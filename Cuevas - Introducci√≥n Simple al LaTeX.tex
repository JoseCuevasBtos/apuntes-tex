\documentclass[11pt,oneside,letterpaper]{book}

% ========= Paquetes utilizados ==============
\usepackage[bookmarks=true, bookmarksnumbered=true, bookmarksopen=true, urlcolor=blue]{hyperref}
\usepackage[utf8x]{inputenc}
\usepackage[T1]{fontenc}
\usepackage[spanish]{babel}
\usepackage{amsmath,amssymb,amsthm}
\usepackage{mathrsfs,cancel}
\usepackage{xcolor}
\usepackage{enumerate}
\usepackage{tikz,pgf}
\usetikzlibrary{arrows,calc}
\tikzset{
every picture/.append style={
  execute at begin picture={\deactivatequoting},
  execute at end picture={\activatequoting}
  }
} % Para que no tenga problemas con el babel en español
\usepackage{longtable}
\usepackage{listings} % Para escribir código
\lstset{inputencoding=latin1,
	extendedchars=true,
	language={[LaTeX]TeX},
	breaklines=true,
	basicstyle=\ttfamily\footnotesize,
	keywordstyle=\color[HTML]{006699},
	stringstyle=\ttfamily\color[HTML]{006699},
	backgroundcolor=\color{gray!10}
}
\lstset{
     literate=%
         {á}{{\'a}}1
         {í}{{\'i}}1
         {é}{{\'e}}1
         {ý}{{\'y}}1
         {ú}{{\'u}}1
         {ó}{{\'o}}1
         {ě}{{\v{e}}}1
         {š}{{\v{s}}}1
         {č}{{\v{c}}}1
         {ř}{{\v{r}}}1
         {ž}{{\v{z}}}1
         {ď}{{\v{d}}}1
         {ť}{{\v{t}}}1
         {ň}{{\v{n}}}1                
         {ů}{{\r{u}}}1
         {Á}{{\'A}}1
         {Í}{{\'I}}1
         {É}{{\'E}}1
         {Ý}{{\'Y}}1
         {Ú}{{\'U}}1
         {Ó}{{\'O}}1
         {Ě}{{\v{E}}}1
         {Š}{{\v{S}}}1
         {Č}{{\v{C}}}1
         {Ř}{{\v{R}}}1
         {Ž}{{\v{Z}}}1
         {Ď}{{\v{D}}}1
         {Ť}{{\v{T}}}1
         {Ň}{{\v{N}}}1                
         {Ů}{{\r{U}}}1    
}

% ======== Ejemplo de aplicación =============
\usepackage[most]{tcolorbox}
\usepackage{mdframed}
\definecolor{thm}{HTML}{3399FF}
\newtheoremstyle{custom}
	{0cm}
	{0cm}
	{\normalfont}
	{0.5cm}
	{\sffamily\bfseries}
	{:}
	{4pt}
	{\color{thm}{\thmname{#1}\thmnumber{ #2}\thmnote{ -- \textit{#3}}}}
\theoremstyle{custom}
\newmdtheoremenv[linewidth=2pt,
linecolor=thm,
backgroundcolor=thm!15,
rightline=false,topline=false,bottomline=false]{ex}{Ejemplo de aplicación}[chapter]

\theoremstyle{definition}
\newtheorem{mythm}{Teorema}[chapter]

% ========= Comandos propios =================
\newcommand{\cbox}[1]{
\begin{center}
	\fboxsep 12pt \fcolorbox{thm}{gray!5}{
	\begin{minipage}[t]{11cm}
	#1
	\end{minipage}}
\end{center}
}
\newcommand{\N}{\mathbb{N}}
\newcommand{\Z}{\mathbb{Z}}
\newcommand{\Q}{\mathbb{Q}}
\newcommand{\R}{\mathbb{R}}
\newcommand{\C}{\mathbb{C}}

\title{Introducción Corta al \LaTeX}
\author{José Cuevas Btos.}
\date\today

\begin{document}
\maketitle
\tableofcontents

\chapter*{Introducción}
Richard Feynman, también conocido como el Gran Profesor, fue un famoso físico principalmente en la década de los 60's, entre una de sus frases célebres y lecciones esta su método de aprendizaje el cual consiste en aprender a través de la enseñanza. Ahora, puede ser difícil para los estudiantes de física y matemática encontrar a alguien a quien enseñarles todo lo que van entendiendo a lo largo de sus cursos, por lo que la escritura entra de forma natural en sus vidas.

En la actualidad, creo que es un hecho que Microsoft Word se ha convertido en el software de escritura predeterminado, si bien es útil para escribir una reseña o incluso un artículo científico tiene dos grandes problemas:
\begin{enumerate}
\item Se ve \textbf{horriblemente mal}.
\item Es difícil de configurar una vez que has escrito mucho, de compilar o hasta corres riesgo de que alguien distribuya tu documento a su nombre.
\end{enumerate}
Es ahí donde \LaTeX{} entra naturalmente y con mucha razón. \LaTeX{} es un distribuidor del lenguaje \TeX{} y sus extensiones con el cual se hacen los artículos, libros e incluso presentaciones científicas por excelencia, esto se debe a que:
\begin{enumerate}
\item Es gratis y de libre acceso.
\item Compila directamente en formato PDF y lo hace sin tener un tamaño elevado.
\item Se ve elegante y es fácil de alterar.
\end{enumerate}
Es por eso que, debido a mi largo tiempo estudiando este lenguaje, he decidido hacer una introducción corta acerca de las herramientas más comunes a la hora de escribir textos científicos.

Como advertencia final debo indicar que obviaré los capítulos de instalación de \LaTeX{}, debido a que esto me fue difícil y no conozco todos los tecnicismos para poder guiar a alguien, sin embargo, aun así recomendaré los software necesarios. La edición de este texto fue hecha con \textbf{\TeX{}Maker} \url{http://www.xm1math.net/texmaker/download.html} y la distribución con \textbf{Mik\TeX{} 2.9} \url{https://miktex.org/howto/install-miktex}. Mik\TeX{} solo funciona en Windows, sin embargo, puede descargar \TeX{}Live de tener Mac OS o Linux.

\chapter{``Hola mundo''}
Todos o casí todos los comandos en \LaTeX{} siguen la estructura de \lstinline|\command[opcional]{necesario}|. De forma que como puede ver, un comando se caracteriza por llamarse con un símbolo \lstinline|/| y tener uno o más campos obligatorios dentro de \lstinline|{}|. El primer comando que conoceremos es \lstinline|\documentclass[opciones]{tipo}| en donde el campo obligatorio se rellena con:
\begin{longtable}{p{3cm}p{9cm}}
\texttt{article} & Sirve para escribir artículos cortos, reportes cortos, presentaciones, invitaciones, etc.\\
\texttt{report} & Sirve para escribir reportes largos, con capítulos, libros cortos, tesis, etc.\\
\texttt{book} & Sirve para escribir libros reales.\\
\texttt{letter} & Sirve para escribir cartas.\\
\texttt{memoir} & Sirve para escribir cualquier clase de documento. Es versátil, aun que no por eso recomendable.\\
\texttt{proc} & Similar a un artículo, pero para cosas más cortas.\\
\texttt{beamer} & Sirve para hacer presentaciones (revisar con cuidado).
\end{longtable}

Las opciones para poner en el comando son, el tamaño de la página:
\begin{longtable}{ll}
\hline Opciones & Tamaño (en mm)\\
\hline \texttt{a0paper} & 841 x 1189\\
\texttt{a1paper} & 594 x 841\\
\texttt{a2paper} & 420 x 594\\
\texttt{a3paper} & 297 x 420\\
\texttt{a4paper} & 210 x 297\\
\texttt{a5paper} & 148 x 210\\
\texttt{a6paper} & 105 x 148\\
\texttt{b0paper} & 1000 x 1414\\
\texttt{b1paper} & 707 x 1000\\
\texttt{b2paper} & 500 x 707\\
\texttt{b3paper} & 353 x 500\\
\texttt{b4paper} & 250 x 353\\
\texttt{b5paper} & 176 x 250\\
\texttt{b6paper} & 125 x 176\\
\texttt{c0paper} & 917 x 1297\\
\texttt{c1paper} & 648 x 917\\
\texttt{c2paper} & 458 x 648\\
\texttt{c3paper} & 324 x 458\\
\texttt{c4paper} & 250 x 353\\
\texttt{c5paper} & 229 x 324\\
\texttt{c6paper} & 114 x 162\\
\texttt{letterpaper} & 216 x 279\\
\texttt{legalpaper} & 216 x 356\\
\texttt{ansiapaper} & 216 x 279\\
\texttt{ansibpaper} & 279 x 432\\
\texttt{ansicpaper} & 432 x 559\\
\texttt{ansidpaper} & 559 x 864\\ \hline
\end{longtable}
El tamaño de letra, que se escribe en \texttt{pt} (abreviación de \textit{punto}, que en tipografía equivale a 1/72 de pulgada). Por defecto los documentos de \LaTeX{} vienen con \texttt{10pt} de tamaño normal, los documentos no suelen tener más de \texttt{12pt} de tamaño.

También en las opciones se puede utilizar \texttt{fleqn} para alinear las ecuaciones a la izquierda en vez de centrado (por defecto) y \texttt{leqno} para poner las etiquetas (números de las ecuaciones) a la izquierda en vez de a la derecha (por defecto).

\texttt{titlepage} y \texttt{notitlepage} sirve para indicar si desea crear o no una página específica para el título (no por defecto en \texttt{article} y si por defecto en \texttt{book} y \texttt{report}).

\texttt{oneside} y \texttt{twoside} indican si el documento debe ser a una o dos caras (una en \texttt{article} y \texttt{report}, dos en \texttt{book}). Nótese que esta configuración es solamente visual, no le dice a su impresora como debería imprimir el documento.

\texttt{onecolumn} y \texttt{twocolumn} determinan el numero de columnas de un documento (uno por defecto).

\texttt{openany} determina que los capítulos partan en cualquier página, mientras que \texttt{openright} determina que los capítulos comienzan desde la página de la derecha\footnote{Solo funciona en documentos con \texttt{twoside} y no en \texttt{article} porque no posee capítulos.}.

\section{Paquetes fundamentales y preámbulo básico}
De por si, \TeX{} es un lenguaje bastante completo, pero para poder complementarlo y adaptarlo a nuestras necesidades necesitaremos ampliarlo un poco, esto se hace incorporando \textbf{paquetes} que son archivos llenos de código para expandir nuestras opciones de manera sencilla.

En general, todo documento en español debería comenzar con las líneas
\begin{lstlisting}
\usepackage[spanish]{babel} % Sirve para traducir ciertos comandos
\usepackage[utf8x]{inputenc} % Sirve para decodificar caracteres especiales
\usepackage[T1]{fontenc} % Sirve para adecuar las fuentes de letras
\end{lstlisting}
En donde, como vera, los comentarios se escriben como \lstinline|% Comentario|. Además, se puede incluir unos paquetes extra para poder complementar tu documento como por ejemplo, el paquete \texttt{geometry}, que nos permite editar la geometría de la página.
\begin{figure}[!ht]
\centering
\begin{tikzpicture}
\draw (0,0) rectangle (6,8);
\fill[black!20] (1,1) rectangle (5,7);
\begin{scope}[|<->|]
\draw (2,1) -- node[above right]{altura} ++(0,6);
\draw (1,2) -- node[below]{largo} ++(4,0);
\draw (0,1) -- node[below right]{margen izquierdo} ++(1,0);
\draw (2,7) -- node[right]{margen superior} ++(0,1);
\end{scope}
\end{tikzpicture}
\caption{Diagrama sencillo de una página}
\end{figure}

Por ejemplo, si quisiéramos hacer un documento de largo 17 cm, alto 24 cm y margenes izquierdo y derecho de 2 cm escribiríamos en el preámbulo
\begin{lstlisting}
\usepackage[total={17cm,24cm},top=2cm,left=2cm]{geometry}
\end{lstlisting}
A esto le podemos sumar el uso de ciertos paquetes para incrementar nuestros símbolos:
\begin{lstlisting}
\usepackage{latexsym,amsmath,amssymb,amsfonts}
\end{lstlisting}
También, por estar en español se recomienda traducir ciertos comandos desde el ya para poder utilizarlos a futuro sin problemas:
\begin{lstlisting}
\renewcommand{\contentsname}{Contenido}
\renewcommand{\partname}{Parte}
\renewcommand{\appendixname}{Apéndice}
\renewcommand{\figurename}{Figura}
\renewcommand{\tablename}{Tabla}
\AtBeginDocument{\renewcommand\tablename{Tabla}}
\renewcommand{\abstractname}{Resumen}
\renewcommand{\refname}{Bibliografía}
\renewcommand{\chaptername}{Capítulo} % para 'book'
\renewcommand{\bibname}{Bibliografía} % para 'book'
\end{lstlisting}

\section{Texto simple}
Hasta ahora solo hemos hablado de un documento sin nada de texto, el primer comando necesario es la inclusión de un título, autor y fecha, para ello, en el preámbulo se escribe
\begin{lstlisting}
\title{Un título interesante}
\author{Yo}
\date\today
\end{lstlisting}
En donde \lstinline|\today| es un comando para escribir la fecha actual según su computador. Luego comienzas todos los contenidos del documento con el comando:
\begin{lstlisting}
\begin{document}
% Aquí va el contenido...
\end{document}
\end{lstlisting}
Para ponerle el titular utilizas el comando \lstinline|\maketitle|, por lo que, su documento debería verse como:
\begin{lstlisting}
\documentclass{article}

\usepackage[spanish]{babel}
\usepackage[utf8x]{inputenc}
\usepackage[T1]{fontenc}
\usepackage[top=2cm,left=2cm]{geometry}

\title{Un artículo}
\author{José Cuevas}
\date\today

\begin{document}
\maketitle

...
\end{document}
\end{lstlisting}
Luego, después de hacer el título con el comando ya indicado puede comenzar a escribir libremente, para \textit{enfatizar} ciertas palabras hay una serie de comandos para modificar el texto normal, estas se activan utilizando un comando de tipo \lstinline|\comando{...}| o a través de modificando todo el texto de un entorno como \lstinline|{\cmd ...}|. Las modificaciones pueden verse en la siguiente tabla:\\
\begin{longtable}{llll}
\hline Comando en linea & Atajo (\TeX{}Maker) & Comando para entorno & Produce\\ \hline\hline
\lstinline|\textit{}| & \verb|Ctrl+I| & \lstinline|{\it ...}| & \textit{Itálica}\\
\lstinline|\textbf{}| & \verb|Ctrl+B| & \lstinline|{\bf ...}| & \textbf{Negritas}\\
\lstinline|\textsc{}| & \verb|Ctrl+Shift+C| & \lstinline|{\sc ...}| & \textsc{Versallitas}\\
\lstinline|\textsl{}| & \verb|Ctrl+Shift+S| & \lstinline|{\sl ...}| & \textsl{Inclinado}\\ \hline
\end{longtable}
Además, todo documento \LaTeX{} cuenta con cuatro tipos de fuentes a las que puede acceder:
\begin{longtable}{lll}
\hline Comando en linea & Comando para entorno & Produce\\ \hline\hline
\lstinline|\textrm{}| & \lstinline|{\rm ...}| & Romana\\
\lstinline|\textsf{}| & \lstinline|{\sf ...}| & \textsf{Sans-Serif}\\
\lstinline|\texttt{}| & \lstinline|{\tt ...}| & \texttt{Mono-espaciada}\\
\lstinline|$...$| & \lstinline|$$...$$| & $Matematicas.$\\ \hline
\end{longtable}
Estos cuatro comandos no generan un texto con una fuente específica en realidad, sino que escriben texto con una \textbf{familia} o \textbf{tipo} de fuente. Esto será retomado a futuro cuando se muestren paquetes que alteran estas familias. Asimismo, la escritura en fuente matemática será estudiada en el próximo capítulo con detención.

Además se puede escribir en los siguientes tamaños de letra:
\begin{longtable}{ll}
\hline Comando & Produce \\ \hline \hline
\lstinline|{\tiny ...}| & {\tiny Pequeñísimo}\\
\lstinline|{\scriptsize ...}| & {\scriptsize Muy pequeñito}\\
\lstinline|{\footnotesize ...}| & {\footnotesize Pequeñito}\\
\lstinline|{\small ...}| & {\small Pequeño}\\
\lstinline|{\normalsize ...}| & {\normalsize Normal}\\
\lstinline|{\large ...}| & {\large Grande}\\
\lstinline|{\Large ...}| & {\Large Muy grande}\\
\lstinline|{\huge ...}| & {\huge Enorme}\\
\lstinline|{\Huge ...}| & {\Huge Gigante}\\\hline
\end{longtable}
\begin{ex}
Esta es la explicación de un teorema:
\begin{lstlisting}
{\large\sc El teorema de Palomar:} Siempre que se tengan $n+1$ palomas distribuidas en $n$ cajas, al menos una caja tendrá \textbf{dos palomas}. 
\end{lstlisting}
Lo que produce:
\cbox{{\large\sc El teorema de Palomar:} Siempre que se tengan $n+1$ palomas distribuidas en $n$ cajas, al menos una caja tendrá \textbf{dos palomas}.}
\end{ex}
Como dato extra, para hacer las comillas en \LaTeX{} se utiliza \lstinline|``...''| (``...''), \lstinline|`'| (`...') o \lstinline|<<...>>| (<<...>>).

\section{Párrafos, símbolos y listas}
Al escribir debe considerar que hay una serie de símbolos especiales que tienen un uso específico en \LaTeX{}, para poder utilizarlos en la escritura siga la siguiente tabla:
\begin{longtable}{p{1cm}p{2.2cm}p{8cm}}
\hline Símbolo & Comando & Uso en \LaTeX{}\\ \hline\hline
\char`\\ & \lstinline|\char`\\| & Iniciar comando\\
\{ \} & \lstinline|\{ \}| & Determinar entorno o bloque de comando.\\
\$ & \lstinline|\$| & Determinar entorno en modo matemático.\\
\_ \^{} & \lstinline|\_ \^{}| & Sub y super índices en modo matemático.\\
\# & \lstinline|\#| & Indicar parámetros en la macros (crear comandos).\\
\~{} & \lstinline|\~{}| & Evitar cortes de renglón.\\
\% & \lstinline|\%| & Realizar comentarios.\\ \hline
\end{longtable}

Para cambiar de línea presionar \texttt{enter} no servirá de mucho. En \LaTeX{} las líneas cambian con el comando \lstinline|\\|. Este comando, sin embargo, no le otorga sangría al siguiente párrafo, para ponerle sangría se debe presionar dos veces \texttt{enter}.

El comando \lstinline|\\| también puede específicar una distancia entre párrafos utilizando los símbolos \lstinline|[]| con la distancia escrita en \texttt{cm} (centímetros), \texttt{mm} (milímetros), \texttt{in} (pulgadas) o una fracción de una medida universal de \LaTeX{}. Con lo último hablamos de comandos como \lstinline|\textwidth| (largo de texto) o \lstinline|\pagewidth| (largo de página).

Para generar espacio de otra forma se pueden utilizar los comandos \lstinline|\hspace{}| y \lstinline|\vspace{}| donde en su campo obligatorio va la distancia horizontal o vertical respectivamente.

También podemos cambiar la alineación de texto con un par de comandos sencillos, utilizando el entorno \texttt{center} puedes centrar el texto, con \texttt{left} ponerlo a la izquierda o con \texttt{right} a la derecha. Si se siente confundido, un entorno se crea en \LaTeX{} con un par de comandos de apertura y cierre \lstinline|\begin{comando}...\end{comando}|. Los comandos para entornos les llamamos a aquellos que modifican todas las propiedades de un entorno, también se puede formar un entorno normal con el uso de corchetes tipo \lstinline|{...}|.

Para hacer listas solemos utilizar el entorno \texttt{enumerate} para que sea enumerada e \texttt{itemize} para que no lo sea. Para enlistar los elementos se escribe \lstinline|\item ...|.
\begin{ex}
Un ejemplo es una lista de compras:
\begin{lstlisting}
\begin{center}
\it Lista de compras:
\end{center}
\begin{enumerate}
\item Ropa.
\item Comida:
	\begin{enumerate}
	\item Frutas:
		\begin{itemize}
		\item Manzanas.
		\item Bananas.
		\end{itemize}
	\item Pan.
	\item Bebidas y jugos.
	\end{enumerate}
\item Útiles.
\end{enumerate}
\end{lstlisting}
Lo que produce:
\cbox{\begin{center}
\it Lista de compras:
\end{center}
\begin{enumerate}
\item Ropa.
\item Comida:
	\begin{enumerate}
	\item Frutas:
		\begin{itemize}
		\item Manzanas.
		\item Bananas.
		\end{itemize}
	\item Pan.
	\item Bebidas y jugos.
	\end{enumerate}
\item Útiles.
\end{enumerate}}
\end{ex}
También se puede utilizar el paquete \texttt{enumerate} para poder modificar de forma sencilla la notación de las listas enumeradas. Por ejemplo, se escribe:
\begin{lstlisting}
\begin{enumerate}[a)]
\item ...
\end{enumerate}
\end{lstlisting}
Y se obtiene una lista que se enumera a), b), etc. También funciona con números romanos (ej, \texttt{I.} o en minúsculas \texttt{i.}) y números arábicos (ej. \texttt{(1)}).

Los símbolos utilizados por el entorno \texttt{enumerate} por defecto vienen dados por comandos \lstinline|\labelenumi|, \lstinline|\labelenumii|, \lstinline|\labelenumiii|, \lstinline|labelenumiv|, etc. De manera que si, por ejemplo, quisiera que la primera enumeración fuese por defecto \texttt{(A.) (B.) (C.)}, debo redefinir el comando. En la redefinición debo indicar el número con \lstinline|{enumi}|, \lstinline|{enumii}| y así. Por ejemplo, el siguiente código:
\begin{lstlisting}
\renewcommand{\labelenumi}{\Alph{enumi}.)}
\renewcommand{\labelenumii}{\Alph{enumi}.\arabic{enumii})}
\end{lstlisting}
Produce que una lista salga así:
\cbox{\begin{enumerate}[\Alph{enumi}.)]
\item Problema A.
\item Problema B.
	\begin{enumerate}[\Alph{enumi}.\arabic{enumii})]
	\item Parte 1.
	\item Parte 2.
	\end{enumerate}
\item Problema C.
\end{enumerate}}

\section{Inventando y redefiniendo comandos}
El catálogo de \LaTeX{} es bastante amplio con lo que a comandos respecta, pero siempre puede mejorar con una que otra opción que te gustaría simplificar, para ello podemos definir nuestros propios comandos con el comando (vaya la redundancia) \lstinline|\newcommand{nombre}[núm]{def}|. El primer campo define el nombre del comando, el campo opcional define la cantidad de entornos que utilizará (si no se escribe, el programa asume cero) y el segundo campo define el comando.

Por ejemplo, supongamos que deseo definir un comando para los símbolos < (menor que) y > (mayor que), para ello utilizo el código
\begin{lstlisting}
\newcommand{\lt}{<}
\newcommand{\gt}{>} % lt y gt son acrónimos para 'less than' y 'greater than'
\end{lstlisting}
El comando genera un error si el nombre que le has dado ya constituye un comando predefinido. Si, por alguna razón, quieres seguir usándolo puedes redefinir el comando con \lstinline|\renewcommand{...}{...}|.

También se puede crear entornos (aquellos que se llaman con \lstinline|\begin{...} \end{...}|) utilizando \lstinline|\newenvironment{nombre}[num]{antes}{después}|. El primer campo determina el nombre, el campo opcional la cantidad de campos obligatorios, el segundo el código antes del texto y el tercero el código despues, por ejemplo:
\begin{lstlisting}
\newenvironment{important}{\begin{center}\bf}{\end{center}}
\end{lstlisting}
Analíticamente puedes redefinir un entorno con \lstinline|\renewenvironment{...}{...}{...}|.

\chapter{Haciendo matemáticas}
La gran razón del uso de \LaTeX{} es su elegancia al momento de hacer o escribir ecuaciones matemáticas. Para poder utilizar matemáticas entre lineas como así, $x+1$ (también llamado \texttt{modo inline}), se utilizan los delimitadores \lstinline|$...$| donde la ecuación va dentro de los símbolos de dolar. Para poder escribir matemática en su propia línea (\texttt{modo display}) como prosigue
$$\int x^2\,dx$$
se utilizan los delimitadores \lstinline|$$...$$| o también \lstinline|\[...\]|.

\section{Lo básico}
En general, no se debe aprender nada en \LaTeX{} para escribir fórmulas con un par de excepciones. Se utilizan los símbolos \lstinline|^| y \lstinline|_| para simular los exponentes y los subíndices en las expresiones. Otros símbolos matemáticos están dados por comandos simples e intuitivos como se enlistan en la siguiente tabla:
\begin{longtable}{ll|ll}
\hline Comando & Produce & Comando & Produce \\ \hline \hline
\lstinline|\sum| & $\displaystyle\sum$ & \lstinline|\int| & $\displaystyle\int$\\
\lstinline|\prod| & $\displaystyle\prod$ & \lstinline|\lim| & $\displaystyle\lim$\\
\lstinline|\sin| & $\sin$ & \lstinline|\cos| & $\cos$\\
\lstinline|\tan| & $\tan$ & \lstinline|\ln| & $\ln$\\
\lstinline|\log| & $\log$ & \lstinline|\exp| & $\exp$\\\hline
\end{longtable}

Aquí puede notar que en general los comandos que son funciones escritas de forma normal suelen escribirse igual pero con el símbolo \lstinline|\| antes.

Además es usual que también se quieran hacer fracciones, las fracciones se describen con el comando \lstinline|\frac{...}{...}| donde dentro de los corchetes van la parte superior e inferior de esta fracción. Esta puede verse muy bien en modo display
$$\frac{x+1}{x-1}$$
Pero puede verse terrible en modo inline $\frac{x+1}{x-1}$. Para poder forzar el modo inline a verse como el display puede utilizar el comando \lstinline|\displaystyle|, pero puede simplificar su código escribiendo \lstinline|\dfrac{...}{...}| en su lugar, este comando automáticamente forma una fracción en modo display $\dfrac{x+1}{x-1}$, el uso de este último es bastante recomendable en fracciones continuas, este es un ejemplo solo con \lstinline|\frac|:
$$1+\frac{1}{1+\frac{1}{1+\frac{1}{1+\cdots}}}$$
Y este es el mismo ejemplo pero con \lstinline|\dfrac|:
$$1+\dfrac{1}{1+\dfrac{1}{1+\dfrac{1}{1+\cdots}}}$$
También cabe destacar que para poder escribir super o subíndices con más de un símbolo se debe escribir tal expresión dentro de \lstinline|{...}|, por ejemplo, el comando \lstinline|10^-1| da $10^-1$, para evitar tal error se escribe \lstinline|10^{-1}| lo que da $10^{-1}$ como esperado.

Para escribir el símbolo de multiplicación hay dos opciones comunes, el punto centrado que se hace con \lstinline|\cdot| ($\cdot$) y la cruz que se hace por \lstinline|\times| ($\times$), también podría querer utilizar el símbolo de división que se denota como \lstinline|\div| ($\div$).

Para escribir letras griegas como $\alpha,\beta,\dots$ se escriben tal cual precedidas por el backslash, como \lstinline|\alpha,\beta,\dots|. Para escribirse en mayúsculas solo se escribe en mayúsculas la primera letra, es decir, \lstinline|\Gamma,\Delta,\dots| produce $\Gamma,\Delta,\dots$ Por tanto, la letra $\pi$ se escribe como \lstinline|\pi|.

Para escribir radicales o raíces cuadradas como $\sqrt{2}$ se utiliza \lstinline|\sqrt{2}|, pero para escribir cualquier raíz se debe utilizar un corchete antes para especificar el índice, por ejemplo, $\sqrt[3]{2}$ con \lstinline|\sqrt[3]{2}|.

Los espacios en modo matemático son suprimidos de forma que escribir \lstinline|$a b$| da $a b$. Para poder escribir espacios se utiliza \lstinline|$a\,b\;c\quad d\qquad e$| los que producen espacios de
$$a\,b\;c\quad d\qquad e$$
Los símbolos de relación entre números como ``mayor que'' y ``menor que'' se escriben como normalmente. Para indicar posible igualdad se utilizan \lstinline|\leq| y \lstinline|\geq| que producen $a\leq b$ y $b\geq c$ respectivamente. La desigualdad se hace con \lstinline|\neq|.
\begin{ex}
El siguiente parrafo habla de la suma de Gauss:
\begin{lstlisting}
La suma de Gauss índica que si tenemos una suma $S_N$ de todos los enteros hasta $N$, la cuál puede denotarse como:
$$S_N=\sum_{k=1}^N k$$
Podemos sumar los elementos del final y del inicio para deducir que:
$$S_N=\frac{N(N+1)}{2}.$$
\end{lstlisting}
Esto produce:
\cbox{La suma de Gauss índica que si tenemos una suma $S_N$ de todos los enteros hasta $N$, la cuál puede denotarse como:
$$S_N=\sum_{k=1}^N k$$
Podemos sumar los elementos del final y del inicio para deducir que:
$$S_N=\frac{N(N+1)}{2}.$$}
\end{ex}

Para utilizar paréntesis de corchetes en \LaTeX{} se requiere del uso de un backslash antes, como \lstinline|\{...\}| lo que produce $\{...\}$. Para indicar pertenencia se utiliza \lstinline|\in| y para negar pertenencia se puede usar \lstinline|\not\in| lo que produce $a\in A$ y $b\not\in A$. Para indicar que dos conjuntos son sub o superconjuntos se utilizan \lstinline|\subset| y \lstinline|\supset| como $A\subset B$ y $B\supset A$ respectivamente. También se le puede añadir un \texttt{eq} al final para hacer la linea bajo el símbolo para identificar posible igualdad, es decir, \lstinline|\subseteq| produce $A\subseteq B$.

Usualmente se utiliza una notación de tipo $\mathbb{N}$ para simbolizar conjuntos importantes, eso se logra modificando los caracteres, para ello se utilizan los siguientes comandos:
\begin{longtable}{ll}
\hline Comando & Produce\\\hline\hline
\lstinline|\mathbb{}| & $\mathbb{ABC}$\\
\lstinline|\mathcal{}| & $\mathcal{ABC}$\\
\lstinline|\mathscr{}|\footnote{Requiere el paquete \texttt{mathrsfs}.} & $\mathscr{ABC}$\\
\lstinline|\mathfrak{}| & $\mathfrak{ABC}$\\ \hline
\end{longtable}

Otros comandos utilizados en lógica matemática y teoría de conjuntos son los símbolos ``para todo'' que se escribe \lstinline|\forall| ($\forall$), ``existe'' \lstinline|\exists| ($\exists$), ``y'' \lstinline|\wedge| ($\wedge$), ``o'' \lstinline|\vee| ($\vee$), ``unión'' \lstinline|\cup| ($\cup$), ``intersección'' \lstinline|\cap| ($\cap$), ``diferencia de conjuntos'' \lstinline|\setminus| ($\setminus$) y ``conjunto vacío'' \lstinline|\emptyset| ($\emptyset$). En particular prefiero como se ve \lstinline|\varnothing| ($\varnothing$), por lo que, para que siempre que escriba \lstinline|\emptyset| salga \lstinline|\varnothing| escribes en el preámbulo:
\begin{lstlisting}
\renewcommand{\emptyset}{\varnothing}
\end{lstlisting}

Otra recomendación personal es que para acortar como escribir los conjuntos $\N$, $\Z$ y otros, puedes definir en el preámbulo los comandos:
\begin{lstlisting}
\newcommand{\N}{\mathbb{N}}
\newcommand{\Z}{\mathbb{Z}}
\newcommand{\Q}{\mathbb{Q}}
\newcommand{\R}{\mathbb{R}}
\newcommand{\C}{\mathbb{C}}
\end{lstlisting}
Una recomendación extra es que siempre que busque un símbolo que no conozca puede acceder gratuitamente al sitio \textbf{Detexify} \url{http://detexify.kirelabs.org/classify.html}, aquí usted dibuja el símbolo que busca y la página le otorga una lista de ellos con la especificación de si se utiliza en modo de texto o matemático y si requiere de algún paquete.

\section{Ecuaciones enumeradas, alineadas y más}
Algo común en los libros es enumerar los fórmulas matemáticas y/o científicas importantes para poder referenciarlas luego, para ello utilizamos el entorno \texttt{equation} como prosigue:
\begin{lstlisting}
\begin{equation}
1+1=2
\end{equation}
\end{lstlisting}
Lo que produce
\cbox{\begin{equation}
1+1=2
\end{equation}}
Un error común es que cuando alumnos quieren escribir varias líneas de fórmulas utilizan \lstinline|\\| dentro de un entorno \lstinline|$$...$$|, lo cual \LaTeX{} no lee apropiadamente. La forma correcta de hacer ecuaciones alineadas es utilizar el entorno \texttt{align}, aquí las ecuaciones se alinean por medio del símbolo \lstinline|&| y se permite el uso \lstinline|\\|, sin embargo, todas las lineas son enumeradas, para evitar eso escribes \lstinline|\nonumber| o \lstinline|\notag| en las líneas que no quieres enumerar. Si quieres escribir una ecuación alineada sin ninguna línea enumerada utilizas el entorno \texttt{align*}.

Para demostrar la aplicación de \texttt{align} y \lstinline|\nonumber| se demuestra el siguiente ejemplo:
\begin{lstlisting}
\begin{align}
(a+b)^2 &= c^2+4\cdot\frac{1}{2}ab\nonumber\\ % Los '&' se utilizan para alinear dichos caracteres...
a^2+2ab+b^2 &= c^2+2ab\nonumber\\
a^2+b^2 &= c^2
\end{align}
Despejando un poco se obtiene la otra expresión para un cateto:
\begin{equation}
a=\sqrt{b^2-c^2}
\end{equation}
\end{lstlisting}
\cbox{\begin{align}
(a+b)^2 &= c^2+4\cdot\frac{1}{2}ab\nonumber\\
a^2+2ab+b^2 &= c^2+2ab\nonumber\\
a^2+b^2 &= c^2
\end{align}
Despejando un poco se obtiene la otra expresión para un cateto:
\begin{equation}
a=\sqrt{b^2-c^2}
\end{equation}}
Lo que nos lleva a la siguiente lección, como cancelar ecuaciones. Con esto hablamos de esa linea diagonal que pasa sobre los símbolos, para ello primero incorporamos el paquete \texttt{cancel} y luego usted escoge que tipo de cancelación quiere realizar:
\begin{longtable}{ll}
\hline Comando & Produce \\ \hline\hline
\lstinline|\cancel{ab}| & $\cancel{ab}$\\
\lstinline|\bcancel{ab}| & $\bcancel{ab}$\\
\lstinline|\xcancel{ab}| & $\xcancel{ab}$\\
\lstinline|\cancelto{1}{ab}| & $\cancelto{1}{ab}$\\\hline
\end{longtable}

Volviendo a las ecuaciones alineadas...

\section{Matrices, vectores y más}
Cuando se utilizan símbolos muy grandes (como fracciones, integrales o matrices) los paréntesis al rededor se vuelven feos o pequeños en comparación, para hacerlos más grandes se utiliza \lstinline|\left(| (o cualquier otro símbolo en el lugar de \lstinline|(|) y \lstinline|\right)|. Si se escribe \texttt{left} es obligatorio cerrarlo con \texttt{right}, si solo quieres utilizar uno, puedes cerrar el otro con un símbolo vacío descrito con \texttt{.}.
\begin{ex}
El comando
\begin{lstlisting}
Para demostrar que...
$$\frac{d}{dx}\left[\int_a^x f(x)\,dx\right]=\left.\frac{d}{dx}F(x)\right|_a^x=f(x)$$
\end{lstlisting}
Genera:
\cbox{Para demostrar que la derivada de una integral es igual a la función utilizaremos el siguiente proceso:
$$\frac{d}{dx}\left[\int_a^x f(x)\,dx\right]=\left.\frac{d}{dx}F(x)\right|_a^x=f(x)$$}
\end{ex}

En general, la mejor forma de distribuir datos es a través del comando \lstinline|\begin{array}{orden} ... \end{array}|. En realidad, este es un comando versátil (funciona tanto en modo texto como matemático), pero no corresponde a un entorno matemático (es decir, debes usarlo dentro de \lstinline|$$ ... $$|). El orden queda determinado por una letra por columna que determina la alineación.
\begin{longtable}{ll}
\hline Letra & Alineación\\ \hline\hline
\texttt{l} & Izquierda\\
\texttt{c} & Centrado\\
\texttt{r} & Derecha\\
\texttt{p\{largo\}} & Párrafo de largo específico\\ \hline
\end{longtable}
Como dije, este es el método más general, pero si desea crear matrices específicas también puede utilizar el entorno \texttt{matrix}, por ejemplo:
\begin{lstlisting}
$$\mathcal{M} =
\begin{matrix}
1 & 0\\
0 & 1
\end{matrix})$$
\end{lstlisting}
\cbox{$$\mathcal{M} = \left(
\begin{matrix}
1 & 0\\
0 & 1
\end{matrix}
\right)$$}
También, puede usar otros tipos de matrices según el paréntesis querido, como \lstinline|bmatrix| $\begin{bmatrix}
1 & 0\\
0 & 1
\end{bmatrix}$, \lstinline|pmatrix| $\begin{pmatrix}
1 & 0\\
0 & 1
\end{pmatrix}$, \lstinline|vmatrix| $\begin{vmatrix}
1 & 0\\
0 & 1
\end{vmatrix}$ y \lstinline|Vmatrix| $\begin{Vmatrix}
1 & 0\\
0 & 1
\end{Vmatrix}$.

Para usar la notación de vector se suele usar la letra en negritas \lstinline|{\bf ...}| (aun que personalmente lo detesto) ${\bf a}$ o una flecha por sobre la letra que se genera con \lstinline|\vec{}| $\vec{a}$. El sombrerito del vector unitario se hace con \lstinline|\hat{}| $\hat{a}$, la comilla de la primera derivada se hace con \lstinline|'| $a'$ (de verse mal utilice \lstinline|\,| para generar más espacio), el punto de la derivada temporal se hace con \lstinline|\dot{}| $\dot{a}$, el doble punto con \lstinline|\ddot{}| $\ddot{a}$, para generar la linea sobre las letras se utiliza \lstinline|\bar{}| $\bar{a}$ y la tilde con \lstinline|\tilde{}| $\tilde{a}$.

Para generar el valor absoluto puede utilizar libremente \lstinline@|x|@ ($|x|$), pero se recomienda \lstinline@\vert x\vert@ ($\vert x\vert$)\footnote{La diferencia en este caso no es muy notoria, pero en ciertos dispositivos lo es.}, para hacer la doble línea se utiliza \lstinline|\Vert x\Vert| ($\Vert x\Vert$), con \lstinline|\langle| y \lstinline|\rangle| se forma $\langle x\rangle$, con \lstinline|\lceil| y \lstinline|\rceil| se forma $\lceil x\rceil$, con \lstinline|\lfloor| y \lstinline|\rfloor| se forma $\lfloor x\rfloor$.

\section{Teoremas}
En \LaTeX{} se les llama \textit{``teoremas''} a ambientes que poseen tipo (ej: teorema, lema, ejercicio), numeración (opcional pero recomendada) y nombre (opcional, ej: teorema pitagórico). Los teoremas se importan con el paquete \texttt{amsmath} y \texttt{amsthm}.

Antes de crear los teoremas se debe decidir su estilo, entre los estilos más básicos están
\begin{longtable}{lp{6cm}}
\hline Estilo & Ejemplo\\ \hline\hline
\texttt{definition} & \textbf{Teorema 1.} Bla bla bla...\\
\texttt{plain} & \textbf{Teorema 2.} \textit{Bla bla bla...}\\
\texttt{remark} & \textit{Teorema 3.} Bla bla bla...\\ \hline
\end{longtable}
Se accede a ellos con la línea \lstinline|\theoremstyle{estilo}| antes de los teoremas que poseeran ese estilo (puede incluir múltiples teoremas con múltiples estilos siempre anteponiendo el estilo antes que los teoremas).

Un teorema básico se llama por la línea
\begin{lstlisting}
\newtheorem{abreviación}[contador]{Tipo}[sección]
\end{lstlisting}
En \textit{abreviación} va la palabra clave para llamar al teorema, los teoremas se llaman a través de entornos con dicha palabra. El tipo del teorema es el texto que tendrá al demostrarse y la sección depende de la numeración requerida, podemos enumerar teoremas por partes, capítulos, secciones, etc.

El contador se utiliza cuando queremos que varios teoremas tengan un sólo contador común. Por ejemplo, supongamos que defino un teorema \texttt{thm} de nombre \textit{``teorema''} y luego, defino otro \texttt{lem} de nombre \textit{``lema''}, sin embargo, tras definir el teorema 1.1 quiero obtener un lema 1.2 y luego un teorema 1.3, entonces en el lugar del contador de \texttt{lem} escribo \texttt{thm} y ya está. Cabe destacar que no se pueden utilizar el contador y la sección al mismo tiempo.

Por ejemplo, el teorema más básico sería:
\begin{lstlisting}
\theoremstyle{definition}
\newtheorem{mythm}{Teorema}[chapter]
\end{lstlisting}
Tal que luego, escribir:
\begin{lstlisting}
\begin{mythm}[Pitagórico]
Dado un triangulo de catetos $a$ y $b$...
\end{mythm}
\end{lstlisting}
Produce:
\begin{mythm}[Pitagórico]
Dado un triangulo de catetos $a$ y $b$, e hipotenusa $c$ se cumple:
$$a^2+b^2=c^2.$$
\end{mythm}
Además puedes crear tu propio estilo para teoremas con el comando \lstinline|\newtheoremstyle| el cual funciona como prosigue:
\begin{lstlisting}
\newtheoremstyle{custom} % nombre del estilo
	{0cm} % espacio sobre el teorema
	{0cm} % espacio bajo el teorema
	{\normalfont} % fuente para el cuerpo
	{0.5cm} % sangría
	{\sffamily\bfseries} % fuente título
	{:} % puntuación entre título y cuerpo
	{4pt} % espacio entre título y cuerpo
	{\thmname{#1}\thmnumber{ #2}\thmnote{ -- \textit{#3}}} % configuración del teorema
\end{lstlisting}
En la configuración \lstinline|#1| representa el título del teorema (ej: ``Definición''), \lstinline|#2| el número (ej: ``1.1''), \lstinline|#3| el nombre (ej: ``Teorema de Pitágoras''). Las palabras \lstinline|\thmname{}| y otras sirven para lo mismo, pero generan su propia estructura opcional, de forma que un teorema sin nombre no tenga la línea \lstinline|--| si no se incluye.

\begin{thebibliography}{99}
\bibitem{mora} Alexánder Borbón y Walter Mora. \textit{Edición de Textos Científicos \LaTeX{} 2017}. 2\textsuperscript{a} edición, 2017.
\bibitem{indiatug} Indian \TeX{} Users Group. \textit{\LaTeX{} Tutorials}. 2003.
\bibitem{notancorta} Tobias Oetiker, Hubert Partl, Irene Hyna y Elisabeth Schlegl. \textit{La Introducción No-tan-corta a \LaTeX{} $2_\varepsilon$}. 2014.
\end{thebibliography}

\end{document}