\documentclass[11pt]{book}

\usepackage[book, es]{../cuevasthm}
\input{../template.tex}
\DeclareMathOperator{\Pol}{\mathscr{F}}
\DeclareMathOperator{\Sing}{Sing}
\DeclareMathOperator{\Reg}{Reg}
\DeclareMathOperator{\Isol}{Isol}

% \newcommand\CaDiv{\Div_c}
\DeclareMathOperator{\CaDiv}{CaDiv}
\DeclareMathOperator{\CaCl}{CaCl}

\DeclareMathOperator{\Num}{Num}

\DeclareMathOperator{\shHom}{\mathscr{H}\!\mathit{om}}
\DeclareMathOperator{\shEnd}{\mathscr{E}\!\mathit{nd}}
\DeclareMathOperator{\shTor}{\mathscr{T}\!\mathit{or}}
\DeclareMathOperator{\shExt}{\mathscr{E}\!\mathit{xt}}
\DeclareMathOperator{\bfSpec}{\mathbf{Spec}}

\newcommand{\Hilb}{\mathrm{  Hilb}}
\newcommand{\bfHilb}{\mathbf{Hilb}} % Esquema de Hilbert y otros
\newcommand{\Quot}{\mathrm{  Quot}}
\newcommand{\bfQuot}{\mathbf{Quot}}
\newcommand{\bfPic}{\mathbf{Pic}}
\newcommand{\bfDiv}{\mathbf{Div}}

\DeclareMathOperator{\opkom}{\mathring{\mathcal{K}}}
\newcommand{\midsharp}{\mathrel{\olddiv}}

% Bibliografía
\DeclareBibliographyCategory{geoalg}
\addtocategory{geoalg}{castillo:GA, cutkosky:geometry, milne:AG, fulton:curves, smith:AG}
\defbibheading{geoalg}{\section*{Geometría algebraica clásica}}

\DeclareBibliographyCategory{scheme}
\addtocategory{scheme}{castillo:esquemas, hartshorne:algebraic, liu:algebraic, vakil:rising_sea, stacks}
\defbibheading{scheme}{\section*{Geometría algebraica moderna}}

\DeclareBibliographyCategory{topics}
\defbibheading{topics}{\section*{Tópicos en la teoría de esquemas}}

\DeclareBibliographyCategory{groth}
\addtocategory{groth}{ega-i, ega-iv4}
\defbibheading{groth}{\section*{Escritos de Grothendieck}}

\DeclareBibliographyCategory{etale}
\defbibheading{etale}{\section*{Cohomología étale}}

\DeclareBibliographyCategory{other}
\addtocategory{other}{
	matsumura:ring, kunz:AG,
	kleiman76misconceptions,
	bombieri73counting,
	hochster1969thesis, banaschewski96ideal, dickmann:spectral,
}
\defbibheading{other}{\section*{Otros recursos}}

\DeclareBibliographyCategory{historical}
\addtocategory{historical}{
	stepanov69number, weil48courbes,
	stone37brouwerian, krull32allgemeine, fontana80topological_rings,
}
\defbibheading{historical}{\section*{Documentos históricos}}

% \DeclareBibliographyCategory{history}
% \addtocategory{history}{dieudonne:history}
% \defbibheading{history}{\section*{Historia}}

\DeclareBibliographyCategory{self}
\addtocategory{self}{Alg, CatTh, SetTh, Top}
\defbibheading{self}{\section*{Libros de autoría propia}}

\title{Geometría algebraica}
\author{José Cuevas Barrientos}
\date\today

\begin{document}

\frontmatter
\maketitle
\tableofcontents

% \input{prologo.tex}

\mainmatter
\part{Geometría Algebraica Clásica}
\input{intro_variedades.tex}

\input{variedades-ii.tex}

% \input{divisores.tex}

\part{El lenguage de Esquemas}
En esta parte introducimos el lenguaje de esquemas y las herramientas fundamentales para su estudio.

La definición de \textit{esquema} es necesaria por varias razones: en primer lugar, permite hacer geometría algebraica sobre cuerpos no algebraicamente cerrados
(pues a diferenia de la geometría algebraica clásica, no se construye sobre el teorema de ceros de Hilbert), pero mejor aún permite incluso definir objetos
sobre anillos y, en particular, da una definición suficientemente flexible como para tener aplicaciones aritméticas.
En segundo lugar, la definición permite que la categoría de esquemas contenga a la categoría de anillos (unitarios conmutativos), de modo que tiene interés
para el álgebra conmutativa y, de hecho, permite un diálogo entre ambas disciplinas (a diferencia del <<monólogo>> que las teorías predecesoras tenían).
En tercer lugar, los esquemas están llenos de puntos genéricos, lo que permite que topológicamente haya una correspondencia entre puntos y subvariedades cerradas;
esta es una ventaja que Grothendieck observó de la teoría de Weil.
En cuarto lugar, los esquemas representan un objeto unificado que, como ejemplos, incluyen a las variedades afines, proyectivas y algebraicas;
más aún, su definición es altamente funtorial, de modo que es fácil definir ciertos espacios de moduli, o construcciones como las explosiones, sin tener
que pasar primero por cálculos concretos.
En último lugar, los esquemas permiten apropiadamente una noción de extensión y restricción de escalares, y existen teorías adecuadas para medir qué cosas
se preservan en ascenso y descenso.
% En último lugar, los esquemas, con su generalidad, son capaces de medir <<deformaciones infinitesimales>>: su flexibilidad es tal que puede captar 

La desventaja es que el reino esquemático es tan vasto que resulta como una jungla amazónica para el lector, y este está obligado a pasar por cientos de páginas
para ubicarse apropiadamente y reencontrar los objetos que le son familiares.
Peor aún, es quizá digno de mención el que resulta difícil probar que, de hecho, el mundo esquemático funciona de manera compatible a los casos base para los
geómetras de siglo \textsc{xix}, es decir, que coinciden con la geometría analítica compleja.
Así, esta sección resulta extensa, pero finalmente otorga todas las herramientas fundamentales para el resto del libro.

Destacamos ciertos resultados importantes:
\begin{enumerate}
	\item Los esquemas extienden a la categoría de anillos conmutativos (cfr., teorema~\ref{thm:hom_aff_sch}).
\end{enumerate}
De hecho podemos ser un poco menos ambiguos.
Uno construirá primero la noción de \textit{esquema afín} asociado a un anillo $A$, llamado su \textit{espectro}, el cual posee toda la información original,
y luego uno define un esquema como un objeto que puede cubrirse de manera adecuada por esquemas afines.
Para el lector familiarizado con topología diferencial: los esquemas afines son a los espacios euclídeos $\R^n$ lo que los esquemas
son a las variedades diferenciales.

En particular, en la primera parte vimos que una variedad algebraica afín viene completamente determinada por su anillo de coordenadas $A$,
así que tomando $\Spec A$ como su sustituto uno obtiene que:
\begin{enumerate}[resume]
	\item Los esquemas extienden a las variedades clásicas (cfr., teorema~\ref{thm:sch_extendend_vars}).
	% \item Un esquema íntegro posee un punto genérico que, en muchos casos, captura información <<genérica>>.
\end{enumerate}
No obstante, las extienden estrictamente, incluso si hablamos de esquemas algebraicos sobre un cuerpo fijo.
Uno de los fenómenos introducidos por éste lenguaje es el de la (no) reducción.

Dado un esquema $X$ definido sobre un cuerpo $k$ y dada una extensión $L/k$, podemos asociarle un esquema $X_L$ (llamado <<cambio de base>>) definido sobre $L$.
Están bien estudiadas las propiedades que $X_L$ preserva de $X$, o que propiedades de $X_L$ son reflejadas en $X$.
% Mejor aún, uno puede definir el cambio de base por cualquier esquema $S$ definido sobre $k$.

% Hay que destacar ciertas definiciones sobre morfismos $f \colon X \to Y$ entre esquemas:
% \begin{description}
% 	\item[De tipo finito:] Si todo punto $x \in X$ está contenido en un abierto afín $x \in U = \Spec B \subseteq X$
% 		y existe un abierto afín $V = \Spec A \subseteq Y$ tal que la restricción $f|_U \colon \Spec B \to \Spec A$ induce un homomorfismo de anillos
% 		$A \to B$ de tipo finito.
% 	\item[Propio:] Si es de tipo finito, universalmente cerrado (i.e.\ todo cambio de base es cerrado) y separado.
% 	\item[Entero:] 
% \end{description}

% Hay que destacar ciertas definiciones:
\begin{enumerate}[resume]
	\item Un esquema separado $X$ es tal que los morfismos $X \dashto Y$ admiten extensiones únicas.
	\item Un morfismo proyectivo es propio.
	% \item Un anillo es a un esquema $X$, lo que un módulo es a un $\mathscr{O}_X$-módulo.
	% 	Los $\mathscr{O}_X$-módulos también reflejan información del esquema, como la amplitud el que tan proyectivo es.
	\item La categoría de $\mathscr{O}_{\Spec A}$-módulos cuasicoherentes sobre un esquema afín $\Spec A$
		es equivalente a la categoría de $A$-módulos.
	\item Un esquema normal $X$ es regular en codimensión 1 y,
		por tanto, toda aplicación racional $f\colon X \dashto Y$ puede extenderse de modo que $X \setminus \Dom f$ tenga codimensión 2.
	\item La regularidad se puede verificar mediante el rango de una matriz llamada el jacobiano.
	% \item Los esquemas normales pueden extender morfismos a 
\end{enumerate}

\chapter{Esquemas}

\section{El espectro de un anillo}

\begin{mydefi}
	Sea $A$ un anillo.%
	\footnote{En éste libro todos los anillos se suponen conmutativos y unitarios.}
	Llamamos su \strong{espectro primo}\index{espectro!primo} (resp. \strong{espectro maximal}\index{espectro!maximal}),
	denotado $\Spec A$ (resp. $\mSpec A$), al conjunto de sus ideales primos (resp. ideales maximales).
\end{mydefi}

En la geometría algebraica moderna se piensa en el espectro no sólo como una colección de ideales,
sino también como \textit{puntos} de un \textit{espacio} a los cuales podemos someter a ecuaciones.
De ese modo, dado un ideal primo $\mathfrak{p} \nsl A$, denotaremos por $x_{\mathfrak{p}}$ al mismo ideal pero pensado como un punto,
o viceversa, dado un punto $x \in \Spec A$ denotaremos por $\mathfrak{p}_x$ al mismo punto pensado como un ideal.

Bajo ésta perspectiva, un elemento $f \in A$ es una función sobre $\Spec A$,
donde a todo punto $x \in \Spec A$ le asocia $f(x) := f \bmod{\mathfrak{p}_x} \in A/\mathfrak{p}_x$.
Entonces, podemos preguntarnos ¿cuándo $f (x)$ se anula?
Y, por definición, ello equivale a preguntarse si $f \in \mathfrak{p}_x$. Se verifica lo siguiente:
\begin{enumerate}
	\item La función $0 \in A$ se anula en todo punto $x \in \Spec A$.
	\item Si dos funciones se anulan en $x$, entonces su suma también.
	\item Si una función se anula en $x$, entonces sus múltiplos también.
	\item Si un producto de funciones se anula en $x$, entonces alguna debe anularse en $x$.
\end{enumerate}

\begin{mydef}
	Dado un conjunto de funciones $S \subseteq A$ se define su \textit{lugar de ceros} como todos los puntos en donde se anula:
	$$ \VV(S) := \{ x \in \Spec A : \forall f \in S \quad f (x) = 0 \} = \{ x \in \Spec A : S \subseteq \mathfrak{p}_x \}. $$
\end{mydef}
\begin{ex}
	Fijemos $A = \Z$. Ahora, podemos pensar el 6 como función y notamos que el 6 sólo se anula en $x_2$ y $x_3$. Luego:
	$$ \VV(6) = \{ x_2, x_3 \}. $$
\end{ex}

\begin{lem}
	Dado un anillo $A$ se cumple:
	\begin{enumerate}
		\item Si $S_1 \subseteq S_2 \subseteq A$, entonces $\VV(S_1) \supseteq \VV(S_2)$.
		\item Si $\mathfrak{a} = (S)$ es el ideal generado por $S$, entonces $\VV(S) = \VV(\mathfrak{a})$.
		\item $\VV(1) = \emptyset$ y $\VV(0) = \Spec A$.
		\item Si $\{ \mathfrak{a}_i \}_{i\in I}$ es una familia de ideales de $A$, entonces:
			$$ \bigcap_{i\in I} \VV(\mathfrak{a}_i) = \VV\left( \bigcup_{i\in I} \mathfrak{a}_i \right)
			= \VV\left( \sum_{i\in I} \mathfrak{a}_i \right). $$
		\item Dados $\mathfrak{a, b} \nsle A$, entonces
			$$ \VV(\mathfrak{a}) \cup \VV(\mathfrak{b}) = \VV(\mathfrak{a \cap b}) = \VV(\mathfrak{a \cdot b}). $$
	\end{enumerate}
\end{lem}

\begin{mydef}
	Los conjuntos de la forma $\VV(\mathfrak{a})$ son exactamente los cerrados de una única topología sobre $\Spec A$,
	llamada la \strong{topología de Zariski}\index{topología!de Zariski}.
\end{mydef}
\begin{ex}
	\begin{itemize}
		\item Por un teorema de Krull, todo anillo no nulo posee un ideal maximal, en particular, un ideal primo.
			Luego $\Spec A = \emptyset$ syss $A = 0$.
		\item Si $A = k$ es un cuerpo, entonces sus ideales son solo $\{ (0), A \}$, por lo que, $\Spec k = \{ (0) \}$ y
			necesariamente adquiere la topología discreta.
		\item Sea $A := k[\varepsilon]/(\varepsilon^2)$, donde $\varepsilon$ es una indeterminada.
			Trivialmente existe una proyección $k[\varepsilon] \epicto A$ que prueba que el único ideal primo de $A$ es
			$(\varepsilon)$; nótese que ni siquiera $(0)$ es primo, pues $\varepsilon \cdot \varepsilon = 0$.
			Así $\Spec A$ es un punto, incluso cuando $A$ no es un cuerpo.
	\end{itemize}
\end{ex}

Para todo $f \in A$ denotamos
$$ \DD(f) := \Spec A \setminus \VV(f) = \{ x \in \Spec A : f \notin \mathfrak{p}_x \}. $$
Compare los dos últimos resultados con el primer capítulo.
\begin{prop}
	Los abiertos de la forma $\DD(f)$ con $f \in A$ forman una base de $\Spec(A)$.
\end{prop}
\begin{prob}
	Describir $\Spec\Z$ como espacio topológico.
\end{prob}

\begin{mydef}
	Dado $X \subseteq \Spec A$ podemos definir:
	$$ \II(X) := \{ f \in A : \forall x \in X \; f(x) = 0 \} = \{ f \in A : \forall x \in X \; f \in \mathfrak{p}_x \} = \bigcap_{x\in X} \mathfrak{p}_x. $$
\end{mydef}

\begin{prop}
	Dado un anillo $A$ y $X \subseteq Y \subseteq \Spec A$ se cumple:
	\begin{enumerate}
		\item $\II(X)$ es un ideal radical.
		\item Si $X \subseteq Y \subseteq \Spec A$, entonces $\II(X) \supseteq \II(Y)$.
		\item $\II(\Spec A) = \nilrad$ e $\II(\emptyset) = A$.
		\item Para todo $\mathfrak{a} \nsle A$ se cumple que $\mathfrak{a} \subseteq \II( \VV(\mathfrak{a}) )$.
		\item Para todo $X \subseteq \Spec A$ se cumple que $X \subseteq \VV( \II(X) )$.
	\end{enumerate}
\end{prop}

También hay resultados análogos al teorema de ceros de Hilbert, aunque aquí éstos resultan radicalmente más simples de probar:
\begin{thm}\label{thm:ring_nullstellensatz}
	Sea $A$ un anillo, se cumplen:
	\begin{enumerate}
		\item Sea $\mathfrak{a} \nsle A$, entonces $\II( \VV(\mathfrak{a}) ) = \rad\mathfrak{a}$ (cf. \cite{Alg}, Teo. 6.26).
		\item Sea $X \subseteq \Spec A$, entonces $\VV( \II(X) ) = X$ su clausura de Zariski.
	\end{enumerate}
\end{thm}
Los paralelos con la geometría algebraica clásica son evidentes.
El anillo $A$ se comporta como el álgebra polinomial $k[\vec x]$ mientras que el $\Spec A$ se comporta como el espacio afín $\A^n(k)$.
Veamos algunas propiedades de la topología del $\Spec A$:
\begin{cor}\label{thm:spec_closure}
	Sea $A$ un anillo y $X := \Spec A$. Entonces:
	\begin{enumerate}
		\item Para todo $x \in X$ se cumple que $\overline{\{ x \}} = \VV(\mathfrak{p}_x)$.
			Así pues, $y \in \overline{\{ x \}}$ syss $\mathfrak{p}_y \supseteq \mathfrak{p}_x$.
		\item Un punto $x \in X$ es cerrado syss $\mathfrak{p}_x$ es maximal.
	\end{enumerate}
\end{cor}

\begin{ex}
	Sea $k$ un cuerpo algebraicamente cerrado.
	Estudiemos $\Spec(k[x])$: como $k[x]$ es un DIP, entonces sus ideales primos son
	generados por polinomios irreducibles, o bien son el ideal nulo.
	Como $k$ es algebraicamente cerrado, entonces sus polinomios irreducibles son de la forma $x - \alpha$ para cada $\alpha \in k$.
	Los puntos $\mathfrak{p}_\alpha := (x - \alpha)$ son cerrados, pues son
	ideales maximales, pero el punto $\xi := (0)$ es tal que $\overline{\{ \xi \}} = \Spec(k[x])$.
\end{ex}
\begin{ex}
	En el $\Spec\Z$ sucede algo parecido.
	Los ideales de la forma $(p) =: x_p$ son maximales, luego los $x_p$'s son puntos cerrados, no obstante, el
	punto $\xi := (0)$ es denso.
	\todo{Rehacer figura de $\Spec\Z$.}
\end{ex}

% 2

% 3

% 5

% 7 11

% \Spec Z
% \xi

% Figura 3.1. Espectro de Z.
\begin{ex}
	Sea $A$ un dominio de valuación discreta, es decir, un anillo local que es DIP.
	Como es DIP, sus primos son o bien $(0) =: \xi$, o bien los maximales, pero como es local,
	sólo tiene un maximal $\mathfrak{m} =: s$.
	Así $\Spec A = \{ \xi, s \}$, donde $\xi$ es abierto (y denso), y $s$ es un punto cerrado.
	Topológicamente, $\Spec A$ es homeomorfo al espacio de Sierpiński (cfr. \cite{Top}, ej.~4).
\end{ex}
\begin{mydef}
	Sean $x, y \in X$ un par de puntos en un espacio topológico.
	Se dice que $y$ es una \strong{especialización}\index{especialización} de $x$, o que $x$ es una generización de $y$,
	denotado $x \speto y$, si $y \in \overline{\{ x \}}$.
	Se dice que $x$ es un \strong{punto genérico}\index{punto!genérico} si $z \speto x$ implica $z = x$.
\end{mydef}
Del ejemplo vemos que si $A$ es un dominio íntegro, entonces siempre el punto $\xi := (0)$ es genérico.
Más generalmente, en un espacio $T_0$, un punto denso es genérico.

\begin{prop}\label{thm:generic_in_spec}
	Sea $F \subseteq \Spec A$. Se cumplen:
	\begin{enumerate}
		\item Un cerrado $F \subseteq \Spec A$ es irreducible syss $\II(F)$ es un ideal primo.
		\item Un cerrado $F \subseteq \Spec A$ es una componente irreducible syss $\II(F)$ es un primo minimal.
		\item El espacio $\Spec A$ es irreducible syss posee un único primo minimal.
			En particular, el $\Spec A$ siempre es irreducible si $A$ es un dominio íntegro.
	\end{enumerate}
\end{prop}
\begin{cor}
	Sea $A$ un anillo arbitrario.
	Todo cerrado irreducible de $\Spec A$ admite exactamente un punto denso.
	En consecuente, existe una biyección entre cerrados irreducibles y puntos genéricos.
\end{cor}
\begin{proof}
	Sea $F$ un cerrado irreducible, entonces $\mathfrak{p} := \II(F)$ es un ideal primo por la proposición anterior.
	Luego claramente, $x_{\mathfrak{p}} \in \overline{\{ x_{\mathfrak{p}} \}} = \VV(\mathfrak{p}) = \VV(\II(F)) = F$.
	Para todo $y \in F$ se cumple que $\mathfrak{q}_y \subseteq \mathfrak{p}$, así que si $y = x_{\mathfrak{p}}$
	se cumple que $\mathfrak{q}_y \supset \mathfrak{p}$, y finalmente, $y \speto z$ syss $\mathfrak{q}_y \subseteq \mathfrak{q}_z$,
	por lo que, $y \not\speto x_{\mathfrak{p}}$.
\end{proof}
Los puntos genéricos serán de utilidad más adelante.

% La teoría de la dimensión en éste contexto da:
% \begin{prop}
% 	Sea $A$ un anillo.
% 	\begin{enumerate}
% 		\item $\dim(\Spec A) = \kdim A = \kdim(A/\nilrad)$.
% 		\item Para todo primo $\mathfrak{p} \nsl A$ se cumple que $\kdim(A_{\mathfrak{p}}) = \alt\mathfrak{p} = \codim(\VV(\mathfrak{p}), \Spec A)$.
% 		\item $\kdim A = \sup\{ \kdim(A_{\mathfrak{m}}) : \mathfrak{m} \in \mSpec A \}$.
% 	\end{enumerate}
% \end{prop}

\begin{thm}
	$\Spec A$ es compacto.
\end{thm}
\begin{proof}
	Sea $\{ U_i \}_{i\in I}$ un cubrimiento abierto de $\Spec A$ y elijamos suficientes $f_j \in A$ tales que $\DD(f_j) \subseteq U_i$ para algún $i \in I$
	y tal que ${ \DD(f_j) }_{j\in J}$ es un cubrimiento abierto de $\Spec A$ (lo cual es válido pues los $\DD(f_j)$'s son una base).
	El hecho de que $\Spec A = \bigcup_{j\in J} \DD(f_j)$ equivale a decir que el ideal generado por $\{ f_j \}_{j\in J}$ es $A$, luego existen
	$f_{j_1}, \dots, f_{j_n}$ y $a_{j_1}, \dots, a_{j_n} \in A$ tales que
	$$ a_{j_1} f_{j_1} + \cdots + a_{j_n} f_{j_n} = 1, $$
	de modo que $A = (f_{j_1}, \dots, f_{j_n})$ y $\Spec A = \bigcup_{\ell=1}^n \DD( f_{j_\ell} )$.
\end{proof}
\begin{thm}
	Si $A$ es un anillo noetheriano, entonces $\Spec A$ es noetheriano.
\end{thm}
El recíproco puede fallar.

\begin{cor}
	En un anillo noetheriano $A$, todo ideal radical $\mathfrak{r} \nsl A$ es una intersección de finitos ideales primos.
\end{cor}
\begin{prop}
	Dado un homomorfismo de anillos $\varphi \colon A \to B$, éste induce una aplicación (la contracción de ideales, cfr. \cite{Alg} \S6.1.3):
	\begin{multicols}{2}
		\begin{align*}
			\varphi^a \colon \Spec B &\longrightarrow \Spec A \\
			x &\longmapsto x^c = f^{-1}[\mathfrak{p}_x]
		\end{align*}
		\begin{center}
			\begin{tikzcd}
				A \dar["\varphi"', ""{name=s}] & \Spec A \\
				B                              & \Spec B \uar["\varphi^a"', ""{name=f}]
				\ar[from=s, to=f, Rightarrow, "\Spec"]
			\end{tikzcd}
		\end{center}
	\end{multicols}
	\begin{enumerate}
		\item $\varphi^a$ es continua. Más aún, $(-)^a \colon \mathsf{CRing} \to \mathsf{Top}$ es un funtor contravariante.
		\item Si $\varphi$ es suprayectiva, entonces $\varphi^a$ induce un homeomorfismo entre $\Spec B$ y el cerrado $\VV(\ker\varphi)$,
			luego $\varphi^a$ es un encaje (topológico) cerrado.
		\item Dado un ideal $\mathfrak{a} \nsl A$, entonces la proyección canónica $\pi \colon A \to A/\mathfrak{a}$ da lugar al encaje
			$\pi^a \colon \Spec(A/\mathfrak{a}) \to \Spec(A)$ que prueba que $\VV(\mathfrak{a})$ es homeomorfo a $\Spec(A/\mathfrak{a})$.
		\item Si $S$ es un sistema multiplicativo de $A$ que no contiene divisores de cero, entonces la inclusión $\lambda \colon A \to S^{-1}A$
			induce un homeomorfismo entre $\Spec(S^{-1} A)$ y $\{ \mathfrak{p} \in \Spec A : \mathfrak{p} \cap S = \emptyset \}$.
		\item En particular, dado $f \in A$, el abierto $\DD(f)$ con la topología subespacio de $\Spec A$ es homeomorfo al espacio $\Spec(A[1/f])$.
	\end{enumerate}
\end{prop}
\begin{proof}
	Probaremos la 2. Para ello, nótese que por el primer teorema de isomorfismos (cfr. \cite{Alg}, Teo. 2.23)
	se cumple que $\overline{\varphi}\colon A/\ker \varphi \to B$ es un isomorfismo, y los ideales primos de $A/\ker \varphi$ están en correspondencia con
	$\VV(\ker \varphi) \subseteq \Spec A$.
\end{proof}

Es sabido que la extensión de ideales no determina una función entre los espectros (dé un ejemplo).
El inciso 3 da el siguiente corolario:
\begin{cor}
	Dado un anillo $A$ se cumple que $\Spec A$ y $\Spec(A/\nilrad)$ son espacios topológicos homeomorfos.
\end{cor}
\begin{cor}
	Sea $\varphi \colon A \to B$ un homomorfismo de anillos.
	La función continua $\varphi^a \colon \Spec B \to \Spec A$ es dominante (i.e., su imagen es densa) syss $\ker \varphi$ es nilpotente.
	En particular, si $A$ es reducido, $\varphi^a$ es dominante syss $\varphi$ es monomorfismo.
\end{cor}
\begin{proof}
	Sabemos que la clausura de la imagen de $\varphi^a$ es $\VV(\ker \varphi) \subseteq \Spec A$.
	Así que, exigir que $\varphi^a$ sea dominante equivale a que $\VV(\ker\varphi) = \Spec A$, lo que equivale, por la proposición 3.8,
	a que $\ker\varphi \subseteq \nilrad$.
\end{proof}

\subsection{Espectro homogéneo}
\begin{mydef}
	Sea $A$ un anillo ($\N$-)graduado $A = \bigoplus_{d\in\N} A_d$ y recordemos que el ideal irrelevante es $A_+ := \bigoplus_{d > 0} A_d$.
	Llamamos su \strong{espectro homogéneo}\index{espectro!homogéneo}, denotado $\Proj A$, al conjunto de los ideal primos $\mathfrak{p} \nsl A$
	homogéneos tales que $\mathfrak{p} \nsupseteq A_+$, llamados \textit{primos relevantes}.

	Dado un conjunto $S \subseteq A$, definimos su lugar de ceros homogéneo:
	$$ \VV_+(S) := \{ x \in \Proj A : \forall f \in S \quad f (x) = 0 \} = \{ x \in \Proj A : \mathfrak{p}_x \supseteq S \}. $$
\end{mydef}

\addtocounter{thmi}{1}
\begin{slem}
	Dado un anillo graduado $A$ se cumple:
	\begin{enumerate}
		\item Si $S_1 \subseteq S_2 \subseteq A$, entonces $\VV_+(S_1) \supseteq \VV_+(S_2)$.
		\item Si $\mathfrak{a} = (S)^h$ es el ideal homogéneo generado por $S$, entonces $\VV_+(S) = \VV_+(\mathfrak{a})$.
		\item $\VV_+(1) = \VV_+(A_+) = \emptyset$ y $\VV_+(0) = \Proj A$.
		\item Si $\{ \mathfrak{a}_i \}_{i\in I}$ es una familia de ideales homogéneos de $A$, entonces:
			$$ \bigcap_{i\in I} \VV_+(\mathfrak{a}_i) = \VV_+\left( \bigcup_{i\in I} \mathfrak{a}_i \right)
			= \VV_+\left( \sum_{i\in I} \mathfrak{a}_i \right). $$
		\item Dados $\mathfrak{a, b} \nsle A$ homogéneos, entonces
			$$ \VV_+(\mathfrak{a}) \cup \VV_+(\mathfrak{b}) = \VV_+(\mathfrak{a \cap b}) = \VV_+(\mathfrak{a \cdot b}). $$
	\end{enumerate}
\end{slem}
\addtocounter{thmi}{-1}
\begin{mydef}
	Sea $A$ un anillo graduado y $\mathfrak{a} \nsle A$ un ideal.
	Se define su \strong{parte homogénea} como
	$$ \mathfrak{a}^h := \bigoplus_{d\in\N} (\mathfrak{a} \cap A_d) $$
	Es claro que a es homogéneo syss $\mathfrak{a} = \mathfrak{a}^h$.
\end{mydef}

\begin{lem}
	Sea $A$ un anillo graduado y $\mathfrak{p} \nsl A$ un primo.
	Entonces $\mathfrak{p}^h$ también es primo.
\end{lem}
\begin{proof}
	Sean $a, b \in A$ con descomposición homogénea:
	$$ a = \sum_{d=0}^{n} a_d, \qquad b = \sum_{d=0}^{m} b_d, $$
	tales que $ab \in \mathfrak{p}^h$, queremos probar que alguno está en $\mathfrak{p}^h$ .

	La prueba será por inducción fuerte sobre $n + m$.
	Si $n + m = 0$, entonces $a = a_0$, $b = b_0$ son homogéneos y trivialmente alguno está en $\mathfrak{p} \subseteq \mathfrak{p}^h$.
	Si no, su producto se escribe como 
	$$ a \cdot b = a_n b_m + \sum_{d=0}^{n+m-1} \sum_{i+j=d} a_i b_j. $$
	donde $a_i b_j$ es homogéneo de grado $< n + m$. Luego $a_n b_m \in \mathfrak{p} \cap A_{n+m}$,
	por lo que $a_n \in \mathfrak{p}$ o $b_m \in \mathfrak{p}$, sin perdida de generalidad supongamos el primer caso.
	Así $(a - a_n )b \in \mathfrak{p}^h$ y su descomposición homogénea llega hasta elementos de grado $< n + m$, luego o bien $a - a_n \in \mathfrak{p}^h$
	o $b \in \mathfrak{p}^h$ por hipótesis inductiva.
\end{proof}

\begin{prop}
	Sea $A$ un anillo graduado.
	\begin{enumerate}
		\item Si $\mathfrak{a, b} \nsle A$ son homogéneos, entonces $\VV_+(\mathfrak{a}) \subseteq \VV_+(\mathfrak{b})$
			syss $\mathfrak{b} \cap A_+ \subseteq \rad\mathfrak{a}$.
		\item $\Proj A = \emptyset$ syss $A_+$ es nilpotente.
	\end{enumerate}
\end{prop}
\begin{proof}
	\begin{enumerate}
		\item $\impliedby.$ Sea $\mathfrak{p} \in \VV_+(\mathfrak{a})$, entonces $\mathfrak{p} \supseteq \rad\mathfrak{a}
			\supseteq \mathfrak{b} \cap A_+ \supseteq \mathfrak{b}$, por lo que $\mathfrak{p} \in \VV_+(\mathfrak{b})$.

			$\implies.$ Sea $\mathfrak{p} \in \VV(\mathfrak{a})$, entonces $\mathfrak{p}^h \supseteq \mathfrak{a}^h = \mathfrak{a}$.
			Si $\mathfrak{p}^h \nsupseteq A_+$, entonces $\mathfrak{p}^h \in \VV_+(\mathfrak{a}) \subseteq \VV_+(\mathfrak{b})$, luego
			$\mathfrak{p} \supseteq \mathfrak{p}^h \supseteq \mathfrak{b} \supseteq \mathfrak{b} \cap A_+$ y, por ende,
			$$ \mathfrak{b} \cap A_+ \subseteq \bigcap_{\mathfrak{p} \in \VV(\mathfrak{a})} \mathfrak{p} = \rad\mathfrak{a}. $$

		\item Si $\Proj A = \VV_+(0) \subseteq \VV_+(A_+) = \emptyset$, entonces $A_+ \subseteq \rad(0) = \nilrad$. \qedhere
	\end{enumerate}
\end{proof}
\begin{mydef}
	Sea $A$ un anillo graduado y $f \in A$ homogéneo. Definimos $\DD_+(f) := \Proj A \setminus \VV_+(f)$.
\end{mydef}
Recuérdese que:
\begin{mydef}
	Sean $A, B$ un par de anillos graduados.
	Un \strong{homomorfismo de anillos graduados}\index{homomorfismo!de anillos graduados} de grado $r \ge 1$ es un
	homomorfismo de anillos $\varphi \colon A \to B$ tal que $\varphi[A_d] \subseteq B_{rd}$ para todo $d \in \N$.
\end{mydef}

\begin{prop}
	Sea $A$ un anillo graduado, $f \in A$ homogéneo de grado $r > 0$, sea $g \in A$ tal que $\DD_+(g) \subseteq \DD_+(f )$
	y sea $\alpha := g^r f^{-\deg g} \in A_{(f)}$. Entonces:
	\begin{enumerate}
		\item Existe un homeomorfismo $\theta \colon \DD_+(f ) \to \Spec(A_{(f)})$.
		\item $\theta[\DD_+(g)] = \DD(\alpha)$.
		\item Existe un homomorfismo canónico $A_{(f)} \to A_{(g)}$ e induce un isomorfismo $(A_{(f)} )[1/\alpha] \cong A_{(g)}$.
		\item Si $a$ es homogéneo en $A$, entonces $\theta[\VV_+(a) \cap \DD_+(f )] = \VV( \mathfrak{a}_{(f)} )$,
			donde $\mathfrak{a}_{(f)} := \mathfrak{a} A[1/f ] \cap A_{(f)}$.
	\end{enumerate}
\end{prop}
\begin{proof}
	\begin{enumerate}
		\item[\rm 1. y 2.] Nótese que $\Proj A \subseteq \Spec A$ hereda la topología subespacio.
			La inclusión $\iota \colon A_{(f)} \to A[1/f ]$ induce la aplicación continua
			$\Spec(A[1/f ]) \approx \DD(f ) \to \Spec(A_{(f)})$ y definimos $\theta \colon \DD_+(f ) := \DD(f )\cap \Proj A \to \Spec(A_{(f)})$
			como su restricción.

			Veamos que $\theta$ es suprayectiva: Sea $\mathfrak{p} \in \Spec(A_{(f)} )$, entonces es fácil
			ver que $\mathfrak{q} := \rad(\mathfrak{p} A[1/f])$ es primo en $A[1/f]$.
			Nótese que $A[1/f]$ es una $A_{(f)}$-álgebra graduada en sentido canónico, donde los elementos homogéneos de grado $n$ son
			de la forma $bf^{-N}$ donde $b \in A$ es homogéneo de grado $\deg b = N r + n$.
			Así $\mathfrak{q}$ es un primo homogéneo de $A[1/f]$.
			Considere $\lambda \colon A \to A[1/f ]$ el homomorfismo canónico, entonces es un homomorfismo de anillos graduados,
			y así $\mathfrak{r} := \lambda^{-1}[\mathfrak{q}]$ es un primo homogéneo de $A$ y $\mathfrak{r} \in \DD_+(f)$.
			Finalmente, queda al lector probar que $\theta(\mathfrak{r}) = \mathfrak{p}$.

			Veamos que $\theta$ es inyectiva: Sean $\mathfrak{p}, \mathfrak{p}' \in \DD_+(f )$ tales que $(\mathfrak{p} A[1/f]) \cap A_{(f)}
			= (\mathfrak{p}' A[1/f ]) \cap A_{(f)}$, entonces para todo $b \in \mathfrak{p}$ homogéneo se cumple que
			$b^r f^{-\deg b} \in (\mathfrak{p}A[1/f]) \cap A_{(f)} \subseteq \mathfrak{p}' A[1/f]$ de modo que $b \in \mathfrak{p}'$
			y $\mathfrak{p \subseteq p}'$ y, por simetría, $\mathfrak{p = p}'$.

			Veamos que $\theta$ es abierta: Basta notar que $\theta[\DD_+(g)] = \DD(\alpha)$ (¿por qué?)
			y emplear que los abiertos principales son una base. Finalmente, toda
			biyección abierta y continua es un homeomorfismo.

		\item Como $\DD_+(g) \subseteq \DD_+(f )$, el lema anterior nos da que $\rad(g)\cap A_+ \supseteq (f)$,
			luego para todo $a \in A$ se cumple que $g^n = f a$ para algún $n \in \N$.
			En particular, fijemos $a$ homogéneo y así determinamos el homomorfismo $bf^{-N} \mapsto (ba^N)g^{-nN}$.
			Queda de ejercicio para el lector verificar que el homomorfismo está bien definido y que determina un homeomorfismo entre los espectros.
			\qedhere
	\end{enumerate}
\end{proof}

\begin{prop}\label{thm:proj_spec_morphisms}
	Dado un homomorfismo de anillos graduados $\varphi \colon A \to B$ con $M := (A_+)^e = \varphi[A_+]B$, éste induce una aplicación:
	\begin{align*}
		\varphi^a \colon \DD_+(M) &\longrightarrow \Proj A \\
		x &\longmapsto x^c = \varphi^{-1}[\mathfrak{p}_x]
	\end{align*}
	\begin{enumerate}
		\item $\varphi^a$ es continua.
		\item Si $\varphi$ es suprayectiva, entonces $\varphi^a$ induce un homeomorfismo entre $\DD_+(M)$ y el cerrado $\VV(\ker \varphi)$,
			luego $\varphi^a$ es un encaje cerrado.
		\item Sea $\mathfrak{a} \nsl A$ un ideal homogéneo, entonces la proyección canónica $\pi \colon A \to A/\mathfrak{a}$ da lugar al encaje
			$\pi^a \colon \Proj(A/\mathfrak{a}) \to \Proj(A)$ que prueba que $\VV_+(\mathfrak{a})$ es homeomorfo a $\Proj(A/\mathfrak{a})$.
		\item Si $S$ es un sistema multiplicativo de $A$ que no contiene divisores de cero, entonces la inclusión $\lambda \colon A \to A_{(S)}$
			induce un homeomorfismo entre $\Spec(A_{(S)})$ y ${ \mathfrak{p} \in \Proj A : \mathfrak{p} \cap S = \emptyset }$.
		\item En particular, dado $f \in A$ homogéneo, el abierto $\DD_+(f)$ con la topología subespacio de $\Proj A$
			es homeomorfo al espacio $\Spec(A_{(f)})$.
	\end{enumerate}
\end{prop}
Ojo que para el inciso 2 hay que verificar que $\pi[A_+] = (A/\mathfrak{a})_+$.

\section{Haces}
Comenzamos con las nociones de haces y espacios anillados.
En esencia, un haz determina una familia de funciones que tienen ciertas <<propiedades locales>> con respecto a un espacio topológico.
\begin{mydefi}
	Dado un espacio topológico $X$, un \strong{prehaz}\index{prehaz} es un funtor contravariante $\mathscr{F} \colon \Open(X) \to \catC$,
	donde $\catC$ es una categoría.
	Un prehaz de conjuntos, de grupos, de anillos, etc., es un prehaz donde $\catC$ es la categoría de conjuntos, de grupos, de anillos, etc.
	Si $\catC$ es una categoría concreta,%
	\footnote{Una categoría cuyos objetos son <<conjuntos con estructura>>.
	Formalmente, una categoría $\catC$ con un funtor fiel canónico $\catC \to \mathsf{Set}$ llamado funtor olvidadizo (cfr. \cite{CatTh}, def.~1.15).}
	diremos que $\mathscr{F}$ es un prehaz concreto y, entonces $\mathscr{F}(U)$ es un conjunto (con estructura)
	y sus elementos se denominan \strong{secciones}\index{sección} sobre $U$; también denotamos $\Gamma(U, F)$ al conjunto de secciones sobre $U$.
	Las secciones sobre $X$ se dice \strong{secciones globales}\index{sección!global}.

	En general, dados $U \subseteq V$ abiertos de $X$, denotamos por $\rho^V_U \colon \mathscr{F}(V) \to \mathscr{F}(U)$
	la imagen de la flecha de inclusión, a los que llamamos \strong{restricciones}.
	Dada una sección $s \in \mathscr{F}(V)$ solemos denotar $s|_U := \rho^V_U(s)$.
\end{mydefi}

\begin{mydef}
	Sea $X$ un espacio topológico y $\mathscr{F}$ un prehaz concreto sobre $X$.
	Se dice que una familia $(s_i \in \mathscr{F}(U_i))_{i\in I}$ de secciones es \strong{compatible}\index{compatible (familia de secciones)}
	si para todo $i, j \in I$ se cumple que $s_i|_{U_i\cap U_j} = s_j|_{U_i\cap U_j}$.

	Decimos que un prehaz concreto $\mathscr{F}$ sobre $X$ es un haz%
	\footnote{fr. \textit{faisceau}, eng. \textit{sheaf}.}
	si para toda familia $(x_i \in \mathscr{F}(U_i))_{i\in I}$ compatible existe un único elemento $x \in \mathfrak{F}\big( \bigcup_{i\in I} U_i \big)$
	tal que $x|_{U_i} = x_i$ para todo $i \in I$ (<<axioma de pegado>>).\index{axioma!de pegado}
\end{mydef}

\citeauthor{hartshorne:algebraic}~\cite{hartshorne:algebraic} trabaja exclusivamente con prehaces de grupos en donde la condición de unicidad la expresa
diciendo que si un elemento se restringe al 0 en todos los $U_i$ es porque dicho elemento es el 0.

Hay una manera de escribir la condición de ser haz mediante flechas:
Sea $\{ U_i \}_{i\in I}$ una familia de abiertos tal que $U = \bigcup_{i\in I} U_i$, entonces considere las siguientes flechas:
\begin{equation}
	\begin{tikzcd}[sep=large]
		\displaystyle
		\mathscr{F}(U) \rar["\alpha"] & \prod_{i\in I} \mathscr{F}(U_i) \rar["\beta", shift left] \rar["\gamma"', shift right] &
		\prod_{i, j\in I} \mathscr{F}(U_i \cap U_j)
	\end{tikzcd}
	\label{cd:sheaf_equalizer}
\end{equation}
donde
\begin{gather*}
	\alpha(s) = \big( \rho_{U_i}^U(s) \big)_{i\in I}, \qquad \beta\big( (s_i)_{i\in I} \big) = ( \rho_{U_i\cap U_j}^{U_i}(s_i) )_{i, j} \\
	\gamma\big( (s_i)_{i\in I} \big) = ( \rho_{U_i\cap U_j}^{U_j}(s_i) )_{i, j}.
\end{gather*}
Y exigimos que el ecualizador de $\beta, \gamma$ sea $\alpha$.
Efectivamente, que $\alpha \circ \beta = \alpha \circ \gamma$ es la condición que una familia de secciones sea compatible.
Ésta condición puede sonar un tanto complicada, pero nos permite definir la categoría de haces sobre una categoría cualquiera.
% además la sucesión exacta dada por \eqref{cd:sheaf_equalizer} será útil más adelante para algunos cálculos.

Podemos extraer el siguiente caso:
\begin{prop}
	Sea $X$ un espacio topológico y $\mathscr{F}$ un prehaz de $A$-módulos sobre $X$.
	Entonces $\mathscr{F}$ es un haz syss para cada $U$ abierto y cada cubrimiento $\bigcup_{i\in I} U_i = U$ la sucesión
	\begin{center}
		\begin{tikzcd}[sep=large]
			\displaystyle
			0 \rar & \mathscr{F}(U) \rar["\alpha"] & \prod_{i\in I} \mathscr{F}(U_i) \rar["\delta"] & \prod_{i, j\in I} \mathscr{F}(U_i \cap U_j)
		\end{tikzcd}
	\end{center}
	es exacta, donde:
	\[
		\alpha(s) = \big( \rho_{U_i}^U(s) \big)_{i\in I}, \qquad \delta\big( (s_i)_{i\in I} \big) = ( s_i|_{U_i\cap U_j} - s_j|_{U_i\cap U_j} )_{i, j}.
	\]
\end{prop}

\begin{prop}
	Sea $X$ un espacio topológico.
	Todo haz $\mathscr{F} \in \mathsf{Sh}(X; \catC)$ satisface que $\mathscr{F}(\emptyset)$ es un objeto final.
\end{prop}

\begin{prop}
	Sean $\catC$ una categoría y $F \colon \catC \to \mathsf{Set}$ un funtor tales que:
	\begin{enumerate}
		\item $F$ es fiel.
		\item $\catC$ es completa y $F$ preserva límites inversos.
		\item $F$ refleja isomorfismos.
	\end{enumerate}
	Entonces, para todo espacio topológico $X$ y todo prehaz $\mathscr{O}$ con codominio $\catC$ se cumple
	que $\mathscr{O}$ es un haz en $\catC$ syss $\mathscr{O \circ F}$ es un haz de conjuntos.
\end{prop}
\begin{proof}
	$\implies$. Claramente $\mathscr{O \circ F}$ es un prehaz de conjuntos y como los ecualizadores son límites inversos, entonces también es un haz.

	$\impliedby$. Si $\mathscr{O \circ F}$ es un haz de conjuntos, entonces para un cubrimiento $\{ U_i \}_{i\in I}$ de un abierto $U$,
	sea $E$ el ecualizador de las flechas $\beta, \gamma$ de \eqref{cd:sheaf_equalizer}.

	Por definición de prehaz, existe una única flecha $\mathscr{O}(U) \to E$.
	Aplicando el funtor $F$, obtenemos $F(\mathscr{O}(U)) \to F(E)$ y como $F$ preserva límites, entonces $F(E)$ también es un ecualizador,
	por lo que la flecha es un isomorfismo y como $F$ refleja isomorfismos, entonces $\mathscr{O}(U) \to E$ es un isomorfismo.
\end{proof}
En particular, las categorías $\mathsf{Grp, Ab, Mod}_A, \mathsf{Vect}_k, \mathsf{Ring, CAlg}_A$ con el funtor olvidadizo satisfacen las hipótesis anteriores,
así que tenemos un buen criterio para verificar que un prehaz es un haz.

\begin{mydef}
	Sea $X$ un espacio topológico y $\mathfrak{F}$ un prehaz concreto sobre $X$.
	Dado un punto $P \in X$ llamamos la \strong{fibra}\index{fibra!(prehaz)}%
	\footnote{fr. \textit{fibre}, eng. \textit{stalk}.}
	\nomenclature{$\mathscr{F}_P$}{Fibra del prehaz $\mathscr{F}$ en el punto $P \in X$}
	sobre $P$, al límite directo (si existe) $\mathscr{F}_P := \limdir_{U \in\Open(X)} \mathscr{F}(U)$.
	Los elementos de $\mathscr{F}_P$ se dicen \strong{gérmenes locales}%
	\footnote{fr. \textit{germe}.
		La terminología, acuñada por el mismo Grothendieck, refiere a una analogía con la agronomía:
		los haces son literalmente atados de tallos de heno, los cuales a su vez germitan de semillas.
	}
	\index{gérmen local} en $P$.

	Trivialmente para cada entorno $U$ de un punto $P$ existe una única flecha $\rho^U_P \in \Hom\big( \mathscr{F}(U), \mathscr{F}_P \big)$ que
	conmuta con todas las restricciones.
	Dada una sección $s \in \Gamma(U, \mathscr{F})$ se denota $s|_P := \rho^U_P(s)$.
	% y denotamos por $s|_P := \rho_P^U(s)$.
\end{mydef}
Si la categoría de codominio fuese, por ejemplo, cocompleta (i.e., admite límites directos de diagramas),
entonces siempre podríamos asegurar la existencia de fibras sobre los puntos;
pero en general no requerimos tanto, sino que podemos aprovecharnos de que $\Open_P(X)$ es una categoría filtrada.
Veamos una construcción:

\begin{prop}
	Si $X$ es un espacio topológico, $\mathscr{F}$ es un prehaz de conjuntos sobre $X$ y $x \in X$,
	entonces podemos considerar el conjunto $C := \bigcup_{x\in U} \{ U \} \times \mathscr{F}(U)$ y la relación:
	$$ (U, f) \sim (V, g) \iff \exists x \in W \subseteq U \cap V \quad f|_W = g|_W $$
	la cual es de equivalencia. Luego el conjunto cociente $C/\sim$ es la fibra $\mathscr{F}_x$ sobre $x$.
\end{prop}
Ésta misma construcción podemos aplicarla sobre un prehaz de grupos, de anillos, de $A$-módulos verificando también que la relación de equivalencia
conmuta con las operaciones requeridas; ésto se asemeja al lema 1.68.

\begin{ex}
	\begin{itemize}
		\item Dado un par de espacios topológicos $X, Y$, el funtor contravariante $\Hom{\mathsf{Top}}(-, Y) \colon \Open(X) \to \mathsf{Set}$ es
			un haz de conjuntos, llamado el \textit{haz de funciones continuas}, donde la restricción es la restricción usual de conjuntos.

			Más aún, si $Y$ es un grupo topológico (resp. anillo topológico, $A$-módulo topológico) entonces $\Hom(-, Y)$ es un
			haz de grupos (resp. anillos, $A$-módulos).

			Nótese que aquí, el axioma de pegado está satisfecho por el hecho de que pegar funciones continuas compatibles sigue siendo continua;
			por tanto, hasta cierto punto se debe entender que un haz es un funtor contravariante dado por una familia de funciones definidas por una
			propiedad <<local>>.

		\item Dado un par de variedades diferenciales $X, Y$, el funtor contravariante $\Hom_{\mathsf{Man}^\infty} (-, Y) \colon \Open(X) \to \mathsf{Set}$
			es un haz de conjuntos, llamado el \textit{haz de funciones diferenciables}, donde la restricción es la restricción usual de conjuntos.

			Más aún, si $Y$ es un grupo topológico (resp. anillo topológico, $A$-módulo topológico) entonces $\Hom(-, Y)$ es un haz de grupos
			(resp. anillos, $A$-módulos).

			Acá las fibras juegan un rol particular y se llaman \textit{gérmenes de funciones diferenciales} en un punto.

		\item Sea $X$ un espacio topológico arbitrario, $A$ un grupo abeliano arbitrario y $P \in X$ un punto fijo.
			El \textit{haz rascacielos} centrado en $P$ es el haz
			$$ A_X^P(U) :=
			\begin{cases}
				A, & P \in U \\
				0, & P \notin U
			\end{cases} $$
			donde dados $U \subseteq V$ la restricción $\rho^V_U$ es la identidad si $P \in U$ o el homomorfismo nulo si $P \notin U$.

			Las fibras del haz rascacielos son
			$$ A_{X, Q}^P(U) :=
			\begin{cases}
				A, & P     \speto Q \\
				0, & P \not\speto Q
			\end{cases} $$

		\item Sea $X$ un espacio topológico y $A$ un conjunto arbitrario.
			Se le llama el \strong{prehaz constante}\index{prehaz!constante} $A^-_X$ al prehaz de conjuntos que corresponde a un funtor constante.

			Éste prehaz no es (en general) un haz: en primer lugar, uno puede argumentar que para que sea un haz se debe dar que $A^-_X(\emptyset)$
			sea el objeto final de la categoría, y podemos admitir sin problemas dicha condición, pero aún así sigue fallando en general.
			Si $A$ tuviese, por ejemplo, más de un elemento, entonces bastaría encontrar dos secciones distintas definidas sobre abiertos disjuntos
			de $X$ para notar que no es un haz.
			Aquí todas las fibras son $A$.

		\item Sea $X$ un espacio topológico y $A$ un conjunto arbitrario.
			Se le llama el \strong{haz constante}\index{haz!constante} (¡no confundir con el prehaz constante!) $A^+_X$ al haz
			de conjuntos que corresponde al funtor contravariante $\Hom_{\mathsf{Top}}(X, A)$, donde $A$ está visto como un espacio topológico discreto.
	\end{itemize}
\end{ex}

Como los (pre)haces son funtores contravariantes tenemos lo siguiente:
\begin{mydef}
	Sea $X$ un espacio topológico y $\mathscr{F}, \mathscr{G}$ prehaces sobre $X$ con codominio $\catC$.
	Un \strong{morfismo de prehaces}\index{morfismo!de prehaces} $\alpha \colon \mathscr{F} \to \mathscr{G}$ es una transformación natural entre
	los funtores contravariantes, i.e., es una familia de flechas $\alpha(U) \in \Hom_{\catC}(\mathscr{F}(U), \mathscr{G}(U))$ tal que el siguiente
	diagrama siempre conmuta:
	\begin{center}
		\begin{tikzcd}[row sep=large]
			U \dar["\subseteq"', hook] & \mathscr{F}(U)              \rar["\alpha(U)"] & \mathscr{G}(U) \\
			V                          & \mathscr{F}(V) \uar["\rho"] \rar["\alpha(V)"] & \mathscr{G}(V) \uar["\rho"']
		\end{tikzcd}
	\end{center}
\end{mydef}

Con ésto tenemos que:
\begin{prop}
	Los prehaces (resp. haces) sobre un espacio topológico fijo $X$ con codominio fijo $\catC$ (como objetos)
	y los homomorfismos entre ellos (como flechas) conforman una categoría denotada $\mathsf{PSh}(X; \catC)$ (resp. $\mathsf{Sh}(X; \catC)$).
	Se obvia la categoría de codominio cuando $\catC = \mathsf{Ab}$.
\end{prop}
Es notorio que $\mathsf{PSh}(X; \catC)$ es la misma categoría que $\mathsf{Fun}(\Open(X), \catC^{\rm op})$,
y la categoría de $\mathsf{Sh}(X; \catC)$ es una subcategoría plena de $\mathsf{PSh}(X; \catC)$.
% Como los (pre)haces son un tipo de funtor contravariante se comprueba que $\mathsf{PSh}(X; \catC)$ es
% una subcategoría plena de ;
% luego en particular tenemos lo siguiente:
\begin{prop}
	Sea $X$ un espacio topológico y $\mathscr{F}, \mathscr{G} \in \mathsf{PSh}(X; \catC)$ con $\alpha \in \Hom_{\mathsf{PSh}}(\mathscr{F}, \mathscr{G})$.
	Luego, las fibras en el punto $P \in X$ corresponden a un funtor covariante $(-)_P \colon \mathsf{PSh}(X; \catC) \to \catC$:
	\begin{center}
		\begin{tikzcd}[row sep=large]
			\mathscr{F} \dar["\varphi"', ""{name=s}] & \mathscr{F}_P \dar["\varphi_P", ""'{name=f}] \\
			\mathscr{G}                              & \mathscr{G}_P
			\ar[from=s, to=f, Rightarrow, "(-)_P"]
		\end{tikzcd}
	\end{center}
	Lo mismo vale para haces.
\end{prop}

\begin{prop}
	Sea $X$ un espacio topológico y $\mathscr{F}$ un prehaz sobre $X$ con valores en una categoría bicompleta $\catC$.
	Para cada abierto $U \subseteq X$ y cada punto $x \in X$ se determinan dos únicas flechas
	\begin{center}
		\begin{tikzcd}
			\displaystyle
			\mathscr{F}(U )
			\rar["\alpha"] &
			\prod_{y \in U} \mathscr{F}_y
				       &
			\coprod_{x\in V} \mathscr{F}(V)
			\rar["\beta"] &
			\mathscr{F}_x,
		\end{tikzcd}
	\end{center}
	tales que $\alpha\circ\pi_x = \rho^U_x$ para cada punto $x \in U$ y tales que $\iota_V \circ \beta = \rho^V_y$ para cada $V$ entorno de $y$.
	Más aún, si $\mathscr{F}$ es un haz, entonces $\alpha$ es un monomorfismo y $\beta$ es un epimorfismo.
\end{prop}
\begin{proof}
	La existencia (y unicidad) de las flechas se desprende de la definición de (co)producto.
	El que $\alpha$ sea un monomorfismo lo verificaremos en el caso concreto: Aquí $\alpha(s) := (s|_x)_{x\in U}$,
	luego dos secciones $s, t \in \Gamma(U, \mathscr{F})$ satisfacen que $\alpha(s) = \alpha(t)$ syss $s|_y = t|_y$ para cada $y \in U$.
	Esto significa que existe un entorno $V_y \subseteq U$ y una sección $u_y \in \Gamma(V_y, \mathscr{F})$ tal que $s|_{V_y} = u_y = t|_{V_y}$.
	Así, vemos que $s, t$ coinciden en abiertos $V_x$ que cubren todo $U$, luego por el axioma de pegado son iguales.

	El que $\beta$ sea un epimorfismo también se puede verificar en el caso concreto, en donde corresponde a decir que todo gérmen local en $y$ viene
	de alguna sección sobre un entorno de $y$.
\end{proof}
\begin{prop}
	Dado un espacio topológico $X$ y un morfismo $\varphi \colon \mathscr{F} \to \mathscr{G}$ de haces de conjuntos sobre $X$. Entonces:
	\begin{enumerate}
		\item $\varphi$ es un monomorfismo syss $\varphi_x$ lo es para todo $x \in X$.
		\item $\varphi$ es un isomorfismo syss $\varphi_x$ lo es para todo $x \in X$.
	\end{enumerate}
\end{prop}
\begin{proof}
	\begin{enumerate}
		\item $\impliedby$. Basta construir el siguiente diagrama conmutativo:
			\begin{center}
				\begin{tikzcd}[sep=large]
					\displaystyle
					\mathscr{F}(U) \rar["\varphi_U"] \dar[hook] & \mathscr{G}(U) \dar[hook] \\
					\prod_{x\in U} \mathscr{F}_x \rar["\prod_{x\in U} \varphi_x"] & \prod_{x\in U} \mathscr{G}_x
				\end{tikzcd}
			\end{center}
			Y notar que como la composición $\varphi_U \circ \alpha$ es un monomorfismo, entonces $\varphi_U$ también.

			$\implies.$ Sean $s, t \in \mathscr{F}(U)$ un par de secciones tales que
			\[
				\varphi(s)|_x = \varphi_x(s|_x) = \varphi_x(t|_x) = \varphi(t)|_x,
			\]
			para algún $x \in U$.
			Entonces existe un subentorno $x \in V \subseteq U$ tal que $\varphi_V(s|_V) = \varphi_U(s)|_V = \varphi_U(t)|_V = \varphi_V(t|_V)$.
			Luego, puesto que $\varphi_V$ es un monomorfismo, se cumple que $s|_V = t|_V$ y así vemos que $s|_x = t|_x$,
			lo que completa la inyectividad de $\alpha$.

		\item $\implies.$ Basta recordar que localizar $(-)_x$ es funtorial.

			$\impliedby.$ Por el inciso anterior ya sabemos que $\varphi$ es un monomorfismo, basta ver que es un epimorfismo.

			Sea $t \in \Gamma(U, \mathscr{G})$ una sección.
			Luego para todo $y \in U$ se tiene que $t|_y = \varphi_y(s_y)$ para algún germén local.
			Así que existe un subenterno $y \in V_y \subseteq U$ tal que $u|_{V_y} = t|_{V_y}$, donde $u \in \Gamma(V_y, \mathscr{G})$
			es tal que $u|_y = \varphi_y(s^y)$.
			Ahora bien, $s^y$ viene de alguna sección ${\tilde{s̃}}^y \in \Gamma({\tilde{Ṽ}}_y, \mathscr{F})$
			y luego $\varphi_{\tilde{Ṽ}_y} (\tilde{s̃}^y)|_y = u|_y$, por lo que coinciden en un subentorno $y \in W_y \subseteq U$
			y finalmente definimos $v^y := \tilde{s}^y|_{W_y}$.
			Así, tenemos una colección de $v^y$'s tales que $\varphi_{W_y}(v^y) = t|_{W_y}$ y como $\varphi$ es inyectivo,
			entonces se verifica que los $v^y$'s son compatibles y se pegan en $v$.
			\qedhere
	\end{enumerate}
\end{proof}

\begin{exn}
	Sea $X$ un espacio topológico, $\catC$ una categoría completa y sea $(A_x)_{x\in X}$ una familia de objetos en $\catC$. Definiendo
	$$ \Gamma(U, \Pi) := \prod_{y\in U} A_y. $$
	con las restricciones naturales (dadas, en el caso concreto, por eliminar coordenadas),
	entonces se comprueba que $\Pi$ es un prehaz.

	Supongamos que $\catC$ es concreta.
	Entonces una sección $(s_y)_y \in \Gamma(U, \Pi)$ es una función de elección $s_y \in A_y$.
	Luego dos secciones $(s_u)_{u\in U}|_{U \cap V} = (t_v)_{v\in V}|_{U \cap V}$ coinciden syss $s_x = t_x$ para cada $x \in U \cap V$,
	por lo que podemos pegarlas de manera única en una sección $(s_y )_{y\in U\cup V}$.
	Así, $\Pi$ es de hecho un haz.

	Ahora bien, es fácil notar que las fibras
	$$ \Pi_x = \prod_{x \speto y} A_y, $$
	donde $y$ recorre los puntos que están en todos los entornos de $x$.
	Así existe un monomorfismo $A_x \to \Pi_x$ y también un epimorfismo $\Pi_x \to A_x$ cuya composición es $1_{A_x} \colon A_x \to A_x$,
	pero no necesariamente son iguales.
	Nótese que la igualdad se alcanza syss $x$ es un punto cerrado.
\end{exn}
Al lector, argumente que el prehaz anterior es un haz cuando $\catC$ no es necesariamente concreta.

\begin{prop}
	Dado un prehaz $\mathscr{F} \in \mathsf{PSh}(X; \catC)$, entonces podemos considerar
	el funtor $\Hom_{\mathsf{PSh}}(\mathscr{F}, -) \colon \mathsf{Sh}(X; \catC) \to \mathsf{Set}$:
	\begin{multicols}{2}
		\begin{align*}
			h^\varphi \colon \Hom_{\mathsf{PSh}}(\mathscr{F, G}) &\longrightarrow \Hom_{\mathsf{PSh}}(\mathscr{F, H}) \\
			\psi &\longmapsto \varphi\circ \psi
		\end{align*}
		\begin{center}
			\begin{tikzcd}
				\mathscr{G} \dar["\varphi"', ""{name=s}] & \Hom_{\mathsf{PSh}}(\mathscr{F, G}) \dar["h^\varphi", ""'{name=f}] \\
				\mathscr{H}			      & \Hom_{\mathsf{PSh}}(\mathscr{F, H})
			\end{tikzcd}
		\end{center}
	\end{multicols}
	Éste funtor es representable por un único objeto y flecha $\iota \colon \mathscr{F \to F}^+$,
	el cual es un haz y se dice la \strong{hazificación}\index{hazificación} de $\mathscr{F}$.

	Más aún, para cada punto $x \in X$ se satisface que $\mathscr{F}_x = \mathscr{F}_x^+$, es decir, las fibras de un prehaz y su hazificación coinciden.
\end{prop}
\begin{proof}
	Definamos $\mathscr{F}^+$ como procede: para cada abierto $U \subseteq X$ una sección $(s_y)_y \in \Gamma(U, \mathscr{F}^+)$,
	corresponde a una tupla $(s_y)_y \in \prod_{y\in U} \mathscr{F}_y$ tal que cada punto $x \in U$ posee un subentorno $x \in V \subseteq U$ y una sección
	$\sigma \in \mathscr{F}(V)$ tal que $s_v = \gamma|_v$ para todo $v \in V$.
	
	Claramente si $U \subseteq V$ son abiertos, entonces hay una proyección natural
	\begin{center}
		\begin{tikzcd}[sep=large]
			\prod_{u\in U} \mathscr{F}_u \rar & \prod_{v\in V} \mathscr{F}_v,
		\end{tikzcd}
	\end{center}
	que luego determina una restricción entre secciones. Así $\mathscr{F}^+$ es un prehaz.

	Además, si $U \subseteq X$ es abierto, entonces la flecha $\mathscr{F}(U) \to \prod_{u\in U} \mathscr{F}_u$ producto de localizaciones
	tiene valores en $\mathscr{F}^+ (U)$, de modo que determina un morfismo de prehaces $\mathscr{F \to F}^+$, y claramente $\mathscr{F}^+$ es
	un subprehaz de $\Pi$ del ejemplo anterior, luego hereda unicidad de pegado.
	Finalmente es fácil notar que en $\mathscr{F}^+$ también se pueden pegar secciones, por lo que, es un haz.

	Por el mismo ejemplo, hay una inyección $\mathscr{F}_x \to \mathscr{F}_x^+$,
	por lo que basta probar que es suprayectiva.
	Sea $s̄ \in \mathscr{F}_x^+$, luego existe un entorno $x \in U$ tal que $s̄ = (s_u)_{u\in U}|_x$.
	Por definición existe un subentorno $x \in V \subseteq U$ y una sección $\sigma \in \mathscr{F}(V)$ tal que $s_v = \sigma|_v$ para todo $v \in V$.
	Trivialmente, la imagen de $\sigma$ es $s̄$. Así $\mathscr{F}_x \to \mathscr{F}_x^+$ es un isomorfismo.
	Finalmente, es fácil notar que la hazificación determina un funtor $(-)^+ \colon \mathsf{PSh} \to \mathsf{Sh}$.
	Así que si $\psi \colon \mathscr{F} \to \mathscr{G}$ es un morfismo de prehaces, determina un morfismo $\psi^+ \colon \mathscr{F}^+ \to \mathscr{G}^+$
	de haces.
	Además tenemos el morfismo de prehaces $\iota_{\mathscr{G}} \colon \mathscr{G} \to \mathscr{G}^+$ que es un isomorfismo en las fibras,
	luego es un isomorfismo de haces por la proposición 3.42, de modo que $\psi$ se factoriza por $\mathscr{F} \to \mathscr{F}^+ \to \mathscr{G}$.
\end{proof}
\begin{ex}
	La hazificación del prehaz constante $A^-_X$ es el haz constante $A^+_X$, de ahí la simbología.
\end{ex}

\begin{cor}
	La hazificación determina un funtor $(-)^+ \colon \mathsf{PSh}(X; \catC) \to \mathsf{Sh}(X; \catC)$ que es la adjunta izquierda
	del funtor semiolvidadizo $U \colon \mathsf{Sh}(X; \catC) \to \mathsf{PSh}(X; \catC)$. En símbolos:
	$$ (-)+ \adjoint U, \qquad \Hom_{\mathsf{Sh}}(\mathscr{F}^+ , \mathscr{G}) \approx \Hom_{\mathsf{PSh}}(\mathscr{F}, U\mathscr{G}). $$
\end{cor}
Hay dos consecuencias que entender del corolario anterior.
La primera es que como $U$ tiene adjunta izquierda, entonces preserva límites inversos, y los límites en la categoría de prehaces se calculan
puntualmente pues es esencialmente una categoría de funtores.
La segunda es que, en general, \textit{no} preserva límites directos, así que no es tan fácil calcularlos entre haces y, la solución tecnica
es calcular puntualmente un límite directo y hazificar.
% Uno de los usos para la hazificación es el siguiente:

% §3.2.1 Límites de (pre)haces. No olvidemos que fundamentalmente la
% categoría de prehaces es una categoría de funtores contravariantes, así que
% veamos resultados relativos a ello:
% Proposición 3.45: Sea X un espacio topológico. Un diagrama F− \colon D \to 
% PSh(X; C ) de prehaces posee un límite inverso (resp. directo)
\subsection{Funtores con haces}
\begin{prop}
	Sea $\catC$ una categoría cocompleta y sea $f \colon X \to Y$ una función continua entre espacios topológicos.
	\begin{enumerate}
		\item Dado un prehaz $\mathscr{F} \in \mathsf{PSh}(X; \catC)$. Entonces definiendo para todo $V \subseteq Y$ abierto:
			$$ \Gamma(V, f_*\mathscr{F}) := \Gamma(f^{-1} [V ], \mathscr{F} ), $$
			se cumple que $f_*\mathscr{F} \in \mathsf{PSh}(Y; \catC)$, llamado el prehaz preimagen.
			Esto determina un funtor $f_*(-) \colon \mathsf{PSh}(X; \catC) \to \mathsf{PSh}(Y ; \catC)$.

		\item  Dado un prehaz $\mathscr{G} \in \mathsf{PSh}(Y; \catC)$.
			Entonces definiendo para todo $U \subseteq X$ abierto:
			$$ \Gamma(U, f_p \mathscr{G} ) := \limdir_{V \supseteq f[U]} \Gamma(V, \mathscr{G} ), $$
			se cumple que $f_p \mathscr{G} \in \mathsf{PSh}(X; \catC)$, llamado el prehaz imagen directa.
			Esto determina un funtor $f_p(-) \colon \mathsf{PSh}(Y; \catC) \to \mathsf{PSh}(X; \catC)$.
	\end{enumerate}
	Más aún, existe un isomorfismo canónico en las fibras: para todo $y \in Y$ se cumple $(f_p \mathscr{G})_y = \mathscr{G}_{f(y)}$.
\end{prop}
\begin{proof}
	Las afirmaciones son todas triviales, exceptuando la de las fibras. Sea $x \in X$, entonces
	\begin{equation}
		(f_p \mathscr{G})_x = \limdir_{x\in U} f_p \mathscr{G}(U) = \limdir_{x\in U} \limdir_{V \supseteq f[U]} \mathscr{G}(V ) = \mathscr{G}_{f(x)}.
		\tqedhere
	\end{equation}
\end{proof}

\begin{prop}
	Sea $\catC$ una categoría cocompleta y sea $f \colon X \to Y$ una función continua entre espacios topológicos.
	Dados $\mathscr{F, G}$ prehaces sobre $X, Y$ resp. con valores en $\catC$, entonces el funtor $f_p(-)$ es la adjunta izquierda de $f_*(-)$.
	En símbolos:
	$$ f_p(-) \adjoint f_*(-), \qquad \Hom_{\mathsf{PSh}(X)} (f_p \mathscr{G}, \mathscr{F}) \approx \Hom{\mathsf{PSh}(Y)} (\mathscr{G}, f_* \mathscr{F}). $$
\end{prop}
\begin{proof}
	Para todo $U \subseteq Y$ abierto se cumple que $f\big[ f^{-1} [U ] \big] \subseteq U$, de modo que existe una flecha canónica
	$\mathscr{G}(U) \to f_p \mathscr{G}(f^{-1} [U ])$.
	Esto determina un morfismo de prehaces $\varepsilon_{\mathscr{G}} \colon \mathscr{G} \to f_* f_p \mathscr{G}$.

	Por otro lado, para todo $U \subseteq X$ abierto se cumple que $f^{-1}\big[ f [U ] \big] \supseteq U$, entonces para todo $V \supseteq f [U ]$
	abierto de $Y$ se cumple que $f^{-1} [V ] \supseteq U$ de modo que tenemos la flecha de la restricción $\mathscr{F}(f^{-1}[V ]) \to F (U )$,
	y luego existe una única flecha canónica $f_p f_* \mathscr{F} (U ) \to \mathscr{F} (U )$ que determina un morfismo
	de prehaces $\eta_{\mathscr{F}} \colon f_p f_* \mathscr{F} \to \mathscr{F}$.

	Finalmente, dado un par de morfismos de prehaces $\alpha \colon \mathscr{G} \to f_* \mathscr{F}$ y $\beta \colon f_*\mathscr{G} \to \mathscr{F}$,
	entonces se inducen los siguientes morfismos:
	\begin{center}
		\includegraphics{geo-alg/adjointness.pdf}
	\end{center}
	Queda al lector verificar que éstas construcciones son una la inversa de la otra.
\end{proof}

\begin{prop}
	Sean $f \colon X \to Y$ una función continua entre espacios topológicos, $\catC$ una categoría cocompleta y $\mathscr{F} \in \mathsf{Sh}(X; \catC)$ un haz.
	Entonces $f_* \mathscr{F} \in \mathsf{Sh}(Y; \catC)$ también es un haz.
\end{prop}
La misma condición no necesariamente se da con $f_p$.
\begin{mydef}
	Sea $f \colon X \to Y$ una función continua entre espacios topológicos y $\catC$ una categoría cocompleta.
	Dado un haz $\mathscr{G} \in \mathsf{Sh}(Y; \catC)$, se define el \strong{haz imagen directa} como la hacificación
	$$ f^{-1} \mathscr{G} := (f_p\mathscr{G})^+ \in \mathsf{Sh}(X; \catC). $$
\end{mydef}
\begin{cor}
	Sea $f \colon X \to Y$ una función continua entre espacios topológicos y $\catC$ una categoría cocompleta.
	El funtor $f^{-1}(-) \colon \mathsf{Sh}(Y ; \catC) \to \mathsf{Sh}(X; \catC)$ es la adjunta izquierda del
	funtor $f_*(-) \colon \mathsf{Sh}(X; \catC) \to \mathsf{Sh}(Y; \catC)$.
	En símbolos:
	$$ f^{-1}(-) \adjoint f_*(-), \qquad \Hom_{\mathsf{Sh}(X)} (f^{-1} \mathscr{G}, \mathscr{F}) \approx \Hom{\mathsf{Sh}(Y)} (\mathscr{G}, f_* \mathscr{F}). $$
	Más aún, para cada $x \in X$ se cumple que $(f^{-1} \mathscr{G})_x = \mathscr{G}_{f(x)}$.
\end{cor}

\section{Esquemas}
\subsection{Espacios anillados y esquemas afínes}
\begin{mydefi}
	Un par $(X, \mathscr{O}_X)$ se dice un espacio anillado si $X$ es un espacio topológico y $\mathscr{O}_X$ es un haz de anillos sobre $X$,
	al que llamamos \strong{haz estructural}\index{haz!estructural} de $X$.
	Obviaremos el haz estructural de no haber ambigüedad en los signos.
	A las secciones de $\Gamma(U, \mathscr{O}_X )$ les llamamos \strong{funciones regulares}\index{función!regular} sobre $U$.

	Un \strong{morfismo de espacios anillados} es un par $(f, f^\sharp ) \colon (X, \mathscr{O}_X ) \to (Y, \mathscr{O}_Y )$
	donde $f \colon X \to Y$ es una función continua y $f^\sharp \colon \mathscr{O}_Y \to f^* \mathscr{O}_X$ es un morfismo de haces sobre $Y$.
\end{mydefi}
Por el corolario anterior, elegir $f^\sharp \colon f^{-1} \mathscr{O}_Y \to \mathscr{O}_X$ determinaría la misma información.

\begin{prop}
	Los espacios anillados (como objetos) y los morfismos entre ellos (como flechas) conforman una categoría.
\end{prop}
\begin{ex}
	Los conjuntos algebraicos (sobre $k$) presentadas en la primera parte, junto con las funciones regulares, forman espacios anillados.
\end{ex}
Fijemos un anillo $A$, podemos construir un haz sobre $\Spec A$ de la siguiente manera:
\begin{mydef}
	Dado un anillo $A$, definamos $\mathscr{O}_{\Spec A} \in \mathsf{PSh}(\Spec A; \mathsf{Set})$ donde,
	para todo abierto $U \subseteq \Spec A$ definimos $\mathscr{O}_{\Spec A}(U )$ como el conjunto de aplicaciones $s \colon U \to
	\prod_{x\in U} A_{\mathfrak{p}_x}$ tales que:
	\begin{enumerate}[{HE}1.]
		\item Para cada $x \in U$ se cumple que $s(x) \in A_{\mathfrak{p}_x}$.
		\item Las aplicaciones $s$ son localmente fracciones sobre $A$.
			Vale decir, para cada $x \in U$ existe un entorno $x \in V \subseteq U$ y unos elementos $a, f \in A$ tales que
			$$ y \in V \setminus \VV(f ) = V \cap \DD(f ) \implies s(y) = a/f. $$
	\end{enumerate}
	Si $U \subseteq V$, entonces $\rho^V_U \colon \mathscr{O}_{\Spec A}(U ) \to \mathscr{O}_{\Spec A}(V)$ es la restricción natural.
	Definiendo la suma y producto coordenada a coordenada, se prueba que $\mathscr{O}_{\Spec A}$ determina un prehaz de anillos.
\end{mydef}
\begin{prop}
	$(\Spec A, \mathscr{O}_{\Spec A})$ es un espacio anillado.
\end{prop}
Nótese que lo que hay que demostrar es que esencialmente $\mathscr{O}_{\Spec A}$ es un haz.
Ahora, tenemos derecho de llamar a $\mathscr{O}_{\Spec A}$  como el haz estructural (canónico) sobre $\Spec A$.

\begin{thm}
	Sea $A$ un anillo y sea $X := \Spec A$. Entonces:
	\begin{enumerate}
		\item Para todo $x \in \Spec A$ se cumple que $\mathscr{O}_{X, x} \cong A_{\mathfrak{p}_x}$ .
		\item Para todo $f \in A$ se cumple que $\Gamma(\DD(f ), \mathscr{O}_X) \cong A[1/f ]$.
		\item En particular, $\Gamma(X, \mathscr{O}_X) \cong A$.
	\end{enumerate}
\end{thm}
\begin{proof}
	\begin{enumerate}
		\item Definamos la siguiente aplicación:
			\begin{align*}
				\varphi := \ev_x \colon \mathscr{O}_{X, x} &\longrightarrow A_{\mathfrak{p}_x} \\
				s &\longmapsto s(x)
			\end{align*}
			la cual claramente está bien definida; es un $A$-homomorfismo de álgebras y es suprayectiva pues, dado $a/f \in A_{\mathfrak{p}_x}$,
			donde $f \notin \mathfrak{p}_x$, basta elegir la aplicación constante $s = a/f \in \Gamma(\DD(f), \mathscr{O}_X)$ y localizar en el haz.

			Veamos que $\varphi$ es un monomorfismo: para ello, sean $s \in \mathscr{O}_X(U_1), t \in \mathscr{O}_X(U_2)$ tales que $s(x) = t(x)$,
			veamos que coinciden en un abierto.
			Por la propiedad HE2, se cumple que $s = a/f$ en $U_1 \cap \DD(f)$ y $t = b/g$ en $U2 \cap \DD(g)$ para algunos $f, g \notin \mathfrak{p}_x$.
			Luego $a/f = s(x) = t(x) = b/g$ y así, coinciden en $U_1 \cap U_2 \cap \DD(f ) \cap \DD(g)$ que es un entorno de $x$.

		\item Definamos $\psi \colon A[1/f ] \to \Gamma(D(f ), \mathscr{O}_X)$ que a cada elemento $a/f^n$ le asigna
			la función $s$ tal que $s(x) = a/f^n \in A_{\mathfrak{p}_x}$.
			Claramente $s$ es un homomorfismo de $A$-álgebras.

			\underline{$\psi$ es monomorfismo:} Si $\psi(a/f^n ) = \psi(b/f^m )$, entonces $a/f^n = b/f^m$ en todo $A_{\mathfrak{p}}$,
			donde $f \notin \mathfrak{p}$.
			Así, existe $h \notin \mathfrak{p}$ tal que $h(f^m a - f^n b) = 0 \in A$.
			Sea $\mathfrak{a} := \Ann(f^m a - f^n b)$, luego $h \in \mathfrak{a \setminus p}_x$ y así $\mathfrak{a} ̸\subseteq Z_x$ para todo
			$x \in \DD(f)$; ergo, $\VV(a) \cap \DD(f) = \emptyset$.
			En conclusión $\VV(f ) \subseteq \VV(\mathfrak{a})$ y, por el teorema de ceros de Hilbert, $f \in \rad\mathfrak{a}$, es decir,
			$f^\ell \in \mathfrak{a}$ y $f^\ell (f^m a - f^n b) = 0$, por lo que, $a/f^n = b/f^m$ en $A[1/f ]$.

			\underline{$\psi$ es epimorfismo:} Sea $s \in \Gamma(\DD(f ), \mathscr{O}_X)$, por HE2, $\DD(f ) \subseteq \bigcup_{i\in I} V_i$
			donde en cada $V_i$ se cumple que $s$ es constante y vale, digamos, $a_i/g_i$ con $V_i \subseteq \DD(g_i)$.
			Como los abiertos principales forman una base, podemos suponer que $V_i = \DD(h_i )$ para algún $h_i$ y luego,
			por el teorema de ceros de Hilbert, $h_i^{m_i} \in g_iA$ para algún $m_i$.
			Luego $h_i^{m_i} = c_i g_i$ para algún $ci \in A$ y, en consecuente, $a_i /g_i = ca_i /h_i^{m_i}$; así, sustituyendo $a_i$ por $c_i a_i$
			y $h_i$ por $h_i^{m_i}$, podemos suponer que $\DD(f )$ está cubierto por abiertos $\DD(h_i)$ en donde $s$ vale $a_i/h_i$.

			Podemos extraer un subcubrimiento finito de $\DD(f )$ pues
			$$ \DD(f ) \subseteq \bigcup_{i\in I} \DD(h_i) = \DD\left( \sum_{i\in I} h_iA \right) $$
			lo que, por el teorema de ceros de Hilbert, equivale a que $f^n \in \sum_{i\in I} h_iA$, pero $f^n$ claramente está en una suma
			de algunos finitos $h_i$'s, luego reescribiendo tenemos que $\DD(f ) \subseteq \bigcup_{i=1}^n \DD(h_i)$.

			Ahora bien, dados $h_i, h_j$ con $i ̸= j$, se tiene que $\DD(h_i) \cap \DD(h_j) = \DD(h_i h_j) ̸= \emptyset$ y $s$ es constante
			y vale $a_i /h_i = a_j /h_j \in A[1/h_i h_j ]$ en $\DD(h_i h_j)$, luego
			$$ (h_i h_j )^n (a_i h_j - a_j h_i ) = 0, $$
			donde como hay finitos índices $i$'s podemos elegir un $n$ suficientemente grande.
			La fórmula se reescribe como $h_j^{n+1} (h^n_i a_i ) - h^{n+1}_i(h_j^n a_j) = 0$, así que, reemplazando $a_i$ por $a_i h^n_i$
			y $h_j$ por $h{n+1}_j$, obtenemos que $h_j a_i = h_i a_j$ para todo $i, j$.
			Ahora, como $\DD(f) \subseteq \DD\left( \sum_{i=1}^{n} h_i \right)$, tenemos que $f^m = \sum_{i=1}^{n} b_ih_i$ para algunos $b_i \in A$.
			Definiendo $a := \sum_{i=1}^{n} b_ia_i$ se tiene que
			$$ h_j a = \sum_{i=1}^{n} b_i a_i h_j = \sum_{i=1}^{n} b_i h_i a_j = f^m a_j. $$
			Luego $s$ toma, en cada $\DD(h_i)$, el valor $a/f^m \in A[1/f ]$ como se quería probar.

		\item Considere el inciso anterior con $f = 1$. \qedhere
	\end{enumerate}
\end{proof}

\begin{mydef}
	Sean $(A, \mathfrak{m}), (B, \mathfrak{n})$ dos anillos locales.
	Una aplicación $\varphi \colon A \to B$ se dice un \strong{homomorfismo de anillos locales}\index{homomorfismo!de anillos locales}
	si es un homomorfismo de anillos tal que $\varphi^{-1}[\mathfrak{n}] = \mathfrak{m}$.
\end{mydef}
\begin{prop}
	Si $\varphi \colon (A, \mathfrak{m}) \to (B, \mathfrak{n})$ es un homomorfismo de anillos locales,
	entonces determina un único monomorfismo $\bar\varphi\colon A/\mathfrak{m} \hookto B/\mathfrak{n}$ de cuerpos.
\end{prop}
\begin{proof}
	Basta aplicar el primer teorema de isomorfismos, notar que $\bar \varphi$ no es nula, y recordar que todo homomorfismo cuyo dominio es un
	cuerpo es necesariamente un monomorfismo.
\end{proof}
Veamos ahora el caso local:
\begin{mydef}
	Un \strong{espacio localmente anillado}\index{espacio!localmente anillado} $(X, \mathscr{O}_X )$ es un espacio anillado en donde
	cada fibra $\mathscr{O}_X,x$ es un anillo local con ideal maximal $\mathfrak{m}_{X,x}$ y con \strong{cuerpo de restos}
	\nomenclature{$\kk(x)$}{$= \mathscr{O}_{X, x} / \mathfrak{m}_{X, x}$, cuerpo de restos de un esquema $X$ en un punto $x$}
	$\kk(x) := \mathscr{O}_{X, x} / \mathfrak{m}_{X, x}$.

	Sean $(X, \mathscr{O}_X ), (Y, \mathscr{O}_Y )$ un par de espacios localmente anillados.
	Un par $(f, f^\sharp ) \colon (X, \mathscr{O}_X ) \to (Y, \mathscr{O}_Y )$ se dice un \strong{morfismo de espacios localmente anillados}%
	\index{morfismo!de espacios localmente anillados} si es un morfismo de espacios anillados, y en cada fibra $f_y^\sharp \colon \mathscr{O}_{Y,y} \to 
	(f_*\mathscr{O}_X)_y$ es un homomorfismo de anillos locales.
\end{mydef}
Como la información $f^\sharp \colon \mathscr{O}_Y \to f_* \mathscr{O}_X$ coincide con una flecha $f^\sharp \colon f^{-1} \mathscr{O}_Y \to \mathscr{O}_X$,
entonces podemos ver que es lo mismo exigir que para $x \in X$ se induce un homomorfismo de anillos locales
$$ f_x^\sharp \colon (f^{-1} \mathscr{O}_Y )_x = \mathscr{O}_{Y,f (x)} \longrightarrow \mathscr{O}_{X,x}. $$
Esto corresponderá mejor con la situación dada entre espectros de anillos.

Sumado a la proposición anterior, nos da que:
\begin{prop}\label{thm:mono_fibers}
	Sea $f \colon X \to Y$ un morfismo de espacios localmente anillados.
	Entonces, para cada punto $x \in X$ se determina un monomorfismo entre los cuerpos de restos $\bar f_x \colon \kk\big( f(x) \big) \hookto \kk(x)$.
\end{prop}
\begin{cor}
	Los espacios de la forma $\Spec A$ con el haz estructural son espacios localmente anillados.
\end{cor}

\begin{mydef}
	Los espacios localmente anillados de la forma $\Spec A$ con el haz estructural se dicen esquemas afínes.
\end{mydef}
\begin{mydef}
	Sea $A$ un anillo.
	El \strong{espacio afín}\index{espacio!afín} de dimensión $n$ sobre $A$, denotado $\A^n_A$, es el esquema afín $\A^n_A := \Spec(A[t_1, \dots, t_n])$.
\end{mydef}
\begin{ex}
	Si $k$ es un cuerpo, entonces dado un punto $P := (\alpha_1, \dots, \alpha_n) \in k^n$ se satisface que
	el ideal $x_P := (t_1 - \alpha_1, \dots, t_n - \alpha_n)$ es maximal, luego es un punto cerrado de $\A^n_k$.
	Más aún, si $k$ es algebraicamente cerrado, entonces por el teorema débil de ceros de Hilbert, todos los puntos cerrados de $\A^n_k$ son de la forma $x_P$.
\end{ex}

\begin{prop}
	Sean $A, B$ anillos. Todo homomorfismo de anillos $\varphi \colon A \to B$ induce un morfismo de espacios localmente anillados
	$$ (\varphi^a, \varphi^\sharp) \colon \Spec B \longrightarrow \Spec A. $$
	Más aún cada morfismo de espacios localmente anillados viene de un homomorfismo de anillos.
\end{prop}
\begin{proof}
	Por brevedad, denotaremos $f := \varphi^a$ el que se define como $f(x) := \varphi^{-1}[\mathfrak{p}_x]$.
	Para cada punto $y \in \Spec B$ se determina un homomorfismo de anillos locales $\varphi_y \colon A_{f(y)} \to B_y$.
	Así, para cada abierto $U \subseteq \Spec A$ y cada sección $s \in \Gamma(U, \mathscr{O}_{\Spec A} )$, definimos:
	\begin{align*}
		\varphi^\sharp \colon f^{-1}[U] &\longrightarrow \coprod_{y \in f^{-1}[U]} B_{\mathfrak{p}_y} \\
		y &\longmapsto \varphi_y\big( s(f(y)) \big).
	\end{align*}
	Queda al lector verificar que efectivamente $\varphi^\sharp (s)$ determina una sección de $\Gamma(f^{-1} [U ], \mathscr{O}_{\Spec B})$.

	Sea $(f, f^\sharp) \colon \Spec B \to \Spec A$ un morfismo de espacios localmente anillados, entonces consideramos las secciones
	globales $f^\sharp \colon \Gamma(\Spec A, \mathscr{O}_{\Spec A}) \to \Gamma(\Spec B, \mathscr{O}_{\Spec B})$ y recordamos que,
	por la proposición anterior,
	\[
		\Gamma(\Spec R, \mathscr{O}_{\Spec R} ) \cong R
	\]
	para todo anillo $R$, luego determina un homomorfismo de anillos $\varphi \colon A \to B$.
	Para cada $\mathfrak{p} \in \Spec B$, mirando las fibras tenemos el siguiente diagrama conmutativo:
	\begin{center}
		\begin{tikzcd}[row sep=large]
			A \dar[hook] \rar["\varphi"] & B \dar[hook] \\
			A_{f(\mathfrak{p})} \rar["f^\sharp_{\mathfrak{p}}"] & B_{\mathfrak{p}}
		\end{tikzcd}
	\end{center}
	con lo que, empleando que $f_{\mathfrak{p}}^\sharp$ es un homomorfismo de anillos locales,
	concluimos que $f(\mathfrak{p}) = \varphi^{-1}[\mathfrak{p}]$ y $f_{\mathfrak{p}}^\sharp = \varphi_{\mathfrak{p}}$.
\end{proof}
\begin{ex}
	Sea $A := \C$ y considere $\sigma \colon \C \to \C$ el isomorfismo de cuerpos dado por la conjugación compleja.
	Entonces induce un isomorfismo de esquemas $\sigma^a \colon \Spec\C \to \Spec\C$.
	Ahora bien, el único punto de $\Spec\C$ es $\xi := (0)$, luego necesariamente $\sigma^a$ (como función continua) es la función que fija a $\xi$.
	Así, $\sigma^a = \Id^a$ (como función continua), pero $\sigma^\sharp ̸\ne \Id^\sharp$.
\end{ex}
Éste ejemplo sencillo ilustra que las funciones entre esquemas no están únicamente determinadas por la función entre espacios subyacentes.

\begin{exn}
	Sea $A$ un dominio de valuación discreta, por ende, un dominio íntegro con sólo dos ideales primos $X := \Spec A = \{ (0), \mathfrak{m} \}$.
	Nótese que $A_{\mathfrak{m}} = A$ y que $A_{(0)} = \Frac A =: K$.
	La inclusión $\lambda \colon A \to K$ da lugar al morfismo de espacios localmente anillados $(f, f^\sharp ) \colon \Spec K \to X$, donde,
	como $\Spec K = \{ (0) \}$, vemos que $f(y_0) = x_0$.

	No obstante, podemos definir otro morfismo de espacios anillados\break $(g, g^\sharp ) \colon \Spec K \to X$ tal que $g(y_0) = x_m$.
	Nótese que $X$ posee tres abiertos: $X, \{ x_0 \}, \emptyset$. Así podemos verificar que $g_*\mathscr{O}_{\Spec K}$ es el haz sobre $X$:
	$$ \Gamma(U, g_*\mathscr{O}_{\Spec K}) =
	\begin{cases}
		K, & U = X \\
		0, & U \ne X
	\end{cases} $$
	y construir $g^\sharp (\{ x_0 \}) := \Id_K \colon K \to K$ y $g^\sharp (X) \colon A \to 0$ el morfismo nulo.

	Finalmente, nótese que $g^\sharp_{x_{\mathfrak{m}}} \colon A \to 0$ no es un homomorfismo de anillos locales, por lo que,
	$(g, g^\sharp )$ no es un morfismo de espacios localmente anillados.
\end{exn}

Para finalizar ésta subsección, introduciremos una noción que jugará un rol fundamental más adelante:
\begin{mydef}
	Un \strong{encaje abierto}\index{encaje abierto} (resp. \strong{cerrado}\index{encaje!cerrado}) de espacios anillados
	es un morfismo $(f, f^\sharp ) \colon (X, \mathscr{O}_X) \to (Y, \mathscr{O}_Y)$ tal que
	$f \colon X \to Y$ es un encaje topológico abierto (resp. cerrado)%
	\footnote{Es decir, si $f \colon X \to f [X]$ es un homeomorfismo, y si $f [X] \subseteq Y$ es abierto (resp. cerrado).}
	y cada $f_x^\sharp$ es un isomorfismo (resp. un epimorfismo) para todo $x \in X$.
\end{mydef}
Se denota por <<\begin{tikzcd}[cramped, sep=small] {} \rar[open] & {} \end{tikzcd}>> a los encajes abiertos,
y por <<\begin{tikzcd}[cramped, sep=small] {} \rar[closed] & {} \end{tikzcd}>> a los encajes cerrados.

Los encajes abiertos tienen una interpretación sencilla:
\begin{prop}
	Sea $(X, \mathscr{O}_X)$ un espacio anillado.
	\begin{enumerate}
		\item Para todo abierto $U \subseteq X$, se cumple que la inclusión
			\begin{center}
				\begin{tikzcd}[sep=large]
					(i, i^\sharp ) \colon (U, \mathscr{O}_X|_U ) \rar[open] & (X, \mathscr{O}_X)
				\end{tikzcd}
			\end{center}
			es un encaje abierto de espacios anillados.

		\item Más aún, todo encaje abierto 
		\begin{tikzcd}[cramped, sep=small]
			f \colon (Y, \mathscr{O}_Y ) \rar[open] & (X, \mathscr{O}_X),
		\end{tikzcd}
		con $U := f [Y ]$, se factoriza por:
		\begin{center}
			\begin{tikzcd}[row sep=large, column sep=small]
				Y \ar[rr, open, "f", near end] \drar["\bar f"'] \drar[draw=none, "\sim", sloped] & {}                  & X \\
				{}                                                                               & U \urar[open, "i"']
			\end{tikzcd}
		\end{center}
	\end{enumerate}
\end{prop}

Los encajes cerrados son más complicados, pero podemos dar un ejemplo canónico:
\begin{prop}
	Sea $A$ un anillo, $\mathfrak{a} \nsle A$ un ideal y sea $\pi \colon A \to A/\mathfrak{a}$ la proyección canónica.
	Entonces induce un encaje cerrado $\pi^a \colon \Spec(A/\mathfrak{a}) \to \Spec A$,
	de modo que $\Spec(A/\mathfrak{a}) \approx \VV(\mathfrak{a})$ (en $\mathsf{Top}$).
\end{prop}
\begin{proof}
	Las propiedades de $\pi^a$ ya las probamos, y es fácil notar que sobre un abierto principal $\DD(f)$
	se cumple que $\pi^\sharp \colon A[1/f ] \to (A/\mathfrak{a})[1/f]$ es la proyección canónica.
\end{proof}
\begin{ex}
	Sea $k$ un cuerpo y considere $A := k[\varepsilon]$, donde $\varepsilon$ es una indeterminada.
	Considere los ideales $\mathfrak{a} := (\varepsilon)$ y $\mathfrak{b} := (\varepsilon^2) = \mathfrak{a}_2$,
	así tenemos los siguientes encajes cerrados:
	\begin{center}
		\begin{tikzcd}[row sep=large, column sep=tiny]
			\Spec k \ar[rr, closed, "\alpha", near end] \drar[closed, "\gamma"'] & & \Spec( k[\varepsilon] ) \\
			{} & \displaystyle \Spec\left( \frac{k[\varepsilon]}{(\varepsilon^2)} \right) \urar[closed, "\beta"']
		\end{tikzcd}
	\end{center}
	Lo interesante es que tanto $\alpha$ como $\beta$ tienen por imagen el mismo cerrado $\VV(\mathfrak{a})$,
	pese a que, los esquemas afínes $\Spec k = \Spec(A/\mathfrak{a})$ y $\Spec(A/\mathfrak{b})$ no son isomorfos (¿por qué?).
	Otra curiosidad es que $\gamma$ es un encaje cerrado y suprayectivo que no es un isomorfismo de espacios localmente anillados.
\end{ex}
Y demostraremos una linda correspondencia, con lo siguiente:

\begin{mydef}
	Sea $(X, \mathscr{O}_X )$ un espacio anillado. Un haz $\mathscr{I}$ sobre $X$
	se dice un \strong{haz de ideales}\index{haz!de ideales} si cada $\mathscr{I}(U)$ es un ideal de $\mathscr{O}_X(U)$ y si la restricción
	$\mathscr{I}(U) \to \mathscr{I}(V)$ es un homomorfismo de $\mathscr{O}_X(U)$-módulos.
	Dado un haz de ideales $\mathscr{I}$ definimos:
	$$ \VV(\mathscr{I}) := \{ x \in X : \mathscr{I}_x ̸= \mathscr{O}_{X,x} \} \subseteq X. $$
\end{mydef}
Esta noción sigue estando dentro de nuestros limites, puesto que los ideales son grupos abelianos.

\addtocounter{thmi}{1}
\begin{slem}
	Sea $(X, \mathscr{O}_X )$ un espacio anillado e $\mathscr{I}$ un haz de ideales de $\mathscr{O}_X$.
	Entonces $Z := \VV(\mathscr{I})$ es un cerrado de $X$ y denotando $j \colon Z \to X$ la
	inclusión se verifica que $(Z, j^{-1} (\mathscr{O}_X /I ))$ es un espacio anillado y existe un
	encaje cerrado $(j, j^\sharp ) \colon Z \to X$ tal que $j^\sharp$ es el epimorfismo:
	\begin{equation}
		\begin{tikzcd}[sep=large]
			\mathscr{O}_X \rar[two heads] & \mathscr{O}_X/\mathscr{I} = j_*^{-1}(\mathscr{O}_X/\mathscr{I}).
		\end{tikzcd}
		\label{eqn:closed_inmersion}
	\end{equation}
\end{slem}
\begin{proof}
	Si $x \notin \VV(\mathscr{I})$, entonces $\mathscr{I}_x = \mathscr{O}_{X,x}$ y existe un entorno
	$x \in U$ y una sección $f \in \Gamma(U, \mathscr{I})$, de modo que su germen $f|_x = 1$, luego
	existe un subentorno $x \in V \subseteq U$ de modo que $f|_V = 1$ y así $V \subseteq X \setminus \VV(\mathscr{I})$.
	Finalmente para cada $x \in \VV(\mathscr{I})$ se cumple que las fibras
	$$ \big( j^{-1} (\mathscr{O}_X /\mathscr{I} ) \big)_x = (\mathscr{O}_X /\mathscr{I} )_x = \mathscr{O}_{X,x}/\mathscr{I}_x $$
	son anillos locales. Finalmente es claro verificar lo demás.
\end{proof}
\addtocounter{thmi}{-1}

\begin{prop}\label{thm:closed_subsch}
	Sea 
	\begin{tikzcd}[cramped, sep=small]
		(f, f^\sharp ) \colon (Y, \mathscr{O}_Y ) \rar[closed] & (X, \mathscr{O}_X )
	\end{tikzcd}
	un encaje cerrado de espacios anillados.
	Sea $Z$ el subespacio anillado con espacio topológico $\VV(\mathscr{I})$ donde $\mathscr{I} := \ker(f^\sharp ) \subseteq \mathscr{O}_X$.
	Entonces $f$ se factoriza por un isomorfismo $g \colon Y \to Z$ y la inclusión 
	\begin{tikzcd}[cramped, sep=small]
		j \colon Z \rar[closed] & X.
	\end{tikzcd}
\end{prop}
\begin{proof}
	Como $f [Y ]$ es cerrado en $X$, se puede comprobar que
	$$ (f_*\mathscr{O}_Y)_x =
	\begin{cases}
		0, & x \notin f[Y] \\
		\mathscr{O}_{Y, y}, & x \in f[Y]
	\end{cases} $$
	Por hipótesis se tiene la sucesión exacta $0 \to \mathscr{I} \to \mathscr{O}_X \to f_* \mathscr{O}_Y \to 0$ de
	haces (sobre X) y, por tanto, se verifica que $\mathscr{I}_x = \mathscr{O}_{X,x}$ syss $x \notin f [Y ]$,
	por lo que, $Z := \VV(\mathscr{I}) = f [Y ]$.
	Sea $g \colon Y \to Z$ el homeomorfismo inducido por $f$, y sea $j \colon Z \hookto X$ la inclusión canónica, claramente $f = g \circ j$
	y por la funtorialidad de $(-)_*$ tenemos que $f_* \mathscr{O}_Y = j_* g_* \mathscr{O}_Y$.
	Finalmente, empleando que $j$ es un encaje es fácil ver que se cumple
	$$ \mathscr{O}_Z = j^{-1} j_* \mathscr{O}_Z \cong j^{-1} j_* g_* \mathscr{O}_Y = g_* \mathscr{O}_Y, $$
	(¿por qué?), de modo que $g$ es un isomorfismo de espacios anillados como se quería ver.
\end{proof}
Ésta proposición aparecerá más adelante al clasificar subesquemas cerrados.

\subsection{Esquemas}
\begin{mydefi}
	Un \strong{esquema}\index{esquema} es un espacio localmente anillado $(X, \mathscr{O}_X)$ tal que todo punto $x \in X$ posee un entorno $x \in U$ tal que
	$(U, \mathscr{O}_X|_U)$ es isomorfo (como espacio localmente anillado) a un esquema afín.
	Un \strong{morfismo de esquemas}\index{morfismo!de esquemas} $(f, f^\sharp ) \colon (X, \mathscr{O}_X) \to (Y, \mathscr{O}_Y)$
	es un morfismo de espacios localmente anillados.
\end{mydefi}

\begin{prop}
	Los esquemas (como objetos) y los morfismos de esquemas (como flechas) conforman una categoría, denotada $\mathsf{Sch}$.
\end{prop}
\begin{prop}
	Sea $(X, \mathscr{O}_X)$ un esquema y $U \subseteq X$ un abierto. Entonces $(U, \mathscr{O}_X|_U )$ también es un esquema.
\end{prop}
\begin{proof}
	Claramente $(U, \mathscr{O}_X|_U )$ es un espacio localmente anillado, veamos que cada punto posee un entorno que es un esquema afín.
	Sea $x \in U$ un punto, luego posee un entorno $x \in V \subseteq X$ tal que $(V, \mathscr{O}_X|_V )$ es isomorfo a un esquema afín $\Spec A$.
	Luego $V \cap U$ es un entorno de $x$ en $V$ y admite un subentorno de la forma $\DD(f ) =: W$ para algún $f \in A$.
	El par $(W, \mathscr{O}_X|_W)$ es un esquema afín, pues $W \cong \Spec(A[1/f ])$ y $W \subseteq U$ es abierto.
\end{proof}

\begin{prop}\label{thm:compatible_morphs_glue}
	Sean $X, Y$ esquemas, sea $\{ U_i \}_{i\in I}$ un cubrimiento abierto de $X$.
	Dada una familia de morfismos de esquemas $f_i \colon U_i \to Y$ donde los $f_i$'s son compatibles
	(es decir, $f_i|_{U_i \cap U_j} = f_j|_{U_i \cap U_j}$ para cada $i, j \in I$), entonces existe un único morfismo de esquemas $f \colon X \to Y$.
\end{prop}
\begin{cor}
	Si $U \subseteq X$ es un abierto (no vacío) de un esquema, entonces la inclusión $\iota \colon U \hookto X$ es un encaje abierto de esquemas.
\end{cor}

\begin{thm}\label{thm:hom_aff_sch}
	Sean $X, Y$ esquemas donde $Y = \Spec A$. Entonces existe una biyección natural entre
	$$ \Hom_{\mathsf{Sch}} (X, \Spec A) \longrightarrow \Hom_{\mathsf{Ring}} (A, \Gamma(X, \mathscr{O}_X )). $$
\end{thm}
\begin{proof}
	Claramente, dado un morfismo de esquemas $(f, f^\sharp ) \colon X \to \Spec A$, entonces da un morfismo de haces sobre $\Spec A$ dado por
	$f^\sharp \colon \mathscr{O}_{\Spec A} \to f_* \mathscr{O}_X$, mirando secciones globales obtenemos un homomorfismo de anillos.

	Ahora bien, sea $X = \bigcup_{i\in I} U_i$ donde cada $U_i$ es un abierto afín, entonces tenemos el siguiente diagrama:
	\begin{center}
		\begin{tikzcd}
			\displaystyle
			\Hom_{\mathsf{Sch}}(X, Y) \rar["\rho"] \dar["\alpha"', hook] & \Hom_{\mathsf{Ring}}(A, \Gamma(X, \mathscr{O}_X)) \dar["\beta"] \\
			\prod_{i\in I} \Hom_{\mathsf{Sch}}(U_i, Y) \rar["\gamma"] \rar[draw=none, "\sim"'] & \prod_{i\in I} \Hom_{\mathsf{Ring}}(A, \Gamma(U_i, \mathscr{O}_X))
		\end{tikzcd}
	\end{center}
	Aquí $\gamma$ es una biyección por tratarse de morfismos entre esquemas afínes, y $\alpha$ es inyectivo por el axioma de pegado de esquemas,
	luego $\rho$ también es inyectivo.
	Sea $\varphi \colon A \to \mathscr{O}_X(X)$ un homomorfismo de anillos, entonces componiendo:
	\begin{center}
		\begin{tikzcd}[sep=large]
			A \rar["\varphi"'] \ar[rr, "\varphi_i", bend left] & \mathscr{O}_X(X) \rar["\rho^X_{U_i}"'] & \mathscr{O}_X(U_i)
		\end{tikzcd}
	\end{center}
	Obtenemos una familia de homomorfismos $\varphi_i \colon A \to \mathscr{O}_X (U_i)$ que son compatibles,
	luego inducen una familia de morfismos de esquemas $\varphi^a_i \colon U_i \to Y$ compatibles que pegamos en un morfismo $f$
	por la proposición anterior, y satisface que $\rho(f) = \varphi$.
\end{proof}
\begin{cor}
	Para todo esquema $(X, \mathscr{O}_X)$ existe un único morfismo de esquemas $(f, f^\sharp ) \colon X \to \Spec\Z$.
	En resumen, $\Spec\Z$ es el objeto final de $\mathsf{Sch}$.
\end{cor}

\begin{prop}\label{thm:amalg_sum_sch}
	Sean $(X, \mathscr{O}_X), (Y, \mathscr{O}_Y)$ un par esquemas, y $U \subseteq X, V \subseteq Y$ abiertos.
	Dado un isomorfismo de esquemas
	$$ \varphi \colon (U, \mathscr{O}_X|_U) \longrightarrow (V, \mathscr{O}_Y|_V) $$
	entonces admite un coproducto fibrado $X \amalg_\varphi Y =: Z$ que también es un esquema.
	Tenemos el siguiente diagrama conmutativo (en $\mathsf{Sch}$):
	% https://q.uiver.app/#q=WzAsNCxbMCwwLCJVXFxjb25nIFYiXSxbMCwxLCJZIl0sWzEsMCwiWCJdLFsxLDEsIlhcXGFtYWxnX1xcdmFycGhpIFkiXSxbMCwxLCJWIiwyXSxbMCwyLCJVIl0sWzEsMywiIiwyLHsic3R5bGUiOnsiYm9keSI6eyJuYW1lIjoiZGFzaGVkIn19fV0sWzIsMywiIiwwLHsic3R5bGUiOnsiYm9keSI6eyJuYW1lIjoiZGFzaGVkIn19fV0sWzMsMCwiIiwxLHsic3R5bGUiOnsibmFtZSI6ImNvcm5lci1pbnZlcnNlIn19XV0=
	\[\begin{tikzcd}
		{U\cong V} & X \\
		Y & {X\amalg_\varphi Y}
		\arrow[open, "V"', from=1-1, to=2-1]
		\arrow[open, "U", from=1-1, to=1-2]
		\arrow[dashed, from=2-1, to=2-2]
		\arrow[dashed, from=1-2, to=2-2]
		\arrow["\ulcorner"{anchor=center, pos=0.125, rotate=180}, draw=none, from=2-2, to=1-1]
	\end{tikzcd}\]
\end{prop}
\begin{proof}
	Explicitaremos la construcción.
	El espacio topológico de $Z$ es el espacio topológico dado por el cociente de la suma de espacios $X \amalg Y$
	bajo la relación $x \sim \varphi(x)$ para $x \in U$.
	Esto determina dos aplicaciones $i \colon X \to Z, j \colon Y \to Z$ con la condición de que $W \subseteq Z$ es abierto syss
	$i^{-1}[W] \subseteq X$ y $j^{-1}[W] \subseteq Y$ son abiertos.
	Para un abierto $W \subseteq Z$, denotemos $U' := i^{-1}[W], V' := j^{-1}[W]$, entonces sus secciones son:
	$$ \Gamma(W, \mathscr{O}_Z) := \{ (s_1, s_2 ) \in \mathscr{O}_X(U') \times \mathscr{O}_Y(V') : \varphi(s_1|_{U \cap U'}) = s_2|_{V \cap V'} \}. $$
	Finalmente, es fácil verificar que $(Z, \mathscr{O}_Z)$ es un espacio localmente anillado, y cada punto de $Z$ está contenido en una copia
	de $X$ o de $Y$, de modo que posee un entorno que es un esquema afín.
\end{proof}
Nótese que la prueba también funciona para diagramas posiblemente infinitos.
Si dejamos que $U = V = \emptyset$, entonces obtenemos un coproducto en el sentido usual.

\begin{ex}
	Sea $A$ un anillo arbitrario, entonces $X := \coprod_{\N} \Spec A$ es un esquema,
	cuyo espacio topológico es una suma infinita de espacios no vacíos, luego no es compacto.
	Como todo esquema afín es compacto, entonces $X$ no es afín.

	Si $A = k$ es un cuerpo, podemos describir más en detalle al esquema $X$.
	Como $\Spec k = \{ x_0 \}$, entonces fácil verificar que para un abierto $U \subseteq X$ (que solo es un conjunto de puntos),
	se cumple que $\Gamma(U, \mathscr{O}_X) = \prod_{x \in U} k$, y la restricción entre abiertos equivale a borrar coordenadas.
\end{ex}
El ejemplo~\ref{thm:plane_without_origin} da otro esquema que no es afín.
% \begin{cor}
% 	La categoría de esquemas es cocompleta.
% \end{cor}
% \begin{proof}
% 	Esto debido a que tiene un objeto inicial (el esquema vacío $\emptyset$) y posee coproductos,
% 	luego es finitamente cocompleta (cfr. [13, teo. 2.12]) y entonces posee conúcleos.
% 	Como posee objeto inicial, conúcleos y coproductos infinitos, entonces es cocompleta (cfr. [13, teo. 2.11]).
% \end{proof}

Con esto podemos convertir al espectro homogéneo $\Proj A$ en un esquema:
\begin{prop}
	Sea $A$ un anillo graduado. Existe un haz $\mathscr{O}_{\Proj A}$ sobre $\Proj A$ de modo que para todo $f \in A$ homogéneo
	se cumpla que
	$$ \Gamma(\DD_+(f), \mathscr{O}_{\Proj A}) \cong A_{(f)}. $$
\end{prop}
\begin{proof}
	Como indica el enunciado, para $f \in A$ homogéneo definamos $\Gamma(\DD_+(f ), \mathscr{O}_{\Proj A}) := A_{(f)}$.
	Si $g \in A$ es homogéneo tal que $\DD_+(f ) \supseteq \DD_+(g)$ entonces, la proposición 3.29 nos da la restricción
	$\mathscr{O}_{\Proj A}(\D_+(f )) = A_{(f)} \to A_{(g)} = \mathscr{O}_{\Proj A}(\DD_+(g))$.
	Esto determina un prehaz sobre la base $\mathcal{B}$ de abiertos principales, lo que, por la misma proposición, determina un haz;
	y se extiende de manera única a un haz sobre $\Proj A$.
\end{proof}

\begin{mydef}
	Sea $A$ un anillo.
	El espacio proyectivo de dimensión $n$ sobre $A$, denotado $\PP^n_A$, es el esquema $\Proj(A[t_0, \dots, t_n])$.
\end{mydef}
\begin{exn}
	Como $\{ \DD_+(t_i) \}^n_{i=0}$ es un cubrimiento finito por abiertos de $\Proj(A[t_0, \dots, t_n]) = \PP^n_A$ , y como
	$A[t_0, \dots, t_n]_{(t_i )} \cong A[t_0/t_i, \dots , t_n/t_i]$, por ello es común ver la descripción de que $\PP^n_A$ es el esquema
	que resulta de pegar $\Spec(A[t_0/t_i, \dots, t_n/t_i])$.
\end{exn}

Ahora, la función continua dada por la proposición 3.30 se traduce en los siguientes enunciados:
\begin{prop}
	Sean $A, B$ un par de anillos graduados, y sea $\varphi \colon A \to B$ un homomorfismo de anillos graduados.
	Sea $M := (A_+)^e = \varphi[A_+]B$, entonces induce un morfismo de esquemas $\varphi^a \colon \DD_+(M ) \to \Proj A$
	tal que para todo $g \in A$ homogéneo se cumple que $(\varphi^a )^{-1}[\DD_+(g)] = \DD_+(\varphi(g))$ y
	que $\varphi^a|_{\DD_+(g)}$ coincide con el morfismo inducido por $A_{(g)} \to B_{(\varphi(g))}$.
\end{prop}
\begin{prop}
	Sea $A$ un anillo y $\mathfrak{a} \nsle A[t_0, \dots, t_n]$ un ideal homogéneo, 
	de modo que $B := A[t_0, \dots, t_n]/\mathfrak{a}$ es una $A$-álgebra graduada.
	Entonces $\Proj B$ es isomorfo a un subesquema cerrado de $\PP^n_A$ con espacio topológico $\VV_+(\mathfrak{a})$.
\end{prop}

\begin{mydef}
	Sea $A$ un anillo. Un \strong{esquema proyectivo}\index{esquema!proyectivo} sobre $A$ es un $A$-esquema isomorfo a un subesquema cerrado de $\PP^n_A$.
\end{mydef}
Acabamos de ver que toda $A$-álgebra graduada de tipo finito $B$ es tal que su espectro homogéneo es un $A$-esquema proyectivo.

\begin{mydef}
	Sea $S$ un esquema fijo. Decimos que $X$ es un esquema sobre $S$ o \strong{$S$-esquema}\index{Sesquema@$S$-esquema} si
	es morfismo de esquemas $\pi \colon X \to S$, a éste morfismo le llamaremos morfismo estructural.
	Si $A$ es un anillo, decimos que $X$ es un esquema sobre $A$ si es un esquema sobre $\Spec A$.
	Un morfismo entre un par $X, Y$ de esquemas sobre $S$ es un diagrama conmutativo (en $\mathsf{Sch}$):
	\begin{center}
		\begin{tikzcd}[row sep=large]
			X \dar["f"'] \rar & S \dar[equals] \\
			Y            \rar & S
		\end{tikzcd}
	\end{center}
	Los esquemas sobre $S$ forman una categoría (de corte), denotada $\mathsf{Sch}/S$.
\end{mydef}

\begin{cor}
	Sea $A$ un anillo.
	Los $A$-esquemas son esquemas dotados de un haz de $A$-álgebras.
\end{cor}
Ésta definición debería hacer eco de la definición de una $A$-álgebra. Trivialmente todo esquema es un esquema sobre $\Z$.

\begin{thm}\label{thm:sch_extendend_vars}
	Sea $k$ un cuerpo algebraicamente cerrado.
	Existe un funtor canónico plenamente fiel $t \colon \mathsf{Var}_k \to \mathsf{Sch}/k$.
	Más aún, una variedad $V$ es homeomorfa a los puntos cerrados de $F(V)$ y su anillo de funciones regulares
	es homeomorfo al dado por la restricción del haz estructural al subespacio de los puntos cerrados.
\end{thm}
\begin{proof}
	Sea $X$ un espacio topológico cualquiera, entonces $t(X)$ es el conjunto de los cerrados no vacíos irreducibles de $X$.
	En ésta demostración emplearemos, sin citar, las propiedades de los conjuntos irreducibles vistas en \S\ref{sec:irred_spaces}.
	Se puede verificar que si $(F_i)_{i\in I}$ son subconjuntos cerrados de $X$, entonces:
	\begin{enumerate}
		\item Si $F_1 \subseteq F_2$, entonces $t(F_1) \subseteq t(F_2)$. En particular, $t(F) \subseteq t(X)$.
		\item $t(F_1 \cup F_2 ) = t(F_1) \cup t(F_2)$.
		\item $t\left( \bigcup_{i\in I} F_i \right) = \bigcap_{i\in I} t(F_i)$.
	\end{enumerate}
	En consecuencia, los conjuntos de la forma $t(F)$ forman los cerrados de una topología sobre $t(X)$.

	Dada una función continua $f \colon X \to Y$, definimos
	\begin{align*}
		t(f) \colon t(X) &\longrightarrow t(Y) \\
		t(F) &\longmapsto \left\{ \overline{f[G]} : G \in t(F) \right\}.
	\end{align*}
	Para un espacio topológico $X$ podemos definir $\alpha \colon X \to t(X)$ dado por $\alpha(P) := \overline{\{ P \}}$.
	Y $\alpha^{-1}[-]$ determina una biyección entre abiertos de $t(X)$ y abiertos de $X$, pues:
	$$ t(X) \setminus t(F ) \longmapsto \alpha^{-1}\big[ t(X) \setminus t(F ) \big] = X \setminus F. $$
	En consecuente, $\alpha$ es continua.
	Sea $V$ una variedad sobre $k$ y sea $\mathscr{O}_V$ su haz de funciones regulares, entonces construimos el $k$-espacio localmente anillado
	$\big( t(V), \alpha_*\mathscr{O}_V \big)$.

	Sea $V$ una variedad afín sobre $k$, y sea $k[V ] := A$ su anillo de coordenadas afínes, entonces definimos
	\begin{align*}
		\beta \colon V &\longrightarrow \Spec(k[V]) \\
		P &\longmapsto \mathfrak{m}_{V, P},
	\end{align*}
	ésta aplicación determina una biyección entre $V$ y $\mSpec( k[V] )$, y más aún, determina un homeomorfismo entre ambos, luego es continua.
	Denotemos $X := \Spec( k[V] )$.
	Sea $\mathfrak{a} \nsle k[V ]$ un ideal de funciones regulares, luego $\beta^{-1}[\DD_X(\mathfrak{a})] = \DD_V(\mathfrak{a})$, es decir,
	es el conjunto de puntos que no se anulan en elementos de $\mathfrak{a}$.
	Luego podemos construir el morfismo $\beta^\sharp \colon \mathscr{O}_X \to \beta_*\mathcal{O}_V$ bajo
	la regla de que para $\mathfrak{a} \nsle k[V ]$:
	$$ \beta^\sharp \colon \Gamma(X \setminus \VV(\mathfrak{a}), \mathscr{O}_X ) \longrightarrow \Gamma(V \setminus \VV(\mathfrak{a}), \mathcal{O}_V ), $$
	dado por la evaluación. Finalmente, es fácil comprobar que $(\beta, \beta^\sharp)$ determina un isomorfismo de espacios localmente anillados,
	así que $\big( t(V), \alpha_*\mathcal{O}_V \big)$ es un esquema afín.

	Como toda variedad admite una base por abiertos afínes, entonces $\big(t(V ), \alpha_*\mathcal{O}_V \big)$ siempre resulta ser un esquema.
	Dadas dos variedades $V, W$ sobre $k$ y un morfismo de variedades $f \colon V \to W$ entre ellos, entonces podemos construir un
	morfismo de esquemas $(t(f ), t(f )^\sharp ) \colon t(V ) \to t(W )$ entre ellos con construcciones similares a las anteriores
	(donde $t(f)^\sharp$ mandará secciones sobre $W$ a secciones sobre $V$, mediante precomposición por $f$).
	Se puede verificar que ésta construcción da lugar a una biyección:
	$$ \Hom_{\mathsf{Var}_k}(V, W ) \longrightarrow \Hom_{\mathsf{Sch}/k}(t(V), t(W )) $$
	que es de hecho natural, luego $t(-)$ es plenamente fiel (cfr. \cite{CatTh}, prop. 1.11).
	
	Finalmente, $t(V)$ es un esquema sobre $k$, donde el morfismo de esquemas viene inducido por el homomorfismo de anillos
	$k \to \Gamma\big( t(V), \alpha_* \mathcal{O}_V \big) = \mathcal{O}_V(V)$
	que a cada $\lambda \in k$ lo manda a la función regular constante sobre $V$.
\end{proof}

Éste teorema nos dice que la teoría de esquemas extiende a la teoría de variedades algebraicas.
Como ejercicio para el lector, ¿dónde empleamos que $k$ sea algebraicamente cerrado? ¿Qué sucede con el funtor si $k$ no fuera algebraicamente cerrado?
\begin{mydef}
	Sea $k$ un cuerpo. Un $k$-esquema $X$ se dice un:
	\begin{description}
		\item[Esquema algebraico afín]\index{esquema!algebraico!afín} Si es el espectro de una $k$-álgebra de tipo finito.
		\item[Esquema algebraico]\index{esquema!algebraico} Si cada punto posee un entorno que es un conjunto algebraico afín.
		\item[Esquema algebraico proyectivo]\index{esquema!algebraico!proyectivo} Si es un $k$-esquema proyectivo.
	\end{description}
\end{mydef}

\subsection{Propiedades de esquemas}
Comencemos por dar un par de definiciones:
\begin{mydef}
	Un espacio topológico $X$ se dice \strong{cuasiseparado}\index{espacio!cuasiseparado} si la intersección de todo par de abiertos compactos es compacto.
	\par
	Un esquema $X$ se dice \strong{compacto}\index{compacto} (resp. \strong{cuasiseparado}, \strong{conexo}, \strong{irreducible})
	si su espacio topológico es compacto (resp. cuasiseparado, conexo, irreducible).
\end{mydef}
Por ejemplificar, los espacios topológicos noetherianos son compactos y cuasiseparados.

\begin{prop}\label{thm:qcqs}
	Sea $X$ un esquema y $f \in \Gamma(X, \mathscr{O}_X )$ una sección global. Denótese
	$$ X_f := \{ x \in X : f|_x \notin \mathfrak{m}_{X,x} \subseteq \mathscr{O}_{X,x} \} = \{ x \in X : \mathscr{O}_{X,x} = f|_x \mathscr{O}_{X,x} \}. $$
	Entonces:
	\begin{enumerate}
		\item Si $\Spec B \cong U \subseteq X$ es un subesquema abierto afín y $\overline{f} := f|_U \in B \cong \Gamma(U, \mathscr{O}_X)$,
			entonces $U \cap X_f = \DD_B(\overline{f})$. En consecuencia, $X_f$ es abierto en $X$.
		\item Si $X$ es compacto, entonces para todo $a \in A := \Gamma(X, \mathscr{O}_X )$ se cumple que $a|_{X_f} = 0$ syss
			$f^n a = 0$ para algún $n \in \N$.
			% Suponga que $X$ posee un cubrimiento finito por abiertos afínes $\{ U_i \}^n_{i=1}$ tales que cada $U_i \cap U_j$ es compacto
			% (e.g., si X es un espacio topológico noetheriano).
		\item Si $X$ es cuasiseparado,
			entonces para todo $b \in \Gamma(X_f, \mathscr{O}_X)$ existe $a \in A$ tal que $a|_{X_f} = f^n b$ para algún $n \in \N$.
			Más aún, $\Gamma(X_f, \mathscr{O}_X) \cong A[1/f ]$.
	\end{enumerate}
\end{prop}
\begin{mydef}
	Un esquema $(X, \mathscr{O}_X)$ se dice \strong{reducido}\index{esquema!reducido} si para cada abierto $U \subseteq X$, el anillo $\mathscr{O}_X(U)$
	es reducido (i.e., no posee nilpotentes).
	Un esquema se dice \strong{íntegro}\index{esquema!integro@íntegro} si para cada abierto no vacío $U \subseteq X$, el anillo $\mathscr{O}_X(U )$ es
	un dominio íntegro (no nulo).
\end{mydef}

\begin{prop}\label{thm:generic_pts}
	Sea $(X, \mathscr{O}_X)$ un esquema.
	\begin{enumerate}
		\item Un punto $\xi$ de $X$ es genérico syss $\overline{\{ \xi \}}$ es una componente irreducible de $X$.
			En consecuencia, hay una biyección entre puntos genéricos y componentes irreducibles de $X$.
		\item Para todo punto $x \in X$, las componentes irreducibles de $\Spec(\mathscr{O}_{X, x})$ están en biyección con
			las componentes irreducibles de $X$ que contienen a $x$.
	\end{enumerate}
\end{prop}
\begin{proof}
	\begin{enumerate}
		\item Sea $F$ una componente irreducible de $X$ y sea $U \subseteq F$ un abierto afín no vacío.
			Nótese que $U$ es denso por el teorema \ref{thm:open_in_irred_sp} y es irreducible,
			luego posee un único punto genérico $\xi \in U$ por la proposición \ref{thm:generic_in_spec} tal que $\overline{\{ \xi \}} = U = F $.
			Si $F$ tuviese otro punto denso $\eta$, entonces éste cortaría a todo abierto de $F$, en particular, $\eta \in U$ sería denso,
			luego $\eta = \xi$ y se comprueba que $\xi$ es punto genérico de $F$ y de $X$.

		\item Sea $x$ arbitrario y sea $U$ un entorno afín de $x$, de modo que $\mathscr{O}_{X, x} = (\mathscr{O}_X|_U)_x$,
			lo que nos reduce al caso afín.
			Luego podemos considerar a $x$ como un primo y luego los puntos genéricos son los primos minimales que están contenidos
			en $\mathfrak{p}_x$ por la proposición \ref{thm:generic_in_spec}. \qedhere
	\end{enumerate}
\end{proof}
\begin{cor}
	Un esquema irreducible posee un único punto genérico que, además, resulta ser denso.
\end{cor}

\begin{prop}\label{thm:reduced_subsch}
	Sea $(X, \mathscr{O}_X)$ un esquema, entonces:
	\begin{enumerate}
		\item $X$ es reducido syss cada anillo local $\mathscr{O}_{X, x}$ es reducido.
		\item Sea $\mathscr{O}_{ X_{\rm red} }$ la hazificación del prehaz $U \mapsto \Gamma(U, \mathscr{O}_X)_{\rm red}$.
			Entonces $(X, \mathscr{O}_{ X_{\rm red} })$ es un esquema reducido, que denotaremos $X_{\rm red}$.
			Más aún, existe un morfismo de esquemas canónico $(f, f^\sharp ) \colon X_{\rm red} \to X$ tal que $f$ es un homeomorfismo.
		\item Sea $g \colon X \to Y$ un morfismo de esquemas.
			Entonces existe un único morfismo $ḡ_{\rm red} \colon X_{\rm red} \to Y_{\rm red}$ tal que el siguiente diagrama conmuta:
			\begin{center}
				\begin{tikzcd}[row sep=large]
					X_{\rm red} \rar["\exists!g_{\rm red}", dashed] \dar & Y_{\rm red} \dar \\
					X           \rar["g"]                                & Y
				\end{tikzcd}
			\end{center}

			En consecuencia, $(-)_{\rm red} \colon \mathsf{Sch} \to \mathsf{Sch}$ es un funtor covariante.
			Denotando por $\mathscr{R}$ la subcategoría plena de esquemas reducidos y denotando $\iota \colon \mathscr{R} \to \mathsf{Sch}$
			el funtor inclusión, tenemos que $(-)_{\rm red} \adjoint \iota$.

		\item Sea $Z$ un cerrado de $X$.
			Existe un único haz de anillos sobre $Z$ tal que $Z$ es un subesquema cerrado reducido de $X$.
	\end{enumerate}
\end{prop}
\begin{proof}
	\begin{enumerate}
		\item $\implies.$ Sea $x \in X$ un punto y $s \in \mathscr{O}_{X, x}$ un gérmen local no nulo.
			Luego, $s^n = 0$ syss existe una sección $t \in \Gamma(U, \mathscr{O}_X)$ sobre un entorno $U$ de $x$, tal que $t|_x = s$ y $t^n = 0$,
			pero dicha sección no existe por que $X$ es reducido.

			$\impliedby.$ Sea $s \in \Gamma(U, \mathscr{O}_X)$ una sección no nula tal que $s^n = 0$. Entonces
			como $s ̸= 0$, existe $x \in U$ tal que el gérmen $s|_x ̸= 0$ y $s^n|_x = 0$, lo que es absurdo pues los anillos locales son reducidos.

		\item Para todo anillo, tenemos el homomorfismo de anillos $\pi \colon A \to A/\nilrad = A_{\rm red}$
			que induce $\pi^a \colon \Spec(A_{\rm red}) = (\Spec A)_{\rm red} \to \Spec A$.
			Pegando estos morfismos de esquemas, pues son compatibles, obtenemos el morfismo $X_{\rm red} \to X$.
		\item Ejercicio para el lector.
		\item Sea $U$ un abierto afín sobre $X$, luego $Z\cap U$ es un cerrado en $U$, por ende,
			$\mathscr{O}_Z(Z \cap U ) = \mathscr{O}_X(U)/\mathfrak{a}$ para algún $\mathfrak{a} \nsle \mathscr{O}_X(U)$,
			donde $Z \cap U = \VV_U(\mathfrak{a})$.
			Si $Z$ es reducido, entonces $\mathfrak{a}$ debe ser radical.
			Si $\mathscr{O}_Z (Z \cap U )$ tuviese otra estructura sería de la forma $Z \cap U = \VV(\mathfrak{b})$,
			donde $\mathfrak{b} = \rad\mathfrak{b} = \rad\mathfrak{a} = \mathfrak{a}$ por el teorema~\ref{thm:ring_nullstellensatz}.
			Para probar la existencia, basta elegir $U_i$ un cubrimiento de $X$ por abiertos afínes y dado $Z \cap U_i = \VV_U(\mathfrak{a}_i)$,
			podemos definir $\mathscr{O}_Z (Z \cap U_i) := \mathscr{O}_X(U_i)/\rad\mathfrak{a}_i$. \qedhere
	\end{enumerate}
\end{proof}
El funtor de reducción $(-)_{\rm red}$ será importante más adelante y su mención no es puramente estética.

\begin{prop}
	Un esquema es íntegro syss es reducido e irreducible.
\end{prop}
\begin{proof}
	$\implies.$ Claramente un esquema íntegro es reducido.
	Si $X$ no fuera irreducible, entonces existirían dos abiertos $U_1, U_2$ no vacíos disjuntos, luego nótese que $\mathscr{O}_X(U_1 \cup U_2)
	\cong \mathscr{O}_X(U_1) \times \mathscr{O}_X(U_2)$ por proposición~\ref{thm:amalg_sum_sch}, el cual no es un dominio íntegro.

	$\impliedby.$ Sea $U$ un abierto de $X$ y sean $f, g \in \mathscr{O}_X(U)$ tales que $fg = 0$.
	Entonces sean $Y := \{ x \in U : f|_x \in \mathscr{m}_{X, x} \}$ y $Z := \{ x \in U : g|_x \in \mathfrak{m}_{X, x} \}$ subconjuntos cerrados,
	por la proposición~\ref{thm:qcqs}, de $U$ e $Y \cup Z = U$.
	Como $X$ es irreducible, entonces $U$ también, por lo que alguno de los dos conjuntos $Y, Z$ es $U$.
	Digamos que $Y = U$, entonces $f|_x \in \mathfrak{m}_{X, x}$ para todo $x \in U$, luego eligiendo un abierto afín $V = \Spec A$,
	entonces vemos que $\emptyset = V \cap U_f = \DD_A(f)$, luego es fácil notar que $f|_V$ debe ser nilpotente, pues $f$ está contenido en todos los
	primos de $A$ y $\nilrad(A) = \bigcap_{\mathfrak{p} \in \Spec A} \mathfrak{p}$ por el teorema~\ref{app:radical}.
	Pero como $f|_V$ es nilpotente y $A$ es reducido, entonces $f|_V = 0$ para todos los abiertos afínes, luego $f = 0$ como se quería probar.
\end{proof}
Por la proposición~\ref{thm:generic_pts}, entonces todo esquema íntegro posee un único punto genérico.

\begin{cor}\label{thm:integral=loc_int+connected}
	Un esquema es íntegro syss es conexo y cada anillo local $\mathscr{O}_{X, x}$ es un dominio íntegro.
\end{cor}
\begin{proof}
	$\implies.$ Trivial.

	$\impliedby.$ Sea $U = \Spec A$ un abierto afín no vacío.
	Sea $\xi$ un punto genérico de $U$, entonces $A_\xi$ es un anillo local artiniano y es un dominio íntegro, luego es un cuerpo
	(teorema~\ref{app:artinian_integral}), por lo que, $\xi = (0)$ es primo y $A$ es un dominio íntegro.
	Ahora veamos que $\mathscr{O}_X(X)$ es íntegro; si $fg \in \mathscr{O}_X(X)$ son tales que $f g = 0$, entonces sin perdida de generalidad
	supongamos que $f|_U = 0$ en el abierto afín $U$.
	Sea $V$ otro abierto afín no vacío, si $V \cap U = \emptyset$, entonces $\mathscr{O}_X(V \cup U) = \mathscr{O}_X(V) \times \mathscr{O}_X(U)$,
	de modo que $V \cup U$ es un abierto afín de espectro no íntegro, lo cual es absurdo.
	Así que $V$ se corta con $U$ y como $\VV_V(f ) \supseteq U \cap V$ es denso, entonces $\VV_V(f ) = V$.
	Así que $f|_V = 0$ para todo abierto afín, y luego $f = 0$.
\end{proof}

\begin{lem}
	Sea $A$ un dominio íntegro, $K := \Frac A$ y sea $\xi := (0) \in \Spec A =: X$.
	Entonces $\mathscr{O}_{X,\xi} = K$ y para todo abierto no vacío $U \subseteq X$ se cumple que $\xi \in U$,
	de modo que el homomorfismo canónico $\mathscr{O}_X (U ) \to \mathscr{O}_{X,\xi}$ es inyectivo.
	Si $U \supseteq V ̸\ne \emptyset$ es abierto, entonces la restricción $\rho_V^U \colon \mathscr{O}_X(U ) \to \mathscr{O}_X(V)$ también es inyectiva.
\end{lem}
\begin{proof}
	Nótese que $\mathscr{O}_{X,\xi} = A(0) = K$, luego para todo $U = \DD(f)$ identificamos $\mathscr{O}_X(\DD(f)) \cong A[1/f] \subseteq K$
	lo que corresponde al homomorfismo canónico.
	Más generalmente, si $s \in \mathscr{O}_X(U)$ con $U = \bigcup_{i=1}^n \DD(f_i)$ es tal que $s|_{\DD(f_i)} = 0$, entonces $s = 0$,
	lo que da que el homomorfismo canónico $\mathscr{O}_X(U) \to K$ sea inyectivo.
	Finalmente como la composición $\mathscr{O}_X(U ) \to \mathscr{O}_X(V ) \to K$ es inyectiva, entonces $\rho_V^U$ lo es.
\end{proof}

\begin{prop}
	Sea $X$ un esquema íntegro con punto genérico $\xi$. Entonces:
	\begin{enumerate}
		\item Sea $V$ un abierto afín, entonces el homomorfismo canónico $\mathscr{O}_X (V ) \to \mathscr{O}_{X,\xi}$
			induce un isomorfismo $\Frac(\mathscr{O}_X(V)) \cong \mathscr{O}_{X,\xi}$.
		\item Para todo abierto $U \subseteq X$ y todo punto $x \in X$, los homomorfismos $\mathscr{O}_X (U ) \to \mathscr{O}_{X,x}$
			y $\mathscr{O}_{X,x} \to \mathscr{O}_{X,\xi}$ son inyectivos.
		\item Identificando $\mathscr{O}_X (U )$ y $\mathscr{O}_{X,x}$ como subanillos de $\mathscr{O}_{X,\xi}$ se tiene que:
			$$ \mathscr{O}_X (U ) = \bigcap_{x\in U} \mathscr{O}_{X, x}. $$
	\end{enumerate}
\end{prop}
\begin{proof}
	\begin{enumerate}
		\item Basta notar que $\xi \in V$ también es punto genérico de $V = \Spec A$ como esquema.
			Luego $\mathscr{O}_X(V) \cong A \to \mathscr{O}_{V,\xi} = \mathscr{O}_{X,\xi} = \Frac A$ por el lema anterior.

		\item Sea $f \in \mathscr{O}_X (U )$ tal que $f|_x = 0$, luego $f|_W = 0$ para algún entorno afín de $x$.
			Sea $V \subseteq U$ un entorno afín arbitrario de $x$, luego como $X$ es irreducible, $W \cap V ̸= \emptyset$ y como
			$f|_{W \cap V} = 0$ entonces $f|_V = 0$, pues $\mathscr{O}_X (V ) \hookto \mathscr{O}_X (V \cap W )$ es inyectiva por el lema anterior.
			Así, por axioma de pegado, $f = 0$.

			El hecho de que $\mathscr{O}_{X,x} \to \mathscr{O}_{X,\xi}$ sea inyectivo puede corroborarse pasando
			a un entorno afín de $x$, donde se reduce a notar que $A_{\mathfrak{p}} \hookto A_{(0)}$ es
			inyectivo para todo $\mathfrak{p} \in \Spec A$.

		\item Podemos reducirnos al caso afín, el cual es fácilmente verificable. \qedhere
	\end{enumerate}
\end{proof}

\begin{exn}[plano afín sin orígen]\label{thm:plane_without_origin}
	Sea $k$ un cuerpo y considere el plano afín $X := \A^2_k$.
	Ahora, considere el abierto $U := X \setminus \{ (x, y) \}$ el cual es un esquema; vamos a probar que $U$ no es un esquema afín.

	Nótese que $U = \DD(x) \cup \DD(y)$, esto es pues $\mathfrak{p} \in \DD(x)$ syss $\mathfrak{p} ̸\supseteq (x)$, y lo
	mismo con $y$, luego $\mathfrak{p} \notin U$ syss $\mathfrak{p} \subseteq (x)$ y $\mathfrak{p} \supseteq (y)$, por lo que
	$\mathfrak{p} \subseteq (x, y)$ y el punto $(x, y)$ es maximal.

	Luego considere la restricción $\rho^X_U \colon k[x, y] \to \Gamma(U, \mathscr{O}_X)$.
	Una sección $s \in \Gamma(U, \mathscr{O}_X )$ es tal que es el pegado de una sección
	$$ f(x, y) := s|_{\DD(x)} \in \Gamma(\DD(x), \mathscr{O}_X ) \cong k[x, y][1/x] $$
	y una sección $g(x, y) := s|_{\DD(y)} \in \Gamma(\DD(y), \mathscr{O}_X )$ tales que $f|_{\DD(x)\cap\DD(y)} = g|_{\DD(x)\cap\DD(y)}$.
	Como $k[x, y]$ es un dominio íntegro, entonces la restricción a abiertos es siempre inyectiva, entonces tenemos que la sección $s \in k(X) = k(x, y)$
	solo tiene como posibles denominadores múltiplos de $x$ (dados en $f$) o múltiplos de $y$ (dados en $g$), y como $f$ no tiene denominador con $y$,
	entonces $s$ tampoco.
	Así, $s \in k[x, y]$ y se ve que $\rho^X_U$ es un isomorfismo.

	Finalmente, si $U$ fuera afín, entonces $\rho^X_U \colon \mathscr{O}_X(X) \to \mathscr{O}_U(U)$ determina de forma única el morfismo de esquemas
	$\iota \colon U \to X$ que es necesariamente la inclusión, y como entre esquemas afínes el funtor $\Spec(-)$ es una equivalencia de categorías
	tendríamos que $\iota$ es un isomorfismo de esquemas, pero claramente no es suprayectivo.
\end{exn}

% \begin{exn}[el plano con dos orígenes]
% 	Sea $A$ un anillo fijo.
% 	Sean $X := Y := \A^2_A$ y sea $U := V = X \setminus \{ (x, y) \}$.
% 	Luego, aplicando la proposición~\ref{thm:amalg_sum_sch} podemos pegar $X, Y$ a través de $U \cong V$ y así obtener un esquema $Z$
% 	que llamaremos el plano con dos orígenes.
% \end{exn}

En espíritu del mismo ejemplo, haga el siguiente ejercicio:
\begin{prob}
	Sea $k$ un cuerpo y sea $\PP^n_k =: X$ un espacio proyectivo.
	Demuestre que las secciones globales son $\Gamma(\PP^n_k , \mathscr{O}_X ) = k$ y concluya que $\PP^n_k$ no es un esquema afín.
\end{prob}

\begin{thm}
	Sean $X, Y$ esquemas íntegros con puntos genéricos $\xi, \eta$ resp.
	Para un morfismo $f \colon X \to Y$ son equivalentes:
	\begin{enumerate}
		\item $f$ es dominante.
		\item $f^\sharp \colon \mathscr{O}_Y \to f_*\mathscr{O}_X$ es inyectiva.
		\item $f(\xi) = \eta$.
		\item $\eta \in f[X]$.
	\end{enumerate}
\end{thm}
\begin{proof}
	$1 \implies 2$. Supongamos, por contradicción, que existe un abierto $V \subseteq Y$ tal que
	$f^\sharp \colon \mathscr{O}_Y (V ) \to \mathscr{O}_X (f^{-1} [V ])$ no es inyectiva.
	Es decir, existe $h \in \mathscr{O}_Y (V )$ no nula, tal que $f_V^\sharp (h) = 0$.
	Como $Y$ es íntegro, las restricciones son inyectivas, luego podemos suponer que $V$ es afín, y podemos reducirnos al abierto $\DD(h)$,
	de modo que $h$ sea invertible en $\mathscr{O}_Y (V )$.

	Ahora bien, como $f$ es dominante, existe $x \in X$ tal que $f (x) \in U $.
	Luego $f_x^\sharp (h|_{f(x)} ) = 0$, pero $h|_{f (x)}$ es invertible en $\mathscr{O}_{Y,f(x)}$,
	así que $(f_x^\sharp)^{-1}[\mathfrak{_{X, x}}] = \mathscr{O}_{Y,f(x)} \supset \mathfrak{m}_{Y, f(x)}$, lo cual es absurdo.

	$2 \implies 1$. Si $f[X]$ no fuera denso, entonces existe un abierto $V \ne \emptyset$ de $Y$
	tal que $f^{-1}[V] = \emptyset$, pero el homomorfismo $f_V^\sharp \colon \mathscr{O}_Y (V ) \to 0$ no es inyectivo.

	$1 \implies 3$. Sea $V ̸\ne \emptyset$ un abierto de $Y$, entonces $f^{-1} [V ]$ es abierto y no vacío en $X$, luego es denso (pues $X$ es irreducible) y,
	por lo tanto, $\xi \in f^{-1} [V ]$, o equivalentemente, $f (\xi) \in V$.
	Así, vemos que $f(\xi)$ es denso en $Y$ y luego debe ser $\eta$.

	Las implicancias $3 \implies 4 \implies 1$ son triviales.
\end{proof}
\begin{mydef}
	Sea $X$ un esquema íntegro con único punto genérico $\xi$.
	Su \strong{cuerpo de funciones racionales}\index{cuerpo!de funciones racionales} es $K(X) := \mathscr{O}_{X,\xi}$.
	Una función racional $f \in K(X)$ se dice \strong{regular}\index{función!regular} en un punto $x \in X$ si $f \in \mathscr{O}_{X,x}$,
	o se dice \strong{regular} en un abierto $U ̸\ne \emptyset$ si $f \in \mathscr{O}_X(U)$.
\end{mydef}
El inciso 3 de la proposición anterior, ahora se escribe como que una función es regular en un abierto syss es regular en todos sus puntos.

% Proposición 3.100: Sea k un cuerpo y X un k-conjunto algebraico íntegro. Entonces para todo U \subseteq X abierto no vacío se tiene que dim U =
% dim X = trdegk k(X).
% Demostración: Como X es un esquema íntegro, entonces posee un único
% punto genérico \xi y como es denso \xi \in U , luego k(U ) = \mathscr{O}_U,\xi = \mathscr{O}_X,\xi = k(X).
% Supongamos que X = \Spec A es un esquema afín. Entonces dim X =
% \kdim A por la proposición 3.14 y \kdim A = trdegk (\Frac(A)) por el teorema A.44 y k(X) = \Frac A (¿por qué?).
% Proposición 3.101: Sea X un esquema. Entonces para todo punto x \in
% X se cumple que codim({x}, X) = \kdim(\mathscr{O}_X,x ).
% Definición 3.102: Se dice que un esquema X es catenario si tiene dimensión finita y toda cadena maximal de cerrados irreducibles tiene la misma
% longitud.
% El inciso 2 del teorema A.44 nos dice:
% Proposición 3.103: Sea k un cuerpo. Todo k-conjunto algebraico afín
% es catenario.
% Corolario 3.104: Sea k un cuerpo y X un k-conjunto algebraico afín
% irreducible. Entonces para todo punto cerrado x \in X se cumple que dim X =
% \kdim(OX,x ).

El opuesto a los puntos genéricos son los puntos cerrados.
Si localmente un punto genérico <<ve todas las funciones racionales>>, entonces localmente un punto cerrado ve muy pocas:
\begin{prop}
	Sea $k$ un cuerpo y $X$ un $k$-esquema algebraico.
	Un punto $x \in X$ es cerrado syss $\kk(x)$ es una extensión finita de $k$.
	En particular, si $k$ es algebraicamente cerrado, entonces $x \in X$ es cerrado syss $\kk(x) = k$.
\end{prop}
\begin{proof}
	Basta notar que en una carta afín un punto es cerrado syss corresponde a un ideal maximal (corolario~\ref{thm:spec_closure}),
	luego su cociente es una extensión de cuerpos de $k$.
	Como en un conjunto algebraico, las cartas afínes son $k$-álgebras de tipo finito, entonces los puntos cerrados tienen cuerpos de
	restos $\kk(x)$ que son extensiones de cuerpo de $k$ de tipo finito, por lo que, por
	el lema de Zariski, deben ser extensiones finitas de cuerpo.
\end{proof}

\begin{mydef}
	Sea $k$ un cuerpo y $X$ un $k$-esquema algebraico. Denotamos por $\clpt X$ a los puntos cerrados de $X$.
\end{mydef}

\begin{cor}
	Sea $k$ un cuerpo y $X$ un $k$-esquema algebraico. Se cumplen:
	\begin{enumerate}
		\item $\clpt X ̸\ne \emptyset$.
		\item Para todo $U \subseteq X$ abierto se cumple que $\clpt U = U \cap \clpt X$.
		\item $\clpt X$ es un subconjunto denso de $X$.
	\end{enumerate}
\end{cor}
\begin{proof}
	Basta notar que cada carta afín admite un punto cerrado, y la proposición anterior nos da un criterio global para detectar puntos
	cerrados. Más aún, como cada abierto afín siempre posee puntos cerrados,
	entonces $\clpt X$ corta a cada abierto no vacío.
\end{proof}
Este corolario será fuertemente mejorado en la versión esquemática del lema de Zariski (vid.\ teo.~\ref{thm:scheme_zariski_lemma}).

\begin{prop}\label{thm:qc_sch_many_closed_pts}
	Sea $X$ un esquema compacto.
	Entonces, todo punto $x \in X$ se especializa $x \speto s$ en un punto cerrado.
\end{prop}
\begin{proof}
	Como todo cerrado es un esquema compacto, basta probar que $X$ posee un punto cerrado.
	Como $X$ es compacto, entonces se cubre por finitos abiertos afines $U_i = \Spec(A_i)$.
	Podemos elegir ésta expresión de modo que sea irredundante, vale decir, tal que $U_i \nsubseteq U_j$ con $i \ne j$,
	para ello, si tenemos una inclusión $U_i \subseteq U_j$ eliminamos a $U_i$.
	Elijamos un punto $s \in U_1 \setminus \left( \bigcup_{j=2}^n U_j \right)$ y que sea cerrado en $U_1$ (es decir, maximal),
	como $s \in \left( \bigcup_{j=2}^n U_j \right)^c$, el cual es cerrado en $X$, entonces $ \overline{\{ s \}} \cap \left( \bigcup_{j=2}^{n} U_j \right)^c$
	también es cerrado en $X$.
	Finalmente, es fácil verificar que si un punto $y \in \overline{\{ s \}}$, entonces debe estar en $U_1$,
	luego $y = s$ y así, vemos que $s$ es cerrado como se quería probar.
	% para ello reemplazamos $U_i \cap (\bigcup_{j\ne i} U_j)$
	% % Esencialmente, emplearemos el hecho de que <<ser cerrado>> puede verificarse localmente.
	% Reordenemos la lista de modo que $x \in U_i$ con $1 \le i \le n$, pero $x \notin U_j$ con $n < j \le m$.
	% Así pues $x \in \left( \bigcup_{j=n+1}^{m} U_j \right)^c$ que es cerrado y podemos hacer verificaciones de ser cerrado aquí.
	% \par
	% De la lista, borraremos aquellos que no contienen a $x$; así que $x \in \bigcap_{i=1}^{n} U_i$ y, por la proposición anterior,
	% sea $V = \Spec A$ un entorno de $x$ que es un abierto principal de cada $U_i$ (esto se logra por finitud de los $U_i$'s).
	% Así, vemos a $x$ como un primo $\mathfrak{p}$ de $A$ y, luego, se especializa en un ideal maximal $\mathfrak{m}$ de $A$,
	% que se corresponde a un punto $s \in V$.
	% \par
	% \underline{Veamos que $s$ es cerrado:}
	% sea $y \in X$ tal que $y \in \overline{\{ s \}}$, es decir, $y$ pertenece a todo entorno de $s$ y, en particular, pertenece a $V$.
	% Así $y \in V \cap \overline{\{ s \}} = \overline{\{ s \}}_V = \{ s \}$ como se quería probar.
	% Consideremos $F := \overline{\{ x \}}$, este conjunto es un cerrado irreducible y es fácil notar que si $U_i$
	% es un entorno afín de $x$, entonces $U \supseteq F$.
	% Sea $\mathfrak{p}_i$ el primo correspondiente a $x$ en $U_i$, luego se especializa en cada abierto a un ideal maximal $\mathfrak{m}_i \in \Spec(A_i)$
	% Mirando al punto $x \in U_i$
	% Sea $X$ compacto. Diremos que un subespacio $S$ posee la propiedad $\mathcal{P}$ si todo cerrado $F \subseteq S$ no vacío contiene un punto cerrado.
	% Como los abiertos afínes poseen la propiedad $\mathcal{P}$ y finitos de éstos cubren a $X$, entonces basta probar que si $U, V$ son abiertos
	% que poseen $\mathcal{P}$, entonces $U \cup V$ posee $\mathcal{P}$.
	% \par
	% Supongamos que $X = U \cup V$, donde $U, V$ poseen $\mathcal{P}$ y donde $U \nsubseteq V$.
	% Luego $U \setminus V ̸= \emptyset$ es un cerrado en $U$, por lo que posee un punto $x \in U \setminus V$ cerrado en $U$.
	% Si $x$ es cerrado en $X$, estamos listos; si no $\overline{\{ x \}} \cap U = \overline{\{ x \}}$ y $\overline{\{ x \}} \cap V$ es un cerrado en $V$
	% que posee un punto $y$ cerrado en $V$.
	% Entonces $\overline{\{ y \}} \cap V = \overline{\{ y \}}$ y $\overline{\{ y \}} \cap U \subseteq \overline{\{ x \}} \cap U = \overline{\{ x \}}$,
	% pero $x \notin \overline{\{ y \}}$ porque $U$ es un entorno de $x$ que no corta a $\overline{\{ y \}}$, así que $\overline{\{ y \}} = \{ y \}$
	% como se quería probar.
\end{proof}
El paso clave en la demostración, el uso de <<compacidad>> está en el encontrar el cubrimiento irreducible.
Si el esquema no es compacto, podríamos tener un cubrimiento por abiertos afines creciente e infinito; así la unión de todos ellos seguiría siendo un abierto,
pero podría no ser afin.
En el ejemplo~\ref{exn:no_closed_points} damos una construcción de un esquema sin puntos cerrados y precisamente esto es lo que falla.

Debido a su naturaleza, los puntos, mientras más especializados, más pequeños resultan sus cuerpos de restos, de modo que son más <<sencillos>>
y así tenemos todo un espectro recorriendo desde los puntos genéricos a los cerrados.
\begin{prop}
	Sea $k$ un cuerpo, sean $X, Y$ un par de $k$-esquemas algebraicos y $f \colon X \to Y$ un morfismo de $k$-esquemas.
	Entonces $f[\clpt X] \subseteq \clpt Y$.
\end{prop}
\begin{proof}
	Basta aplicar la proposición~\ref{thm:mono_fibers}.
\end{proof}

\begin{prop}
	Sea $X$ un esquema, y sean $U, V \subseteq X$ dos abiertos afines en $X$.
	Todo punto $x \in U \cap V$ posee un entorno $W \subseteq U \cap V$ tal que $W$ es un abierto principal tanto en $U$ como en $V$.
\end{prop}
\begin{proof}
	Como los abiertos principales son una base, existe $f \in \Gamma(V, \mathscr{O}_X)$ tal que $x \in \DD_V(f) \subseteq U \cap V$.
	Nótese que $V' := \DD_V(f) \cong \Spec(V[1/f])$ y todo abierto principal $\DD_{V'}(g/f^n)$ de $V'$ corresponde a un abierto principal $\DD_V(fg)$
	de $V$, de modo que sustituyendo $V$ por $V'$ podemos suponer que $V \subseteq U$.

	Como los abiertos principales son base, existe $h \in \Gamma(U, \mathscr{O}_X)$ tal que $x \in \DD_U(h) \subseteq V$.
	Consideremos la restricción $\rho_V^U \colon \Gamma(U, \mathscr{O}_X) \to \Gamma(V, \mathscr{O}_X)$, así vemos que $\DD_U(h) = \DD_V(h|_V)$
	(¿por qué?), así que éste abierto cumple lo exigido.
\end{proof}

% \begin{mydef}
% 	Un espacio topológico $X$ se dice \strong{cuasiseparado}\index{cuasiseparado (espacio)} si la intersección de dos abiertos compactos es compacto.
% \end{mydef}
% \begin{prop}
% 	Un esquema es compacto y cuasiseparado syss puede cubrirse por finitos abiertos afínes.
% \end{prop}
% \begin{proof}
% 	$\implies .$ Trivial.

% 	$\impliedby.$ Los esquemas afínes son compactos, así que $X$ también.
% 	Para ver que $X$ es cuasiseparado, basta probar que todo esquema afín $X = \Spec A$ es cuasiseparado.
% 	Sean $U_1, U_2$ dos abiertos compactos de $X$, entonces se pueden escribir como unión finita de abiertos principales:
% 	$$ U_1 = \bigcup_{i=1}^n \DD(f_i), \qquad U_2 = \bigcup_{j=1}^{m} \DD(g_j) $$
% 	Nótese que como $\DD(f_i) \simeq \Spec(A[1/f_i])$ (en $\mathsf{Top}$), vemos que cada $\DD(f_i)$ es compacto. Luego
% 	$$ U1 \cap U2 = \bigcup_{i=1}^{n} \bigcup_{j=1}^{m} (\DD(f_i) \cap \DD(g_j)) = \bigcup_{i=1}^{n} \bigcup_{j=1}^{m} \DD(f_ig_j), $$
% 	el cual es abierto y compacto.
% \end{proof}
% Por ejemplo, los espacios topológicos noetherianos son compactos y cuasiseparados.

\begin{mydef}
	Sea $X$ un esquema.
	Se dice que una propiedad $\mathcal{P}$ de subesquemas abiertos afines es \strong{local para afines}\index{local para afines (propiedad)} si:
	\begin{enumerate}[{LAf}1.]
		\item Si $\Spec A \subseteq X$ satisface $\mathcal{P}$, entonces todo abierto principal $\Spec(A[1/f]) \subseteq X$
			también satisface $\mathcal{P}$.
		\item Supongamos que $\Spec A = \bigcup_{i=1}^{n} \DD_A(f_i) $, o equivalentemente $A = (f_1, \dots, f_n)$,
			donde cada $\DD_A(f_i)$ satisface $\mathcal{P}$.
			Entonces $\Spec A$ satisface $\mathcal{P}$.
	\end{enumerate}
\end{mydef}
\begin{lem}[comunicación afín]\index{lema!de comunicación afín}
	Sea $X$ un esquema.
	Sea $\mathcal{P}$ una propiedad local para afines en $X$.
	Si $X = \bigcup_{i\in I} U_i$, donde cada $U_i$ es un abierto afín que satisface $\mathcal{P}$,
	entonces cada abierto afín de $X$ satisface $\mathcal{P}$.
\end{lem}
\begin{proof}
	Sea $V \subseteq X$ un abierto afín.
	Luego $V = \bigcup_{i\in I} (U_i \cap V)$, pero mejor aún, cada intersección $U_i \cap V$ puede cubrirse por abiertos $U_{ij}$
	que son principales tanto en $U_i$ como en $V$ por la proposición anterior.
	Así, $V = \bigcup_{ij} U_{ij}$ y, como los esquemas afines son compactos, podemos suponer que la unión es finita.
	Por LAf1, sabemos que cada $U_{ij}$ satisface $\mathcal{P}$ y, por LAf2, concluimos que $V$ también satisface $\mathcal{P}$ como se quería probar.
\end{proof}
Éste es un metalema muy conocido que nos permitirá demostrar que varias propiedades para abiertos afines de un esquema
pueden verificarse en cubrimientos.
Éste resultado es bastante sencillo, al punto de que rara vez se menciona en un libro de geometría algebraica y se le utiliza en demasía;
la denominación <<lema de comunicación afín>> es de \citeauthor{vakil:rising_sea}~\cite{vakil:rising_sea}.

\begin{mydef}
	Un esquema $X$ se dice \strong{localmente noetheriano}\index{esquema!localmente noetheriano}
	si todo punto $x \in X$ posee un entorno afín $x \in U$ tal que $\Gamma(U, \mathscr{O}_X)$ es un anillo noetheriano.
	% si posee un cubrimiento por esquemas afínes $\Spec(A_i)$, donde cada $A_i$ es un anillo noetheriano.
	Un esquema se dice \strong{noetheriano}\index{esquema!noetheriano} si es localmente noetheriano y compacto, o equivalentemente,
	si posee un cubrimiento finito de esquemas afínes noetherianos.
\end{mydef}
Claramente, todo esquema noetheriano posee un espacio topológico noetheriano, pero el recíproco es falso:
para verlo puede notar que todo espacio topológico finito es trivialmente noetheriano y hay varios anillos no noetherianos con espectro finito.

\begin{prop}
	Un esquema $X$ es localmente noetheriano syss para todo abierto afín $U = \Spec A$ se cumple que $A$ es noetheriano.
	En particular, un esquema afín $\Spec A$ es localmente noetheriano syss $A$ es noetheriano.
\end{prop}
\begin{proof}
	$\impliedby$. Es trivial, puesto que los abiertos afínes son una base de cualquier esquema.

	$\implies$.
	Para ello basta verificar que la propiedad $\mathcal{P}$ de que el abierto afín $U$ sea tal que su anillo $\Gamma(U, \mathscr{O}_X)$ es noetheriano
	es local para afines.
	La propiedad LAf1 es trivial, puesto que si $A$ es noetheriano, entonces $A[1/f]$ también.
	% Sea $B$ un anillo noetheriano, entonces cada localización $B[1/f ]$ es noetheriana y $\DD(f) \cong \Spec(B[1/f ])$,
	% así que un esquema es localmente noetheriano syss posee una base de esquemas afínes dados por anillos noetherianos.

	% Nos reducimos a probar lo siguiente: un esquema afín $\Spec A$ es localmente noetheriano syss $A$ es noetheriano.
	% Sea $U \cong \Spec B$ con $B$ noetheriano y sea $f \in A$ tal que $\DD(f) \subseteq U $, entonces $A[1/f ] \cong B[1/f ]$ es noetheriano.
	% Así, podemos cubrir el $\Spec A$ con finitos $\{ \DD(f_i) \}^r_{i=1} $, lo que equivale a que $(f_1, \dots, f_r) = (1)$, tales que cada $A[1/f_i]$
	% es noetheriano.
	Probaremos LAf2: sea $\Spec A \subseteq X$ y sea $\Spec A = \bigcup_{i=1}^{n} \DD(f_i)$, donde cada $A[1/f_i]$ es noetheriano.
	Afirmamos que para todo ideal $\mathfrak{a} \nsle A$ se cumple
	$$ \mathfrak{a} = \bigcap_{i=1}^{r} \varphi_i^{-1}\big[ \varphi_i[\mathfrak{a}] \cdot A[1/f_i] \big], $$
	donde $\varphi_i \colon A \to A[1/f_i ]$ es el homomorfismo canónico.

	Si fijamos un $\varphi_i$, entonces vemos que la expresión $\varphi_i^{-1}\big[ \varphi_i[\mathfrak{a}] \cdot A[1/f_i] \big] = \mathfrak{a}^{ec} $,
	en notación de extensión y contracción de ideales, y siempre se cumple que $\mathfrak{a} \subseteq \mathfrak{a}^{ec}$ (cfr. \cite{Alg}, prop. 6.38).
	Veamos la inclusión <<$\supseteq$>>: Sea $b \in \bigcap_{i=1}^{r} \mathfrak{a}^{ec}$, es decir, para cada $i$ se cumple que $\varphi_i(b) = a_i/f_i^n$
	para algún $a_i \in \mathfrak{a}$ y algún $n \in N$ suficientemente grande.
	Esto equivale a que $f_i^m(f_i^nb - a_i) = 0$ para algún $m$ suficientemente grande, y como los $f_i$'s generan el 1, también los $f_i^N$'s
	con $N = n + m$, luego
	$$ b = \sum_{i=1}^{r} f_i^N g_i b = \sum_{i=1}^{r} a_i g_i f_i^m \in \mathfrak{a}. $$
	Ahora, dada una cadena de ideales $\mathfrak{a}_1 \subseteq \mathfrak{a}_2 \subseteq \cdots $, podemos extenderla
	mediante un homomorfismo $\varphi_i \colon A \to A[1/f_i]$:
	$$ \mathfrak{a}^e_1 \subseteq \mathfrak{a}^e_2 \subseteq \mathfrak{a}^e_3 \subseteq \cdots, $$
	y como $A[1/f_i]$ es noetheriano, entonces ésta cadena se estabiliza, dijamos en $n_i$.
	Eligiendo el máximo $n := \max\{ n_1, \dots , n_r\}$ vemos que la cadena en $A$ se estabiliza en dicho índice.

	Finalmente, concluimos por el lema de comunicación afín.
\end{proof}

% \begin{prop}
% 	Todo esquema localmente noetheriano tiene puntos cerrados.
% \end{prop}


\input{morfismos.tex}

\input{haces_cuasicoherentes.tex}

\chapter{Propiedades locales}

\section{Dimensión}
Recuérdese que la \strong{dimensión (combinatorial)} de un espacio topológico $X$ es el $n$ supremo tal que existe una cadena $F_0 \subset F_1 \subset \cdots
\subset F_n$ de cerrados irreducibles de $X$.
Se sigue de la definición que $\dim\emptyset = -\infty$ (pues es el supremo de un conjunto vacío).
Dado un cerrado irreducible $Y \subseteq X$ habíamos definido la \strong{codimensión (combinatorial)}, denotada $\codim(Y, X) = n$, como el supremo
tal que existe una cadena de cerrados irreducibles que comienza en $Y = F_0 \subset F_1 \subset \cdots \subset F_n$.
Claramente, se sigue que la dimensión de $X$ es el supremo de las codimensiones.

\begin{prop}
	Sea $X$ un espacio topológico. Entonces:
	\begin{enumerate}
		\item Para todo $Y \subseteq X$ se cumple que $\dim Y \le \dim X$.
		\item Si $X = \bigcup_{i=1}^n X_i$, donde $X_i$ son cerrados en $X$, entonces $\dim X = \max_i \{ \dim X_i \}$.
	\end{enumerate}
\end{prop}

\begin{mydef}
	Sea $X$ un espacio topológico y sea $x \in X$ un punto.
	Entonces se define la \strong{dimensión local} en $x$ como
	$$ \dim_x X := \inf\{ \dim U : x \in U \text{ abierto} \}. $$
\end{mydef}

\begin{prop}
	Sea $X$ un espacio topológico.
	\begin{enumerate}
		\item Dado un punto $x \in X$, un entorno $U$ de $x$ y una familia $Y_i \subseteq X$ finita de cerrados tales que
			$x \in Y_i$ para todo $i$ y $U \subseteq \bigcup_{i=1}^n Y_i$, entonces $\dim_x X = \max_i\{ \dim_x Y_i \}$.
		\item Se cumple que $\dim X = \sup_{x\in X} \dim_x X$.
		\item Si $\{ X_i \}_{i\in I}$ es un cubrimiento abierto o un cubrimiento cerrado localmente finito de $X$,
			entonces \smash{$\displaystyle \dim X = \sup_{i\in I} \dim X_i$}.
		\item Sea $X$ un espacio $T_0$ y noetheriano, y sea $F$ el conjunto de puntos cerrados de $X$,
			entonces \smash{$\displaystyle \dim X = \sup_{x\in F} \dim_x X$}.
		\item Para todo cerrado irreducible $Y \subseteq X$ se cumple que
			$$ \dim Y + \codim(Y, X) \le \dim X. $$
		\item Para todo par de cerrados irreducibles $Y \subseteq Z \subseteq X$ se cumple que
			$$ \codim(Y, Z) + \codim(Z, X) \le \codim(Y, X). $$
		\item Un cerrado irreducible $Y$ tiene codimensión 0 syss es una componente irreducible.
	\end{enumerate}
\end{prop}
\begin{proof}
	\begin{enumerate}
		\item De la proposición anterior, para todo $V \subseteq U$ entorno de $x$ se tiene que
			$$ \dim_x X = \inf_V \max_i \{ \dim(Y_i \cap V) \}, $$
			y para cada $Y_i$ vemos que $\dim_x Y_i = \inf_V \dim(Y_i \cap V)$.
			Si para algún $i$ tenemos que $\dim_x Y_i = \infty$ es evidente que $\dim_x X = \infty$,
			así que, si no, tenemos que $\dim_x Y_i$ alcanza un valor finito (mínimo) para cada $i$, así que existe un entorno
			$V_0 \subseteq U$ de $x$ tal que $\dim_x Y_i = \dim(Y_i \cap V_0)$ y así para todo $V \subseteq V_0$ tenemos que
			el valor es el mismo, de lo que se sigue el enunciado.
		\item Claramente $\dim_x X \le \dim X$ para todo punto $x \in X$.
			Por otro lado, sea $\emptyset \ne F_0 \subset \cdots \subset F_n$ una cadena de irreducibles en $X$ y sea $x \in F_0$;
			para todo entorno $U$ de $x$, vemos que $U \cap F_i$ es cerrado en $U$ y $\overline{U \cap F_i} = F_i$ (¿por qué?),
			así que los $U \cap F_i$'s son cerrados irreducibles en $U$ y tenemos que $\dim U \ge n$.
		\item Si el cubrimiento es abierto y $x \in X_i$, entonces $\dim_x X \ge \dim_x X_i$.
			Si el cubrimiento es cerrado, $x \in X$ y $U$ es un entorno de $x$ que corta a finitos cerrados, entonces
			$$ \dim_x X \le \dim U = \max_i \{ \dim(U \cap X_i) \} \le \sup_i \dim X_i. $$
		\item Basta notar que si un espacio es $T_0$, entonces todo cerrado irreducible minimal $F$ corresponde a un punto cerrado.
		\item[5-7.] Triviales.
			\qedhere
	\end{enumerate}
\end{proof}
Como ejericio, ¿dónde empleamos que $X$ es noetheriano en el inciso 4?

\begin{cor}\label{thm:isol_iff_loc_dim_zero}
	Sea $X$ un espacio topológico $T_0$ y noetheriano.
	\begin{enumerate}
		\item $X$ tiene $\dim X = 0$ syss es discreto, finito (y no vacío).
		\item Un punto $x \in X$ tiene $\dim_x X = 0$ syss está aislado.
	\end{enumerate}
\end{cor}
\begin{proof}
	\begin{enumerate}
		\item Todas las componentes irreducibles tienen dimensión 0, luego, como $X$ es $T_0$ es fácil ver que corresponden a puntos,
			y como es noetheriano, posee finitas componentes irreducibles, es decir, finitos puntos cerrados, luego abiertos.
		\item Basta notar que el punto posee un entorno $T_0$ y noetheriano de dimensión 0. \qedhere
	\end{enumerate}
\end{proof}

\begin{prop}
	Sea $A$ un anillo, y sea $X := \Spec A$.
	\begin{enumerate}
		\item $\dim X = \kdim A = \kdim(A/\nilrad) = \dim(X_{\rm red})$.
		\item Para todo primo $\mathfrak{p} \nsl A$, se cumple que:
			$$ \kdim(A_{\mathfrak{p}}) = \alt\mathfrak{p} = \codim(\overline{\{ x_{\mathfrak{p}} \}}, X), \qquad
			\kdim(A/\mathfrak{p}) = \dim\overline{\{ x_{\mathfrak{p}} \}}. $$
		\item $\dim X = \sup\{ \kdim(A_{\mathfrak{m}}) : \mathfrak{m} \in \mSpec A \}$.
		\item $\dim X = \sup\{ \kdim(A/\mathfrak{p}) : \mathfrak{p} \text{ minimal} \}$.
	\end{enumerate}
\end{prop}
Antes de seguir vamos a dar un ejemplo relevante de la clase de posibles desastres en la teoría de la dimensión:
\begin{ex}
	Sea $(A, \mathfrak{m})$ un dominio de valuación discreta y sea $X := \Spec A$.
	Luego $\dim X = \kdim A = 1$, y $X = \{ \xi, x \}$, donde $\xi = (0)$ y $x = \mathfrak{m}$.
	Como $x$ corresponde a un ideal maximal, entonces es un punto cerrado del espacio, luego $U := X \setminus \{ x \} = \{ \xi \}$ es abierto
	y, además, es denso.
	Pero $\dim U = 0$ pues es un punto y $\dim X = 1 \ne \dim U$.
\end{ex}
Así, por ejemplo, vemos que en general no podemos calcular dimensión en un abierto, realmente es necesario un cubrimiento.

El teorema del ascenso y del descenso de Cohen-Seidenberg tienen la siguiente consecuencia esquemática:
\begin{cor}
	Sea $f \colon X \to Y$ un morfismo entero (e.g.\ finito) de esquemas.
	Entonces $\dim X = \dim Y$.
\end{cor}

% \begin{mydef}
% 	Sea $k$ un cuerpo.
% 	Un $k$-esquema $X$ se dice una \strong{variedad afín}\index{variedad!afín} sobre $k$ si es un conjunto algebraico afín íntegro sobre $k$.
% \end{mydef}
Y una traducción directa del teorema~\ref{app:dimension_kAlg_ft}:
\begin{thm}\label{thm:integral_affine_are_biequidim}
	Sea $k$ un cuerpo y sea $X$ una variedad afín sobre $k$. Entonces:
	\begin{enumerate}
		\item $\dim X = \trdeg_k K(X) = \trdeg_k K(U) = \dim U$, para cualquier abierto no vacío $U \subseteq X$.
		\item Toda cadena maximal de cerrados irreducibles en $X$ tiene la misma longitud.%
			\footnote{Una cadena de cerrados irreducibles $F_0 \subset F_1 \subset \cdots$ se dice \textit{maximal} si no existe un cerrado irreducible
				$Y$ tal que $F_j \subset Y \subset F_{j+1}$. \citeauthor{ega-iv1}~\cite{ega-iv1} emplea el adjetivo \textit{saturado}.}
		\item Para todo punto $x \in X$ se tiene que
			$$ \dim\overline{\{ x \}} + \codim(\overline{\{ x \}}, X) = \dim X. $$
	\end{enumerate}
\end{thm}

\begin{ex}
	Sea $k$ un cuerpo. Entonces:
	\begin{enumerate}
		\item $\A^n_k$ es una variedad afín y $\dim(\A^n_k) = n$.
		\item $\dim(\PP^n_k) = n$, pues $\PP^n_k = \bigcup_{i=0}^{n} U_i$, donde cada
			\[
				U_i \cong \Spec(A[t_0/t_i, \dots, t_n/t_0]) \cong \A^n_k.
			\]
		\item Sea $X = \Spec( k[x, y]/(y^2 - x^3) )$, entonces $\dim X = 1$,
			pues $k[x, y](y^2 - x^3) = k[x, \sqrt{x^3}]$ es una extensión entera de $k[x]$.
	\end{enumerate}
\end{ex}

\begin{mydef}
	Se dice que un espacio topológico $X$ es \strong{equidimensional}\index{equidimensional (espacio topológico)} o que tiene
	\strong{dimensión pura}\index{dimensión!pura} $n$ si todas sus componentes irreducibles tienen dimensión $n$.
	% Se dice que $X$ es \strong{equicodimensional}\index{equicodimensional} si todo cerrado irreducible minimal tiene igual codimensión.
\end{mydef}
Así, sustituyendo ser un $k$-esquema \textit{íntegro} por \textit{de dimensión pura}, entonces los resultados del teorema anterior se preservan.

\begin{mydef}
	Se dice que un espacio topológico $X$ es \strong{catenario}\index{espacio topológico!catenario}\index{catenario (espacio topológico)}
	si para todo par de cerrados irreducibles $Y \subseteq Z$ se cumple que $\codim(Y, Z) < \infty$ y toda cadena maximal de cerrados irreducibles
	$$ Y = F_0 \subset F_1 \subset \cdots \subset F_n = Z $$
	tiene igual longitud $n = \codim(Y, Z)$.

	Se dice que un esquema $X$ es \strong{universalmente catenario}\index{esquema!universalmente catenario} si todo $X$-esquema de tipo finito es catenario.
\end{mydef}
\begin{prop}
	% Sea $X$ un espacio topológico tal que para todo par de cerrados irreducibles $Y \subseteq Z$ se cumpla que $\codim(Y, Z) < \infty$
	% (e.g., si $X$ es un espacio topológico noetheriano).
	Para un espacio topológico $X$, son equivalentes:
	\begin{enumerate}
		\item $X$ es catenario.
		\item $X$ admite un cubrimiento por abiertos catenarios.
		\item Para todo par de cerrados irreducibles $Y \subseteq Z$ se cumple que $\codim(Y, Z) < \infty$,
			y para todo trío de cerrados irreducibles $Y \subseteq Z \subseteq W$ se cumple que
			$$ \codim(Y, Z) + \codim(Z, W) = \codim(Y, W). $$
	\end{enumerate}
	% Si además $X$ es $T_0$, noetheriano y de dimensión pura finita, las anteriores equivalen a:
	% \begin{enumerate}
	% 	\item 
	% \end{enumerate}
\end{prop}
\begin{proof}
	$1 \implies 2$ y $1 \implies 3$ son triviales.

	$2 \implies 1$. Basta notar que la función $F \mapsto \overline{F}$ determina una biyección entre los cerrados de un subespacio abierto $U \subseteq X$
	y los cerrados de $X$ que cortan a $U$.

	$3 \implies 1$. Fijemos $Y \subseteq Z$ dos cerrados irreducibles en $X$ como extremos.
	Basta probar que toda cadena con extremos $Y, Z$ tiene igual longitud por inducción sobre la longitud, notando que si hay una cadena maximal de longitud 1,
	entonces es trivial, y si hay una cadena más larga, la descomponemos en dos para emplear la hipótesis inductiva.
\end{proof}
\begin{cor}
	Los espectros de cuerpos son universalmente catenarios y, en consecuencia, todo esquema algebraico sobre un cuerpo $k$ es universalmente catenario.
\end{cor}
\begin{proof}
	Por el teorema~\ref{thm:integral_affine_are_biequidim} se sigue que todo esquema algebraico afín irreducible sobre $k$ es catenario, y claramente
	todo esquema algebraico admite un cubrimiento por afines irreducibles, luego también es catenario.
\end{proof}
\warn
Ojo que ser catenario no implica ser equidimensional.
Es claro que un esquema sobre un cuerpo puede constar de varias componentes irreducibles de distinta dimensión.
Nagata construyó el primer ejemplo de un anillo noetheriano que no es catenario; en el apéndice hay un ejemplo de un esquema afín que no es catenario.

\begin{prop}\label{thm:dimension_on_alg_base_change}
	Sea $k$ un cuerpo y sea $X$ un esquema algebraico sobre $k$ de dimensión pura $n$.
	Para toda extensión algebraica $L/k$ se cumple que $X_L$ tiene dimensión pura $n$.
\end{prop}

\begin{mydef}
	Sea $X$ un espacio topológico.
	Una función $f \colon X \to \R$ se dice \strong{semicontinua superior}\index{semicontinua!superior} si todo punto $x \in X$
	posee un entorno abierto $U$ tal que para todo $z \in U$ se tiene que $f(x) \ge f(z)$.
	Una función $g \colon X \to \R$ se dice \strong{semicontinua inferior}\index{semicontinua!inferior} si $-g$ es semicontinua superior.
\end{mydef}
\begin{prop}
	Sea $X$ un espacio topológico.
	Si $f$ es semicontinua superior, entonces para todo $x \speto y$ se tiene que $f(x) \le f(y)$.
\end{prop}

La proposición anterior da una idea de cómo pensar la semicontinuidad. Por ejemplo:
\begin{prop}
	Sea $X$ un espacio topológico $T_0$.
	La función $x \mapsto \dim_x X$ es semicontinua superior.
\end{prop}
\begin{proof}
	Si $x$ es tal que $\dim_x X = \infty$ entonces es claro.
	Si $\dim_x X = d < \infty$, entonces, por definición de dimensión local, existe un entorno $x \in U_0$
	tal que $\dim U = d$ para todo $U$ abierto que satisfaga $x \in U \subseteq U_0$.
	Así pues, para todo $y \in U_0$, vemos que todo entorno $V \subseteq U_0$ satisface $\dim V \le \dim U_0$, por lo que
	$\dim_y X \le d$.
\end{proof}
Nótese que $x \mapsto \codim( \overline{\{ x \}}, X )$ no es ni semicontinua superior, ni semicontinua inferior.
Si $A$ es un anillo de valuación y $X := \Spec A = \{ \eta, s \}$, donde $\eta$ es genérico, entonces el único entorno de $s$ es $X$ mismo,
y $\codim( \overline{\{ \eta \}}, X ) = 0 < 1 = \codim(\overline{\{ s \}}, X)$.
No obstante, si $Y := \Spec\Z$ y $\xi \in Y$ es el punto genérico de $Y$, vemos que $\codim(\overline{\{ \xi \}}, Y) = 0$,
pero todo entorno de $\xi$ contiene puntos de codimensión 1.

\section{Esquemas normales}
Aquí haremos fuerte uso de definiciones y propiedades de dependencia
íntegra, un resumen de los resultados se encuentra en el apéndice \S\ref{sec:integral_dep}.

\begin{mydef}
	Un esquema $X$ se dice \strong{normal} si para cada punto $x \in X$
	se cumple que el anillo local $\mathscr{O}_{X, x}$ es normal.
	Un esquema $X$ normal de $\dim X \le 1$
	se dice un esquema de Dedekind.
\end{mydef}
\begin{cor}
	Un anillo $A$ es normal syss el esquema $\Spec A$ es normal.
\end{cor}
\begin{prop}
	Todo esquema compacto posee puntos cerrados.
\end{prop}
\begin{prop}
	Sea $X$ un esquema irreducible. Son equivalentes:
	\begin{enumerate}
		\item $X$ es normal.
		\item Para todo abierto $U \subseteq X$, el anillo $\mathscr{O}_X(U)$ es normal.
	\end{enumerate}
	Si además $X$ es compacto, entonces ambas son equivalente a:
	\begin{enumerate}[resume]
		\item Las fibras $\mathscr{O}_{X,x}$ son normales en todos los puntos cerrados.
	\end{enumerate}
\end{prop}
\begin{proof}
	La equivalencia $1 \iff 2$ sale de la definición de anillo
	normal y es trivial que $1 \implies 3$.

	$3 \implies 1$. Sea $x$ un punto arbitrario del esquema. Entonces existe un
	punto cerrado $y \in \overline{\{ x \}}$, y así $\mathscr{O}_{X, y}$ es normal, y como $\mathscr{O}_{X, x}$ es una localización
	de $\mathscr{O}_{X, y}$, entonces $\mathscr{O}_{X, x}$ también es normal.
\end{proof}
\begin{ex}
	Sea $A$ un anillo normal (e.g., $A = k$ un cuerpo).
	Como $A[x_1, \dots, x_n]$ es normal (proposición A.36), entonces $\A^n_A$ y $\PP^n_A$ son esquemas normales.
\end{ex}
\begin{cor}
	Sea $X$ un esquema íntegro noetheriano. Entonces $X$ es
	de Dedekind syss cada $\Gamma(U, \mathscr{O}_X)$ es de Dedekind.
\end{cor}
\begin{mydef}
	Un esquema $X$ se dice \strong{factorial} si para cada punto
	$x \in X$, se cumple que la fibra $\mathscr{O}_{X, x}$ es un DFU.
\end{mydef}

\begin{prop}
	Se cumple:
	\begin{enumerate}
		\item Todo esquema factorial es normal.
		\item Si $X$ es un esquema de Dedekind, entonces todos los anillos locales $\mathscr{O}_{X, x}$ son DIPs.
		\item Sea $X$ un esquema irreducible de $\dim X = 1$.
			Entonces $X$ es de Dedekind syss $X$ es normal.
	\end{enumerate}
\end{prop}

\begin{thm}
	Sea $X$ un esquema normal localmente noetheriano. Para
	todo $F \subseteq X$ cerrado de $\codim(F, X) \ge 2$, la restricción
	$$ \Gamma(X, \mathscr{O}_X ) \longrightarrow \Gamma(X \setminus F, \mathscr{O}_X ) $$
	es un isomorfismo. Es decir, toda función regular en $X \setminus F$ se extiende a $X$.
\end{thm}
\begin{proof}
	Sea $X = \Spec A$ afín. Para todo $\mathfrak{p} \in \Spec A$ de $\alt\mathfrak{p} = 1$
	se cumple que $\mathfrak{p} \in X \setminus F$, luego concluimos el teorema pues
	\begin{equation}
		A = \bigcap_{\substack{\mathfrak{p} \in \Spec A \\ \alt\mathfrak{p} = 1}} A_{\mathfrak{p}}.
		\tqedhere
	\end{equation}
\end{proof}

\begin{prop}\label{thm:rational_map_extension}
	Sea $S$ un esquema localmente noetheriano, $X$ un $S$-esquema íntegro normal de tipo finito e $Y$ un $S$-esquema propio.
	Sea $f \colon U \to Y$ un morfismo definido sobre un abierto $\emptyset ̸= U \subseteq X$.
	Entonces $f$ se extiende de manera única a un morfismo $\overline{f}\colon V \to Y$, donde $V$ es un abierto de $X$ que
	contiene a todos los puntos de codimensión 1.
	Más aún, si $X$ es de Dedekind, podemos exigir $V = X$.
\end{prop}
\begin{proof}
	La unicidad sale del hecho de que el morfismo estructural $Y \to S$ es separado, y del teorema~\ref{thm:separatedness_ext}.
	Sea $\xi \in X$ el punto genérico,
	entonces $\xi \in U$ e induce un morfismo $f_\xi \colon \Spec k(X) \to Y$.
	Sea $x \in X$ de codimensión 1, entonces $\mathscr{O}_{X,x}$ es un dominio de valuación discreta con
	$\Frac(\mathscr{O}_{X,x}) = k(X)$, luego se extiende a $f_x \colon \Spec \mathscr{O}_{X,x} \to Y$.
	Como $Y$ es de tipo finito sobre $S$, entonces $f_x$ se extiende a $g\colon U_x \to Y$ para un entorno $U_x$
	de $x$.
	Sea $W \subseteq Y$ un entorno afín de $g(x)$ y considere las restricciones de $f, g$ a $U^\prime := f^{-1} [W ] \cap g^{-1} [W ]$.
	Así, $U^\prime$ es un abierto no vacío pues contiene a $\xi$.
	Los homomorfismos de anillos $\mathscr{O}_Y (W ) \to \mathscr{O}_X (U^\prime )$ inducidos por $f, g$ son
	idénticos pues coinciden en $k(X) \supseteq U^\prime$, así que $f|_{U^\prime} = g|_{U^\prime}$ (pues el codominio
	es afín, teorema 3.75), así que coinciden en $U \cap U_x$ (nuevamente, por el teorema~\ref{thm:separatedness_ext}).
	Así iteramos el proceso para cada punto de codimensión 1 y pegamos en $V$.
\end{proof}

\begin{lem}
	Sea $A$ un dominio de valuación discreta con $K := \Frac A$,
	y sea $k$ el cuerpo de restos de $A$. Sea $X$ un $A$-esquema con $\Gamma(U, \mathscr{O}_X)$ una
	$A$-álgebra plana para todo abierto afín $U \subseteq X$. Si $X_K$ es normal y $X_k$ es
	reducido, entonces $X$ es normal.
\end{lem}
\begin{proof}
	Podemos reducirnos al caso en que $X = \Spec B$ es afín.
	Por hipótesis, el homomorfismo canónico $B = B \otimes_A A \to B \otimes_A K$ es inyectivo.
	Sea $t$ un uniformizador de $A$. Sea $\beta \in \Frac B$ entero sobre $B$; como $B \otimes_A K$
	es normal, existe $b \in B$ y $r \in \Z$ tal que $\beta = bt^{-r}$.
	Sea $b \notin tB$, probaremos que $r \le 0$. Como $\beta$ es entero, sea
	$$ \beta^n + c_{n-1} \beta^{n-1} + \cdots + c_1 \beta + c_0 = 0, \qquad c_i \in B. $$
	Si $r > 0$, multiplicando por $t^{rn}$, vemos que $b$ es nilpotente en $B/tB$, así que
	$b \in tB$ lo que es absurdo. Así que $r \le 0$ y $\beta \in B$.
\end{proof}

\begin{mydef}
	Sea $X$ un esquema íntegro. Un morfismo $\pi \colon X^\prime \to X$
	se dice una \strong{normalización}\index{normalización} si $X^\prime$ es normal y todo morfismo $f \colon Y \to X$
	dominante con $Y$ normal se factoriza:
	\begin{center}
		\begin{tikzcd}
			{} & Y \dar["f"] \dlar["\exists!"'] \\
			X' \rar["\pi"'] & X
		\end{tikzcd}
	\end{center}
\end{mydef}

\begin{lem}
	Sea $A$ un dominio íntegro con $K := \Frac A$. Sea $A' := \mathcal{O}_K/A$
	la clausura íntegra de $A$ en $K$, entonces $\pi := (\iota^a) \colon \Spec(A') \to \Spec A$ es la
	normalización.
\end{lem}
\begin{proof}
	Sea $f \colon Y \to X$ un morfismo dominante con $Y$ normal,
	entonces viene de un homomorfismo de anillos $\varphi \colon A \to \mathscr{O}_Y (Y)$, el cual es
	inyectivo pues $f$ es dominante. Así que se factoriza como
	\begin{tikzcd}[cramped, sep=small]
		A \rar & A' \rar["g"] & \mathscr{O}_Y(Y)
	\end{tikzcd}
	lo que induce el morfismo de esquemas $g^a \colon Y \to \Spec(A')$.
\end{proof}

Notamos que la normalización descrita por el lema anterior es una aplicación birracional por el teorema 4.89.

\begin{prop}
	Sea $X$ un esquema íntegro.
	Entonces existe una única normalización $\pi \colon X^\prime \to X$ salvo isomorfismo (de $\mathsf{Sch}/X$).
	Más aún, $f \colon Y \to X$ es una normalización syss $Y$ es normal y $f$ es un morfismo entero y birracional.
\end{prop}
\begin{hint}
	Basta ir pegando abiertos afínes y empleando la unicidad de la normalización.
\end{hint}

\begin{mydef}
	Sea $X$ un esquema íntegro y sea $L \supseteq k(X)$ una extensión algebraica de cuerpos.
	Se le llama una \strong{normalización} de $X$ en $L$ a un
	morfismo $\pi \colon X^\prime \to X$ entero con $X^\prime$ normal y $k(X^\prime) = L$ que extiende al morfismo canónico $\Spec L \to X$.
\end{mydef}

Del mismo modo que se probaba la proposición anterior se prueba:
\begin{prop}
	Sea $X$ un esquema íntegro y sea $L/k(X)$ una extensión algebraica.
	Entonces existe una única normalización $\pi \colon X^\prime \to X$ de $X$
	en $L$ salvo isomorfismo (de $\mathsf{Sch}/X$).
\end{prop}

\begin{prop}
	Sea $X$ un esquema noetheriano normal y sea $L/k(X)$ una extensión algebraica separable.
	La normalización $X^\prime \to X$ de $X$ en $L$ es un morfismo finito y, en consecuencia, $\dim(X') = \dim X$.
\end{prop}
\begin{proof}
	Ésta es una traducción de la proposición A.42.
\end{proof}

\begin{prop}
	Sea $k$ un cuerpo, sea $A := k[t_1, \dots, t_n]$ el álgebra
	polinomial, sea $K := \Frac A$ y sea $L/K$ una extensión finita (posiblemente inseparable).
	Entonces la clausura íntegra de $A$ en $L$ es un $A$-módulo finitamente generado.
\end{prop}
\begin{proof}
	Sabemos que $L$ puede verse como $L/L_i/K$, donde $L_i/K$
	es una extensión puramente inseparable y $L/L_i$ es separable. Así, por la
	proposición anterior, podemos reducirnos al caso en que $L/K$ sea puramente
	inseparable. Supongamos que $\car K =: p > 0$ y sea $\alpha_1, \dots, \alpha_n$ una $K$-base
	de $L$. Sabemos que existe $q := p^r$ para algún $r > 0$ tal que cada $\alpha_i^q \in
	K$ (teorema A.15). Fijemos una clausura algebraica%
	\footnote{Tecnicamente aquí fijamos dos cosas: una clausura algebraica de $K$ y un monomorfismo $K \hookto \algcl K$,
		y también un $K$-monomorfismo $L \hookto \algcl K$.}
	$\algcl K \supseteq L$, de modo que $L \subseteq k^\prime[\beta_1, \dots, \beta_n]$ donde cada $\beta_i = t_i^{1/q} \in \algcl K$,
	y donde $k^\prime/k$ es la extensión generada por añadir las raíces $q$-ésimas de los coeficientes de los
	$\alpha_i$'s. Así $k^\prime[\beta_1, \dots, \beta_n]/A$ es una extensión de anillos entera y finita, luego
	$B \subseteq k^\prime[\beta_1, \dots, \beta_n]$ también es un $A$-módulo finitamente generado.
\end{proof}

\begin{prop}
	Sea $X$ un esquema algebraico íntegro sobre un cuerpo $k$, y sea $L/K(X)$ una extensión finita de cuerpos.
	Entonces la normalización $X^\prime \to X$ de $X$ en $L$ es un morfismo finito, de modo que $X^\prime$ es un esquema algebraico íntegro sobre $L$.
\end{prop}
\begin{proof}
	Podemos reducirnos al caso de $X = \Spec A$ afín. Sea
	$k[t_1, \dots, t_n] \hookto A$ un monomorfismo finito (dado por el teorema de normalización de Noether),
	luego la clausura íntegra $B$ de $A$ en $L$ es la misma que la de $k[\vec t]$ en $L$, la cual es finitamente generada como $k[\vec t]$-módulo, por tanto
	también como $A$-módulo; así el morfismo $X^\prime \to X$ es finito.
\end{proof}

\begin{prop}
	Sea $X$ un esquema íntegro cuya normalización sea finita.
	Entonces el conjunto de puntos normales de $X$ es abierto.
\end{prop}
\begin{proof}
	Podemos reducirnos al caso en que $X = \Spec A$ es afín.
	Ahora la normalización de $X$ es $\Spec B$ con $B := \mathcal{O}_{\Frac(A)/A}$.
	Sea $U$ el conjunto de puntos normales de $X$, de modo que $x_{\mathfrak{p}} \in U$ syss $(B/A) \otimes_A A_{\mathfrak{p}} = 0$,
	luego sea $\mathfrak{a} := \Ann_A(B/A)$. Claramente $\DD(\mathfrak{a}) = \Spec A \setminus \VV(\mathfrak{a}) \subseteq U$.
	Recíprocamente, sea $\mathfrak{p} \in U$, entonces como $B/A$-módulo finitamente generado, existe
	$a \in A \setminus \mathfrak{p}$ tal que $a(B/A) = 0$ (¿por qué?), luego $\mathfrak{p} \notin \DD(a)$ y así se tiene
	igualdad, donde $\DD(a)$ es claramente abierto.
\end{proof}
\begin{cor}
	Sea $X$ un esquema algebraico íntegro sobre un cuerpo $k$. Entonces la normalización sí es finita y el conjunto de puntos normales de $X$ es abierto.
\end{cor}
\begin{proof}
	Resulta de aplicar los dos últimos resultados.
\end{proof}

\begin{thm}[Krull-Akizuki]
	Sea $X$ un esquema íntegro noetheriano de $\dim X = 1$, sea $L/K(X)$ una extensión finita de cuerpos y sea $\pi\colon X' \to X$ su normalización en $L$.
	Entonces $X'$ es de Dedekind y para todo subesquema cerrado propio $Z \subset X$, la restricción $\pi^{-1}[Z] \to Z$ es un morfismo finito.
\end{thm}
\begin{proof}
	Claramente $\dim(X') = 1$ y $X'$ es normal, luego basta probar que es noetheriano y que la restricción $\pi^{-1}[Z] \to Z$.
	Ambas propiedades son locales y podemos suponer que $X$ es afín, de lo que se deduce del teorema algebraico de Krull-Akizuki.
\end{proof}

\section{Esquemas regulares}
\begin{mydef}
	Sea $X$ un esquema y $x \in X$ un punto. Sea $(\mathscr{O}_{X,x}, \mathfrak{m}_x , \kk(x))$
	el anillo local en $x$, entonces el \strong{espacio tangente de Zariski}\index{espacio!tangente de Zariski} en $x$ es:
	\begin{equation*}
		T_{X,x} := (\mathfrak{m}_x /\mathfrak{m}^2_x )^\vee
		= \Hom_{\kk(x)}(\mathfrak{m}_x /\mathfrak{m}^2_x , \kk(x)) = \mathfrak{m}_x  \otimes_{\mathscr{O}_{X,x}} \kk(x).
	\end{equation*}
\end{mydef}
Sea $f \colon X \to Y$ un morfismo entre esquemas. Para todo $x \in X$, el morfismo
$f$ induce un homomorfismo de anillos:
\[
	T_{f,x} \colon T_{X,x} \to T_{Y,y} \otimes_{\kk(y)} \kk(x).
\]
\begin{prop}
	Si $X$ es un esquema localmente noetheriano, entonces
	para todo punto $x \in X$ se tiene que $\dim_{\kk(x)} T_{X,x} \ge \kdim(\mathscr{O}_{X,x} )$.
\end{prop}
\begin{proof}
	Es una traducción de la proposición A.52.
\end{proof}

\begin{prop}
	Sean $f \colon X \to Y, g \colon Y \to Z$ un par de morfismos
	entre esquemas y sea $x \in X$. Entonces $T_{f \circ g,x} = T_{f,x} \circ  (T_{g,f(x)} \otimes_{\kk(y)} \Id_{\kk(x)})$.
\end{prop}

\begin{mydef}
	Sea $X$ un esquema.
	Se dice que un punto $x\in X$ es \strong{regular}\index{punto!regular} en $X$ si su anillo local $\mathscr{O}_{X, x}$ es regular,
	vale decir, si $\kdim(\mathscr{O}_{X, x}) = \kdim( T_{X, x} ) = \dim_{\kk(x)}( \mathfrak{m}_x / \mathfrak{m}_x^2 )$;
	de lo contrario se dice que $x$ es \strong{singular}\index{punto!singular}.
	Se dice que $X$ es un esquema \strong{regular}\index{esquema!regular} si todos sus puntos son regulares.
\end{mydef}
Geométricamente uno piensa que <<singular>> se traduce o en una variedad con <<puntas>> o <<con cruces>>;
el segundo tipo es más fácil de ilustrar:
\begin{ex}
	Sea $k$ un cuerpo y tómese $X = \VV(x) \cup \VV(y) \subseteq \A^2_k$ la unión de los dos ejes.
	Este subesquema es afín y $A := \Gamma(X, \mathscr{O}_X) = k[x, y]/(xy)$; aquí denotaremos por $u$ la imagen de $x$, y por $v$ la imagen de $y$,
	los cuales satisfacen $uv = 0 \in A$.

	Un punto $P := (a, b) \in k^2$ lo asociamos al primo $\mathfrak{m}_P = (x - a, y - b) \in \A^2_k$;
	de modo que podemos verificar que $X$ es regular en los puntos de la forma $(a, 0), (0, a)$ con $a \in k^\times$ y en los puntos genéricos.
	El razonamiento es que en la localización $B := A_{(u - a, v)}$, por definición, $u$ es invertible, de modo que $v = (1/u)uv = 0$.
	Así que $B \cong k[x]_{(x-a)}$ el cual es un dominio de valuación discreta (¿por qué?), luego es regular.%
	\footnote{Tecnicamente verificamos regularidad en los puntos $k$-racionales; el resto de puntos se trabaja análogamente.}
	Si localizamos $A_{(u)}$ (pues $uA$ es un primo minimal), entonces aquí $v$ es invertible, de modo que $u = uv(1/v) = 0$,
	por lo que $A_{(u)} = k(v)$, el cual es un cuerpo.

	Finalmente, el punto $(0, 0)$ correspondiente al ideal $(u, v)$ es singular en $X$ y la razón está en que la localización $B := k[u, v]_{(u, v)}$
	satisface que su maximal es $(u, v)B$, luego su espacio tangente de Zariski es
	$$ T_{X, (0,0)} = \frac{(u, v)B}{(u^2, v^2)B} \cong k\langle 1, \overline{u}, \overline{v} \rangle, $$
	donde el símbolo de la derecha significa <<el $k$-espacio vectorial con base $1, \overline{u}, \overline{v}$>>, donde $\overline{()}$ representa
	la imagen salvo el cociente.
	Este $k$-espacio vectorial tiene dimensión 2, mientras que $X$ tiene dimensión 1.
\end{ex}

% Ahora bien, por regularidad de Serre, ser regular es cerrado bajo generización, luego:
\begin{prop}\label{thm:regularity_under_gez}
	Sea $X$ un esquema noetheriano.
	\begin{enumerate}
		\item Si $X$ es regular en un punto $x$ y $x' \speto x$ es una generización,
			entonces $x'$ es regular.
		\item En consecuencia, $X$ es regular syss todos sus puntos cerrados son regulares.
		\item Si $X$ es regular, entonces todas sus componentes conexas son normales.
	\end{enumerate}
\end{prop}
\begin{proof}
	Los primeros dos incisos son una aplicación del teorema de regularidad de Serre,
	y el tercero es una aplicación del teorema de Auslander-Buchsbaum-Nagata.
\end{proof}

Ahora nos dedicaremos a probar el criterio del jacobiano para probar regularidad:
\begin{mydef}
	Sea $k$ un cuerpo, sea $Y := \A^n_k = \Spec(k[t_1, \dots, t_n])$ y sea $y \in Y(k)$ un punto racional.
	Definimos el homomorfismo $D_y\colon k[\vec t] \to \Hom_k(k^n, k) = (k^n)^\vee$ que para un polinomio $f(\vec t) \in k[\vec t]$ actúa como
	$$ D_y(f)(\vec u) := \sum_{i=1}^{n} \left.\frac{\partial f}{\partial t_i}\right|_{\vec t = y} u_i, $$
	a esta aplicación le llamamos el \strong{diferencial} de $f$ en $y$.
\end{mydef}

\begin{lem}
	Sea $k$ un cuerpo.
	Sea $Y := \A^n_k$, sea $y \in Y(k)$ un punto racional y $\mathfrak{m} := \mathfrak{p}_y$. Entonces:
	\begin{enumerate}
		\item $\mathfrak{m}^2 \subseteq \ker D_y$ de modo que $D_y\colon \mathfrak{m/m}^2 \to (k^n)^\vee$ es un isomorfismo.
		\item Existe un isomorfismo canónico $T_{Y, y} \cong k^n$.
	\end{enumerate}
\end{lem}

\begin{prop}
	Sea $k$ un cuerpo.
	Sea $X = \VV(\mathfrak{a})$ un subesquema cerrado de $Y := \A^n_k$, sea 
	\begin{tikzcd}[cramped, sep=small]
		f\colon X \rar[closed] & Y
	\end{tikzcd}
	el encaje cerrado canónico y sea $x \in X(k)$.
	La transformación $k$-lineal $T_{f, x}\colon T_{X, x} \to T_{Y, y}$ induce un isomorfismo de $T_{X, x}$ con $(D_x[\mathfrak{a}])^\perp \subseteq k^n$,
	así que
	$$ T_{X, x} \cong \left\{ \vec u \in k^n : \sum_{i=1}^{n} \left. \frac{\partial f}{\partial t_i} \right|_{\vec t = x} \cdot u_i = 0 \right\}. $$
\end{prop}

\begin{thmi}[Criterio del jacobiano]\index{criterio!del jacobiano}
	Sea $k$ un cuerpo.
	Sea $X = \VV(\mathfrak{a})$ un subesquema cerrado de $Y := \A^n_k$, sea 
	\begin{tikzcd}[cramped, sep=small]
		f\colon X \rar[closed] & Y
	\end{tikzcd}
	el encaje cerrado canónico, sea $x \in X(k)$ y sean $f_1, \dots, f_r$ tales que generan $\mathfrak{a}$.
	Considere la siguiente matriz, denominada el \strong{jacobiano}\index{jacobiano (matriz)} en $x$:
	$$ \mathcal{J}_x := \left[ \left. \frac{\partial f_i}{\partial t_j} \right|_{\vec t = x} \right] \in \Mat_{r\times n}(k), $$
	entonces el punto $x$ es regular en $X$ syss $\rang \mathcal{J}_x = n - \kdim(\mathscr{O}_{X, x})$.
\end{thmi}

\begin{thm}\label{thm:geo_irred_has_reg_clpt}
	Sea $k$ un cuerpo.
	Todo conjunto algebraico geométricamente irreducible sobre $k$ tiene un punto cerrado regular.
\end{thm}

\begin{mydef}
	Sea $X$ un esquema localmente noetheriano.
	Se denota por $\Sing X$ el conjunto de puntos singulares de $X$ y por $\Reg X$ el conjunto de puntos regulares.
\end{mydef}

\begin{lem}\label{lem:regular_pts_speto_closed_reg}
	Sea $K$ un cuerpo algebraicamente cerrado y sea $X$ un esquema algebraico sobre $K$.
	Entonces todo punto regular de $X$ se especializa en algún punto regular y cerrado.
	% Si $x \in X$ es un punto regular, entonces existe $x' \in X$ tal que $x \speto x'$ y es regular y cerrado.
\end{lem}
\begin{prop}
	Sea $K$ un cuerpo algebraicamente cerrado y sea $X$ un esquema algebraico sobre $K$.
	Entonces $\Reg X$ es un abierto en $X$.
	% Más aún, si $X$ es normal, entonces $\codim(\Sing X, X) \ge 2$.
\end{prop}

Los siguientes forman parte de un criterio fundamental de Serre en álgebra conmutativa.
\begin{mydef}
	Se dice que un esquema $X$ es \strong{regular en codimensión $r$}%
	\index{regular!en codimensión $r$}%
	\index{esquema!regular!en codimensión $r$}
	si para toda localización $\mathscr{O}_{X,x}$ de dimensión $\le r$ se cumple que $\mathscr{O}_{X, x}$ es regular.

	Se dice que un esquema $X$ satisface la propiedad $(S_r)$ si para todo punto $x \in X$ se satisface que
	$$ \prof(\mathscr{O}_{X, x}) \ge \min\{ \codim(\overline{\{ x \}}, X), r \}, $$
	donde <<$\prof$>> denota la profundidad de un anillo local.
\end{mydef}
La condición $S_r$ podemos leerla como que si $\prof(\mathscr{O}_{X, x}) < s$, entonces también $\codim( \overline{\{ x \}}, X ) < s$
para todo $s \le r$.
Ambas definiciones son generalizaciones naturales de sus análogos en álgebra conmutativa;
por definición un anillo es regular si es regular en codimensión $r$ para todo $r$, y un anillo es de Cohen-Macaulay si satisface $(S_r)$ para todo $r$.

Si $X$ es localmente noetheriano, entonces ser regular en codimensión 0 significa que para todo punto genérico $\xi \in X$
el anillo local $\mathscr{O}_{X, \xi}$, que es artiniano, sea regular.
Pero es sabido que un anillo local artiniano es regular syss es un cuerpo, por lo que $X$ debe ser reducido en $\xi$.

\begin{thm}[criterio de normalidad de Serre]
	Sea $X$ un esquema localmente noetheriano. Entonces:
	\begin{enumerate}
		\item $X$ es reducido syss es regular en codimensión 0 (o es genéricamente reducido) y satisface la propiedad $(S_1)$.
		\item $X$ es normal syss es regular en codimensión 1 y satisface la propiedad $(S_2)$.
	\end{enumerate}
	En consecuencia, si $X$ es normal, entonces $\codim(\Sing X, X) \ge 2$.
\end{thm}

\section{Morfismos}
\subsection{Morfismos planos}
\begin{mydef}
	Sea $f\colon X \to Y$ un morfismo de esquemas y sea $\mathscr{F}$ un $\mathscr{O}_X$-módulo.
	Se dice que $\mathscr{F}$ es \strong{$f$-plano sobre $Y$}\index{fplano@$f$-plano} en $x \in X$ si el homomorfismo de anillos
	$f_x^\sharp\colon \mathscr{O}_{Y, f(x)} \to \mathscr{O}_{X, x}$ induce que $\mathscr{F}_{x}$ sea un $\mathscr{O}_{Y, f(x)}$-módulo plano.
	Se dice que $\mathscr{F}$ es \strong{$f$-plano sobre $Y$} (a secas) si lo es en cada punto.

	Se dice que $\mathscr{F}$ es \strong{plano sobre $X$}\index{OXmodulo@$\mathscr{O}_X$-módulo!plano} en el punto $x$ si es $\Id_X$-plano en $x$,
	vale decir, si la fibra $\mathscr{F}_x$ es un $\mathscr{O}_{X, x}$-módulo plano.
	% Se dice que $\mathscr{F}$ es \strong{plano sobre $X$} (a secas) si es plano en cada punto.
	Se dice que el morfismo $f$ es \strong{plano}\index{morfismo!plano} (en $x \in X$) si $\mathscr{O}_X$ es $f$-plano sobre $Y$ (en $x$).
\end{mydef}

\begin{prop}
	Se cumplen:
	\begin{enumerate}
		\item Los encajes abiertos son planos.
		\item Los morfismos planos son estables salvo cambio de base.
		\item La composición de morfismos planos es plana.
		\item El producto fibrado de morfismos planos es plano.
		\item Sea $\varphi\colon A \to B$ un homomorfismo de anillos.
			Entonces $\varphi^a\colon \Spec B \to \Spec A$ es plano syss $\varphi$ es plano.
		\item Si $X$ es un esquema localmente noetheriano, entonces un haz $\mathscr{F}$ finitamente generado es plano syss es localmente libre.
	\end{enumerate}
\end{prop}

\begin{lem}\label{thm:flat_are_very_dominant}
	Sea $Y$ un esquema irreducible y sea $f\colon X \to Y$ un morfismo plano.
	Entonces todo abierto no vacío $U \subseteq X$ domina a $Y$ (i.e., $f[U]$ es denso).
	Más aún, si $X$ tiene finitas componentes irreducibles, entonces cada componente domina a $Y$.
\end{lem}

\begin{prop}
	Sea $Y$ un esquema con finitas componentes irreducibles y sea $f\colon X \to Y$ un morfismo plano.
	Si $Y$ es reducido (resp. irreducible, íntegro) y las fibras genéricas son reducidas (resp. irreducibles, íntegras),
	entonces $X$ es reducido (resp. irreducible, íntegro).
\end{prop}

\begin{prop}\label{thm:flat_morph_open}
	Sea $f\colon X \to Y$ un morfismo plano de tipo finito entre esquemas noetherianos, entonces es abierto.
\end{prop}
\begin{proof}
	Por el lema~\ref{thm:flat_are_very_dominant}, entonces $f$ es dominante y luego por el teorema~\ref{thm:dominant_image_noeth_sch} tiene que
	satisfacerse que $f[X]$ contiene un abierto denso $V$ en $Y$.
	Luego considere el cambio de base $f^{-1}[Y \setminus V] = X \times_Y (Y \setminus V) \to Y \setminus V$ el cual también es plano y de tipo finito
	entre esquemas noetherianos, por lo que la imagen contiene un abierto $U_1 \subseteq Y \setminus V$ el cual induce un abierto $V_1 \supseteq V$,
	y así sucesivamente construimos una sucesión de abiertos $V \subseteq V_1 \subseteq V_2 \subseteq \cdots$ (o por complementos, una sucesión descendente
	de cerrados) de modo que se estabiliza y prueba que $f[X]$ es abierto.

	Cambiando $X$ por un abierto $U$, es claro que $U$ sigue siendo noetheriano y que el morfismo sigue siendo plano de tipo finito, así que se ve que $f[U]$
	es abierto.
\end{proof}
\begin{ex}
	Es fácil comprobar que la proyección canónica $\A^{n+m}_k \to \A^n_k$ con $m \ge 0$ es un morfismo plano;
	por tanto la proyección es abierta (recuérdese que ya comprobamos que esta proyección no es cerrada en general).
	Esto coincide con la situación en topología general.
\end{ex}

\begin{prop}
	Sea $X$ un esquema reducido e $Y$ un esquema de Dedekind.
	Un morfismo $f\colon X \to Y$ es plano syss toda componentes irreducible domina a $Y$.
\end{prop}
\begin{cor}
	Sea $X$ un esquema íntegro e $Y$ un esquema de Dedekind.
	Todo morfismo no constante $X \to Y$ es plano.
\end{cor}

\begin{thm}\label{thm:flat_dim_formula}
	Sea $f\colon X \to Y$ un morfismo de esquemas localmente noetherianos.
	Sean $x \in X$ e $y := f(x)$, entonces
	$$ \kdim(\mathscr{O}_{X_y, x}) \ge \kdim( \mathscr{O}_{X, x} ) - \kdim(\mathscr{O}_{Y, y}). $$
	Más aún, si $f$ es plano, entonces se alcanza igualdad.
\end{thm}

\begin{cor}\label{thm:flat_fibers_dimension}
	Sea $k$ un cuerpo y sean $X, Y$ un par de conjuntos algebraicos sobre $k$ con $X$ equidimensional e $Y$ irreducible.
	Para todo morfismo $f\colon X \to Y$ plano y todo $y \in Y$ se comprueba que la fibra $X_y$ es equidimensional y
	$$ \dim X_y = \dim X - \dim Y. $$
\end{cor}

Si volvemos a tomar el ejemplo de la proyección canónica entre espacios afines podemos notar que trivialmente se cumple el teorema anterior,
puesto que las fibras son asímismas espacios afines (de dimensión constante).
% Un contraejemplo:
\begin{ex}
	Considere la proyección $\Spec(k[x, y, z]) \to \Spec(k[z]) = \A^1_k$ y precompongamos con el encaje cerrado por el subesquema cerrado
	$X := \VV(x^2 + y^2 - z^2)$.
	Denotando $f \colon X \to \A^1_k$ (ver fig.~\ref{fig:geo-alg/non_flat_morph}) notamos que las fibras $X_w$ con $w \ne 0 \in \A^1_k$ (un punto cerrado)
	son circunferencias y, en particular, tienen $\dim(X_w) = 1$ (formalice esto).
	Por el contrario, la fibra $X_0$ solo consta del punto $(0, 0, 0)$, de modo que tiene $\dim(X_0) = 0$, por lo que el morfismo $f$ no puede ser plano.
\end{ex}
\begin{figure}[!hbt]
	\centering
	\includegraphics[scale=1]{geo-alg/non_flat_morph.pdf}
	\caption{Un morfismo que no es plano.}%
	\label{fig:geo-alg/non_flat_morph}
\end{figure}

Finalmente, un último resultado que será útil:
\begin{thm}[planitud genérica]\index{teorema!de planitud genérica}
	Sea $f \colon X \to Y$ un morfismo dominante de tipo finito entre esquemas íntegros noetherianos.
	Entonces existe un abierto no vacío $U \subseteq Y$ tal que la restricción $f^{-1}[U] \to U$ es un morfismo fielmente plano.
\end{thm}
\begin{proof}
	En primer lugar, por el teorema~\ref{thm:dominant_image_noeth_sch} sabemos que $f[X]$ contiene a un abierto denso $V$,
	de modo que $f^{-1}[V] \to V$ es sobreyectivo.
	Restringiéndose a un abierto afín de $V$ y así mismo con $f^{-1}[V]$ podemos suponer que tenemos un morfismo dominante de tipo finito $\Spec B \to \Spec A$
	entre espectros de anillos noetherianos.
	Es decir, $A \subseteq B$ es una extensión de dominios íntegros noetherianos de tipo finito y podemos aplicar la planitud genérica de álgebra conmutativa.
\end{proof}
Como el teorema anterior engloba el caso afín, nos referiremos a éste como <<planitud genérica>>.
\begin{cor}\label{thm:dim_ineq_dominant_ft}
	Sea $f \colon X \to Y$ un morfismo dominante de tipo finito entre esquemas íntegros noetherianos, y sea $d$ la dimensión de la fibra genérica de $f$.
	Entonces:
	\begin{enumerate}
		\item Las componentes irreducibles de las fibras no vacías de $f$ tienen dimensión $\ge d$.
		\item Existe un abierto denso $U \subseteq Y$ tal que para todo $y \in U$ la fibra $X_y$ tiene dimensión pura $d$.
	\end{enumerate}
\end{cor}
\begin{proof}
	Por planitud genérica se obtiene inmediatamente el inciso 2.
	Pasando al abierto $U \subseteq Y$ y restringiéndose a $f|_{f^{-1}[U]}$ podemos suponer que tenemos un morfismo plano.
	Sean $\eta \in X, \xi \in Y$ los puntos genéricos; como $f$ es dominante, sabemos que $f(\eta) = \xi$ y, como $\xi \in U$,
	por el teorema~\ref{thm:flat_dim_formula} tenemos que $\dim(X_\xi) = \dim X - \dim Y$ y, aplicándo el mismo teorema concluimos el inciso 1.
\end{proof}

\subsection{Morfismos étale}
\begin{mydef}
	Sea $f \colon X \to Y$ un morfismo de tipo finito entre esquemas localmente noetherianos.
	% Sea $x \in X$ un punto e $y := f(x) \in Y$.
	Se dice que $f$ es \strong{no ramificado}\index{morfismo!no ramificado}\index{no ramificado!(morfismo)} en $x\in X$
	si el homomorfismo de anillos $f^\sharp_x \colon \mathscr{O}_{Y, f(x)} \to \mathscr{O}_{X, x}$ satisface que
	$$ \mathfrak{m}_{Y, f(x)} \mathscr{O}_{X, x} = \mathfrak{m}_{X, x} \iff \frac{\mathscr{O}_{X, x}}{\mathfrak{m}_{Y, f(x)} \mathscr{O}_{X, x}} \cong \kk(x) $$
	y si la extensión finita de cuerpos $\kk(x) \supseteq \kk(y)$ es separable.
	Se dice que $f$ es \strong{étale}\index{morfismo!etale@étale} en $x \in X$ si es no ramificado y plano en $x$.

	Se dice que $f$ es \strong{no ramificado} (resp. \strong{étale}) si lo es en cada punto.
\end{mydef}
\begin{ex}
	Sea $L/k$ una extensión finita de cuerpos.
	El morfismo $\Spec L \to \Spec k$ es no ramificado (o equivalentemente, étale) syss la extensión $L/k$ es separable.
\end{ex}

\begin{lem}
	Sea $f \colon X \to Y$ un morfismo de tipo finito entre esquemas localmente noetherianos.
	Entonces $f$ es no ramificado syss es cuasifinito, reducido y para cada $y \in Y$ y $x \in X_y$ se cumple que la extensión $\kk(x) / \kk(y)$ es separable.
\end{lem}
\begin{prop}\label{thm:unram_et_prop}
	Se cumplen:
	\begin{enumerate}
		\item Los encajes cerrados entre esquemas localmente noetherianos son no ramificados.
		\item Los encajes abiertos entre esquemas localmente noetherianos son étale.
		\item Los morfismos no ramificados (resp. étale) son estables salvo composición.
		\item Los morfismos no ramificados (resp. étale) son estables salvo cambio de base.
		\item Los morfismos no ramificados (resp. étale) son estables salvo productos fibrados.
		\item Sean $f \colon X \to Y, g \colon Y \to Z$ un par de morfismos tales que $f\circ g$ es no ramificado (resp. étale)
			y $g$ es separado. Entonces $f$ es no ramificado (resp. étale).
	\end{enumerate}
\end{prop}

\begin{prop}
	Sea $f \colon X \to Y$ un morfismo étale.
	Sean $x \in X$ e $y := f(x)$, entonces se cumplen:
	\begin{enumerate}
		\item $\kdim(\mathscr{O}_{X, x}) = \kdim(\mathscr{O}_{Y, y})$.
		\item El morfismo tangente $T_{f, x} \colon T_{X, x} \to T_{Y, y} \otimes_{\kk(x)} \kk(x)$ es un isomorfismo.
	\end{enumerate}
\end{prop}
\begin{cor}
	Sea $Y$ un esquema localmente noetheriano, y sea $f \colon X \to Y$ un morfismo de tipo finito que es étale en $x \in X$.
	Entonces $X$ es regular en $x$ syss $Y$ es regular en $f(x)$.
\end{cor}

\begin{prop}
	Sea $Y$ un esquema localmente noetheriano, $f \colon X \to Y$ un morfismo de tipo finito y $x \in X_y$ un punto tal que $\kk(x) = \kk(y)$.
	Sea $\widehat{\mathscr{O}}_{Y, y} \to \widehat{\mathscr{O}}_{X, x}$ el homomorfismo entre las compleciones formales.
	Entonces $f$ es étale en $x$ syss $\widehat{\mathscr{O}}_{Y, y} \to \widehat{\mathscr{O}}_{X, x}$ es un isomorfismo.
\end{prop}

\subsection{Morfismos suaves}
\begin{mydef}
	Sea $k$ un cuerpo y $X$ un esquema algebraico sobre $k$.
	Se dice que $X$ es \strong{suave}\index{esquema!suave}\index{suave!(esquema)} en un punto $x \in X$ si $X_{\algcl k}$ es regular
	en todos los puntos de la fibra de $x$.
	Decimos que $X$ es \strong{suave} (a secas) si es suave en todos sus puntos (si es \textit{geométricamente regular}).
\end{mydef}
\begin{ex}
	Sea $k$ un cuerpo.
	Entonces el espacio afín $\A^n_k$ y el espacio proyectivo $\PP^n_k$ son variedades suaves.
\end{ex}

\begin{exn}
	Sea $k$ un cuerpo de $\car k \ne 2$.
	Y sea $C := \Spec( k[x, y] / (y^2 - f(x)) )$, donde $f(x) \in k[x]$ es un polinomio no constante.
	Entonces, aplicando el criterio del jacobiano, uno puede notar que $C$ es suave (syss $C_{\algcl k}$ es regular) syss $f(x)$ no posee raíces
	repetidas en $\algcl k$; lo que equivale a que su discriminante sea no nulo.
\end{exn}
El ejemplo anterior es importante pues reaparecerá en el estudio de curvas elípticas.

\begin{prop}
	Sea $k$ un cuerpo, $X$ un esquema algebraico sobre $k$ y $x \in X$ un punto cerrado. Se cumplen:
	\begin{enumerate}
		\item Si $X$ es suave en $x$, entonces $X$ es regular en $x$.
		\item Si $\kk(x)/k$ es separable (e.g., si $k$ es perfecto) y $X$ es regular en $x$, entonces $X$ es suave en $x$.
	\end{enumerate}
\end{prop}
\begin{proof}
	\begin{enumerate}
		\item Sea $x' \in X_{\algcl k}$ en la fibra de $x$, de modo que es regular.
			Por el lema~\ref{lem:regular_pts_speto_closed_reg} tenemos que $x'$ se especializa en un punto $z' \in X_{\algcl k}$ cerrado y regular;
			por tanto, sea $z \in X$ la imagen de $z'$.
			Entonces $z$ es regular, por la proposición anterior, y es claramente especializa a $x$, por lo que $x$ es regular
			(por la proposición~\ref{thm:regularity_under_gez}).
			\qedhere
	\end{enumerate}
\end{proof}
\begin{cor}\label{thm:geometrically_red_is_generically_smooth}
	Sea $X$ un esquema algebraico sobre un cuerpo $k$. Se cumplen:
	\begin{enumerate}
		\item El conjunto de puntos suaves de $X$, denominado \strong{locus suave}\index{locus!suave}, es abierto.
		\item Si $X$ es geométricamente reducido o si $X$ es irreducible y es reducido en el punto genérico,
			entonces el locus suave es denso.
	\end{enumerate}
\end{cor}
\begin{proof}
	Basta probar la primera, la cual se deduce de leer la proposición anterior como que $X$ es suave en el punto $x$ syss
	$X$ es regular y geométricamente reducido en $x$, por lo que el locus suave es la intersección del locus regular
	con el locus geométricamente reducido, donde ambos son abiertos.
\end{proof}

\begin{mydef}
	Sea $Y$ un esquema localmente noetheriano y sea $f \colon X \to Y$ un morfismo de tipo finito.
	Se dice que $f$ es \strong{suave}\index{suave!(morfismo)} en un punto $x \in X$ si es plano en $x$ y si la fibra $X_y$ con $y = f(x)$ es un
	$\kk(y)$-esquema suave en $x$.
	Se dice que $f$ es un \strong{morfismo suave de dimensión relativa $n$}\index{morfismo!suave} si es suave en todo punto de $X$ y
	si todas las fibras son de dimensión pura $n$.
	El conjunto de puntos donde $f$ es suave, se le llama el \strong{locus suave}\index{locus!suave} de $f$.
\end{mydef}

\section{Haz de diferenciales de Kähler}
\begin{mydef}
	Sea $A$ un anillo, $B$ una $A$-álgebra conmutativa y $M$ un $B$-módulo.
	Una \strong{$A$-derivación}\index{Aderivación@$A$-derivación} es un homomorfismo de $A$-módulos $D \colon B \to M$ tal que para todo $u, v\in B$
	se satisface la \strong{regla de Leibniz}\index{regla!de Leibniz}:
	$$ D(u\cdot v) = vD(u) + uD(v). $$
	El conjunto de $A$-derivaciones $D \colon B \to M$ se denota $\Der_A(B, M)$ y es un $A$-módulo.
\end{mydef}

Es claro que $\Der_A(B, -) \colon \mathsf{Mod}_B \to \mathsf{Mod}_A$ determina un funtor (si el lector lo prefiere, puede suponer que el codominio es
$\mathsf{Set}$) y es un resultado de álgebra conmutativa su representabilidad:
\begin{mydef}
	Sea $B/A$ una álgebra conmutativa.
	Se define el \strong{módulo de diferenciales de Kähler}\index{modulo@módulo!de diferenciales de Kähler} como el $B$-módulo
	$\Omega_{B/A}^1$ junto con una $A$-derivación $\ud \colon B \to \Omega_{B/A}^1$ tal que para toda $A$-derivación $D \colon B \to M$
	\nomenclature{$\Omega_{B/A}$}{Módulo de diferenciales de Kähler}
	existe un único homomorfismo de $B$-módulos $\overline{D} \colon \Omega_{B/A}^1 \to M$ tal que el siguiente diagrama conmuta:
	\begin{center}
		\begin{tikzcd}
			B \rar["\ud"] \drar["D"'] & \Omega_{B/A}^1 \dar[dashed, "\exists!\overline{D}"] \\
			{}                        & M
		\end{tikzcd}
	\end{center}
\end{mydef}

Claramente una manera de construirlo es construir el $B$-módulo libre $B^{\oplus B}$ y cocientar por el submódulo
generado por la restricción de la regla de Leibniz.

Las propiedades generales del módulo $\Omega_{B/A}$ están contenidas en el apéndice \S\ref{sec:commalg_kahler_diff}.
\begin{prop}
	Sea $f \colon X \to Y$ un morfismo de esquemas.
	Existe un único haz cuasicoherente $\mathscr{O}_{X/Y}^1$ sobre $X$ tal que para todo punto $x \in X$,
	y para todo abierto afín $U \subseteq f^{-1}[V]$ contenido en la preimagen de un abierto afín $V \subseteq Y$ tenemos que
	$$ \Omega_{X/Y}^1 |_U \simeq \widetilde{ \Omega^1_{ {\mathscr{O}}_X(U) / {\mathscr{O}}_Y(V) } }, \qquad
	( \Omega_{X/Y}^1 )_x \cong \Omega_{\mathscr{O}_{X, x} / \mathscr{O}_{Y, f(x)}}^1. $$
	Éste haz se llama el \strong{haz de diferenciales relativos de grado 1}\index{haz!de diferenciales relativos} sobre $X/Y$.
	Si $Y = \Spec A$ se denota $\Omega_{X/A}^1$ y obviaremos el subíndice <<$Y$>> de no haber ambigüedad.
\end{prop}
\begin{proof}
	Sea $\varphi \colon A \to B$ un homomorfismo de anillos, sea $\mathfrak{q} \in \Spec B$ un primo y sea $\mathfrak{p} := \varphi^{-1}[\mathfrak{q}] \in
	\Spec A$, entonces la proposición~\ref{app:Omega_base_change} nos da
	$$ \Omega_{B/A}^1 \otimes_B B_{\mathfrak{q}} \cong \Omega_{B_{\mathfrak{q}}/A}^1 \cong \Omega_{B_{\mathfrak{q}} / A_{\mathfrak{p}}}^1. $$
	Por simplicidad, denotemos $\Omega_x^1 := \Omega_{\mathscr{O}_{X, x} / \mathscr{O}_{Y, f(x)}}^1$ para $x \in X$.
	Dado un abierto afín $V \subseteq Y$ y un abierto afín $U \subseteq f^{-1}[V]$, sea $\omega \in \Omega_{\mathscr{O}_X(U) / \mathscr{O}_Y(V)}^1$ y
	sea $x \in U$.
	Denotemos por $\omega|_x$ la imagen de $\omega$ en $\Omega_{\mathscr{O}_X(U) / \mathscr{O}_Y(V)}^1 \otimes \mathscr{O}_{X, x} \cong \Omega_x^1$.

	Ahora, finalmente definimos $\Omega_{X/Y}^1(U)$ como las funciones $s \colon U \to \coprod_{x\in U} \Omega_x^1$ tales que para todo $x \in U$
	existe un entorno afín $V_x \subseteq Y$ de $f(x)$, un entorno afín $x \in U_x \subseteq f^{-1}[V_x] \cap U$ y un elemento $\omega
	\in \Omega_{\mathscr{O}_X(U) / \mathscr{O}_Y(V)}^1$ tal que
	$$ \forall z \in U_x \quad s(z) = \omega|_z. $$
	Con las restricciones triviales es fácil ver que $\Omega_{X/Y}^1$ es un $\mathscr{O}_X$-módulo y también es fácil verificar que
	$( \Omega_{X/Y}^1 )_x \cong \Omega_x^1$.
	Sea $U \subseteq f^{-1}[V]$ un abierto afín con $V \subseteq Y$ abierto afín, entonces tenemos un homomorfismo natural de $\mathscr{O}_X(U)$-módulos
	$\Omega_{\mathscr{O}_X(U) / \mathscr{O}_Y(V)}^1 \to \Omega_{X/Y}^1(U)$ que induce un morfismo de $\mathscr{O}_X|_U$-módulos:
	\[
		\widetilde{ \Omega^1_{ {\mathscr{O}}_X(U) / {\mathscr{O}}_Y(V) } } \longrightarrow \Omega_{X/Y}^1|_U,
	\]
	el cual es un isomorfismo pues lo es en las fibras.
	Así pues $\Omega_{X/Y}^1$ satisface todo lo exigido.
\end{proof}
% \warn
% Hay que tener ojo con la expresión <<un único haz tal que...>>, en realidad nos referimos a las restricciones (que son aplicar $(-)_*$ sobre los morfismos
% de inclusión) dan canónicamente el haz 

\begin{cor}
	Si $f \colon X \to Y$ es un morfismo de tipo finito entre esquemas noetherianos,
	entonces $\Omega_{X/Y}^1$ es un haz coherente sobre $X$.
\end{cor}
\begin{proof}
	Basta aplicar el corolario~\ref{app:kahler_diff_finite_type}.
\end{proof}

Las propiedades de \S\ref{sec:commalg_kahler_diff} ahora se reescriben como:
\begin{thm}\label{thm:rel_diff_props}
	Sea $f \colon X \to Y$ un morfismo de esquemas. Entonces:
	\begin{enumerate}
		\item (Cambio de base) Para todo $Y$-esquema $W$ se cumple que
			$$ \Omega_{X_W/Y_W}^1 = \Omega_{X_W/W}^1 \simeq p^*\Omega_{X/Y}^1, $$
			donde $p \colon X_W \to X$ es la proyección canónica.
		\item\label{thm:rel_diff_props_exact}
			Para todo morfismo de esquemas $Y \to Z$ se tiene la siguiente sucesión exacta:
			\begin{center}
				\begin{tikzcd}
					f^*\Omega_{Y/Z}^1 \rar & \Omega_{X/Z}^1 \rar & \Omega_{X/Y}^1 \rar & 0
				\end{tikzcd}
			\end{center}
		\item Para todo abierto $U \subseteq X$ tenemos $\Omega_{X/Y}^1|_U \simeq \Omega_{U/Y}^1$.
			Para todo punto $x \in X$ tenemos $(\Omega_{X/Y}^1)_x \cong \Omega_{\mathscr{O}_{X, x}/\mathscr{O}_{Y, f(x)}}^1$.
		\item Si 
			\begin{tikzcd}[cramped, sep=small]
				Z \rar[closed] & X
			\end{tikzcd}
			es un subesquema cerrado dado por $Z = \VV(\mathscr{I})$, entonces tenemos la sucesión exacta:
			\begin{center}
				\begin{tikzcd}
					\mathscr{I/I}^2 \rar & \Omega_{X/Y}^1 \otimes_{\mathscr{O}_X} \mathscr{O}_Z \rar & \Omega_{Z/Y}^1 \rar & 0
				\end{tikzcd}
			\end{center}
	\end{enumerate}
\end{thm}

Veamos unos ejemplos sobre el cómo calcular el haz de diferenciales relativos:
\begin{exn}\label{ex:affine_space_diff}
	Sea $Y$ un esquema y $X := \A_Y^n$.
	Si $Y = \Spec A$ fuese afín, entonces por el hecho de que $\Omega_{A[x_1, \dots, x_n]/A}^1 = A[x_1, \dots, x_n]^n$
	(proposición~\ref{app:Omega_calc_example}) implica que, sobre abiertos afines $U$, tenemos $\Omega_{X/Y}^1|_U \simeq \mathscr{O}_X^n|_U$.
	Estos isomorfismos son claramente compatibles entre sí de lo que se sigue que $\Omega_{X/Y}^1 \simeq \mathscr{O}_X^n$.
\end{exn}

\begin{thm}\label{thm:kahler_diff_proy_sp}
	Sea $Y := \Spec A$ un esquema afín y sea $X := \Proj(A[x_0, \dots, \break x_n]) = \PP_A^n.$
	Entonces existe una sucesión exacta de $\mathscr{O}_X$-módulos:
	\begin{center}
		\begin{tikzcd}[sep=large]
			0 \rar & \Omega_{X/Y}^1 \rar & \mathscr{O}_X(-1)^{\oplus(n+1)} \rar & \mathscr{O}_X \rar & 0.
		\end{tikzcd}
	\end{center}
\end{thm}
\begin{proof}
	Sea $B := A[\vec x]$, la cual es una $A$-álgebra graduada y sea $M := B(-1)^{n+1}$ el $B$-módulo graduado, donde $B(-1)$ denota un torcimiento
	(en los grados).
	Sean $\vec e_0, \dots, \vec e_n$ la base en grado 1 y sea $\varphi \colon M \to B$ el homomorfismo de $B$-módulos dado por $\vec e_i \mapsto x_i$.
	Entonces tenemos una sucesión exacta 
	\begin{tikzcd}[cramped, sep=small]
		0 \rar & K \rar & M \rar & B
	\end{tikzcd}
	de $B$-módulos graduados que induce una sucesión exacta
	\begin{center}
		\begin{tikzcd}[sep=large]
			0 \rar & \widetilde{K} \rar & \mathscr{O}_X(-1)^{\oplus(n+1)} \rar[red] & \mathscr{O}_X \rar & 0
		\end{tikzcd}
	\end{center}
	donde la flecha roja es un epimorfismo pues lo es en grados suficientemente grandes.

	Veremos que $\widetilde{K} \simeq \Omega_{X/Y}^1$.
	Localizando por $x_i$, vemos que $M[x_i^{-1}] \to B[x_i^{-1}]$ es un epimorfismo de $B[x_i^{-1}]$-módulos libres, de modo que $K[x_i^{-1}]$
	es un módulo libre de rango $n$ generado por $\{ \vec e_j - \frac{x_j}{x_i}\vec e_i \}_{i\ne j}$.
	Tomando $U_i := \DD_+(x_i)$, lo anterior se traduce en que $\widetilde{K}|_{U_i}$ es un $\mathscr{O}_X|_{U_i}$-módulo
	libre generado por secciones globales
	\[
		\left\{ \frac{1}{x_i}\vec e_j - \frac{x_j}{x_i^2}\vec e_i \right\}_{i\ne j},
	\]
	(donde el factor $1/x_i$ es para que los elementos tengan grado 0, recordando el torcimiento.)

	Recuérdese que $U_i \cong \Spec(A[x_0/x_i, x_1/x_i, \dots, x_n/x_i])$, así que definamos $\varphi_i \colon \Omega_{X/Y}^1|_{U_i} \to \widetilde{K}|_{U_i}$
	mediante la regla
	$$ (\varphi_i)_{U_i}( \ud(x_i/x_j) ) := \frac{1}{x_i^2}( x_i\vec e_j - x_j\vec e_i ), $$
	(por el ejemplo anterior, $\Omega_{\A^n_A/A}^1$ es libre y generado por los símbolos $\ud(x_i/x_j)$.)
	Así cada $\varphi_i$ es un isomorfismo y son compatibles, pues en $U_i \cap U_j = \DD_+(x_ix_j)$
	para cada $\ell$ tenemos que $x_\ell/x_i = (x_\ell/x_j) \cdot (x_j/x_i)$ y en $\Omega_{X/Y}^1|_{U_i \cap U_j}$ se tiene
	$$ \varphi_i\big( \ud(x_\ell/x_i) \big) = \ud\left( \frac{x_\ell}{x_i} \right) - \frac{x_\ell}{x_j} \ud\left( \frac{x_j}{x_i} \right)
	= \frac{x_j}{x_i} \ud\left( \frac{x_\ell}{x_i} \right) = \varphi_j\big( \ud(x_\ell/x_i) \big), $$
	lo cual verifica que son compatibles.
\end{proof}

\begin{ex}
	Sea $A$ un anillo, $B := A[x_1, \dots, x_n]$, sea $F \in B$ y $C := B/(F)$.
	Definiendo $X = \Spec B$, vemos que 
	\begin{tikzcd}[cramped, sep=small]
		Z = \Spec C = \VV(F) \rar[closed] & X.
	\end{tikzcd}
	Luego, empleando el ejemplo~\ref{ex:affine_space_diff} y el inciso 4 de la proposición~\ref{thm:rel_diff_props} nos dan que
	$$ \Omega_{C/A}^1 \cong \frac{ \bigoplus_{i=1}^n C \, \ud x_i }{C \, \ud F}, $$
	donde $\ud F = \sum_{i=1}^{n} (\partial F/\partial x_i) \, \ud x_i$.
\end{ex}
\begin{exn}\label{exn:diff_over_smooth_affine_curve}
	Sea $k$ un cuerpo y sea $X := \Spec( k[T, S]/f(T, S) ) = \VV(f) \subseteq \A^2_k$ una curva afín suave.
	Denotemos por $t, s$ las imágenes de $T, S$ en $\Gamma(X, \mathscr{O}_X)$ y denotemos por $\partial_t F$ la imagen de $\partial F/\partial T \in k[T, S]$
	en $\Gamma(X, \mathscr{O}_X)$ y análogamente con $s$.
	Por el ejemplo anterior, sobre el abierto principal $\DD( \partial_s F )$, el ejemplo anterior nos da que
	$\Gamma\big(\DD( \partial_s F ), \Omega_{X/k}^1\big)$ es libre generado por $\ud t/\partial_s F$ y en el otro abierto principal
	$\Gamma\big(\DD( \partial_t F ), \Omega_{X/k}^1\big)$ es libre generado por $\ud s/\partial_t F$.

	Como $\partial_t F \, \ud t = - \partial_s F \, \ud s$ y como $\DD( \partial_s F ) \cup \DD( \partial_t F ) = X$ por el criterio del jacobiano,
	vemos que $\Gamma(X, \Omega_{X/k}^1)$ es libre generado por $\ud t/\partial_s F$.
\end{exn}

\begin{mydefi}
	Sea $k$ un cuerpo.
	Una \strong{curva elíptica}\index{curva!elíptica} sobre $k$ es una curva suave proyectiva $E$ sobre $k$ que es isomorfa a una subvariedad
	cerrada de $\PP^2_k$ dada por una \strong{ecuación de Weierstrass (larga)}\index{ecuación!de Weierstrass!larga}:
	\begin{equation}
		v^2w + a_1uvw + a_3vw^2 = u^3 + a_2u^2w + a_4uw^2 + a_6w^3,
		\label{eq:long_wei_form}
	\end{equation}
	con el punto distinguido $o = [0 : 1 : 0]$.
\end{mydefi}
La ecuación de Weierstrass usual sale de cortar con el abierto afín $w = 1$;
uno puede memorizar los índices asignandole los <<pesos>> $u \to 2$ e $v \to 3$, en consecuencia, no hay $a_5$.
\begin{prop}
	Sea $E$ una curva elíptica sobre un cuerpo $k$ dada por una ecuación de Weierstrass \eqref{eq:long_wei_form}.
	Sean $x := u/w, y := v/w \in K(E)$ y
	$$ \omega := \frac{\ud x}{2y + a_1x + a_3} \in \Omega_{K(E)/k}^1. $$
	Entonces $\Omega_{E/k}^1 = \omega \mathscr{O}_E$.
\end{prop}
\begin{proof}
	Nótese que $E \subseteq \PP^2_k$ es la unión de los abiertos principales $U := \DD_+(w) \cap E, V := \DD_+(v) \cap E$ cuyos anillo de coordenadas son
	\begin{align*}
		\Gamma(U, \mathscr{O}_E) &= \frac{k[x, y]}{( y^2 + (a_1x + a_3)y - (x^3 + a_2x^2 + a_4x + a_6) )}, \\
		\Gamma(V, \mathscr{O}_E) &= \frac{k[t, z]}{( z + a_1tz + a_3z^2 - (t^3 + a_2t^2z + a_4tz^2 + a_6z^3) )},
	\end{align*}
	donde $t := u/v = x/y$ y $z := w/v = 1/y$.

	Por el ejemplo~\ref{exn:diff_over_smooth_affine_curve} vemos que $\Omega_{U/k}^1$ es libre y generado por $\omega$,
	mientras que $\Omega_{V/k}^1$ es libre y generado por
	$$ \omega' := \frac{\ud z}{a_1z - (3t^2 + 2a_2tz + a_4z^2)}. $$
	Finalmente, en $\Omega_{K(E)/k}^1$ tenemos que $\ud z = -\ud y/y^2$, por lo que es fácil comprobar que
	\begin{equation}
		\omega' = \frac{-\ud y}{a_1y - (3x^2 + 2a_2x + a_4)} = \omega.
		\tqedhere
	\end{equation}
\end{proof}

\begin{lem}
	Sea $k$ un cuerpo y $A$ una $k$-álgebra de tipo finito.
	Sea $x \in (\Spec A)(k)$ un punto $k$-racional y sea $\mathfrak{m} := \mathfrak{p}_x \in \Spec A$.
	Entonces el homomorfismo de la segunda sucesión fundamental
	\begin{center}
		\begin{tikzcd}[sep=large]
			\delta \colon \mathfrak{m/m}^2 \rar["\sim"] & \Omega_{A/k}^1 \otimes_A \kk(x)
		\end{tikzcd}
	\end{center}
	es un isomorfismo.
\end{lem}
\begin{proof}
	Por la segunda sucesión fundamental $\coker\delta = 0$, así que basta ver que $\ker\delta = 0$.
	Sea $\pi \colon B := k[\vec t] \epicto A$ un $k$-homomorfismo suprayectivo, y sea $\mathfrak{n} := \pi^{-1}[\mathfrak{m}] \in \Spec B$.
	Entonces, se tiene el siguiente diagrama con filas exactas:
	\begin{center}
		\begin{tikzcd}
			\mathfrak{a} \dar[equals] \rar[dotted] & \mathfrak{n/n}^2 \dar["\delta'"] \rar & \mathfrak{m/m}^2 \dar["\delta"] \rar & 0 \\
			\mathfrak{a} \rar["\gamma"] & \Omega_{B/k}^1 \otimes \kk(x) \rar[dotted] & \Omega_{A/k}^1 \otimes \kk(x) \rar[dotted] & 0
		\end{tikzcd}
	\end{center}
	donde $\gamma$ es la composición $\mathfrak{a \to a/a}^2 \to \Omega_{B/k}^1 \otimes_B A$ tensorizado por $\kk(x)$.
	Una aplicación del lema de la serpiente da que $\ker\delta = \ker(\delta')$, el cual es cero.
\end{proof}

\begin{prop}
	Sea $X$ una variedad algebraica sobre un cuerpo $k$ y sea $x \in X$.
	Son equivalentes:
	\begin{enumerate}
		\item $\Omega_{X, x}^1$ es libre de rango $\dim_x X$.
		\item $X$ es suave en un entorno de $x$.
		\item $X$ es suave en $x$.
	\end{enumerate}
\end{prop}
\begin{proof}
	$1 \implies 2$.
	Como $X$ es localmente noetheriano, si $\Omega_{X, x}^1$ es libre de rango $n := \dim_x X$ en $x$,
	entonces es libre de rango $n$ en un entorno $U$ de $x$ (prop.~\ref{thm:local_freeness_on_noeth}).
	Podemos suponer que $U$ es conexo y que tiene $\dim U = n$.
	Sea $V := U_{\algcl k}$, el cual tiene $\dim V = n$ (prop.~\ref{thm:dimension_on_alg_base_change}).
	Aplicando cambio de base, vemos que $\Omega_{V/\algcl k}^1$ también es libre de rango $n$ y, por el lema anterior,
	para todo punto cerrado $y \in \clpt V$ tenemos que
	$$ \dim_{\kk(y)}( T_{V, y} ) = \dim_{\kk(y)}\big( \Omega_{V, y}^1 \otimes \kk(y) \big) = n, $$
	de modo que $V$ es suave en todos los puntos $y$ tales que $\dim_y V = n$.

	Veamos $U$ es irreducible. En efecto, de lo contrario habrían dos componentes irreducibles que se cortan en un punto cerrado $x_0 \in \clpt U$
	de $\dim_{x_0}(U)$.
	Pero hemos visto que tal punto sería suave, luego íntegro.
	Por lo tanto, $U$ es irreducible y $V$ tiene dimensión $n$ en todos los puntos cerrados, de modo que $U$ es suave.

	$2 \implies 3$. Trivial.

	$3 \implies 1$. Sea $x' \in X_{\algcl k}$ en la fibra de $x$.
	...
\end{proof}
La proposición anterior da una demostración (muchísimo más sencilla) de que el conjunto de puntos regulares es abierto.

\begin{cor}\label{thm:unram_omega_criteria}
	Sea $f \colon X \to S$ un morfismo de tipo finito entre esquemas localmente noetherianos.
	Entonces $f$ es no ramificado sobre un punto $x \in X$ syss $\Omega_{X/S, x}^1 = 0$.
\end{cor}

\begin{lem}
	Sea $X \to S$ un morfismo de tipo finito entre esquemas localmente noetherianos.
	Sean $s \in S, x \in X_s$ y defínase
	\[
		d := \dim_{\kk(x)}\big( \Omega_{X_s/\kk(s), x}^1 \otimes_{\mathscr{O}_{X_s, x}} \kk(x) \big).
	\]
	Entonces en un entorno de $x$ existe un encaje cerrado 
	\begin{tikzcd}[cramped, sep=small]
		X \rar[closed] & Z
	\end{tikzcd}
	donde $Z$ es un $S$-esquema suave en $x$ tal que $\dim_x(Z_s) = d$ y tal que $ \Omega_{Z/S, x}^1 $ es un $\mathscr{O}_{Z, x}$-módulo libre
	de rango $d$.
\end{lem}

\begin{prop}
	Sea $S$ un esquema localmente noetheriano y $f\colon X \to S$ un morfismo de tipo finito. Se cumplen:
	\begin{enumerate}
		\item Si $f$ es suave en $x$, entonces existe un entorno $x \in U$ tal que $\Omega_{X/S}^1|_U$ es libre de rango $\dim_x(X_s)$,
			donde $s := f(x)$.
		\item Si $f$ es plano y las fibras son de dimensión pura $n$.
			Entonces $f$ es suave syss $\Omega_{X/S}^1$ es localmente libre de rango $n$.
	\end{enumerate}
\end{prop}

\begin{cor}\label{thm:first_fund_seq_smooth}
	Sea $S$ un esquema localmente noetheriano y sea $f\colon X \to Y$ un morfismo entre $S$-esquemas suaves.
	Entonces se tiene la siguiente sucesión exacta:
	\begin{center}
		\begin{tikzcd}[sep=large]
			0 \rar & f^* \Omega_{Y/S}^1 \rar["\alpha"] & \Omega_{X/S}^1 \rar & \Omega_{X/Y}^1 \rar & 0
		\end{tikzcd}
	\end{center}
\end{cor}
\begin{hint}
	Falta verificar la inyectividad de la flecha $\alpha$, lo cual se verifica en fibras empleando que todos son haces localmente libres.
\end{hint}

\begin{thm}
	Sea $S$ un esquema localmente noetheriano, regular y conexo, $X$ un esquema irreducible y $f \colon X \to S$ un morfismo de tipo finito.
	Sea $x \in X$ un punto tal que para $s := f(x)$ tenemos
	\begin{equation}
		\kdim\mathscr{O}_{X, x} = \kdim\mathscr{O}_{X_s, x} + \kdim\mathscr{O}_{S, s}.
		\label{eqn:smooth_dimension_condition}
	\end{equation}
	Entonces $f$ es suave en $x$ syss $\Omega_{X/S, x}^1$ es un $\mathscr{O}_{X, x}$-módulo libre de rango $\dim_x(X_s)$.
\end{thm}

\begin{cor}\label{thm:vars_sm=gen_sm}
	Sea $X$ un esquema algebraico sobre un cuerpo $k$.
	Entonces $X$ es suave syss $\Omega_{X/k}^1$ es localmente libre y para todo punto genérico $\xi \in X$ se satisface que
	la extensión $\kk(\xi)/k$ es separable.
	En particular, si $k$ es perfecto, entonces $X$ es suave syss $\Omega_{X/k}^1$ es localmente libre.
\end{cor}
% \todo{Pensar demostración, ver \citeauthor{liu:algebraic}~\cite{liu:algebraic}, ex.~6.2.2.}
\begin{proof}
	% Si el esquema $X$ es suave, entonces debe serlo en al menos el punto genérico $\xi \in X$.
	Por evitamiento de primos podemos restringirnos a un abierto irreducible y así suponer que $X$ es irreducible con punto genérico $\xi$.
	Nótese que $X$ siempre posee al menos un punto regular, luego sus generizaciones son regulares y, en particular, $\xi$ es regular en $X$, de modo que
	la extensión $\kk(\xi)/k$ es separable syss $X$ es suave en $\xi$.
	Como $\Omega_{X/k}^1$ es localmente libre, tenemos que $\Omega_{X/k, \xi}^1 = \Omega_{\kk(x)/k}^1 \cong \kk(x)^n$ y, como es suave,
	de hecho $n = \trdeg_k\big( \kk(x) \big) = \dim X$.
	Así concluimos que $\Omega_{X/k}^1$ es localmente libre de rango $\dim X$ y, por el teorema anterior, estamos listos.
\end{proof}

\begin{prop}
	Sea $S$ un esquema localmente noetheriano y sea $f \colon X \to Y$ un morfismo entre $S$-esquemas de tipo finito.
	\begin{enumerate}
		\item Si $f$ es étale en un punto $x \in X$, entonces el homomorfismo canónico
			\begin{equation}
				\varphi \colon (f^* \Omega_{Y/S})_x \longrightarrow ( \Omega_{X/S}^1 )_x
				\label{eqn:relative_diff_hom}
			\end{equation}
			es un isomorfismo.
		\item Si $X$ (resp. $Y$) es suave sobre $S$ en $x$ (resp. $f(x)$) y el homomorfismo canónico \eqref{eqn:relative_diff_hom} es un isomorfismo,
			entonces $f$ es étale en $x$.
	\end{enumerate}
\end{prop}
\begin{cor}\label{thm:smooth_morph_decomp}
	Sea $S$ un esquema localmente noetheriano y $f \colon X \to S$ un morfismo suave en $x \in X$.
	Existe un entorno $U$ de $x$ tal que el siguiente diagrama conmuta:
	\begin{center}
		\begin{tikzcd}[row sep=large]
			U \dar[open] \rar["g"] & \A_S^n \dar \\
			X            \rar["f"] & S
		\end{tikzcd}
	\end{center}
	donde $g$ es étale en $x$.
\end{cor}
\begin{proof}
	Restringiéndose a un abierto de $S$ (y recordando que la suavidad es local), podemos suponer que $S = \Spec A$ es afín.
	Sea $\ud f_1, \dots, \ud f_n \in \Omega_{X, x}^1$ una $\mathscr{O}_{X, x}$-base; restringiéndose a un abierto $U$ de $X$
	podemos suponer que cada $f_i$ es regular en $U$; por lo que determinan un $A$-homomorfismo:
	$$ \ev_{f_1, \dots, f_n}\colon A[t_1, \dots, t_n] \longrightarrow \Gamma(U, \mathscr{O}_X) $$
	que manda cada $t_j \mapsto f_j$.
	Por adjunción esto determina un $S$-morfismo $g \colon U \to \A^n_S =: Y$, veamos que es étale en $x$.
	Nótese que induce el homomorfismo sobre haces
	$$ g^* \Omega_{Y/S}^1 \longrightarrow \Omega_{X/S}^1 $$
	que manda $\ud t_j \mapsto \ud f_j$, de modo que en la fibra de $x$ determina un epimorfismo $(g^* \Omega_{Y/S}^1)_x \to (\Omega_{X/S}^1)_x$.
	Finalmente, como ambos son localmente libres, en realidad determina un isomorfismo y concluimos por la proposición anterior.
\end{proof}

\begin{mydef}
	Sea $X$ un esquema.
	Un \strong{engrosamiento}\index{engrosamiento}\footnotemark{} de $X$ es un subesquema cerrado 
	\begin{tikzcd}[cramped, sep=small]
		Z \rar[closed] & X
	\end{tikzcd}
	tal que comparten el mismo espacio topológico.
	En otras palabras, es un subesquema cerrado $Z = \VV(\mathscr{I})$, donde $\mathscr{I \subseteq O}_X$ es un haz de ideales
	tal que para todo punto $x \in X$, el ideal $\mathscr{I}_x \nsle \mathscr{O}_{X, x}$ es nilpotente.
	Se dice que el engrosamiento $Z = \VV(\mathscr{I})$ es de \strong{orden finito}\index{engrosamiento!de orden finito}
	(resp.\ \emph{de primer orden}) si existe $n \in \N$ tal que $\mathscr{I}^n = 0$ (resp.\ si $\mathscr{I}^2 = 0$).
\end{mydef}
\footnotetext{Eng.\ \textit{thickening}. El término es original de \cite{stacks}, \href{https://stacks.math.columbia.edu/tag/04EX}{\urlstyle Tag 04EX}.}

\warn
% Ojo que un engrosamiento no es lo mismo que un $\VV_X(\mathscr{I})$, donde $\mathscr{I \subseteq O}_X$ es nilpotente.
Ojo que no todo engrosamiento es de orden finito;
la razón está en que en cada punto la potencia que anule a la fibra $\mathscr{I}_x$ puede crecer.
Cuando $X$ es noetheriano, entonces el haz de ideales $\mathscr{I}$ sí es nilpotente y si hay equivalencia.

\begin{prop}
	Sea $S$ un esquema localmente noetheriano y $f \colon X \to S$ un morfismo suave (resp.\ étale, no ramificado).
	Para todo $S$-esquema $Y$ y todo $S$-engrosamiento 
	\begin{tikzcd}[cramped, sep=small]
		i\colon Y' \rar[closed] & Y,
	\end{tikzcd}
	la función canónica
	\begin{equation*}
		\Hom_S(Y, X) \longrightarrow \Hom_S(Y', X), \qquad f \mapsto i\circ f
	\end{equation*}
	es sobreyectiva (resp.\ biyectiva, inyectiva).
\end{prop}
% Esto nos diría, en lenguaje de \cite{stacks}, secciones \href{https://stacks.math.columbia.edu/tag/02H7}{\urlstyle 02H7},
% \href{https://stacks.math.columbia.edu/tag/02HF}{\urlstyle 02HF} y \href{https://stacks.math.columbia.edu/tag/02GZ}{\urlstyle 02GZ},
La proposición anterior nos dice, en lenguaje de \citeauthor{gortz:algebraic_ii}~\cite[36-40]{gortz:algebraic_ii},
que un morfismo suave (resp.\ étale, no ramificado) es lo mismo que
un morfismo \textit{formalmente} suave (resp.\ formalmente étale, formalmente no ramificado) de tipo finito.

\subsection{Intersección completa local}
Recuérdese que dado un $A$-módulo $M$ decimos que $a \in A$ es \strong{$M$-regular} si la $a$-torsión $M[a] = 0$.
Decimos que una sucesión $(a_1, \dots, a_n)$ es \strong{débilmente $M$-regular} si $a_1$ es $M$-regular y cada $a_r$
es $M/(a_1, \dots, a_{r-1})M$-regular.

\warn
Esta definición de álgebra conmutativa es dependiente del orden en general, pero si $(A, \mathfrak{m})$ es un anillo local noetheriano,
$M$ es finitamente generado y cada $a_i \in \mathfrak{m}$, entonces es independiente del orden.

\begin{mydef}
	Sea $Y$ un esquema localmente noetheriano y $f \colon X \to Y$ un encaje.
	Se dice que $f$ es un \strong{encaje regular}\index{encaje!regular} (resp. \strong{encaje regular de codimensión $n$})%
	\index{encaje!regular!de codimensión $n$} en un punto $x \in X$ si $\ker( \mathscr{O}_{Y, f(x)} \to \mathscr{O}_{X, x} )$ es un
	$\mathscr{O}_{Y, f(x)}$-módulo generado por una sucesión regular (resp. sucesión regular de $n$ elementos).
	Se dice que $f$ es un \strong{encaje regular} (resp. \strong{encaje regular de codimensión $r$}) si lo es en cada punto de $X$.
\end{mydef}

\begin{lem}\label{thm:a2quot_free}
	Sea $A$ un anillo y $\mathfrak{a} \nsl A$.
	Si $\mathfrak{a}$ está generado por una sucesión (débilmente) $\mathfrak{a}$-regular $a_1, \dots, a_n$,
	entonces $\mathfrak{a/a}^2$ es un $A/\mathfrak{a}$-módulo libre con base $a_1, \dots, a_n \mod{\mathfrak{a}}$.
\end{lem}
\begin{proof}
	del 
	Dados $u_1, \dots, u_n \in A$ tales que
	$$ \sum_{i=1}^{n} a_iu_i \in \mathfrak{a}^2 = \sum_{i=1}^{n} a_i\mathfrak{a}, $$
	existen $w_i \in \mathfrak{a}$ tales que $\sum_{i=1}^{n} a_i(u_i - w_i) = 0$
	Luego, por el lema~\ref{app:regular_base}, concluimos que $u_i \in \mathfrak{a}$, lo que prueba que los $a_i \mod{\mathfrak{a}}$
	conforman una base.
\end{proof}

\begin{mydef}
	% Sea $X$ una variedad suave sobre un cuerpo $k$.
	% Definimos el \strong{haz tangente}\index{haz!tangente} como $\mathscr{T}_X := \shHom( \Omega_{X/k}^1, \mathscr{O}_X ) = ( \Omega_{X/k}^1 )^\vee$.

	Sea $f \colon X \to Y$ un encaje.
	Entonces sea $V \subseteq Y$ un subesquema abierto tal que $f$ se factoriza por un encaje cerrado 
	\begin{tikzcd}[cramped, sep=small]
		i\colon X \rar[closed] & V
	\end{tikzcd}
	y sea $X = \VV_V(\mathscr{I})$.
	El haz $\mathscr{C}_{X/Y} := i^*(\mathscr{I/I}^2)$ se dice el \strong{haz conormal de $X$ en $Y$}\index{haz!conormal} y su dual
	es $\mathscr{N}_{X/Y} := (\mathscr{C}_{X/Y})^\vee$ el \strong{haz normal}\index{haz!normal}.
	% Como $\mathscr{T}_X$ es un haz localmente libre de rango $n := \dim X$, definimos el \strong{haz canónico}\index{haz!canónico} como
	% $ \omega_X := $
\end{mydef}
\begin{cor}\label{thm:conormal_rank}
	Sea $x \colon X \to Y$ un encaje regular (de codimensión $r$), entonces el haz conormal $\mathscr{C}_{X/Y}$ es localmente libre (de rango $r$).
\end{cor}
\begin{proof}
	Igual que en la definición sea $V \subseteq Y$ abierto tal que mediante $f$ tenemos que $X = \VV_V(\mathscr{I})$.
	Sea $x \in X$ un punto y sea $y := f(x)$.
	Entonces $f^*( \mathscr{I/I}^2 )_x = \mathscr{I}_y / \mathscr{I}_y^2$ y $\ker( \mathscr{O}_{Y, y} \epicto \mathscr{O}_{X, x} ) = \mathscr{I}_y$.
	El lema anterior ahora dice que $f^*( \mathscr{I/I}^2 )_x$ es libre (de rango $n$ si $f$ es de codimensión $n$) sobre $\mathscr{O}_{X, x}$.
\end{proof}

\begin{prop}\label{thm:regular_inm_prop}
	Sean $X, Y, W, Z$ un conjunto de esquemas localmente noetherianos. Se cumplen:
	\begin{enumerate}
		\item Sean $f \colon X \to Y, g \colon Y \to Z$ encajes regulares (de codimensión $n, m$ resp.).
			Entonces $f\circ g$ es un encaje regular (de codimensión $n+m$) y tenemos la sucesión exacta
			\begin{equation}
				\begin{tikzcd}
					0 \rar & f^*\mathscr{C}_{Y/Z} \rar & \mathscr{C}_{X/Z} \rar & \mathscr{C}_{X/Y} \rar & 0.
				\end{tikzcd}
				\label{eqn:conormal_sh_seq}
			\end{equation}

		\item Sea 
			\begin{tikzcd}[cramped, sep=small]
				i \colon X \rar[closed] & Y
			\end{tikzcd}
			un encaje cerrado y regular de codimensión $n$.
			Entonces para toda componente irreducible $Y'$ de $Y$ que corta a $X$ tenemos que $\codim(X \cap Y', Y') = n$.
			Más aún, $\kdim(\mathscr{O}_{X, x}) = \kdim(\mathscr{O}_{Y, x}) - n$.

		\item Sea $f \colon X \to Y$ un encaje regular.
			Para todo $Y$-esquema $W$ el homomorfismo canónico $p^* \mathscr{C}_{X/Y} \epicto \mathscr{C}_{X_W/W}$
			es un epimorfismo, donde $p \colon X \times_Y W \to X$ es la proyección.

		\item Sea $f \colon X \to Y$ un encaje regular (de codimensión $n$).
			Entonces para todo $Y$-esquema plano $W$ el cambio de base $f_W\colon X_W \to W$ es un encaje regular (de codimensión $n$)
			y el homomorfismo canónico 
			\begin{tikzcd}[cramped, sep=small]
				p^* \mathscr{C}_{X/Y} \rar["\sim"] & \mathscr{C}_{X_W/W}
			\end{tikzcd}
			es un isomorfismo.
	\end{enumerate}
\end{prop}
\begin{proof}
	\begin{enumerate}
		\item Lo único no trivial es probar la exactitud de \eqref{eqn:conormal_sh_seq}.
			Inmediatamente tenemos una sucesión exacta
			\begin{center}
				\begin{tikzcd}[sep=large]
					f^*\mathscr{C}_{Y/Z} \rar & \mathscr{C}_{X/Z} \rar["\alpha"] & \mathscr{C}_{X/Y} \rar & 0.
				\end{tikzcd}
			\end{center}
			Los haces en la sucesión son coherentes y localmente libres, de modo que $\ker\alpha$ es plano (se puede verificar en abiertos afines,
			donde se reduce a un problema de álgebra) y coherente, por tanto es localmente libre.
			Luego el homomorfismo de haces abelianos $f^*\mathscr{C}_{Y/Z} \epicto \ker\alpha$ es un epimorfismo y, las fibras de ambos
			tienen el mismo rango en las fibras, de modo que debe ser un isomorfismo (¿por qué?).

		\item Pasando a un abierto denso podemos suponer que $Y = \Spec A$ es afín con $A$ noetheriano local de punto cerrado $y$.
			Por hipótesis $\mathscr{I}_y$ tiene una base (como $\mathscr{O}_{Y, y}$-módulo) $a_1, \dots, a_n$ y, por una inducción
			sencilla, basta reducirnos al caso de $n = 1$ (i.e., $X = \VV(a)$).
			Sea $\mathfrak{p}$ el primo minimal de $A$ tal que $\overline{\{ x_{\mathfrak{p}} \}} = Y'$ y sea $b := a\mod{\mathfrak{p}}
			\in A/\mathfrak{p}$.
			Entonces $b\ne 0$ (¿por qué?) y $\dim(X \cap Y') = \dim(Y') - 1$ por el teorema de ideales principales de Krull.
			Finalmente $\dim X = \dim Y - 1$.

		\item Ejercicio.
		\item El que $f_W\colon X_W \to W$ sea un encaje regular viene del lema~\ref{thm:a2quot_free}.
			El que $p^*\mathscr{C}_{X/Y} \to \mathscr{C}_{X_W/W}$ sea un isomorfismo se sigue de que es un epimorfismo entre haces
			localmente libres del mismo rango sobre $X$. \qedhere
	\end{enumerate}
\end{proof}

\begin{prop}\label{thm:inm_between_smooth_are_reg}
	Sea $S$ un esquema localmente noetheriano.
	Sean $X, Y$ esquemas suaves sobre $S$.
	Todo encaje $f \colon X \to Y$ de $S$-esquemas es un encaje regular y tenemos la siguiente sucesión exacta:
	\begin{center}
		\begin{tikzcd}[sep=large]
			0 \rar & \mathscr{C}_{X/Y} \rar & f^* \Omega_{Y/S}^1 \rar & \Omega_{X/S}^1 \rar & 0.
		\end{tikzcd}
	\end{center}
\end{prop}
\begin{proof}
	Sea $x \in X, y := f(x)$ y $s$ la imagen de $y$ (y luego de $x$) en $S$ bajo el morfismo estructural.
	Por inducción sobre $e := \dim_y(Y_s) - \dim_x(X_s)$ probaremos que $f$ es regular de codimensión $e$ en $x$.
	Si $e = 0$, entonces $f$ es un isomorfismo por la proposición~\ref{}.
	\todo{Hacer demostración, \citeauthor{liu:algebraic}~\cite[222]{liu:algebraic}.}
	Si $e \ge 1$, entonces existe $f_1 \in \mathscr{I}_y \setminus \{ 0 \}$ tal que $Z := \VV_Y(f_1)$ es un entorno de $y$
	suave sobre $S$ en $x$.
	Así, la inclusión canónica 
	\begin{tikzcd}[cramped, sep=small]
		Z \rar[closed] & Y
	\end{tikzcd}
	es un encaje regular en $x$ y $\dim_y(Z_s) - \dim_x(X_s) = e - 1$, por lo que concluimos por hipótesis inductiva
	y por el inciso 1 de la proposición anterior.

	Para la sucesión exacta, en primer lugar cambiando $Y$ por un subesquema abierto podemos suponer que 
	\begin{tikzcd}[cramped, sep=small]
		f \colon X \rar[closed] & Y
	\end{tikzcd}
	es un encaje cerrado.
	Luego aplicando el inciso 4 del teorema~\ref{thm:rel_diff_props} construimos la siguiente sucesión exacta:
	\begin{center}
		\begin{tikzcd}
			\mathscr{C}_{X/Y} = \mathscr{I/I}^2 \rar["\delta"] &
			f^* \Omega_{Y/S}^1 = \Omega_{Y/S}^1 \otimes_{\mathscr{O}_Y} \mathscr{O}_X \rar &
			\Omega_{X/S}^1 \rar & 0,
		\end{tikzcd}
	\end{center}
	así que solo queda verificar que $\delta$ es un monomorfismo.
	Ahora bien, los términos de la sucesión son localmente libres y, por el corolario~\ref{thm:conormal_rank} vemos que
	$$ \rang( f^* \Omega_{Y/S}^1 )_x - \rang( \Omega_{X/S}^1 )_x = \dim_y(Y_s) - \dim_x(X_s) = e = \rang( \mathscr{C}_{X/Y} )_s. $$
	Y así $\delta$ establece un epimorfismo con su imagen, y como tienen el mismo rango, ha de ser un isomorfismo, por lo que $\delta$ es un monomorfismo.
\end{proof}

\begin{cor}\label{thm:section_smooth_sep_are_reg}
	Sea $Y$ un esquema localmente noetheriano y sea $f \colon X \to Y$ un morfismo suave y separado.
	Entonces toda sección $\pi \colon Y \to X$ (tal que $\pi \circ f = \Id_Y$) es un encaje cerrado regular
	y tenemos el isomorfismo canónico $\mathscr{C}_{Y/X} \simeq \pi^* \Omega_{X/Y}^1$.
\end{cor}
\begin{proof}
	Por la proposición~\ref{thm:separated_prop}, $X$ es un $Y$-esquema separado y $X, Y$ son $X$-esquemas, de modo que
	\begin{center}
		\begin{tikzcd}[sep=large]
			Y = X \times_X Y \rar[closed, "\pi", near end] & X \times_Y Y = X
		\end{tikzcd}
	\end{center}
	es un encaje cerrado.
	Luego aplicamos la sucesión exacta de la proposición anterior con $\Omega_{Y/Y}^1 = 0$ para concluir el enunciado.
	% ver que $\mathscr{C}_{Y/X} \simeq \pi^* \Omega_{X/Y}^1$.
\end{proof}

\begin{mydef}
	Sea $Y$ un esquema localmente noetheriano y $f \colon X \to Y$ un morfismo de tipo finito.
	Se dice que $f$ es \strong{intersección completa local} (abrev., \strong{i.c.l.})\index{intersección completa local (morfismo)}%
	\index{morfismo!intersección completa local (i.c.l.)}
	en un punto $x \in X$ si existe un entorno $U$ de $x$ y un diagrama conmutativo
	\begin{center}
		\begin{tikzcd}[row sep=large]
			{}                               & Z \dar["g"] \\
			U \rar["f|_U"'] \urar["i", hook] & Y
		\end{tikzcd}
	\end{center}
	donde $i$ es un encaje regular y $g$ es un morfismo suave.
	Se dice que $f$ es de \strong{intersección completa local} (i.c.l.) si lo es en todos los puntos de $X$.
\end{mydef}
\begin{prop}
	Se cumplen:
	\begin{enumerate}
		\item Todo encaje regular y todo morfismo suave son i.c.l.
		\item La composición de morfismos i.c.l. es i.c.l.
		\item Sea $f \colon X \to Y$ un morfismo i.c.l., y sea $W$ un $Y$-esquema plano.
			Entonces $f_W \colon X_W \to W$ es i.c.l.
	\end{enumerate}
\end{prop}
\begin{proof}
	\begin{enumerate}
		\item Trivial.
		\item Como los encajes regulares y los morfismos suaves son estables salvo composición,
			basta probar que si $f \colon X \to Y$ es suave y $g \colon Y \to Z$ es un encaje regular, entonces $f \circ g$ es i.c.l.
			Como $f$ es suave, por el corolario~\ref{thm:smooth_morph_decomp}, en un abierto tenemos que $f$ se factoriza por un encaje cerrado 
			\begin{tikzcd}[cramped, sep=small]
				i\colon X \rar[closed] & \A_Y^n
			\end{tikzcd}
			y la proyección canónica $\A_Y^n \to Y$.
			Luego, haciendo cambio de base por $\A_Z^n$ tenemos que $g'\colon \A_Y^n \hookto \A_Z^n$ es un encaje regular.
			Por la proposición~\ref{thm:inm_between_smooth_are_reg} tenemos que $i$ es regular, luego $i\circ g'$ es un encaje regular
			y claramente la proyección $\A_Z^n \to Z$ es suave.
		\item Basta recordar que tanto los encajes regulares como los morfismos suaves son estables salvo cambio de base plano.
			\qedhere
	\end{enumerate}
\end{proof}


\begin{prop}\label{thm:lci_fact_inmersion_reg}
	Sea $f \colon X \to Y$ un morfismo i.c.l.
	\begin{enumerate}
		\item Si $f$ es un encaje, entonces es un encaje regular.
		\item Si $f$ se descompone en un encaje $i \colon X \hookto Z$ y un morfismo suave $g \colon Z \to Y$, entonces $i$ es un encaje regular.
			Más aún, si $f$ es un encaje regular, se induce la siguiente sucesión exacta
			\begin{center}
				\begin{tikzcd}[sep=large]
					0 \rar & \mathscr{C}_{X/Y} \rar & \mathscr{C}_{X/Z} \rar & i^* \Omega_{Z/Y}^1 \rar & 0.
				\end{tikzcd}
			\end{center}
	\end{enumerate}
\end{prop}
\begin{proof}
	\begin{enumerate}
		\item Pasando a un abierto $U$ podemos emplear la descomposición siguiente (cor.~\ref{thm:smooth_morph_decomp}):
			\begin{center}
				\begin{tikzcd}[row sep=large]
					V := X \times_Z U \dar[open] \rar[hook, "\text{regular}"] & U \dar[open] \rar["\text{étale}"] & \A_Y^n \dar \\
					X \rar[hook, "\text{regular}"']                           & Z \rar["\text{suave}"']            & Y
				\end{tikzcd}
			\end{center}
			de modo que es fácil ver que se factoriza por un encaje regular $V \hookto \A_Y^n$ con la proyección $\A_Y^n \to Y$.
			Luego queda al lector verificar que componer con $\A^Y_n$ da un encaje regular (aplíquese la proposición~\ref{app:flat_tensor_depths}).
		\item Cambiando $Z$ por un subesquema abierto de $Y$ podemos suponer que $i$ es un encaje cerrado.
			Defínase $X \times_Y Z$. entonces tenemos el siguiente diagrama conmutativo:
			% https://q.uiver.app/#q=WzAsNSxbMCwwLCJYIl0sWzEsMCwiVyJdLFsxLDEsIlgiXSxbMiwwLCJaIl0sWzIsMSwiWSJdLFsxLDJdLFsxLDNdLFszLDQsImciXSxbMiw0LCJmIiwyXSxbMCwzLCJpIiwwLHsibGFiZWxfcG9zaXRpb24iOjgwLCJjdXJ2ZSI6LTN9XSxbMCwyLCIiLDIseyJsZXZlbCI6Miwic3R5bGUiOnsiaGVhZCI6eyJuYW1lIjoibm9uZSJ9fX1dLFswLDEsIlxccGkiLDAseyJsYWJlbF9wb3NpdGlvbiI6ODB9XSxbMSw0LCIiLDAseyJzdHlsZSI6eyJuYW1lIjoiY29ybmVyLWludmVyc2UifX1dXQ==
			\[\begin{tikzcd}
				X & W & Z \\
				& X & Y
				\arrow[from=1-2, to=2-2]
				\arrow[from=1-2, to=1-3]
				\arrow["g", from=1-3, to=2-3]
				\arrow["f"', from=2-2, to=2-3]
				\arrow[closed, "i"{pos=0.8}, bend left, from=1-1, to=1-3]
				\arrow[Rightarrow, no head, from=1-1, to=2-2]
				\arrow[closed, "\pi"{pos=0.8}, from=1-1, to=1-2]
				\arrow["\ulcorner"{anchor=center, pos=0.125}, draw=none, from=1-2, to=2-3]
			\end{tikzcd}\]
			donde $\pi$ es la sección de $g_X \colon Z_X \to X$, el cual es suave y separado por ser cambio de base de un morfismo suave y separado,
			de modo que es un encaje cerrado regular.
			Como $Z$ es plano sobre $Y$, el morfismo $f_Z$ es i.c.l.
			Luego $\pi \circ f_Z = i$ es i.c.l., y es un encaje, así que el inciso anterior prueba que es un encaje regular.

			Si suponemos que $f$ es un encaje regular, entonces por el inciso 1 de la proposición~\ref{thm:regular_inm_prop} tenemos la sucesión exacta
			\begin{center}
				\begin{tikzcd}[sep=large]
					0 \rar & \pi^* \mathscr{C}_{W/Z} \rar & \mathscr{C}_{X/Z} \rar & \mathscr{C}_{X/W} \rar & 0
				\end{tikzcd}
			\end{center}
			y por el corolario~\ref{thm:section_smooth_sep_are_reg} tenemos
			\begin{gather*}
				\pi^* \mathscr{C}_{W/Z} \simeq \pi^* p^* \mathscr{C}_{X/Z} \simeq \mathscr{C}_{X/Y}, \\
				\mathscr{C}_{X/W} \simeq \pi^* \Omega_{W/X}^1 \simeq \pi^* q^* \Omega_{Z/Y}^1 \simeq i^* \Omega_{Z/Y}^1,
			\end{gather*}
			que es precisamente lo que queríamos probar.
			\qedhere
	\end{enumerate}
\end{proof}

\begin{lem}
	Sea $S$ un esquema localmente noetheriano y sea $i \colon X \hookto Y$ un encaje de $S$-esquemas localmente noetherianos y planos.
	Sea $s \in S$ un punto y $x \in X_s$ en la fibra, si $i_s \colon X_s \hookto Y_s$ es un encaje regular en $x$,
	entonces $i$ es un encaje regular en $x$.
\end{lem}
\begin{proof}
	Podemo ssuponer que $i$ es un encaje cerrado y definir $\mathscr{I} \subseteq \mathscr{O}_Y$ el haz de ideales
	tal que $X := \VV_Y(\mathscr{I})$. Entonces tenemos la sucesión exacta
	\begin{center}
		\begin{tikzcd}
			0 \rar & \mathscr{I}_x \rar & \mathscr{O}_{Y, x} \rar & \mathscr{O}_{X, x} \rar & 0
		\end{tikzcd}
	\end{center}
	que, localmente sobre $s \in S$ induce
	\begin{center}
		\begin{tikzcd}
			0 \rar & \mathscr{I}_x \otimes_{\mathscr{O}_{S, s}} \kk(s) = \mathscr{I}_x\mathscr{O}_{Y_s, x} \rar &
			\mathscr{O}_{Y_s, x} \rar & \mathscr{O}_{X_s, x} \rar & 0.
		\end{tikzcd}
	\end{center}
	Por hipótesis, existen $a_1, \dots, a_n \in \mathscr{I}_x$ cuyas imágenes $b_1, \dots, b_n \in \mathscr{I}_x\mathscr{O}_{Y_s, x}$
	forman un sistema generador débilmente $\mathscr{O}_{Y_s, x}$-regular. Es decir
	$$ \frac{\mathscr{I}_x}{(a_1, \dots, a_n)} \otimes_{\mathscr{O}_{S, s}} \kk(s) = 0, $$
	lo que implica, por el lema de Nakayama, que $\mathscr{I}_x = (a_1, \dots, a_n)$.

	Finalmente, tenemos la situación de un homomorfismo de anillos locales noetherianos $A := \mathscr{O}_{S, s} \to \mathscr{O}_{Y, x} =: B$
	con un $B$-módulo $N := \mathscr{I}_x$ que es $A$-plano y $M := A$, y concluimos por la proposición~\ref{app:flat_tensor_depths}.
\end{proof}
Este es un primer ejemplo de <<descenso plano>>.

\begin{cor}\label{lci_local_on_fibers}
	Sea $Y$ un esquema localmente noetheriano y $f \colon X \to Y$ un morfismo plano de tipo finito.
	Entonces $f$ es i.c.l. syss para todo $y \in Y$ la fibra $f_y \colon X_y \to \Spec\kk(y)$ es i.c.l.
\end{cor}
\begin{proof}
	$\impliedby$.
	Sobre un abierto afín uno puede descomponer $f$ como un encaje $i \colon X \to Z := \A_Y^n$ con la proyección $p\colon Z \to Y$.
	Sea $y \in Y$ y $x \in X_y$, de modo que $f_y$ es i.c.l.; como $p_y \colon \A_{\kk(y)}^n \to \kk(y)$ es suave, entonces $i_y$ es un encaje regular,
	luego como $Z_y$ es $Z$-plano, entonces descendemos para ver que $i$ es un encaje regular.
	Así $f$ es i.c.l.

	$\implies$. Supongamos la misma situación anterior donde $i$ es un encaje regular y $p$ es suave.
	Sea $i$ de manera que $\mathscr{O}_{X, x}$ está generado por una sucesión débilmente regular $b_1, \dots, b_m \in \mathscr{O}_{Z, m}$.
	En particular, $\kdim\mathscr{O}_{X, x} = \kdim\mathscr{O}_{Z, x} - m$ y, además, por el corolario~\ref{thm:flat_fibers_dimension}
	\begin{align*}
		\kdim\mathscr{O}_{X_y, x} &= \kdim\mathscr{O}_{X, x} - \kdim\mathscr{O}_{Y, y}, \\
		\kdim\mathscr{O}_{Z_y, x} &= \kdim\mathscr{O}_{Z, x} - \kdim\mathscr{O}_{Y, y};
	\end{align*}
	por lo que $\kdim\mathscr{O}_{X_y, x} = \kdim\mathscr{O}_{Z_y, x} - m$.
	Luego al localizar (que es tensorizar por un módulo plano) las imágenes de los $b_i$ siguen forman una sucesión débilmente regular y,
	por tanto, $X_y \to Z_y$ es un en encaje regular.
\end{proof}

\section*{Notas históricas}
Los morfismos étale fueron una invención de Grothendieck.
Varios autores señalan que la palabra \textit{étale} es poco frecuente en francés, salvo en poesía y usualmente referido al mar donde tendría el
equivalente español de <<calmado, tranquilo o profundo>>.
Señala \citeauthor{mumford:red}~\cite{mumford:red}:
<<La palabra aparentemente se refiere a la apariencia del mar en marea alta bajo una luna llena en determinados tipos de clima.>>

\addtocategory{scheme}{mumford:red}


\input{cohomologia.tex}

% \section{La categoría derivada de un esquema}
% Recordemos las siguientes definiciones de álgebra homológica.
% Dada una categoría abeliana $\catA$, un \strong{complejo (de cocadenas)} es un diagrama $(A^\bullet, d^\bullet) \colon \Poset(\Z) \to \catA$,
% generalmente representado como
% \begin{center}
% 	\begin{tikzcd}[sep=large]
% 		\cdots \rar & A^{n-1} \rar["d^{n-1}"] & A^n \rar["d^n"] & A^{n+1} \rar["d^{n+1}"] & \cdots
% 	\end{tikzcd}
% \end{center}
% tal que $d^n \circ d^{n+1} = 0$ para todo $n \in \Z$.
% Los \strong{morfismos} de complejos son simplemente transformaciones naturales entre los diagramas.
% Así, los complejos conforman la categoría $\mathsf{Ch}(\catA)$.
% Se definen los \strong{objetos de cohomología} del complejo $A^\bullet$ como
% $$ h^p(A^\bullet) := \frac{\ker(d^p)}{\Img(d^{p-1})}. $$
% Es fácil notar que un morfismo de complejos $f^\bullet \colon A^\bullet \to B^\bullet$ induce flechas en cohomología $h^p(f^\bullet) \colon h^p(A^\bullet)
% \to h^p(B^\bullet)$, de modo que una flecha se dice un \strong{cuasi-isomorfismo} si induce un isomorfismo en toda la cohomología.
% Se define la \strong{categoría derivada} de $\catA$, denotada $\dercat(\catA)$, como la localización de $\mathsf{Ch}(\catA)$ respecto a los cuasi-isomorfismos.

\input{diferenciales.tex}

\input{proyectivos.tex}

\part{Geometría de variedades}
\chapter{El sitio étale}
En la sección \S\ref{sec:weil_conj_curves} ya vimos la introducción a las conjeturas de Weil y vimos dos soluciones, la primera en la misma sección mediante
métodos de conteo de Bombieri-Stepanov y la segunda adaptando la demostración original de Weil (vid., \S\ref{sec:weils_proof}).
Las llamadas \strong{conjeturas de Weil}\index{conjetura!de Weil} forman una generalización del caso de curvas y fueron un elemento determinante para
el desarrollo de la geometría algebraica en los años sucesivos. He aquí el enunciado.

Sea $X/\Fp[q]$ una variedad proyectiva, suave y geométricamente irreducible de dimensión $d$ sobre un cuerpo finito.
Definiendo las funciones $\zeta(X, s)$ de Hasse-Weil y $Z(X, t)$ (como en \S\ref{sec:weil_conj_curves}) las conjeturas son las siguientes:
\begin{enumerate}[{CW}1.]
	\item \textbf{Racionalidad:} $Z(X, t)$ es una función racional y de hecho:
		$$ Z(X, t) = \frac{P_1(t) P_3(t) \cdots P_{2d-1}(t)}{P_0(t) P_2(t) \cdots P_{2d}(t)} \in \Q(t), $$
		donde cada $P_i(t) \in \Z[t]$, donde $P_0(t) = 1 - t, P_{2d}(t) = 1 - q^d t$ y cada $P_i(t) = \prod_{j} (1 - \alpha_{ij}t) \in \algcl\Q[t]$
		con $0 < i < 2d$.
	\item \textbf{Ecuación funcional:} Si $\chi := \chi(X, \mathscr{O}_X)$ es la característica de Euler, entonces
		$$ Z\left( X, \frac{1}{q^d t} \right) = \pm q^{n \chi/2} t^\chi Z(X, t). $$
	\item \textbf{Hipótesis de Riemann:} Se cumple que $|\alpha_{ij}| = q^{i/2}$ para $0 < i < 2d$.
\end{enumerate}
También hay una cuarta conjetura CW4 que de momento no tocaremos, ya que no podemos enunciarla.

Históricamente, y siguiendo una acorazonada de Weil, el problema sería resuelto con una teoría cohomológica análoga
a la homología simplicial de topología algebraica.
El primer problema al que nos enfrentaremos es que la topología de Zariski que hemos dado a las variedades es irreducible, por lo que nuestros haces dan
muy poca información con la cohomología de \v Cech, así que hemos de redefinir nuestros espacios para que las propiedades se resuelvan adecuadamente.

\addtocategory{etale}{freitag:etale, poonen:rational}

\section{Descenso}
Comencemos con dos resultados sencillos de álgebra conmutativa:
% Comenzaremos la sección con algunos resultados 
\begin{prop}\label{thm:modules_satisfy_fpqc_desc}
	Sea $\varphi\colon A \to B$ un homomorfismo fielmente plano de anillos, sea $M$ un $A$-módulo y denótese $M_B := M \otimes_A B$ visto como $B$-módulo.
	Si $M_B$ es plano (resp.\ finitamente generado, de presentación finita, localmente libre de rango $n$),
	entonces $M$ también lo es.
\end{prop}
\begin{proof}
	Que la planitud satisfaga descenso plano es trivial.

	Sea $e_1, \dots, e_n$ un sistema generador de $M_B$ sobre $B$, donde cada $e_i := \sum_{j} m_{ij} \otimes b_{ij} \in M_B$ para $m_{ij} \in M$
	y $b_{ij} \in B$.
	Luego, el homomorfismo $\varphi\colon A^N \to M$ de evaluación sobre los $m_{ij}$'s satisface que $\varphi_B \colon B^N \to M_B$ sea sobreyectivo y,
	por definición de <<fielmente plano>>, se sigue que $\varphi$ debe ser sobreyectivo.
	Si $M_B$ es de presentación finita, en ésta situación tenemos que $\ker(\varphi_B)$ es un $B$-módulo finitamente generado,
	pero $\ker(\varphi_B) \cong \ker\varphi \otimes_A B$ por planitud, luego $\ker\varphi$ es finitamente generado.

	Finalmente, ser <<localmente libre>> equivale a ser <<plano de presentación finita>>, ambas propiedades que satisfacen descenso plano.
	La igualdad de los rangos se verifica pasando a un cuerpo de restos.
\end{proof}
\begin{prop}\label{thm:algs_fpqc_desc}
	Sea $\varphi \colon A \to A'$ un homomorfismo fielmente plano de anillos, sea $B$ una $A$-álgebra y sea $B' := B \otimes_A A'$ visto como $A'$-álgebra.
	Si $B'$ es una $A'$-álgebra de tipo finito (resp.\ de presentación finita), entonces $B$ también lo es como $A$-álgebra.
\end{prop}
\begin{hint}
	La demostración es análoga a la anterior empleando $A[t_1, \dots, t_N]$ en lugar de $A^{\oplus N}$.
\end{hint}

El problema del \textit{descenso} es principalmente el siguiente: sea $\mathbf{P}$ una propiedad de morfismos de esquemas
y supongamos que tenemos el siguiente diagrama conmutativo:
\begin{equation}
	% https://q.uiver.app/#q=WzAsNixbMCwxLCJYIl0sWzEsMSwiWSJdLFsyLDEsIlMiXSxbMiwwLCJTJyJdLFsxLDAsIlknIl0sWzAsMCwiWCciXSxbMCwxLCJmIiwyXSxbMSwyLCIiLDIseyJzdHlsZSI6eyJib2R5Ijp7Im5hbWUiOiJkb3R0ZWQifX19XSxbMywyLCJnIl0sWzUsMCwiIiwyLHsic3R5bGUiOnsiYm9keSI6eyJuYW1lIjoiZG90dGVkIn19fV0sWzUsNCwiZiciXSxbNCwxLCIiLDAseyJzdHlsZSI6eyJib2R5Ijp7Im5hbWUiOiJkb3R0ZWQifX19XSxbNCwzLCIiLDAseyJzdHlsZSI6eyJib2R5Ijp7Im5hbWUiOiJkb3R0ZWQifX19XSxbNSwxLCIiLDAseyJzdHlsZSI6eyJuYW1lIjoiY29ybmVyLWludmVyc2UifX1dLFs0LDIsIiIsMCx7InN0eWxlIjp7Im5hbWUiOiJjb3JuZXItaW52ZXJzZSJ9fV1d
	\begin{tikzcd}
		{X'} & {Y'} & {S'} \\
		X & Y & S
		\arrow["{f'}", from=1-1, to=1-2]
		\arrow[dotted, from=1-1, to=2-1]
		\arrow["\ulcorner"{anchor=center, pos=0.125}, draw=none, from=1-1, to=2-2]
		\arrow[dotted, from=1-2, to=1-3]
		\arrow[dotted, from=1-2, to=2-2]
		\arrow["\ulcorner"{anchor=center, pos=0.125}, draw=none, from=1-2, to=2-3]
		\arrow["g", from=1-3, to=2-3]
		\arrow["f"', from=2-1, to=2-2]
		\arrow[dotted, from=2-2, to=2-3]
	\end{tikzcd}
	\label{cd:descent_sit}
\end{equation}
¿Qué propiedad podemos imponer en $g$ de modo que si $f'$ satisface $\mathbf{P}$ entonces $f$ también satisface $\mathbf{P}$?
% Como el lector puede apreciar, la situación es \textit{inversa} a los teoremas de cambio de base, en donde una propiedad de $f$ se eleva a una propiedad de $f'$.

Un ejemplo sencillo:
\begin{prop}
	En el diagrama conmutativo \eqref{cd:descent_sit} supongamos que $g$ es sobreyectivo.
	Entonces:
	\begin{enumerate}
		\item El morfismo $f$ es sobreyectivo syss $f'$ lo es.
		\item Si $f'$ es inyectivo, entonces $f$ también lo es.
		\item Si $f'$ tiene fibras de cardinalidad finita, entonces $f$ también.
		\item El morfismo $f$ es universalmente inyectivo syss $f'$ lo es.
	\end{enumerate}
\end{prop}
\begin{hint}
	Los tres primeros incisos aplican para funciones en general en un diagrama conmutativo de conjuntos.
	El cuarto se sigue del segundo.
\end{hint}
% \begin{proof}
% 	\begin{enumerate}
% 		\item Ya hemos visto que si $f$ es sobreyectivo, entonces el cambio de base $f' = f_{S'}$ también.
% 			El recíproco es trivial de que $f' \circ g = g' \circ f$ es sobreyectivo, y luego $f$ lo será.
% 		\item E
% 	\end{enumerate}
% \end{proof}

% \begin{prop}
% 	En el diagrama conmutativo \eqref{cd:descent_sit} supongamos que $g$ es compacto y sobreyectivo.
% 	Entonces $f$ es un morfismo compacto syss $f'$ lo es.
% \end{prop}

\begin{mydef}
	% Sean $\mathbf{P}(f)$ y $\mathbf{Q}(f)$ propiedades de morfismos $f$ de esquemas.
	Sean $\mathbf{P}$ y $\mathbf{Q}$ propiedades de morfismos de esquemas.
	Diremos que una propiedad $\mathbf{P}$ de morfismos satisface <<descenso por morfismos $\mathbf{Q}$>> si en todo diagrama conmutativo
	\eqref{cd:descent_sit} donde $g$ satisface $\mathbf{Q}$ se cumple que
	% $\mathbf{P}(f') \implies \mathbf{P}(f)$.
	$f$ satisface $\mathbf{P}$ siempre que $f'$ satisface $\mathbf{P}$.
\end{mydef}
Más brevemente, diremos que una propiedad satisface descenso fpqc (resp.\ fppf, étale) si satisface descenso por morfismos fpqc (resp.\ fppf, étale).

He aquí el interés que podría tener la teoría de descenso para teoristas de números:
\begin{ex}
	Sea $k$ un cuerpo.
	El morfismo $\Spec(\algcl k) \to \Spec k$ siempre es fpqc, aunque casi nunca es fppf.
	Para toda extensión finita $L/k$ el morfismo $\Spec L \to \Spec k$ es fppf, aunque no siempre es étale.
	Para toda extensión finita separable $L/k$ el morfismo $\Spec L \to \Spec k$ es étale.
\end{ex}

\begin{mydef}
	Sea $Y$ un esquema compacto y cuasiseparado, un subconjunto $E \subseteq Y$ se dice \strong{proconstructible}\index{proconstructible (conjunto)}
	\index{conjunto!proconstructible} si existe un morfismo $f \colon \Spec A \to Y$ desde un esquema afín con $f[\Spec A] = E$.
\end{mydef}
\begin{cor}
	Sea $Y$ un esquema compacto y cuasiseparado. Entonces:
	\begin{enumerate}
		\item Todo subconjunto localmente constructible de $Y$ es proconstructible.
		\item La imagen de todo morfismo compacto $X \to Y$ es proconstructible.
		\item La intersección de finitos conjuntos proconstructibles de $Y$ es proconstructible.
	\end{enumerate}
\end{cor}

Recuérdese que decíamos que un morfismo de esquemas $f \colon X \to Y$ \textit{refleja generizaciones} si para todo punto $x \in X$
y toda generización $y' \speto f(x)$, existe $x' \speto x$ en $X$ tal que $y' = f(x')$.
Esto era equivalente a que $f[\Spec(\mathscr{O}_{X, x})] = \Spec(\mathscr{O}_{Y, f(x)})$.
% Como corolario al teorema de constructibilidad de Chevalley vimos que un morfismo localmente de presentación finita refleja generizaciones syss es abierto.
\addtocounter{thmi}{1}
\begin{slem}
	Sea $Y$ un esquema compacto y cuasiseparado, sea $Z \subseteq Y$ un subconjunto proconstructible y
	sea $f \colon X \to Y$ un morfismo que refleja generizaciones.
	Entonces $\overline{f^{-1}[Z]} = f^{-1}[ \,\overline{Z}\, ]$.
\end{slem}
\begin{proof}
	Nótese que podemos suponer que $Y = \Spec A$ sea afín (¿por qué?).
	Siempre se satisface la inclusión $\overline{f^{-1}[Z]} \subseteq f^{-1}[ \,\overline{Z}\, ]$, probemos <<$\supseteq$>>:
	sea $x \in U := X \setminus \overline{f^{-1}[Z]}$.
	Como $\Spec(\mathscr{O}_{X, x}) \subseteq U \subseteq f^{-1}[Y \setminus Z]$, vemos que
	$$ f[\Spec(\mathscr{O}_{X, x})] = \Spec(\mathscr{O}_{Y, f(x)}) \subseteq Y \setminus Z. $$
	Afirmamos que $y := f(x) \notin \overline{Z}$.
	Sea $A'$ una $A$-álgebra tal que $g\colon Y' := \Spec(A') \to Y$ tiene imagen $Z$;
	sea $\mathfrak{p} := \mathfrak{p}_y$ el primo asociado a $y \in Y$, como $g[\Spec(\mathscr{O}_{Y, y})] = \emptyset$, se verifica que
	$$ \limdir_{s \notin \mathfrak{p}} (A[1/s] \otimes_A A') = A_{\mathfrak{p}} \otimes_A A' = 0, $$
	por lo que existe $s \in A \setminus \mathfrak{p}$ tal que $A[1/s] \otimes_A A' = 0$, es decir, tal que $\DD(s) \cap Z = \emptyset$.
\end{proof}

\begin{slem}
	Sea $f \colon X \to Y$ un morfismo compacto que refleja generizaciones.
	Entonces $f \colon X \to f[X]$ es una identificación topológica (i.e., el subespacio topológico $f[X] \subseteq Y$ es un cociente topológico de $X$).
\end{slem}
\begin{proof}
	Nuevamente podemos suponer que $Y = \Spec A$ sea afín.
	Sea $Z \subseteq f[X]$ tal que $F := f^{-1}[Z] \subseteq X$ sea cerrado, queremos ver que $Z$ es cerrado en la topología subespacio;
	vale decir, que $Z = \overline{Z} \cap f[X]$.
	Dotándolo de la estructura reducida, vemos a $F$ como subesquema cerrado de $X$ y la composición 
	\begin{tikzcd}[cramped, sep=small]
		F \rar[closed] & X \rar & Y
	\end{tikzcd}
	da un morfismo compacto, por lo que su imagen $Z$ es proconstructible. Así pues
	$$ Z = f\big[ f^{-1}[Z] \big] = f\big[ \,\overline{f^{-1}[Z]}\, \big] = f^{-1}\big[ f[ \,\overline{Z}\, ] \big] = \overline{Z} \cap f[X] $$
	como se quería ver.
\end{proof}
\addtocounter{thmi}{-1}

Es inmediato del lema anterior que:
\begin{prop}
	Todo morfismo fpqc es una identificación.
\end{prop}

\begin{prop}
	Las siguientes propiedades satisfacen descenso fpqc:
	% \begin{multicols}{3}
	% 	\noindent
	% 	Ser morfismo abierto. \\
	% 	Ser morfismo compacto. \\
	% 	Ser morfismo cerrado. \\
	% 	Ser cuasiseparado. \\
	% 	Ser homeomorfismo. \\
	% 	Ser separado.
	% \end{multicols}
	\begin{center}
		\begin{tabular}{lll}
			Ser morfismo  abierto. & Ser morfismo cerrado. & Ser homeomorfismo. \\
			Ser morfismo compacto. &    Ser cuasiseparado. &      Ser separado.   
		\end{tabular}
	\end{center}
\end{prop}
\begin{proof}
	Considere la situación del diagrama \eqref{cd:descent_sit}.
	Las primeras dos se siguen inmediatamente de que las flechas verticales sean identificaciones topológicas;
	la tercera se sigue de que un homeomorfismo es una biyección abierta.

	Para ver que los morfismos compactos satisfacen descenso fpqc basta notar que si $V \subseteq Y$ es un abierto compacto,
	entonces
	$$ f^{-1}[V] = g_X[ (f')^{-1}[ g_Y^{-1}[V] ] ] $$
	es compacto, pues $g_Y$ es un morfismo compacto por cambio de base.

	Finalmente, un morfismo se dice cuasiseparado (resp.\ separado) syss la diagonal $\Delta_f \colon X \to X\times_Y X$ es un morfismo compacto
	(resp.\ un morfismo cerrado) y ya probamos que ambas propiedades satisfacen descenso fpqc.
\end{proof}
\begin{cor}
	Las propiedades de ser <<universalmente abierto>>, <<universalmente cerrado>> y de ser un <<homeomorfismo universal>> satisfacen descenso fpqc.
\end{cor}

\begin{prop}
	Las siguientes propiedades satisfacen descenso fpqc:
	\begin{center}
		\begin{tabular}{lll}
			Ser (localmente) de         tipo finito. &  Ser un isomorfismo. & Ser propio. \\
			Ser (localmente) de presentación finita. & Ser un monomorfismo. &   Ser afín. \\
			Ser                         cuasifinito. &       Ser un encaje. & Ser finito.   
		\end{tabular}
	\end{center}
\end{prop}
\begin{proof}
	En la situación del diagrama \eqref{cd:descent_sit} vemos que $g_Y$ es fpqc, así que supondremos $Y = S$.
	Ser (localmente) de tipo finito y de presentación finita satisface descenso fpqc por la proposición~\ref{thm:algs_fpqc_desc}.
	Ser cuasifinito es ser de tipo finito con fibras finitas, por lo que se sigue de las proposiciones anteriores.

	Si $f'$ es un isomorfismo, entonces también es un homeomorfismo universal y $f$ también por descenso.
	Así, basta ver que $f^\sharp \colon \mathscr{O}_Y \to f_*\mathscr{O}_X$ es un isomorfismo de haces; aplicándole el funtor $g^*$
	obtenemos el morfismo de haces
	\[\begin{tikzcd}[column sep=large]
		(f')^\sharp \colon \mathscr{O}_{Y'} = g^*\mathscr{O}_Y \rar["\sim"] & f^\prime_* \mathscr{O}_{X'} = g^*f_*\mathscr{O}_X
	\end{tikzcd}\]
	el cual es un isomorfismo por hipótesis.
	Así que $g^*f^\sharp$ es un isomorfismo, pero como $g$ es fielmente plano, $f^\sharp$ debe serlo.
	Un monomorfismo es un morfismo cuya diagonal es un isomorfismo, así que trivialmente satisface descenso fpqc.

	Ser propio equivale a ser de tipo finito, separado y universalmente cerrado; todas propiedades que satisfacen descenso fpqc.

	Si $f'$ es afín, entonces es compacto y cuasiseparado (por lo que $f$ también) y la $\mathscr{O}_{Y'}$-álgebra cuasicoherente
	$\mathscr{A}' := f^\prime_*\mathscr{O}_{X'}$ satisface que $X' = \bfSpec_{Y'}(\mathscr{A}')$.
	Definiendo $\mathscr{A} := f_*\mathscr{O}_X$, vemos que $g^*\mathscr{A}' = \mathscr{A}$ luego tenemos el siguiente diagrama conmutativo
	% https://q.uiver.app/#q=WzAsNixbMiwwLCJZJyJdLFsyLDEsIlkiXSxbMSwwLCJBJyJdLFsxLDEsIkEiXSxbMCwwLCJYJyJdLFswLDEsIlgiXSxbMCwxLCJnIl0sWzQsNSwiIiwwLHsic3R5bGUiOnsiYm9keSI6eyJuYW1lIjoiZG90dGVkIn19fV0sWzIsMywiIiwwLHsic3R5bGUiOnsiYm9keSI6eyJuYW1lIjoiZG90dGVkIn19fV0sWzUsM10sWzMsMSwiIiwyLHsic3R5bGUiOnsiYm9keSI6eyJuYW1lIjoiZG90dGVkIn19fV0sWzIsMCwiIiwxLHsic3R5bGUiOnsiYm9keSI6eyJuYW1lIjoiZG90dGVkIn19fV0sWzQsMiwiXFxzaW0iXSxbNCwzLCIiLDIseyJzdHlsZSI6eyJuYW1lIjoiY29ybmVyLWludmVyc2UifX1dLFsyLDEsIiIsMix7InN0eWxlIjp7Im5hbWUiOiJjb3JuZXItaW52ZXJzZSJ9fV0sWzUsMSwiZiIsMix7ImN1cnZlIjoxfV1d
	\[\begin{tikzcd}
		{X'} & {\bfSpec_{Y'}(\mathscr{A}')} & {Y'} \\
		X & {\bfSpec_Y(\mathscr{A})} & Y
		\arrow["\sim", from=1-1, to=1-2]
		\arrow[dotted, from=1-1, to=2-1]
		\arrow["\ulcorner"{anchor=center, pos=0.125}, draw=none, from=1-1, to=2-2]
		\arrow[dotted, from=1-2, to=1-3]
		\arrow[dotted, from=1-2, to=2-2]
		\arrow["\ulcorner"{anchor=center, pos=0.125}, draw=none, from=1-2, to=2-3]
		\arrow["g", from=1-3, to=2-3]
		\arrow["\bar{f}"', from=2-1, to=2-2]
		% \arrow["f"', curve={height=6pt}, from=2-1, to=2-3]
		\arrow[dotted, from=2-2, to=2-3]
	\end{tikzcd}\]
	y por descenso fpqc comprobamos que $\bar{f}$ es un isomorfismo, de modo que $f$ es afín.

	Ser finito equivale a ser afín y propio, las cuales satisfacen descenso.
\end{proof}

% \begin{thm}
% 	% En el diagrama conmutativo \eqref{cd:descent_sit} supongamos que $g$ es un morfismo fpqc.
% 	% Entonces:
% 	% \begin{enumerate}
% 	% 	\item El morfismo $f$ es 
% 	% \end{enumerate}
% 	Las siguientes propiedades satisfacen descenso fpqc:
% 	Ser de tipo finito.
% 	Ser universalmente inyectivo
% 	Ser separado.
% 	Ser propio.
% \end{thm}

Ahora seguimos a \citeauthor{bosch:neron}~\cite[129\psqq]{bosch:neron}, \S 6.1.
El problema que pretendemos resolver es el siguiente: dado un morfismo de esquemas $S' \to S$,
¿bajo qué condiciones el funtor de pull-back $\mathscr{F} \mapsto p^*\mathscr{F}$ es una equivalencia de categorías (canónicamente)?
O en otras palabras, ¿cuándo un haz sobre $S'$ \emph{desciende} a un haz de $S$?
% Más aún, sería útil poder decir que esta equivalencia respete ciertas propiedades de haces (como <<ser localmente libre>>, )

% Considere el siguiente resultado elemental: dados dos esquemas $X$ e $Y$ con subesquemas abiertos $U$ y $V$,
% tales que existe un isomorfismo 
% \begin{tikzcd}[cramped, sep=small]
% 	\varphi\colon U \rar["\sim"] & V
% \end{tikzcd}
% se pueden pegar en un esquema $X \amalg_\varphi Y$.
% El lector podría releer esto en el lenguaje de que se pueden pegar esquemas Zariski-localmente,
% y podría preguntarse bajo qué condiciones se pueden pegar étale-localmente o fppf-localmente.
\begin{mydef}
	Sea $p \colon S' \to S$ un morfismo de esquemas y denótese $S'' := S' \times_S S'$.
	Dado un $\mathscr{O}_{S'}$-módulo $\mathscr{F}'$,
	un par $(\mathscr{F}', \varphi)$ se dice un \strong{dato de cubrimiento} si $\varphi \colon \pi_1^*\mathscr{F}' \to \pi_2^*\mathscr{F}'$
	es un isomorfismo de $\mathscr{O}_{S''}$-módulos, donde $\pi_1, \pi_2 \colon S'' \to S'$ denotan las proyecciones.
	Los haces con datos de cubrimiento forman una categoría $\catC_{S'/S}$ donde una flecha $f \colon (\mathscr{F}', \varphi) \to (\mathscr{G}', \psi)$
	es un morfismo de haces $f \colon \mathscr{F}' \to \mathscr{G}'$ tal que el siguiente diagrama conmuta
	\[\begin{tikzcd}[row sep=large]
		\pi_1^*\mathscr{F}' \dar[sloped, "\sim"'] \rar["{\pi_1^*f}"'] & \pi_1^*\mathscr{G}' \dar[sloped, "\sim"] \\
		\pi_2^*\mathscr{F}'                        \rar["{\pi_2^*f}"] & \pi_2^*\mathscr{G}'
	\end{tikzcd}\]
\end{mydef}
Nótese que si $\mathscr{F}$ es un $\mathscr{O}_S$-módulo, entonces $p^*\mathscr{F}$ es un haz sobre $S'$ con la identidad como dato de cubrimiento. 

\begin{prop}\label{thm:pullback_fpqc_is_fully_faith}
	Sea $p \colon S' \to S$ un morfismo fpqc, denótense $\pi_j \colon S'' := S'\times_S S' \to S'$ las proyecciones y $q := \pi_1 \circ p = \pi_2 \circ p$.
	Dado un par de $\mathscr{O}_S$-módulos cuasicoherentes $\mathscr{F}$ y $\mathscr{G}$, el siguiente diagrama es un ecualizador
	\[\begin{tikzcd}
		\Hom_S(\mathscr{F}, \mathscr{G}) \rar["p^*"] & \Hom_{S'}(p^*\mathscr{F}, p^*\mathscr{G}) \rar[shift left, "\pi_1^*"] \rar[shift right, "\pi_2^*"']
							     & \Hom_{S''}(q^*\mathscr{F}, q^*\mathscr{G}).
	\end{tikzcd}\]
	En consecuencia, el funtor $p^*(-) \colon \mathsf{QCoh}_S \to \catC_{S'/S}$ es plenamente fiel.
\end{prop}
\begin{proof}
	Podemos suponer que $S = \Spec A$ es afín.
	Como el morfismo es compacto, podemos cubrir a $S'$ con finitos abiertos afines $\{ U_i \}_i$, de modo que el morfismo $\coprod_i U_i \to \Spec A$
	es fielmente plano, de modo que podemos suponer que $S' = \Spec B$ es afín.

	Ahora, $\mathscr{F} = \widetilde{M}$ y $\mathscr{G} = \widetilde{N}$ y
	$$ \Hom_{S'}(p^*\mathscr{F}, p^*\mathscr{G}) = \Hom_B(M\otimes_A B, N\otimes_A B) \cong \Hom_A(M, N) \otimes_A B, $$
	por lo que, definiendo $H := \Hom_A(M, N)$, nos reducimos a probar que la sucesión
	\[\begin{tikzcd}
		0 \rar & H \rar & H \otimes_A B \rar["\gamma"] & H \otimes_A B \otimes_A B
	\end{tikzcd}\]
	es exacta, donde $\gamma$ es la tensorización del homomorfismo $b \mapsto b\otimes 1 - 1\otimes b$ de $B \to B\otimes_A B$.
	Es decir, la sucesión de arriba es la tensorización de la sucesión exacta $0 \to A \to B \to B\otimes_A B$ lo cual induce la sucesión exacta
	\[\begin{tikzcd}
		\Tor_1^A(H, B\otimes_A B) \rar & H \rar & H \otimes_A B \rar["\gamma"] & H \otimes_A B \otimes_A B,
	\end{tikzcd}\]
	donde $\Tor_1(H, B\otimes_A B) = 0$ pues $B \otimes_A B$ es un $A$-módulo fielmente plano (¿por qué?).
\end{proof}
En consecuencia, $p^*(-)$ establece una equivalencia con una subcategoría de $\catC_{S'/S}$.

\begin{sit}\label{sit:cocycle_fpqc}
	Sea $p \colon S' \to S$ un morfismo de esquemas.
	Denotaremos por $S'' := S' \times_S S'$ y $S''' := S'' \times_S S'$,
	las cuales vendrán dotadas de las proyecciones $\pi_1, \pi_2 \colon S'' \to S'$ y
	$\pi_{ij} \colon S''' \to S''$ con $1 \le i < j \le 3$.
\end{sit}
\begin{mydef}
	En la situación~\ref{sit:cocycle_fpqc}, un haz cuasicoherente sobre $S'$ con dato de cubrimiento $(\mathscr{F}', \varphi)$
	se dice que conforma un \strong{dato de descenso}\index{dato!de descenso} si el siguiente diagrama conmuta (llamado \textit{condición de cociclos}):
	\begin{center}
		\includegraphics[scale=1]{geo-alg/cocycle_cond.pdf}
	\end{center}
	% satisface la siguiente \textit{condición de cociclos}:
	% \[
	% 	\pi_{13}^* \varphi = \pi_{12}^* \varphi \circ \pi_{23}^* \varphi.
	% \]
	Los haces cuasicoherentes con datos de descenso conforman una subcategoría plena $\catD_{S'/S} \subseteq \catC_{S'/S}$.
	Se dice que el dato de descenso $\varphi$ es \strong{efectivo}\index{dato!de descenso!efectivo} si el par $(\mathscr{F}', \varphi)$
	está en la imagen esencial\footnotemark{} del funtor de pull-back $p^*(-)$.
\end{mydef}
\footnotetext{Vale decir, si $(\mathscr{F}', \varphi)$ es isomorfo (en $\catC_{S'/S}$)
a $p^*\mathscr{F}$, para algún haz cuasicoherente $\mathscr{F}$ sobre $S$.}
\begin{ex}
	En la situación~\ref{sit:cocycle_fpqc}, sea $\mathscr{F}$ un haz cuasicoherente sobre $S$.
	Entonces, se sigue que $\Id$ es un dato de descenso para $p^*\mathscr{F}$,
	de modo que $p^*(-) \colon \mathsf{QCoh}_S \to \catD_{S'/S}$ es plenamente fiel.
\end{ex}

\addtocounter{thmi}{1}
\begin{slem}
	Supongamos que $p \colon S' \to S$ es un morfismo con una sección (o equivalentemente, que $S'(S) \ne \emptyset$).
	Entonces todo dato de descenso sobre un $\mathscr{O}_{S'}$-módulo cuasicoherente $\mathscr{F}'$ es efectivo.
\end{slem}
\begin{proof}
	Sea $s \in S'(S)$.
	Vamos a probar que la cuasi-inversa de $p^*(-) \colon \mathsf{QCoh}_S \to \catD_{S'/S}$ es $s^*$.
	Sea $(\mathscr{F}', \varphi)$ un haz cuasicoherente sobre $S'$ con un dato de descenso;
	dados tres puntos $T$-valuados $t_1, t_2, t_3 \in S'(T)$ y empleando pull-back por $(t_i, t_j) \colon T \to S''$ obtenemos
	el isomorfismo $\varphi_{t_i, t_j} \colon t_i^* \mathscr{F}' \to t_j^* \mathscr{F}'$ (con $1 \le i < j \le 3$) de <<ser un dato de cubrimiento>>
	y la condición de cociclos como
	$$ \varphi_{t_1, t_3} = \varphi_{t_1, t_2} \circ \varphi_{t_2, t_3}. $$
	Empleando los puntos $t_1 = \Id_{S'}$ y $t_2 = s \circ p \in S'(S')$ obtenemos un isomorfismo
	\[
		f := \varphi_{t_1, t_2} \colon \mathscr{F}' = t_1^*\mathscr{F}' \to t_2^*\mathscr{F}' = p^*\mathscr{F},
	\]
	donde $\mathscr{F} := s^*\mathscr{F}'$.
	Para probar que $f$ es un isomorfismo en $\catC_{S'/S}$ queremos probar que el diagrama
	\begin{equation}
		\begin{tikzcd}[row sep=large]
			\pi_1^*\mathscr{F}' \dar["\pi_1^*f"'] \rar["\varphi"] & \pi_2^*\mathscr{F}' \dar["\pi_2^*f"] \\
			\pi_1^*p^*\mathscr{F}                 \rar[equals]    & \pi_2^*p^*\mathscr{F}
		\end{tikzcd}
		\label{eqn:effective_desc_datum}
	\end{equation}
	conmuta, para ello considere los puntos valuados
	$$ \pi_1, \quad \pi_2, \quad t_3 := \pi_1 \circ p \circ s = \pi_2 \circ p \circ s \in S'(S''), $$
	y nótese que
	\[
		\varphi_{ \pi_1, \pi_2} = \varphi, \qquad \varphi_{\pi_1, t_3} = \pi_1^* f, \qquad \varphi_{\pi_2, t_3} = \pi_2^* f,
	\]
	de modo que la condición de cociclos se traduce en que $\pi_1^*f = \varphi \circ \pi_2^*f$, es decir, que \eqref{eqn:effective_desc_datum} conmuta.
\end{proof}
\addtocounter{thmi}{-1}

\begin{thm}[Grothendieck]
	Sea $p \colon S' \to S$ un morfismo fpqc.
	Entonces el funtor $p^*(-) \colon \mathsf{QCoh}_S \to \catD_{S'/S}$ es una equivalencia de categorías.
	Dicho de otro modo, todo dato de descenso sobre un $\mathscr{O}_{S'}$-módulo cuasicoherente es efectivo.
\end{thm}
\begin{proof}
	Sea $u \colon T' \to S'$ un morfismo fpqc y sea $\overline{p} := u\circ p \colon T' \to S$,
	tenemos el siguiente diagrama conmutativo de categorías y funtores
	\[\begin{tikzcd}
		\mathsf{QCoh}_S \rar["p^*"] \drar["\overline{p}^*"'] & \catD_{S'/S} \dar["u^*", hook] \\
		{}                                                   & \catD_{T'/S'},
	\end{tikzcd}\]
	donde $u^*$ es plenamente fiel por la proposición~\ref{thm:pullback_fpqc_is_fully_faith}.
	Así, si probamos el enunciado para $\overline{p}^*$, veremos que $u^*$ es una equivalencia y luego se sigue que $p^*$ también.

	Por ello, podemos suponer que $S = \Spec A$ y $S' = \Spec B$.
	Sea $(\mathscr{F}' = \widetilde{N}, \varphi)$ un $B$-módulo con dato de descenso $\varphi \colon N \otimes_A B \to B \otimes_A N$
	(donde en el tensor identificamos a $N$ con un $A$-módulo).
	Los $A$-homomorfismos
	\begin{tikzcd}[cramped, sep=small]
		B \rar[shift left] \rar[shift right] & B \otimes_A B
	\end{tikzcd}
	dados por $b\mapsto b\otimes 1$ y $b \mapsto 1\otimes b$ inducen, al tensorizar por $-\otimes_B N$ y componer con $\varphi$, un par de $A$-homomorfismos 
	\begin{tikzcd}[cramped, sep=small]
		N \rar[shift left] \rar[shift right] & N \otimes_A B,
	\end{tikzcd}
	y definimos el $A$-módulo
	\[
		K := \ker\mathopen{}\left(
			\begin{tikzcd}
				N \rar[shift left] \rar[shift right] & N \otimes_A B
			\end{tikzcd}
		\right)\mathclose{}.
	\]
	Queremos probar que $K \otimes_A B = N$, pero, de momento, solo podemos construir el siguiente diagrama conmutativo:
	% https://q.uiver.app/#q=WzAsNixbMCwwLCJLIl0sWzEsMCwiS1xcb3RpbWVzX0FCIl0sWzEsMSwiTiJdLFswLDEsIksiXSxbMiwxLCJOXFxvdGltZXNfQSBCIl0sWzIsMCwiS1xcb3RpbWVzX0EgQlxcb3RpbWVzX0EgQiJdLFswLDMsIiIsMCx7ImxldmVsIjoyLCJzdHlsZSI6eyJoZWFkIjp7Im5hbWUiOiJub25lIn19fV0sWzEsMiwiXFxhbHBoYSIsMCx7InN0eWxlIjp7ImJvZHkiOnsibmFtZSI6ImRhc2hlZCJ9fX1dLFswLDFdLFszLDJdLFsxLDUsIiIsMSx7Im9mZnNldCI6LTF9XSxbMSw1LCIiLDEseyJvZmZzZXQiOjF9XSxbMiw0LCIiLDEseyJvZmZzZXQiOi0xfV0sWzIsNCwiIiwxLHsib2Zmc2V0IjoxfV0sWzUsNCwiXFxiZXRhIiwwLHsic3R5bGUiOnsiYm9keSI6eyJuYW1lIjoiZGFzaGVkIn19fV1d
	\[\begin{tikzcd}
		K & {K\otimes_AB} & {K\otimes_A B\otimes_A B} \\
		K & N & {N\otimes_A B}
		\arrow[from=1-1, to=1-2]
		\arrow[Rightarrow, no head, from=1-1, to=2-1]
		\arrow[shift left, from=1-2, to=1-3]
		\arrow[shift right, from=1-2, to=1-3]
		\arrow["\alpha", dashed, from=1-2, to=2-2]
		\arrow["\beta", dashed, from=1-3, to=2-3]
		\arrow[from=2-1, to=2-2]
		\arrow[shift left, from=2-2, to=2-3]
		\arrow[shift right, from=2-2, to=2-3]
	\end{tikzcd}\]
	donde queremos probar que $\alpha$ es un isomorfismo.
	Para ello, podemos hacer cambio de base fielmente plano $A \to B$ correspondiente al morfismo fpqc de esquemas $S'' \to S'$ que posee una sección,
	por lo que $N_B$ desciende a un $A_B$-módulo, por el lema anterior, y este necesariamente es $K_B$,
	es decir, $\alpha_B$ es un isomorfismo y, finalmente, $\alpha$ también.
\end{proof}

\section{Teoría de Galois-Grothendieck}
\subsection{Preludio: categorías de Galois}
El resultado al que queremos aspirar es el teorema~\ref{thm:Groth_Gal_connection} más adelante (pág.~\pageref{thm:Groth_Gal_connection}).
\begin{mydef}
	Sea $\catC$ una categoría.
	Dado un objeto $X \in \Obj\catC$ y un subgrupo de automorfismos $G \le \Aut(X)$, decimos que una flecha $q\colon X \to G\coquot X$ es un \strong{cociente
	categorial} si:
	\begin{enumerate}
		\item Dado $f \in G$ se cumple que $f \circ q = q$.
		\item Si $p \colon X \to Y$ satisface que para todo $f \in G$ se cumpla que $f\circ p = p$.
			Entonces existe una única flecha tal que el diagrama conmuta:
			\[\begin{tikzcd}[column sep=small]
				{} & X \dlar["q"'] \drar["p"] \\
				G\coquot X \ar[rr, "\exists!", dashed] & & Y
			\end{tikzcd}\]
	\end{enumerate}
	Por propiedad universal es claro que si existe el cociente categorial es único salvo $X$-isomorfismo.

	Sea $\catC$ una categoría con un funtor $F\colon \catC \to \mathsf{Fin}$.
	Se dice que $\catC$ es una \strong{categoría de Galois}\index{categoría!de Galois} con \strong{funtor fundamental}\index{funtor!fundamental} $F$ si:
	\begin{enumerate}[{G}1]
		\item $\catC$ es finitamente completa (i.e., posee objeto final y productos fibrados finitos).
		\item $\catC$ posee objeto inicial y coproductos finitos.
			Dado $X \in \catC$ y un subgrupo finito $G \le \Aut(X)$, existe el cociente categorial $G \coquot X$.
		\item\label{ax:galois_split_monos}
			Toda flecha $f \in \Morf\catC$ se factoriza $f = g\circ h$, donde $g$ es un epimorfismo y $h$ un monomorfismo.
			Todo monomorfismo $h \colon X \to Y$ se extiende a un diagrama conmutativo:
			% https://q.uiver.app/#q=WzAsMyxbMSwxLCJYIl0sWzAsMCwiWSJdLFsyLDAsIlhcXGFtYWxnIFoiXSxbMCwyLCIiLDAseyJzdHlsZSI6eyJ0YWlsIjp7Im5hbWUiOiJob29rIiwic2lkZSI6InRvcCJ9fX1dLFswLDEsImgiXSxbMSwyLCJcXHNpbSIsMCx7InN0eWxlIjp7ImJvZHkiOnsibmFtZSI6ImRhc2hlZCJ9fX1dXQ==
			\[\begin{tikzcd}[column sep=small]
				Y && {X\amalg Z} \\
				& X
				\arrow["\sim", dashed, from=1-1, to=1-3]
				\arrow["h", from=2-2, to=1-1]
				\arrow[hook, from=2-2, to=1-3]
			\end{tikzcd}\]
		\item $F$ preserva límites inversos.
		\item $F$ preserva epimorfismos, coproductos finitos y cocientes categoriales.
		\item $F$ refleja isomorfismos.
	\end{enumerate}
\end{mydef}
\newcommand{\init}{\vec 0}
\newcommand{ \fin}{\vec 1}
La segunda parte de la condición \ref{ax:galois_split_monos} puede expresarse como que <<todo monomorfismo se escinde>>.
Desde ahora, adquirimos el convenio que $\init$ (resp.\ $\fin$) denota el objeto inicial (resp.\ objeto final) de $\catC$ si existen.
También, emplearemos el abuso de notación de que $X = \init$ cuando $X$ es \emph{un} objeto inicial.

Para el siguiente lema, recuérdese que un \emph{objeto conexo} $X$ de una categoría $\catC$ con coproductos y objeto inicial
es aquel que se escribe $X \cong Y \amalg Z$ solo cuando $Y$ ó $Z$ es un objeto inicial.
Cuando $\catC$ es de Galois, esto equivale a que $X$ posea solamente por subobjetos a $\init$ y a $X$.
\begin{prop}\label{thm:Gal_cat_connected_prop}
	Sea $\catC$ una categoría de Galois con funtor fundamental $F$.
	Entonces:
	\begin{enumerate}
		\item Todo objeto en $\catC$ se escribe (de manera única) como coproducto finito de objetos conexos $\ne\init$.
		\item Si $A \in \Obj\catC$ es conexo y $a \in F(A)$, entonces la función
			\[
				\ev_a \colon \Hom_\catC(A, X) = (\yoneda A)(X) \longrightarrow F(X), \qquad f \longmapsto F(f)(a)
			\]
			es inyectiva.
		\item Si $X \ne \init$ e $Y$ es conexo, entonces toda flecha $X \to Y$ es un epimorfismo.
		\item Si $Y$ es conexo, entonces todo epimorfismo es un automorfismo.
	\end{enumerate}
\end{prop}
\begin{proof}
	\begin{enumerate}
		\item Basta emplear inducción sobre $|F(X)|$ para un objeto $X \in \Obj\catC$.
		\item Sean $f, g \in \Hom(A, X)$ tales que $F(f)(a) = F(g)(a)$.
			Definiendo $K := \ker(f, g)$ el ecualizador, vemos que $K$ es un subobjeto de $A$ y que $a \in F(K)$.
			Pero $F(\init) = \emptyset$, así que $K = A$ y $f = g$.
		\item Basta factorizar la flecha $f \in \Hom(X, Y)$ como
			\[\begin{tikzcd}
				X \rar[two heads, "g"] & Z \rar[hook, "i"] & Y \cong Z \amalg W.
			\end{tikzcd}\]
			Como $X \ne \init$, entonces $FX \ne \emptyset$ y luego $FZ \ne \emptyset$, por lo que $Y \cong Z$ y $W = \init$.
		\item Basta notar que $f \in \End(Y)$ es tal que $F(f) \colon FY \to FY$ es sobreyectiva entre conjuntos finitos, luego es biyectiva
			y, por tanto, $f$ es un isomorfismo. \qedhere
	\end{enumerate}
\end{proof}

De la proposición anterior, se sigue el siguiente lema como ejercicio:
\addtocounter{thmi}{1}
\begin{slem}
	Sea $\catC$ una categoría de Galois con funtor fundamental $F$.
	Para un objeto conexo $A$ son equivalentes:
	\begin{enumerate}
		\item Para un elemento $a \in FA$, la evaluación $\ev_a \colon \End(A) \to FA$ es sobreyectiva (y luego, biyectiva).
		\item La acción $\Aut(A) \acts F(A)$ es transitiva.
		\item La acción $\Aut(A) \acts F(A)$ es fielmente transitiva
			(i.e., para cada $a, b \in FA$ existe un único $\sigma \in \Aut(A)$ tal que $F(\sigma)(a) = b$).
	\end{enumerate}
\end{slem}
\addtocounter{thmi}{-1}

\begin{mydef}
	Un objeto $A$ en una categoría de Galois $\catC$ (con funtor fundamental $F$) se dice un \strong{objeto de Galois}\index{objeto!de Galois}
	si satisface las condiciones del lema anterior.
\end{mydef}
\begin{cor}
	Sea $\catC$ una categoría de Galois con funtor fundamental $F$.
	Dado un objeto de Galois $A$ y un elemento $a \in F(A)$, entonces para todo objeto $X \in \Obj\catC$,
	la función de evaluación $\ev_a \colon (\yoneda A)(X) \to F(X)$ es biyectiva.
\end{cor}

Dado un funtor $F \colon \catC \to \catD$ arbitrario entre categorías, podemos construir la clase $\Aut_{\mathsf{Nat}}(F)$
de transformaciones naturales $\sigma\colon F \To F$ con inversas.
También, recuérdese que, si $G$ es un grupo topológico, un \strong{$G$-conjunto} es un espacio topológico discreto $X$
con una acción $G \times X \to X$ que es conjunto.
\newcommand{\mhyph}{\text{-}}
\addtocounter{thmi}{1}
\begin{slem}
	Sea $F \colon \catC \to \catD$ un funtor arbitrario entre categorías, y supongamos que $\catC$ es (esencialmente) pequeña.
	Se cumplen:
	\begin{enumerate}
		\item La clase $\Aut(F)$ es un conjunto y, de hecho, un grupo topológico extremadamente disconexo.
		\item Si $\catD \subseteq \mathsf{Set}$ es una subcategoría de conjuntos,
			entonces $F$ se <<factoriza>> canónicamente por $F \colon \catC \to \Aut(F)\mathsf{\mhyph Set}$
		\item Si $\catD \subseteq \mathsf{Fin}$ es una subcategoría de conjuntos finitos, entonces $\Aut F$ es un grupo profinito.
			y $F$ se factoriza canónica por $F \colon \catC \to \Aut(F)\mathsf{\mhyph Fin}$.
	\end{enumerate}
\end{slem}
\begin{proof}
	Basta notar que, un automorfismo $\sigma \in \Aut(F)$, al evaluarse da un automorfismo $\sigma_X \in \Aut_\catD(FX)$ para todo $X \in \Obj\catC$.
	Más aún, claramente dos automorfismos de $F$ son iguales si todas las evaluaciones lo son, por lo que
	\begin{equation}
		\Aut(F) \subseteq \prod_{X} \Aut_\catD(FX),
		\label{eqn:Gal_aut_prod}
	\end{equation}
	donde $X$ recorre las clases de isomorfismo de objetos en $\catC$ y donde cada $\Aut(FX)$ es un conjunto,
	lo que prueba 1.

	Para la segunda, nótese que la evaluación determina un homomorfismo de grupos $\pi := \Aut F \to \Aut_\catD(FX)$ y,
	por tanto, una acción $\pi \acts FX$, de modo que $FX$ es un conjunto con una acción de $\pi$.
	Verificar que la acción es continua equivale a ver que los estabilizadores son subgrupos abiertos
	\[
		\Stab_x = \left\{ (\sigma_Y) \in \prod_{Y} \Aut(FY) : \sigma_X(x) = \sigma_X \right\},
	\]
	el cual es un producto de cada $\Aut(FY)$ para todo $Y \not\cong X$, y de un subconjunto de $\Aut(FX)$;
	este es abierto en la topología producto y.
	Finalmente, la definición de <<transformación natural>> prueba que las imágenes $F(f)$ de flechas respetan la acción de $\pi$.

	Cuando $\catD \subseteq \mathsf{Fin}$, entonces el lado derecho de \eqref{eqn:Gal_aut_prod} es un grupo profinito.
	Así, queda verificar que $\pi$ es un subgrupo \emph{cerrado} del lado derecho de \eqref{eqn:Gal_aut_prod}, esto del hecho de que,
	dada una flecha $Y \morf{f} Z$, se sigue que $\pi$ es la intersección de los subgrupos cerrados
	\begin{equation}
		\left\{ (\sigma_X) \in \prod_{X} \Aut(FX) : \sigma_Y\circ F(f) = F(f) \circ \sigma_Z \right\}.
		\tqedhere
	\end{equation}
\end{proof}

\begin{slem}
	Sea $\catC$ una categoría de Galois con funtor fundamental $F$.
	Entonces $F$ preserva y refleja monomorfismos (i.e., $X \morf{f} Y$ es un monomorfismo syss $F(f) \colon FX \to FY$ es inyectiva).
\end{slem}
\begin{proof}
	Basta notar que $F$ preserva monomorfismos pues preserva coproductos.
	Para ver que también refleja monomorfismos, basta notar que $f$ es un monomorfismo syss $(f, f) \colon X\times_Y X \to X$ es un isomorfismo
	y $F$ los refleja.
\end{proof}

\begin{slem}
	Sea $\catC$ una categoría de Galois esencialmente pequeña con funtor fundamental $F$.
	Entonces:
	\begin{enumerate}
		\item La familia $\mathcal{I}$ de pares $(A, a)$ salvo isomorfismo, donde $A \in \Obj\catC$ es conexo y $a \in F(A)$;
			con la relación $(A, a) \ge (B, b)$ cuando existe $A \morf{f} B$ con $F(f)(a) = b$, es un conjunto dirigido.
		\item La subfamilia $J \subseteq \mathcal{I}$ de pares $(A, a)$ con $A$ un objeto de Galois, es cofinal.
	\end{enumerate}
\end{slem}
\begin{proof}
	Para ver que el orden parcial sobre $\mathcal{I}$ está dirigio, sean $(A, a), (B, b) \in \mathcal{I}$.
	Definamos a $C \in \Obj\catC$ como la componente conexa de $(a, b) \in FA \times FB = F(A\times B)$ por la proposición~\ref{thm:Gal_cat_connected_prop},
	de modo que $(C, (a, b)) \ge (A, a)$ y $(C, (a, b)) \ge (B, b)$.

	Para ver que $J$ es cofinal, sea $(X, x) \in \mathcal{I}$, construyamos $Y := X^{\prod F(X)}$,
	y sea $\vec a \in FY = F(X)^{F(X)}$ cuya $x$-ésima coordenada es $x$ para cada $x \in FX$.
	Finalmente, sea $A$ la componente conexa de $\vec a$, entonces claramente $(A, a) \ge (X, x)$.
	Denotemos por $\pi_x$ la composición 
	\begin{tikzcd}[cramped, sep=small]
		A \rar[hook] & X^{F(X)} \rar[two heads, "p"] & X
	\end{tikzcd}
	donde $p$ es la proyección en la $x$-ésima coordenada.
	Luego $\ev_{\vec a}(\pi_x) = x$, por lo que $\ev_{\vec a} \colon \Hom(A, X) \to FX$ es sobreyectivo,
	e inyectivo por prop.~\ref{thm:Gal_cat_connected_prop}.

	Sea $\vec b \in FA$ otro elemento. Entonces $\ev_{\vec b} \colon \Hom(A, X) \to FX$ es inyectivo y, por cardinalidad, biyectivo.
	Es decir, que $\vec b \in F(Y) = F(X)^{F(X)}$ corresponde a una permutación de los elementos de $F(X)$; así que, como existe un
	automorfismo $\sigma \in \Aut(Y)$ que permuta las proyecciones vemos que $F(\sigma)(\vec a) = \vec b$.
	Es claro que $\sigma$ transforma la componente conexa $A$ en otra componente $B$, pero como $\vec b \in FA \cap FB$,
	se sigue que $A = B$, por lo que se restringe a un automorfismo de $A$.
	Así vemos que la acción $\Aut A \acts FA$ es transitiva.
\end{proof}

\begin{slem}
	Sea $\catC$ una categoría de Galois esencialmente pequeña con funtor fundamental $F$.
	Entonces $F$ es prorrepresentable (i.e., $F$ es el límite inverso de funtores representables).
\end{slem}
\begin{proof}
	Podemos construir el diagrama de funtores $\{ \yoneda A \}_{(A, a) \in \mathcal{I}}$ y vemos que
	$\ev_a \colon \yoneda A \To F$ determina un co-cono, por lo que tenemos una transformación natural:
	\[
		\alpha_X\colon \left( \limdir_{(A, a) \in \mathcal{I}} \yoneda A \right)(X)
		= \limdir_{(A, a) \in \mathcal{I}} \Hom(A, X) \longrightarrow FX.
	\]
	La inyectividad es clara del inciso 1 de la proposición~\ref{thm:Gal_cat_connected_prop},
	mientras que la sobreyectividad se sigue de que, dado $x \in F(X)$ existe una componente conexa $A \hookto X$ tal que $x \in F(A)$.
\end{proof}
\addtocounter{thmi}{-1}

\begin{thm}[Galois-Grothendieck]\label{thm:Groth_Gal_connection}
	Sea $\catC$ una categoría esencialmente pequeña de Galois con funtor fundamental $F$. Entonces:
	\begin{enumerate}
		\item El funtor canónico $F \colon \catC \to \Aut(F)\mathsf{\mhyph Fin}$ es una equivalencia de categorías.
		\item Si $\pi$ es un grupo profinito tal que $F\colon \catC \to \mathsf{Fin}$ es isomorfo a $\pi \mathsf{\mhyph Fin} \to \mathsf{Fin}$,
			entonces $\pi$ y $\Aut F$ son canónicamente isomorfos (en $\mathsf{TopGrp}$).
		\item En consecuencia, dado otro funtor fundamental $F'$, entonces $F$ y $F'$ son isomorfos.
		\item Si $F, G\colon \catC \to \pi\mathsf{\mhyph Fin}$ son funtores fundamentales, entonces la equivalencia determina
			un automorfismo interno del grupo profinito $\pi$.
	\end{enumerate}
\end{thm}

\subsection{De vuelta a los cubrimientos étale}
El objetivo es ahora poder demostrar, aplicar e interpretar la correspondencia de Galois-Grothendieck para cubrimientos étale.
Para ello, primero debemos probar que los recubrimientos étale finitos conforman una categoría de Galois; la única parte
que no es directa es la existencia de cocientes:
\begin{prop}
	Fijemos un esquema base $S$ y denotemos por $\mathscr{A}_S$ la categoría de $S$-esquemas cuyo morfismo estructural
	es afín y sobreyectivo.
	Dado un objeto $X \in \Obj(\mathscr{A}_S)$ y un subgrupo finito $G \le \Aut_S(X)$, existe el cociente categorial $G\coquot X$ (en $\mathscr{A}_S$).
\end{prop}
\begin{proof}
	Denotaremos por $p \colon X \to S$ al morfismo estructural.
	Vamos a manualmente construir el espacio anillado $(G \coquot X, \mathscr{O}_{G\coquot X})$:
	su espacio topológico subyacente es el espacio cociente $\pi \colon X \to G\coquot X$,
	mientras que el haz estructural $\mathscr{O}_{G \coquot X}$ es el haz $(\pi_*\mathscr{O}_X)^G$ de elementos $G$-invariantes.
	Este es claramente un cociente categorial entre espacios anillados.

	Para ver que $G \coquot X$ es un esquema, supondremos, ya que $p$ es afín, que $S = \Spec A$ y $X = \Spec B$ son afines
	y denotaremos $\varphi := p^\sharp \colon A \to B$ el homomorfismo estructural.
	Así, queremos ver que $G \coquot X = \Spec(B^G)$.
	Sea $q \colon X \to X_G := \Spec(B^G)$ el morfismo canónico dado por la inclusión $B^G \hookto B$;
	nótese que la extensión de anillos $B / B^G$ es entera ya que cada $b \in B$ es raíz del polinomio mónico $\prod_{\sigma \in G} (x - \sigma(b))
	\in B^G[x]$, por lo que $X \to X_G$ es un morfismo entero y el morfismo $X_G \to S$ es inmediatamente afín y sobreyectivo.
	Nótese que $q$ manda el cerrado $\VV(\mathfrak{a}) \subseteq X$ en $\VV(\mathfrak{a}^G) \subseteq X^G$ por lo que la sobreyectividad es clara.

	Veamos que las fibras de $q$ son $G$-órbitas:
	claramente la fibra de $\mathfrak{p}^G$ contiene a $\{ \sigma[\mathfrak{p}] \}_{\sigma \in G}$;
	si ademas contuviese a otra órbita $\{ \sigma[\mathfrak{q}] \}_{\sigma \in G}$, entonces $\sigma\mathfrak{p}$ y $\sigma\mathfrak{q}$
	inducen ideales maximales en el anillo artiniano
	\[
		\overline{B} := B \otimes_{B^G} \kk(\mathfrak{p}^G) = \Gamma(X_{\mathfrak{p}^G}, \mathscr{O}_{ X_{\mathfrak{p}^G} }),
	\]
	con $\bigcap_{\sigma} \sigma\mathfrak{p} = \bigcap_{\sigma} \sigma\mathfrak{q} = 0$.
	Pero esto es absurdo por el teorema chino del resto.

	Finalmente, nótese que $(\pi_* \mathscr{O}_X)^G$ es un haz cuasicoherente sobre $X_G$ ya que es el núcleo del morfismo de haces cuasicoherentes:
	\[
		\pi_*\mathscr{O}_X \longrightarrow \bigoplus_{\sigma \in G} \pi_*\mathscr{O}_X, \qquad s \longmapsto (s - \sigma(s))_\sigma,
	\]
	por lo que para verificar que $(\pi_* \mathscr{O}_X)^G \simeq \mathscr{O}_{X_G}$ basta verificarlo en secciones globales,
	donde ambos corresponden a $B^G$.
\end{proof}

\begin{cor}
	Si $X \to S$ es un recubrimiento étale finito y conexo, entonces para todo subgrupo $H \le \Aut(X/S)$
	se cumple que los morfismos $X \to H \coquot X$ y $H \coquot X \to S$ son recubrimientos étale finitos.
\end{cor}
\begin{proof}
	Como los morfismos étale y los morfismos finitos admiten cancelación derecha por morfismos separados,
	basta verificar que $H \coquot X \to S$ es étale finito.
	% Por la proposición anterior, $H\coquot X \to S$ es afín y sobreyectivo, y es claro que es cuasifinito, así que es un morfismo finito.
	Que $X \to S$ sea étale finito significa que existe un $S$-esquema $T$ tal que $X \times_S T \cong F \times_S T$,
	donde $F$ es un $S$-esquema étale finito escindido.
	La acción $H \acts X$ se induce una acción natural $H \acts X_T$ y, como $X_T$ es un $T$-esquema étale escindido,
	se comprueba que $G \coquot (X \times_S T) = (G\coquot F) \times_S T$.
	Por otro lado, es claro que $X_T \to (G \coquot X)_T$ es constante en órbitas de $G$, de modo que hay un $T$-morfismo
	$G \coquot (X \times_S T) \epicto (G \coquot X) \times_S T$ que es afín y sobreyectivo.
	Finalmente, para ver que es un isomorfismo basta verificarlo en abiertos afines, donde es claro.
\end{proof}

\begin{prop}
	Sea $S$ un esquema conexo y sea $\overline{s} \in S(\Omega)$ un punto geométrico.
	La categoría $\mathsf{F\acute Et}_S$ es una categoría de Galois esencialmente pequeña con el funtor fibra
	\[
		F_{\overline{s}}\colon \mathsf{F\acute Et}_S \to \mathsf{F\acute Et}_\Omega \cong \mathsf{Set}, \qquad
		X \mapsto X_{\overline{s}} = X \times_S \Spec(\Omega)
	\]
	como funtor fundamental.
	Definimos $\pi_1^{\text{ét}}(S, \overline{s}) := \Aut(F_{\overline{s}})$ y lo llamamos
	el \strong{grupo fundamental étale}\index{grupo!fundamental!(étale)} de $S$ basado en $\overline{s}$.
\end{prop}
\begin{cor}
	Sea $S$ un esquema conexo y sea $\overline{s} \in S(\Omega)$ un punto geométrico.
	La categoría $\mathsf{\acute Et}_S$ es una categoría de Galois y existe una equivalencia de categorías
	\[
		\mathsf{\acute Et}_S \longrightarrow \pi_1^{\text{ét}}(S, \overline{s})\mathsf{\mhyph Set}.
	\]
\end{cor}

Otra consecuencia del formalismo de Grothendieck es <<la invarianza topológica del punto>>:
\begin{cor}
	Sea $S$ un esquema conexo y sean $s_1 \in S(\Omega_1)$ y $s_2 \in S(\Omega_2)$ un par de puntos geométricos.
	Entonces existe un isomorfismo entre los funtores fibra 
	\begin{tikzcd}[cramped, sep=small]
		F_{s_1} \rar["\sim"] & F_{s_2},
	\end{tikzcd}
	que induce un isomorfismo (continuo) entre grupos fundamentales étale 
	\begin{tikzcd}[cramped, sep=small]
		\pi_1^{\text{ét}}(S, s_1) \rar["\sim"] & \pi_1^{\text{ét}}(S, s_2).
	\end{tikzcd}
\end{cor}
En consecuente, podemos hablar libremente del grupo fundamental étale $\pi_1^{\text{ét}}(X)$ sin referencia a un punto geométrico base
si es que solo nos interesa su clase de isomorfismo como grupo profinito.

\begin{exn}\label{ex:etale_cohom_is_Gal}
	Sea $S = \Spec k$ el espectro de un cuerpo $k$.
	La categoría $\mathsf{F\acute Et}_k$ es la categoría opuesta de $k$-álgebras conmutativas separables
	de dimensión finita (como $k$-espacios vectoriales).
	Sus objetos conexos corresponden a extensiones finitas separables y sus objetos de Galois son extensiones de Galois finitas.
	Finalmente, como el grupo fundamental étale se calcula como un límite inverso de los grupos de automorfismos de los objetos de Galois,
	se comprueba que $\pi_1^{\text{ét}}(\Spec k, \overline{s}) = \Gal(\sepcl k/k)$.
	(Recuerde que canónicamente $\Gal(\sepcl k/k) \cong \Gal(\algcl k/k)$.)
\end{exn}

Es fácil probar que los recubrimientos étale conexos son los objetos conexos de $\mathsf{\acute Et}_S$,
así que queda preguntarse por los objetos de Galois:
\begin{mydef}
	Sea $S$ un esquema conexo y sea $X$ un recubrimiento étale conexo de $S$.
	Se dice que $X$ es un \strong{recubrimiento de Galois}\index{recubrimiento!de Galois} si para todo punto geométrico $\overline{s} \in S(\Omega)$
	la acción de $\Aut(X/S)$ en la fibra geométrica $X_{\overline{s}}$ es transitiva.
\end{mydef}

La correspondencia de Galois clásica es ahora una mera formalidad:
\begin{prop} % Szamuely, Prop. 5.3.8
	Sea $f\colon X \to S$ un recubrimiento de Galois finito.
	Dado un recubrimiento étale finito conexo $Z \to S$ en un diagrama conmutativo
	\[\begin{tikzcd}
		X \drar["f"'] \rar["\pi"] & Z \dar \\
				          & S
	\end{tikzcd}\]
	entonces $\pi \colon X \to Z$ es un recubrimiento de Galois (finito) y, de hecho, $Z \cong H \coquot X$ con $H = \Aut(X/Z) \le \Aut(X/S)$.
	Así, hay una biyección entre subgrupos de $\Aut(X/S)$ y recubrimientos étale conexos intermedios;
	más aún, $Z \to S$ es de Galois syss $\Aut(X/Z)$ es un subgrupo normal de $\Aut(X/S)$, en cuyo caso
	\[
		\Aut(Z/S) \cong \Aut(X/S)/\Aut(X/Z).
	\]
\end{prop}
\begin{prop} % Szamuely, Prop. 5.3.9
	Sea $p\colon X \to S$ un recubrimiento étale finito conexo.
	Existe un morfismo $\pi \colon Y \to X$ tal que:
	\begin{enumerate}
		\item $\pi \circ p \colon Y \to X$ es un recubrimiento de Galois.
		\item Si $g \colon Z \to X$ es tal que $g\circ p \colon Z \to X$ es un recubrimiento de Galois,
			entonces existe un morfismo $Z \to Y$ tal que el siguiente diagrama conmuta:
			\[\begin{tikzcd}
				Z \ar[rr] \drar["\exists"']  &                 & X \\
							     & Y \urar["\pi"']
			\end{tikzcd}\]
	\end{enumerate}
\end{prop}

\begin{prop} % EGA IV_2 4.5.13 (p. 61)
	Sea $X$ un esquema conexo sobre un cuerpo $k$.
	Si existe un morfismo $Y \to X$, donde $Y$ es un esquema geométricamente conexo y no vacío sobre $k$,
	entonces $X$ es geométricamente conexo.
\end{prop}
\begin{proof}
	Sea $\overline{f} := f_{\algcl k}\colon \overline{Y} := Y_{\algcl k} \to X_{\sepcl k} =: \overline{X}$.
	Dado un subconjunto $\overline{U} \subseteq \overline{X}$ que es abierto, cerrado y no vacío; como las proyecciones $p\colon \overline{X} \to X$
	y $q\colon \overline{Y} \to Y$ son funciones abiertas y cerradas (¿por qué?), se sigue que $p[ \,\overline{U}\, ] = X$.
	Así, $\overline{f}^{-1}[ \,\overline{U}\, ]$ ha de ser un abierto y cerrado no vacío de $\overline{Y}$, de modo que $\overline{U} = \overline{X}$.
	Aplicando el mismo razonamiento a $\overline{X} \setminus \overline{U}$ comprobamos que éste ha de ser vacío, es decir, que $\overline{X}$ es conexo.
\end{proof}

De esto se siguen inmediatamente dos corolarios:
\begin{cor}
	Sea $X$ un esquema sobre un cuerpo $k$. Entonces:
	\begin{enumerate}
		\item Dado un $k$-morfismo $f \colon Y \to X$, donde $Y$ es geométricamente conexo sobre $k$,
			entonces su imagen es una componente geométricamente conexa de $X$.
		\item Si $Y \subseteq X$ es una componente irreducible que es también geométricamente irreducible sobre $k$,
			entonces la componente conexa $X_0 \subseteq X$ que le contiene es también geométricamente conexa.
	\end{enumerate}
\end{cor}
\begin{cor}
	Sea $X$ un esquema conexo sobre un cuerpo $k$.
	Si $X$ posee un punto $x \in X$ cuyo cuerpo de restos sea puramente inseparable sobre $k$ (e.g., si $X(k) \ne \emptyset$),
	entonces $X$ es geométricamente conexo.
\end{cor}

\begin{prop}
	Sea $X$ un esquema sobre un cuerpo $k$, sea $x \in X$ un punto y sea $K := \kk(x) \cap \sepcl k$.
	Supongamos que se satisfacen las siguientes condiciones:
	\begin{enumerate}[(a)]
		\item $X$ es conexo.
		\item La extensión $K/k$ es finita (e.g., si $X$ es localmente de tipo finito).
	\end{enumerate}
	Entonces $X_{\sepcl k}$ posee $\le [K : k]$ componentes geométricamente conexas y si $L \supseteq K$ es una extensión de Galois,
	entonces la componentes conexas de $X_L$ son geométricamente conexas.
\end{prop}
\begin{proof}
	Sea $L/K/k$ una extensión de Galois finita.
	La proyección $p\colon X_L \to X$ es un morfismo étale finito, luego la imagen de toda componente conexa $Y_\alpha \subseteq X_L$ es todo $X$.
	Así, la fibra $(X_L)_x$ tiene tantos puntos como $X_L$ tiene componentes conexas; y $(X_L)_x = \Spec(\kk(x) \otimes_k L)$
	el cual tiene $\le [K : k]$ puntos.
	Más aún, para todo $y \in (X_L)_x$ vemos que $\kk(y)$ es una extensión puramente inseparable de $L$,
	así que concluimos por el corolario anterior.
\end{proof}

\begin{prop}
	Sea $X$ un esquema localmente de tipo finito sobre un cuerpo $k$.
	Existe un un $k$-morfismo $q \colon X \to \pi_0(X)$ con la siguiente propiedad universal:
	\begin{enumerate}
		\item $\pi_0(X)$ es étale sobre $k$.
		\item Dado otro $k$-morfismo $g \colon X \to Y$, donde $Y$ es étale sobre $k$, existe $h \colon \pi_0(X) \to Y$ tal que $g = q\circ h$.
	\end{enumerate}
	Más aún, $q$ es un morfismo fielmente plano y sus fibras son las componentes conexas de $X$.
\end{prop}
\begin{proof}
	Sea $\overline{X} := X \times_k \Spec(\sepcl k)$ el cambio de base y sea $\pi_0^{\rm tg}(X) := \pi_0\big( \overline{X}_{\rm top} \big)$,
	donde $\pi_0$ ahora denota el conjunto usual de las componentes conexas del espacio topológico $\overline{X}_{\rm top}$.
	Nótese que $\pi_0^{\rm tg}(X)$ viene dotado de una acción del grupo de Galois absoluto $\Gamma_k := \Gal(\sepcl k/k)$.
	Sea $C \subseteq X_{\sepcl k}$ una componente conexa, entonces mediante la proyección $X_{\sepcl k} \epicto X$,
	su imagen es una componente conexa $D \subseteq X$.
	Ahora, $D$ es un esquema conexo y localmente de tipo finito sobre $k$, luego $D_{\sepcl k}$ tiene finitas componentes conexas,
	por lo que la $\Gamma_k$-órbita de $C$ es finita; esto prueba que la acción es continua.

	Definamos
	$$ \pi_0(\overline{X}) := \coprod_{\pi_0^{\rm tg}(X)} \Spec(\algcl k). $$
	Éste es un esquema étale sobre $\sepcl k$ y podemos construir el $\sepcl k$-morfismo $\overline{q} \colon \overline{X} \to \pi_0(\overline{X})$
	como aquel que manda cada componente conexa en un punto distinto y verificar que se satisfacen las hipótesis.
	Como $\pi_0(\overline{X})$, visto como conjunto, está dotado de una acción continua de $\Gamma_k$, existe un esquema étale $\pi_0(X)$ sobre $k$
	y un isomorfismo de $\Gamma_k$-conjuntos $\alpha\colon \pi_0(X)(\sepcl k) \to \pi_0(\overline{X})$;
	así, mediante $\alpha$, tenemos el $\sepcl k$-morfismo
	\[
		\overline{q} \colon X \times_k \Spec(\sepcl k) \longrightarrow \pi_0(X) \times_k \Spec(\sepcl k),
	\]
	el cual determina un $k$-morfismo $q \colon X \to \pi_0(X)$ por descenso de Galois que,
	del mismo modo, puede verificarse que satisface las hipótesis necesarias.
\end{proof}
\begin{cor}
	Sean $X, Y$ un par de esquemas localmente de tipo finito sobre un cuerpo $k$.
	Entonces $\pi_0(X) \times_k \pi_0(Y) \cong \pi_0(X \times_k Y)$ canónicamente.
	En particular, para toda extensión algebraica $K/k$ se cumple que $\pi_0(X_K) = \pi_0(X)_K$.
\end{cor}
\begin{hint}
	Puede hacer cambio de base por una extensión finita para poder suponer que $X$ e $Y$ poseen puntos racionales,
	y luego aplicar descenso fpqc.
\end{hint}

\subsection{Sucesiones exactas en homotopía}
En ésta sección veremos las propiedades funtoriales del grupo fundamental étale $\pi_1^{\text{ét}}$.
\begin{sit}\label{sit:etale_exact}
	Sean $S, S'$ un par de esquemas conexos, sea $\Omega$ un cuerpo separablemente cerrado,
	sean $\overline{s} \in S(\Omega)$ y $\overline{s}' \in S'(\Omega)$ un par de puntos geométricos
	y $f \colon S' \to S$ un morfismo tal que $\overline{s}' \circ f = \overline{s}$.
	Denotaremos por
	\[
		\operatorname{BC}_{S, S'} \colon \mathsf{\acute Et}_S \longrightarrow \mathsf{\acute Et}_{S'}, \qquad X \longmapsto X \times_S S'
	\]
	al funtor de cambio de base.
	Así, $\mathcal{F}_{\overline{s}} = \operatorname{BC}_{S, S'} \circ \mathcal{F}_{\overline{s}'}$, lo que determina un homomorfismo (continuo)
	entre los grupos fundamentales étale
	\[
		f_* \colon \pi_1^{\text{ét}}(S', \overline{s}') \longrightarrow \pi_1^{\text{ét}}(S, \overline{s}).
	\]
\end{sit}

Al siguiente resultado se le llama <<invarianza topológica>> ya que los engrosamientos de orden finito son homeomorfismos universales.
\begin{prop}
	En la situación~\ref{sit:etale_exact},
	si $S'$ es un engrosamiento de orden finito de $S$, entonces $\operatorname{BC}_{S, S'}$ es un isomorfismo de categorías y,
	en consecuencia, 
	\begin{tikzcd}[cramped, sep=small]
		f_* \colon \pi_1^{\text{ét}}(S', \overline{s}') \rar["\sim"] & \pi_1^{\text{ét}}(S, \overline{s})
	\end{tikzcd}
	es un isomorfismo (continuo) de grupos topológicos.
\end{prop}
% \begin{proof}
% 	La consecuencia se sigue formalmente
% \end{proof}

\begin{prop}
	En la situación~\ref{sit:etale_exact}, se cumplen:
	\begin{enumerate}
		\item El homomorfismo $f_*$ es nulo syss para cada recubrimiento étale finito conexo $X \to S$
			se cumple que el cambio de base $X \times_S S'$ es un recubrimiento étale finito escindido.
		\item El homomorfismo $f_*$ es sobreyectivo syss para cada recubrimiento étale finito conexo $X \to S$
			se cumple que el cambio de base $X \times_S S'$ también es conexo.
	\end{enumerate}
\end{prop}
\begin{proof}
	\begin{enumerate}
		\item $\implies$.
			Basta recordar que un recubrimiento étale finito $X' \to S'$ se escinde
			syss la acción de $\pi_1^{\text{ét}}(S', \overline{s}')$ es trivial.

			$\impliedby$.
			Por contrarrecíproca, si $\Img(f_*)$ es no nula, existe un subgrupo normal abierto $U \nsupseteq \Img(f_*)$
			de modo que la acción de $\pi_1^{\text{ét}}(S', \overline{s}')$ sobre el grupo finito $f_* \pi_1^{\text{ét}}(S, \overline{s}) / U$
			no es trivial.
			Ahora bien, tomando preimagen, existe un recubrimiento de Galois $X \to S$ cuyo grupo de $S$-automorfismos es
			$\pi_1^{\text{ét}}(S, \overline{s}) / (f_*)^{-1}U$ y su cambio de base tiene
			acción no trivial de $\pi_1^{\text{ét}}(S', \overline{s}')$

		\item $\implies$.
			Basta notar que un recubrimiento étale finito es conexo
			syss corresponde a un conjunto con acción de $\pi_1^{\text{ét}}(S, \overline{s})$ transitiva.

			$\impliedby$.
			Si $\Img(f_*) \subset \pi_1^{\text{ét}}(S', \overline{s}')$, entonces es un subgrupo cerrado compacto,
			de modo que existe un subgrupo abierto normal $U \nsl_o \pi_1^{\text{ét}}(S', \overline{s}')$ tal que
			$U \subseteq \Img(f_*)$, luego existe un recubrimiento étale finito conexo $X \to S$ con
			\[
				\Aut(X/S) = (f_*)^{-1}U \coquot \pi_1^{\text{ét}}(S, \overline{s}),
			\]
			pero $X' = X \times_S S'$ tiene la acción trivial de $U \coquot \Img(f_*)$, mediante $f_*$, por lo que no es conexo.
			\qedhere
	\end{enumerate}
\end{proof}

\addtocounter{thmi}{1}
\begin{slem}
	En la situación~\ref{sit:etale_exact}, sea $U \nsle_o \pi_1^{\text{ét}}(S, \overline{s})$ un subgrupo normal abierto
	y sea $X \to S$ el recubrimiento étale conexo con $\Aut(X/S) = \pi_1^{\text{ét}}(S, \overline{s})/U$.
	Sea $\overline{x} \in X_{\overline{s}}$ el punto en la fibra geométrica correspondiente a la clase $1\mod U$.

	Entonces $U \supseteq \Img(f_*)$ syss el cambio de base $X' := X_{S'} \to S'$ posee una sección $S' \to X'$
	que manda $\overline{s}'$ en $\overline{x}$.
\end{slem}
\begin{proof}
	Nótese que $\Img(f_*) \subseteq U$ syss la acción $X \acts \pi_1^{\text{ét}}(S', \overline{s}')$ (mediante $f_*$) fija a $\overline{x}$,
	es decir, que la componente conexa de $\overline{x}$ en $X'$ está fija bajo la acción de $\pi_1^{\text{ét}}(S', \overline{s}')$.
	Por tanto, esta componente es isomorfa a $S'$ meduante $X' \to S'$.
\end{proof}
\addtocounter{thmi}{-1}
\begin{prop}
	En la situación~\ref{sit:etale_exact}, sea $U' \nsle_o \pi_1^{\text{ét}}(S', \overline{s}')$ un subgrupo normal abierto
	y sea $X' \to S'$ el recubrimiento étale conexo con $\Aut(X/S) = \pi_1^{\text{ét}}(S', \overline{s}')/U'$.

	Entonces $U' \supseteq \ker(f_*)$ syss existe un recubrimiento étale finito $X \to S$ y un $S'$-morfismo $X_j \to X'$,
	donde $X_j$ es una componente conexa del cambio de base $X \times_S S'$.
\end{prop}
\begin{proof}
	$\implies$.
	Sea $U \le_o \pi_1^{\text{ét}}(S, \overline{s})$ tal que $\Aut(X/S) \cong U \coquot \pi_1^{\text{ét}}(S, \overline{s})$.
	Así, tras elegir un punto geométrico, la componente conexa $X_j \subseteq X \times_S S'$ se identifica
	con $(f_*)^{-1}U \coquot \pi_1^{\text{ét}}(S', \overline{s}')$ y $(f_*)^{-1}U$ es un subgrupo abierto que contiene a $\ker(f_*)$.
	Finalmente, la existencia de un $S'$-morfismo $X_j \to X'$ equivale a que $(f_*)^{-1}U \subseteq U'$.

	$\impliedby$.
	Si $U' \supseteq \ker(f_*)$, entonces $V' := f_*[U']$ es un subgrupo abierto del grupo profinito $H := \Img(f_*)$;
	luego existe $V \le_o \pi_1^{\text{ét}}(S', \overline{s}')$ tal que $V' = V \cap H$, y existe $X \to S$ asociado
	al espacio cociente $V \coquot \pi_1^{\text{ét}}(S, \overline{s})$.
	Así pues, una componente conexa $X_j \subseteq X \times_S S'$ corresponde al espacio cociente de
	algún subgrupo abierto $U'' \le_o \pi_1^{\text{ét}}(S', \overline{s}')$ y hay un $S'$-morfismo $X_j \to X'$ syss $U'' \subseteq U'$.
	Finalmente, como ambos contienen a $\ker(f_*)$, por le teorema de la correspondencia, esta inclusión equivale a que
	$f_*[U''] \subseteq f_*[U']$ lo cual se verifica por construcción.
\end{proof}

\begin{cor}
	En la situación~\ref{sit:etale_exact}, el homomorfismo $f_*$ es inyectivo syss para todo recubrimiento étale finito $X' \to S'$
	existe un recubrimiento étale finito $X \to S$ y un $S'$-morfismo $X_j \to X'$, donde $X_j$ es una componente conexa de $X \times_S S'$.
\end{cor}
\begin{cor}\label{thm:homotopy_exactness}
	Sean $f \colon S_1 \to S_2$ y $g \colon S_2 \to S_3$ morfismos entre esquemas conexos,
	y sean $\overline{s}_j \in S_j(\Omega)$ puntos geométricos
	con $\overline{s}_1 \circ f = \overline{s}_2$ y $\overline{s}_2 \circ g = \overline{s}_3$.
	Entonces, la sucesión
	\[\begin{tikzcd}[sep=large]
		\pi_1^{\text{ét}}(S_1, \overline{s}_1) \rar["f_*"] & \pi_1^{\text{ét}}(S_2, \overline{s}_2) \rar["g_*"] & \pi_1^{\text{ét}}(S_3, \overline{s}_3)
	\end{tikzcd}\]
	es exacta syss se satisfacen las siguientes condiciones:
	\begin{enumerate}[(i)]
		\item\label{thm:homotopy_exactness_i}
			Para todo recubrimiento étale finito $X_3 \to S_3$,
			el cambio de base $X_1 \times_{S_1} S_3 \to S_3$ es un recubrimiento étale escindido.
		\item\label{thm:homotopy_exactness_ii}
			Para todo recubrimiento étale finito conexo $X_2 \to S_2$ tal que el $S_3$-esquema $X_2 \times_{S_2} S_3$ posea sección,
			existe un recubrimiento étale finito conexo $X_1 \to S_1$ y un $S_2$-morfismo desde una componente conexa
			de $X_1 \times_{S_1} S_2$ hacia $X_2$.
	\end{enumerate}
\end{cor}

\addtocounter{thmi}{1}
\begin{slem}
	Sea $X$ un esquema compacto y geométricamente íntegro sobre un cuerpo $k$, y sea $\overline{X} := X \times_k \Spec(\sepcl k)$.
	Dado un recubrimiento étale finito $\overline{Y} \to \overline{X}$, existe una extensión finita separable $L/k$
	y un recubrimiento étale finito $Y_L \to X_L$ tal que $Y_L \times_L \Spec(\sepcl k) \cong \overline{Y}$.
\end{slem}
\begin{proof}
	Como $X$ es compacto, admite un cubrimiento finito por abiertos afines $\{ U_j = \Spec(A_j) \}_{j=1}^n$ y,
	como los morfismos finitos son afines, podemos suponer que la preimagen de $\overline{U}_j := U_j \times_k \Spec(\sepcl k)$ en $\overline{Y}$
	es un abierto afín $\Spec(\overline{B}_j)$, donde $\overline{B}_j$ es un $A_j \otimes_k \sepcl k$-módulo finitamente generado.
	Así, $\overline{B}_j = (A_j \otimes_k \sepcl k)[x_1, \dots, x_m]/\mathfrak{a}_j$, donde $\mathfrak{a}_j$ es un ideal generado
	por finitos polinomios $f_1, \dots, f_r$, donde cada uno con finitos términos involucra finitos elementos separables sobre $k$.
	Es decir, existe $L/k$ finita separable, donde cada $f_i \in (A_j \otimes_k L)[\vec x]$,
	de modo que el esquema afín $B_j := (A_j \otimes_k L)[\vec x]/(f_1, \dots, f_r)$ satisface que $B_j \otimes_L \sepcl k \cong \overline{B}_j$.
	Pegándolos, obtenemos al recubrimiento étale $Y_L \to X_L$.
\end{proof}
\addtocounter{thmi}{-1}
\begin{prop}
	Sea $X$ un esquema compacto y geométricamente íntegro sobre un cuerpo $k$, y sea $\overline{X} := X \times_k \Spec(\sepcl k)$.
	Fijemos un punto geométrico $\overline{x} \in \overline{X}(\Omega)$ que, mediante $\overline{X} \to X$, también identificamos con un
	punto geométrico de $X$. Entonces la siguiente sucesión es exacta:
	\[\begin{tikzcd}
		1 \rar & \pi_1^{\text{ét}}(\overline{X}, \overline{x}) \rar & \pi_1^{\text{ét}}(X, \overline{x}) \rar & \Gal(\sepcl k/k) \rar & 1.
	\end{tikzcd}\]
\end{prop}
\begin{proof}
	Para probar inyectividad de $\pi_1^{\text{ét}}(\overline{X}, \overline{x}) \to \pi_1^{\text{ét}}(X, \overline{x})$
	y sobreyectividad de $\pi_1^{\text{ét}}(X, \overline{x}) \to \Gal(\sepcl k/k)$ basta aplicar las proposiciones anteriores.
	Así, queda verificar exactitud en $\pi_1^{\text{ét}}(X, \overline{x})$, para lo que aplicamos el corolario~\ref{thm:homotopy_exactness}.
	En él, la condición \ref{thm:homotopy_exactness_i} es trivial, así que verifiquemos \ref{thm:homotopy_exactness_ii}:
	sea $Y \to X$ un recubrimiento de Galois finito tal que el recubrimiento $Y_{\sepcl k} \to \overline{X}$ posee una sección.
	Como $X$ es íntegro, la fibra genérica de $Y \to X$ es el espectro de una extensión finita de Galois $K$
	del cuerpo de funciones $k(X)$ que, tras tensorizar por $\sepcl{k}$, se escribe como suma directa de $\sepcl{k(X)}$.
	Por tanto, $K \cong k(X) \otimes_k L$ para una extensión de Galois $L/k$.
	Luego el recubrimiento de Galois finito $X_L \to X$ tiene la misma fibra genérica que $Y$, es decir, existe un abierto denso $U \subseteq X$
	tal que $X_L \times_X U \cong Y \times_X U$. Como $Y$ y $X_L$ son $U$-esquemas finitos localmente libres, se sigue que $Y \cong X_L$.
\end{proof}

\section{Sitios, haces y cohomologías}
\newcommand{\Cov}{{\rm Cov}}
\newcommand{\Cat}{\operatorname{Cat}}
\begin{mydef}
	Sea $\catC$ una categoría pequeña con productos fibrados.
	Fijado un objeto $U \in \Obj\catC$ y dado un par de conjuntos $\mathcal{S}_1 := \{ U_i \}_{i\in I}, \mathcal{S}_2 := \{ V_j \}_{j\in J}$
	de objetos de $\catC/U$, se dice que $\mathcal{S}_2$ es un \strong{refinamiento}\index{refinamiento} de $\mathcal{S}_1$ si para todo $i \in I$
	existe un $j \in J$ y una flecha $U_i \to V_j$ (de $\catC/U$).

	Se le llama una \strong{(pre)topología de Grothendieck}\index{topología!(de Grothendieck)} a una familia $J := \{ \Cov_U \}_{U \in \Obj\catC}$,
	tal que para cada objeto $U \in \Obj\catC$, se cumple que $\Cov_U$ es una familia de conjuntos de flechas, llamados \strong{cubrimientos}\index{cubrimiento}
	que satisface lo siguiente:
	\begin{enumerate}[{COV}1., left=.8em]
		\item Para todo isomorfismo $V \to U$ se cumple que $\{ V \to U \}$ es un cubrimiento.
		\item El refinamiento de un cubrimiento es también un cubrimiento.
		% \item Si $\mathcal{S}_1 \in \Cov_U$ y $\mathcal{S}_2$ es un refinamiento de $\mathcal{S}_1$, entonces $\mathcal{S}_2 \in \Cov_U$.
		\item El cambio de base de un cubrimiento induce un cubrimiento.
			Vale decir, dado un cubrimiento $\mathcal{S} := \{ U_i \to U \}_{i\in I} \in \Cov_U$ y una flecha $V \to U$,
			la familia $\mathcal{S} \times_U V := \{ U_i \times_U V \to V \}_{i\in I} \in \Cov_V$.
		\item Sea $\mathcal{S}_1 := \{ U_i \to U \}_{i\in I} \in \Cov_U$ y $\mathcal{S}_2 := \{ V_j \to U \}_{j\in J}$ una familia de flechas.
			Si para cada $U_i$ se cumple que $\mathcal{S}_2 \times_U U_i \in \Cov_{U_i}$, entonces $\mathcal{S}_2 \in \Cov_U$.
	\end{enumerate}
	Un \strong{sitio}\index{sitio} es un par $X := (\catC, J)$, donde $J$ es una (pre)topología de Grothendieck sobre $\catC$.
	Denotamos $\Cat(X) := \catC$.
\end{mydef}
La definición de una topología de Grothendieck es otra, pero uno puede probar que una pretopología induce de manera única una topología.

\begin{ex}
	Sea $X$ un espacio topológico.
	Definimos el \textit{sitio topológico} $X_{\rm top}$ como el sitio con $\Cat(X_{\rm top}) := \Open(X)$
	y donde los cubrimientos coinciden con la definición previa de <<cubrimiento>>.
\end{ex}
Éste es un ejemplo sencillo, pero queremos ampliar un poco más la definición, en particular a sitios que capturen ciertas propiedades de morfismos de esquemas.

\begin{mydef}
	Sea $\catC$ una categoría concreta (e.g., $\mathsf{Sch}$).
	Se dice que una familia de flechas de codominio fijo $\{ f_i\colon V_i \to U \}$ es \strong{colectivamente suprayectiva}\index{colectivamente!suprayectiva}
	si $U = \bigcup_{i\in I} f_i[V_i]$.
\end{mydef}

% La solución que vamos a aplicar es un tanto radical.
% Las categorías subyacentes casi siempre serán $\mathsf{Sch}/S$ con condiciones sobre los cubrimientos:
\newcommand{ \zarsite}[1]{#1_{\rm Zar}}
\newcommand{  \etsite}[1]{#1_{\text{ét}}}
\begin{exn}
	Sea $S$ un esquema.
	Un \strong{cubrimiento por Zariski abiertos} (resp. \strong{cubrimiento étale}\index{cubrimiento!étale}) es una familia $\{ \varphi_i \colon U_i \to X \}$
	de encajes abiertos (resp. morfismos étale) que es colectivamente suprayectiva.

	El \strong{gran sitio de Zariski} (resp. \strong{gran sitio étale}), denotado $\zarsite{(\mathsf{Sch}/S)}$ (resp. $\etsite{(\mathsf{Sch}/S)}$),
	es aquél que tiene por categoría subyacente a $\mathsf{Sch}/S$ con los cubrimientos por Zariski abiertos (resp. cubrimientos étale).

	El \strong{(pequeño) sitio de Zariski}\index{sitio!de Zariski}, denotado $\zarsite S$, es aquél que tiene por categoría los
	subesquemas abiertos de $S$ (por objetos) con los encajes abiertos (por flechas), y cuyos cubrimientos son los cubrimientos por Zariski abiertos.
	El \strong{(pequeño) sitio étale}\index{sitio!étale}, denotado $\etsite S$, es aquél que tiene por categoría
	los $S$-esquemas étale (por objetos) con los morfismos de esquemas (por flechas), y cuyos cubrimientos son los cubrimientos étale.
	% En general tendremos dos clases de sitios relacionados a $S$, los <<pequeños>> y los <<grandes>>,
	% donde el sitio grande tendrá por categoría subyacente a $\mathsf{Sch}/S$.
	% \begin{enumerate}
	% 	\item Decimos que un \strong{cubrimiento por Zariski abiertos} es una familia de morfismos de esquemas $\{ \varphi_i \colon U_i \to X \}$
	% 		que son colectivamente suprayectivos y tales que cada uno es un encaje abierto.
	% 		El \strong{gran sitio de Zariski}, denotado $\zarsite{(\mathsf{Sch}/S)}$, es aquél que tiene por categoría subyacente a $\mathsf{Sch}/S$
	% 		con los cubrimientos por Zariski abiertos.
	% 		El \strong{(pequeño) sitio de Zariski}\index{sitio!de Zariski}, denotado $S_{\rm Zar}$, es el subsitio de $\zarsite{(\mathsf{Sch}/S)}$
	% 		cuya categoría son los subesquemas abiertos de $S$ (por objetos) con los encajes abiertos (por flechas).
	% 	\item Un \strong{cubrimiento étale}\index{cubrimiento!étale} es una familia de morfismos étale $\{ \varphi_i \colon U_i \to X \}$
	% 		que son colectivamente suprayectivos.
	% 		El \strong{gran sitio étale}, denotado $\etsite{(\mathsf{Sch}/S)}$, tiene por categoría subyacente a $\mathsf{Sch}/S$
	% 		con los cubrimientos étale.
	% 	\item Un \strong{cubrimiento fpqc}\index{cubrimiento!fpqc}%
	% 		es una familia colectivamente suprayectiva de morfismos planos $\{ \varphi_i \colon X_i \to Y \}$ que satisfacen lo siguiente:
	% 		\paragraph{(fpqc):} Para cada abierto afín $U \subseteq Y$, existen finitos abiertos afines $\{ U_j \subseteq X_{i_j} \}_{j \in J}$
	% 		tales que la familia $\{ \varphi_{i_j}|_{U_j} \colon U_j \to U \}_{j\in J}$ es colectivamente suprayectiva.
	% 		El \strong{(gran) sitio fpqc}\index{sitio!fpqc}, denotado $\fpqcsite X$
	% 		\qedhere
	% \end{enumerate}
\end{exn}

Vamos a ver más ejemplos, pero antes un par de definiciones:
\begin{mydef}
	Un morfismo de esquemas $f \colon X \to Y$ se dice \strong{localmente de presentación finita}\index{localmente!de presentación finita (morfismo)}
	si para todo punto $x \in X$ existen un par de entornos afines $x \in U$ y $f(x) \in V$ tales que $f[U] \subseteq V$,
	y tales que $\mathscr{O}_X(U)$ es una $\mathscr{O}_Y(V)$-álgebra de presentación finita.

	Un morfismo de esquemas $f \colon X \to Y$ se dice \strong{fppf}\index{morfismo!fppf}%
	\footnote{Del fr., abrev. de \textit{fidèlement plat et presentation fini}.}
	si es fielmente plano y localmente de presentación finita.
	Se dice que $f$ es \strong{fpqc}\index{morfismo!fpqc}%
	\footnote{Del fr., abrev. de \textit{fidèlement plat et quasi-compact:} fielmente plano y (cuasi)compacto.}
	si es fielmente plano y todo abierto compacto de $Y$ es la imagen de un abierto compacto de $X$.
\end{mydef}
\begin{ex}
	Si $X, Y$ son localmente noetherianos, entonces para un morfismo $X \to Y$ ser <<localmente de presentación finita>> equivale a
	ser <<localmente de tipo finito>>.
\end{ex}

\newcommand{\fpqcsite}[1]{#1_{\rm fpqc}}
\newcommand{\fppfsite}[1]{#1_{\rm fppf}}
Ahora dos ejemplos más:
\begin{exn}
	Sea $S$ un esquema.
	Un \strong{cubrimiento fppf}\index{cubrimiento!fppf} (resp. \strong{cubrimiento fpqc}\index{cubrimiento!fpqc}) es una familia
	$\{ \varphi_i \colon Y_i \to Y \}_{i\in I}$ tal que $\coprod_{i\in I} Y_i \to X$ es un morfismo fppf (resp. fpqc).

	El \strong{(gran) sitio fppf}\index{sitio!fppf} (resp. \strong{(gran) sitio fpqc}\index{sitio!fpqc}), denotado $\fppfsite S$
	(resp. $\fpqcsite S$), es aquel que tiene por categoría subyacente $\mathsf{Sch}/S$ y cuyos cubrimientos son los fppf (resp. fpqc).
\end{exn}
En general respecto a los sitios de Zariski y étale nos referiremos a ellos como los sitios pequeños,
mientras que en los sitios fppf y fpqc nos referiremos a los sitios grandes.

La noción de sitios tiene como finalidad poder establecer la siguiente definición:
\begin{mydef}
	Sea $X$ un sitio.
	Un \strong{prehaz} (con valores en $\catC$) es un funtor $\mathscr{F} \colon \Cat(X)^{\rm op} \to \catC$.
	Se dice que un prehaz $\mathscr{F}$ es un \strong{haz} si la sucesión
	\begin{equation}
		\begin{tikzcd}[row sep=large]
			\mathscr{F}(U) \rar["\alpha"] & \prod_{i} \mathscr{F}(U_i) \rar["\beta", shift left] \rar["\gamma"', shift right]
						      & \prod_{i,j} \mathscr{F}(U_i \times_U U_j)
		\end{tikzcd}
		\label{cd:sheaf_site}
	\end{equation}
	es exacta para todo cubrimiento $\{ U_i \to U \}$,
	donde las flechas son las mismas que en \eqref{cd:sheaf_equalizer}.

	Los prehaces (resp. haces) sobre $X$ con valores en $\catC$ conforman una categoría denotada $\mathsf{PSh}(X; \catC)$ (resp. $\mathsf{Sh}(X; \catC)$).
	Cuando $\catC = \mathsf{Ab}$ decimos que los objetos de $\mathsf{PSh}(X; \catC)$ (resp. $\mathsf{Sh}(X; \catC)$) se dicen prehaces (resp. haces) abelianos.
\end{mydef}
\begin{ex}
	Si $X$ es un espacio topológico, entonces las nociones de haz sobre $X$ (como haz sobre un espacio)
	y haz sobre $X_{\rm top}$ (como haz sobre un sitio) coinciden.
\end{ex}
La condición \eqref{cd:sheaf_site} se verifica sobre cubrimientos, por tanto, mientras más \emph{fina} sea la topología de Grothendieck
(<<tiene más abiertos>>), más verificaciones.
% En consecuencia, si tenemos un prehaz sobre $\mathsf{Sch}/S$ que es un haz en la topología fpqc, entonces será un haz en las topologías fppf, étale y de Zariski.

Nótese que, por la proposición~\ref{thm:unram_et_prop}, todo cubrimiento por Zariski abiertos es étale;
en consecuencia, si nos restringimos a los sitios grandes (para tener la misma categoría subyacente), la topología étale es más fina que la de Zariski.
\begin{mydef}
	Sean $X, Y$ un par de sitios.
	Una \strong{aplicación continua}\index{aplicación!continua (sitios)} $f \colon X \to Y$ es un funtor $F := f^t \colon \Cat(Y) \to \Cat(X)$
	que preserva cubrimientos, vale decir:
	\begin{enumerate}[{AC}1.]
		\item Para todo cubrimiento $\{ V_i \morf{\varphi_i} V \}_{i\in I}$ en $Y$,
			se cumple que $\{ FV_i \morf{F(\varphi_i)} FV \}_{i\in I}$ es un cubrimiento en $X$.
		\item Para todo cubrimiento $\{ V_i \morf{\varphi_i} V \}_{i\in I}$ y toda flecha $W \to V$ en $Y$,
			se cumple que $F(V_i \times_V W) \cong FV_i \times_{FV} FW$.
	\end{enumerate}
\end{mydef}
\begin{ex}
	Sean $X, Y$ un par de espacios topológicos y sea $f \colon X \to Y$ una función continua.
	Definimos $f_{\rm top} \colon X_{\rm top} \to Y_{\rm top}$ como el funtor $V \mapsto f^{-1}[V]$; entonces $f_{\rm top}$ es continuo.
\end{ex}

\warn
Empleamos la nomenclatura <<aplicación continua>> (siguiendo a \citeauthor{poonen:rational}~\cite{poonen:rational}) en lugar de <<morfismo de sitios>>,
pues éste último lo reservamos para otros propósitos (\cite{stacks}).

Sobre un conjunto $X$, dadas dos topologías $\tau_1, \tau_2$, se cumple que $\tau_1$ es más fina que $\tau_2$ es equivalente a que la identidad
$\Id \colon (X, \tau_1) \to (X, \tau_2)$ sea continua.
Análogamente, sobre una categoría fija $\catC$, dos topologías de Grothendieck $\mathcal{T}_1, \mathcal{T}_2$ satisfacen que $\mathcal{T}_1$ es más fina
que $\mathcal{T}_2$ si $\Id \colon (\catC, \mathcal{T}_1) \to (\catC, \mathcal{T}_2)$ es continua.
Así, tenemos lo siguiente:
\begin{prop}
	Sea $S$ un esquema, entonces son aplicaciones continuas:
	\begin{center}
		\begin{tikzcd}
			\fpqcsite S \rar & \fppfsite S \rar & \etsite{(\mathsf{Sch}/S)} \rar & \zarsite{(\mathsf{Sch}/S)}.
		\end{tikzcd}
	\end{center}
	Vale decir, todo cubrimiento por Zariski abiertos es étale, todo cubrimiento étale es fppf, y todo cubrimiento fppf es fpqc.
\end{prop}
\begin{cor}
	Sea $S$ un esquema y sea $\mathscr{F} \colon (\mathsf{Sch}/S)^{\rm op} \to \mathsf{Set}$ un funtor.
	Si $\mathscr{F}$ es un haz en la topología fpqc, entonces también lo es en las topologías fppf, étale y de Zariski.
\end{cor}

Nótese que el recíproco no es cierto.
Así que convendría saber un criterio para cuando un funtor es un haz en la
topología fpqc y afortunadamente tenemos el siguiente corolario de la teoría de descenso:
\begin{thm}\label{thm:repr_fun_are_fpqc_sh}
	Sea $S$ un esquema.
	Todo funtor representable $(\mathsf{Sch}/S)^{\rm op} \to \mathsf{Set}$ es un haz en la topología fpqc.
\end{thm}

% La situación es un tanto delicada.
% Por un lado, hay que elegir una categoría cuyos elementos sean <<candidatos a abiertos>> del esquema, y por otro lado, hay que elegir cubrimientos para los
% abiertos.
% Vale decir, no nos basta simplemente que 

% \begin{mydef}
% 	Sea $S$ un esquema.
% 	Una familia de morfismos de esquemas $\mathcal{F} := \{ \varphi_i \colon X_i \to S \}$ se dice un \strong{cubrimiento fpqc}\index{cubrimiento!fpqc} si:
% 	\begin{enumerate}
% 		\item Cada $\varphi_i$ es un morfismo plano y $\mathcal{F}$ es colectivamente suprayectiva.
% 		\item Para cada abierto afín $U \subseteq S$, existen finitos abiertos afines $\{ U_j \subseteq X_{i_j} \}_{j\in J}$,
% 			tales que $U = \bigcup_{j\in J} \varphi_{i_j}[U_j]$.
% 	\end{enumerate}
% 	El \strong{sitio fpqc grande}\index{sitio!fpqc!grande} es el que tiene por categoría subyacente a los esquemas,
% 	y por cubrimientos a los cubrimientos fpqc.
% \end{mydef}
% \begin{ex}
% 	Son cubrimientos fpqc:
% 	\begin{enumerate}
% 		\item Todo cubrimiento de Zariski abiertos.
% 		\item Todo cubrimiento étale.
% 		\item Un morfismo $\{ \varphi^a\colon \Spec B \to \Spec A \}$ es un cubrimiento fpqc syss $\varphi \colon A \to B$ es fielmente plano.
% 		\item Si $\{ f \colon X \to Y \}$ es un morfismo plano, suprayectivo y compacto, entonces $\{ f \colon X \to Y \}$ es un cubrimiento fpqc.
% 	\end{enumerate}
% \end{ex}

% La razón para estudiar la topología fpqc es que, por el ejemplo anterior, es más fina que la topología de Zariski o la topología étale sobre $\mathsf{Sch}/S$.
% Así que, veremos cómo estudiarla apropiadamente:

\begin{mydef}
	Sea $X$ un esquema y $\catC \subseteq \mathsf{Sch}/X$.
	Sea $\mathscr{F}$ un $\mathscr{O}_X$-módulo cuasicoherente (en particular, un haz sobre $\zarsite X$).
	Se define un prehaz $\mathscr{F}_\catC$ dado por:
	$$ \Gamma(U, \mathscr{F}_\catC) := \Gamma(U, p^*\mathscr{F}) $$
	para cada objeto $U \morf{p} X$ en $\catC$.
\end{mydef}
\begin{prop}
	Sea $X$ un esquema y $\mathscr{F}$ un $\mathscr{O}_X$-módulo cuasicoherente.
	Entonces $\mathscr{F}_\tau$ es un haz sobre $X_\tau$ para todo $\tau \in \{ \rm fpqc, fppf, \text{ét}, Zar \}$.
\end{prop}

\begin{prop}
	Sea $X$ un sitio y $\mathscr{F} \colon \Cat(X)^{\rm op} \to \catC$ un prehaz de conjuntos o de grupos abelianos.
	Entonces $\mathscr{F}$ posee una hazificación $\mathscr{F}^+$ y este determina un funtor.
\end{prop}
La demostración es esencialmente la misma que detallamos anteriormente.
De hecho, las construcciones relativas a haces sobre sitios es análoga a la de haces sobre espacios topológicos, por lo que
dejaremos los detalles al lector.

\begin{mydef}
	Sea $X$ un sitio y $\alpha \colon \mathscr{F \to G}$ un morfismo de prehaces abelianos sobre $X$.
	Se definen los siguientes prehaces:
	\begin{gather*}
		\Gamma(U, \ker\alpha) := \ker(\alpha_U), \qquad \Gamma(U, (\Img\alpha)^-) := \Img(\alpha_U), \\
		\Gamma(U, (\coker\alpha)^-) := \coker(\alpha_U).
	\end{gather*}
\end{mydef}
\begin{prop}
	Sea $X$ un sitio y $\alpha \colon \mathscr{F \to G}$ un morfismo de haces abelianos sobre $X$.
	Entonces $\ker\alpha$ es un haz sobre $X$, y
	$$ \Img\alpha := \big( (\Img\alpha)^- \big)^+, \qquad \coker\alpha := \big( (\coker\alpha)^- \big)^+ $$
	son haces sobre $X$; más aún, efectivamente son el núcleo, la imagen y el conúcleo en $\mathsf{Sh}(X; \mathsf{Ab})$.
\end{prop}

\begin{prop}
	Sea $X$ un sitio y 
	\begin{tikzcd}[cramped, sep=small]
		0 \rar & \mathscr{F} \rar["\alpha"] & \mathscr{G} \rar["\beta"] & \mathscr{H}
	\end{tikzcd}
	una sucesión exacta de haces abelianos.
	Entonces para todo $U \in \Obj\Cat(X)$, la sucesión inducida
	\begin{center}
		\begin{tikzcd}
			0 \rar & \Gamma(U, \mathscr{F}) \rar["\alpha"] & \Gamma(U, \mathscr{G}) \rar["\beta"] & \Gamma(U, \mathscr{H})
		\end{tikzcd}
	\end{center}
	es exacta (en $\mathsf{Ab}$).
	Vale decir, el funtor $\Gamma(U, -) \colon \mathsf{Sh}(X; \mathsf{Ab}) \to \mathsf{Ab}$ es exacto por la izquierda.
\end{prop}

\begin{thm}
	Para todo sitio $X$, la categoría $\mathsf{Sh}(X; \mathsf{Ab})$ es abeliana y tiene suficientes inyectivos.
\end{thm}
\begin{proof}
	Mejor aún, la categoría es de Grothendieck por el Thm.~18.1.6 en \citeauthor{kashiwara:sheaves}~\cite[437]{kashiwara:sheaves}.
\end{proof}

Como corolario, todo funtor desde $\mathsf{Sh}(X; \mathsf{Ab})$ que sea exacto por la izquierda admite funtores derivados,
he aquí una lista de los principales ejemplos:
\begin{exn}
	Sean $X$ y $S_\tau$ un par de sitios, donde $S$ es un esquema y $\tau \in \{ \rm fpqc, fppf, \text{ét}, Zar \}$.
	\begin{enumerate}
		\item Para todo objeto $U \in \Cat(X)$,
			el funtor de secciones globales
			\[
				\Gamma(U, -) \colon \mathsf{Sh}(X; \mathsf{Ab}) \longrightarrow \mathsf{Ab}
			\]
			es exacto por la izquierda y su funtor derivado se denota $H_X^q(U, -) := \rder^p \Gamma(U, -)$.
			Cuando $X = S_\tau$, se denota $H_\tau^q(U, -) := H_{S_\tau}^q(U, -)$.
		\item El funtor inclusión $i \colon \mathsf{Sh}(X; \mathsf{Ab}) \to \mathsf{PSh}(X; \mathsf{Ab})$ es exacto por la izquierda
			y su funtor derivado se denota $\mathscr{H}^q(X, -) := \rder^q i$.
			Cuando $X = S_\tau$, de denota $\mathscr{H}^q_\tau(S, -) := \mathscr{H}^q(S_\tau, -)$.
		\item Para todo haz $\mathscr{F} \in \mathsf{Sh}(X; \mathsf{Ab})$,
			el funtor $\Hom(\mathscr{F}, -)$ es exacto por la izquierda
			y su funtor derivado se denota $\Ext_X^q(\mathscr{F}, -) := \rder^p\Hom_X(\mathscr{F}, -)$.
			Cuando $X = S_\tau$, se denota $\Ext_\tau^q(\mathscr{F}, -) := \Ext_{S_\tau}^q(\mathscr{F}_\tau, -)$.
		\item Para todo haz $\mathscr{F} \in \mathsf{Sh}(X; \mathsf{Ab})$,
			el haz Hom $\shHom_X(\mathscr{F}, -)$ es exacto por la izquierda y su funtor derivado se denota
			$\shExt_X^q(\mathscr{F}, -) := \rder^q \shHom_X(\mathscr{F}, -)$.
		\item Sea $f \colon X \to Y$ una aplicación continua de sitios.
			Entonces el funtor $f_* \colon \mathsf{Sh}(Y; \mathsf{Ab}) \to \mathsf{Sh}(X; \mathsf{Ab})$ es exacto por la izquierda.
			A su funtor derivado $\rder^q f_*(-)$ se le llaman \strong{imágenes directas superiores}.
			\qedhere
	\end{enumerate}
\end{exn}

\begin{cor}
	Sea $X$ un sitio y $U \in \Cat(X)$ un objeto.
	Para todo haz abeliano $\mathscr{F} \in \mathsf{Sh}(X; \mathsf{Ab})$, se cumplen:
	\begin{enumerate}
		\item $H_X^q(U, \mathscr{F}) := \Ext_X^q(\Z, \mathscr{F})$, donde $\Z$ denota el haz constante.
		\item $\Gamma( U, \mathscr{H}^q(X, \mathscr{F}) ) = H^q_X(U, \mathscr{F})$.
		\item Sea $X = S_\tau$, donde $S$ es un esquema y $\tau \in \{ \rm fpqc, fppf, \text{ét}, Zar \}$.
			Dado un homeomorfismo universal $\pi \colon S' \to S$ (de esquemas),
			se cumple que $\pi_*\colon \mathsf{Sh}(S_\tau; \mathsf{Ab}) \cong \mathsf{Sh}(S_\tau^\prime; \mathsf{Ab})$ es una
			equivalencia de categorías.
			Denotando $\mathscr{F}' := \pi_*\mathscr{F}$ se satisface entonces:
			\begin{gather*}
				H^q_\tau(S', \mathscr{F}') \cong H^q_\tau(S, \mathscr{F}), \qquad
				\Ext^q_\tau(S', \mathscr{F}') \cong \Ext^q_\tau(S, \mathscr{F}), \\
				\pi_*\mathscr{H}^q_\tau(S^\prime, \mathscr{F}') = \mathscr{H}^q_\tau(S, \mathscr{F}).
			\end{gather*}
	\end{enumerate}
\end{cor}

% \begin{mydef}
% 	Sea $X$ un sitio.
% 	Definimos el funtor
% 	\[
% 		H^q(X, -) \colon \mathsf{Sh}(X; \mathsf{Ab}) \longrightarrow \mathsf{Ab}
% 	\]
% 	como el $q$-ésimo funtor derivado derecho del funtor $\Gamma(X, -)$.
% 	Si $S$ es un esquema y $X = S_\tau$ (con ), entonces denotamos $H^q_\tau(S, -)$ para enfatizar
% 	la topología de Grothendieck.
% \end{mydef}
% Es decir, para toda sucesión exacta 
% \begin{tikzcd}[cramped, sep=small]
% 	0 \rar & \mathscr{F} \rar & \mathscr{G} \rar & \mathscr{H} \rar & 0
% \end{tikzcd}
% de haces abelianos sobre $S_\tau$, se induce la siguiente sucesión exacta larga:
% \begin{center}
% 	\includegraphics{cats/top_cohomology.pdf}
% \end{center}

\subsection{Cohomología de \v Cech}
Sea $X$ un sitio y sea $\mathcal{U} := \{ U_i \to U \}_{i\in I}$ un cubrimiento.
Para una tupla de índices $(i_0, \dots, i_p) \in I^{p+1}$ se define
$$ U_{i_0, \dots, i_p} := U_{i_0} \times_U U_{i_1} \times_U \cdots \times_U U_{i_p}. $$
Sea $(i_0, \dots, i_p) \in I^{p+1}$ una tupla y $j \in \{ 0, 1, \dots, p \}$; entonces obtenemos
una proyección $\rho_j \colon U_{i_0, \dots, i_p} \to U_{i_0, \dots, \hat{i}_j, \dots, i_p}$, donde <<$\hat{i}_j$>> denota borrar la $j$-ésima coordenada.
\begin{mydef}
	Se define
	$$ \check C^q(\mathcal{U}, \mathscr{F}) := \prod_{(i_0, \dots, i_p) \in I^{p+1}} \mathscr{F}(U_{i_0, \dots, i_p}), $$
	y definimos el homomorfismo de grupos:
	\begin{align*}
		\ud^q \colon \check C^q &\longrightarrow \check C^{q-1} \\
		s &\longmapsto \sum_{j=0}^{q} (-1)^j \rho_j(s)
	\end{align*}
	Así, es fácil verificar que $\big(\check C^\bullet(\mathcal{U}, \mathscr{F}), \ud^\bullet\big)$ es un complejo de cocadenas,
	llamado el \strong{complejo de \v Cech}\index{complejo!de \v Cech}.
	
	Siguiendo la terminología usual, los elementos de $\check C^q, \ker(\ud^q), \Img(\ud^{q-1})$
	se dicen \strong{cocadenas}, \strong{cociclos} y \strong{cobordes $q$-ésimos de \v Cech} resp.
	También se denomina \strong{$q$-ésimo grupo de cohomología de \v Cech}\index{grupo!de cohomología!de \v Cech} a:
	$$ \check H^q(\mathcal{U}, \mathscr{F}) := H^q( \check C^\bullet(\mathcal{U}, \mathscr{F}) ) = \frac{\ker(\ud^q)}{\Img(\ud^{q-1})}. $$
\end{mydef}

\begin{prop}
	Sea $\mathscr{F}$ un haz abeliano sobre un sitio $X$.
	Entonces, las flechas canónicas satisfacen lo siguiente:
	\begin{center}
		\begin{tikzcd}[row sep=tiny]
			\check H^0(U, \mathscr{F}) \rar["\sim"] & H^0(U, \mathscr{F}) = \mathscr{F}(U) \\
			\check H^1(U, \mathscr{F}) \rar["\sim"] & H^1(U, \mathscr{F})                  \\
			\check H^2(U, \mathscr{F}) \rar[hook]   & H^2(U, \mathscr{F})
		\end{tikzcd}
	\end{center}
\end{prop}
\begin{proof}
	Esto es una aplicación de la sucesión espectral de cohomología de \v Cech.
\end{proof}

% Ahora veamos la noción de \textit{cohomología de \v Cech}.
\begin{thm}[M. Artin]
	Sea $X$ un esquema compacto tal que cada conjunto finito de puntos de $X$
	esté contenido en un entorno afín (e.g., si $X$ es cuasiproyectivo sobre un esquema afín).
	Entonces, para todo haz abeliano $\etsite{\mathscr{F}}$ sobre $\etsite X$ tenemos que el homomorfismo canónico:
	\begin{center}
		\begin{tikzcd}[row sep=large]
			\etsite{\check H}^q(X, \mathscr{F}) \rar["\sim"] & \etsite H^q(X, \mathscr{F})
		\end{tikzcd}
	\end{center}
	es un isomorfismo para $q \in \N$.
\end{thm}

\begin{prop}[lema de Cartan]
	Sea $\mathscr{F}$ un haz abeliano sobre un esquema $S$ y sea $\mathscr{K}$ la clase de morfismos étale $U \to X$ que satisfacen lo siguiente:
	\begin{enumerate}
		\item Si $U, V \in \mathscr{K}$ entonces $U \times_X V \in \mathscr{K}$.
		\item Toda extensión étale $Y \to X$ admite un cubrimiento étale $\{ U_i \to Y \}_{i\in I}$ tal que cada composición $(U_i \to X) \in \mathscr{K}$.
		\item Se tiene que $\etsite{\check H}^i(U, \mathscr{F}) = 0$ para todo $U \in \mathscr{K}$ y todo $i > 0$.
	\end{enumerate}
	Entonces los homomorfismos canónicos
	\begin{center}
		\begin{tikzcd}[sep=large]
			\etsite{\check H}^i(X, \mathscr{F}) \rar["\sim"] & \etsite H^i(X, \mathscr{F})
		\end{tikzcd}
	\end{center}
	son isomorfismos.
\end{prop}

\begin{mydef}
	Sea $X$ un esquema.
	Un objeto acíclico en $\mathsf{Sh}(X_\tau, \mathsf{Ab})$ con $\tau \in \{ \rm fppf, fpqc, \text{ét}, Zar \}$ se dice un
	\strong{haz flácido}\index{haz!flácido} (en la topología respectiva).
\end{mydef}
Igual que antes, todo haz inyectivo es flácido, pero el recíproco no es cierto.

\begin{mydef}
	Sea $X$ un esquema y sea $\mathscr{L}_\tau$ un haz en la topología $\tau \in \{ \rm fppf, fpqc, \text{ét}, Zar \}$.
	Se dice que $\mathscr{L}_\tau$ es \strong{invertible}\index{haz!invertible} en la topología $\tau$ si $X$
	posee un cubrimiento $\{ U_i \to X \}$ en la respectiva topología tal que $\mathscr{L}|_{U_i} \cong \mathscr{O}_{U_i, \tau}$.

	La clase de haces invertibles (en la topología $\tau$) salvo isomorfismo de haces, se denota $\Pic_\tau X$.
	\nomenclature{$\Pic_\tau X$}{Grupo de Picard de $X$ con la topología $\tau$}
\end{mydef}
\begin{prop}
	Sea $X$ un esquema. Entonces:
	\begin{enumerate}
		\item $\zarsite H^0(X, \GG_m) \cong \etsite H^0(X, \GG_m) \cong \fppfsite H^0(X, \GG_m) \cong \Gamma(X, \mathscr{O}_X)^\times$.
		\item (Teorema de Hilbert 90) $\zarsite H^1(X, \GG_m) \cong \etsite H^1(X, \GG_m) \cong \fppfsite H^1(X, \GG_m) \cong \Pic X$.
	\end{enumerate}
\end{prop}

% \begin{mydef}
% 	Sea $\mathcal{U} := \{ t_i \colon T_i \to T \}_{i \in I}$ una familia de morfismos de esquemas.
% 	Un \strong{dato de descenso}\index{dato de descenso} es una colección $(\{ \mathscr{F}_i \}_i, \{ \varphi_{ij} \}_{i,j\in I})$ tal que:
% 	\begin{enumerate}[{Des}1.]
% 		\item Cada $\mathscr{F}_i$ es un haz cuasicoherente sobre $T_i$.
% 		\item Tenemos que $\varphi_{ij} \colon (\pi^{ij}_i)^* \mathscr{F}_i \to (\pi^{ij}_j)^* \mathscr{F}_j$ es un isomorfismo de haces;
% 			donde $T_{ij} := T_i\times_T T_j$ y $\pi^{ij}_i\colon T_{ij} \to T_i$ denota el morfismo canónico.
% 		\item \textbf{Condición de cociclos:}
% 			El siguiente diagrama conmuta:
% 			\begin{center}
% 				\begin{tikzcd}[row sep=large]
% 					(\pi^{ijk}_i)^* \mathscr{F}_i \drar["(\pi^{ijk}_{ik})^* \varphi_{ik}"'] \ar[rr, "(\pi^{ijk}_{ij})^* \varphi_{ij}"] & & (\pi^{ijk}_j)^* \mathscr{F}_j \ular["(\pi^{ijk}_{jk})^* \varphi_{jk}"] \\
% 					{} & (\pi^{ijk}_k)^* \mathscr{F}_k
% 				\end{tikzcd}
% 			\end{center}
% 	\end{enumerate}
% \end{mydef}

\subsection{Puntos geométricos y entornos étale}
\begin{mydef}
	Sea $X$ un esquema.
	Un \strong{punto geométrico}\index{punto!geométrico} de $X$ es un punto $k$-valuado (i.e., un morfismo $\Spec k \to X$),
	donde $k$ es un cuerpo separablemente cerrado.
	Se dice que $U \to X$ es un \strong{entorno étale}\index{entorno!étale} de un punto $k$-valuado $x \in X(k)$ si $U \to X$ es un morfismo étale
	y se tiene el siguiente diagrama conmutativo:
	\begin{center}
		\begin{tikzcd}[row sep=large]
			U \rar                        & X \\
			\Spec k \uar["y"] \urar["x"']
		\end{tikzcd}
	\end{center}
\end{mydef}
\begin{lem}
	Sea $X$ un esquema y $x$ un punto geométrico.
	Los entornos étale de $x$ forman un categoría filtrada.
\end{lem}

\begin{mydef}
	Sea $X$ un esquema y $x$ un punto geométrico.
	El \strong{anillo local estricto}\index{anillo!local!estricto} en $x$ es el límite directo
	$$ \mathscr{O}_{\etsite X, x} := \limdir_U \etsite\Gamma(U, \mathscr{O}_X), $$
	donde $U$ recorre los entornos étale de $x$.

	Más generalmente, si $\mathscr{F}$ es un haz sobre $\etsite X$ y $x$ es un punto geométrico de $X$,
	entonces se define la \strong{fibra}\index{fibra!(haz étale)} en $x$ como:
	$$ \mathscr{F}_{\text{ét}, x} := \limdir_U \etsite\Gamma(U, \mathscr{F}), $$
	donde $U$ recorre los entornos étale de $x$.
\end{mydef}

Ahora se sigue lo siguiente:
\begin{prop}
	Sea $X$ un esquema noetheriano.
	\begin{enumerate}
		\item Sea $\varphi \colon \mathscr{F} \to \mathscr{G}$ un morfismo de haces abelianos sobre $\etsite X$. 
			Entonces $\varphi$ es un isomorfismo syss para todo punto geométrico $x$ tenemos que $\varphi_x \colon \mathscr{F}_{\text{ét}, x}
			\to \mathscr{G}_{\text{ét}, x}$ es un isomorfismo.
		\item Sea $\mathscr{S \colon F \morf{\varphi} G \morf{\psi} H}$ una sucesión de haces abelianos sobre $\etsite X$.
			Entonces $\mathscr{S}$ es exacta (en $\mathsf{Sh}(\etsite X, \mathsf{Ab})$) syss para todo punto geométrico $x$ tenemos
			que $\mathscr{S}_{\text{ét}, x}$ es exacta (en $\mathsf{Ab}$).
	\end{enumerate}
	Más aún, si $X/k$ es un esquema algebraico podemos hacer las verificaciones sobre puntos geométricos cuya imagen sea cerrada.
\end{prop}

\warn
Al contrario de lo que sucedía antes, ésta proposición sí depende de que el sitio sea una topología étale sobre un esquema noetheriano
y es falso en un sitio general (¿cuál sería la definición de un punto geométrico?).
A ésta propiedad a veces se le llama \textit{tener suficientes puntos}.

\section{Anillos henselianos}
Recuérdese que un polinomio $f \in A[t_1, \dots, t_n]$ se dice \strong{primitivo}\index{primitivo (polinomio)} si sus coeficientes son coprimos en conjunto,
vale decir, si el ideal en $A$ generado por sus coeficientes es $(1)$.

\begin{lem}
	Sea $A$ un anillo y sea $f \in B := A[t_1, \dots, t_n]$.
	\begin{enumerate}
		\item Si $f$ es primitivo, entonces es un elemento regular de $B$ y $B/(f)$ es una $A$-álgebra plana.
		\item Supongamos que $n = 1$.
			Entonces $f$ es primitivo syss $\Spec(A[t]/(f)) \to \Spec A$ tiene fibras finitas.
	\end{enumerate}
\end{lem}
\begin{proof}
	\begin{enumerate}
		\item Nótese que $f$ es primitivo syss para todo $\mathfrak{p} \in \Spec A$ se cumple que $f \mod{\mathfrak{p}} \ne 0$.
			Veamos el endomorfismo (de $A$-módulos) $\times f \colon B \to B$ y sea $K := \ker(\times f) = B[f]$ la $f$-torsión.
			Como $B$ es de presentación finita y plano sobre $A$, entonces podemos tensorizar $\otimes\kk(\mathfrak{p})$ para todo $\mathfrak{p} \in
			\Spec A$, y vemos que $K \otimes \kk(\mathfrak{p}) = 0$, por lo que $K = 0$.
			Así $\times f$ es inyectivo y el conúcleo $B/(f)$ es plano sobre $A$ (por la sucesión exacta en $\Tor$).
		\item Basta notar que exigir que las fibras de $\Spec(A[t]/(f)) \to \Spec A$ sean finitas equivale a ver que
			$\kk(\mathfrak{p})[t]/(f \mod{\mathfrak{p}})$ sea un $\kk(\mathfrak{p})$-espacio vectorial de dimensión finita,
			lo que equivale a que $f \mod{\mathfrak{p}} \ne 0$.
			\qedhere
	\end{enumerate}
\end{proof}

\begin{prop}
	Sea $(A, \mathfrak{m}, k)$ un anillo local, y denotemos por $s := x_{\mathfrak{m}} \in \Spec A =: S$.
	Son equivalentes:
	\begin{enumerate}
		\item Toda $A$-álgebra finitamente generada (como módulo) es un producto de anillos locales.
		\item Para todo $f \in A[t]$ mónico, la $A$-álgebra $A[t]/(f)$ es un producto de anillos locales.
		\item Sea $X$ un $S$-esquema separado de tipo finito.
			Entonces existe una partición $X = Y \amalg X_1 \amalg \cdots \amalg X_r$ en subesquemas abiertos y cerrados,
			tal que toda componente irreducible de $Y_s$ tiene dimensión $\ge 1$ y tal que cada $X_i = \Spec(B_i)$,
			donde cada $B_i$ es una $A$-álgebra local finitamente generada.
		\item Dado $f \in A[t]$ y una factorización $f \equiv g_0\cdot h_0 \pmod{k[t]}$ tal que $g_0$ sea mónico y
			$g_0, h_0 \in k[t]$ son coprimos, existen $g, h \in A[t]$ tales que $g$ es mónico, $f = g\cdot h$ y $g \equiv g_0, h \equiv h_0 \pmod{k[t]}$.
	\end{enumerate}
\end{prop}

\begin{prop}
	Sea $f \colon X \to S$ un morfismo suave.
	Sea $s \in S$ y sea $x \in X_s$ un punto cerrado de la fibra tal que la extensión $\kk(x)/\kk(s)$ sea separable.
	Entonces existe un entorno abierto $U$ de $s$ y un subesquema $Z \subseteq f^{-1}[U]$ con $x \in Z$ tal que $f|_Z \colon Z \to U$ es étale.
	\begin{center}
		\begin{tikzcd}[sep=small]
			\Spec\kk(x) \drar \ar[dd, closed] \ar[dddd, bend right, "\text{étale}"'] \\
			{} & Z \drar[hook] \\
			X_s \ar[dd] \ar[rr] & & X \ar[dd, "f"', "\text{suave}"] \\
			{} & U \ar["\text{étale}"', crossing over, near start, from=uu] \drar[open] \\
			\Spec\kk(s) \ar[rr] \urar & & S
		\end{tikzcd}
	\end{center}
\end{prop}
\begin{cor}
	Sea $f\colon X \to S$ un morfismo sobreyectivo suave.
	Existe un morfismo étale sobreyectivo $T \to S$ y una sección $\sigma \colon T \to X_T$
	(i.e., tal que $\sigma \circ f_T = \Id_T$).
\end{cor}

\begin{prop}
	Sea $f \colon X \to Y$ un morfismo de esquemas.
	Son equivalentes:
	\begin{enumerate}
		\item $f$ es afín y $f_*\mathscr{O}_X$ es un $\mathscr{O}_Y$-módulo localmente libre de rango finito.
		\item $f$ es un morfismo finito, plano y de presentación finita.
		\item Para todo abierto afín $V = \Spec A \subseteq Y$ su preimagen es afín $f^{-1}[V] = \Spec B$
			y la $A$-álgebra $B$ es un $A$-módulo finitamente generado proyectivo.
	\end{enumerate}
	En cuyo caso, decimos que $f$ es un \strong{morfismo finito localmente libre}\index{morfismo!finito!localmente libre}.
\end{prop}
Si $Y$ es un esquema localmente noetheriano, entonces por el inciso 2, un morfismo finito es localmente libre syss es plano.
Nótese que como todo morfismo finito localmente libre es plano y de presentación finita, entonces es un morfismo universalmente abierto;
y, por ser un morfismo finito, también es un morfismo universalmente cerrado.

Así pues, si $Y$ es conexo y $X \ne \emptyset$, entonces es sobreyectivo.

\begin{lem}
	Sea $S$ un esquema conexo y sea $\pi \colon X \to S$ un morfismo finito localmete libre (e.g., morfismo finito étale).
	Entonces $X$ es la unión disjunta finita de subesquemas conexos abiertos y cerrados.
\end{lem}
\begin{proof}
	Basta probarlo por inducción sobre $d := \deg\pi$.
	Para $d = 0$ se tiene que $X = \emptyset$, y para $d = 1$ tenemos que $X \cong S$, así que están listos.
	Si $X$ no es conexo, entonces $X = X_1 \amalg X_2$ y $\deg\pi = \deg(\pi|_{X_1}) + \deg(\pi|_{X_2})$.
\end{proof}
\begin{mydef}
	Sea $S$ un esquema y $\overline{s} \colon \Spec\Omega \to S$ un punto geométrico.
	Dado un $S$-esquema finito étale $X$, podemos definir la \strong{fibra}\index{fibra!(punto geométrico)} $X_{\overline{s}} := X \times_S \Spec\Omega$
	el cual es una unión disjunta de copias de $\Spec\Omega$, de modo que viene completamente determinado por su conjunto $\Hom_S(\Spec\Omega, X)$.
	El \strong{funtor de fibras}\index{funtor!de fibras} es
	$$ \mathcal{F}_{\overline{s}} \colon \mathsf{F\acute Et}/S \longrightarrow \mathsf{Set}, \qquad
	X \mapsto \Hom_S(\Spec\Omega, X) = X_{\overline{s}}. $$
	Si $S$ es conexo, la subcategoría plena de $\mathsf{Fun}(\mathsf{F\acute Et}/S, \mathsf{Set})$ cuyos objetos son isomorfos a
	funtores de fibras $\mathcal{F}_{\overline{s}}$, donde $\overline{s} \colon \Spec\Omega \to S$ recorre los puntos geométricos de $S$,
	se denomina el \strong{grupoide fundamental}\index{grupoide fundamental} de $S$ y se denota $\Pi_{\rm alg}(S)$.
\end{mydef}

% \begin{lem}
% 	Sean $f \colon Y \to X$ y $g \colon X \to S$ un par de morfismos.
% 	Si $g$ es un morfismo separado y $f\circ g$ es un morfismo (étale) finito, entonces $f$ también es (étale) finito.
% \end{lem}
% \begin{proof}
% 	La diagonal 
% 	\begin{tikzcd}[cramped, sep=small]
% 		\Delta_{X/S} \colon X \rar[closed] & X\times_S X
% 	\end{tikzcd}
% 	es un encaje cerrado por definición y, en particular, es un morfismo finito.
% 	...
% \end{proof}
Recuérdese que los encajes cerrados son morfismos finitos, que estos son estables salvo composición y cambio de base, de modo que admiten cancelación izquierda
por morfismos separados.
Los morfismos étale también poseen cancelación por morfismos separados, aunque las razones son distintas.
\begin{prop}
	Sea $f \colon X \to S$ un cubrimiento étale finito, y sea $s \colon S \to X$ una sección suya (i.e., $s\circ f = \Id_S$).
	Entonces $s$ induce un isomorfismo de $S$ con un subesquema abierto y cerrado de $X$.
\end{prop}
\begin{proof}
	Como $s\circ f = \Id_S$ es étale finito y $f$ es separado, entonces $s$ también es étale finito,
	luego es un morfismo universalmente abierto y cerrado por las observaciones anteriores.
\end{proof}
\begin{cor}
	Sea $Z$ un $S$-esquema conexo, $X$ un $S$-esquema étale finito y sean $f, g \colon Z \to X$ un par de $S$-morfismos.
	Si existe un punto geométrico $\overline{z} \colon \Spec k \to Z$ tal que $\overline{z}\circ f = \overline{z}\circ g$, entonces $f = g$.
\end{cor}
\begin{proof}
	Haciendo cambio de base $X_Z \to Z$ es también un morfismo étale finito, por lo que podemos suponer que $Z = S$.
	Así, $S$ es conexo y $X$ posee dos $S$-secciones, por lo que inducen un isomorfismo $f \colon S \cong X_1$ y $g \colon S \cong X_2$,
	donde $X_1$ y $X_2$ son componentes conexas de $X$.
	Como $f(\overline{z}) = g(\overline{z})$, viendo la imagen de $\overline{z}$, comprobamos que $z \in X_1 \cap X_2$, por lo que $X_1 = X_2$
	y $f = g$ ya que $S$ posee un único $S$-endomorfismo.
\end{proof}

\begin{cor}
	Sea $X$ un $S$-esquema étale finito y conexo.
	Entonces, fijando un punto geométrico $\overline{s} \in S(\Omega)$,
	el grupo $\Aut(X/S)$ actúa fielmente sobre la fibra geométrica de $X_{\overline{s}}$ y, en particular, es finito.
\end{cor}
\begin{proof}
	Basta aplicar el corolario anterior con $Z = X$.
\end{proof}

% Es claro que los morfismos finitos localmente libres se preservan salvo composición y cambio de base.

% Sea 
% \begin{lem}
% 	Sea $\pi \colon X \to S$ un morfismo finito étale.
% \end{lem}

% \section{Cambio de base propio}
% ...

\section*{Notas históricas}
La teoría del descenso plano, que es indudablemente una de las herramientas centrales de la geometría algebraica contemporánea,
fue concebida en el seminario de Geometría Algebraica Fundamental y varias de las exposiciones han rápidamente adquirido el estatus
de lecturas obligatorias en el tópico.
El descenso fpqc es quizá una de las técnicas más famosas de Grothendieck y, para fortuna del lector, están bastante bien expuestas en varias fuentes.

La definición de <<topología de Grothendieck>> (y, por consiguiente, la de <<sitio>>) es original de M.~Artin.
El grupo fundamental étale fue estudiado en detalle en el primer
Seminario de Geometría Algebraica (SGA) del Bosque Marie dirigido por Grothendieck \cite{sga1};
aquí mismo fueron definidos por vez primera la noción de morfismos étale y no ramificados,
y el formalismo de las categorías de Galois (vid.\ la exposición~V.4-V.6).


\input{hilbert.tex}

\input{esquema_en_grupos.tex}

\input{neron.tex}

\input{curvas.tex}

\chapter{Superficies regulares}

\section{Teoría de intersección}
\begin{mydef}
	Sea $X$ un esquema localmente noetheriano, conexo y factorial de dimensión 2.
	Sean $C, D$ dos divisores de Cartier efectivos sin componentes irreducibles en común y sea $x \in X$ un punto cerrado.
	En un entorno de $x$ se tiene que $\Supp C \cap \Supp D \in \{ \emptyset, \{ x \} \}$, por lo tanto se tiene que
	$$ \mathfrak{m}_{X, x} \subseteq \rad\big( \mathscr{O}_X(-C)_x + \mathscr{O}_X(-D)_x \big), $$
	así pues, $A := \mathscr{O}_{X, x}/\big( \mathscr{O}_X(-C)_x + \mathscr{O}_X(-D)_x \big)$ es un anillo artiniano.
	Definimos la \strong{multiplicidad de intersección}\index{multiplicidad!de intersección} de $C, D$ en $x$:
	$$ i_x(C, D) := \ell_{\mathscr{O}_{X, x}}\left( \frac{\mathscr{O}_{X, x}}{\mathscr{O}_X(-C)_x + \mathscr{O}_X(-D)_x} \right). $$
\end{mydef}
Así, $i_x(C, D) = 0$ syss $x \notin \Supp C \cap \Supp D$.

\begin{lem}
	Sea $X$ un esquema localmente noetheriano, conexo y factorial de dimensión 2.
	Sean $D, E, F$ una terna de divisores efectivos sin ciclos primos en común dos a dos, entonces:
	\begin{enumerate}
		\item $i_x(D, E) = i_x(E, D)$.
		\item Sea 
			\begin{tikzcd}[cramped, sep=small]
				j\colon E \rar[closed] & X
			\end{tikzcd}
			un encaje cerrado.
			Luego, el divisor $D|_E := j^* D$ es un divisor efectivo de Cartier en $E$ y se tienen:
			$$ \mathscr{O}_E( D|_E ) \simeq \mathscr{O}_X(D)|_E, \qquad i_x(D, E) = \nu_x(D|_E), $$
			para todo punto cerrado $x \in E$.
		\item $i_x(D + F, E) = i_x(D, E) + i_x(F, E)$.
	\end{enumerate}
\end{lem}
\begin{proof}
	\begin{enumerate}
		\item Trivial.
		\item Basta emplear que los anillos locales son DFUs y el tercer teorema de isomorfismos para ver que
			$$ \frac{\mathscr{O}_{X, x}}{\mathscr{O}_X(-D)_x + \mathscr{O}_X(-E)_x} \cong \frac{\mathscr{O}_{E, x}}{\mathscr{O}_E(-D|_E)_x}, $$
			y luego recordar que
			$$ \nu_x(D|_E) = \ell_{\mathscr{O}_{X, x}}\left( \frac{\mathscr{O}_{E, x}}{\mathscr{O}_E(-D|_E)_x} \right) = i_x(D, E). $$
		\item Empleando el inciso anterior, se reduce a aplicar propiedades de las valuaciones. \qedhere
	\end{enumerate}
\end{proof}

\begin{mydef}
	Sea $X$ un esquema localmente noetheriano, conexo y factorial de dimensión 2.
	Sean $D, E$ un par de divisores arbitrarios y escribamos $D = D_1 - D_2, E = E_1 - E_2$ con $D_i, E_i$'s efectivos.
	Luego la multiplicidad
	$$ i_x(D, E) := i_x(D_1, E_1) - i_x(D_2, E_1) - i_x(D_1, E_2) + i_x(D_2, E_2) $$
	es independiente de la elección de $D_i, E_i$.
\end{mydef}
Estaríamos tentados a decir que la multiplicidad ahora determina una forma bilineal simétrica, pero $i_x(D, D)$ no está definido, por ejemplo.
% Más aún, la multiplicidad ahora determina una forma bilineal simétrica entre divisores en un dominio.

\begin{mydef}
	Sea $Y$ un esquema localmente noetheriano y regular, y sea $D$ un divisor efectivo de $Y$.
	Decimos que $D$ tiene \strong{cruces normales estrictos}\index{cruces normales estrictos (divisor)} en un punto $y \in Y$
	si existe un sistema de parámetros $a_1, \dots, a_n$ de $\mathscr{O}_{Y, y}$ y exponentes $\nu_1, \dots, \nu_n \in \Z_{>0}$
	tales que $\mathscr{O}_Y(-D)_y = (a_1^{\nu_1}, \dots, a_n^{\nu_n})\mathscr{O}_{Y, y}$.
	Se dice que $D$ tiene \strong{cruces normales estrictos} si los tiene en todo punto de $Y$.
	Decimos que divisores primos $D_1, \dots, D_\ell$ se \strong{cruzan transversalmente}\index{cruzarse transversalmente (divisores)} en un punto $y \in Y$
	si son distintos y el divisor $D_1 + \cdots + D_\ell$ tiene cruces normales estrictos en $y$.
\end{mydef}
\begin{prop}
	Sea $X$ un esquema localmente noetheriano, conexo y regular de dimensión 2.
	Sean $D, E$ un par de divisores primos y sea $x \in \Supp D$, entonces:
	\begin{enumerate}
		\item $D$ tiene cruces normales estrictos en $x$ syss $D$ es regular en $x$.
		\item Sea $x \in \Supp D \cap \Supp E$, son equivalentes:
			\begin{enumerate}
				\item $D$ y $E$ se cruzan transversalmente en $x$.
				\item $\mathfrak{m}_{X, x} = \mathscr{O}_X(-D)_x + \mathscr{O}_X(-E)_x$.
				\item $i_x(D, E) = 1$.
				\item $D$ y $E$ son regulares en $x$, y $T_{D, x} \oplus T_{E, x} = T_{X, x}$.
			\end{enumerate}
	\end{enumerate}
\end{prop}

\begin{prop}
	Sea $k$ un cuerpo algebraicamente cerrado y $X$ una superficie suave proyectiva sobre $k$.
	Sean $D, E$ un par de divisores primos, con $D$ suave, que se cruzan transversalmente, entonces
	$$ |C \cap D| = \deg_C( \mathscr{O}_X(D)|_C ). $$
\end{prop}

\begin{mydef}
	Sea $S$ un esquema de Dedekind.
	Una \strong{superficie fibrada sobre $S$}\index{superficie!fibrada sobre $S$} es un $S$-esquema $\pi \colon X \to S$ íntegro, proyectivo
	y plano de dimensión 2.
\end{mydef}

\begin{thmi}
	Sea $X \to S$ una superficie fibrada regular y $s \in S$ un punto cerrado.
	Existe una única forma ($\Z$-)bilineal
	$$ i_s \colon \Div X \times \Div_s X \to \Z, $$
	tal que:
	\begin{enumerate}
		\item Si $C \in \Div X, D \in \Div_s X$ no tienen componentes irreducibles comunes, entonces
			$$ i_s(C, D) = \sum_{x \in \clpt{(X_s)}} i_x(C, D) \deg x, $$
			donde $x$ recorre los puntos cerrados de la fibra $X_s$.
		\item La restricción $i_s \colon \Div_s(X) \times \Div_s(X) \to \Z$ da una forma $\Z$-bilineal simétrica.
		\item $i_s$ se preserva salvo equivalencia lineal, es decir, si $C_1 \sim C_2$ entonces $i_s(C_1, D) = i_s(C_2, D)$.
		\item Si $0 \le D \le X_s$, entonces
			$$ i_s(C, D) = \deg_{\kk(s)}( \mathscr{O}_X(D)|_E ). $$
	\end{enumerate}
\end{thmi}

\begin{thmi}
	Sea $k$ un cuerpo algebraicamente cerrado y $X$ una superficie suave proyectiva sobre $k$.
	Existe una única forma ($\Z$-)bilineal
	$$ (-.-) \colon \Div X \times \Div X \to \Z, $$
	tal que:
	% Existe una única forma $(-.-) \colon \Pic X \times \Pic X \to \Z$ tal que:
	\begin{enumerate}
		\item Si $C, D$ son ciclos primos que se intersectan transversalmente, entonces $(C.D) = |C \cap D|$.
		\item $(-.-)$ es una forma $\Z$-bilineal simétrica.
		\item $(-.-)$ se preserva salvo equivalencia lineal, es decir, si $C_1 \sim C_2$ entonces $C_1.D = C_2.D$.
	\end{enumerate}
\end{thmi}

\begin{mydef}
	Sea $k$ un cuerpo algebraicamente cerrado y $X$ una superficie suave proyectiva sobre $k$.
	Un par de divisores $D_1, D_2 \in \Div X$ se dicen \strong{numéricamente equivalentes}\index{numéricamente equivalentes (divisores)}
	(denotado <<$D_1 \equiv D_2$>>) si para todo divisor $E \in \Div X$ se cumple que $D_1.E = D_2.E$.

	Como la multiplicidad de intersección se preserva salvo equivalencia lineal es claro que $D_1 \sim D_2$ implica que $D_1 \equiv D_2$.
	Así, denotamos por $\Num X$ al cociente del grupo de Picard $\Pic X$ sobre el subgrupo de divisores numéricamente equivalentes a cero.
\end{mydef}
Nótese que la multiplicidad de intersección induce una forma bilineal no degenerada sobre $\Num(X)$.

\begin{thm}[del índice de Hodge]
	Sea $k$ un cuerpo algebraicamente cerrado y $X$ una superficie suave proyectiva sobre $k$.
	Entonces:
	\begin{enumerate}
		\item Sea $H \in \Div X$ un divisor amplio y sea $D \in \Div X$ tal que $D \not\equiv 0$ pero $D.H = 0$.
			Entonces $D^2 < 0$.
		\item $\Num(X) \otimes_\Z \R$ es un $\R$-espacio de forma bilineal (con la multiplicidad de intersección) que,
			al diagonalizar, tiene un único $+1$ en la diagonal.
		\item $\Num(X) \otimes_\Z \Q$ es un $\Q$-espacio de forma bilineal con descomposición ortogonal $V \otimes W$,
			donde $V$ es $\Q$-subespacio invariante de $\dim_\Q V = 1$ donde la forma es definida positiva y en $W$ es definida negativa.
	\end{enumerate}
\end{thm}

\subsection{Aplicación: Hipótesis de Riemann sobre curvas II}%
\label{sec:weils_proof}
En la sección~\ref{sec:weil_conj_curves} vimos una demostración de las conjeturas de Weil, donde seguimos la prueba de Bombieri-Stepanov
para las hipótesis de Riemann.
Aquí veremos una demostración alternativa de la hipótesis de Riemann empleando teoría de intersección, la cual es una adaptación de la
demostración de \citet{weil48courbes}.

Nuevamente, fijemos $k := \Fp[q]$ un cuerpo finito, $X$ una curva suave proyectiva geométricamente irreducible sobre $k$ y $\overline{X} := X_{\algcl k}$.
Sobre $\overline{X}$ tenemos dos endomorfismos de Frobenius:
$$ \Frob_{\overline{X}/k} := \Frob_{X/k} \times_{\algcl k} \Id_{\algcl k}, \qquad \psi := \Id_X \times_{\algcl k} \Frob_{\algcl k/k}. $$
Sea $Y := \overline{X} \times_{\algcl k} \overline{X}$.
Denótese $\Delta := \Img\Delta_{\overline{X}/\algcl k}$ y $F_n$ la imagen del gráfico de $\Frob_{\overline{X}/k}^n$;
ambos son ciclos primos de $Y$ (¿por qué?).

\begin{lem}\label{lem:frob_is_def_of_diag}
	Se tiene que $[ F_n ] = ( (\Frob_{\overline{X}/k} \times \Id_{\overline{X}})^* )^n[\Delta]$.
\end{lem}
\begin{proof}
	Es claro que $( (\Frob_{\overline{X}/k} \times \Id_{\overline{X}})^* )^n = (\Frob_{\overline{X}/k} \times \Id_{\overline{X}})^*$,
	así que bastará probar en general que para todo endomorfismo $g \colon \overline{X} \to \overline{X}$ se cumple que
	$[ \Img(\Gamma_g) ] = (g \times \Id_{\overline{X}})^*[ \Delta ]$.
\end{proof}

\begin{lem}
	Para todo $n \in \N$ tenemos que $|X(\Fp[q^n])| = ( [ F_n ].[ \Delta ] )$.
\end{lem}
\begin{proof}
	Nótese que $F_n, \Delta$ se cruzan transversalmente.
	\todo{Revisar conclusión.}
	Más aún, como $F_n, \Delta$ son irreducibles de dimensión 1, solo deben cortarse en puntos cerrados.
	Ahora bien, por el teorema de ceros de Hilbert, dado que $\overline{X}$ es un esquema de tipo finito sobre un cuerpo algebraicamente cerrado,
	tenemos que los puntos cerrados de $Y$ están en correspondencia con pares ordenados en $\overline{X}$.
	Finalmente, $\Frob_{\overline{X}}^n(x) = x$ (es decir, $x$ es un punto $\Fp[q^n]$-valuado) syss
	está fijado por $\Id_{\overline{X}} \times \Frob_{\algcl k/k}^n$.
\end{proof}

Finalmente, estamos listos para probar la hipótesis de Riemann:
\begin{proof}
	Sea $W := \Num(Y) \otimes \Q$. Sean $H, V$ las curvas horizontal y vertical en $Y = \overline{X} \times \overline{X}$.
	Nótese que $[H], [V]$ son distintos en $\Num(Y)$, puesto que $[H].[V] = 1$ y $[H].[H] = 0$.
	Sean $U := [H]\Q \oplus [V]\Q \le W$, y sea $U'$ el complemento ortogonal de $U$, de modo que $W = U \oplus U'$.
	La matriz de la forma bilineal sobre $U$, respecto a la base $[H], [V]$ es:
	$$ \begin{bmatrix}
		0 & 1 \\
		1 & 0
	\end{bmatrix}, $$
	y ésta matriz posee un valor propio positivo, luego el subespacio definido positivo está contenido en $U$ y, como tiene dimensión 1,
	no está en $U'$.
	Luego la forma es definida negativa en $U'$.

	Sea $\Gamma_F := \Frob_{\overline{X}} \times \Id_{\overline{X}} \colon \overline{X} \to Y$,
	y sea $T \colon W \to W$ la transformación lineal dada por $T(D) := (\Gamma_F)^* D$.
	Nótese que $\deg(\Gamma_F) = q$, luego, para divisores $D, E \in \Num(Y)$ se da
	$$ \big( (\Gamma_F)^*D, (\Gamma_F)^*E \big) = \big( D, (\Gamma_F)_*(\Gamma_F)^*E \big) = (D, qE) = q(D, E). $$
	Así, para $u, v \in W$, vemos que $(Tu, Tv) = q(u, v)$.

	Ahora bien, por el lema~\ref{lem:frob_is_def_of_diag}, sabemos que $T^n[\Delta] = [F_n]$
	Finalmente, es fácil verificar que $T[H] = q[H], T[V] = [V]$ y descomponer $[\Delta] = [H] + [V] + [w]$ donde $w \in U'$,
	luego calculamos:
	\begin{align*}
		|X(\Fp[q^n])| &= ([F_n], [\Delta]) = (T^n[\Delta], [\Delta]) \\
			      &= \big( T^n([H] + [V] + w), ([H] + [V] + w) \big) = q^n + 1 + (T^n w, w).
	\end{align*}
	Aplicando la desigualdad de Cauchy-Schwarz sobre $U'$, donde la forma es definida negativa, vemos que
	$$ |(T^n w, w)| \le \sqrt{|(T^nw, T^nw)| \, |(w, w)|} = \sqrt{q^n \, |u, v|} = O(q^{n/2}). $$
	La hipótesis de Riemann queda probada por la equivalencia~\ref{thm:riemann_hyp_equiv}.
\end{proof}


% \part{Cohomología étale}
% \chapter{El grupo fundamental algebraico de esquemas}

% \section{La teoría de Galois según Grothendieck}
% \begin{mydef}
% 	Sea $X$ un esquema.
% 	Un \strong{entorno étale}\index{entorno!etale@étale} de un punto $k$-valuado $x \in X(k)$, donde $k$ es un cuerpo arbitrario, es un morfismo étale $U \to X$
% 	con un punto $y \in U_x$ tal que $\kk(y) = k$; equivalentemente es un diagrama conmutativo:
% 	\begin{center}
% 		\begin{tikzcd}[column sep=small]
% 			U \ar[rr, "\text{étale}"] & {}                                                                 & X \\
% 			{}                        & \Spec k \ular["\langle y \rangle"'] \urar["\langle x \rangle"]
% 		\end{tikzcd}
% 	\end{center}
% 	Un \strong{punto geométrico}\index{punto!geométrico} de $X$ es un morfismo $\Spec\Omega \to X$ tal que $\Omega$ es separablemente cerrado.
% \end{mydef}
% En la cohomología étale será útil emplear puntos geométricos y entornos étale.

% Recuérdese que dadas categorías $\catA, \catB$, los funtores covariantes $F \colon \catA \to \catB$ (como objetos) y las transformaciones naturales entre ellos
% (como flechas) conforman una categoría denotada $\mathsf{Nat}(\catA, \catB)$.
% \begin{mydef}
% 	Sea $S$ un esquema y $\overline{s} \in S(\Omega)$ un punto geométrico.
% 	Denotamos por $F := \mathscr{F}ib_{\overline{s}} \colon \mathsf{FEt}_S \to \mathsf{Set}$ el funtor dado por $- \times_S \overline{s}$.
% 	Denotamos por $\pi_1(S, \overline{s}) := \Aut_{\mathsf{Nat}}(F)$ al grupo de automorfismos de $F$.
% \end{mydef}
% Categorialmente, uno puede probar que para todo funtor $F \colon \catC \to \mathsf{Set}$, se cumple que $G := \Aut_{\mathsf{Nat}}(F)$ es un grupo
% que actúa en cada $FX$, es decir, puede factorizarse mediante $\catC \to \mathsf{Set}_G$.

% \begin{mydefi}
% 	Un anillo local $(A, \mathfrak{m}, k)$ se dice \strong{henseliano}\index{anillo!henseliano} si para todo polinomio mónico $f(x) \in A[x]$
% 	tal que se factoriza como $f(x) \equiv g_0(x) h_0(x) \pmod{\mathfrak{m}}$ para algunos $g_0, h_0 \in k[x]$ mónicos y coprimos,
% 	se cumple que existen $g(x), h(x) \in A[x]$ tales que $g(x) \mod{\mathfrak{m}} = g_0(x)$ y $h(x) \mod{\mathfrak{m}} = h_0(x)$.
% \end{mydefi}
% \begin{thm}
% 	Sea $A$ un anillo local, sea $X := \Spec A$ su espectro y $s \in X$ su único punto cerrado.
% 	Son equivalentes:
% 	\begin{enumerate}
% 		\item $A$ es henseliano.
% 		\item Toda $A$-álgebra de tipo finito $B$ es un producto de anillos locales.
% 		\item Todo $X$-esquema $f \colon Y \to X$ cuasifinito y separado es de la forma $Y \cong Y_0 \amalg \coprod_{j=1}^n Y_j$, donde $f[Y_0]$ no contiene
% 			a $x$ y donde cada $Y_j$ es un $X$-esquema finito local para $j \ge 1$.
% 		\item Sea $f\colon Y \to X$ un morfismo étale con un punto $y \in Y_x$ que satisface $\kk(y) = \kk(x)$.
% 			Entonces $f$ posee una sección $s \colon X \to Y$ (i.e., tal que $s \circ f = \Id_X$).
% 		\item Sean $f_1, \dots, f_n \in A[x_1, \dots, x_n]$ y sea $\vec a \in k^n$ un punto que es una solución para cada
% 			$f_i(\vec a) \equiv 0\pmod{\mathfrak{m}}$ y tal que $\det[ (\partial f_i/\partial x_j)(\vec a) ]_{ij} \ne 0$.
% 			Entonces existe $\vec b \in A^n$ tal que cada $f_i(\vec b) = 0$ y $\vec b \mod{\mathfrak{m}} = \vec a$.
% 	\end{enumerate}
% \end{thm}

% \begin{mydef}
% 	Sea $A$ un anillo local.
% 	Se le llama \strong{hensalización}\index{hensalización} a un anillo local con un homomorfismo de anillos locales $A \to A^{\rm h}$ tal que para todo
% 	homomorfismo de anillos locales $A \to H$, donde $H$ es henseliano, ...
% \end{mydef}

% \begin{mydef}
% 	Un \strong{cubrimiento étale}\index{cubrimiento!etale@étale} es un morfismo étale suprayectivo $f\colon X \to Y$.
% 	Se dice que el cubrimiento $f$ es \strong{trivial}\index{cubrimiento!etale@étale!trivial} si $X \cong \coprod_{i\in I} Y$ como $Y$-esquema.
% \end{mydef}
% \begin{prop}
% 	Sea $S$ un esquema conexo y $f \colon X \to S$ un morfismo afín suprayectivo.
% 	Entonces $f$ es un cubrimiento étale syss existe un morfismo $g \colon Y \to S$ finito localmente libre tal que el cambio de base $X_Y \to Y$
% 	es un cubrimiento étale trivial.
% \end{prop}

% \begin{mydef}
% 	Decimos que un morfismo de esquemas $f \colon X \to Y$ es
% 	\strong{finito localmente libre}\index{morfismo!finito localmente libre}\index{finito localmente libre (morfismo)}
% 	si es finito y $f_*\mathscr{O}_X$ es un $\mathscr{O}_Y$-módulo localmente libre de rango finito (constante).
% \end{mydef}

% Dado un esquema $S$, denotaremos por $\mathsf{F\tilde Et}_S$ la categoría 

% \backmatter
\appendix
\input{app_algebraico.tex}

\input{espectrales.tex}

\printnomenclature

\nocite{*}
\printbibheading[heading=bibintoc]
Las fechas empleadas son aquellas de la primera publicación o del primer registro de Copyright.
% \printbibliography[heading=subbibintoc, keyword={algebraic_geometry}, notkeyword={self_published}, nottype={article}, title={Geometría algebraica}]
% \printbibliography[heading=subbibintoc, keyword={homological_alg, commutative_alg}, nottype={article}, title={Álgebra conmutativa y homológica}]
% % \printbibliography[heading=subbibintoc, filter={article}, title={Artículos}]
% \printbibliography[heading=subbibintoc, keyword={self_published}, keyword={algebraic_geometry}, title={Libros de autoría propia}]
\bibbycategory

\printindex

\listoftodos

\end{document}
