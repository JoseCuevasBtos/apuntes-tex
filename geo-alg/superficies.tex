\chapter{Superficies regulares}

\section{Teoría de intersección}
\begin{mydef}
	Sea $X$ un esquema localmente noetheriano, conexo y factorial de dimensión 2.
	Sean $C, D$ dos divisores de Cartier efectivos sin componentes irreducibles en común y sea $x \in X$ un punto cerrado.
	En un entorno de $x$ se tiene que $\Supp C \cap \Supp D \in \{ \emptyset, \{ x \} \}$, por lo tanto se tiene que
	$$ \mathfrak{m}_{X, x} \subseteq \rad\big( \mathscr{O}_X(-C)_x + \mathscr{O}_X(-D)_x \big), $$
	así pues, $A := \mathscr{O}_{X, x}/\big( \mathscr{O}_X(-C)_x + \mathscr{O}_X(-D)_x \big)$ es un anillo artiniano.
	Definimos la \strong{multiplicidad de intersección}\index{multiplicidad!de intersección} de $C, D$ en $x$:
	$$ i_x(C, D) := \ell_{\mathscr{O}_{X, x}}\left( \frac{\mathscr{O}_{X, x}}{\mathscr{O}_X(-C)_x + \mathscr{O}_X(-D)_x} \right). $$
\end{mydef}
Así, $i_x(C, D) = 0$ syss $x \notin \Supp C \cap \Supp D$.

\begin{lem}
	Sea $X$ un esquema localmente noetheriano, conexo y factorial de dimensión 2.
	Sean $D, E, F$ una terna de divisores efectivos sin ciclos primos en común dos a dos, entonces:
	\begin{enumerate}
		\item $i_x(D, E) = i_x(E, D)$.
		\item Sea 
			\begin{tikzcd}[cramped, sep=small]
				j\colon E \rar[closed] & X
			\end{tikzcd}
			un encaje cerrado.
			Luego, el divisor $D|_E := j^* D$ es un divisor efectivo de Cartier en $E$ y se tienen:
			$$ \mathscr{O}_E( D|_E ) \simeq \mathscr{O}_X(D)|_E, \qquad i_x(D, E) = \nu_x(D|_E), $$
			para todo punto cerrado $x \in E$.
		\item $i_x(D + F, E) = i_x(D, E) + i_x(F, E)$.
	\end{enumerate}
\end{lem}
\begin{proof}
	\begin{enumerate}
		\item Trivial.
		\item Basta emplear que los anillos locales son DFUs y el tercer teorema de isomorfismos para ver que
			$$ \frac{\mathscr{O}_{X, x}}{\mathscr{O}_X(-D)_x + \mathscr{O}_X(-E)_x} \cong \frac{\mathscr{O}_{E, x}}{\mathscr{O}_E(-D|_E)_x}, $$
			y luego recordar que
			$$ \nu_x(D|_E) = \ell_{\mathscr{O}_{X, x}}\left( \frac{\mathscr{O}_{E, x}}{\mathscr{O}_E(-D|_E)_x} \right) = i_x(D, E). $$
		\item Empleando el inciso anterior, se reduce a aplicar propiedades de las valuaciones. \qedhere
	\end{enumerate}
\end{proof}

\begin{mydef}
	Sea $X$ un esquema localmente noetheriano, conexo y factorial de dimensión 2.
	Sean $D, E$ un par de divisores arbitrarios y escribamos $D = D_1 - D_2, E = E_1 - E_2$ con $D_i, E_i$'s efectivos.
	Luego la multiplicidad
	$$ i_x(D, E) := i_x(D_1, E_1) - i_x(D_2, E_1) - i_x(D_1, E_2) + i_x(D_2, E_2) $$
	es independiente de la elección de $D_i, E_i$.
\end{mydef}
Estaríamos tentados a decir que la multiplicidad ahora determina una forma bilineal simétrica, pero $i_x(D, D)$ no está definido, por ejemplo.
% Más aún, la multiplicidad ahora determina una forma bilineal simétrica entre divisores en un dominio.

\begin{mydef}
	Sea $Y$ un esquema localmente noetheriano y regular, y sea $D$ un divisor efectivo de $Y$.
	Decimos que $D$ tiene \strong{cruces normales estrictos}\index{cruces normales estrictos (divisor)} en un punto $y \in Y$
	si existe un sistema de parámetros $a_1, \dots, a_n$ de $\mathscr{O}_{Y, y}$ y exponentes $\nu_1, \dots, \nu_n \in \Z_{>0}$
	tales que $\mathscr{O}_Y(-D)_y = (a_1^{\nu_1}, \dots, a_n^{\nu_n})\mathscr{O}_{Y, y}$.
	Se dice que $D$ tiene \strong{cruces normales estrictos} si los tiene en todo punto de $Y$.
	Decimos que divisores primos $D_1, \dots, D_\ell$ se \strong{cruzan transversalmente}\index{cruzarse transversalmente (divisores)} en un punto $y \in Y$
	si son distintos y el divisor $D_1 + \cdots + D_\ell$ tiene cruces normales estrictos en $y$.
\end{mydef}
\begin{prop}
	Sea $X$ un esquema localmente noetheriano, conexo y regular de dimensión 2.
	Sean $D, E$ un par de divisores primos y sea $x \in \Supp D$, entonces:
	\begin{enumerate}
		\item $D$ tiene cruces normales estrictos en $x$ syss $D$ es regular en $x$.
		\item Sea $x \in \Supp D \cap \Supp E$, son equivalentes:
			\begin{enumerate}
				\item $D$ y $E$ se cruzan transversalmente en $x$.
				\item $\mathfrak{m}_{X, x} = \mathscr{O}_X(-D)_x + \mathscr{O}_X(-E)_x$.
				\item $i_x(D, E) = 1$.
				\item $D$ y $E$ son regulares en $x$, y $T_{D, x} \oplus T_{E, x} = T_{X, x}$.
			\end{enumerate}
	\end{enumerate}
\end{prop}

\begin{prop}
	Sea $k$ un cuerpo algebraicamente cerrado y $X$ una superficie suave proyectiva sobre $k$.
	Sean $D, E$ un par de divisores primos, con $D$ suave, que se cruzan transversalmente, entonces
	$$ |C \cap D| = \deg_C( \mathscr{O}_X(D)|_C ). $$
\end{prop}

\begin{mydef}
	Sea $S$ un esquema de Dedekind.
	Una \strong{superficie fibrada sobre $S$}\index{superficie!fibrada sobre $S$} es un $S$-esquema $\pi \colon X \to S$ íntegro, proyectivo
	y plano de dimensión 2.
\end{mydef}

\begin{thmi}
	Sea $X \to S$ una superficie fibrada regular y $s \in S$ un punto cerrado.
	Existe una única forma ($\Z$-)bilineal
	$$ i_s \colon \Div X \times \Div_s X \to \Z, $$
	tal que:
	\begin{enumerate}
		\item Si $C \in \Div X, D \in \Div_s X$ no tienen componentes irreducibles comunes, entonces
			$$ i_s(C, D) = \sum_{x \in \clpt{(X_s)}} i_x(C, D) \deg x, $$
			donde $x$ recorre los puntos cerrados de la fibra $X_s$.
		\item La restricción $i_s \colon \Div_s(X) \times \Div_s(X) \to \Z$ da una forma $\Z$-bilineal simétrica.
		\item $i_s$ se preserva salvo equivalencia lineal, es decir, si $C_1 \sim C_2$ entonces $i_s(C_1, D) = i_s(C_2, D)$.
		\item Si $0 \le D \le X_s$, entonces
			$$ i_s(C, D) = \deg_{\kk(s)}( \mathscr{O}_X(D)|_E ). $$
	\end{enumerate}
\end{thmi}

\begin{thmi}
	Sea $k$ un cuerpo algebraicamente cerrado y $X$ una superficie suave proyectiva sobre $k$.
	Existe una única forma ($\Z$-)bilineal
	$$ (-.-) \colon \Div X \times \Div X \to \Z, $$
	tal que:
	% Existe una única forma $(-.-) \colon \Pic X \times \Pic X \to \Z$ tal que:
	\begin{enumerate}
		\item Si $C, D$ son ciclos primos que se intersectan transversalmente, entonces $(C.D) = |C \cap D|$.
		\item $(-.-)$ es una forma $\Z$-bilineal simétrica.
		\item $(-.-)$ se preserva salvo equivalencia lineal, es decir, si $C_1 \sim C_2$ entonces $C_1.D = C_2.D$.
	\end{enumerate}
\end{thmi}

\begin{mydef}
	Sea $k$ un cuerpo algebraicamente cerrado y $X$ una superficie suave proyectiva sobre $k$.
	Un par de divisores $D_1, D_2 \in \Div X$ se dicen \strong{numéricamente equivalentes}\index{numéricamente equivalentes (divisores)}
	(denotado <<$D_1 \equiv D_2$>>) si para todo divisor $E \in \Div X$ se cumple que $D_1.E = D_2.E$.

	Como la multiplicidad de intersección se preserva salvo equivalencia lineal es claro que $D_1 \sim D_2$ implica que $D_1 \equiv D_2$.
	Así, denotamos por $\Num X$ al cociente del grupo de Picard $\Pic X$ sobre el subgrupo de divisores numéricamente equivalentes a cero.
\end{mydef}
Nótese que la multiplicidad de intersección induce una forma bilineal no degenerada sobre $\Num(X)$.

\begin{thm}[del índice de Hodge]
	Sea $k$ un cuerpo algebraicamente cerrado y $X$ una superficie suave proyectiva sobre $k$.
	Entonces:
	\begin{enumerate}
		\item Sea $H \in \Div X$ un divisor amplio y sea $D \in \Div X$ tal que $D \not\equiv 0$ pero $D.H = 0$.
			Entonces $D^2 < 0$.
		\item $\Num(X) \otimes_\Z \R$ es un $\R$-espacio de forma bilineal (con la multiplicidad de intersección) que,
			al diagonalizar, tiene un único $+1$ en la diagonal.
		\item $\Num(X) \otimes_\Z \Q$ es un $\Q$-espacio de forma bilineal con descomposición ortogonal $V \otimes W$,
			donde $V$ es $\Q$-subespacio invariante de $\dim_\Q V = 1$ donde la forma es definida positiva y en $W$ es definida negativa.
	\end{enumerate}
\end{thm}

\subsection{Aplicación: Hipótesis de Riemann sobre curvas II}%
\label{sec:weils_proof}
En la sección~\ref{sec:weil_conj_curves} vimos una demostración de las conjeturas de Weil, donde seguimos la prueba de Bombieri-Stepanov
para las hipótesis de Riemann.
Aquí veremos una demostración alternativa de la hipótesis de Riemann empleando teoría de intersección, la cual es una adaptación de la
demostración de \citet{weil48courbes}.

Nuevamente, fijemos $k := \Fp[q]$ un cuerpo finito, $X$ una curva suave proyectiva geométricamente irreducible sobre $k$ y $\overline{X} := X_{\algcl k}$.
Sobre $\overline{X}$ tenemos dos endomorfismos de Frobenius:
$$ \Frob_{\overline{X}/k} := \Frob_{X/k} \times_{\algcl k} \Id_{\algcl k}, \qquad \psi := \Id_X \times_{\algcl k} \Frob_{\algcl k/k}. $$
Sea $Y := \overline{X} \times_{\algcl k} \overline{X}$.
Denótese $\Delta := \Img\Delta_{\overline{X}/\algcl k}$ y $F_n$ la imagen del gráfico de $\Frob_{\overline{X}/k}^n$;
ambos son ciclos primos de $Y$ (¿por qué?).

\begin{lem}\label{lem:frob_is_def_of_diag}
	Se tiene que $[ F_n ] = ( (\Frob_{\overline{X}/k} \times \Id_{\overline{X}})^* )^n[\Delta]$.
\end{lem}
\begin{proof}
	Es claro que $( (\Frob_{\overline{X}/k} \times \Id_{\overline{X}})^* )^n = (\Frob_{\overline{X}/k} \times \Id_{\overline{X}})^*$,
	así que bastará probar en general que para todo endomorfismo $g \colon \overline{X} \to \overline{X}$ se cumple que
	$[ \Img(\Gamma_g) ] = (g \times \Id_{\overline{X}})^*[ \Delta ]$.
\end{proof}

\begin{lem}
	Para todo $n \in \N$ tenemos que $|X(\Fp[q^n])| = ( [ F_n ].[ \Delta ] )$.
\end{lem}
\begin{proof}
	Nótese que $F_n, \Delta$ se cruzan transversalmente.
	\todo{Revisar conclusión.}
	Más aún, como $F_n, \Delta$ son irreducibles de dimensión 1, solo deben cortarse en puntos cerrados.
	Ahora bien, por el teorema de ceros de Hilbert, dado que $\overline{X}$ es un esquema de tipo finito sobre un cuerpo algebraicamente cerrado,
	tenemos que los puntos cerrados de $Y$ están en correspondencia con pares ordenados en $\overline{X}$.
	Finalmente, $\Frob_{\overline{X}}^n(x) = x$ (es decir, $x$ es un punto $\Fp[q^n]$-valuado) syss
	está fijado por $\Id_{\overline{X}} \times \Frob_{\algcl k/k}^n$.
\end{proof}

Finalmente, estamos listos para probar la hipótesis de Riemann:
\begin{proof}
	Sea $W := \Num(Y) \otimes \Q$. Sean $H, V$ las curvas horizontal y vertical en $Y = \overline{X} \times \overline{X}$.
	Nótese que $[H], [V]$ son distintos en $\Num(Y)$, puesto que $[H].[V] = 1$ y $[H].[H] = 0$.
	Sean $U := [H]\Q \oplus [V]\Q \le W$, y sea $U'$ el complemento ortogonal de $U$, de modo que $W = U \oplus U'$.
	La matriz de la forma bilineal sobre $U$, respecto a la base $[H], [V]$ es:
	$$ \begin{bmatrix}
		0 & 1 \\
		1 & 0
	\end{bmatrix}, $$
	y ésta matriz posee un valor propio positivo, luego el subespacio definido positivo está contenido en $U$ y, como tiene dimensión 1,
	no está en $U'$.
	Luego la forma es definida negativa en $U'$.

	Sea $\Gamma_F := \Frob_{\overline{X}} \times \Id_{\overline{X}} \colon \overline{X} \to Y$,
	y sea $T \colon W \to W$ la transformación lineal dada por $T(D) := (\Gamma_F)^* D$.
	Nótese que $\deg(\Gamma_F) = q$, luego, para divisores $D, E \in \Num(Y)$ se da
	$$ \big( (\Gamma_F)^*D, (\Gamma_F)^*E \big) = \big( D, (\Gamma_F)_*(\Gamma_F)^*E \big) = (D, qE) = q(D, E). $$
	Así, para $u, v \in W$, vemos que $(Tu, Tv) = q(u, v)$.

	Ahora bien, por el lema~\ref{lem:frob_is_def_of_diag}, sabemos que $T^n[\Delta] = [F_n]$
	Finalmente, es fácil verificar que $T[H] = q[H], T[V] = [V]$ y descomponer $[\Delta] = [H] + [V] + [w]$ donde $w \in U'$,
	luego calculamos:
	\begin{align*}
		|X(\Fp[q^n])| &= ([F_n], [\Delta]) = (T^n[\Delta], [\Delta]) \\
			      &= \big( T^n([H] + [V] + w), ([H] + [V] + w) \big) = q^n + 1 + (T^n w, w).
	\end{align*}
	Aplicando la desigualdad de Cauchy-Schwarz sobre $U'$, donde la forma es definida negativa, vemos que
	$$ |(T^n w, w)| \le \sqrt{|(T^nw, T^nw)| \, |(w, w)|} = \sqrt{q^n \, |u, v|} = O(q^{n/2}). $$
	La hipótesis de Riemann queda probada por la equivalencia~\ref{thm:riemann_hyp_equiv}.
\end{proof}
