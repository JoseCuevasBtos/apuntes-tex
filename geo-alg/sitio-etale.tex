\chapter{El sitio étale}
En la sección \S\ref{sec:weil_conj_curves} ya vimos la introducción a las conjeturas de Weil y vimos dos soluciones, la primera en la misma sección mediante
métodos de conteo de Bombieri-Stepanov y la segunda adaptando la demostración original de Weil (vid., \S\ref{sec:weils_proof}).
Las llamadas \strong{conjeturas de Weil}\index{conjetura!de Weil} forman una generalización del caso de curvas y fueron un elemento determinante para
el desarrollo de la geometría algebraica en los años sucesivos. He aquí el enunciado.

Sea $X/\Fp[q]$ una variedad proyectiva, suave y geométricamente irreducible de dimensión $d$ sobre un cuerpo finito.
Definiendo las funciones $\zeta(X, s)$ de Hasse-Weil y $Z(X, t)$ (como en \S\ref{sec:weil_conj_curves}) las conjeturas son las siguientes:
\begin{enumerate}[{CW}1.]
	\item \textbf{Racionalidad:} $Z(X, t)$ es una función racional y de hecho:
		$$ Z(X, t) = \frac{P_1(t) P_3(t) \cdots P_{2d-1}(t)}{P_0(t) P_2(t) \cdots P_{2d}(t)} \in \Q(t), $$
		donde cada $P_i(t) \in \Z[t]$, donde $P_0(t) = 1 - t, P_{2d}(t) = 1 - q^d t$ y cada $P_i(t) = \prod_{j} (1 - \alpha_{ij}t) \in \algcl\Q[t]$
		con $0 < i < 2d$.
	\item \textbf{Ecuación funcional:} Si $\chi := \chi(X, \mathscr{O}_X)$ es la característica de Euler, entonces
		$$ Z\left( X, \frac{1}{q^d t} \right) = \pm q^{n \chi/2} t^\chi Z(X, t). $$
	\item \textbf{Hipótesis de Riemann:} Se cumple que $|\alpha_{ij}| = q^{i/2}$ para $0 < i < 2d$.
\end{enumerate}
También hay una cuarta conjetura CW4 que de momento no tocaremos, ya que no podemos enunciarla.

Históricamente, y siguiendo una acorazonada de Weil, el problema sería resuelto con una teoría cohomológica análoga
a la homología simplicial de topología algebraica.
El primer problema al que nos enfrentaremos es que la topología de Zariski que hemos dado a las variedades es irreducible, por lo que nuestros haces dan
muy poca información con la cohomología de \v Cech, así que hemos de redefinir nuestros espacios para que las propiedades se resuelvan adecuadamente.

\addtocategory{etale}{freitag:etale, poonen:rational}

\section{Descenso}
Comencemos con dos resultados sencillos de álgebra conmutativa:
% Comenzaremos la sección con algunos resultados 
\begin{prop}\label{thm:modules_satisfy_fpqc_desc}
	Sea $\varphi\colon A \to B$ un homomorfismo fielmente plano de anillos, sea $M$ un $A$-módulo y denótese $M_B := M \otimes_A B$ visto como $B$-módulo.
	Si $M_B$ es plano (resp.\ finitamente generado, de presentación finita, localmente libre de rango $n$),
	entonces $M$ también lo es.
\end{prop}
\begin{proof}
	Que la planitud satisfaga descenso plano es trivial.

	Sea $e_1, \dots, e_n$ un sistema generador de $M_B$ sobre $B$, donde cada $e_i := \sum_{j} m_{ij} \otimes b_{ij} \in M_B$ para $m_{ij} \in M$
	y $b_{ij} \in B$.
	Luego, el homomorfismo $\varphi\colon A^N \to M$ de evaluación sobre los $m_{ij}$'s satisface que $\varphi_B \colon B^N \to M_B$ sea sobreyectivo y,
	por definición de <<fielmente plano>>, se sigue que $\varphi$ debe ser sobreyectivo.
	Si $M_B$ es de presentación finita, en ésta situación tenemos que $\ker(\varphi_B)$ es un $B$-módulo finitamente generado,
	pero $\ker(\varphi_B) \cong \ker\varphi \otimes_A B$ por planitud, luego $\ker\varphi$ es finitamente generado.

	Finalmente, ser <<localmente libre>> equivale a ser <<plano de presentación finita>>, ambas propiedades que satisfacen descenso plano.
	La igualdad de los rangos se verifica pasando a un cuerpo de restos.
\end{proof}
\begin{prop}\label{thm:algs_fpqc_desc}
	Sea $\varphi \colon A \to A'$ un homomorfismo fielmente plano de anillos, sea $B$ una $A$-álgebra y sea $B' := B \otimes_A A'$ visto como $A'$-álgebra.
	Si $B'$ es una $A'$-álgebra de tipo finito (resp.\ de presentación finita), entonces $B$ también lo es como $A$-álgebra.
\end{prop}
\begin{hint}
	La demostración es análoga a la anterior empleando $A[t_1, \dots, t_N]$ en lugar de $A^{\oplus N}$.
\end{hint}

El problema del \textit{descenso} es principalmente el siguiente: sea $\mathbf{P}$ una propiedad de morfismos de esquemas
y supongamos que tenemos el siguiente diagrama conmutativo:
\begin{equation}
	% https://q.uiver.app/#q=WzAsNixbMCwxLCJYIl0sWzEsMSwiWSJdLFsyLDEsIlMiXSxbMiwwLCJTJyJdLFsxLDAsIlknIl0sWzAsMCwiWCciXSxbMCwxLCJmIiwyXSxbMSwyLCIiLDIseyJzdHlsZSI6eyJib2R5Ijp7Im5hbWUiOiJkb3R0ZWQifX19XSxbMywyLCJnIl0sWzUsMCwiIiwyLHsic3R5bGUiOnsiYm9keSI6eyJuYW1lIjoiZG90dGVkIn19fV0sWzUsNCwiZiciXSxbNCwxLCIiLDAseyJzdHlsZSI6eyJib2R5Ijp7Im5hbWUiOiJkb3R0ZWQifX19XSxbNCwzLCIiLDAseyJzdHlsZSI6eyJib2R5Ijp7Im5hbWUiOiJkb3R0ZWQifX19XSxbNSwxLCIiLDAseyJzdHlsZSI6eyJuYW1lIjoiY29ybmVyLWludmVyc2UifX1dLFs0LDIsIiIsMCx7InN0eWxlIjp7Im5hbWUiOiJjb3JuZXItaW52ZXJzZSJ9fV1d
	\begin{tikzcd}
		{X'} & {Y'} & {S'} \\
		X & Y & S
		\arrow["{f'}", from=1-1, to=1-2]
		\arrow[dotted, from=1-1, to=2-1]
		\arrow["\ulcorner"{anchor=center, pos=0.125}, draw=none, from=1-1, to=2-2]
		\arrow[dotted, from=1-2, to=1-3]
		\arrow[dotted, from=1-2, to=2-2]
		\arrow["\ulcorner"{anchor=center, pos=0.125}, draw=none, from=1-2, to=2-3]
		\arrow["g", from=1-3, to=2-3]
		\arrow["f"', from=2-1, to=2-2]
		\arrow[dotted, from=2-2, to=2-3]
	\end{tikzcd}
	\label{cd:descent_sit}
\end{equation}
¿Qué propiedad podemos imponer en $g$ de modo que si $f'$ satisface $\mathbf{P}$ entonces $f$ también satisface $\mathbf{P}$?
% Como el lector puede apreciar, la situación es \textit{inversa} a los teoremas de cambio de base, en donde una propiedad de $f$ se eleva a una propiedad de $f'$.

Un ejemplo sencillo:
\begin{prop}
	En el diagrama conmutativo \eqref{cd:descent_sit} supongamos que $g$ es sobreyectivo.
	Entonces:
	\begin{enumerate}
		\item El morfismo $f$ es sobreyectivo syss $f'$ lo es.
		\item Si $f'$ es inyectivo, entonces $f$ también lo es.
		\item Si $f'$ tiene fibras de cardinalidad finita, entonces $f$ también.
		\item El morfismo $f$ es universalmente inyectivo syss $f'$ lo es.
	\end{enumerate}
\end{prop}
\begin{hint}
	Los tres primeros incisos aplican para funciones en general en un diagrama conmutativo de conjuntos.
	El cuarto se sigue del segundo.
\end{hint}
% \begin{proof}
% 	\begin{enumerate}
% 		\item Ya hemos visto que si $f$ es sobreyectivo, entonces el cambio de base $f' = f_{S'}$ también.
% 			El recíproco es trivial de que $f' \circ g = g' \circ f$ es sobreyectivo, y luego $f$ lo será.
% 		\item E
% 	\end{enumerate}
% \end{proof}

% \begin{prop}
% 	En el diagrama conmutativo \eqref{cd:descent_sit} supongamos que $g$ es compacto y sobreyectivo.
% 	Entonces $f$ es un morfismo compacto syss $f'$ lo es.
% \end{prop}

\begin{mydef}
	% Sean $\mathbf{P}(f)$ y $\mathbf{Q}(f)$ propiedades de morfismos $f$ de esquemas.
	Sean $\mathbf{P}$ y $\mathbf{Q}$ propiedades de morfismos de esquemas.
	Diremos que una propiedad $\mathbf{P}$ de morfismos satisface <<descenso por morfismos $\mathbf{Q}$>> si en todo diagrama conmutativo
	\eqref{cd:descent_sit} donde $g$ satisface $\mathbf{Q}$ se cumple que
	% $\mathbf{P}(f') \implies \mathbf{P}(f)$.
	$f$ satisface $\mathbf{P}$ siempre que $f'$ satisface $\mathbf{P}$.
\end{mydef}
Más brevemente, diremos que una propiedad satisface descenso fpqc (resp.\ fppf, étale) si satisface descenso por morfismos fpqc (resp.\ fppf, étale).

He aquí el interés que podría tener la teoría de descenso para teoristas de números:
\begin{ex}
	Sea $k$ un cuerpo.
	El morfismo $\Spec(\algcl k) \to \Spec k$ siempre es fpqc, aunque casi nunca es fppf.
	Para toda extensión finita $L/k$ el morfismo $\Spec L \to \Spec k$ es fppf, aunque no siempre es étale.
	Para toda extensión finita separable $L/k$ el morfismo $\Spec L \to \Spec k$ es étale.
\end{ex}

\begin{mydef}
	Sea $Y$ un esquema compacto y cuasiseparado, un subconjunto $E \subseteq Y$ se dice \strong{proconstructible}\index{proconstructible (conjunto)}
	\index{conjunto!proconstructible} si existe un morfismo $f \colon \Spec A \to Y$ desde un esquema afín con $f[\Spec A] = E$.
\end{mydef}
\begin{cor}
	Sea $Y$ un esquema compacto y cuasiseparado. Entonces:
	\begin{enumerate}
		\item Todo subconjunto localmente constructible de $Y$ es proconstructible.
		\item La imagen de todo morfismo compacto $X \to Y$ es proconstructible.
		\item La intersección de finitos conjuntos proconstructibles de $Y$ es proconstructible.
	\end{enumerate}
\end{cor}

Recuérdese que decíamos que un morfismo de esquemas $f \colon X \to Y$ \textit{refleja generizaciones} si para todo punto $x \in X$
y toda generización $y' \speto f(x)$, existe $x' \speto x$ en $X$ tal que $y' = f(x')$.
Esto era equivalente a que $f[\Spec(\mathscr{O}_{X, x})] = \Spec(\mathscr{O}_{Y, f(x)})$.
% Como corolario al teorema de constructibilidad de Chevalley vimos que un morfismo localmente de presentación finita refleja generizaciones syss es abierto.
\addtocounter{thmi}{1}
\begin{slem}
	Sea $Y$ un esquema compacto y cuasiseparado, sea $Z \subseteq Y$ un subconjunto proconstructible y
	sea $f \colon X \to Y$ un morfismo que refleja generizaciones.
	Entonces $\overline{f^{-1}[Z]} = f^{-1}[ \,\overline{Z}\, ]$.
\end{slem}
\begin{proof}
	Nótese que podemos suponer que $Y = \Spec A$ sea afín (¿por qué?).
	Siempre se satisface la inclusión $\overline{f^{-1}[Z]} \subseteq f^{-1}[ \,\overline{Z}\, ]$, probemos <<$\supseteq$>>:
	sea $x \in U := X \setminus \overline{f^{-1}[Z]}$.
	Como $\Spec(\mathscr{O}_{X, x}) \subseteq U \subseteq f^{-1}[Y \setminus Z]$, vemos que
	$$ f[\Spec(\mathscr{O}_{X, x})] = \Spec(\mathscr{O}_{Y, f(x)}) \subseteq Y \setminus Z. $$
	Afirmamos que $y := f(x) \notin \overline{Z}$.
	Sea $A'$ una $A$-álgebra tal que $g\colon Y' := \Spec(A') \to Y$ tiene imagen $Z$;
	sea $\mathfrak{p} := \mathfrak{p}_y$ el primo asociado a $y \in Y$, como $g[\Spec(\mathscr{O}_{Y, y})] = \emptyset$, se verifica que
	$$ \limdir_{s \notin \mathfrak{p}} (A[1/s] \otimes_A A') = A_{\mathfrak{p}} \otimes_A A' = 0, $$
	por lo que existe $s \in A \setminus \mathfrak{p}$ tal que $A[1/s] \otimes_A A' = 0$, es decir, tal que $\DD(s) \cap Z = \emptyset$.
\end{proof}

\begin{slem}
	Sea $f \colon X \to Y$ un morfismo compacto que refleja generizaciones.
	Entonces $f \colon X \to f[X]$ es una identificación topológica (i.e., el subespacio topológico $f[X] \subseteq Y$ es un cociente topológico de $X$).
\end{slem}
\begin{proof}
	Nuevamente podemos suponer que $Y = \Spec A$ sea afín.
	Sea $Z \subseteq f[X]$ tal que $F := f^{-1}[Z] \subseteq X$ sea cerrado, queremos ver que $Z$ es cerrado en la topología subespacio;
	vale decir, que $Z = \overline{Z} \cap f[X]$.
	Dotándolo de la estructura reducida, vemos a $F$ como subesquema cerrado de $X$ y la composición 
	\begin{tikzcd}[cramped, sep=small]
		F \rar[closed] & X \rar & Y
	\end{tikzcd}
	da un morfismo compacto, por lo que su imagen $Z$ es proconstructible. Así pues
	$$ Z = f\big[ f^{-1}[Z] \big] = f\big[ \,\overline{f^{-1}[Z]}\, \big] = f^{-1}\big[ f[ \,\overline{Z}\, ] \big] = \overline{Z} \cap f[X] $$
	como se quería ver.
\end{proof}
\addtocounter{thmi}{-1}

Es inmediato del lema anterior que:
\begin{prop}
	Todo morfismo fpqc es una identificación.
\end{prop}

\begin{prop}
	Las siguientes propiedades satisfacen descenso fpqc:
	% \begin{multicols}{3}
	% 	\noindent
	% 	Ser morfismo abierto. \\
	% 	Ser morfismo compacto. \\
	% 	Ser morfismo cerrado. \\
	% 	Ser cuasiseparado. \\
	% 	Ser homeomorfismo. \\
	% 	Ser separado.
	% \end{multicols}
	\begin{center}
		\begin{tabular}{lll}
			Ser morfismo  abierto. & Ser morfismo cerrado. & Ser homeomorfismo. \\
			Ser morfismo compacto. &    Ser cuasiseparado. &      Ser separado.   
		\end{tabular}
	\end{center}
\end{prop}
\begin{proof}
	Considere la situación del diagrama \eqref{cd:descent_sit}.
	Las primeras dos se siguen inmediatamente de que las flechas verticales sean identificaciones topológicas;
	la tercera se sigue de que un homeomorfismo es una biyección abierta.

	Para ver que los morfismos compactos satisfacen descenso fpqc basta notar que si $V \subseteq Y$ es un abierto compacto,
	entonces
	$$ f^{-1}[V] = g_X[ (f')^{-1}[ g_Y^{-1}[V] ] ] $$
	es compacto, pues $g_Y$ es un morfismo compacto por cambio de base.

	Finalmente, un morfismo se dice cuasiseparado (resp.\ separado) syss la diagonal $\Delta_f \colon X \to X\times_Y X$ es un morfismo compacto
	(resp.\ un morfismo cerrado) y ya probamos que ambas propiedades satisfacen descenso fpqc.
\end{proof}
\begin{cor}
	Las propiedades de ser <<universalmente abierto>>, <<universalmente cerrado>> y de ser un <<homeomorfismo universal>> satisfacen descenso fpqc.
\end{cor}

\begin{prop}
	Las siguientes propiedades satisfacen descenso fpqc:
	\begin{center}
		\begin{tabular}{lll}
			Ser (localmente) de         tipo finito. &  Ser un isomorfismo. & Ser propio. \\
			Ser (localmente) de presentación finita. & Ser un monomorfismo. &   Ser afín. \\
			Ser                         cuasifinito. &       Ser un encaje. & Ser finito.   
		\end{tabular}
	\end{center}
\end{prop}
\begin{proof}
	En la situación del diagrama \eqref{cd:descent_sit} vemos que $g_Y$ es fpqc, así que supondremos $Y = S$.
	Ser (localmente) de tipo finito y de presentación finita satisface descenso fpqc por la proposición~\ref{thm:algs_fpqc_desc}.
	Ser cuasifinito es ser de tipo finito con fibras finitas, por lo que se sigue de las proposiciones anteriores.

	Si $f'$ es un isomorfismo, entonces también es un homeomorfismo universal y $f$ también por descenso.
	Así, basta ver que $f^\sharp \colon \mathscr{O}_Y \to f_*\mathscr{O}_X$ es un isomorfismo de haces; aplicándole el funtor $g^*$
	obtenemos el morfismo de haces
	\[\begin{tikzcd}[column sep=large]
		(f')^\sharp \colon \mathscr{O}_{Y'} = g^*\mathscr{O}_Y \rar["\sim"] & f^\prime_* \mathscr{O}_{X'} = g^*f_*\mathscr{O}_X
	\end{tikzcd}\]
	el cual es un isomorfismo por hipótesis.
	Así que $g^*f^\sharp$ es un isomorfismo, pero como $g$ es fielmente plano, $f^\sharp$ debe serlo.
	Un monomorfismo es un morfismo cuya diagonal es un isomorfismo, así que trivialmente satisface descenso fpqc.

	Ser propio equivale a ser de tipo finito, separado y universalmente cerrado; todas propiedades que satisfacen descenso fpqc.

	Si $f'$ es afín, entonces es compacto y cuasiseparado (por lo que $f$ también) y la $\mathscr{O}_{Y'}$-álgebra cuasicoherente
	$\mathscr{A}' := f^\prime_*\mathscr{O}_{X'}$ satisface que $X' = \bfSpec_{Y'}(\mathscr{A}')$.
	Definiendo $\mathscr{A} := f_*\mathscr{O}_X$, vemos que $g^*\mathscr{A}' = \mathscr{A}$ luego tenemos el siguiente diagrama conmutativo
	% https://q.uiver.app/#q=WzAsNixbMiwwLCJZJyJdLFsyLDEsIlkiXSxbMSwwLCJBJyJdLFsxLDEsIkEiXSxbMCwwLCJYJyJdLFswLDEsIlgiXSxbMCwxLCJnIl0sWzQsNSwiIiwwLHsic3R5bGUiOnsiYm9keSI6eyJuYW1lIjoiZG90dGVkIn19fV0sWzIsMywiIiwwLHsic3R5bGUiOnsiYm9keSI6eyJuYW1lIjoiZG90dGVkIn19fV0sWzUsM10sWzMsMSwiIiwyLHsic3R5bGUiOnsiYm9keSI6eyJuYW1lIjoiZG90dGVkIn19fV0sWzIsMCwiIiwxLHsic3R5bGUiOnsiYm9keSI6eyJuYW1lIjoiZG90dGVkIn19fV0sWzQsMiwiXFxzaW0iXSxbNCwzLCIiLDIseyJzdHlsZSI6eyJuYW1lIjoiY29ybmVyLWludmVyc2UifX1dLFsyLDEsIiIsMix7InN0eWxlIjp7Im5hbWUiOiJjb3JuZXItaW52ZXJzZSJ9fV0sWzUsMSwiZiIsMix7ImN1cnZlIjoxfV1d
	\[\begin{tikzcd}
		{X'} & {\bfSpec_{Y'}(\mathscr{A}')} & {Y'} \\
		X & {\bfSpec_Y(\mathscr{A})} & Y
		\arrow["\sim", from=1-1, to=1-2]
		\arrow[dotted, from=1-1, to=2-1]
		\arrow["\ulcorner"{anchor=center, pos=0.125}, draw=none, from=1-1, to=2-2]
		\arrow[dotted, from=1-2, to=1-3]
		\arrow[dotted, from=1-2, to=2-2]
		\arrow["\ulcorner"{anchor=center, pos=0.125}, draw=none, from=1-2, to=2-3]
		\arrow["g", from=1-3, to=2-3]
		\arrow["\bar{f}"', from=2-1, to=2-2]
		% \arrow["f"', curve={height=6pt}, from=2-1, to=2-3]
		\arrow[dotted, from=2-2, to=2-3]
	\end{tikzcd}\]
	y por descenso fpqc comprobamos que $\bar{f}$ es un isomorfismo, de modo que $f$ es afín.

	Ser finito equivale a ser afín y propio, las cuales satisfacen descenso.
\end{proof}

% \begin{thm}
% 	% En el diagrama conmutativo \eqref{cd:descent_sit} supongamos que $g$ es un morfismo fpqc.
% 	% Entonces:
% 	% \begin{enumerate}
% 	% 	\item El morfismo $f$ es 
% 	% \end{enumerate}
% 	Las siguientes propiedades satisfacen descenso fpqc:
% 	Ser de tipo finito.
% 	Ser universalmente inyectivo
% 	Ser separado.
% 	Ser propio.
% \end{thm}

Ahora seguimos a \citeauthor{bosch:neron}~\cite[129\psqq]{bosch:neron}, \S 6.1.
El problema que pretendemos resolver es el siguiente: dado un morfismo de esquemas $S' \to S$,
¿bajo qué condiciones el funtor de pull-back $\mathscr{F} \mapsto p^*\mathscr{F}$ es una equivalencia de categorías (canónicamente)?
O en otras palabras, ¿cuándo un haz sobre $S'$ \emph{desciende} a un haz de $S$?
% Más aún, sería útil poder decir que esta equivalencia respete ciertas propiedades de haces (como <<ser localmente libre>>, )

% Considere el siguiente resultado elemental: dados dos esquemas $X$ e $Y$ con subesquemas abiertos $U$ y $V$,
% tales que existe un isomorfismo 
% \begin{tikzcd}[cramped, sep=small]
% 	\varphi\colon U \rar["\sim"] & V
% \end{tikzcd}
% se pueden pegar en un esquema $X \amalg_\varphi Y$.
% El lector podría releer esto en el lenguaje de que se pueden pegar esquemas Zariski-localmente,
% y podría preguntarse bajo qué condiciones se pueden pegar étale-localmente o fppf-localmente.
\begin{mydef}
	Sea $p \colon S' \to S$ un morfismo de esquemas y denótese $S'' := S' \times_S S'$.
	Dado un $\mathscr{O}_{S'}$-módulo $\mathscr{F}'$,
	un par $(\mathscr{F}', \varphi)$ se dice un \strong{dato de cubrimiento} si $\varphi \colon \pi_1^*\mathscr{F}' \to \pi_2^*\mathscr{F}'$
	es un isomorfismo de $\mathscr{O}_{S''}$-módulos, donde $\pi_1, \pi_2 \colon S'' \to S'$ denotan las proyecciones.
	Los haces con datos de cubrimiento forman una categoría $\catC_{S'/S}$ donde una flecha $f \colon (\mathscr{F}', \varphi) \to (\mathscr{G}', \psi)$
	es un morfismo de haces $f \colon \mathscr{F}' \to \mathscr{G}'$ tal que el siguiente diagrama conmuta
	\[\begin{tikzcd}[row sep=large]
		\pi_1^*\mathscr{F}' \dar[sloped, "\sim"'] \rar["{\pi_1^*f}"'] & \pi_1^*\mathscr{G}' \dar[sloped, "\sim"] \\
		\pi_2^*\mathscr{F}'                        \rar["{\pi_2^*f}"] & \pi_2^*\mathscr{G}'
	\end{tikzcd}\]
\end{mydef}
Nótese que si $\mathscr{F}$ es un $\mathscr{O}_S$-módulo, entonces $p^*\mathscr{F}$ es un haz sobre $S'$ con la identidad como dato de cubrimiento. 

\begin{prop}\label{thm:pullback_fpqc_is_fully_faith}
	Sea $p \colon S' \to S$ un morfismo fpqc, denótense $\pi_j \colon S'' := S'\times_S S' \to S'$ las proyecciones y $q := \pi_1 \circ p = \pi_2 \circ p$.
	Dado un par de $\mathscr{O}_S$-módulos cuasicoherentes $\mathscr{F}$ y $\mathscr{G}$, el siguiente diagrama es un ecualizador
	\[\begin{tikzcd}
		\Hom_S(\mathscr{F}, \mathscr{G}) \rar["p^*"] & \Hom_{S'}(p^*\mathscr{F}, p^*\mathscr{G}) \rar[shift left, "\pi_1^*"] \rar[shift right, "\pi_2^*"']
							     & \Hom_{S''}(q^*\mathscr{F}, q^*\mathscr{G}).
	\end{tikzcd}\]
	En consecuencia, el funtor $p^*(-) \colon \mathsf{QCoh}_S \to \catC_{S'/S}$ es plenamente fiel.
\end{prop}
\begin{proof}
	Podemos suponer que $S = \Spec A$ es afín.
	Como el morfismo es compacto, podemos cubrir a $S'$ con finitos abiertos afines $\{ U_i \}_i$, de modo que el morfismo $\coprod_i U_i \to \Spec A$
	es fielmente plano, de modo que podemos suponer que $S' = \Spec B$ es afín.

	Ahora, $\mathscr{F} = \widetilde{M}$ y $\mathscr{G} = \widetilde{N}$ y
	$$ \Hom_{S'}(p^*\mathscr{F}, p^*\mathscr{G}) = \Hom_B(M\otimes_A B, N\otimes_A B) \cong \Hom_A(M, N) \otimes_A B, $$
	por lo que, definiendo $H := \Hom_A(M, N)$, nos reducimos a probar que la sucesión
	\[\begin{tikzcd}
		0 \rar & H \rar & H \otimes_A B \rar["\gamma"] & H \otimes_A B \otimes_A B
	\end{tikzcd}\]
	es exacta, donde $\gamma$ es la tensorización del homomorfismo $b \mapsto b\otimes 1 - 1\otimes b$ de $B \to B\otimes_A B$.
	Es decir, la sucesión de arriba es la tensorización de la sucesión exacta $0 \to A \to B \to B\otimes_A B$ lo cual induce la sucesión exacta
	\[\begin{tikzcd}
		\Tor_1^A(H, B\otimes_A B) \rar & H \rar & H \otimes_A B \rar["\gamma"] & H \otimes_A B \otimes_A B,
	\end{tikzcd}\]
	donde $\Tor_1(H, B\otimes_A B) = 0$ pues $B \otimes_A B$ es un $A$-módulo fielmente plano (¿por qué?).
\end{proof}
En consecuencia, $p^*(-)$ establece una equivalencia con una subcategoría de $\catC_{S'/S}$.

\begin{sit}\label{sit:cocycle_fpqc}
	Sea $p \colon S' \to S$ un morfismo de esquemas.
	Denotaremos por $S'' := S' \times_S S'$ y $S''' := S'' \times_S S'$,
	las cuales vendrán dotadas de las proyecciones $\pi_1, \pi_2 \colon S'' \to S'$ y
	$\pi_{ij} \colon S''' \to S''$ con $1 \le i < j \le 3$.
\end{sit}
\begin{mydef}
	En la situación~\ref{sit:cocycle_fpqc}, un haz cuasicoherente sobre $S'$ con dato de cubrimiento $(\mathscr{F}', \varphi)$
	se dice que conforma un \strong{dato de descenso}\index{dato!de descenso} si el siguiente diagrama conmuta (llamado \textit{condición de cociclos}):
	\begin{center}
		\includegraphics[scale=1]{geo-alg/cocycle_cond.pdf}
	\end{center}
	% satisface la siguiente \textit{condición de cociclos}:
	% \[
	% 	\pi_{13}^* \varphi = \pi_{12}^* \varphi \circ \pi_{23}^* \varphi.
	% \]
	Los haces cuasicoherentes con datos de descenso conforman una subcategoría plena $\catD_{S'/S} \subseteq \catC_{S'/S}$.
	Se dice que el dato de descenso $\varphi$ es \strong{efectivo}\index{dato!de descenso!efectivo} si el par $(\mathscr{F}', \varphi)$
	está en la imagen esencial\footnotemark{} del funtor de pull-back $p^*(-)$.
\end{mydef}
\footnotetext{Vale decir, si $(\mathscr{F}', \varphi)$ es isomorfo (en $\catC_{S'/S}$)
a $p^*\mathscr{F}$, para algún haz cuasicoherente $\mathscr{F}$ sobre $S$.}
\begin{ex}
	En la situación~\ref{sit:cocycle_fpqc}, sea $\mathscr{F}$ un haz cuasicoherente sobre $S$.
	Entonces, se sigue que $\Id$ es un dato de descenso para $p^*\mathscr{F}$,
	de modo que $p^*(-) \colon \mathsf{QCoh}_S \to \catD_{S'/S}$ es plenamente fiel.
\end{ex}

\addtocounter{thmi}{1}
\begin{slem}
	Supongamos que $p \colon S' \to S$ es un morfismo con una sección (o equivalentemente, que $S'(S) \ne \emptyset$).
	Entonces todo dato de descenso sobre un $\mathscr{O}_{S'}$-módulo cuasicoherente $\mathscr{F}'$ es efectivo.
\end{slem}
\begin{proof}
	Sea $s \in S'(S)$.
	Vamos a probar que la cuasi-inversa de $p^*(-) \colon \mathsf{QCoh}_S \to \catD_{S'/S}$ es $s^*$.
	Sea $(\mathscr{F}', \varphi)$ un haz cuasicoherente sobre $S'$ con un dato de descenso;
	dados tres puntos $T$-valuados $t_1, t_2, t_3 \in S'(T)$ y empleando pull-back por $(t_i, t_j) \colon T \to S''$ obtenemos
	el isomorfismo $\varphi_{t_i, t_j} \colon t_i^* \mathscr{F}' \to t_j^* \mathscr{F}'$ (con $1 \le i < j \le 3$) de <<ser un dato de cubrimiento>>
	y la condición de cociclos como
	$$ \varphi_{t_1, t_3} = \varphi_{t_1, t_2} \circ \varphi_{t_2, t_3}. $$
	Empleando los puntos $t_1 = \Id_{S'}$ y $t_2 = s \circ p \in S'(S')$ obtenemos un isomorfismo
	\[
		f := \varphi_{t_1, t_2} \colon \mathscr{F}' = t_1^*\mathscr{F}' \to t_2^*\mathscr{F}' = p^*\mathscr{F},
	\]
	donde $\mathscr{F} := s^*\mathscr{F}'$.
	Para probar que $f$ es un isomorfismo en $\catC_{S'/S}$ queremos probar que el diagrama
	\begin{equation}
		\begin{tikzcd}[row sep=large]
			\pi_1^*\mathscr{F}' \dar["\pi_1^*f"'] \rar["\varphi"] & \pi_2^*\mathscr{F}' \dar["\pi_2^*f"] \\
			\pi_1^*p^*\mathscr{F}                 \rar[equals]    & \pi_2^*p^*\mathscr{F}
		\end{tikzcd}
		\label{eqn:effective_desc_datum}
	\end{equation}
	conmuta, para ello considere los puntos valuados
	$$ \pi_1, \quad \pi_2, \quad t_3 := \pi_1 \circ p \circ s = \pi_2 \circ p \circ s \in S'(S''), $$
	y nótese que
	\[
		\varphi_{ \pi_1, \pi_2} = \varphi, \qquad \varphi_{\pi_1, t_3} = \pi_1^* f, \qquad \varphi_{\pi_2, t_3} = \pi_2^* f,
	\]
	de modo que la condición de cociclos se traduce en que $\pi_1^*f = \varphi \circ \pi_2^*f$, es decir, que \eqref{eqn:effective_desc_datum} conmuta.
\end{proof}
\addtocounter{thmi}{-1}

\begin{thm}[Grothendieck]
	Sea $p \colon S' \to S$ un morfismo fpqc.
	Entonces el funtor $p^*(-) \colon \mathsf{QCoh}_S \to \catD_{S'/S}$ es una equivalencia de categorías.
	Dicho de otro modo, todo dato de descenso sobre un $\mathscr{O}_{S'}$-módulo cuasicoherente es efectivo.
\end{thm}
\begin{proof}
	Sea $u \colon T' \to S'$ un morfismo fpqc y sea $\overline{p} := u\circ p \colon T' \to S$,
	tenemos el siguiente diagrama conmutativo de categorías y funtores
	\[\begin{tikzcd}
		\mathsf{QCoh}_S \rar["p^*"] \drar["\overline{p}^*"'] & \catD_{S'/S} \dar["u^*", hook] \\
		{}                                                   & \catD_{T'/S'},
	\end{tikzcd}\]
	donde $u^*$ es plenamente fiel por la proposición~\ref{thm:pullback_fpqc_is_fully_faith}.
	Así, si probamos el enunciado para $\overline{p}^*$, veremos que $u^*$ es una equivalencia y luego se sigue que $p^*$ también.

	Por ello, podemos suponer que $S = \Spec A$ y $S' = \Spec B$.
	Sea $(\mathscr{F}' = \widetilde{N}, \varphi)$ un $B$-módulo con dato de descenso $\varphi \colon N \otimes_A B \to B \otimes_A N$
	(donde en el tensor identificamos a $N$ con un $A$-módulo).
	Los $A$-homomorfismos
	\begin{tikzcd}[cramped, sep=small]
		B \rar[shift left] \rar[shift right] & B \otimes_A B
	\end{tikzcd}
	dados por $b\mapsto b\otimes 1$ y $b \mapsto 1\otimes b$ inducen, al tensorizar por $-\otimes_B N$ y componer con $\varphi$, un par de $A$-homomorfismos 
	\begin{tikzcd}[cramped, sep=small]
		N \rar[shift left] \rar[shift right] & N \otimes_A B,
	\end{tikzcd}
	y definimos el $A$-módulo
	\[
		K := \ker\mathopen{}\left(
			\begin{tikzcd}
				N \rar[shift left] \rar[shift right] & N \otimes_A B
			\end{tikzcd}
		\right)\mathclose{}.
	\]
	Queremos probar que $K \otimes_A B = N$, pero, de momento, solo podemos construir el siguiente diagrama conmutativo:
	% https://q.uiver.app/#q=WzAsNixbMCwwLCJLIl0sWzEsMCwiS1xcb3RpbWVzX0FCIl0sWzEsMSwiTiJdLFswLDEsIksiXSxbMiwxLCJOXFxvdGltZXNfQSBCIl0sWzIsMCwiS1xcb3RpbWVzX0EgQlxcb3RpbWVzX0EgQiJdLFswLDMsIiIsMCx7ImxldmVsIjoyLCJzdHlsZSI6eyJoZWFkIjp7Im5hbWUiOiJub25lIn19fV0sWzEsMiwiXFxhbHBoYSIsMCx7InN0eWxlIjp7ImJvZHkiOnsibmFtZSI6ImRhc2hlZCJ9fX1dLFswLDFdLFszLDJdLFsxLDUsIiIsMSx7Im9mZnNldCI6LTF9XSxbMSw1LCIiLDEseyJvZmZzZXQiOjF9XSxbMiw0LCIiLDEseyJvZmZzZXQiOi0xfV0sWzIsNCwiIiwxLHsib2Zmc2V0IjoxfV0sWzUsNCwiXFxiZXRhIiwwLHsic3R5bGUiOnsiYm9keSI6eyJuYW1lIjoiZGFzaGVkIn19fV1d
	\[\begin{tikzcd}
		K & {K\otimes_AB} & {K\otimes_A B\otimes_A B} \\
		K & N & {N\otimes_A B}
		\arrow[from=1-1, to=1-2]
		\arrow[Rightarrow, no head, from=1-1, to=2-1]
		\arrow[shift left, from=1-2, to=1-3]
		\arrow[shift right, from=1-2, to=1-3]
		\arrow["\alpha", dashed, from=1-2, to=2-2]
		\arrow["\beta", dashed, from=1-3, to=2-3]
		\arrow[from=2-1, to=2-2]
		\arrow[shift left, from=2-2, to=2-3]
		\arrow[shift right, from=2-2, to=2-3]
	\end{tikzcd}\]
	donde queremos probar que $\alpha$ es un isomorfismo.
	Para ello, podemos hacer cambio de base fielmente plano $A \to B$ correspondiente al morfismo fpqc de esquemas $S'' \to S'$ que posee una sección,
	por lo que $N_B$ desciende a un $A_B$-módulo, por el lema anterior, y este necesariamente es $K_B$,
	es decir, $\alpha_B$ es un isomorfismo y, finalmente, $\alpha$ también.
\end{proof}

\section{Teoría de Galois-Grothendieck}
\subsection{Preludio: categorías de Galois}
El resultado al que queremos aspirar es el teorema~\ref{thm:Groth_Gal_connection} más adelante (pág.~\pageref{thm:Groth_Gal_connection}).
\begin{mydef}
	Sea $\catC$ una categoría.
	Dado un objeto $X \in \Obj\catC$ y un subgrupo de automorfismos $G \le \Aut(X)$, decimos que una flecha $q\colon X \to G\coquot X$ es un \strong{cociente
	categorial} si:
	\begin{enumerate}
		\item Dado $f \in G$ se cumple que $f \circ q = q$.
		\item Si $p \colon X \to Y$ satisface que para todo $f \in G$ se cumpla que $f\circ p = p$.
			Entonces existe una única flecha tal que el diagrama conmuta:
			\[\begin{tikzcd}[column sep=small]
				{} & X \dlar["q"'] \drar["p"] \\
				G\coquot X \ar[rr, "\exists!", dashed] & & Y
			\end{tikzcd}\]
	\end{enumerate}
	Por propiedad universal es claro que si existe el cociente categorial es único salvo $X$-isomorfismo.

	Sea $\catC$ una categoría con un funtor $F\colon \catC \to \mathsf{Fin}$.
	Se dice que $\catC$ es una \strong{categoría de Galois}\index{categoría!de Galois} con \strong{funtor fundamental}\index{funtor!fundamental} $F$ si:
	\begin{enumerate}[{G}1]
		\item $\catC$ es finitamente completa (i.e., posee objeto final y productos fibrados finitos).
		\item $\catC$ posee objeto inicial y coproductos finitos.
			Dado $X \in \catC$ y un subgrupo finito $G \le \Aut(X)$, existe el cociente categorial $G \coquot X$.
		\item\label{ax:galois_split_monos}
			Toda flecha $f \in \Morf\catC$ se factoriza $f = g\circ h$, donde $g$ es un epimorfismo y $h$ un monomorfismo.
			Todo monomorfismo $h \colon X \to Y$ se extiende a un diagrama conmutativo:
			% https://q.uiver.app/#q=WzAsMyxbMSwxLCJYIl0sWzAsMCwiWSJdLFsyLDAsIlhcXGFtYWxnIFoiXSxbMCwyLCIiLDAseyJzdHlsZSI6eyJ0YWlsIjp7Im5hbWUiOiJob29rIiwic2lkZSI6InRvcCJ9fX1dLFswLDEsImgiXSxbMSwyLCJcXHNpbSIsMCx7InN0eWxlIjp7ImJvZHkiOnsibmFtZSI6ImRhc2hlZCJ9fX1dXQ==
			\[\begin{tikzcd}[column sep=small]
				Y && {X\amalg Z} \\
				& X
				\arrow["\sim", dashed, from=1-1, to=1-3]
				\arrow["h", from=2-2, to=1-1]
				\arrow[hook, from=2-2, to=1-3]
			\end{tikzcd}\]
		\item $F$ preserva límites inversos.
		\item $F$ preserva epimorfismos, coproductos finitos y cocientes categoriales.
		\item $F$ refleja isomorfismos.
	\end{enumerate}
\end{mydef}
\newcommand{\init}{\vec 0}
\newcommand{ \fin}{\vec 1}
La segunda parte de la condición \ref{ax:galois_split_monos} puede expresarse como que <<todo monomorfismo se escinde>>.
Desde ahora, adquirimos el convenio que $\init$ (resp.\ $\fin$) denota el objeto inicial (resp.\ objeto final) de $\catC$ si existen.
También, emplearemos el abuso de notación de que $X = \init$ cuando $X$ es \emph{un} objeto inicial.

Para el siguiente lema, recuérdese que un \emph{objeto conexo} $X$ de una categoría $\catC$ con coproductos y objeto inicial
es aquel que se escribe $X \cong Y \amalg Z$ solo cuando $Y$ ó $Z$ es un objeto inicial.
Cuando $\catC$ es de Galois, esto equivale a que $X$ posea solamente por subobjetos a $\init$ y a $X$.
\begin{prop}\label{thm:Gal_cat_connected_prop}
	Sea $\catC$ una categoría de Galois con funtor fundamental $F$.
	Entonces:
	\begin{enumerate}
		\item Todo objeto en $\catC$ se escribe (de manera única) como coproducto finito de objetos conexos $\ne\init$.
		\item Si $A \in \Obj\catC$ es conexo y $a \in F(A)$, entonces la función
			\[
				\ev_a \colon \Hom_\catC(A, X) = (\yoneda A)(X) \longrightarrow F(X), \qquad f \longmapsto F(f)(a)
			\]
			es inyectiva.
		\item Si $X \ne \init$ e $Y$ es conexo, entonces toda flecha $X \to Y$ es un epimorfismo.
		\item Si $Y$ es conexo, entonces todo epimorfismo es un automorfismo.
	\end{enumerate}
\end{prop}
\begin{proof}
	\begin{enumerate}
		\item Basta emplear inducción sobre $|F(X)|$ para un objeto $X \in \Obj\catC$.
		\item Sean $f, g \in \Hom(A, X)$ tales que $F(f)(a) = F(g)(a)$.
			Definiendo $K := \ker(f, g)$ el ecualizador, vemos que $K$ es un subobjeto de $A$ y que $a \in F(K)$.
			Pero $F(\init) = \emptyset$, así que $K = A$ y $f = g$.
		\item Basta factorizar la flecha $f \in \Hom(X, Y)$ como
			\[\begin{tikzcd}
				X \rar[two heads, "g"] & Z \rar[hook, "i"] & Y \cong Z \amalg W.
			\end{tikzcd}\]
			Como $X \ne \init$, entonces $FX \ne \emptyset$ y luego $FZ \ne \emptyset$, por lo que $Y \cong Z$ y $W = \init$.
		\item Basta notar que $f \in \End(Y)$ es tal que $F(f) \colon FY \to FY$ es sobreyectiva entre conjuntos finitos, luego es biyectiva
			y, por tanto, $f$ es un isomorfismo. \qedhere
	\end{enumerate}
\end{proof}

De la proposición anterior, se sigue el siguiente lema como ejercicio:
\addtocounter{thmi}{1}
\begin{slem}
	Sea $\catC$ una categoría de Galois con funtor fundamental $F$.
	Para un objeto conexo $A$ son equivalentes:
	\begin{enumerate}
		\item Para un elemento $a \in FA$, la evaluación $\ev_a \colon \End(A) \to FA$ es sobreyectiva (y luego, biyectiva).
		\item La acción $\Aut(A) \acts F(A)$ es transitiva.
		\item La acción $\Aut(A) \acts F(A)$ es fielmente transitiva
			(i.e., para cada $a, b \in FA$ existe un único $\sigma \in \Aut(A)$ tal que $F(\sigma)(a) = b$).
	\end{enumerate}
\end{slem}
\addtocounter{thmi}{-1}

\begin{mydef}
	Un objeto $A$ en una categoría de Galois $\catC$ (con funtor fundamental $F$) se dice un \strong{objeto de Galois}\index{objeto!de Galois}
	si satisface las condiciones del lema anterior.
\end{mydef}
\begin{cor}
	Sea $\catC$ una categoría de Galois con funtor fundamental $F$.
	Dado un objeto de Galois $A$ y un elemento $a \in F(A)$, entonces para todo objeto $X \in \Obj\catC$,
	la función de evaluación $\ev_a \colon (\yoneda A)(X) \to F(X)$ es biyectiva.
\end{cor}

Dado un funtor $F \colon \catC \to \catD$ arbitrario entre categorías, podemos construir la clase $\Aut_{\mathsf{Nat}}(F)$
de transformaciones naturales $\sigma\colon F \To F$ con inversas.
También, recuérdese que, si $G$ es un grupo topológico, un \strong{$G$-conjunto} es un espacio topológico discreto $X$
con una acción $G \times X \to X$ que es conjunto.
\newcommand{\mhyph}{\text{-}}
\addtocounter{thmi}{1}
\begin{slem}
	Sea $F \colon \catC \to \catD$ un funtor arbitrario entre categorías, y supongamos que $\catC$ es (esencialmente) pequeña.
	Se cumplen:
	\begin{enumerate}
		\item La clase $\Aut(F)$ es un conjunto y, de hecho, un grupo topológico extremadamente disconexo.
		\item Si $\catD \subseteq \mathsf{Set}$ es una subcategoría de conjuntos,
			entonces $F$ se <<factoriza>> canónicamente por $F \colon \catC \to \Aut(F)\mathsf{\mhyph Set}$
		\item Si $\catD \subseteq \mathsf{Fin}$ es una subcategoría de conjuntos finitos, entonces $\Aut F$ es un grupo profinito.
			y $F$ se factoriza canónica por $F \colon \catC \to \Aut(F)\mathsf{\mhyph Fin}$.
	\end{enumerate}
\end{slem}
\begin{proof}
	Basta notar que, un automorfismo $\sigma \in \Aut(F)$, al evaluarse da un automorfismo $\sigma_X \in \Aut_\catD(FX)$ para todo $X \in \Obj\catC$.
	Más aún, claramente dos automorfismos de $F$ son iguales si todas las evaluaciones lo son, por lo que
	\begin{equation}
		\Aut(F) \subseteq \prod_{X} \Aut_\catD(FX),
		\label{eqn:Gal_aut_prod}
	\end{equation}
	donde $X$ recorre las clases de isomorfismo de objetos en $\catC$ y donde cada $\Aut(FX)$ es un conjunto,
	lo que prueba 1.

	Para la segunda, nótese que la evaluación determina un homomorfismo de grupos $\pi := \Aut F \to \Aut_\catD(FX)$ y,
	por tanto, una acción $\pi \acts FX$, de modo que $FX$ es un conjunto con una acción de $\pi$.
	Verificar que la acción es continua equivale a ver que los estabilizadores son subgrupos abiertos
	\[
		\Stab_x = \left\{ (\sigma_Y) \in \prod_{Y} \Aut(FY) : \sigma_X(x) = \sigma_X \right\},
	\]
	el cual es un producto de cada $\Aut(FY)$ para todo $Y \not\cong X$, y de un subconjunto de $\Aut(FX)$;
	este es abierto en la topología producto y.
	Finalmente, la definición de <<transformación natural>> prueba que las imágenes $F(f)$ de flechas respetan la acción de $\pi$.

	Cuando $\catD \subseteq \mathsf{Fin}$, entonces el lado derecho de \eqref{eqn:Gal_aut_prod} es un grupo profinito.
	Así, queda verificar que $\pi$ es un subgrupo \emph{cerrado} del lado derecho de \eqref{eqn:Gal_aut_prod}, esto del hecho de que,
	dada una flecha $Y \morf{f} Z$, se sigue que $\pi$ es la intersección de los subgrupos cerrados
	\begin{equation}
		\left\{ (\sigma_X) \in \prod_{X} \Aut(FX) : \sigma_Y\circ F(f) = F(f) \circ \sigma_Z \right\}.
		\tqedhere
	\end{equation}
\end{proof}

\begin{slem}
	Sea $\catC$ una categoría de Galois con funtor fundamental $F$.
	Entonces $F$ preserva y refleja monomorfismos (i.e., $X \morf{f} Y$ es un monomorfismo syss $F(f) \colon FX \to FY$ es inyectiva).
\end{slem}
\begin{proof}
	Basta notar que $F$ preserva monomorfismos pues preserva coproductos.
	Para ver que también refleja monomorfismos, basta notar que $f$ es un monomorfismo syss $(f, f) \colon X\times_Y X \to X$ es un isomorfismo
	y $F$ los refleja.
\end{proof}

\begin{slem}
	Sea $\catC$ una categoría de Galois esencialmente pequeña con funtor fundamental $F$.
	Entonces:
	\begin{enumerate}
		\item La familia $\mathcal{I}$ de pares $(A, a)$ salvo isomorfismo, donde $A \in \Obj\catC$ es conexo y $a \in F(A)$;
			con la relación $(A, a) \ge (B, b)$ cuando existe $A \morf{f} B$ con $F(f)(a) = b$, es un conjunto dirigido.
		\item La subfamilia $J \subseteq \mathcal{I}$ de pares $(A, a)$ con $A$ un objeto de Galois, es cofinal.
	\end{enumerate}
\end{slem}
\begin{proof}
	Para ver que el orden parcial sobre $\mathcal{I}$ está dirigio, sean $(A, a), (B, b) \in \mathcal{I}$.
	Definamos a $C \in \Obj\catC$ como la componente conexa de $(a, b) \in FA \times FB = F(A\times B)$ por la proposición~\ref{thm:Gal_cat_connected_prop},
	de modo que $(C, (a, b)) \ge (A, a)$ y $(C, (a, b)) \ge (B, b)$.

	Para ver que $J$ es cofinal, sea $(X, x) \in \mathcal{I}$, construyamos $Y := X^{\prod F(X)}$,
	y sea $\vec a \in FY = F(X)^{F(X)}$ cuya $x$-ésima coordenada es $x$ para cada $x \in FX$.
	Finalmente, sea $A$ la componente conexa de $\vec a$, entonces claramente $(A, a) \ge (X, x)$.
	Denotemos por $\pi_x$ la composición 
	\begin{tikzcd}[cramped, sep=small]
		A \rar[hook] & X^{F(X)} \rar[two heads, "p"] & X
	\end{tikzcd}
	donde $p$ es la proyección en la $x$-ésima coordenada.
	Luego $\ev_{\vec a}(\pi_x) = x$, por lo que $\ev_{\vec a} \colon \Hom(A, X) \to FX$ es sobreyectivo,
	e inyectivo por prop.~\ref{thm:Gal_cat_connected_prop}.

	Sea $\vec b \in FA$ otro elemento. Entonces $\ev_{\vec b} \colon \Hom(A, X) \to FX$ es inyectivo y, por cardinalidad, biyectivo.
	Es decir, que $\vec b \in F(Y) = F(X)^{F(X)}$ corresponde a una permutación de los elementos de $F(X)$; así que, como existe un
	automorfismo $\sigma \in \Aut(Y)$ que permuta las proyecciones vemos que $F(\sigma)(\vec a) = \vec b$.
	Es claro que $\sigma$ transforma la componente conexa $A$ en otra componente $B$, pero como $\vec b \in FA \cap FB$,
	se sigue que $A = B$, por lo que se restringe a un automorfismo de $A$.
	Así vemos que la acción $\Aut A \acts FA$ es transitiva.
\end{proof}

\begin{slem}
	Sea $\catC$ una categoría de Galois esencialmente pequeña con funtor fundamental $F$.
	Entonces $F$ es prorrepresentable (i.e., $F$ es el límite inverso de funtores representables).
\end{slem}
\begin{proof}
	Podemos construir el diagrama de funtores $\{ \yoneda A \}_{(A, a) \in \mathcal{I}}$ y vemos que
	$\ev_a \colon \yoneda A \To F$ determina un co-cono, por lo que tenemos una transformación natural:
	\[
		\alpha_X\colon \left( \limdir_{(A, a) \in \mathcal{I}} \yoneda A \right)(X)
		= \limdir_{(A, a) \in \mathcal{I}} \Hom(A, X) \longrightarrow FX.
	\]
	La inyectividad es clara del inciso 1 de la proposición~\ref{thm:Gal_cat_connected_prop},
	mientras que la sobreyectividad se sigue de que, dado $x \in F(X)$ existe una componente conexa $A \hookto X$ tal que $x \in F(A)$.
\end{proof}
\addtocounter{thmi}{-1}

\begin{thm}[Galois-Grothendieck]\label{thm:Groth_Gal_connection}
	Sea $\catC$ una categoría esencialmente pequeña de Galois con funtor fundamental $F$. Entonces:
	\begin{enumerate}
		\item El funtor canónico $F \colon \catC \to \Aut(F)\mathsf{\mhyph Fin}$ es una equivalencia de categorías.
		\item Si $\pi$ es un grupo profinito tal que $F\colon \catC \to \mathsf{Fin}$ es isomorfo a $\pi \mathsf{\mhyph Fin} \to \mathsf{Fin}$,
			entonces $\pi$ y $\Aut F$ son canónicamente isomorfos (en $\mathsf{TopGrp}$).
		\item En consecuencia, dado otro funtor fundamental $F'$, entonces $F$ y $F'$ son isomorfos.
		\item Si $F, G\colon \catC \to \pi\mathsf{\mhyph Fin}$ son funtores fundamentales, entonces la equivalencia determina
			un automorfismo interno del grupo profinito $\pi$.
	\end{enumerate}
\end{thm}

\subsection{De vuelta a los cubrimientos étale}
El objetivo es ahora poder demostrar, aplicar e interpretar la correspondencia de Galois-Grothendieck para cubrimientos étale.
Para ello, primero debemos probar que los recubrimientos étale finitos conforman una categoría de Galois; la única parte
que no es directa es la existencia de cocientes:
\begin{prop}
	Fijemos un esquema base $S$ y denotemos por $\mathscr{A}_S$ la categoría de $S$-esquemas cuyo morfismo estructural
	es afín y sobreyectivo.
	Dado un objeto $X \in \Obj(\mathscr{A}_S)$ y un subgrupo finito $G \le \Aut_S(X)$, existe el cociente categorial $G\coquot X$ (en $\mathscr{A}_S$).
\end{prop}
\begin{proof}
	Denotaremos por $p \colon X \to S$ al morfismo estructural.
	Vamos a manualmente construir el espacio anillado $(G \coquot X, \mathscr{O}_{G\coquot X})$:
	su espacio topológico subyacente es el espacio cociente $\pi \colon X \to G\coquot X$,
	mientras que el haz estructural $\mathscr{O}_{G \coquot X}$ es el haz $(\pi_*\mathscr{O}_X)^G$ de elementos $G$-invariantes.
	Este es claramente un cociente categorial entre espacios anillados.

	Para ver que $G \coquot X$ es un esquema, supondremos, ya que $p$ es afín, que $S = \Spec A$ y $X = \Spec B$ son afines
	y denotaremos $\varphi := p^\sharp \colon A \to B$ el homomorfismo estructural.
	Así, queremos ver que $G \coquot X = \Spec(B^G)$.
	Sea $q \colon X \to X_G := \Spec(B^G)$ el morfismo canónico dado por la inclusión $B^G \hookto B$;
	nótese que la extensión de anillos $B / B^G$ es entera ya que cada $b \in B$ es raíz del polinomio mónico $\prod_{\sigma \in G} (x - \sigma(b))
	\in B^G[x]$, por lo que $X \to X_G$ es un morfismo entero y el morfismo $X_G \to S$ es inmediatamente afín y sobreyectivo.
	Nótese que $q$ manda el cerrado $\VV(\mathfrak{a}) \subseteq X$ en $\VV(\mathfrak{a}^G) \subseteq X^G$ por lo que la sobreyectividad es clara.

	Veamos que las fibras de $q$ son $G$-órbitas:
	claramente la fibra de $\mathfrak{p}^G$ contiene a $\{ \sigma[\mathfrak{p}] \}_{\sigma \in G}$;
	si ademas contuviese a otra órbita $\{ \sigma[\mathfrak{q}] \}_{\sigma \in G}$, entonces $\sigma\mathfrak{p}$ y $\sigma\mathfrak{q}$
	inducen ideales maximales en el anillo artiniano
	\[
		\overline{B} := B \otimes_{B^G} \kk(\mathfrak{p}^G) = \Gamma(X_{\mathfrak{p}^G}, \mathscr{O}_{ X_{\mathfrak{p}^G} }),
	\]
	con $\bigcap_{\sigma} \sigma\mathfrak{p} = \bigcap_{\sigma} \sigma\mathfrak{q} = 0$.
	Pero esto es absurdo por el teorema chino del resto.

	Finalmente, nótese que $(\pi_* \mathscr{O}_X)^G$ es un haz cuasicoherente sobre $X_G$ ya que es el núcleo del morfismo de haces cuasicoherentes:
	\[
		\pi_*\mathscr{O}_X \longrightarrow \bigoplus_{\sigma \in G} \pi_*\mathscr{O}_X, \qquad s \longmapsto (s - \sigma(s))_\sigma,
	\]
	por lo que para verificar que $(\pi_* \mathscr{O}_X)^G \simeq \mathscr{O}_{X_G}$ basta verificarlo en secciones globales,
	donde ambos corresponden a $B^G$.
\end{proof}

\begin{cor}
	Si $X \to S$ es un recubrimiento étale finito y conexo, entonces para todo subgrupo $H \le \Aut(X/S)$
	se cumple que los morfismos $X \to H \coquot X$ y $H \coquot X \to S$ son recubrimientos étale finitos.
\end{cor}
\begin{proof}
	Como los morfismos étale y los morfismos finitos admiten cancelación derecha por morfismos separados,
	basta verificar que $H \coquot X \to S$ es étale finito.
	% Por la proposición anterior, $H\coquot X \to S$ es afín y sobreyectivo, y es claro que es cuasifinito, así que es un morfismo finito.
	Que $X \to S$ sea étale finito significa que existe un $S$-esquema $T$ tal que $X \times_S T \cong F \times_S T$,
	donde $F$ es un $S$-esquema étale finito escindido.
	La acción $H \acts X$ se induce una acción natural $H \acts X_T$ y, como $X_T$ es un $T$-esquema étale escindido,
	se comprueba que $G \coquot (X \times_S T) = (G\coquot F) \times_S T$.
	Por otro lado, es claro que $X_T \to (G \coquot X)_T$ es constante en órbitas de $G$, de modo que hay un $T$-morfismo
	$G \coquot (X \times_S T) \epicto (G \coquot X) \times_S T$ que es afín y sobreyectivo.
	Finalmente, para ver que es un isomorfismo basta verificarlo en abiertos afines, donde es claro.
\end{proof}

\begin{prop}
	Sea $S$ un esquema conexo y sea $\overline{s} \in S(\Omega)$ un punto geométrico.
	La categoría $\mathsf{F\acute Et}_S$ es una categoría de Galois esencialmente pequeña con el funtor fibra
	\[
		F_{\overline{s}}\colon \mathsf{F\acute Et}_S \to \mathsf{F\acute Et}_\Omega \cong \mathsf{Set}, \qquad
		X \mapsto X_{\overline{s}} = X \times_S \Spec(\Omega)
	\]
	como funtor fundamental.
	Definimos $\pi_1^{\text{ét}}(S, \overline{s}) := \Aut(F_{\overline{s}})$ y lo llamamos
	el \strong{grupo fundamental étale}\index{grupo!fundamental!(étale)} de $S$ basado en $\overline{s}$.
\end{prop}
\begin{cor}
	Sea $S$ un esquema conexo y sea $\overline{s} \in S(\Omega)$ un punto geométrico.
	La categoría $\mathsf{\acute Et}_S$ es una categoría de Galois y existe una equivalencia de categorías
	\[
		\mathsf{\acute Et}_S \longrightarrow \pi_1^{\text{ét}}(S, \overline{s})\mathsf{\mhyph Set}.
	\]
\end{cor}

Otra consecuencia del formalismo de Grothendieck es <<la invarianza topológica del punto>>:
\begin{cor}
	Sea $S$ un esquema conexo y sean $s_1 \in S(\Omega_1)$ y $s_2 \in S(\Omega_2)$ un par de puntos geométricos.
	Entonces existe un isomorfismo entre los funtores fibra 
	\begin{tikzcd}[cramped, sep=small]
		F_{s_1} \rar["\sim"] & F_{s_2},
	\end{tikzcd}
	que induce un isomorfismo (continuo) entre grupos fundamentales étale 
	\begin{tikzcd}[cramped, sep=small]
		\pi_1^{\text{ét}}(S, s_1) \rar["\sim"] & \pi_1^{\text{ét}}(S, s_2).
	\end{tikzcd}
\end{cor}
En consecuente, podemos hablar libremente del grupo fundamental étale $\pi_1^{\text{ét}}(X)$ sin referencia a un punto geométrico base
si es que solo nos interesa su clase de isomorfismo como grupo profinito.

\begin{exn}\label{ex:etale_cohom_is_Gal}
	Sea $S = \Spec k$ el espectro de un cuerpo $k$.
	La categoría $\mathsf{F\acute Et}_k$ es la categoría opuesta de $k$-álgebras conmutativas separables
	de dimensión finita (como $k$-espacios vectoriales).
	Sus objetos conexos corresponden a extensiones finitas separables y sus objetos de Galois son extensiones de Galois finitas.
	Finalmente, como el grupo fundamental étale se calcula como un límite inverso de los grupos de automorfismos de los objetos de Galois,
	se comprueba que $\pi_1^{\text{ét}}(\Spec k, \overline{s}) = \Gal(\sepcl k/k)$.
	(Recuerde que canónicamente $\Gal(\sepcl k/k) \cong \Gal(\algcl k/k)$.)
\end{exn}

Es fácil probar que los recubrimientos étale conexos son los objetos conexos de $\mathsf{\acute Et}_S$,
así que queda preguntarse por los objetos de Galois:
\begin{mydef}
	Sea $S$ un esquema conexo y sea $X$ un recubrimiento étale conexo de $S$.
	Se dice que $X$ es un \strong{recubrimiento de Galois}\index{recubrimiento!de Galois} si para todo punto geométrico $\overline{s} \in S(\Omega)$
	la acción de $\Aut(X/S)$ en la fibra geométrica $X_{\overline{s}}$ es transitiva.
\end{mydef}

La correspondencia de Galois clásica es ahora una mera formalidad:
\begin{prop} % Szamuely, Prop. 5.3.8
	Sea $f\colon X \to S$ un recubrimiento de Galois finito.
	Dado un recubrimiento étale finito conexo $Z \to S$ en un diagrama conmutativo
	\[\begin{tikzcd}
		X \drar["f"'] \rar["\pi"] & Z \dar \\
				          & S
	\end{tikzcd}\]
	entonces $\pi \colon X \to Z$ es un recubrimiento de Galois (finito) y, de hecho, $Z \cong H \coquot X$ con $H = \Aut(X/Z) \le \Aut(X/S)$.
	Así, hay una biyección entre subgrupos de $\Aut(X/S)$ y recubrimientos étale conexos intermedios;
	más aún, $Z \to S$ es de Galois syss $\Aut(X/Z)$ es un subgrupo normal de $\Aut(X/S)$, en cuyo caso
	\[
		\Aut(Z/S) \cong \Aut(X/S)/\Aut(X/Z).
	\]
\end{prop}
\begin{prop} % Szamuely, Prop. 5.3.9
	Sea $p\colon X \to S$ un recubrimiento étale finito conexo.
	Existe un morfismo $\pi \colon Y \to X$ tal que:
	\begin{enumerate}
		\item $\pi \circ p \colon Y \to X$ es un recubrimiento de Galois.
		\item Si $g \colon Z \to X$ es tal que $g\circ p \colon Z \to X$ es un recubrimiento de Galois,
			entonces existe un morfismo $Z \to Y$ tal que el siguiente diagrama conmuta:
			\[\begin{tikzcd}
				Z \ar[rr] \drar["\exists"']  &                 & X \\
							     & Y \urar["\pi"']
			\end{tikzcd}\]
	\end{enumerate}
\end{prop}

\begin{prop} % EGA IV_2 4.5.13 (p. 61)
	Sea $X$ un esquema conexo sobre un cuerpo $k$.
	Si existe un morfismo $Y \to X$, donde $Y$ es un esquema geométricamente conexo y no vacío sobre $k$,
	entonces $X$ es geométricamente conexo.
\end{prop}
\begin{proof}
	Sea $\overline{f} := f_{\algcl k}\colon \overline{Y} := Y_{\algcl k} \to X_{\sepcl k} =: \overline{X}$.
	Dado un subconjunto $\overline{U} \subseteq \overline{X}$ que es abierto, cerrado y no vacío; como las proyecciones $p\colon \overline{X} \to X$
	y $q\colon \overline{Y} \to Y$ son funciones abiertas y cerradas (¿por qué?), se sigue que $p[ \,\overline{U}\, ] = X$.
	Así, $\overline{f}^{-1}[ \,\overline{U}\, ]$ ha de ser un abierto y cerrado no vacío de $\overline{Y}$, de modo que $\overline{U} = \overline{X}$.
	Aplicando el mismo razonamiento a $\overline{X} \setminus \overline{U}$ comprobamos que éste ha de ser vacío, es decir, que $\overline{X}$ es conexo.
\end{proof}

De esto se siguen inmediatamente dos corolarios:
\begin{cor}
	Sea $X$ un esquema sobre un cuerpo $k$. Entonces:
	\begin{enumerate}
		\item Dado un $k$-morfismo $f \colon Y \to X$, donde $Y$ es geométricamente conexo sobre $k$,
			entonces su imagen es una componente geométricamente conexa de $X$.
		\item Si $Y \subseteq X$ es una componente irreducible que es también geométricamente irreducible sobre $k$,
			entonces la componente conexa $X_0 \subseteq X$ que le contiene es también geométricamente conexa.
	\end{enumerate}
\end{cor}
\begin{cor}
	Sea $X$ un esquema conexo sobre un cuerpo $k$.
	Si $X$ posee un punto $x \in X$ cuyo cuerpo de restos sea puramente inseparable sobre $k$ (e.g., si $X(k) \ne \emptyset$),
	entonces $X$ es geométricamente conexo.
\end{cor}

\begin{prop}
	Sea $X$ un esquema sobre un cuerpo $k$, sea $x \in X$ un punto y sea $K := \kk(x) \cap \sepcl k$.
	Supongamos que se satisfacen las siguientes condiciones:
	\begin{enumerate}[(a)]
		\item $X$ es conexo.
		\item La extensión $K/k$ es finita (e.g., si $X$ es localmente de tipo finito).
	\end{enumerate}
	Entonces $X_{\sepcl k}$ posee $\le [K : k]$ componentes geométricamente conexas y si $L \supseteq K$ es una extensión de Galois,
	entonces la componentes conexas de $X_L$ son geométricamente conexas.
\end{prop}
\begin{proof}
	Sea $L/K/k$ una extensión de Galois finita.
	La proyección $p\colon X_L \to X$ es un morfismo étale finito, luego la imagen de toda componente conexa $Y_\alpha \subseteq X_L$ es todo $X$.
	Así, la fibra $(X_L)_x$ tiene tantos puntos como $X_L$ tiene componentes conexas; y $(X_L)_x = \Spec(\kk(x) \otimes_k L)$
	el cual tiene $\le [K : k]$ puntos.
	Más aún, para todo $y \in (X_L)_x$ vemos que $\kk(y)$ es una extensión puramente inseparable de $L$,
	así que concluimos por el corolario anterior.
\end{proof}

\begin{prop}
	Sea $X$ un esquema localmente de tipo finito sobre un cuerpo $k$.
	Existe un un $k$-morfismo $q \colon X \to \pi_0(X)$ con la siguiente propiedad universal:
	\begin{enumerate}
		\item $\pi_0(X)$ es étale sobre $k$.
		\item Dado otro $k$-morfismo $g \colon X \to Y$, donde $Y$ es étale sobre $k$, existe $h \colon \pi_0(X) \to Y$ tal que $g = q\circ h$.
	\end{enumerate}
	Más aún, $q$ es un morfismo fielmente plano y sus fibras son las componentes conexas de $X$.
\end{prop}
\begin{proof}
	Sea $\overline{X} := X \times_k \Spec(\sepcl k)$ el cambio de base y sea $\pi_0^{\rm tg}(X) := \pi_0\big( \overline{X}_{\rm top} \big)$,
	donde $\pi_0$ ahora denota el conjunto usual de las componentes conexas del espacio topológico $\overline{X}_{\rm top}$.
	Nótese que $\pi_0^{\rm tg}(X)$ viene dotado de una acción del grupo de Galois absoluto $\Gamma_k := \Gal(\sepcl k/k)$.
	Sea $C \subseteq X_{\sepcl k}$ una componente conexa, entonces mediante la proyección $X_{\sepcl k} \epicto X$,
	su imagen es una componente conexa $D \subseteq X$.
	Ahora, $D$ es un esquema conexo y localmente de tipo finito sobre $k$, luego $D_{\sepcl k}$ tiene finitas componentes conexas,
	por lo que la $\Gamma_k$-órbita de $C$ es finita; esto prueba que la acción es continua.

	Definamos
	$$ \pi_0(\overline{X}) := \coprod_{\pi_0^{\rm tg}(X)} \Spec(\algcl k). $$
	Éste es un esquema étale sobre $\sepcl k$ y podemos construir el $\sepcl k$-morfismo $\overline{q} \colon \overline{X} \to \pi_0(\overline{X})$
	como aquel que manda cada componente conexa en un punto distinto y verificar que se satisfacen las hipótesis.
	Como $\pi_0(\overline{X})$, visto como conjunto, está dotado de una acción continua de $\Gamma_k$, existe un esquema étale $\pi_0(X)$ sobre $k$
	y un isomorfismo de $\Gamma_k$-conjuntos $\alpha\colon \pi_0(X)(\sepcl k) \to \pi_0(\overline{X})$;
	así, mediante $\alpha$, tenemos el $\sepcl k$-morfismo
	\[
		\overline{q} \colon X \times_k \Spec(\sepcl k) \longrightarrow \pi_0(X) \times_k \Spec(\sepcl k),
	\]
	el cual determina un $k$-morfismo $q \colon X \to \pi_0(X)$ por descenso de Galois que,
	del mismo modo, puede verificarse que satisface las hipótesis necesarias.
\end{proof}
\begin{cor}
	Sean $X, Y$ un par de esquemas localmente de tipo finito sobre un cuerpo $k$.
	Entonces $\pi_0(X) \times_k \pi_0(Y) \cong \pi_0(X \times_k Y)$ canónicamente.
	En particular, para toda extensión algebraica $K/k$ se cumple que $\pi_0(X_K) = \pi_0(X)_K$.
\end{cor}
\begin{hint}
	Puede hacer cambio de base por una extensión finita para poder suponer que $X$ e $Y$ poseen puntos racionales,
	y luego aplicar descenso fpqc.
\end{hint}

\subsection{Sucesiones exactas en homotopía}
En ésta sección veremos las propiedades funtoriales del grupo fundamental étale $\pi_1^{\text{ét}}$.
\begin{sit}\label{sit:etale_exact}
	Sean $S, S'$ un par de esquemas conexos, sea $\Omega$ un cuerpo separablemente cerrado,
	sean $\overline{s} \in S(\Omega)$ y $\overline{s}' \in S'(\Omega)$ un par de puntos geométricos
	y $f \colon S' \to S$ un morfismo tal que $\overline{s}' \circ f = \overline{s}$.
	Denotaremos por
	\[
		\operatorname{BC}_{S, S'} \colon \mathsf{\acute Et}_S \longrightarrow \mathsf{\acute Et}_{S'}, \qquad X \longmapsto X \times_S S'
	\]
	al funtor de cambio de base.
	Así, $\mathcal{F}_{\overline{s}} = \operatorname{BC}_{S, S'} \circ \mathcal{F}_{\overline{s}'}$, lo que determina un homomorfismo (continuo)
	entre los grupos fundamentales étale
	\[
		f_* \colon \pi_1^{\text{ét}}(S', \overline{s}') \longrightarrow \pi_1^{\text{ét}}(S, \overline{s}).
	\]
\end{sit}

Al siguiente resultado se le llama <<invarianza topológica>> ya que los engrosamientos de orden finito son homeomorfismos universales.
\begin{prop}
	En la situación~\ref{sit:etale_exact},
	si $S'$ es un engrosamiento de orden finito de $S$, entonces $\operatorname{BC}_{S, S'}$ es un isomorfismo de categorías y,
	en consecuencia, 
	\begin{tikzcd}[cramped, sep=small]
		f_* \colon \pi_1^{\text{ét}}(S', \overline{s}') \rar["\sim"] & \pi_1^{\text{ét}}(S, \overline{s})
	\end{tikzcd}
	es un isomorfismo (continuo) de grupos topológicos.
\end{prop}
% \begin{proof}
% 	La consecuencia se sigue formalmente
% \end{proof}

\begin{prop}
	En la situación~\ref{sit:etale_exact}, se cumplen:
	\begin{enumerate}
		\item El homomorfismo $f_*$ es nulo syss para cada recubrimiento étale finito conexo $X \to S$
			se cumple que el cambio de base $X \times_S S'$ es un recubrimiento étale finito escindido.
		\item El homomorfismo $f_*$ es sobreyectivo syss para cada recubrimiento étale finito conexo $X \to S$
			se cumple que el cambio de base $X \times_S S'$ también es conexo.
	\end{enumerate}
\end{prop}
\begin{proof}
	\begin{enumerate}
		\item $\implies$.
			Basta recordar que un recubrimiento étale finito $X' \to S'$ se escinde
			syss la acción de $\pi_1^{\text{ét}}(S', \overline{s}')$ es trivial.

			$\impliedby$.
			Por contrarrecíproca, si $\Img(f_*)$ es no nula, existe un subgrupo normal abierto $U \nsupseteq \Img(f_*)$
			de modo que la acción de $\pi_1^{\text{ét}}(S', \overline{s}')$ sobre el grupo finito $f_* \pi_1^{\text{ét}}(S, \overline{s}) / U$
			no es trivial.
			Ahora bien, tomando preimagen, existe un recubrimiento de Galois $X \to S$ cuyo grupo de $S$-automorfismos es
			$\pi_1^{\text{ét}}(S, \overline{s}) / (f_*)^{-1}U$ y su cambio de base tiene
			acción no trivial de $\pi_1^{\text{ét}}(S', \overline{s}')$

		\item $\implies$.
			Basta notar que un recubrimiento étale finito es conexo
			syss corresponde a un conjunto con acción de $\pi_1^{\text{ét}}(S, \overline{s})$ transitiva.

			$\impliedby$.
			Si $\Img(f_*) \subset \pi_1^{\text{ét}}(S', \overline{s}')$, entonces es un subgrupo cerrado compacto,
			de modo que existe un subgrupo abierto normal $U \nsl_o \pi_1^{\text{ét}}(S', \overline{s}')$ tal que
			$U \subseteq \Img(f_*)$, luego existe un recubrimiento étale finito conexo $X \to S$ con
			\[
				\Aut(X/S) = (f_*)^{-1}U \coquot \pi_1^{\text{ét}}(S, \overline{s}),
			\]
			pero $X' = X \times_S S'$ tiene la acción trivial de $U \coquot \Img(f_*)$, mediante $f_*$, por lo que no es conexo.
			\qedhere
	\end{enumerate}
\end{proof}

\addtocounter{thmi}{1}
\begin{slem}
	En la situación~\ref{sit:etale_exact}, sea $U \nsle_o \pi_1^{\text{ét}}(S, \overline{s})$ un subgrupo normal abierto
	y sea $X \to S$ el recubrimiento étale conexo con $\Aut(X/S) = \pi_1^{\text{ét}}(S, \overline{s})/U$.
	Sea $\overline{x} \in X_{\overline{s}}$ el punto en la fibra geométrica correspondiente a la clase $1\mod U$.

	Entonces $U \supseteq \Img(f_*)$ syss el cambio de base $X' := X_{S'} \to S'$ posee una sección $S' \to X'$
	que manda $\overline{s}'$ en $\overline{x}$.
\end{slem}
\begin{proof}
	Nótese que $\Img(f_*) \subseteq U$ syss la acción $X \acts \pi_1^{\text{ét}}(S', \overline{s}')$ (mediante $f_*$) fija a $\overline{x}$,
	es decir, que la componente conexa de $\overline{x}$ en $X'$ está fija bajo la acción de $\pi_1^{\text{ét}}(S', \overline{s}')$.
	Por tanto, esta componente es isomorfa a $S'$ meduante $X' \to S'$.
\end{proof}
\addtocounter{thmi}{-1}
\begin{prop}
	En la situación~\ref{sit:etale_exact}, sea $U' \nsle_o \pi_1^{\text{ét}}(S', \overline{s}')$ un subgrupo normal abierto
	y sea $X' \to S'$ el recubrimiento étale conexo con $\Aut(X/S) = \pi_1^{\text{ét}}(S', \overline{s}')/U'$.

	Entonces $U' \supseteq \ker(f_*)$ syss existe un recubrimiento étale finito $X \to S$ y un $S'$-morfismo $X_j \to X'$,
	donde $X_j$ es una componente conexa del cambio de base $X \times_S S'$.
\end{prop}
\begin{proof}
	$\implies$.
	Sea $U \le_o \pi_1^{\text{ét}}(S, \overline{s})$ tal que $\Aut(X/S) \cong U \coquot \pi_1^{\text{ét}}(S, \overline{s})$.
	Así, tras elegir un punto geométrico, la componente conexa $X_j \subseteq X \times_S S'$ se identifica
	con $(f_*)^{-1}U \coquot \pi_1^{\text{ét}}(S', \overline{s}')$ y $(f_*)^{-1}U$ es un subgrupo abierto que contiene a $\ker(f_*)$.
	Finalmente, la existencia de un $S'$-morfismo $X_j \to X'$ equivale a que $(f_*)^{-1}U \subseteq U'$.

	$\impliedby$.
	Si $U' \supseteq \ker(f_*)$, entonces $V' := f_*[U']$ es un subgrupo abierto del grupo profinito $H := \Img(f_*)$;
	luego existe $V \le_o \pi_1^{\text{ét}}(S', \overline{s}')$ tal que $V' = V \cap H$, y existe $X \to S$ asociado
	al espacio cociente $V \coquot \pi_1^{\text{ét}}(S, \overline{s})$.
	Así pues, una componente conexa $X_j \subseteq X \times_S S'$ corresponde al espacio cociente de
	algún subgrupo abierto $U'' \le_o \pi_1^{\text{ét}}(S', \overline{s}')$ y hay un $S'$-morfismo $X_j \to X'$ syss $U'' \subseteq U'$.
	Finalmente, como ambos contienen a $\ker(f_*)$, por le teorema de la correspondencia, esta inclusión equivale a que
	$f_*[U''] \subseteq f_*[U']$ lo cual se verifica por construcción.
\end{proof}

\begin{cor}
	En la situación~\ref{sit:etale_exact}, el homomorfismo $f_*$ es inyectivo syss para todo recubrimiento étale finito $X' \to S'$
	existe un recubrimiento étale finito $X \to S$ y un $S'$-morfismo $X_j \to X'$, donde $X_j$ es una componente conexa de $X \times_S S'$.
\end{cor}
\begin{cor}\label{thm:homotopy_exactness}
	Sean $f \colon S_1 \to S_2$ y $g \colon S_2 \to S_3$ morfismos entre esquemas conexos,
	y sean $\overline{s}_j \in S_j(\Omega)$ puntos geométricos
	con $\overline{s}_1 \circ f = \overline{s}_2$ y $\overline{s}_2 \circ g = \overline{s}_3$.
	Entonces, la sucesión
	\[\begin{tikzcd}[sep=large]
		\pi_1^{\text{ét}}(S_1, \overline{s}_1) \rar["f_*"] & \pi_1^{\text{ét}}(S_2, \overline{s}_2) \rar["g_*"] & \pi_1^{\text{ét}}(S_3, \overline{s}_3)
	\end{tikzcd}\]
	es exacta syss se satisfacen las siguientes condiciones:
	\begin{enumerate}[(i)]
		\item\label{thm:homotopy_exactness_i}
			Para todo recubrimiento étale finito $X_3 \to S_3$,
			el cambio de base $X_1 \times_{S_1} S_3 \to S_3$ es un recubrimiento étale escindido.
		\item\label{thm:homotopy_exactness_ii}
			Para todo recubrimiento étale finito conexo $X_2 \to S_2$ tal que el $S_3$-esquema $X_2 \times_{S_2} S_3$ posea sección,
			existe un recubrimiento étale finito conexo $X_1 \to S_1$ y un $S_2$-morfismo desde una componente conexa
			de $X_1 \times_{S_1} S_2$ hacia $X_2$.
	\end{enumerate}
\end{cor}

\addtocounter{thmi}{1}
\begin{slem}
	Sea $X$ un esquema compacto y geométricamente íntegro sobre un cuerpo $k$, y sea $\overline{X} := X \times_k \Spec(\sepcl k)$.
	Dado un recubrimiento étale finito $\overline{Y} \to \overline{X}$, existe una extensión finita separable $L/k$
	y un recubrimiento étale finito $Y_L \to X_L$ tal que $Y_L \times_L \Spec(\sepcl k) \cong \overline{Y}$.
\end{slem}
\begin{proof}
	Como $X$ es compacto, admite un cubrimiento finito por abiertos afines $\{ U_j = \Spec(A_j) \}_{j=1}^n$ y,
	como los morfismos finitos son afines, podemos suponer que la preimagen de $\overline{U}_j := U_j \times_k \Spec(\sepcl k)$ en $\overline{Y}$
	es un abierto afín $\Spec(\overline{B}_j)$, donde $\overline{B}_j$ es un $A_j \otimes_k \sepcl k$-módulo finitamente generado.
	Así, $\overline{B}_j = (A_j \otimes_k \sepcl k)[x_1, \dots, x_m]/\mathfrak{a}_j$, donde $\mathfrak{a}_j$ es un ideal generado
	por finitos polinomios $f_1, \dots, f_r$, donde cada uno con finitos términos involucra finitos elementos separables sobre $k$.
	Es decir, existe $L/k$ finita separable, donde cada $f_i \in (A_j \otimes_k L)[\vec x]$,
	de modo que el esquema afín $B_j := (A_j \otimes_k L)[\vec x]/(f_1, \dots, f_r)$ satisface que $B_j \otimes_L \sepcl k \cong \overline{B}_j$.
	Pegándolos, obtenemos al recubrimiento étale $Y_L \to X_L$.
\end{proof}
\addtocounter{thmi}{-1}
\begin{prop}
	Sea $X$ un esquema compacto y geométricamente íntegro sobre un cuerpo $k$, y sea $\overline{X} := X \times_k \Spec(\sepcl k)$.
	Fijemos un punto geométrico $\overline{x} \in \overline{X}(\Omega)$ que, mediante $\overline{X} \to X$, también identificamos con un
	punto geométrico de $X$. Entonces la siguiente sucesión es exacta:
	\[\begin{tikzcd}
		1 \rar & \pi_1^{\text{ét}}(\overline{X}, \overline{x}) \rar & \pi_1^{\text{ét}}(X, \overline{x}) \rar & \Gal(\sepcl k/k) \rar & 1.
	\end{tikzcd}\]
\end{prop}
\begin{proof}
	Para probar inyectividad de $\pi_1^{\text{ét}}(\overline{X}, \overline{x}) \to \pi_1^{\text{ét}}(X, \overline{x})$
	y sobreyectividad de $\pi_1^{\text{ét}}(X, \overline{x}) \to \Gal(\sepcl k/k)$ basta aplicar las proposiciones anteriores.
	Así, queda verificar exactitud en $\pi_1^{\text{ét}}(X, \overline{x})$, para lo que aplicamos el corolario~\ref{thm:homotopy_exactness}.
	En él, la condición \ref{thm:homotopy_exactness_i} es trivial, así que verifiquemos \ref{thm:homotopy_exactness_ii}:
	sea $Y \to X$ un recubrimiento de Galois finito tal que el recubrimiento $Y_{\sepcl k} \to \overline{X}$ posee una sección.
	Como $X$ es íntegro, la fibra genérica de $Y \to X$ es el espectro de una extensión finita de Galois $K$
	del cuerpo de funciones $k(X)$ que, tras tensorizar por $\sepcl{k}$, se escribe como suma directa de $\sepcl{k(X)}$.
	Por tanto, $K \cong k(X) \otimes_k L$ para una extensión de Galois $L/k$.
	Luego el recubrimiento de Galois finito $X_L \to X$ tiene la misma fibra genérica que $Y$, es decir, existe un abierto denso $U \subseteq X$
	tal que $X_L \times_X U \cong Y \times_X U$. Como $Y$ y $X_L$ son $U$-esquemas finitos localmente libres, se sigue que $Y \cong X_L$.
\end{proof}

\section{Sitios, haces y cohomologías}
\newcommand{\Cov}{{\rm Cov}}
\newcommand{\Cat}{\operatorname{Cat}}
\begin{mydef}
	Sea $\catC$ una categoría pequeña con productos fibrados.
	Fijado un objeto $U \in \Obj\catC$ y dado un par de conjuntos $\mathcal{S}_1 := \{ U_i \}_{i\in I}, \mathcal{S}_2 := \{ V_j \}_{j\in J}$
	de objetos de $\catC/U$, se dice que $\mathcal{S}_2$ es un \strong{refinamiento}\index{refinamiento} de $\mathcal{S}_1$ si para todo $i \in I$
	existe un $j \in J$ y una flecha $U_i \to V_j$ (de $\catC/U$).

	Se le llama una \strong{(pre)topología de Grothendieck}\index{topología!(de Grothendieck)} a una familia $J := \{ \Cov_U \}_{U \in \Obj\catC}$,
	tal que para cada objeto $U \in \Obj\catC$, se cumple que $\Cov_U$ es una familia de conjuntos de flechas, llamados \strong{cubrimientos}\index{cubrimiento}
	que satisface lo siguiente:
	\begin{enumerate}[{COV}1., left=.8em]
		\item Para todo isomorfismo $V \to U$ se cumple que $\{ V \to U \}$ es un cubrimiento.
		\item El refinamiento de un cubrimiento es también un cubrimiento.
		% \item Si $\mathcal{S}_1 \in \Cov_U$ y $\mathcal{S}_2$ es un refinamiento de $\mathcal{S}_1$, entonces $\mathcal{S}_2 \in \Cov_U$.
		\item El cambio de base de un cubrimiento induce un cubrimiento.
			Vale decir, dado un cubrimiento $\mathcal{S} := \{ U_i \to U \}_{i\in I} \in \Cov_U$ y una flecha $V \to U$,
			la familia $\mathcal{S} \times_U V := \{ U_i \times_U V \to V \}_{i\in I} \in \Cov_V$.
		\item Sea $\mathcal{S}_1 := \{ U_i \to U \}_{i\in I} \in \Cov_U$ y $\mathcal{S}_2 := \{ V_j \to U \}_{j\in J}$ una familia de flechas.
			Si para cada $U_i$ se cumple que $\mathcal{S}_2 \times_U U_i \in \Cov_{U_i}$, entonces $\mathcal{S}_2 \in \Cov_U$.
	\end{enumerate}
	Un \strong{sitio}\index{sitio} es un par $X := (\catC, J)$, donde $J$ es una (pre)topología de Grothendieck sobre $\catC$.
	Denotamos $\Cat(X) := \catC$.
\end{mydef}
La definición de una topología de Grothendieck es otra, pero uno puede probar que una pretopología induce de manera única una topología.

\begin{ex}
	Sea $X$ un espacio topológico.
	Definimos el \textit{sitio topológico} $X_{\rm top}$ como el sitio con $\Cat(X_{\rm top}) := \Open(X)$
	y donde los cubrimientos coinciden con la definición previa de <<cubrimiento>>.
\end{ex}
Éste es un ejemplo sencillo, pero queremos ampliar un poco más la definición, en particular a sitios que capturen ciertas propiedades de morfismos de esquemas.

\begin{mydef}
	Sea $\catC$ una categoría concreta (e.g., $\mathsf{Sch}$).
	Se dice que una familia de flechas de codominio fijo $\{ f_i\colon V_i \to U \}$ es \strong{colectivamente suprayectiva}\index{colectivamente!suprayectiva}
	si $U = \bigcup_{i\in I} f_i[V_i]$.
\end{mydef}

% La solución que vamos a aplicar es un tanto radical.
% Las categorías subyacentes casi siempre serán $\mathsf{Sch}/S$ con condiciones sobre los cubrimientos:
\newcommand{ \zarsite}[1]{#1_{\rm Zar}}
\newcommand{  \etsite}[1]{#1_{\text{ét}}}
\begin{exn}
	Sea $S$ un esquema.
	Un \strong{cubrimiento por Zariski abiertos} (resp. \strong{cubrimiento étale}\index{cubrimiento!étale}) es una familia $\{ \varphi_i \colon U_i \to X \}$
	de encajes abiertos (resp. morfismos étale) que es colectivamente suprayectiva.

	El \strong{gran sitio de Zariski} (resp. \strong{gran sitio étale}), denotado $\zarsite{(\mathsf{Sch}/S)}$ (resp. $\etsite{(\mathsf{Sch}/S)}$),
	es aquél que tiene por categoría subyacente a $\mathsf{Sch}/S$ con los cubrimientos por Zariski abiertos (resp. cubrimientos étale).

	El \strong{(pequeño) sitio de Zariski}\index{sitio!de Zariski}, denotado $\zarsite S$, es aquél que tiene por categoría los
	subesquemas abiertos de $S$ (por objetos) con los encajes abiertos (por flechas), y cuyos cubrimientos son los cubrimientos por Zariski abiertos.
	El \strong{(pequeño) sitio étale}\index{sitio!étale}, denotado $\etsite S$, es aquél que tiene por categoría
	los $S$-esquemas étale (por objetos) con los morfismos de esquemas (por flechas), y cuyos cubrimientos son los cubrimientos étale.
	% En general tendremos dos clases de sitios relacionados a $S$, los <<pequeños>> y los <<grandes>>,
	% donde el sitio grande tendrá por categoría subyacente a $\mathsf{Sch}/S$.
	% \begin{enumerate}
	% 	\item Decimos que un \strong{cubrimiento por Zariski abiertos} es una familia de morfismos de esquemas $\{ \varphi_i \colon U_i \to X \}$
	% 		que son colectivamente suprayectivos y tales que cada uno es un encaje abierto.
	% 		El \strong{gran sitio de Zariski}, denotado $\zarsite{(\mathsf{Sch}/S)}$, es aquél que tiene por categoría subyacente a $\mathsf{Sch}/S$
	% 		con los cubrimientos por Zariski abiertos.
	% 		El \strong{(pequeño) sitio de Zariski}\index{sitio!de Zariski}, denotado $S_{\rm Zar}$, es el subsitio de $\zarsite{(\mathsf{Sch}/S)}$
	% 		cuya categoría son los subesquemas abiertos de $S$ (por objetos) con los encajes abiertos (por flechas).
	% 	\item Un \strong{cubrimiento étale}\index{cubrimiento!étale} es una familia de morfismos étale $\{ \varphi_i \colon U_i \to X \}$
	% 		que son colectivamente suprayectivos.
	% 		El \strong{gran sitio étale}, denotado $\etsite{(\mathsf{Sch}/S)}$, tiene por categoría subyacente a $\mathsf{Sch}/S$
	% 		con los cubrimientos étale.
	% 	\item Un \strong{cubrimiento fpqc}\index{cubrimiento!fpqc}%
	% 		es una familia colectivamente suprayectiva de morfismos planos $\{ \varphi_i \colon X_i \to Y \}$ que satisfacen lo siguiente:
	% 		\paragraph{(fpqc):} Para cada abierto afín $U \subseteq Y$, existen finitos abiertos afines $\{ U_j \subseteq X_{i_j} \}_{j \in J}$
	% 		tales que la familia $\{ \varphi_{i_j}|_{U_j} \colon U_j \to U \}_{j\in J}$ es colectivamente suprayectiva.
	% 		El \strong{(gran) sitio fpqc}\index{sitio!fpqc}, denotado $\fpqcsite X$
	% 		\qedhere
	% \end{enumerate}
\end{exn}

Vamos a ver más ejemplos, pero antes un par de definiciones:
\begin{mydef}
	Un morfismo de esquemas $f \colon X \to Y$ se dice \strong{localmente de presentación finita}\index{localmente!de presentación finita (morfismo)}
	si para todo punto $x \in X$ existen un par de entornos afines $x \in U$ y $f(x) \in V$ tales que $f[U] \subseteq V$,
	y tales que $\mathscr{O}_X(U)$ es una $\mathscr{O}_Y(V)$-álgebra de presentación finita.

	Un morfismo de esquemas $f \colon X \to Y$ se dice \strong{fppf}\index{morfismo!fppf}%
	\footnote{Del fr., abrev. de \textit{fidèlement plat et presentation fini}.}
	si es fielmente plano y localmente de presentación finita.
	Se dice que $f$ es \strong{fpqc}\index{morfismo!fpqc}%
	\footnote{Del fr., abrev. de \textit{fidèlement plat et quasi-compact:} fielmente plano y (cuasi)compacto.}
	si es fielmente plano y todo abierto compacto de $Y$ es la imagen de un abierto compacto de $X$.
\end{mydef}
\begin{ex}
	Si $X, Y$ son localmente noetherianos, entonces para un morfismo $X \to Y$ ser <<localmente de presentación finita>> equivale a
	ser <<localmente de tipo finito>>.
\end{ex}

\newcommand{\fpqcsite}[1]{#1_{\rm fpqc}}
\newcommand{\fppfsite}[1]{#1_{\rm fppf}}
Ahora dos ejemplos más:
\begin{exn}
	Sea $S$ un esquema.
	Un \strong{cubrimiento fppf}\index{cubrimiento!fppf} (resp. \strong{cubrimiento fpqc}\index{cubrimiento!fpqc}) es una familia
	$\{ \varphi_i \colon Y_i \to Y \}_{i\in I}$ tal que $\coprod_{i\in I} Y_i \to X$ es un morfismo fppf (resp. fpqc).

	El \strong{(gran) sitio fppf}\index{sitio!fppf} (resp. \strong{(gran) sitio fpqc}\index{sitio!fpqc}), denotado $\fppfsite S$
	(resp. $\fpqcsite S$), es aquel que tiene por categoría subyacente $\mathsf{Sch}/S$ y cuyos cubrimientos son los fppf (resp. fpqc).
\end{exn}
En general respecto a los sitios de Zariski y étale nos referiremos a ellos como los sitios pequeños,
mientras que en los sitios fppf y fpqc nos referiremos a los sitios grandes.

La noción de sitios tiene como finalidad poder establecer la siguiente definición:
\begin{mydef}
	Sea $X$ un sitio.
	Un \strong{prehaz} (con valores en $\catC$) es un funtor $\mathscr{F} \colon \Cat(X)^{\rm op} \to \catC$.
	Se dice que un prehaz $\mathscr{F}$ es un \strong{haz} si la sucesión
	\begin{equation}
		\begin{tikzcd}[row sep=large]
			\mathscr{F}(U) \rar["\alpha"] & \prod_{i} \mathscr{F}(U_i) \rar["\beta", shift left] \rar["\gamma"', shift right]
						      & \prod_{i,j} \mathscr{F}(U_i \times_U U_j)
		\end{tikzcd}
		\label{cd:sheaf_site}
	\end{equation}
	es exacta para todo cubrimiento $\{ U_i \to U \}$,
	donde las flechas son las mismas que en \eqref{cd:sheaf_equalizer}.

	Los prehaces (resp. haces) sobre $X$ con valores en $\catC$ conforman una categoría denotada $\mathsf{PSh}(X; \catC)$ (resp. $\mathsf{Sh}(X; \catC)$).
	Cuando $\catC = \mathsf{Ab}$ decimos que los objetos de $\mathsf{PSh}(X; \catC)$ (resp. $\mathsf{Sh}(X; \catC)$) se dicen prehaces (resp. haces) abelianos.
\end{mydef}
\begin{ex}
	Si $X$ es un espacio topológico, entonces las nociones de haz sobre $X$ (como haz sobre un espacio)
	y haz sobre $X_{\rm top}$ (como haz sobre un sitio) coinciden.
\end{ex}
La condición \eqref{cd:sheaf_site} se verifica sobre cubrimientos, por tanto, mientras más \emph{fina} sea la topología de Grothendieck
(<<tiene más abiertos>>), más verificaciones.
% En consecuencia, si tenemos un prehaz sobre $\mathsf{Sch}/S$ que es un haz en la topología fpqc, entonces será un haz en las topologías fppf, étale y de Zariski.

Nótese que, por la proposición~\ref{thm:unram_et_prop}, todo cubrimiento por Zariski abiertos es étale;
en consecuencia, si nos restringimos a los sitios grandes (para tener la misma categoría subyacente), la topología étale es más fina que la de Zariski.
\begin{mydef}
	Sean $X, Y$ un par de sitios.
	Una \strong{aplicación continua}\index{aplicación!continua (sitios)} $f \colon X \to Y$ es un funtor $F := f^t \colon \Cat(Y) \to \Cat(X)$
	que preserva cubrimientos, vale decir:
	\begin{enumerate}[{AC}1.]
		\item Para todo cubrimiento $\{ V_i \morf{\varphi_i} V \}_{i\in I}$ en $Y$,
			se cumple que $\{ FV_i \morf{F(\varphi_i)} FV \}_{i\in I}$ es un cubrimiento en $X$.
		\item Para todo cubrimiento $\{ V_i \morf{\varphi_i} V \}_{i\in I}$ y toda flecha $W \to V$ en $Y$,
			se cumple que $F(V_i \times_V W) \cong FV_i \times_{FV} FW$.
	\end{enumerate}
\end{mydef}
\begin{ex}
	Sean $X, Y$ un par de espacios topológicos y sea $f \colon X \to Y$ una función continua.
	Definimos $f_{\rm top} \colon X_{\rm top} \to Y_{\rm top}$ como el funtor $V \mapsto f^{-1}[V]$; entonces $f_{\rm top}$ es continuo.
\end{ex}

\warn
Empleamos la nomenclatura <<aplicación continua>> (siguiendo a \citeauthor{poonen:rational}~\cite{poonen:rational}) en lugar de <<morfismo de sitios>>,
pues éste último lo reservamos para otros propósitos (\cite{stacks}).

Sobre un conjunto $X$, dadas dos topologías $\tau_1, \tau_2$, se cumple que $\tau_1$ es más fina que $\tau_2$ es equivalente a que la identidad
$\Id \colon (X, \tau_1) \to (X, \tau_2)$ sea continua.
Análogamente, sobre una categoría fija $\catC$, dos topologías de Grothendieck $\mathcal{T}_1, \mathcal{T}_2$ satisfacen que $\mathcal{T}_1$ es más fina
que $\mathcal{T}_2$ si $\Id \colon (\catC, \mathcal{T}_1) \to (\catC, \mathcal{T}_2)$ es continua.
Así, tenemos lo siguiente:
\begin{prop}
	Sea $S$ un esquema, entonces son aplicaciones continuas:
	\begin{center}
		\begin{tikzcd}
			\fpqcsite S \rar & \fppfsite S \rar & \etsite{(\mathsf{Sch}/S)} \rar & \zarsite{(\mathsf{Sch}/S)}.
		\end{tikzcd}
	\end{center}
	Vale decir, todo cubrimiento por Zariski abiertos es étale, todo cubrimiento étale es fppf, y todo cubrimiento fppf es fpqc.
\end{prop}
\begin{cor}
	Sea $S$ un esquema y sea $\mathscr{F} \colon (\mathsf{Sch}/S)^{\rm op} \to \mathsf{Set}$ un funtor.
	Si $\mathscr{F}$ es un haz en la topología fpqc, entonces también lo es en las topologías fppf, étale y de Zariski.
\end{cor}

Nótese que el recíproco no es cierto.
Así que convendría saber un criterio para cuando un funtor es un haz en la
topología fpqc y afortunadamente tenemos el siguiente corolario de la teoría de descenso:
\begin{thm}\label{thm:repr_fun_are_fpqc_sh}
	Sea $S$ un esquema.
	Todo funtor representable $(\mathsf{Sch}/S)^{\rm op} \to \mathsf{Set}$ es un haz en la topología fpqc.
\end{thm}

% La situación es un tanto delicada.
% Por un lado, hay que elegir una categoría cuyos elementos sean <<candidatos a abiertos>> del esquema, y por otro lado, hay que elegir cubrimientos para los
% abiertos.
% Vale decir, no nos basta simplemente que 

% \begin{mydef}
% 	Sea $S$ un esquema.
% 	Una familia de morfismos de esquemas $\mathcal{F} := \{ \varphi_i \colon X_i \to S \}$ se dice un \strong{cubrimiento fpqc}\index{cubrimiento!fpqc} si:
% 	\begin{enumerate}
% 		\item Cada $\varphi_i$ es un morfismo plano y $\mathcal{F}$ es colectivamente suprayectiva.
% 		\item Para cada abierto afín $U \subseteq S$, existen finitos abiertos afines $\{ U_j \subseteq X_{i_j} \}_{j\in J}$,
% 			tales que $U = \bigcup_{j\in J} \varphi_{i_j}[U_j]$.
% 	\end{enumerate}
% 	El \strong{sitio fpqc grande}\index{sitio!fpqc!grande} es el que tiene por categoría subyacente a los esquemas,
% 	y por cubrimientos a los cubrimientos fpqc.
% \end{mydef}
% \begin{ex}
% 	Son cubrimientos fpqc:
% 	\begin{enumerate}
% 		\item Todo cubrimiento de Zariski abiertos.
% 		\item Todo cubrimiento étale.
% 		\item Un morfismo $\{ \varphi^a\colon \Spec B \to \Spec A \}$ es un cubrimiento fpqc syss $\varphi \colon A \to B$ es fielmente plano.
% 		\item Si $\{ f \colon X \to Y \}$ es un morfismo plano, suprayectivo y compacto, entonces $\{ f \colon X \to Y \}$ es un cubrimiento fpqc.
% 	\end{enumerate}
% \end{ex}

% La razón para estudiar la topología fpqc es que, por el ejemplo anterior, es más fina que la topología de Zariski o la topología étale sobre $\mathsf{Sch}/S$.
% Así que, veremos cómo estudiarla apropiadamente:

\begin{mydef}
	Sea $X$ un esquema y $\catC \subseteq \mathsf{Sch}/X$.
	Sea $\mathscr{F}$ un $\mathscr{O}_X$-módulo cuasicoherente (en particular, un haz sobre $\zarsite X$).
	Se define un prehaz $\mathscr{F}_\catC$ dado por:
	$$ \Gamma(U, \mathscr{F}_\catC) := \Gamma(U, p^*\mathscr{F}) $$
	para cada objeto $U \morf{p} X$ en $\catC$.
\end{mydef}
\begin{prop}
	Sea $X$ un esquema y $\mathscr{F}$ un $\mathscr{O}_X$-módulo cuasicoherente.
	Entonces $\mathscr{F}_\tau$ es un haz sobre $X_\tau$ para todo $\tau \in \{ \rm fpqc, fppf, \text{ét}, Zar \}$.
\end{prop}

\begin{prop}
	Sea $X$ un sitio y $\mathscr{F} \colon \Cat(X)^{\rm op} \to \catC$ un prehaz de conjuntos o de grupos abelianos.
	Entonces $\mathscr{F}$ posee una hazificación $\mathscr{F}^+$ y este determina un funtor.
\end{prop}
La demostración es esencialmente la misma que detallamos anteriormente.
De hecho, las construcciones relativas a haces sobre sitios es análoga a la de haces sobre espacios topológicos, por lo que
dejaremos los detalles al lector.

\begin{mydef}
	Sea $X$ un sitio y $\alpha \colon \mathscr{F \to G}$ un morfismo de prehaces abelianos sobre $X$.
	Se definen los siguientes prehaces:
	\begin{gather*}
		\Gamma(U, \ker\alpha) := \ker(\alpha_U), \qquad \Gamma(U, (\Img\alpha)^-) := \Img(\alpha_U), \\
		\Gamma(U, (\coker\alpha)^-) := \coker(\alpha_U).
	\end{gather*}
\end{mydef}
\begin{prop}
	Sea $X$ un sitio y $\alpha \colon \mathscr{F \to G}$ un morfismo de haces abelianos sobre $X$.
	Entonces $\ker\alpha$ es un haz sobre $X$, y
	$$ \Img\alpha := \big( (\Img\alpha)^- \big)^+, \qquad \coker\alpha := \big( (\coker\alpha)^- \big)^+ $$
	son haces sobre $X$; más aún, efectivamente son el núcleo, la imagen y el conúcleo en $\mathsf{Sh}(X; \mathsf{Ab})$.
\end{prop}

\begin{prop}
	Sea $X$ un sitio y 
	\begin{tikzcd}[cramped, sep=small]
		0 \rar & \mathscr{F} \rar["\alpha"] & \mathscr{G} \rar["\beta"] & \mathscr{H}
	\end{tikzcd}
	una sucesión exacta de haces abelianos.
	Entonces para todo $U \in \Obj\Cat(X)$, la sucesión inducida
	\begin{center}
		\begin{tikzcd}
			0 \rar & \Gamma(U, \mathscr{F}) \rar["\alpha"] & \Gamma(U, \mathscr{G}) \rar["\beta"] & \Gamma(U, \mathscr{H})
		\end{tikzcd}
	\end{center}
	es exacta (en $\mathsf{Ab}$).
	Vale decir, el funtor $\Gamma(U, -) \colon \mathsf{Sh}(X; \mathsf{Ab}) \to \mathsf{Ab}$ es exacto por la izquierda.
\end{prop}

\begin{thm}
	Para todo sitio $X$, la categoría $\mathsf{Sh}(X; \mathsf{Ab})$ es abeliana y tiene suficientes inyectivos.
\end{thm}
\begin{proof}
	Mejor aún, la categoría es de Grothendieck por el Thm.~18.1.6 en \citeauthor{kashiwara:sheaves}~\cite[437]{kashiwara:sheaves}.
\end{proof}

Como corolario, todo funtor desde $\mathsf{Sh}(X; \mathsf{Ab})$ que sea exacto por la izquierda admite funtores derivados,
he aquí una lista de los principales ejemplos:
\begin{exn}
	Sean $X$ y $S_\tau$ un par de sitios, donde $S$ es un esquema y $\tau \in \{ \rm fpqc, fppf, \text{ét}, Zar \}$.
	\begin{enumerate}
		\item Para todo objeto $U \in \Cat(X)$,
			el funtor de secciones globales
			\[
				\Gamma(U, -) \colon \mathsf{Sh}(X; \mathsf{Ab}) \longrightarrow \mathsf{Ab}
			\]
			es exacto por la izquierda y su funtor derivado se denota $H_X^q(U, -) := \rder^p \Gamma(U, -)$.
			Cuando $X = S_\tau$, se denota $H_\tau^q(U, -) := H_{S_\tau}^q(U, -)$.
		\item El funtor inclusión $i \colon \mathsf{Sh}(X; \mathsf{Ab}) \to \mathsf{PSh}(X; \mathsf{Ab})$ es exacto por la izquierda
			y su funtor derivado se denota $\mathscr{H}^q(X, -) := \rder^q i$.
			Cuando $X = S_\tau$, de denota $\mathscr{H}^q_\tau(S, -) := \mathscr{H}^q(S_\tau, -)$.
		\item Para todo haz $\mathscr{F} \in \mathsf{Sh}(X; \mathsf{Ab})$,
			el funtor $\Hom(\mathscr{F}, -)$ es exacto por la izquierda
			y su funtor derivado se denota $\Ext_X^q(\mathscr{F}, -) := \rder^p\Hom_X(\mathscr{F}, -)$.
			Cuando $X = S_\tau$, se denota $\Ext_\tau^q(\mathscr{F}, -) := \Ext_{S_\tau}^q(\mathscr{F}_\tau, -)$.
		\item Para todo haz $\mathscr{F} \in \mathsf{Sh}(X; \mathsf{Ab})$,
			el haz Hom $\shHom_X(\mathscr{F}, -)$ es exacto por la izquierda y su funtor derivado se denota
			$\shExt_X^q(\mathscr{F}, -) := \rder^q \shHom_X(\mathscr{F}, -)$.
		\item Sea $f \colon X \to Y$ una aplicación continua de sitios.
			Entonces el funtor $f_* \colon \mathsf{Sh}(Y; \mathsf{Ab}) \to \mathsf{Sh}(X; \mathsf{Ab})$ es exacto por la izquierda.
			A su funtor derivado $\rder^q f_*(-)$ se le llaman \strong{imágenes directas superiores}.
			\qedhere
	\end{enumerate}
\end{exn}

\begin{cor}
	Sea $X$ un sitio y $U \in \Cat(X)$ un objeto.
	Para todo haz abeliano $\mathscr{F} \in \mathsf{Sh}(X; \mathsf{Ab})$, se cumplen:
	\begin{enumerate}
		\item $H_X^q(U, \mathscr{F}) := \Ext_X^q(\Z, \mathscr{F})$, donde $\Z$ denota el haz constante.
		\item $\Gamma( U, \mathscr{H}^q(X, \mathscr{F}) ) = H^q_X(U, \mathscr{F})$.
		\item Sea $X = S_\tau$, donde $S$ es un esquema y $\tau \in \{ \rm fpqc, fppf, \text{ét}, Zar \}$.
			Dado un homeomorfismo universal $\pi \colon S' \to S$ (de esquemas),
			se cumple que $\pi_*\colon \mathsf{Sh}(S_\tau; \mathsf{Ab}) \cong \mathsf{Sh}(S_\tau^\prime; \mathsf{Ab})$ es una
			equivalencia de categorías.
			Denotando $\mathscr{F}' := \pi_*\mathscr{F}$ se satisface entonces:
			\begin{gather*}
				H^q_\tau(S', \mathscr{F}') \cong H^q_\tau(S, \mathscr{F}), \qquad
				\Ext^q_\tau(S', \mathscr{F}') \cong \Ext^q_\tau(S, \mathscr{F}), \\
				\pi_*\mathscr{H}^q_\tau(S^\prime, \mathscr{F}') = \mathscr{H}^q_\tau(S, \mathscr{F}).
			\end{gather*}
	\end{enumerate}
\end{cor}

% \begin{mydef}
% 	Sea $X$ un sitio.
% 	Definimos el funtor
% 	\[
% 		H^q(X, -) \colon \mathsf{Sh}(X; \mathsf{Ab}) \longrightarrow \mathsf{Ab}
% 	\]
% 	como el $q$-ésimo funtor derivado derecho del funtor $\Gamma(X, -)$.
% 	Si $S$ es un esquema y $X = S_\tau$ (con ), entonces denotamos $H^q_\tau(S, -)$ para enfatizar
% 	la topología de Grothendieck.
% \end{mydef}
% Es decir, para toda sucesión exacta 
% \begin{tikzcd}[cramped, sep=small]
% 	0 \rar & \mathscr{F} \rar & \mathscr{G} \rar & \mathscr{H} \rar & 0
% \end{tikzcd}
% de haces abelianos sobre $S_\tau$, se induce la siguiente sucesión exacta larga:
% \begin{center}
% 	\includegraphics{cats/top_cohomology.pdf}
% \end{center}

\subsection{Cohomología de \v Cech}
Sea $X$ un sitio y sea $\mathcal{U} := \{ U_i \to U \}_{i\in I}$ un cubrimiento.
Para una tupla de índices $(i_0, \dots, i_p) \in I^{p+1}$ se define
$$ U_{i_0, \dots, i_p} := U_{i_0} \times_U U_{i_1} \times_U \cdots \times_U U_{i_p}. $$
Sea $(i_0, \dots, i_p) \in I^{p+1}$ una tupla y $j \in \{ 0, 1, \dots, p \}$; entonces obtenemos
una proyección $\rho_j \colon U_{i_0, \dots, i_p} \to U_{i_0, \dots, \hat{i}_j, \dots, i_p}$, donde <<$\hat{i}_j$>> denota borrar la $j$-ésima coordenada.
\begin{mydef}
	Se define
	$$ \check C^q(\mathcal{U}, \mathscr{F}) := \prod_{(i_0, \dots, i_p) \in I^{p+1}} \mathscr{F}(U_{i_0, \dots, i_p}), $$
	y definimos el homomorfismo de grupos:
	\begin{align*}
		\ud^q \colon \check C^q &\longrightarrow \check C^{q-1} \\
		s &\longmapsto \sum_{j=0}^{q} (-1)^j \rho_j(s)
	\end{align*}
	Así, es fácil verificar que $\big(\check C^\bullet(\mathcal{U}, \mathscr{F}), \ud^\bullet\big)$ es un complejo de cocadenas,
	llamado el \strong{complejo de \v Cech}\index{complejo!de \v Cech}.
	
	Siguiendo la terminología usual, los elementos de $\check C^q, \ker(\ud^q), \Img(\ud^{q-1})$
	se dicen \strong{cocadenas}, \strong{cociclos} y \strong{cobordes $q$-ésimos de \v Cech} resp.
	También se denomina \strong{$q$-ésimo grupo de cohomología de \v Cech}\index{grupo!de cohomología!de \v Cech} a:
	$$ \check H^q(\mathcal{U}, \mathscr{F}) := H^q( \check C^\bullet(\mathcal{U}, \mathscr{F}) ) = \frac{\ker(\ud^q)}{\Img(\ud^{q-1})}. $$
\end{mydef}

\begin{prop}
	Sea $\mathscr{F}$ un haz abeliano sobre un sitio $X$.
	Entonces, las flechas canónicas satisfacen lo siguiente:
	\begin{center}
		\begin{tikzcd}[row sep=tiny]
			\check H^0(U, \mathscr{F}) \rar["\sim"] & H^0(U, \mathscr{F}) = \mathscr{F}(U) \\
			\check H^1(U, \mathscr{F}) \rar["\sim"] & H^1(U, \mathscr{F})                  \\
			\check H^2(U, \mathscr{F}) \rar[hook]   & H^2(U, \mathscr{F})
		\end{tikzcd}
	\end{center}
\end{prop}
\begin{proof}
	Esto es una aplicación de la sucesión espectral de cohomología de \v Cech.
\end{proof}

% Ahora veamos la noción de \textit{cohomología de \v Cech}.
\begin{thm}[M. Artin]
	Sea $X$ un esquema compacto tal que cada conjunto finito de puntos de $X$
	esté contenido en un entorno afín (e.g., si $X$ es cuasiproyectivo sobre un esquema afín).
	Entonces, para todo haz abeliano $\etsite{\mathscr{F}}$ sobre $\etsite X$ tenemos que el homomorfismo canónico:
	\begin{center}
		\begin{tikzcd}[row sep=large]
			\etsite{\check H}^q(X, \mathscr{F}) \rar["\sim"] & \etsite H^q(X, \mathscr{F})
		\end{tikzcd}
	\end{center}
	es un isomorfismo para $q \in \N$.
\end{thm}

\begin{prop}[lema de Cartan]
	Sea $\mathscr{F}$ un haz abeliano sobre un esquema $S$ y sea $\mathscr{K}$ la clase de morfismos étale $U \to X$ que satisfacen lo siguiente:
	\begin{enumerate}
		\item Si $U, V \in \mathscr{K}$ entonces $U \times_X V \in \mathscr{K}$.
		\item Toda extensión étale $Y \to X$ admite un cubrimiento étale $\{ U_i \to Y \}_{i\in I}$ tal que cada composición $(U_i \to X) \in \mathscr{K}$.
		\item Se tiene que $\etsite{\check H}^i(U, \mathscr{F}) = 0$ para todo $U \in \mathscr{K}$ y todo $i > 0$.
	\end{enumerate}
	Entonces los homomorfismos canónicos
	\begin{center}
		\begin{tikzcd}[sep=large]
			\etsite{\check H}^i(X, \mathscr{F}) \rar["\sim"] & \etsite H^i(X, \mathscr{F})
		\end{tikzcd}
	\end{center}
	son isomorfismos.
\end{prop}

\begin{mydef}
	Sea $X$ un esquema.
	Un objeto acíclico en $\mathsf{Sh}(X_\tau, \mathsf{Ab})$ con $\tau \in \{ \rm fppf, fpqc, \text{ét}, Zar \}$ se dice un
	\strong{haz flácido}\index{haz!flácido} (en la topología respectiva).
\end{mydef}
Igual que antes, todo haz inyectivo es flácido, pero el recíproco no es cierto.

\begin{mydef}
	Sea $X$ un esquema y sea $\mathscr{L}_\tau$ un haz en la topología $\tau \in \{ \rm fppf, fpqc, \text{ét}, Zar \}$.
	Se dice que $\mathscr{L}_\tau$ es \strong{invertible}\index{haz!invertible} en la topología $\tau$ si $X$
	posee un cubrimiento $\{ U_i \to X \}$ en la respectiva topología tal que $\mathscr{L}|_{U_i} \cong \mathscr{O}_{U_i, \tau}$.

	La clase de haces invertibles (en la topología $\tau$) salvo isomorfismo de haces, se denota $\Pic_\tau X$.
	\nomenclature{$\Pic_\tau X$}{Grupo de Picard de $X$ con la topología $\tau$}
\end{mydef}
\begin{prop}
	Sea $X$ un esquema. Entonces:
	\begin{enumerate}
		\item $\zarsite H^0(X, \GG_m) \cong \etsite H^0(X, \GG_m) \cong \fppfsite H^0(X, \GG_m) \cong \Gamma(X, \mathscr{O}_X)^\times$.
		\item (Teorema de Hilbert 90) $\zarsite H^1(X, \GG_m) \cong \etsite H^1(X, \GG_m) \cong \fppfsite H^1(X, \GG_m) \cong \Pic X$.
	\end{enumerate}
\end{prop}

% \begin{mydef}
% 	Sea $\mathcal{U} := \{ t_i \colon T_i \to T \}_{i \in I}$ una familia de morfismos de esquemas.
% 	Un \strong{dato de descenso}\index{dato de descenso} es una colección $(\{ \mathscr{F}_i \}_i, \{ \varphi_{ij} \}_{i,j\in I})$ tal que:
% 	\begin{enumerate}[{Des}1.]
% 		\item Cada $\mathscr{F}_i$ es un haz cuasicoherente sobre $T_i$.
% 		\item Tenemos que $\varphi_{ij} \colon (\pi^{ij}_i)^* \mathscr{F}_i \to (\pi^{ij}_j)^* \mathscr{F}_j$ es un isomorfismo de haces;
% 			donde $T_{ij} := T_i\times_T T_j$ y $\pi^{ij}_i\colon T_{ij} \to T_i$ denota el morfismo canónico.
% 		\item \textbf{Condición de cociclos:}
% 			El siguiente diagrama conmuta:
% 			\begin{center}
% 				\begin{tikzcd}[row sep=large]
% 					(\pi^{ijk}_i)^* \mathscr{F}_i \drar["(\pi^{ijk}_{ik})^* \varphi_{ik}"'] \ar[rr, "(\pi^{ijk}_{ij})^* \varphi_{ij}"] & & (\pi^{ijk}_j)^* \mathscr{F}_j \ular["(\pi^{ijk}_{jk})^* \varphi_{jk}"] \\
% 					{} & (\pi^{ijk}_k)^* \mathscr{F}_k
% 				\end{tikzcd}
% 			\end{center}
% 	\end{enumerate}
% \end{mydef}

\subsection{Puntos geométricos y entornos étale}
\begin{mydef}
	Sea $X$ un esquema.
	Un \strong{punto geométrico}\index{punto!geométrico} de $X$ es un punto $k$-valuado (i.e., un morfismo $\Spec k \to X$),
	donde $k$ es un cuerpo separablemente cerrado.
	Se dice que $U \to X$ es un \strong{entorno étale}\index{entorno!étale} de un punto $k$-valuado $x \in X(k)$ si $U \to X$ es un morfismo étale
	y se tiene el siguiente diagrama conmutativo:
	\begin{center}
		\begin{tikzcd}[row sep=large]
			U \rar                        & X \\
			\Spec k \uar["y"] \urar["x"']
		\end{tikzcd}
	\end{center}
\end{mydef}
\begin{lem}
	Sea $X$ un esquema y $x$ un punto geométrico.
	Los entornos étale de $x$ forman un categoría filtrada.
\end{lem}

\begin{mydef}
	Sea $X$ un esquema y $x$ un punto geométrico.
	El \strong{anillo local estricto}\index{anillo!local!estricto} en $x$ es el límite directo
	$$ \mathscr{O}_{\etsite X, x} := \limdir_U \etsite\Gamma(U, \mathscr{O}_X), $$
	donde $U$ recorre los entornos étale de $x$.

	Más generalmente, si $\mathscr{F}$ es un haz sobre $\etsite X$ y $x$ es un punto geométrico de $X$,
	entonces se define la \strong{fibra}\index{fibra!(haz étale)} en $x$ como:
	$$ \mathscr{F}_{\text{ét}, x} := \limdir_U \etsite\Gamma(U, \mathscr{F}), $$
	donde $U$ recorre los entornos étale de $x$.
\end{mydef}

Ahora se sigue lo siguiente:
\begin{prop}
	Sea $X$ un esquema noetheriano.
	\begin{enumerate}
		\item Sea $\varphi \colon \mathscr{F} \to \mathscr{G}$ un morfismo de haces abelianos sobre $\etsite X$. 
			Entonces $\varphi$ es un isomorfismo syss para todo punto geométrico $x$ tenemos que $\varphi_x \colon \mathscr{F}_{\text{ét}, x}
			\to \mathscr{G}_{\text{ét}, x}$ es un isomorfismo.
		\item Sea $\mathscr{S \colon F \morf{\varphi} G \morf{\psi} H}$ una sucesión de haces abelianos sobre $\etsite X$.
			Entonces $\mathscr{S}$ es exacta (en $\mathsf{Sh}(\etsite X, \mathsf{Ab})$) syss para todo punto geométrico $x$ tenemos
			que $\mathscr{S}_{\text{ét}, x}$ es exacta (en $\mathsf{Ab}$).
	\end{enumerate}
	Más aún, si $X/k$ es un esquema algebraico podemos hacer las verificaciones sobre puntos geométricos cuya imagen sea cerrada.
\end{prop}

\warn
Al contrario de lo que sucedía antes, ésta proposición sí depende de que el sitio sea una topología étale sobre un esquema noetheriano
y es falso en un sitio general (¿cuál sería la definición de un punto geométrico?).
A ésta propiedad a veces se le llama \textit{tener suficientes puntos}.

\section{Anillos henselianos}
Recuérdese que un polinomio $f \in A[t_1, \dots, t_n]$ se dice \strong{primitivo}\index{primitivo (polinomio)} si sus coeficientes son coprimos en conjunto,
vale decir, si el ideal en $A$ generado por sus coeficientes es $(1)$.

\begin{lem}
	Sea $A$ un anillo y sea $f \in B := A[t_1, \dots, t_n]$.
	\begin{enumerate}
		\item Si $f$ es primitivo, entonces es un elemento regular de $B$ y $B/(f)$ es una $A$-álgebra plana.
		\item Supongamos que $n = 1$.
			Entonces $f$ es primitivo syss $\Spec(A[t]/(f)) \to \Spec A$ tiene fibras finitas.
	\end{enumerate}
\end{lem}
\begin{proof}
	\begin{enumerate}
		\item Nótese que $f$ es primitivo syss para todo $\mathfrak{p} \in \Spec A$ se cumple que $f \mod{\mathfrak{p}} \ne 0$.
			Veamos el endomorfismo (de $A$-módulos) $\times f \colon B \to B$ y sea $K := \ker(\times f) = B[f]$ la $f$-torsión.
			Como $B$ es de presentación finita y plano sobre $A$, entonces podemos tensorizar $\otimes\kk(\mathfrak{p})$ para todo $\mathfrak{p} \in
			\Spec A$, y vemos que $K \otimes \kk(\mathfrak{p}) = 0$, por lo que $K = 0$.
			Así $\times f$ es inyectivo y el conúcleo $B/(f)$ es plano sobre $A$ (por la sucesión exacta en $\Tor$).
		\item Basta notar que exigir que las fibras de $\Spec(A[t]/(f)) \to \Spec A$ sean finitas equivale a ver que
			$\kk(\mathfrak{p})[t]/(f \mod{\mathfrak{p}})$ sea un $\kk(\mathfrak{p})$-espacio vectorial de dimensión finita,
			lo que equivale a que $f \mod{\mathfrak{p}} \ne 0$.
			\qedhere
	\end{enumerate}
\end{proof}

\begin{prop}
	Sea $(A, \mathfrak{m}, k)$ un anillo local, y denotemos por $s := x_{\mathfrak{m}} \in \Spec A =: S$.
	Son equivalentes:
	\begin{enumerate}
		\item Toda $A$-álgebra finitamente generada (como módulo) es un producto de anillos locales.
		\item Para todo $f \in A[t]$ mónico, la $A$-álgebra $A[t]/(f)$ es un producto de anillos locales.
		\item Sea $X$ un $S$-esquema separado de tipo finito.
			Entonces existe una partición $X = Y \amalg X_1 \amalg \cdots \amalg X_r$ en subesquemas abiertos y cerrados,
			tal que toda componente irreducible de $Y_s$ tiene dimensión $\ge 1$ y tal que cada $X_i = \Spec(B_i)$,
			donde cada $B_i$ es una $A$-álgebra local finitamente generada.
		\item Dado $f \in A[t]$ y una factorización $f \equiv g_0\cdot h_0 \pmod{k[t]}$ tal que $g_0$ sea mónico y
			$g_0, h_0 \in k[t]$ son coprimos, existen $g, h \in A[t]$ tales que $g$ es mónico, $f = g\cdot h$ y $g \equiv g_0, h \equiv h_0 \pmod{k[t]}$.
	\end{enumerate}
\end{prop}

\begin{prop}
	Sea $f \colon X \to S$ un morfismo suave.
	Sea $s \in S$ y sea $x \in X_s$ un punto cerrado de la fibra tal que la extensión $\kk(x)/\kk(s)$ sea separable.
	Entonces existe un entorno abierto $U$ de $s$ y un subesquema $Z \subseteq f^{-1}[U]$ con $x \in Z$ tal que $f|_Z \colon Z \to U$ es étale.
	\begin{center}
		\begin{tikzcd}[sep=small]
			\Spec\kk(x) \drar \ar[dd, closed] \ar[dddd, bend right, "\text{étale}"'] \\
			{} & Z \drar[hook] \\
			X_s \ar[dd] \ar[rr] & & X \ar[dd, "f"', "\text{suave}"] \\
			{} & U \ar["\text{étale}"', crossing over, near start, from=uu] \drar[open] \\
			\Spec\kk(s) \ar[rr] \urar & & S
		\end{tikzcd}
	\end{center}
\end{prop}
\begin{cor}
	Sea $f\colon X \to S$ un morfismo sobreyectivo suave.
	Existe un morfismo étale sobreyectivo $T \to S$ y una sección $\sigma \colon T \to X_T$
	(i.e., tal que $\sigma \circ f_T = \Id_T$).
\end{cor}

\begin{prop}
	Sea $f \colon X \to Y$ un morfismo de esquemas.
	Son equivalentes:
	\begin{enumerate}
		\item $f$ es afín y $f_*\mathscr{O}_X$ es un $\mathscr{O}_Y$-módulo localmente libre de rango finito.
		\item $f$ es un morfismo finito, plano y de presentación finita.
		\item Para todo abierto afín $V = \Spec A \subseteq Y$ su preimagen es afín $f^{-1}[V] = \Spec B$
			y la $A$-álgebra $B$ es un $A$-módulo finitamente generado proyectivo.
	\end{enumerate}
	En cuyo caso, decimos que $f$ es un \strong{morfismo finito localmente libre}\index{morfismo!finito!localmente libre}.
\end{prop}
Si $Y$ es un esquema localmente noetheriano, entonces por el inciso 2, un morfismo finito es localmente libre syss es plano.
Nótese que como todo morfismo finito localmente libre es plano y de presentación finita, entonces es un morfismo universalmente abierto;
y, por ser un morfismo finito, también es un morfismo universalmente cerrado.

Así pues, si $Y$ es conexo y $X \ne \emptyset$, entonces es sobreyectivo.

\begin{lem}
	Sea $S$ un esquema conexo y sea $\pi \colon X \to S$ un morfismo finito localmete libre (e.g., morfismo finito étale).
	Entonces $X$ es la unión disjunta finita de subesquemas conexos abiertos y cerrados.
\end{lem}
\begin{proof}
	Basta probarlo por inducción sobre $d := \deg\pi$.
	Para $d = 0$ se tiene que $X = \emptyset$, y para $d = 1$ tenemos que $X \cong S$, así que están listos.
	Si $X$ no es conexo, entonces $X = X_1 \amalg X_2$ y $\deg\pi = \deg(\pi|_{X_1}) + \deg(\pi|_{X_2})$.
\end{proof}
\begin{mydef}
	Sea $S$ un esquema y $\overline{s} \colon \Spec\Omega \to S$ un punto geométrico.
	Dado un $S$-esquema finito étale $X$, podemos definir la \strong{fibra}\index{fibra!(punto geométrico)} $X_{\overline{s}} := X \times_S \Spec\Omega$
	el cual es una unión disjunta de copias de $\Spec\Omega$, de modo que viene completamente determinado por su conjunto $\Hom_S(\Spec\Omega, X)$.
	El \strong{funtor de fibras}\index{funtor!de fibras} es
	$$ \mathcal{F}_{\overline{s}} \colon \mathsf{F\acute Et}/S \longrightarrow \mathsf{Set}, \qquad
	X \mapsto \Hom_S(\Spec\Omega, X) = X_{\overline{s}}. $$
	Si $S$ es conexo, la subcategoría plena de $\mathsf{Fun}(\mathsf{F\acute Et}/S, \mathsf{Set})$ cuyos objetos son isomorfos a
	funtores de fibras $\mathcal{F}_{\overline{s}}$, donde $\overline{s} \colon \Spec\Omega \to S$ recorre los puntos geométricos de $S$,
	se denomina el \strong{grupoide fundamental}\index{grupoide fundamental} de $S$ y se denota $\Pi_{\rm alg}(S)$.
\end{mydef}

% \begin{lem}
% 	Sean $f \colon Y \to X$ y $g \colon X \to S$ un par de morfismos.
% 	Si $g$ es un morfismo separado y $f\circ g$ es un morfismo (étale) finito, entonces $f$ también es (étale) finito.
% \end{lem}
% \begin{proof}
% 	La diagonal 
% 	\begin{tikzcd}[cramped, sep=small]
% 		\Delta_{X/S} \colon X \rar[closed] & X\times_S X
% 	\end{tikzcd}
% 	es un encaje cerrado por definición y, en particular, es un morfismo finito.
% 	...
% \end{proof}
Recuérdese que los encajes cerrados son morfismos finitos, que estos son estables salvo composición y cambio de base, de modo que admiten cancelación izquierda
por morfismos separados.
Los morfismos étale también poseen cancelación por morfismos separados, aunque las razones son distintas.
\begin{prop}
	Sea $f \colon X \to S$ un cubrimiento étale finito, y sea $s \colon S \to X$ una sección suya (i.e., $s\circ f = \Id_S$).
	Entonces $s$ induce un isomorfismo de $S$ con un subesquema abierto y cerrado de $X$.
\end{prop}
\begin{proof}
	Como $s\circ f = \Id_S$ es étale finito y $f$ es separado, entonces $s$ también es étale finito,
	luego es un morfismo universalmente abierto y cerrado por las observaciones anteriores.
\end{proof}
\begin{cor}
	Sea $Z$ un $S$-esquema conexo, $X$ un $S$-esquema étale finito y sean $f, g \colon Z \to X$ un par de $S$-morfismos.
	Si existe un punto geométrico $\overline{z} \colon \Spec k \to Z$ tal que $\overline{z}\circ f = \overline{z}\circ g$, entonces $f = g$.
\end{cor}
\begin{proof}
	Haciendo cambio de base $X_Z \to Z$ es también un morfismo étale finito, por lo que podemos suponer que $Z = S$.
	Así, $S$ es conexo y $X$ posee dos $S$-secciones, por lo que inducen un isomorfismo $f \colon S \cong X_1$ y $g \colon S \cong X_2$,
	donde $X_1$ y $X_2$ son componentes conexas de $X$.
	Como $f(\overline{z}) = g(\overline{z})$, viendo la imagen de $\overline{z}$, comprobamos que $z \in X_1 \cap X_2$, por lo que $X_1 = X_2$
	y $f = g$ ya que $S$ posee un único $S$-endomorfismo.
\end{proof}

\begin{cor}
	Sea $X$ un $S$-esquema étale finito y conexo.
	Entonces, fijando un punto geométrico $\overline{s} \in S(\Omega)$,
	el grupo $\Aut(X/S)$ actúa fielmente sobre la fibra geométrica de $X_{\overline{s}}$ y, en particular, es finito.
\end{cor}
\begin{proof}
	Basta aplicar el corolario anterior con $Z = X$.
\end{proof}

% Es claro que los morfismos finitos localmente libres se preservan salvo composición y cambio de base.

% Sea 
% \begin{lem}
% 	Sea $\pi \colon X \to S$ un morfismo finito étale.
% \end{lem}

% \section{Cambio de base propio}
% ...

\section*{Notas históricas}
La teoría del descenso plano, que es indudablemente una de las herramientas centrales de la geometría algebraica contemporánea,
fue concebida en el seminario de Geometría Algebraica Fundamental y varias de las exposiciones han rápidamente adquirido el estatus
de lecturas obligatorias en el tópico.
El descenso fpqc es quizá una de las técnicas más famosas de Grothendieck y, para fortuna del lector, están bastante bien expuestas en varias fuentes.

La definición de <<topología de Grothendieck>> (y, por consiguiente, la de <<sitio>>) es original de M.~Artin.
El grupo fundamental étale fue estudiado en detalle en el primer
Seminario de Geometría Algebraica (SGA) del Bosque Marie dirigido por Grothendieck \cite{sga1};
aquí mismo fueron definidos por vez primera la noción de morfismos étale y no ramificados,
y el formalismo de las categorías de Galois (vid.\ la exposición~V.4-V.6).
