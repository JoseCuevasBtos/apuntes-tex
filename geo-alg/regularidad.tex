\chapter{Propiedades locales}

\section{Dimensión}
Recuérdese que la \strong{dimensión (combinatorial)} de un espacio topológico $X$ es el $n$ supremo tal que existe una cadena $F_0 \subset F_1 \subset \cdots
\subset F_n$ de cerrados irreducibles de $X$.
Se sigue de la definición que $\dim\emptyset = -\infty$ (pues es el supremo de un conjunto vacío).
Dado un cerrado irreducible $Y \subseteq X$ habíamos definido la \strong{codimensión (combinatorial)}, denotada $\codim(Y, X) = n$, como el supremo
tal que existe una cadena de cerrados irreducibles que comienza en $Y = F_0 \subset F_1 \subset \cdots \subset F_n$.
Claramente, se sigue que la dimensión de $X$ es el supremo de las codimensiones.

\begin{prop}
	Sea $X$ un espacio topológico. Entonces:
	\begin{enumerate}
		\item Para todo $Y \subseteq X$ se cumple que $\dim Y \le \dim X$.
		\item Si $X = \bigcup_{i=1}^n X_i$, donde $X_i$ son cerrados en $X$, entonces $\dim X = \max_i \{ \dim X_i \}$.
	\end{enumerate}
\end{prop}

\begin{mydef}
	Sea $X$ un espacio topológico y sea $x \in X$ un punto.
	Entonces se define la \strong{dimensión local} en $x$ como
	$$ \dim_x X := \inf\{ \dim U : x \in U \text{ abierto} \}. $$
\end{mydef}

\begin{prop}
	Sea $X$ un espacio topológico.
	\begin{enumerate}
		\item Dado un punto $x \in X$, un entorno $U$ de $x$ y una familia $Y_i \subseteq X$ finita de cerrados tales que
			$x \in Y_i$ para todo $i$ y $U \subseteq \bigcup_{i=1}^n Y_i$, entonces $\dim_x X = \max_i\{ \dim_x Y_i \}$.
		\item Se cumple que $\dim X = \sup_{x\in X} \dim_x X$.
		\item Si $\{ X_i \}_{i\in I}$ es un cubrimiento abierto o un cubrimiento cerrado localmente finito de $X$,
			entonces \smash{$\displaystyle \dim X = \sup_{i\in I} \dim X_i$}.
		\item Sea $X$ un espacio $T_0$ y noetheriano, y sea $F$ el conjunto de puntos cerrados de $X$,
			entonces \smash{$\displaystyle \dim X = \sup_{x\in F} \dim_x X$}.
		\item Para todo cerrado irreducible $Y \subseteq X$ se cumple que
			$$ \dim Y + \codim(Y, X) \le \dim X. $$
		\item Para todo par de cerrados irreducibles $Y \subseteq Z \subseteq X$ se cumple que
			$$ \codim(Y, Z) + \codim(Z, X) \le \codim(Y, X). $$
		\item Un cerrado irreducible $Y$ tiene codimensión 0 syss es una componente irreducible.
	\end{enumerate}
\end{prop}
\begin{proof}
	\begin{enumerate}
		\item De la proposición anterior, para todo $V \subseteq U$ entorno de $x$ se tiene que
			$$ \dim_x X = \inf_V \max_i \{ \dim(Y_i \cap V) \}, $$
			y para cada $Y_i$ vemos que $\dim_x Y_i = \inf_V \dim(Y_i \cap V)$.
			Si para algún $i$ tenemos que $\dim_x Y_i = \infty$ es evidente que $\dim_x X = \infty$,
			así que, si no, tenemos que $\dim_x Y_i$ alcanza un valor finito (mínimo) para cada $i$, así que existe un entorno
			$V_0 \subseteq U$ de $x$ tal que $\dim_x Y_i = \dim(Y_i \cap V_0)$ y así para todo $V \subseteq V_0$ tenemos que
			el valor es el mismo, de lo que se sigue el enunciado.
		\item Claramente $\dim_x X \le \dim X$ para todo punto $x \in X$.
			Por otro lado, sea $\emptyset \ne F_0 \subset \cdots \subset F_n$ una cadena de irreducibles en $X$ y sea $x \in F_0$;
			para todo entorno $U$ de $x$, vemos que $U \cap F_i$ es cerrado en $U$ y $\overline{U \cap F_i} = F_i$ (¿por qué?),
			así que los $U \cap F_i$'s son cerrados irreducibles en $U$ y tenemos que $\dim U \ge n$.
		\item Si el cubrimiento es abierto y $x \in X_i$, entonces $\dim_x X \ge \dim_x X_i$.
			Si el cubrimiento es cerrado, $x \in X$ y $U$ es un entorno de $x$ que corta a finitos cerrados, entonces
			$$ \dim_x X \le \dim U = \max_i \{ \dim(U \cap X_i) \} \le \sup_i \dim X_i. $$
		\item Basta notar que si un espacio es $T_0$, entonces todo cerrado irreducible minimal $F$ corresponde a un punto cerrado.
		\item[5-7.] Triviales.
			\qedhere
	\end{enumerate}
\end{proof}
Como ejericio, ¿dónde empleamos que $X$ es noetheriano en el inciso 4?

\begin{cor}\label{thm:isol_iff_loc_dim_zero}
	Sea $X$ un espacio topológico $T_0$ y noetheriano.
	\begin{enumerate}
		\item $X$ tiene $\dim X = 0$ syss es discreto, finito (y no vacío).
		\item Un punto $x \in X$ tiene $\dim_x X = 0$ syss está aislado.
	\end{enumerate}
\end{cor}
\begin{proof}
	\begin{enumerate}
		\item Todas las componentes irreducibles tienen dimensión 0, luego, como $X$ es $T_0$ es fácil ver que corresponden a puntos,
			y como es noetheriano, posee finitas componentes irreducibles, es decir, finitos puntos cerrados, luego abiertos.
		\item Basta notar que el punto posee un entorno $T_0$ y noetheriano de dimensión 0. \qedhere
	\end{enumerate}
\end{proof}

\begin{prop}
	Sea $A$ un anillo, y sea $X := \Spec A$.
	\begin{enumerate}
		\item $\dim X = \kdim A = \kdim(A/\nilrad) = \dim(X_{\rm red})$.
		\item Para todo primo $\mathfrak{p} \nsl A$, se cumple que:
			$$ \kdim(A_{\mathfrak{p}}) = \alt\mathfrak{p} = \codim(\overline{\{ x_{\mathfrak{p}} \}}, X), \qquad
			\kdim(A/\mathfrak{p}) = \dim\overline{\{ x_{\mathfrak{p}} \}}. $$
		\item $\dim X = \sup\{ \kdim(A_{\mathfrak{m}}) : \mathfrak{m} \in \mSpec A \}$.
		\item $\dim X = \sup\{ \kdim(A/\mathfrak{p}) : \mathfrak{p} \text{ minimal} \}$.
	\end{enumerate}
\end{prop}
Antes de seguir vamos a dar un ejemplo relevante de la clase de posibles desastres en la teoría de la dimensión:
\begin{ex}
	Sea $(A, \mathfrak{m})$ un dominio de valuación discreta y sea $X := \Spec A$.
	Luego $\dim X = \kdim A = 1$, y $X = \{ \xi, x \}$, donde $\xi = (0)$ y $x = \mathfrak{m}$.
	Como $x$ corresponde a un ideal maximal, entonces es un punto cerrado del espacio, luego $U := X \setminus \{ x \} = \{ \xi \}$ es abierto
	y, además, es denso.
	Pero $\dim U = 0$ pues es un punto y $\dim X = 1 \ne \dim U$.
\end{ex}
Así, por ejemplo, vemos que en general no podemos calcular dimensión en un abierto, realmente es necesario un cubrimiento.

El teorema del ascenso y del descenso de Cohen-Seidenberg tienen la siguiente consecuencia esquemática:
\begin{cor}
	Sea $f \colon X \to Y$ un morfismo entero (e.g.\ finito) de esquemas.
	Entonces $\dim X = \dim Y$.
\end{cor}

% \begin{mydef}
% 	Sea $k$ un cuerpo.
% 	Un $k$-esquema $X$ se dice una \strong{variedad afín}\index{variedad!afín} sobre $k$ si es un conjunto algebraico afín íntegro sobre $k$.
% \end{mydef}
Y una traducción directa del teorema~\ref{app:dimension_kAlg_ft}:
\begin{thm}\label{thm:integral_affine_are_biequidim}
	Sea $k$ un cuerpo y sea $X$ una variedad afín sobre $k$. Entonces:
	\begin{enumerate}
		\item $\dim X = \trdeg_k K(X) = \trdeg_k K(U) = \dim U$, para cualquier abierto no vacío $U \subseteq X$.
		\item Toda cadena maximal de cerrados irreducibles en $X$ tiene la misma longitud.%
			\footnote{Una cadena de cerrados irreducibles $F_0 \subset F_1 \subset \cdots$ se dice \textit{maximal} si no existe un cerrado irreducible
				$Y$ tal que $F_j \subset Y \subset F_{j+1}$. \citeauthor{ega-iv1}~\cite{ega-iv1} emplea el adjetivo \textit{saturado}.}
		\item Para todo punto $x \in X$ se tiene que
			$$ \dim\overline{\{ x \}} + \codim(\overline{\{ x \}}, X) = \dim X. $$
	\end{enumerate}
\end{thm}

\begin{ex}
	Sea $k$ un cuerpo. Entonces:
	\begin{enumerate}
		\item $\A^n_k$ es una variedad afín y $\dim(\A^n_k) = n$.
		\item $\dim(\PP^n_k) = n$, pues $\PP^n_k = \bigcup_{i=0}^{n} U_i$, donde cada
			\[
				U_i \cong \Spec(A[t_0/t_i, \dots, t_n/t_0]) \cong \A^n_k.
			\]
		\item Sea $X = \Spec( k[x, y]/(y^2 - x^3) )$, entonces $\dim X = 1$,
			pues $k[x, y](y^2 - x^3) = k[x, \sqrt{x^3}]$ es una extensión entera de $k[x]$.
	\end{enumerate}
\end{ex}

\begin{mydef}
	Se dice que un espacio topológico $X$ es \strong{equidimensional}\index{equidimensional (espacio topológico)} o que tiene
	\strong{dimensión pura}\index{dimensión!pura} $n$ si todas sus componentes irreducibles tienen dimensión $n$.
	% Se dice que $X$ es \strong{equicodimensional}\index{equicodimensional} si todo cerrado irreducible minimal tiene igual codimensión.
\end{mydef}
Así, sustituyendo ser un $k$-esquema \textit{íntegro} por \textit{de dimensión pura}, entonces los resultados del teorema anterior se preservan.

\begin{mydef}
	Se dice que un espacio topológico $X$ es \strong{catenario}\index{espacio topológico!catenario}\index{catenario (espacio topológico)}
	si para todo par de cerrados irreducibles $Y \subseteq Z$ se cumple que $\codim(Y, Z) < \infty$ y toda cadena maximal de cerrados irreducibles
	$$ Y = F_0 \subset F_1 \subset \cdots \subset F_n = Z $$
	tiene igual longitud $n = \codim(Y, Z)$.

	Se dice que un esquema $X$ es \strong{universalmente catenario}\index{esquema!universalmente catenario} si todo $X$-esquema de tipo finito es catenario.
\end{mydef}
\begin{prop}
	% Sea $X$ un espacio topológico tal que para todo par de cerrados irreducibles $Y \subseteq Z$ se cumpla que $\codim(Y, Z) < \infty$
	% (e.g., si $X$ es un espacio topológico noetheriano).
	Para un espacio topológico $X$, son equivalentes:
	\begin{enumerate}
		\item $X$ es catenario.
		\item $X$ admite un cubrimiento por abiertos catenarios.
		\item Para todo par de cerrados irreducibles $Y \subseteq Z$ se cumple que $\codim(Y, Z) < \infty$,
			y para todo trío de cerrados irreducibles $Y \subseteq Z \subseteq W$ se cumple que
			$$ \codim(Y, Z) + \codim(Z, W) = \codim(Y, W). $$
	\end{enumerate}
	% Si además $X$ es $T_0$, noetheriano y de dimensión pura finita, las anteriores equivalen a:
	% \begin{enumerate}
	% 	\item 
	% \end{enumerate}
\end{prop}
\begin{proof}
	$1 \implies 2$ y $1 \implies 3$ son triviales.

	$2 \implies 1$. Basta notar que la función $F \mapsto \overline{F}$ determina una biyección entre los cerrados de un subespacio abierto $U \subseteq X$
	y los cerrados de $X$ que cortan a $U$.

	$3 \implies 1$. Fijemos $Y \subseteq Z$ dos cerrados irreducibles en $X$ como extremos.
	Basta probar que toda cadena con extremos $Y, Z$ tiene igual longitud por inducción sobre la longitud, notando que si hay una cadena maximal de longitud 1,
	entonces es trivial, y si hay una cadena más larga, la descomponemos en dos para emplear la hipótesis inductiva.
\end{proof}
\begin{cor}
	Los espectros de cuerpos son universalmente catenarios y, en consecuencia, todo esquema algebraico sobre un cuerpo $k$ es universalmente catenario.
\end{cor}
\begin{proof}
	Por el teorema~\ref{thm:integral_affine_are_biequidim} se sigue que todo esquema algebraico afín irreducible sobre $k$ es catenario, y claramente
	todo esquema algebraico admite un cubrimiento por afines irreducibles, luego también es catenario.
\end{proof}
\warn
Ojo que ser catenario no implica ser equidimensional.
Es claro que un esquema sobre un cuerpo puede constar de varias componentes irreducibles de distinta dimensión.
Nagata construyó el primer ejemplo de un anillo noetheriano que no es catenario; en el apéndice hay un ejemplo de un esquema afín que no es catenario.

\begin{prop}\label{thm:dimension_on_alg_base_change}
	Sea $k$ un cuerpo y sea $X$ un esquema algebraico sobre $k$ de dimensión pura $n$.
	Para toda extensión algebraica $L/k$ se cumple que $X_L$ tiene dimensión pura $n$.
\end{prop}

\begin{mydef}
	Sea $X$ un espacio topológico.
	Una función $f \colon X \to \R$ se dice \strong{semicontinua superior}\index{semicontinua!superior} si todo punto $x \in X$
	posee un entorno abierto $U$ tal que para todo $z \in U$ se tiene que $f(x) \ge f(z)$.
	Una función $g \colon X \to \R$ se dice \strong{semicontinua inferior}\index{semicontinua!inferior} si $-g$ es semicontinua superior.
\end{mydef}
\begin{prop}
	Sea $X$ un espacio topológico.
	Si $f$ es semicontinua superior, entonces para todo $x \speto y$ se tiene que $f(x) \le f(y)$.
\end{prop}

La proposición anterior da una idea de cómo pensar la semicontinuidad. Por ejemplo:
\begin{prop}
	Sea $X$ un espacio topológico $T_0$.
	La función $x \mapsto \dim_x X$ es semicontinua superior.
\end{prop}
\begin{proof}
	Si $x$ es tal que $\dim_x X = \infty$ entonces es claro.
	Si $\dim_x X = d < \infty$, entonces, por definición de dimensión local, existe un entorno $x \in U_0$
	tal que $\dim U = d$ para todo $U$ abierto que satisfaga $x \in U \subseteq U_0$.
	Así pues, para todo $y \in U_0$, vemos que todo entorno $V \subseteq U_0$ satisface $\dim V \le \dim U_0$, por lo que
	$\dim_y X \le d$.
\end{proof}
Nótese que $x \mapsto \codim( \overline{\{ x \}}, X )$ no es ni semicontinua superior, ni semicontinua inferior.
Si $A$ es un anillo de valuación y $X := \Spec A = \{ \eta, s \}$, donde $\eta$ es genérico, entonces el único entorno de $s$ es $X$ mismo,
y $\codim( \overline{\{ \eta \}}, X ) = 0 < 1 = \codim(\overline{\{ s \}}, X)$.
No obstante, si $Y := \Spec\Z$ y $\xi \in Y$ es el punto genérico de $Y$, vemos que $\codim(\overline{\{ \xi \}}, Y) = 0$,
pero todo entorno de $\xi$ contiene puntos de codimensión 1.

\section{Esquemas normales}
Aquí haremos fuerte uso de definiciones y propiedades de dependencia
íntegra, un resumen de los resultados se encuentra en el apéndice \S\ref{sec:integral_dep}.

\begin{mydef}
	Un esquema $X$ se dice \strong{normal} si para cada punto $x \in X$
	se cumple que el anillo local $\mathscr{O}_{X, x}$ es normal.
	Un esquema $X$ normal de $\dim X \le 1$
	se dice un esquema de Dedekind.
\end{mydef}
\begin{cor}
	Un anillo $A$ es normal syss el esquema $\Spec A$ es normal.
\end{cor}
\begin{prop}
	Todo esquema compacto posee puntos cerrados.
\end{prop}
\begin{prop}
	Sea $X$ un esquema irreducible. Son equivalentes:
	\begin{enumerate}
		\item $X$ es normal.
		\item Para todo abierto $U \subseteq X$, el anillo $\mathscr{O}_X(U)$ es normal.
	\end{enumerate}
	Si además $X$ es compacto, entonces ambas son equivalente a:
	\begin{enumerate}[resume]
		\item Las fibras $\mathscr{O}_{X,x}$ son normales en todos los puntos cerrados.
	\end{enumerate}
\end{prop}
\begin{proof}
	La equivalencia $1 \iff 2$ sale de la definición de anillo
	normal y es trivial que $1 \implies 3$.

	$3 \implies 1$. Sea $x$ un punto arbitrario del esquema. Entonces existe un
	punto cerrado $y \in \overline{\{ x \}}$, y así $\mathscr{O}_{X, y}$ es normal, y como $\mathscr{O}_{X, x}$ es una localización
	de $\mathscr{O}_{X, y}$, entonces $\mathscr{O}_{X, x}$ también es normal.
\end{proof}
\begin{ex}
	Sea $A$ un anillo normal (e.g., $A = k$ un cuerpo).
	Como $A[x_1, \dots, x_n]$ es normal (proposición A.36), entonces $\A^n_A$ y $\PP^n_A$ son esquemas normales.
\end{ex}
\begin{cor}
	Sea $X$ un esquema íntegro noetheriano. Entonces $X$ es
	de Dedekind syss cada $\Gamma(U, \mathscr{O}_X)$ es de Dedekind.
\end{cor}
\begin{mydef}
	Un esquema $X$ se dice \strong{factorial} si para cada punto
	$x \in X$, se cumple que la fibra $\mathscr{O}_{X, x}$ es un DFU.
\end{mydef}

\begin{prop}
	Se cumple:
	\begin{enumerate}
		\item Todo esquema factorial es normal.
		\item Si $X$ es un esquema de Dedekind, entonces todos los anillos locales $\mathscr{O}_{X, x}$ son DIPs.
		\item Sea $X$ un esquema irreducible de $\dim X = 1$.
			Entonces $X$ es de Dedekind syss $X$ es normal.
	\end{enumerate}
\end{prop}

\begin{thm}
	Sea $X$ un esquema normal localmente noetheriano. Para
	todo $F \subseteq X$ cerrado de $\codim(F, X) \ge 2$, la restricción
	$$ \Gamma(X, \mathscr{O}_X ) \longrightarrow \Gamma(X \setminus F, \mathscr{O}_X ) $$
	es un isomorfismo. Es decir, toda función regular en $X \setminus F$ se extiende a $X$.
\end{thm}
\begin{proof}
	Sea $X = \Spec A$ afín. Para todo $\mathfrak{p} \in \Spec A$ de $\alt\mathfrak{p} = 1$
	se cumple que $\mathfrak{p} \in X \setminus F$, luego concluimos el teorema pues
	\begin{equation}
		A = \bigcap_{\substack{\mathfrak{p} \in \Spec A \\ \alt\mathfrak{p} = 1}} A_{\mathfrak{p}}.
		\tqedhere
	\end{equation}
\end{proof}

\begin{prop}\label{thm:rational_map_extension}
	Sea $S$ un esquema localmente noetheriano, $X$ un $S$-esquema íntegro normal de tipo finito e $Y$ un $S$-esquema propio.
	Sea $f \colon U \to Y$ un morfismo definido sobre un abierto $\emptyset ̸= U \subseteq X$.
	Entonces $f$ se extiende de manera única a un morfismo $\overline{f}\colon V \to Y$, donde $V$ es un abierto de $X$ que
	contiene a todos los puntos de codimensión 1.
	Más aún, si $X$ es de Dedekind, podemos exigir $V = X$.
\end{prop}
\begin{proof}
	La unicidad sale del hecho de que el morfismo estructural $Y \to S$ es separado, y del teorema~\ref{thm:separatedness_ext}.
	Sea $\xi \in X$ el punto genérico,
	entonces $\xi \in U$ e induce un morfismo $f_\xi \colon \Spec k(X) \to Y$.
	Sea $x \in X$ de codimensión 1, entonces $\mathscr{O}_{X,x}$ es un dominio de valuación discreta con
	$\Frac(\mathscr{O}_{X,x}) = k(X)$, luego se extiende a $f_x \colon \Spec \mathscr{O}_{X,x} \to Y$.
	Como $Y$ es de tipo finito sobre $S$, entonces $f_x$ se extiende a $g\colon U_x \to Y$ para un entorno $U_x$
	de $x$.
	Sea $W \subseteq Y$ un entorno afín de $g(x)$ y considere las restricciones de $f, g$ a $U^\prime := f^{-1} [W ] \cap g^{-1} [W ]$.
	Así, $U^\prime$ es un abierto no vacío pues contiene a $\xi$.
	Los homomorfismos de anillos $\mathscr{O}_Y (W ) \to \mathscr{O}_X (U^\prime )$ inducidos por $f, g$ son
	idénticos pues coinciden en $k(X) \supseteq U^\prime$, así que $f|_{U^\prime} = g|_{U^\prime}$ (pues el codominio
	es afín, teorema 3.75), así que coinciden en $U \cap U_x$ (nuevamente, por el teorema~\ref{thm:separatedness_ext}).
	Así iteramos el proceso para cada punto de codimensión 1 y pegamos en $V$.
\end{proof}

\begin{lem}
	Sea $A$ un dominio de valuación discreta con $K := \Frac A$,
	y sea $k$ el cuerpo de restos de $A$. Sea $X$ un $A$-esquema con $\Gamma(U, \mathscr{O}_X)$ una
	$A$-álgebra plana para todo abierto afín $U \subseteq X$. Si $X_K$ es normal y $X_k$ es
	reducido, entonces $X$ es normal.
\end{lem}
\begin{proof}
	Podemos reducirnos al caso en que $X = \Spec B$ es afín.
	Por hipótesis, el homomorfismo canónico $B = B \otimes_A A \to B \otimes_A K$ es inyectivo.
	Sea $t$ un uniformizador de $A$. Sea $\beta \in \Frac B$ entero sobre $B$; como $B \otimes_A K$
	es normal, existe $b \in B$ y $r \in \Z$ tal que $\beta = bt^{-r}$.
	Sea $b \notin tB$, probaremos que $r \le 0$. Como $\beta$ es entero, sea
	$$ \beta^n + c_{n-1} \beta^{n-1} + \cdots + c_1 \beta + c_0 = 0, \qquad c_i \in B. $$
	Si $r > 0$, multiplicando por $t^{rn}$, vemos que $b$ es nilpotente en $B/tB$, así que
	$b \in tB$ lo que es absurdo. Así que $r \le 0$ y $\beta \in B$.
\end{proof}

\begin{mydef}
	Sea $X$ un esquema íntegro. Un morfismo $\pi \colon X^\prime \to X$
	se dice una \strong{normalización}\index{normalización} si $X^\prime$ es normal y todo morfismo $f \colon Y \to X$
	dominante con $Y$ normal se factoriza:
	\begin{center}
		\begin{tikzcd}
			{} & Y \dar["f"] \dlar["\exists!"'] \\
			X' \rar["\pi"'] & X
		\end{tikzcd}
	\end{center}
\end{mydef}

\begin{lem}
	Sea $A$ un dominio íntegro con $K := \Frac A$. Sea $A' := \mathcal{O}_K/A$
	la clausura íntegra de $A$ en $K$, entonces $\pi := (\iota^a) \colon \Spec(A') \to \Spec A$ es la
	normalización.
\end{lem}
\begin{proof}
	Sea $f \colon Y \to X$ un morfismo dominante con $Y$ normal,
	entonces viene de un homomorfismo de anillos $\varphi \colon A \to \mathscr{O}_Y (Y)$, el cual es
	inyectivo pues $f$ es dominante. Así que se factoriza como
	\begin{tikzcd}[cramped, sep=small]
		A \rar & A' \rar["g"] & \mathscr{O}_Y(Y)
	\end{tikzcd}
	lo que induce el morfismo de esquemas $g^a \colon Y \to \Spec(A')$.
\end{proof}

Notamos que la normalización descrita por el lema anterior es una aplicación birracional por el teorema 4.89.

\begin{prop}
	Sea $X$ un esquema íntegro.
	Entonces existe una única normalización $\pi \colon X^\prime \to X$ salvo isomorfismo (de $\mathsf{Sch}/X$).
	Más aún, $f \colon Y \to X$ es una normalización syss $Y$ es normal y $f$ es un morfismo entero y birracional.
\end{prop}
\begin{hint}
	Basta ir pegando abiertos afínes y empleando la unicidad de la normalización.
\end{hint}

\begin{mydef}
	Sea $X$ un esquema íntegro y sea $L \supseteq k(X)$ una extensión algebraica de cuerpos.
	Se le llama una \strong{normalización} de $X$ en $L$ a un
	morfismo $\pi \colon X^\prime \to X$ entero con $X^\prime$ normal y $k(X^\prime) = L$ que extiende al morfismo canónico $\Spec L \to X$.
\end{mydef}

Del mismo modo que se probaba la proposición anterior se prueba:
\begin{prop}
	Sea $X$ un esquema íntegro y sea $L/k(X)$ una extensión algebraica.
	Entonces existe una única normalización $\pi \colon X^\prime \to X$ de $X$
	en $L$ salvo isomorfismo (de $\mathsf{Sch}/X$).
\end{prop}

\begin{prop}
	Sea $X$ un esquema noetheriano normal y sea $L/k(X)$ una extensión algebraica separable.
	La normalización $X^\prime \to X$ de $X$ en $L$ es un morfismo finito y, en consecuencia, $\dim(X') = \dim X$.
\end{prop}
\begin{proof}
	Ésta es una traducción de la proposición A.42.
\end{proof}

\begin{prop}
	Sea $k$ un cuerpo, sea $A := k[t_1, \dots, t_n]$ el álgebra
	polinomial, sea $K := \Frac A$ y sea $L/K$ una extensión finita (posiblemente inseparable).
	Entonces la clausura íntegra de $A$ en $L$ es un $A$-módulo finitamente generado.
\end{prop}
\begin{proof}
	Sabemos que $L$ puede verse como $L/L_i/K$, donde $L_i/K$
	es una extensión puramente inseparable y $L/L_i$ es separable. Así, por la
	proposición anterior, podemos reducirnos al caso en que $L/K$ sea puramente
	inseparable. Supongamos que $\car K =: p > 0$ y sea $\alpha_1, \dots, \alpha_n$ una $K$-base
	de $L$. Sabemos que existe $q := p^r$ para algún $r > 0$ tal que cada $\alpha_i^q \in
	K$ (teorema A.15). Fijemos una clausura algebraica%
	\footnote{Tecnicamente aquí fijamos dos cosas: una clausura algebraica de $K$ y un monomorfismo $K \hookto \algcl K$,
		y también un $K$-monomorfismo $L \hookto \algcl K$.}
	$\algcl K \supseteq L$, de modo que $L \subseteq k^\prime[\beta_1, \dots, \beta_n]$ donde cada $\beta_i = t_i^{1/q} \in \algcl K$,
	y donde $k^\prime/k$ es la extensión generada por añadir las raíces $q$-ésimas de los coeficientes de los
	$\alpha_i$'s. Así $k^\prime[\beta_1, \dots, \beta_n]/A$ es una extensión de anillos entera y finita, luego
	$B \subseteq k^\prime[\beta_1, \dots, \beta_n]$ también es un $A$-módulo finitamente generado.
\end{proof}

\begin{prop}
	Sea $X$ un esquema algebraico íntegro sobre un cuerpo $k$, y sea $L/K(X)$ una extensión finita de cuerpos.
	Entonces la normalización $X^\prime \to X$ de $X$ en $L$ es un morfismo finito, de modo que $X^\prime$ es un esquema algebraico íntegro sobre $L$.
\end{prop}
\begin{proof}
	Podemos reducirnos al caso de $X = \Spec A$ afín. Sea
	$k[t_1, \dots, t_n] \hookto A$ un monomorfismo finito (dado por el teorema de normalización de Noether),
	luego la clausura íntegra $B$ de $A$ en $L$ es la misma que la de $k[\vec t]$ en $L$, la cual es finitamente generada como $k[\vec t]$-módulo, por tanto
	también como $A$-módulo; así el morfismo $X^\prime \to X$ es finito.
\end{proof}

\begin{prop}
	Sea $X$ un esquema íntegro cuya normalización sea finita.
	Entonces el conjunto de puntos normales de $X$ es abierto.
\end{prop}
\begin{proof}
	Podemos reducirnos al caso en que $X = \Spec A$ es afín.
	Ahora la normalización de $X$ es $\Spec B$ con $B := \mathcal{O}_{\Frac(A)/A}$.
	Sea $U$ el conjunto de puntos normales de $X$, de modo que $x_{\mathfrak{p}} \in U$ syss $(B/A) \otimes_A A_{\mathfrak{p}} = 0$,
	luego sea $\mathfrak{a} := \Ann_A(B/A)$. Claramente $\DD(\mathfrak{a}) = \Spec A \setminus \VV(\mathfrak{a}) \subseteq U$.
	Recíprocamente, sea $\mathfrak{p} \in U$, entonces como $B/A$-módulo finitamente generado, existe
	$a \in A \setminus \mathfrak{p}$ tal que $a(B/A) = 0$ (¿por qué?), luego $\mathfrak{p} \notin \DD(a)$ y así se tiene
	igualdad, donde $\DD(a)$ es claramente abierto.
\end{proof}
\begin{cor}
	Sea $X$ un esquema algebraico íntegro sobre un cuerpo $k$. Entonces la normalización sí es finita y el conjunto de puntos normales de $X$ es abierto.
\end{cor}
\begin{proof}
	Resulta de aplicar los dos últimos resultados.
\end{proof}

\begin{thm}[Krull-Akizuki]
	Sea $X$ un esquema íntegro noetheriano de $\dim X = 1$, sea $L/K(X)$ una extensión finita de cuerpos y sea $\pi\colon X' \to X$ su normalización en $L$.
	Entonces $X'$ es de Dedekind y para todo subesquema cerrado propio $Z \subset X$, la restricción $\pi^{-1}[Z] \to Z$ es un morfismo finito.
\end{thm}
\begin{proof}
	Claramente $\dim(X') = 1$ y $X'$ es normal, luego basta probar que es noetheriano y que la restricción $\pi^{-1}[Z] \to Z$.
	Ambas propiedades son locales y podemos suponer que $X$ es afín, de lo que se deduce del teorema algebraico de Krull-Akizuki.
\end{proof}

\section{Esquemas regulares}
\begin{mydef}
	Sea $X$ un esquema y $x \in X$ un punto. Sea $(\mathscr{O}_{X,x}, \mathfrak{m}_x , \kk(x))$
	el anillo local en $x$, entonces el \strong{espacio tangente de Zariski}\index{espacio!tangente de Zariski} en $x$ es:
	\begin{equation*}
		T_{X,x} := (\mathfrak{m}_x /\mathfrak{m}^2_x )^\vee
		= \Hom_{\kk(x)}(\mathfrak{m}_x /\mathfrak{m}^2_x , \kk(x)) = \mathfrak{m}_x  \otimes_{\mathscr{O}_{X,x}} \kk(x).
	\end{equation*}
\end{mydef}
Sea $f \colon X \to Y$ un morfismo entre esquemas. Para todo $x \in X$, el morfismo
$f$ induce un homomorfismo de anillos:
\[
	T_{f,x} \colon T_{X,x} \to T_{Y,y} \otimes_{\kk(y)} \kk(x).
\]
\begin{prop}
	Si $X$ es un esquema localmente noetheriano, entonces
	para todo punto $x \in X$ se tiene que $\dim_{\kk(x)} T_{X,x} \ge \kdim(\mathscr{O}_{X,x} )$.
\end{prop}
\begin{proof}
	Es una traducción de la proposición A.52.
\end{proof}

\begin{prop}
	Sean $f \colon X \to Y, g \colon Y \to Z$ un par de morfismos
	entre esquemas y sea $x \in X$. Entonces $T_{f \circ g,x} = T_{f,x} \circ  (T_{g,f(x)} \otimes_{\kk(y)} \Id_{\kk(x)})$.
\end{prop}

\begin{mydef}
	Sea $X$ un esquema.
	Se dice que un punto $x\in X$ es \strong{regular}\index{punto!regular} en $X$ si su anillo local $\mathscr{O}_{X, x}$ es regular,
	vale decir, si $\kdim(\mathscr{O}_{X, x}) = \kdim( T_{X, x} ) = \dim_{\kk(x)}( \mathfrak{m}_x / \mathfrak{m}_x^2 )$;
	de lo contrario se dice que $x$ es \strong{singular}\index{punto!singular}.
	Se dice que $X$ es un esquema \strong{regular}\index{esquema!regular} si todos sus puntos son regulares.
\end{mydef}
Geométricamente uno piensa que <<singular>> se traduce o en una variedad con <<puntas>> o <<con cruces>>;
el segundo tipo es más fácil de ilustrar:
\begin{ex}
	Sea $k$ un cuerpo y tómese $X = \VV(x) \cup \VV(y) \subseteq \A^2_k$ la unión de los dos ejes.
	Este subesquema es afín y $A := \Gamma(X, \mathscr{O}_X) = k[x, y]/(xy)$; aquí denotaremos por $u$ la imagen de $x$, y por $v$ la imagen de $y$,
	los cuales satisfacen $uv = 0 \in A$.

	Un punto $P := (a, b) \in k^2$ lo asociamos al primo $\mathfrak{m}_P = (x - a, y - b) \in \A^2_k$;
	de modo que podemos verificar que $X$ es regular en los puntos de la forma $(a, 0), (0, a)$ con $a \in k^\times$ y en los puntos genéricos.
	El razonamiento es que en la localización $B := A_{(u - a, v)}$, por definición, $u$ es invertible, de modo que $v = (1/u)uv = 0$.
	Así que $B \cong k[x]_{(x-a)}$ el cual es un dominio de valuación discreta (¿por qué?), luego es regular.%
	\footnote{Tecnicamente verificamos regularidad en los puntos $k$-racionales; el resto de puntos se trabaja análogamente.}
	Si localizamos $A_{(u)}$ (pues $uA$ es un primo minimal), entonces aquí $v$ es invertible, de modo que $u = uv(1/v) = 0$,
	por lo que $A_{(u)} = k(v)$, el cual es un cuerpo.

	Finalmente, el punto $(0, 0)$ correspondiente al ideal $(u, v)$ es singular en $X$ y la razón está en que la localización $B := k[u, v]_{(u, v)}$
	satisface que su maximal es $(u, v)B$, luego su espacio tangente de Zariski es
	$$ T_{X, (0,0)} = \frac{(u, v)B}{(u^2, v^2)B} \cong k\langle 1, \overline{u}, \overline{v} \rangle, $$
	donde el símbolo de la derecha significa <<el $k$-espacio vectorial con base $1, \overline{u}, \overline{v}$>>, donde $\overline{()}$ representa
	la imagen salvo el cociente.
	Este $k$-espacio vectorial tiene dimensión 2, mientras que $X$ tiene dimensión 1.
\end{ex}

% Ahora bien, por regularidad de Serre, ser regular es cerrado bajo generización, luego:
\begin{prop}\label{thm:regularity_under_gez}
	Sea $X$ un esquema noetheriano.
	\begin{enumerate}
		\item Si $X$ es regular en un punto $x$ y $x' \speto x$ es una generización,
			entonces $x'$ es regular.
		\item En consecuencia, $X$ es regular syss todos sus puntos cerrados son regulares.
		\item Si $X$ es regular, entonces todas sus componentes conexas son normales.
	\end{enumerate}
\end{prop}
\begin{proof}
	Los primeros dos incisos son una aplicación del teorema de regularidad de Serre,
	y el tercero es una aplicación del teorema de Auslander-Buchsbaum-Nagata.
\end{proof}

Ahora nos dedicaremos a probar el criterio del jacobiano para probar regularidad:
\begin{mydef}
	Sea $k$ un cuerpo, sea $Y := \A^n_k = \Spec(k[t_1, \dots, t_n])$ y sea $y \in Y(k)$ un punto racional.
	Definimos el homomorfismo $D_y\colon k[\vec t] \to \Hom_k(k^n, k) = (k^n)^\vee$ que para un polinomio $f(\vec t) \in k[\vec t]$ actúa como
	$$ D_y(f)(\vec u) := \sum_{i=1}^{n} \left.\frac{\partial f}{\partial t_i}\right|_{\vec t = y} u_i, $$
	a esta aplicación le llamamos el \strong{diferencial} de $f$ en $y$.
\end{mydef}

\begin{lem}
	Sea $k$ un cuerpo.
	Sea $Y := \A^n_k$, sea $y \in Y(k)$ un punto racional y $\mathfrak{m} := \mathfrak{p}_y$. Entonces:
	\begin{enumerate}
		\item $\mathfrak{m}^2 \subseteq \ker D_y$ de modo que $D_y\colon \mathfrak{m/m}^2 \to (k^n)^\vee$ es un isomorfismo.
		\item Existe un isomorfismo canónico $T_{Y, y} \cong k^n$.
	\end{enumerate}
\end{lem}

\begin{prop}
	Sea $k$ un cuerpo.
	Sea $X = \VV(\mathfrak{a})$ un subesquema cerrado de $Y := \A^n_k$, sea 
	\begin{tikzcd}[cramped, sep=small]
		f\colon X \rar[closed] & Y
	\end{tikzcd}
	el encaje cerrado canónico y sea $x \in X(k)$.
	La transformación $k$-lineal $T_{f, x}\colon T_{X, x} \to T_{Y, y}$ induce un isomorfismo de $T_{X, x}$ con $(D_x[\mathfrak{a}])^\perp \subseteq k^n$,
	así que
	$$ T_{X, x} \cong \left\{ \vec u \in k^n : \sum_{i=1}^{n} \left. \frac{\partial f}{\partial t_i} \right|_{\vec t = x} \cdot u_i = 0 \right\}. $$
\end{prop}

\begin{thmi}[Criterio del jacobiano]\index{criterio!del jacobiano}
	Sea $k$ un cuerpo.
	Sea $X = \VV(\mathfrak{a})$ un subesquema cerrado de $Y := \A^n_k$, sea 
	\begin{tikzcd}[cramped, sep=small]
		f\colon X \rar[closed] & Y
	\end{tikzcd}
	el encaje cerrado canónico, sea $x \in X(k)$ y sean $f_1, \dots, f_r$ tales que generan $\mathfrak{a}$.
	Considere la siguiente matriz, denominada el \strong{jacobiano}\index{jacobiano (matriz)} en $x$:
	$$ \mathcal{J}_x := \left[ \left. \frac{\partial f_i}{\partial t_j} \right|_{\vec t = x} \right] \in \Mat_{r\times n}(k), $$
	entonces el punto $x$ es regular en $X$ syss $\rang \mathcal{J}_x = n - \kdim(\mathscr{O}_{X, x})$.
\end{thmi}

\begin{thm}\label{thm:geo_irred_has_reg_clpt}
	Sea $k$ un cuerpo.
	Todo conjunto algebraico geométricamente irreducible sobre $k$ tiene un punto cerrado regular.
\end{thm}

\begin{mydef}
	Sea $X$ un esquema localmente noetheriano.
	Se denota por $\Sing X$ el conjunto de puntos singulares de $X$ y por $\Reg X$ el conjunto de puntos regulares.
\end{mydef}

\begin{lem}\label{lem:regular_pts_speto_closed_reg}
	Sea $K$ un cuerpo algebraicamente cerrado y sea $X$ un esquema algebraico sobre $K$.
	Entonces todo punto regular de $X$ se especializa en algún punto regular y cerrado.
	% Si $x \in X$ es un punto regular, entonces existe $x' \in X$ tal que $x \speto x'$ y es regular y cerrado.
\end{lem}
\begin{prop}
	Sea $K$ un cuerpo algebraicamente cerrado y sea $X$ un esquema algebraico sobre $K$.
	Entonces $\Reg X$ es un abierto en $X$.
	% Más aún, si $X$ es normal, entonces $\codim(\Sing X, X) \ge 2$.
\end{prop}

Los siguientes forman parte de un criterio fundamental de Serre en álgebra conmutativa.
\begin{mydef}
	Se dice que un esquema $X$ es \strong{regular en codimensión $r$}%
	\index{regular!en codimensión $r$}%
	\index{esquema!regular!en codimensión $r$}
	si para toda localización $\mathscr{O}_{X,x}$ de dimensión $\le r$ se cumple que $\mathscr{O}_{X, x}$ es regular.

	Se dice que un esquema $X$ satisface la propiedad $(S_r)$ si para todo punto $x \in X$ se satisface que
	$$ \prof(\mathscr{O}_{X, x}) \ge \min\{ \codim(\overline{\{ x \}}, X), r \}, $$
	donde <<$\prof$>> denota la profundidad de un anillo local.
\end{mydef}
La condición $S_r$ podemos leerla como que si $\prof(\mathscr{O}_{X, x}) < s$, entonces también $\codim( \overline{\{ x \}}, X ) < s$
para todo $s \le r$.
Ambas definiciones son generalizaciones naturales de sus análogos en álgebra conmutativa;
por definición un anillo es regular si es regular en codimensión $r$ para todo $r$, y un anillo es de Cohen-Macaulay si satisface $(S_r)$ para todo $r$.

Si $X$ es localmente noetheriano, entonces ser regular en codimensión 0 significa que para todo punto genérico $\xi \in X$
el anillo local $\mathscr{O}_{X, \xi}$, que es artiniano, sea regular.
Pero es sabido que un anillo local artiniano es regular syss es un cuerpo, por lo que $X$ debe ser reducido en $\xi$.

\begin{thm}[criterio de normalidad de Serre]
	Sea $X$ un esquema localmente noetheriano. Entonces:
	\begin{enumerate}
		\item $X$ es reducido syss es regular en codimensión 0 (o es genéricamente reducido) y satisface la propiedad $(S_1)$.
		\item $X$ es normal syss es regular en codimensión 1 y satisface la propiedad $(S_2)$.
	\end{enumerate}
	En consecuencia, si $X$ es normal, entonces $\codim(\Sing X, X) \ge 2$.
\end{thm}

\section{Morfismos}
\subsection{Morfismos planos}
\begin{mydef}
	Sea $f\colon X \to Y$ un morfismo de esquemas y sea $\mathscr{F}$ un $\mathscr{O}_X$-módulo.
	Se dice que $\mathscr{F}$ es \strong{$f$-plano sobre $Y$}\index{fplano@$f$-plano} en $x \in X$ si el homomorfismo de anillos
	$f_x^\sharp\colon \mathscr{O}_{Y, f(x)} \to \mathscr{O}_{X, x}$ induce que $\mathscr{F}_{x}$ sea un $\mathscr{O}_{Y, f(x)}$-módulo plano.
	Se dice que $\mathscr{F}$ es \strong{$f$-plano sobre $Y$} (a secas) si lo es en cada punto.

	Se dice que $\mathscr{F}$ es \strong{plano sobre $X$}\index{OXmodulo@$\mathscr{O}_X$-módulo!plano} en el punto $x$ si es $\Id_X$-plano en $x$,
	vale decir, si la fibra $\mathscr{F}_x$ es un $\mathscr{O}_{X, x}$-módulo plano.
	% Se dice que $\mathscr{F}$ es \strong{plano sobre $X$} (a secas) si es plano en cada punto.
	Se dice que el morfismo $f$ es \strong{plano}\index{morfismo!plano} (en $x \in X$) si $\mathscr{O}_X$ es $f$-plano sobre $Y$ (en $x$).
\end{mydef}

\begin{prop}
	Se cumplen:
	\begin{enumerate}
		\item Los encajes abiertos son planos.
		\item Los morfismos planos son estables salvo cambio de base.
		\item La composición de morfismos planos es plana.
		\item El producto fibrado de morfismos planos es plano.
		\item Sea $\varphi\colon A \to B$ un homomorfismo de anillos.
			Entonces $\varphi^a\colon \Spec B \to \Spec A$ es plano syss $\varphi$ es plano.
		\item Si $X$ es un esquema localmente noetheriano, entonces un haz $\mathscr{F}$ finitamente generado es plano syss es localmente libre.
	\end{enumerate}
\end{prop}

\begin{lem}\label{thm:flat_are_very_dominant}
	Sea $Y$ un esquema irreducible y sea $f\colon X \to Y$ un morfismo plano.
	Entonces todo abierto no vacío $U \subseteq X$ domina a $Y$ (i.e., $f[U]$ es denso).
	Más aún, si $X$ tiene finitas componentes irreducibles, entonces cada componente domina a $Y$.
\end{lem}

\begin{prop}
	Sea $Y$ un esquema con finitas componentes irreducibles y sea $f\colon X \to Y$ un morfismo plano.
	Si $Y$ es reducido (resp. irreducible, íntegro) y las fibras genéricas son reducidas (resp. irreducibles, íntegras),
	entonces $X$ es reducido (resp. irreducible, íntegro).
\end{prop}

\begin{prop}\label{thm:flat_morph_open}
	Sea $f\colon X \to Y$ un morfismo plano de tipo finito entre esquemas noetherianos, entonces es abierto.
\end{prop}
\begin{proof}
	Por el lema~\ref{thm:flat_are_very_dominant}, entonces $f$ es dominante y luego por el teorema~\ref{thm:dominant_image_noeth_sch} tiene que
	satisfacerse que $f[X]$ contiene un abierto denso $V$ en $Y$.
	Luego considere el cambio de base $f^{-1}[Y \setminus V] = X \times_Y (Y \setminus V) \to Y \setminus V$ el cual también es plano y de tipo finito
	entre esquemas noetherianos, por lo que la imagen contiene un abierto $U_1 \subseteq Y \setminus V$ el cual induce un abierto $V_1 \supseteq V$,
	y así sucesivamente construimos una sucesión de abiertos $V \subseteq V_1 \subseteq V_2 \subseteq \cdots$ (o por complementos, una sucesión descendente
	de cerrados) de modo que se estabiliza y prueba que $f[X]$ es abierto.

	Cambiando $X$ por un abierto $U$, es claro que $U$ sigue siendo noetheriano y que el morfismo sigue siendo plano de tipo finito, así que se ve que $f[U]$
	es abierto.
\end{proof}
\begin{ex}
	Es fácil comprobar que la proyección canónica $\A^{n+m}_k \to \A^n_k$ con $m \ge 0$ es un morfismo plano;
	por tanto la proyección es abierta (recuérdese que ya comprobamos que esta proyección no es cerrada en general).
	Esto coincide con la situación en topología general.
\end{ex}

\begin{prop}
	Sea $X$ un esquema reducido e $Y$ un esquema de Dedekind.
	Un morfismo $f\colon X \to Y$ es plano syss toda componentes irreducible domina a $Y$.
\end{prop}
\begin{cor}
	Sea $X$ un esquema íntegro e $Y$ un esquema de Dedekind.
	Todo morfismo no constante $X \to Y$ es plano.
\end{cor}

\begin{thm}\label{thm:flat_dim_formula}
	Sea $f\colon X \to Y$ un morfismo de esquemas localmente noetherianos.
	Sean $x \in X$ e $y := f(x)$, entonces
	$$ \kdim(\mathscr{O}_{X_y, x}) \ge \kdim( \mathscr{O}_{X, x} ) - \kdim(\mathscr{O}_{Y, y}). $$
	Más aún, si $f$ es plano, entonces se alcanza igualdad.
\end{thm}

\begin{cor}\label{thm:flat_fibers_dimension}
	Sea $k$ un cuerpo y sean $X, Y$ un par de conjuntos algebraicos sobre $k$ con $X$ equidimensional e $Y$ irreducible.
	Para todo morfismo $f\colon X \to Y$ plano y todo $y \in Y$ se comprueba que la fibra $X_y$ es equidimensional y
	$$ \dim X_y = \dim X - \dim Y. $$
\end{cor}

Si volvemos a tomar el ejemplo de la proyección canónica entre espacios afines podemos notar que trivialmente se cumple el teorema anterior,
puesto que las fibras son asímismas espacios afines (de dimensión constante).
% Un contraejemplo:
\begin{ex}
	Considere la proyección $\Spec(k[x, y, z]) \to \Spec(k[z]) = \A^1_k$ y precompongamos con el encaje cerrado por el subesquema cerrado
	$X := \VV(x^2 + y^2 - z^2)$.
	Denotando $f \colon X \to \A^1_k$ (ver fig.~\ref{fig:geo-alg/non_flat_morph}) notamos que las fibras $X_w$ con $w \ne 0 \in \A^1_k$ (un punto cerrado)
	son circunferencias y, en particular, tienen $\dim(X_w) = 1$ (formalice esto).
	Por el contrario, la fibra $X_0$ solo consta del punto $(0, 0, 0)$, de modo que tiene $\dim(X_0) = 0$, por lo que el morfismo $f$ no puede ser plano.
\end{ex}
\begin{figure}[!hbt]
	\centering
	\includegraphics[scale=1]{geo-alg/non_flat_morph.pdf}
	\caption{Un morfismo que no es plano.}%
	\label{fig:geo-alg/non_flat_morph}
\end{figure}

Finalmente, un último resultado que será útil:
\begin{thm}[planitud genérica]\index{teorema!de planitud genérica}
	Sea $f \colon X \to Y$ un morfismo dominante de tipo finito entre esquemas íntegros noetherianos.
	Entonces existe un abierto no vacío $U \subseteq Y$ tal que la restricción $f^{-1}[U] \to U$ es un morfismo fielmente plano.
\end{thm}
\begin{proof}
	En primer lugar, por el teorema~\ref{thm:dominant_image_noeth_sch} sabemos que $f[X]$ contiene a un abierto denso $V$,
	de modo que $f^{-1}[V] \to V$ es sobreyectivo.
	Restringiéndose a un abierto afín de $V$ y así mismo con $f^{-1}[V]$ podemos suponer que tenemos un morfismo dominante de tipo finito $\Spec B \to \Spec A$
	entre espectros de anillos noetherianos.
	Es decir, $A \subseteq B$ es una extensión de dominios íntegros noetherianos de tipo finito y podemos aplicar la planitud genérica de álgebra conmutativa.
\end{proof}
Como el teorema anterior engloba el caso afín, nos referiremos a éste como <<planitud genérica>>.
\begin{cor}\label{thm:dim_ineq_dominant_ft}
	Sea $f \colon X \to Y$ un morfismo dominante de tipo finito entre esquemas íntegros noetherianos, y sea $d$ la dimensión de la fibra genérica de $f$.
	Entonces:
	\begin{enumerate}
		\item Las componentes irreducibles de las fibras no vacías de $f$ tienen dimensión $\ge d$.
		\item Existe un abierto denso $U \subseteq Y$ tal que para todo $y \in U$ la fibra $X_y$ tiene dimensión pura $d$.
	\end{enumerate}
\end{cor}
\begin{proof}
	Por planitud genérica se obtiene inmediatamente el inciso 2.
	Pasando al abierto $U \subseteq Y$ y restringiéndose a $f|_{f^{-1}[U]}$ podemos suponer que tenemos un morfismo plano.
	Sean $\eta \in X, \xi \in Y$ los puntos genéricos; como $f$ es dominante, sabemos que $f(\eta) = \xi$ y, como $\xi \in U$,
	por el teorema~\ref{thm:flat_dim_formula} tenemos que $\dim(X_\xi) = \dim X - \dim Y$ y, aplicándo el mismo teorema concluimos el inciso 1.
\end{proof}

\subsection{Morfismos étale}
\begin{mydef}
	Sea $f \colon X \to Y$ un morfismo de tipo finito entre esquemas localmente noetherianos.
	% Sea $x \in X$ un punto e $y := f(x) \in Y$.
	Se dice que $f$ es \strong{no ramificado}\index{morfismo!no ramificado}\index{no ramificado!(morfismo)} en $x\in X$
	si el homomorfismo de anillos $f^\sharp_x \colon \mathscr{O}_{Y, f(x)} \to \mathscr{O}_{X, x}$ satisface que
	$$ \mathfrak{m}_{Y, f(x)} \mathscr{O}_{X, x} = \mathfrak{m}_{X, x} \iff \frac{\mathscr{O}_{X, x}}{\mathfrak{m}_{Y, f(x)} \mathscr{O}_{X, x}} \cong \kk(x) $$
	y si la extensión finita de cuerpos $\kk(x) \supseteq \kk(y)$ es separable.
	Se dice que $f$ es \strong{étale}\index{morfismo!etale@étale} en $x \in X$ si es no ramificado y plano en $x$.

	Se dice que $f$ es \strong{no ramificado} (resp. \strong{étale}) si lo es en cada punto.
\end{mydef}
\begin{ex}
	Sea $L/k$ una extensión finita de cuerpos.
	El morfismo $\Spec L \to \Spec k$ es no ramificado (o equivalentemente, étale) syss la extensión $L/k$ es separable.
\end{ex}

\begin{lem}
	Sea $f \colon X \to Y$ un morfismo de tipo finito entre esquemas localmente noetherianos.
	Entonces $f$ es no ramificado syss es cuasifinito, reducido y para cada $y \in Y$ y $x \in X_y$ se cumple que la extensión $\kk(x) / \kk(y)$ es separable.
\end{lem}
\begin{prop}\label{thm:unram_et_prop}
	Se cumplen:
	\begin{enumerate}
		\item Los encajes cerrados entre esquemas localmente noetherianos son no ramificados.
		\item Los encajes abiertos entre esquemas localmente noetherianos son étale.
		\item Los morfismos no ramificados (resp. étale) son estables salvo composición.
		\item Los morfismos no ramificados (resp. étale) son estables salvo cambio de base.
		\item Los morfismos no ramificados (resp. étale) son estables salvo productos fibrados.
		\item Sean $f \colon X \to Y, g \colon Y \to Z$ un par de morfismos tales que $f\circ g$ es no ramificado (resp. étale)
			y $g$ es separado. Entonces $f$ es no ramificado (resp. étale).
	\end{enumerate}
\end{prop}

\begin{prop}
	Sea $f \colon X \to Y$ un morfismo étale.
	Sean $x \in X$ e $y := f(x)$, entonces se cumplen:
	\begin{enumerate}
		\item $\kdim(\mathscr{O}_{X, x}) = \kdim(\mathscr{O}_{Y, y})$.
		\item El morfismo tangente $T_{f, x} \colon T_{X, x} \to T_{Y, y} \otimes_{\kk(x)} \kk(x)$ es un isomorfismo.
	\end{enumerate}
\end{prop}
\begin{cor}
	Sea $Y$ un esquema localmente noetheriano, y sea $f \colon X \to Y$ un morfismo de tipo finito que es étale en $x \in X$.
	Entonces $X$ es regular en $x$ syss $Y$ es regular en $f(x)$.
\end{cor}

\begin{prop}
	Sea $Y$ un esquema localmente noetheriano, $f \colon X \to Y$ un morfismo de tipo finito y $x \in X_y$ un punto tal que $\kk(x) = \kk(y)$.
	Sea $\widehat{\mathscr{O}}_{Y, y} \to \widehat{\mathscr{O}}_{X, x}$ el homomorfismo entre las compleciones formales.
	Entonces $f$ es étale en $x$ syss $\widehat{\mathscr{O}}_{Y, y} \to \widehat{\mathscr{O}}_{X, x}$ es un isomorfismo.
\end{prop}

\subsection{Morfismos suaves}
\begin{mydef}
	Sea $k$ un cuerpo y $X$ un esquema algebraico sobre $k$.
	Se dice que $X$ es \strong{suave}\index{esquema!suave}\index{suave!(esquema)} en un punto $x \in X$ si $X_{\algcl k}$ es regular
	en todos los puntos de la fibra de $x$.
	Decimos que $X$ es \strong{suave} (a secas) si es suave en todos sus puntos (si es \textit{geométricamente regular}).
\end{mydef}
\begin{ex}
	Sea $k$ un cuerpo.
	Entonces el espacio afín $\A^n_k$ y el espacio proyectivo $\PP^n_k$ son variedades suaves.
\end{ex}

\begin{exn}
	Sea $k$ un cuerpo de $\car k \ne 2$.
	Y sea $C := \Spec( k[x, y] / (y^2 - f(x)) )$, donde $f(x) \in k[x]$ es un polinomio no constante.
	Entonces, aplicando el criterio del jacobiano, uno puede notar que $C$ es suave (syss $C_{\algcl k}$ es regular) syss $f(x)$ no posee raíces
	repetidas en $\algcl k$; lo que equivale a que su discriminante sea no nulo.
\end{exn}
El ejemplo anterior es importante pues reaparecerá en el estudio de curvas elípticas.

\begin{prop}
	Sea $k$ un cuerpo, $X$ un esquema algebraico sobre $k$ y $x \in X$ un punto cerrado. Se cumplen:
	\begin{enumerate}
		\item Si $X$ es suave en $x$, entonces $X$ es regular en $x$.
		\item Si $\kk(x)/k$ es separable (e.g., si $k$ es perfecto) y $X$ es regular en $x$, entonces $X$ es suave en $x$.
	\end{enumerate}
\end{prop}
\begin{proof}
	\begin{enumerate}
		\item Sea $x' \in X_{\algcl k}$ en la fibra de $x$, de modo que es regular.
			Por el lema~\ref{lem:regular_pts_speto_closed_reg} tenemos que $x'$ se especializa en un punto $z' \in X_{\algcl k}$ cerrado y regular;
			por tanto, sea $z \in X$ la imagen de $z'$.
			Entonces $z$ es regular, por la proposición anterior, y es claramente especializa a $x$, por lo que $x$ es regular
			(por la proposición~\ref{thm:regularity_under_gez}).
			\qedhere
	\end{enumerate}
\end{proof}
\begin{cor}\label{thm:geometrically_red_is_generically_smooth}
	Sea $X$ un esquema algebraico sobre un cuerpo $k$. Se cumplen:
	\begin{enumerate}
		\item El conjunto de puntos suaves de $X$, denominado \strong{locus suave}\index{locus!suave}, es abierto.
		\item Si $X$ es geométricamente reducido o si $X$ es irreducible y es reducido en el punto genérico,
			entonces el locus suave es denso.
	\end{enumerate}
\end{cor}
\begin{proof}
	Basta probar la primera, la cual se deduce de leer la proposición anterior como que $X$ es suave en el punto $x$ syss
	$X$ es regular y geométricamente reducido en $x$, por lo que el locus suave es la intersección del locus regular
	con el locus geométricamente reducido, donde ambos son abiertos.
\end{proof}

\begin{mydef}
	Sea $Y$ un esquema localmente noetheriano y sea $f \colon X \to Y$ un morfismo de tipo finito.
	Se dice que $f$ es \strong{suave}\index{suave!(morfismo)} en un punto $x \in X$ si es plano en $x$ y si la fibra $X_y$ con $y = f(x)$ es un
	$\kk(y)$-esquema suave en $x$.
	Se dice que $f$ es un \strong{morfismo suave de dimensión relativa $n$}\index{morfismo!suave} si es suave en todo punto de $X$ y
	si todas las fibras son de dimensión pura $n$.
	El conjunto de puntos donde $f$ es suave, se le llama el \strong{locus suave}\index{locus!suave} de $f$.
\end{mydef}

\section{Haz de diferenciales de Kähler}
\begin{mydef}
	Sea $A$ un anillo, $B$ una $A$-álgebra conmutativa y $M$ un $B$-módulo.
	Una \strong{$A$-derivación}\index{Aderivación@$A$-derivación} es un homomorfismo de $A$-módulos $D \colon B \to M$ tal que para todo $u, v\in B$
	se satisface la \strong{regla de Leibniz}\index{regla!de Leibniz}:
	$$ D(u\cdot v) = vD(u) + uD(v). $$
	El conjunto de $A$-derivaciones $D \colon B \to M$ se denota $\Der_A(B, M)$ y es un $A$-módulo.
\end{mydef}

Es claro que $\Der_A(B, -) \colon \mathsf{Mod}_B \to \mathsf{Mod}_A$ determina un funtor (si el lector lo prefiere, puede suponer que el codominio es
$\mathsf{Set}$) y es un resultado de álgebra conmutativa su representabilidad:
\begin{mydef}
	Sea $B/A$ una álgebra conmutativa.
	Se define el \strong{módulo de diferenciales de Kähler}\index{modulo@módulo!de diferenciales de Kähler} como el $B$-módulo
	$\Omega_{B/A}^1$ junto con una $A$-derivación $\ud \colon B \to \Omega_{B/A}^1$ tal que para toda $A$-derivación $D \colon B \to M$
	\nomenclature{$\Omega_{B/A}$}{Módulo de diferenciales de Kähler}
	existe un único homomorfismo de $B$-módulos $\overline{D} \colon \Omega_{B/A}^1 \to M$ tal que el siguiente diagrama conmuta:
	\begin{center}
		\begin{tikzcd}
			B \rar["\ud"] \drar["D"'] & \Omega_{B/A}^1 \dar[dashed, "\exists!\overline{D}"] \\
			{}                        & M
		\end{tikzcd}
	\end{center}
\end{mydef}

Claramente una manera de construirlo es construir el $B$-módulo libre $B^{\oplus B}$ y cocientar por el submódulo
generado por la restricción de la regla de Leibniz.

Las propiedades generales del módulo $\Omega_{B/A}$ están contenidas en el apéndice \S\ref{sec:commalg_kahler_diff}.
\begin{prop}
	Sea $f \colon X \to Y$ un morfismo de esquemas.
	Existe un único haz cuasicoherente $\mathscr{O}_{X/Y}^1$ sobre $X$ tal que para todo punto $x \in X$,
	y para todo abierto afín $U \subseteq f^{-1}[V]$ contenido en la preimagen de un abierto afín $V \subseteq Y$ tenemos que
	$$ \Omega_{X/Y}^1 |_U \simeq \widetilde{ \Omega^1_{ {\mathscr{O}}_X(U) / {\mathscr{O}}_Y(V) } }, \qquad
	( \Omega_{X/Y}^1 )_x \cong \Omega_{\mathscr{O}_{X, x} / \mathscr{O}_{Y, f(x)}}^1. $$
	Éste haz se llama el \strong{haz de diferenciales relativos de grado 1}\index{haz!de diferenciales relativos} sobre $X/Y$.
	Si $Y = \Spec A$ se denota $\Omega_{X/A}^1$ y obviaremos el subíndice <<$Y$>> de no haber ambigüedad.
\end{prop}
\begin{proof}
	Sea $\varphi \colon A \to B$ un homomorfismo de anillos, sea $\mathfrak{q} \in \Spec B$ un primo y sea $\mathfrak{p} := \varphi^{-1}[\mathfrak{q}] \in
	\Spec A$, entonces la proposición~\ref{app:Omega_base_change} nos da
	$$ \Omega_{B/A}^1 \otimes_B B_{\mathfrak{q}} \cong \Omega_{B_{\mathfrak{q}}/A}^1 \cong \Omega_{B_{\mathfrak{q}} / A_{\mathfrak{p}}}^1. $$
	Por simplicidad, denotemos $\Omega_x^1 := \Omega_{\mathscr{O}_{X, x} / \mathscr{O}_{Y, f(x)}}^1$ para $x \in X$.
	Dado un abierto afín $V \subseteq Y$ y un abierto afín $U \subseteq f^{-1}[V]$, sea $\omega \in \Omega_{\mathscr{O}_X(U) / \mathscr{O}_Y(V)}^1$ y
	sea $x \in U$.
	Denotemos por $\omega|_x$ la imagen de $\omega$ en $\Omega_{\mathscr{O}_X(U) / \mathscr{O}_Y(V)}^1 \otimes \mathscr{O}_{X, x} \cong \Omega_x^1$.

	Ahora, finalmente definimos $\Omega_{X/Y}^1(U)$ como las funciones $s \colon U \to \coprod_{x\in U} \Omega_x^1$ tales que para todo $x \in U$
	existe un entorno afín $V_x \subseteq Y$ de $f(x)$, un entorno afín $x \in U_x \subseteq f^{-1}[V_x] \cap U$ y un elemento $\omega
	\in \Omega_{\mathscr{O}_X(U) / \mathscr{O}_Y(V)}^1$ tal que
	$$ \forall z \in U_x \quad s(z) = \omega|_z. $$
	Con las restricciones triviales es fácil ver que $\Omega_{X/Y}^1$ es un $\mathscr{O}_X$-módulo y también es fácil verificar que
	$( \Omega_{X/Y}^1 )_x \cong \Omega_x^1$.
	Sea $U \subseteq f^{-1}[V]$ un abierto afín con $V \subseteq Y$ abierto afín, entonces tenemos un homomorfismo natural de $\mathscr{O}_X(U)$-módulos
	$\Omega_{\mathscr{O}_X(U) / \mathscr{O}_Y(V)}^1 \to \Omega_{X/Y}^1(U)$ que induce un morfismo de $\mathscr{O}_X|_U$-módulos:
	\[
		\widetilde{ \Omega^1_{ {\mathscr{O}}_X(U) / {\mathscr{O}}_Y(V) } } \longrightarrow \Omega_{X/Y}^1|_U,
	\]
	el cual es un isomorfismo pues lo es en las fibras.
	Así pues $\Omega_{X/Y}^1$ satisface todo lo exigido.
\end{proof}
% \warn
% Hay que tener ojo con la expresión <<un único haz tal que...>>, en realidad nos referimos a las restricciones (que son aplicar $(-)_*$ sobre los morfismos
% de inclusión) dan canónicamente el haz 

\begin{cor}
	Si $f \colon X \to Y$ es un morfismo de tipo finito entre esquemas noetherianos,
	entonces $\Omega_{X/Y}^1$ es un haz coherente sobre $X$.
\end{cor}
\begin{proof}
	Basta aplicar el corolario~\ref{app:kahler_diff_finite_type}.
\end{proof}

Las propiedades de \S\ref{sec:commalg_kahler_diff} ahora se reescriben como:
\begin{thm}\label{thm:rel_diff_props}
	Sea $f \colon X \to Y$ un morfismo de esquemas. Entonces:
	\begin{enumerate}
		\item (Cambio de base) Para todo $Y$-esquema $W$ se cumple que
			$$ \Omega_{X_W/Y_W}^1 = \Omega_{X_W/W}^1 \simeq p^*\Omega_{X/Y}^1, $$
			donde $p \colon X_W \to X$ es la proyección canónica.
		\item\label{thm:rel_diff_props_exact}
			Para todo morfismo de esquemas $Y \to Z$ se tiene la siguiente sucesión exacta:
			\begin{center}
				\begin{tikzcd}
					f^*\Omega_{Y/Z}^1 \rar & \Omega_{X/Z}^1 \rar & \Omega_{X/Y}^1 \rar & 0
				\end{tikzcd}
			\end{center}
		\item Para todo abierto $U \subseteq X$ tenemos $\Omega_{X/Y}^1|_U \simeq \Omega_{U/Y}^1$.
			Para todo punto $x \in X$ tenemos $(\Omega_{X/Y}^1)_x \cong \Omega_{\mathscr{O}_{X, x}/\mathscr{O}_{Y, f(x)}}^1$.
		\item Si 
			\begin{tikzcd}[cramped, sep=small]
				Z \rar[closed] & X
			\end{tikzcd}
			es un subesquema cerrado dado por $Z = \VV(\mathscr{I})$, entonces tenemos la sucesión exacta:
			\begin{center}
				\begin{tikzcd}
					\mathscr{I/I}^2 \rar & \Omega_{X/Y}^1 \otimes_{\mathscr{O}_X} \mathscr{O}_Z \rar & \Omega_{Z/Y}^1 \rar & 0
				\end{tikzcd}
			\end{center}
	\end{enumerate}
\end{thm}

Veamos unos ejemplos sobre el cómo calcular el haz de diferenciales relativos:
\begin{exn}\label{ex:affine_space_diff}
	Sea $Y$ un esquema y $X := \A_Y^n$.
	Si $Y = \Spec A$ fuese afín, entonces por el hecho de que $\Omega_{A[x_1, \dots, x_n]/A}^1 = A[x_1, \dots, x_n]^n$
	(proposición~\ref{app:Omega_calc_example}) implica que, sobre abiertos afines $U$, tenemos $\Omega_{X/Y}^1|_U \simeq \mathscr{O}_X^n|_U$.
	Estos isomorfismos son claramente compatibles entre sí de lo que se sigue que $\Omega_{X/Y}^1 \simeq \mathscr{O}_X^n$.
\end{exn}

\begin{thm}\label{thm:kahler_diff_proy_sp}
	Sea $Y := \Spec A$ un esquema afín y sea $X := \Proj(A[x_0, \dots, \break x_n]) = \PP_A^n.$
	Entonces existe una sucesión exacta de $\mathscr{O}_X$-módulos:
	\begin{center}
		\begin{tikzcd}[sep=large]
			0 \rar & \Omega_{X/Y}^1 \rar & \mathscr{O}_X(-1)^{\oplus(n+1)} \rar & \mathscr{O}_X \rar & 0.
		\end{tikzcd}
	\end{center}
\end{thm}
\begin{proof}
	Sea $B := A[\vec x]$, la cual es una $A$-álgebra graduada y sea $M := B(-1)^{n+1}$ el $B$-módulo graduado, donde $B(-1)$ denota un torcimiento
	(en los grados).
	Sean $\vec e_0, \dots, \vec e_n$ la base en grado 1 y sea $\varphi \colon M \to B$ el homomorfismo de $B$-módulos dado por $\vec e_i \mapsto x_i$.
	Entonces tenemos una sucesión exacta 
	\begin{tikzcd}[cramped, sep=small]
		0 \rar & K \rar & M \rar & B
	\end{tikzcd}
	de $B$-módulos graduados que induce una sucesión exacta
	\begin{center}
		\begin{tikzcd}[sep=large]
			0 \rar & \widetilde{K} \rar & \mathscr{O}_X(-1)^{\oplus(n+1)} \rar[red] & \mathscr{O}_X \rar & 0
		\end{tikzcd}
	\end{center}
	donde la flecha roja es un epimorfismo pues lo es en grados suficientemente grandes.

	Veremos que $\widetilde{K} \simeq \Omega_{X/Y}^1$.
	Localizando por $x_i$, vemos que $M[x_i^{-1}] \to B[x_i^{-1}]$ es un epimorfismo de $B[x_i^{-1}]$-módulos libres, de modo que $K[x_i^{-1}]$
	es un módulo libre de rango $n$ generado por $\{ \vec e_j - \frac{x_j}{x_i}\vec e_i \}_{i\ne j}$.
	Tomando $U_i := \DD_+(x_i)$, lo anterior se traduce en que $\widetilde{K}|_{U_i}$ es un $\mathscr{O}_X|_{U_i}$-módulo
	libre generado por secciones globales
	\[
		\left\{ \frac{1}{x_i}\vec e_j - \frac{x_j}{x_i^2}\vec e_i \right\}_{i\ne j},
	\]
	(donde el factor $1/x_i$ es para que los elementos tengan grado 0, recordando el torcimiento.)

	Recuérdese que $U_i \cong \Spec(A[x_0/x_i, x_1/x_i, \dots, x_n/x_i])$, así que definamos $\varphi_i \colon \Omega_{X/Y}^1|_{U_i} \to \widetilde{K}|_{U_i}$
	mediante la regla
	$$ (\varphi_i)_{U_i}( \ud(x_i/x_j) ) := \frac{1}{x_i^2}( x_i\vec e_j - x_j\vec e_i ), $$
	(por el ejemplo anterior, $\Omega_{\A^n_A/A}^1$ es libre y generado por los símbolos $\ud(x_i/x_j)$.)
	Así cada $\varphi_i$ es un isomorfismo y son compatibles, pues en $U_i \cap U_j = \DD_+(x_ix_j)$
	para cada $\ell$ tenemos que $x_\ell/x_i = (x_\ell/x_j) \cdot (x_j/x_i)$ y en $\Omega_{X/Y}^1|_{U_i \cap U_j}$ se tiene
	$$ \varphi_i\big( \ud(x_\ell/x_i) \big) = \ud\left( \frac{x_\ell}{x_i} \right) - \frac{x_\ell}{x_j} \ud\left( \frac{x_j}{x_i} \right)
	= \frac{x_j}{x_i} \ud\left( \frac{x_\ell}{x_i} \right) = \varphi_j\big( \ud(x_\ell/x_i) \big), $$
	lo cual verifica que son compatibles.
\end{proof}

\begin{ex}
	Sea $A$ un anillo, $B := A[x_1, \dots, x_n]$, sea $F \in B$ y $C := B/(F)$.
	Definiendo $X = \Spec B$, vemos que 
	\begin{tikzcd}[cramped, sep=small]
		Z = \Spec C = \VV(F) \rar[closed] & X.
	\end{tikzcd}
	Luego, empleando el ejemplo~\ref{ex:affine_space_diff} y el inciso 4 de la proposición~\ref{thm:rel_diff_props} nos dan que
	$$ \Omega_{C/A}^1 \cong \frac{ \bigoplus_{i=1}^n C \, \ud x_i }{C \, \ud F}, $$
	donde $\ud F = \sum_{i=1}^{n} (\partial F/\partial x_i) \, \ud x_i$.
\end{ex}
\begin{exn}\label{exn:diff_over_smooth_affine_curve}
	Sea $k$ un cuerpo y sea $X := \Spec( k[T, S]/f(T, S) ) = \VV(f) \subseteq \A^2_k$ una curva afín suave.
	Denotemos por $t, s$ las imágenes de $T, S$ en $\Gamma(X, \mathscr{O}_X)$ y denotemos por $\partial_t F$ la imagen de $\partial F/\partial T \in k[T, S]$
	en $\Gamma(X, \mathscr{O}_X)$ y análogamente con $s$.
	Por el ejemplo anterior, sobre el abierto principal $\DD( \partial_s F )$, el ejemplo anterior nos da que
	$\Gamma\big(\DD( \partial_s F ), \Omega_{X/k}^1\big)$ es libre generado por $\ud t/\partial_s F$ y en el otro abierto principal
	$\Gamma\big(\DD( \partial_t F ), \Omega_{X/k}^1\big)$ es libre generado por $\ud s/\partial_t F$.

	Como $\partial_t F \, \ud t = - \partial_s F \, \ud s$ y como $\DD( \partial_s F ) \cup \DD( \partial_t F ) = X$ por el criterio del jacobiano,
	vemos que $\Gamma(X, \Omega_{X/k}^1)$ es libre generado por $\ud t/\partial_s F$.
\end{exn}

\begin{mydefi}
	Sea $k$ un cuerpo.
	Una \strong{curva elíptica}\index{curva!elíptica} sobre $k$ es una curva suave proyectiva $E$ sobre $k$ que es isomorfa a una subvariedad
	cerrada de $\PP^2_k$ dada por una \strong{ecuación de Weierstrass (larga)}\index{ecuación!de Weierstrass!larga}:
	\begin{equation}
		v^2w + a_1uvw + a_3vw^2 = u^3 + a_2u^2w + a_4uw^2 + a_6w^3,
		\label{eq:long_wei_form}
	\end{equation}
	con el punto distinguido $o = [0 : 1 : 0]$.
\end{mydefi}
La ecuación de Weierstrass usual sale de cortar con el abierto afín $w = 1$;
uno puede memorizar los índices asignandole los <<pesos>> $u \to 2$ e $v \to 3$, en consecuencia, no hay $a_5$.
\begin{prop}
	Sea $E$ una curva elíptica sobre un cuerpo $k$ dada por una ecuación de Weierstrass \eqref{eq:long_wei_form}.
	Sean $x := u/w, y := v/w \in K(E)$ y
	$$ \omega := \frac{\ud x}{2y + a_1x + a_3} \in \Omega_{K(E)/k}^1. $$
	Entonces $\Omega_{E/k}^1 = \omega \mathscr{O}_E$.
\end{prop}
\begin{proof}
	Nótese que $E \subseteq \PP^2_k$ es la unión de los abiertos principales $U := \DD_+(w) \cap E, V := \DD_+(v) \cap E$ cuyos anillo de coordenadas son
	\begin{align*}
		\Gamma(U, \mathscr{O}_E) &= \frac{k[x, y]}{( y^2 + (a_1x + a_3)y - (x^3 + a_2x^2 + a_4x + a_6) )}, \\
		\Gamma(V, \mathscr{O}_E) &= \frac{k[t, z]}{( z + a_1tz + a_3z^2 - (t^3 + a_2t^2z + a_4tz^2 + a_6z^3) )},
	\end{align*}
	donde $t := u/v = x/y$ y $z := w/v = 1/y$.

	Por el ejemplo~\ref{exn:diff_over_smooth_affine_curve} vemos que $\Omega_{U/k}^1$ es libre y generado por $\omega$,
	mientras que $\Omega_{V/k}^1$ es libre y generado por
	$$ \omega' := \frac{\ud z}{a_1z - (3t^2 + 2a_2tz + a_4z^2)}. $$
	Finalmente, en $\Omega_{K(E)/k}^1$ tenemos que $\ud z = -\ud y/y^2$, por lo que es fácil comprobar que
	\begin{equation}
		\omega' = \frac{-\ud y}{a_1y - (3x^2 + 2a_2x + a_4)} = \omega.
		\tqedhere
	\end{equation}
\end{proof}

\begin{lem}
	Sea $k$ un cuerpo y $A$ una $k$-álgebra de tipo finito.
	Sea $x \in (\Spec A)(k)$ un punto $k$-racional y sea $\mathfrak{m} := \mathfrak{p}_x \in \Spec A$.
	Entonces el homomorfismo de la segunda sucesión fundamental
	\begin{center}
		\begin{tikzcd}[sep=large]
			\delta \colon \mathfrak{m/m}^2 \rar["\sim"] & \Omega_{A/k}^1 \otimes_A \kk(x)
		\end{tikzcd}
	\end{center}
	es un isomorfismo.
\end{lem}
\begin{proof}
	Por la segunda sucesión fundamental $\coker\delta = 0$, así que basta ver que $\ker\delta = 0$.
	Sea $\pi \colon B := k[\vec t] \epicto A$ un $k$-homomorfismo suprayectivo, y sea $\mathfrak{n} := \pi^{-1}[\mathfrak{m}] \in \Spec B$.
	Entonces, se tiene el siguiente diagrama con filas exactas:
	\begin{center}
		\begin{tikzcd}
			\mathfrak{a} \dar[equals] \rar[dotted] & \mathfrak{n/n}^2 \dar["\delta'"] \rar & \mathfrak{m/m}^2 \dar["\delta"] \rar & 0 \\
			\mathfrak{a} \rar["\gamma"] & \Omega_{B/k}^1 \otimes \kk(x) \rar[dotted] & \Omega_{A/k}^1 \otimes \kk(x) \rar[dotted] & 0
		\end{tikzcd}
	\end{center}
	donde $\gamma$ es la composición $\mathfrak{a \to a/a}^2 \to \Omega_{B/k}^1 \otimes_B A$ tensorizado por $\kk(x)$.
	Una aplicación del lema de la serpiente da que $\ker\delta = \ker(\delta')$, el cual es cero.
\end{proof}

\begin{prop}
	Sea $X$ una variedad algebraica sobre un cuerpo $k$ y sea $x \in X$.
	Son equivalentes:
	\begin{enumerate}
		\item $\Omega_{X, x}^1$ es libre de rango $\dim_x X$.
		\item $X$ es suave en un entorno de $x$.
		\item $X$ es suave en $x$.
	\end{enumerate}
\end{prop}
\begin{proof}
	$1 \implies 2$.
	Como $X$ es localmente noetheriano, si $\Omega_{X, x}^1$ es libre de rango $n := \dim_x X$ en $x$,
	entonces es libre de rango $n$ en un entorno $U$ de $x$ (prop.~\ref{thm:local_freeness_on_noeth}).
	Podemos suponer que $U$ es conexo y que tiene $\dim U = n$.
	Sea $V := U_{\algcl k}$, el cual tiene $\dim V = n$ (prop.~\ref{thm:dimension_on_alg_base_change}).
	Aplicando cambio de base, vemos que $\Omega_{V/\algcl k}^1$ también es libre de rango $n$ y, por el lema anterior,
	para todo punto cerrado $y \in \clpt V$ tenemos que
	$$ \dim_{\kk(y)}( T_{V, y} ) = \dim_{\kk(y)}\big( \Omega_{V, y}^1 \otimes \kk(y) \big) = n, $$
	de modo que $V$ es suave en todos los puntos $y$ tales que $\dim_y V = n$.

	Veamos $U$ es irreducible. En efecto, de lo contrario habrían dos componentes irreducibles que se cortan en un punto cerrado $x_0 \in \clpt U$
	de $\dim_{x_0}(U)$.
	Pero hemos visto que tal punto sería suave, luego íntegro.
	Por lo tanto, $U$ es irreducible y $V$ tiene dimensión $n$ en todos los puntos cerrados, de modo que $U$ es suave.

	$2 \implies 3$. Trivial.

	$3 \implies 1$. Sea $x' \in X_{\algcl k}$ en la fibra de $x$.
	...
\end{proof}
La proposición anterior da una demostración (muchísimo más sencilla) de que el conjunto de puntos regulares es abierto.

\begin{cor}\label{thm:unram_omega_criteria}
	Sea $f \colon X \to S$ un morfismo de tipo finito entre esquemas localmente noetherianos.
	Entonces $f$ es no ramificado sobre un punto $x \in X$ syss $\Omega_{X/S, x}^1 = 0$.
\end{cor}

\begin{lem}
	Sea $X \to S$ un morfismo de tipo finito entre esquemas localmente noetherianos.
	Sean $s \in S, x \in X_s$ y defínase
	\[
		d := \dim_{\kk(x)}\big( \Omega_{X_s/\kk(s), x}^1 \otimes_{\mathscr{O}_{X_s, x}} \kk(x) \big).
	\]
	Entonces en un entorno de $x$ existe un encaje cerrado 
	\begin{tikzcd}[cramped, sep=small]
		X \rar[closed] & Z
	\end{tikzcd}
	donde $Z$ es un $S$-esquema suave en $x$ tal que $\dim_x(Z_s) = d$ y tal que $ \Omega_{Z/S, x}^1 $ es un $\mathscr{O}_{Z, x}$-módulo libre
	de rango $d$.
\end{lem}

\begin{prop}
	Sea $S$ un esquema localmente noetheriano y $f\colon X \to S$ un morfismo de tipo finito. Se cumplen:
	\begin{enumerate}
		\item Si $f$ es suave en $x$, entonces existe un entorno $x \in U$ tal que $\Omega_{X/S}^1|_U$ es libre de rango $\dim_x(X_s)$,
			donde $s := f(x)$.
		\item Si $f$ es plano y las fibras son de dimensión pura $n$.
			Entonces $f$ es suave syss $\Omega_{X/S}^1$ es localmente libre de rango $n$.
	\end{enumerate}
\end{prop}

\begin{cor}\label{thm:first_fund_seq_smooth}
	Sea $S$ un esquema localmente noetheriano y sea $f\colon X \to Y$ un morfismo entre $S$-esquemas suaves.
	Entonces se tiene la siguiente sucesión exacta:
	\begin{center}
		\begin{tikzcd}[sep=large]
			0 \rar & f^* \Omega_{Y/S}^1 \rar["\alpha"] & \Omega_{X/S}^1 \rar & \Omega_{X/Y}^1 \rar & 0
		\end{tikzcd}
	\end{center}
\end{cor}
\begin{hint}
	Falta verificar la inyectividad de la flecha $\alpha$, lo cual se verifica en fibras empleando que todos son haces localmente libres.
\end{hint}

\begin{thm}
	Sea $S$ un esquema localmente noetheriano, regular y conexo, $X$ un esquema irreducible y $f \colon X \to S$ un morfismo de tipo finito.
	Sea $x \in X$ un punto tal que para $s := f(x)$ tenemos
	\begin{equation}
		\kdim\mathscr{O}_{X, x} = \kdim\mathscr{O}_{X_s, x} + \kdim\mathscr{O}_{S, s}.
		\label{eqn:smooth_dimension_condition}
	\end{equation}
	Entonces $f$ es suave en $x$ syss $\Omega_{X/S, x}^1$ es un $\mathscr{O}_{X, x}$-módulo libre de rango $\dim_x(X_s)$.
\end{thm}

\begin{cor}\label{thm:vars_sm=gen_sm}
	Sea $X$ un esquema algebraico sobre un cuerpo $k$.
	Entonces $X$ es suave syss $\Omega_{X/k}^1$ es localmente libre y para todo punto genérico $\xi \in X$ se satisface que
	la extensión $\kk(\xi)/k$ es separable.
	En particular, si $k$ es perfecto, entonces $X$ es suave syss $\Omega_{X/k}^1$ es localmente libre.
\end{cor}
% \todo{Pensar demostración, ver \citeauthor{liu:algebraic}~\cite{liu:algebraic}, ex.~6.2.2.}
\begin{proof}
	% Si el esquema $X$ es suave, entonces debe serlo en al menos el punto genérico $\xi \in X$.
	Por evitamiento de primos podemos restringirnos a un abierto irreducible y así suponer que $X$ es irreducible con punto genérico $\xi$.
	Nótese que $X$ siempre posee al menos un punto regular, luego sus generizaciones son regulares y, en particular, $\xi$ es regular en $X$, de modo que
	la extensión $\kk(\xi)/k$ es separable syss $X$ es suave en $\xi$.
	Como $\Omega_{X/k}^1$ es localmente libre, tenemos que $\Omega_{X/k, \xi}^1 = \Omega_{\kk(x)/k}^1 \cong \kk(x)^n$ y, como es suave,
	de hecho $n = \trdeg_k\big( \kk(x) \big) = \dim X$.
	Así concluimos que $\Omega_{X/k}^1$ es localmente libre de rango $\dim X$ y, por el teorema anterior, estamos listos.
\end{proof}

\begin{prop}
	Sea $S$ un esquema localmente noetheriano y sea $f \colon X \to Y$ un morfismo entre $S$-esquemas de tipo finito.
	\begin{enumerate}
		\item Si $f$ es étale en un punto $x \in X$, entonces el homomorfismo canónico
			\begin{equation}
				\varphi \colon (f^* \Omega_{Y/S})_x \longrightarrow ( \Omega_{X/S}^1 )_x
				\label{eqn:relative_diff_hom}
			\end{equation}
			es un isomorfismo.
		\item Si $X$ (resp. $Y$) es suave sobre $S$ en $x$ (resp. $f(x)$) y el homomorfismo canónico \eqref{eqn:relative_diff_hom} es un isomorfismo,
			entonces $f$ es étale en $x$.
	\end{enumerate}
\end{prop}
\begin{cor}\label{thm:smooth_morph_decomp}
	Sea $S$ un esquema localmente noetheriano y $f \colon X \to S$ un morfismo suave en $x \in X$.
	Existe un entorno $U$ de $x$ tal que el siguiente diagrama conmuta:
	\begin{center}
		\begin{tikzcd}[row sep=large]
			U \dar[open] \rar["g"] & \A_S^n \dar \\
			X            \rar["f"] & S
		\end{tikzcd}
	\end{center}
	donde $g$ es étale en $x$.
\end{cor}
\begin{proof}
	Restringiéndose a un abierto de $S$ (y recordando que la suavidad es local), podemos suponer que $S = \Spec A$ es afín.
	Sea $\ud f_1, \dots, \ud f_n \in \Omega_{X, x}^1$ una $\mathscr{O}_{X, x}$-base; restringiéndose a un abierto $U$ de $X$
	podemos suponer que cada $f_i$ es regular en $U$; por lo que determinan un $A$-homomorfismo:
	$$ \ev_{f_1, \dots, f_n}\colon A[t_1, \dots, t_n] \longrightarrow \Gamma(U, \mathscr{O}_X) $$
	que manda cada $t_j \mapsto f_j$.
	Por adjunción esto determina un $S$-morfismo $g \colon U \to \A^n_S =: Y$, veamos que es étale en $x$.
	Nótese que induce el homomorfismo sobre haces
	$$ g^* \Omega_{Y/S}^1 \longrightarrow \Omega_{X/S}^1 $$
	que manda $\ud t_j \mapsto \ud f_j$, de modo que en la fibra de $x$ determina un epimorfismo $(g^* \Omega_{Y/S}^1)_x \to (\Omega_{X/S}^1)_x$.
	Finalmente, como ambos son localmente libres, en realidad determina un isomorfismo y concluimos por la proposición anterior.
\end{proof}

\begin{mydef}
	Sea $X$ un esquema.
	Un \strong{engrosamiento}\index{engrosamiento}\footnotemark{} de $X$ es un subesquema cerrado 
	\begin{tikzcd}[cramped, sep=small]
		Z \rar[closed] & X
	\end{tikzcd}
	tal que comparten el mismo espacio topológico.
	En otras palabras, es un subesquema cerrado $Z = \VV(\mathscr{I})$, donde $\mathscr{I \subseteq O}_X$ es un haz de ideales
	tal que para todo punto $x \in X$, el ideal $\mathscr{I}_x \nsle \mathscr{O}_{X, x}$ es nilpotente.
	Se dice que el engrosamiento $Z = \VV(\mathscr{I})$ es de \strong{orden finito}\index{engrosamiento!de orden finito}
	(resp.\ \emph{de primer orden}) si existe $n \in \N$ tal que $\mathscr{I}^n = 0$ (resp.\ si $\mathscr{I}^2 = 0$).
\end{mydef}
\footnotetext{Eng.\ \textit{thickening}. El término es original de \cite{stacks}, \href{https://stacks.math.columbia.edu/tag/04EX}{\urlstyle Tag 04EX}.}

\warn
% Ojo que un engrosamiento no es lo mismo que un $\VV_X(\mathscr{I})$, donde $\mathscr{I \subseteq O}_X$ es nilpotente.
Ojo que no todo engrosamiento es de orden finito;
la razón está en que en cada punto la potencia que anule a la fibra $\mathscr{I}_x$ puede crecer.
Cuando $X$ es noetheriano, entonces el haz de ideales $\mathscr{I}$ sí es nilpotente y si hay equivalencia.

\begin{prop}
	Sea $S$ un esquema localmente noetheriano y $f \colon X \to S$ un morfismo suave (resp.\ étale, no ramificado).
	Para todo $S$-esquema $Y$ y todo $S$-engrosamiento 
	\begin{tikzcd}[cramped, sep=small]
		i\colon Y' \rar[closed] & Y,
	\end{tikzcd}
	la función canónica
	\begin{equation*}
		\Hom_S(Y, X) \longrightarrow \Hom_S(Y', X), \qquad f \mapsto i\circ f
	\end{equation*}
	es sobreyectiva (resp.\ biyectiva, inyectiva).
\end{prop}
% Esto nos diría, en lenguaje de \cite{stacks}, secciones \href{https://stacks.math.columbia.edu/tag/02H7}{\urlstyle 02H7},
% \href{https://stacks.math.columbia.edu/tag/02HF}{\urlstyle 02HF} y \href{https://stacks.math.columbia.edu/tag/02GZ}{\urlstyle 02GZ},
La proposición anterior nos dice, en lenguaje de \citeauthor{gortz:algebraic_ii}~\cite[36-40]{gortz:algebraic_ii},
que un morfismo suave (resp.\ étale, no ramificado) es lo mismo que
un morfismo \textit{formalmente} suave (resp.\ formalmente étale, formalmente no ramificado) de tipo finito.

\subsection{Intersección completa local}
Recuérdese que dado un $A$-módulo $M$ decimos que $a \in A$ es \strong{$M$-regular} si la $a$-torsión $M[a] = 0$.
Decimos que una sucesión $(a_1, \dots, a_n)$ es \strong{débilmente $M$-regular} si $a_1$ es $M$-regular y cada $a_r$
es $M/(a_1, \dots, a_{r-1})M$-regular.

\warn
Esta definición de álgebra conmutativa es dependiente del orden en general, pero si $(A, \mathfrak{m})$ es un anillo local noetheriano,
$M$ es finitamente generado y cada $a_i \in \mathfrak{m}$, entonces es independiente del orden.

\begin{mydef}
	Sea $Y$ un esquema localmente noetheriano y $f \colon X \to Y$ un encaje.
	Se dice que $f$ es un \strong{encaje regular}\index{encaje!regular} (resp. \strong{encaje regular de codimensión $n$})%
	\index{encaje!regular!de codimensión $n$} en un punto $x \in X$ si $\ker( \mathscr{O}_{Y, f(x)} \to \mathscr{O}_{X, x} )$ es un
	$\mathscr{O}_{Y, f(x)}$-módulo generado por una sucesión regular (resp. sucesión regular de $n$ elementos).
	Se dice que $f$ es un \strong{encaje regular} (resp. \strong{encaje regular de codimensión $r$}) si lo es en cada punto de $X$.
\end{mydef}

\begin{lem}\label{thm:a2quot_free}
	Sea $A$ un anillo y $\mathfrak{a} \nsl A$.
	Si $\mathfrak{a}$ está generado por una sucesión (débilmente) $\mathfrak{a}$-regular $a_1, \dots, a_n$,
	entonces $\mathfrak{a/a}^2$ es un $A/\mathfrak{a}$-módulo libre con base $a_1, \dots, a_n \mod{\mathfrak{a}}$.
\end{lem}
\begin{proof}
	del 
	Dados $u_1, \dots, u_n \in A$ tales que
	$$ \sum_{i=1}^{n} a_iu_i \in \mathfrak{a}^2 = \sum_{i=1}^{n} a_i\mathfrak{a}, $$
	existen $w_i \in \mathfrak{a}$ tales que $\sum_{i=1}^{n} a_i(u_i - w_i) = 0$
	Luego, por el lema~\ref{app:regular_base}, concluimos que $u_i \in \mathfrak{a}$, lo que prueba que los $a_i \mod{\mathfrak{a}}$
	conforman una base.
\end{proof}

\begin{mydef}
	% Sea $X$ una variedad suave sobre un cuerpo $k$.
	% Definimos el \strong{haz tangente}\index{haz!tangente} como $\mathscr{T}_X := \shHom( \Omega_{X/k}^1, \mathscr{O}_X ) = ( \Omega_{X/k}^1 )^\vee$.

	Sea $f \colon X \to Y$ un encaje.
	Entonces sea $V \subseteq Y$ un subesquema abierto tal que $f$ se factoriza por un encaje cerrado 
	\begin{tikzcd}[cramped, sep=small]
		i\colon X \rar[closed] & V
	\end{tikzcd}
	y sea $X = \VV_V(\mathscr{I})$.
	El haz $\mathscr{C}_{X/Y} := i^*(\mathscr{I/I}^2)$ se dice el \strong{haz conormal de $X$ en $Y$}\index{haz!conormal} y su dual
	es $\mathscr{N}_{X/Y} := (\mathscr{C}_{X/Y})^\vee$ el \strong{haz normal}\index{haz!normal}.
	% Como $\mathscr{T}_X$ es un haz localmente libre de rango $n := \dim X$, definimos el \strong{haz canónico}\index{haz!canónico} como
	% $ \omega_X := $
\end{mydef}
\begin{cor}\label{thm:conormal_rank}
	Sea $x \colon X \to Y$ un encaje regular (de codimensión $r$), entonces el haz conormal $\mathscr{C}_{X/Y}$ es localmente libre (de rango $r$).
\end{cor}
\begin{proof}
	Igual que en la definición sea $V \subseteq Y$ abierto tal que mediante $f$ tenemos que $X = \VV_V(\mathscr{I})$.
	Sea $x \in X$ un punto y sea $y := f(x)$.
	Entonces $f^*( \mathscr{I/I}^2 )_x = \mathscr{I}_y / \mathscr{I}_y^2$ y $\ker( \mathscr{O}_{Y, y} \epicto \mathscr{O}_{X, x} ) = \mathscr{I}_y$.
	El lema anterior ahora dice que $f^*( \mathscr{I/I}^2 )_x$ es libre (de rango $n$ si $f$ es de codimensión $n$) sobre $\mathscr{O}_{X, x}$.
\end{proof}

\begin{prop}\label{thm:regular_inm_prop}
	Sean $X, Y, W, Z$ un conjunto de esquemas localmente noetherianos. Se cumplen:
	\begin{enumerate}
		\item Sean $f \colon X \to Y, g \colon Y \to Z$ encajes regulares (de codimensión $n, m$ resp.).
			Entonces $f\circ g$ es un encaje regular (de codimensión $n+m$) y tenemos la sucesión exacta
			\begin{equation}
				\begin{tikzcd}
					0 \rar & f^*\mathscr{C}_{Y/Z} \rar & \mathscr{C}_{X/Z} \rar & \mathscr{C}_{X/Y} \rar & 0.
				\end{tikzcd}
				\label{eqn:conormal_sh_seq}
			\end{equation}

		\item Sea 
			\begin{tikzcd}[cramped, sep=small]
				i \colon X \rar[closed] & Y
			\end{tikzcd}
			un encaje cerrado y regular de codimensión $n$.
			Entonces para toda componente irreducible $Y'$ de $Y$ que corta a $X$ tenemos que $\codim(X \cap Y', Y') = n$.
			Más aún, $\kdim(\mathscr{O}_{X, x}) = \kdim(\mathscr{O}_{Y, x}) - n$.

		\item Sea $f \colon X \to Y$ un encaje regular.
			Para todo $Y$-esquema $W$ el homomorfismo canónico $p^* \mathscr{C}_{X/Y} \epicto \mathscr{C}_{X_W/W}$
			es un epimorfismo, donde $p \colon X \times_Y W \to X$ es la proyección.

		\item Sea $f \colon X \to Y$ un encaje regular (de codimensión $n$).
			Entonces para todo $Y$-esquema plano $W$ el cambio de base $f_W\colon X_W \to W$ es un encaje regular (de codimensión $n$)
			y el homomorfismo canónico 
			\begin{tikzcd}[cramped, sep=small]
				p^* \mathscr{C}_{X/Y} \rar["\sim"] & \mathscr{C}_{X_W/W}
			\end{tikzcd}
			es un isomorfismo.
	\end{enumerate}
\end{prop}
\begin{proof}
	\begin{enumerate}
		\item Lo único no trivial es probar la exactitud de \eqref{eqn:conormal_sh_seq}.
			Inmediatamente tenemos una sucesión exacta
			\begin{center}
				\begin{tikzcd}[sep=large]
					f^*\mathscr{C}_{Y/Z} \rar & \mathscr{C}_{X/Z} \rar["\alpha"] & \mathscr{C}_{X/Y} \rar & 0.
				\end{tikzcd}
			\end{center}
			Los haces en la sucesión son coherentes y localmente libres, de modo que $\ker\alpha$ es plano (se puede verificar en abiertos afines,
			donde se reduce a un problema de álgebra) y coherente, por tanto es localmente libre.
			Luego el homomorfismo de haces abelianos $f^*\mathscr{C}_{Y/Z} \epicto \ker\alpha$ es un epimorfismo y, las fibras de ambos
			tienen el mismo rango en las fibras, de modo que debe ser un isomorfismo (¿por qué?).

		\item Pasando a un abierto denso podemos suponer que $Y = \Spec A$ es afín con $A$ noetheriano local de punto cerrado $y$.
			Por hipótesis $\mathscr{I}_y$ tiene una base (como $\mathscr{O}_{Y, y}$-módulo) $a_1, \dots, a_n$ y, por una inducción
			sencilla, basta reducirnos al caso de $n = 1$ (i.e., $X = \VV(a)$).
			Sea $\mathfrak{p}$ el primo minimal de $A$ tal que $\overline{\{ x_{\mathfrak{p}} \}} = Y'$ y sea $b := a\mod{\mathfrak{p}}
			\in A/\mathfrak{p}$.
			Entonces $b\ne 0$ (¿por qué?) y $\dim(X \cap Y') = \dim(Y') - 1$ por el teorema de ideales principales de Krull.
			Finalmente $\dim X = \dim Y - 1$.

		\item Ejercicio.
		\item El que $f_W\colon X_W \to W$ sea un encaje regular viene del lema~\ref{thm:a2quot_free}.
			El que $p^*\mathscr{C}_{X/Y} \to \mathscr{C}_{X_W/W}$ sea un isomorfismo se sigue de que es un epimorfismo entre haces
			localmente libres del mismo rango sobre $X$. \qedhere
	\end{enumerate}
\end{proof}

\begin{prop}\label{thm:inm_between_smooth_are_reg}
	Sea $S$ un esquema localmente noetheriano.
	Sean $X, Y$ esquemas suaves sobre $S$.
	Todo encaje $f \colon X \to Y$ de $S$-esquemas es un encaje regular y tenemos la siguiente sucesión exacta:
	\begin{center}
		\begin{tikzcd}[sep=large]
			0 \rar & \mathscr{C}_{X/Y} \rar & f^* \Omega_{Y/S}^1 \rar & \Omega_{X/S}^1 \rar & 0.
		\end{tikzcd}
	\end{center}
\end{prop}
\begin{proof}
	Sea $x \in X, y := f(x)$ y $s$ la imagen de $y$ (y luego de $x$) en $S$ bajo el morfismo estructural.
	Por inducción sobre $e := \dim_y(Y_s) - \dim_x(X_s)$ probaremos que $f$ es regular de codimensión $e$ en $x$.
	Si $e = 0$, entonces $f$ es un isomorfismo por la proposición~\ref{}.
	\todo{Hacer demostración, \citeauthor{liu:algebraic}~\cite[222]{liu:algebraic}.}
	Si $e \ge 1$, entonces existe $f_1 \in \mathscr{I}_y \setminus \{ 0 \}$ tal que $Z := \VV_Y(f_1)$ es un entorno de $y$
	suave sobre $S$ en $x$.
	Así, la inclusión canónica 
	\begin{tikzcd}[cramped, sep=small]
		Z \rar[closed] & Y
	\end{tikzcd}
	es un encaje regular en $x$ y $\dim_y(Z_s) - \dim_x(X_s) = e - 1$, por lo que concluimos por hipótesis inductiva
	y por el inciso 1 de la proposición anterior.

	Para la sucesión exacta, en primer lugar cambiando $Y$ por un subesquema abierto podemos suponer que 
	\begin{tikzcd}[cramped, sep=small]
		f \colon X \rar[closed] & Y
	\end{tikzcd}
	es un encaje cerrado.
	Luego aplicando el inciso 4 del teorema~\ref{thm:rel_diff_props} construimos la siguiente sucesión exacta:
	\begin{center}
		\begin{tikzcd}
			\mathscr{C}_{X/Y} = \mathscr{I/I}^2 \rar["\delta"] &
			f^* \Omega_{Y/S}^1 = \Omega_{Y/S}^1 \otimes_{\mathscr{O}_Y} \mathscr{O}_X \rar &
			\Omega_{X/S}^1 \rar & 0,
		\end{tikzcd}
	\end{center}
	así que solo queda verificar que $\delta$ es un monomorfismo.
	Ahora bien, los términos de la sucesión son localmente libres y, por el corolario~\ref{thm:conormal_rank} vemos que
	$$ \rang( f^* \Omega_{Y/S}^1 )_x - \rang( \Omega_{X/S}^1 )_x = \dim_y(Y_s) - \dim_x(X_s) = e = \rang( \mathscr{C}_{X/Y} )_s. $$
	Y así $\delta$ establece un epimorfismo con su imagen, y como tienen el mismo rango, ha de ser un isomorfismo, por lo que $\delta$ es un monomorfismo.
\end{proof}

\begin{cor}\label{thm:section_smooth_sep_are_reg}
	Sea $Y$ un esquema localmente noetheriano y sea $f \colon X \to Y$ un morfismo suave y separado.
	Entonces toda sección $\pi \colon Y \to X$ (tal que $\pi \circ f = \Id_Y$) es un encaje cerrado regular
	y tenemos el isomorfismo canónico $\mathscr{C}_{Y/X} \simeq \pi^* \Omega_{X/Y}^1$.
\end{cor}
\begin{proof}
	Por la proposición~\ref{thm:separated_prop}, $X$ es un $Y$-esquema separado y $X, Y$ son $X$-esquemas, de modo que
	\begin{center}
		\begin{tikzcd}[sep=large]
			Y = X \times_X Y \rar[closed, "\pi", near end] & X \times_Y Y = X
		\end{tikzcd}
	\end{center}
	es un encaje cerrado.
	Luego aplicamos la sucesión exacta de la proposición anterior con $\Omega_{Y/Y}^1 = 0$ para concluir el enunciado.
	% ver que $\mathscr{C}_{Y/X} \simeq \pi^* \Omega_{X/Y}^1$.
\end{proof}

\begin{mydef}
	Sea $Y$ un esquema localmente noetheriano y $f \colon X \to Y$ un morfismo de tipo finito.
	Se dice que $f$ es \strong{intersección completa local} (abrev., \strong{i.c.l.})\index{intersección completa local (morfismo)}%
	\index{morfismo!intersección completa local (i.c.l.)}
	en un punto $x \in X$ si existe un entorno $U$ de $x$ y un diagrama conmutativo
	\begin{center}
		\begin{tikzcd}[row sep=large]
			{}                               & Z \dar["g"] \\
			U \rar["f|_U"'] \urar["i", hook] & Y
		\end{tikzcd}
	\end{center}
	donde $i$ es un encaje regular y $g$ es un morfismo suave.
	Se dice que $f$ es de \strong{intersección completa local} (i.c.l.) si lo es en todos los puntos de $X$.
\end{mydef}
\begin{prop}
	Se cumplen:
	\begin{enumerate}
		\item Todo encaje regular y todo morfismo suave son i.c.l.
		\item La composición de morfismos i.c.l. es i.c.l.
		\item Sea $f \colon X \to Y$ un morfismo i.c.l., y sea $W$ un $Y$-esquema plano.
			Entonces $f_W \colon X_W \to W$ es i.c.l.
	\end{enumerate}
\end{prop}
\begin{proof}
	\begin{enumerate}
		\item Trivial.
		\item Como los encajes regulares y los morfismos suaves son estables salvo composición,
			basta probar que si $f \colon X \to Y$ es suave y $g \colon Y \to Z$ es un encaje regular, entonces $f \circ g$ es i.c.l.
			Como $f$ es suave, por el corolario~\ref{thm:smooth_morph_decomp}, en un abierto tenemos que $f$ se factoriza por un encaje cerrado 
			\begin{tikzcd}[cramped, sep=small]
				i\colon X \rar[closed] & \A_Y^n
			\end{tikzcd}
			y la proyección canónica $\A_Y^n \to Y$.
			Luego, haciendo cambio de base por $\A_Z^n$ tenemos que $g'\colon \A_Y^n \hookto \A_Z^n$ es un encaje regular.
			Por la proposición~\ref{thm:inm_between_smooth_are_reg} tenemos que $i$ es regular, luego $i\circ g'$ es un encaje regular
			y claramente la proyección $\A_Z^n \to Z$ es suave.
		\item Basta recordar que tanto los encajes regulares como los morfismos suaves son estables salvo cambio de base plano.
			\qedhere
	\end{enumerate}
\end{proof}


\begin{prop}\label{thm:lci_fact_inmersion_reg}
	Sea $f \colon X \to Y$ un morfismo i.c.l.
	\begin{enumerate}
		\item Si $f$ es un encaje, entonces es un encaje regular.
		\item Si $f$ se descompone en un encaje $i \colon X \hookto Z$ y un morfismo suave $g \colon Z \to Y$, entonces $i$ es un encaje regular.
			Más aún, si $f$ es un encaje regular, se induce la siguiente sucesión exacta
			\begin{center}
				\begin{tikzcd}[sep=large]
					0 \rar & \mathscr{C}_{X/Y} \rar & \mathscr{C}_{X/Z} \rar & i^* \Omega_{Z/Y}^1 \rar & 0.
				\end{tikzcd}
			\end{center}
	\end{enumerate}
\end{prop}
\begin{proof}
	\begin{enumerate}
		\item Pasando a un abierto $U$ podemos emplear la descomposición siguiente (cor.~\ref{thm:smooth_morph_decomp}):
			\begin{center}
				\begin{tikzcd}[row sep=large]
					V := X \times_Z U \dar[open] \rar[hook, "\text{regular}"] & U \dar[open] \rar["\text{étale}"] & \A_Y^n \dar \\
					X \rar[hook, "\text{regular}"']                           & Z \rar["\text{suave}"']            & Y
				\end{tikzcd}
			\end{center}
			de modo que es fácil ver que se factoriza por un encaje regular $V \hookto \A_Y^n$ con la proyección $\A_Y^n \to Y$.
			Luego queda al lector verificar que componer con $\A^Y_n$ da un encaje regular (aplíquese la proposición~\ref{app:flat_tensor_depths}).
		\item Cambiando $Z$ por un subesquema abierto de $Y$ podemos suponer que $i$ es un encaje cerrado.
			Defínase $X \times_Y Z$. entonces tenemos el siguiente diagrama conmutativo:
			% https://q.uiver.app/#q=WzAsNSxbMCwwLCJYIl0sWzEsMCwiVyJdLFsxLDEsIlgiXSxbMiwwLCJaIl0sWzIsMSwiWSJdLFsxLDJdLFsxLDNdLFszLDQsImciXSxbMiw0LCJmIiwyXSxbMCwzLCJpIiwwLHsibGFiZWxfcG9zaXRpb24iOjgwLCJjdXJ2ZSI6LTN9XSxbMCwyLCIiLDIseyJsZXZlbCI6Miwic3R5bGUiOnsiaGVhZCI6eyJuYW1lIjoibm9uZSJ9fX1dLFswLDEsIlxccGkiLDAseyJsYWJlbF9wb3NpdGlvbiI6ODB9XSxbMSw0LCIiLDAseyJzdHlsZSI6eyJuYW1lIjoiY29ybmVyLWludmVyc2UifX1dXQ==
			\[\begin{tikzcd}
				X & W & Z \\
				& X & Y
				\arrow[from=1-2, to=2-2]
				\arrow[from=1-2, to=1-3]
				\arrow["g", from=1-3, to=2-3]
				\arrow["f"', from=2-2, to=2-3]
				\arrow[closed, "i"{pos=0.8}, bend left, from=1-1, to=1-3]
				\arrow[Rightarrow, no head, from=1-1, to=2-2]
				\arrow[closed, "\pi"{pos=0.8}, from=1-1, to=1-2]
				\arrow["\ulcorner"{anchor=center, pos=0.125}, draw=none, from=1-2, to=2-3]
			\end{tikzcd}\]
			donde $\pi$ es la sección de $g_X \colon Z_X \to X$, el cual es suave y separado por ser cambio de base de un morfismo suave y separado,
			de modo que es un encaje cerrado regular.
			Como $Z$ es plano sobre $Y$, el morfismo $f_Z$ es i.c.l.
			Luego $\pi \circ f_Z = i$ es i.c.l., y es un encaje, así que el inciso anterior prueba que es un encaje regular.

			Si suponemos que $f$ es un encaje regular, entonces por el inciso 1 de la proposición~\ref{thm:regular_inm_prop} tenemos la sucesión exacta
			\begin{center}
				\begin{tikzcd}[sep=large]
					0 \rar & \pi^* \mathscr{C}_{W/Z} \rar & \mathscr{C}_{X/Z} \rar & \mathscr{C}_{X/W} \rar & 0
				\end{tikzcd}
			\end{center}
			y por el corolario~\ref{thm:section_smooth_sep_are_reg} tenemos
			\begin{gather*}
				\pi^* \mathscr{C}_{W/Z} \simeq \pi^* p^* \mathscr{C}_{X/Z} \simeq \mathscr{C}_{X/Y}, \\
				\mathscr{C}_{X/W} \simeq \pi^* \Omega_{W/X}^1 \simeq \pi^* q^* \Omega_{Z/Y}^1 \simeq i^* \Omega_{Z/Y}^1,
			\end{gather*}
			que es precisamente lo que queríamos probar.
			\qedhere
	\end{enumerate}
\end{proof}

\begin{lem}
	Sea $S$ un esquema localmente noetheriano y sea $i \colon X \hookto Y$ un encaje de $S$-esquemas localmente noetherianos y planos.
	Sea $s \in S$ un punto y $x \in X_s$ en la fibra, si $i_s \colon X_s \hookto Y_s$ es un encaje regular en $x$,
	entonces $i$ es un encaje regular en $x$.
\end{lem}
\begin{proof}
	Podemo ssuponer que $i$ es un encaje cerrado y definir $\mathscr{I} \subseteq \mathscr{O}_Y$ el haz de ideales
	tal que $X := \VV_Y(\mathscr{I})$. Entonces tenemos la sucesión exacta
	\begin{center}
		\begin{tikzcd}
			0 \rar & \mathscr{I}_x \rar & \mathscr{O}_{Y, x} \rar & \mathscr{O}_{X, x} \rar & 0
		\end{tikzcd}
	\end{center}
	que, localmente sobre $s \in S$ induce
	\begin{center}
		\begin{tikzcd}
			0 \rar & \mathscr{I}_x \otimes_{\mathscr{O}_{S, s}} \kk(s) = \mathscr{I}_x\mathscr{O}_{Y_s, x} \rar &
			\mathscr{O}_{Y_s, x} \rar & \mathscr{O}_{X_s, x} \rar & 0.
		\end{tikzcd}
	\end{center}
	Por hipótesis, existen $a_1, \dots, a_n \in \mathscr{I}_x$ cuyas imágenes $b_1, \dots, b_n \in \mathscr{I}_x\mathscr{O}_{Y_s, x}$
	forman un sistema generador débilmente $\mathscr{O}_{Y_s, x}$-regular. Es decir
	$$ \frac{\mathscr{I}_x}{(a_1, \dots, a_n)} \otimes_{\mathscr{O}_{S, s}} \kk(s) = 0, $$
	lo que implica, por el lema de Nakayama, que $\mathscr{I}_x = (a_1, \dots, a_n)$.

	Finalmente, tenemos la situación de un homomorfismo de anillos locales noetherianos $A := \mathscr{O}_{S, s} \to \mathscr{O}_{Y, x} =: B$
	con un $B$-módulo $N := \mathscr{I}_x$ que es $A$-plano y $M := A$, y concluimos por la proposición~\ref{app:flat_tensor_depths}.
\end{proof}
Este es un primer ejemplo de <<descenso plano>>.

\begin{cor}\label{lci_local_on_fibers}
	Sea $Y$ un esquema localmente noetheriano y $f \colon X \to Y$ un morfismo plano de tipo finito.
	Entonces $f$ es i.c.l. syss para todo $y \in Y$ la fibra $f_y \colon X_y \to \Spec\kk(y)$ es i.c.l.
\end{cor}
\begin{proof}
	$\impliedby$.
	Sobre un abierto afín uno puede descomponer $f$ como un encaje $i \colon X \to Z := \A_Y^n$ con la proyección $p\colon Z \to Y$.
	Sea $y \in Y$ y $x \in X_y$, de modo que $f_y$ es i.c.l.; como $p_y \colon \A_{\kk(y)}^n \to \kk(y)$ es suave, entonces $i_y$ es un encaje regular,
	luego como $Z_y$ es $Z$-plano, entonces descendemos para ver que $i$ es un encaje regular.
	Así $f$ es i.c.l.

	$\implies$. Supongamos la misma situación anterior donde $i$ es un encaje regular y $p$ es suave.
	Sea $i$ de manera que $\mathscr{O}_{X, x}$ está generado por una sucesión débilmente regular $b_1, \dots, b_m \in \mathscr{O}_{Z, m}$.
	En particular, $\kdim\mathscr{O}_{X, x} = \kdim\mathscr{O}_{Z, x} - m$ y, además, por el corolario~\ref{thm:flat_fibers_dimension}
	\begin{align*}
		\kdim\mathscr{O}_{X_y, x} &= \kdim\mathscr{O}_{X, x} - \kdim\mathscr{O}_{Y, y}, \\
		\kdim\mathscr{O}_{Z_y, x} &= \kdim\mathscr{O}_{Z, x} - \kdim\mathscr{O}_{Y, y};
	\end{align*}
	por lo que $\kdim\mathscr{O}_{X_y, x} = \kdim\mathscr{O}_{Z_y, x} - m$.
	Luego al localizar (que es tensorizar por un módulo plano) las imágenes de los $b_i$ siguen forman una sucesión débilmente regular y,
	por tanto, $X_y \to Z_y$ es un en encaje regular.
\end{proof}

\section*{Notas históricas}
Los morfismos étale fueron una invención de Grothendieck.
Varios autores señalan que la palabra \textit{étale} es poco frecuente en francés, salvo en poesía y usualmente referido al mar donde tendría el
equivalente español de <<calmado, tranquilo o profundo>>.
Señala \citeauthor{mumford:red}~\cite{mumford:red}:
<<La palabra aparentemente se refiere a la apariencia del mar en marea alta bajo una luna llena en determinados tipos de clima.>>

\addtocategory{scheme}{mumford:red}
