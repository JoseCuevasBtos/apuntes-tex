\documentclass[11pt, reqno]{amsart}

\usepackage[article, es]{../cuevasthm}
\input{../amsart-template.tex}

\makeatletter
\let\@wraptoccontribs\wraptoccontribs
\makeatother

\title{Formas lineales en logaritmos}
\author{Francisco Gallardo}
\email{francisco.gallardo@uc.cl}
\address{Departamento de Matemáticas, Pontificia Universidad Católica de Chile.
Facultad de Matemáticas, 4860 Av. Vicuña Mackenna, Macul, RM, Chile}

% \contrib[con apuntes y un apéndice de]{José Cuevas Barrientos}
% \email{josecuevasbtos@uc.cl}
% \notes{José Cuevas Barrientos}

\date{17 de mayo de 2024}

\begin{document}

\maketitle

\section{Introducción histórica}
La primera definición de número trascendente, como conocemos hoy, fue dada por Euler en el siglo \textsc{xviii}.
En 1768, Lambert conjetura que $e$ y $\pi$ son trascendentes en el mismo artículo donde prueba que $\pi$ es irracional.
Sin embargo, no fue hasta 1844 que Liouville logró demostrar la existencia de números trascendentes.
En 1873, Hermite demuestra que $e$ es trascendente y en 1874 Cantor demuestra que estos abundan usando su teoría de cardinales infinitos.
En 1882, Lindemann demuestra que $\pi$ es trascendente y, más en general:
\begin{thm}[Lindemann, 1882]
	Si $\alpha\neq 0$ es algebraico, entonces $e^{\alpha}$ es trascendente.
\end{thm}
Así, si $\pi$ fuese algebraico, entonces $2\pi\ui$ también, y luego $e^{2\pi\ui}=1$ sería trascendente por el teorema anterior, lo que es absurdo.

En 1885, Weierstrass generaliza el teorema de Lindemann:
\begin{thm}[Lindemann-Weierstrass, 1885]
	\label{LW}
	Sean $\alpha_1,\ldots,\alpha_n\in \overline{\Q}$ distintos.
	Luego $e^{\alpha_1},\ldots,e^{\alpha_n}$ son linealmente independientes sobre $\overline{\Q}$.
\end{thm}

Y en el año 1900 Hilbert postula, dentro de su lista de 23 problemas, uno que tendría relación con trascendencia:
% \begin{prob}[Séptimo problema de Hilbert]
\begin{displayquote}
	\textbf{Séptimo problema de Hilbert:}
	Si $\alpha\neq 0,1$ es algebraico y $b\notin\Q$, ¿es $a^b := \exp(b\log a)$ trascendente?
\end{displayquote}
% \end{prob}
Nótese que $a^b$ es multivaluado, puesto que el logaritmo complejo lo es, así que la pregunta hace alusión a cualquiera de los posibles valores.

En 1934, Gelfond y Schneider independientemente dan una respuesta afirmativa al
séptimo problema de Hilbert y, por ende, el resultado adquiere el nombre de
<<teorema de Gelfond-Schneider>>.

En 1966, Alan Baker generalizó el teorema de Gelfond-Schneider usando
\textit{formas lineales en logaritmos}, de las cuales tratará esta exposición.
Cabe destacar que, después de las ideas de Baker, la llamada \textit{teoría de
la trascendencia} ha tenido sorpresivas repercusiones en las ecuaciones
diofántincas, la geometría diofántica y la aproximación diofántica; por lo que,
resulta una herramienta esencial para teoristas de números.

\section{Formas lineales en logaritmos}
Empezamos con nuestro estudio de logaritmos de números algebraicos. 

\begin{prop}
	Sea $\alpha\neq 0,1$ algebraico. Luego $\log\alpha$ es trascendente. 
\end{prop}
\begin{proof}
	Como $\alpha\neq 1$, entonces $\log\alpha\neq 0$.
	Si $\beta=\log\alpha$ fuera algebraico, entonces por el teorema de Lindemann se tendría que $\alpha=e^{\beta}$ sería trascendente, lo que es absurdo.
\end{proof}

La pregunta a continuación es si existe un Lindemann-Weierstrass logarítmico, vale decir:
\begin{mydef}
	Diremos que se satisface la propiedad $L(n)$ para $n \ge 1$ entero si:
	Para todo conjunto $\alpha_1,\ldots,\alpha_n$ de números algebraicos no nulos
	tales que $\log\alpha_1,\ldots,\log\alpha_n$ son $\Q$-linealmente independientes,
	se satisface que $\log\alpha_1,\ldots,\log\alpha_n$ son $\algcl\Q$-linealmente independientes. 
\end{mydef}

\begin{obs}
	Los $\alpha$'s del problema son distintos de 1 y distintos entre si, pues de lo contrario el conjunto de logaritmos no sería $\Q$-linealmente independiente.
	Una combinación $\algcl\Q$-lineal de tales logaritmos se denomina una \strong{forma lineal en logaritmos}.
\end{obs}

El caso $n=1$ es por supuesto trivial. Sin embargo, el caso $n=2$ ya es bastante complicado:
\begin{prop}
	Son equivalentes:
	\begin{enumerate}
		\item Se satisface el teorema de Gelfond-Schneider.
		\item La afirmación $L(2)$ es válida.
	\end{enumerate}
\end{prop}
\begin{proof}
	$1 \implies 2$. Supongamos que
	\[
		\beta_1\log\alpha_1 + \beta_2\log\alpha_2 = 0.
	\]
	Si $\beta_1=0$, entonces $\beta_2$ también pues $\log\alpha_2\neq 0$.
	Luego $\log\alpha_2=-\frac{\beta_1}{\beta_2}\log\alpha_1$ y entonces $\alpha_2=\alpha_1^{\gamma}$ con $\gamma=-\beta_1/\beta_2$.
	Como $\alpha_2$ es algebraico, el teorema de Gelfond-Schneider implica que $\alpha_1 \in \{ 0, 1 \}$ o bien que $\gamma\in \Q$.
	% Como la primera opción no ocurre, debe ocurrir la segunda.
	Así pues, necesariamente $\gamma \in \Q$.
	En dicho caso dividiendo la relación de dependencia original por $\beta_2$ da un relación de dependencia sobre $\Q$ entre $\log\alpha_1$ y
	$\log\alpha_2$, lo que es absurdo.

	$2 \implies 1$. Sea $\alpha \notin \{ 0, 1 \}$ algebraico y sea $\beta\notin \Q$.
	Supongamos, por contradicción, que $\gamma:=\alpha^{\beta}$ es algebraico.
	Luego $1\cdot \log\gamma+ (-\beta)\log\alpha=0$ y $\alpha,\gamma$ son algebraicos distintos de 0 y 1.
	Por $L(2)$ se tiene que $\beta=0\in \Q$, lo que es absurdo.
\end{proof}

Sorprendentemente, $L(n)$ es cierto para todo $n$. Esto fue una consecuencia del teorema de Baker:
\begin{thm}[Baker, 1966]
	Sean $\alpha_1,\ldots,\alpha_n\neq 0$ algebraicos tales que $\log\alpha_1,\ldots,\log\alpha_n$ son $\Q$-linealmente independientes. Luego
	\[
		1, \; \log\alpha_1, \; \ldots, \; \log\alpha_n
	\]
	son $\algcl\Q$-linealmente independientes.
\end{thm}

En otras palabras, dados $\alpha_1,\ldots,\alpha_n\neq 0,1$ algebraicos el teorema nos da condiciones para que la expresión
\[
	\Lambda=\beta_0+\beta_1\log\alpha_1+\cdots+\beta_n\log\alpha_n,\quad\beta_i\in \overline{\Q},\quad 1\leq i\leq n.
\]
sea no nula. Por ejemplo, si los logaritmos de los $\alpha_i$'s son $\Q$-linealmente independientes, entonces $\Lambda\neq 0$ por el teorema.
% Un corolario inmediato es el siguiente:
Como corolario tenemos lo siguiente:
\begin{cor}
	Si $\beta_0\neq 0$, entonces $\Lambda\neq 0$. 
	Si $\beta_0=0$ y $\beta_1,\ldots,\beta_n$ son $\Q$-linealmente independientes.
\end{cor}

Es importante que no se pida la independencia de los logaritmos.

Si bien conocer que $\Lambda\neq 0$ es ya de interés, para aplicaciones es necesario dar una cota inferior de $|\Lambda|$ en caso de que sea no nulo.
En esta línea tenemos el siguiente resultado de Baker, 9 años después:
\begin{thm}[Baker, 1975]
	Sean $\alpha_1,\ldots,\alpha_n\neq 0,1$ algebraicos.
	Existe una constante (computable) $C>0$ que depende solo de los $\alpha_j$ que satisface lo siguiente: 

	Si $\beta_1,\ldots,\beta_n\in \Z$ cumplen que $\Lambda\neq 0$, entonces $$|\Lambda|\geq (1+B)^{-C}$$ donde $B=\max_{j}|\beta_j|$.
	Es decir, $-\log|\Lambda|\leq C\log(1+B)$.
\end{thm}

Esto ha sido mejorado y generalizado en varias direcciones. Al día de hoy el mejor resultado en $\C$ es de Matveev. Para enunciarlo definimos la altura de un número racional $p/q$ con $\gcd(p,q)=1$ como $$H(p/q)=\log\max\{|p|,|q|\}.$$ Además, para $t\geq 0$ definimos $\log^*t:=\max\{1,\log{t}\}$.

\begin{thm}[Matveev, 2000]
	Sean $\alpha_1,\ldots,\alpha_n\in \Q$ no nulos y sean $b_1,\ldots,b_n\in \Z$ tales que $\Lambda\neq 0$.
	Entonces
	\[
		-\log|\Lambda|\leq \kappa^n(\log^*B) \prod_{j=1}^n\log^*H(\alpha_j),
	\]
	donde $B=\max_j|b_j|$ y $\kappa \le 100$ es una constante computable.
\end{thm}

En particular, nos da una versión explícita del teorema de Baker, con
\[
	C=100^n\prod_{j=1}^n\log^*H(\alpha_j).
\]
Para aplicaciones esta fórmula para $C$ es crucial.

Además, uno puede traducir los resultados de Baker y Matveev a una versión multiplicativa usando la siguiente observación de cálculo:

\begin{lem}
	Sea $L\in \R$ con $|L|\leq 1$. Entonces $$|1-e^L|\geq \frac{|L|}{2}.$$
\end{lem}
\begin{proof}
	Sea $-1\leq L\leq 1$ un número real. La expansión de Taylor-MacLaurin de $f(L)=\frac{e^L-1}{L}$ es
	\[
		1+\frac{L}{2}+\frac{L^2}{6}+\frac{L^3}{24}+\cdots
	\]
	Si $L\geq 0$, es claro que $|f(L)|\geq 1\geq 1/2$. Si $L$ es negativo luego es claro que $|f(L)|\geq \frac{1}{2}$.
\end{proof}
En particular, los <<espacios>> (eng.\ \textit{gaps}) se vuelven arbitrariamente grandes.

\begin{thm}[Baker, versión multiplicativa]
	Sean $\alpha_1,\ldots,\alpha_n\neq 0,1$ algebraicos. Existe una constante computable $C>0$ tal que, si $\beta_1,\ldots,\beta_n\in \Z$ son tales que $\Lambda\neq 0$, entonces $$-\log|1-\alpha^{\beta_1}\cdots\alpha_n^{\beta_n}|\leq C\log^*B.$$
\end{thm}
\begin{proof}
	Por el lema anterior,
	\begin{align*}
		-\log\lvert 1 - \alpha^{\beta_1}\cdots\alpha_n^{\beta_n} \rvert &= -\log\lvert 1 - e^{\Lambda} \rvert \\
										&\leq -\log \frac{|L|}{2}\leq C\log\mathopen{}\left( 1+\frac{B}{2} \right)\mathclose{}
										\leq C\log^*B.
										\tqedhere
	\end{align*}
\end{proof}

Similarmente obtenemos:
\begin{thm}[Matveev, versión multiplicativa]
	Sean $\alpha_1,\ldots,\alpha_n\in \Q$ no nulos y sean $b_1,\ldots,b_n\in \Z$ tales que $\alpha_1^{\beta_1}\cdots\alpha_n^{\beta_n}=:e^\Lambda\neq 1$.
	Entonces
	\[
		-\log|1-\alpha_1^{\beta_1}\cdots\alpha_n^{\beta_n}|\leq \kappa^n(\log^*B) \prod_{j=1}^n\log^*H(\alpha_j),
	\]
	donde $B=\max_j|b_j|$ y $\kappa \le 100$ es una constante computable.  
\end{thm}

Cuando las formas cuentan con solo dos términos hay cotas mucho mejores.
% En 1995, Laurent, Mignotte y Festerenko probaron lo siguiente:
\begin{thm}[Laurent-Mignotte-Festerenko, 1995]
	\label{LMF}
	Sean $a_1,a_2\neq 1$ racionales y $b_1,b_2$ enteros no nulos tales que $\Lambda=b_1\log a_2-b_2\log a_1\neq 0$.
	Luego
	\begin{multline*}
		\log\lvert\Lambda\rvert \geq \\
		-22\left( \max\left\{ 21, \log\mathopen{}\left( \frac{|b_1|}{\log H(a_2)}+\frac{|b_2|}{\log H(a_1)} \right)\mathclose{} + 0.06\right\} \right)^2
		\log H(a_1)\log H(a_2).
	\end{multline*}
\end{thm}

Antes de continuar a las aplicaciones, cabe mencionar que estos resultados
tienen análogos en el mundo $p$-ádico y estos son también importantes en muchas
aplicaciones. Por ejemplo:
% Por ejemplo, tenemos un análogo al teorema de Baker versión multiplicativa que
% fue probado por Yu en 1986 usando análisis $p$-ádico:

\begin{thm}[Yu, 1986]
	\label{Yu86}
	Sea $p$ un número primo y $a_1,\ldots,a_m$ racionales no-nulos y no divisibles por $p$. Sean además $b_1,\ldots,b_m$ enteros tales que $a_1^{b_1}\cdots a_m^{b_m}\neq 1$ y sea $B=\max_{1\leq i\leq m}|b_i|$.
	Luego
	\[
		|a_1^{b_1}\cdots a_m^{b_m}-1|_p\geq (eB)^{-C}
	\]
	para una constante computable $C$ que depende de $p$ y $a_1,\ldots,a_m$. 
\end{thm}

\section{Aplicaciones de formas lineales en logaritmos}
\begin{prop}
	Sean $a,b\geq 2$.
	Luego existe una constante computable $C$, dependiendo de $a$ y $b$, tal que para todo $m,n\in \Z_{>0}$ se cumple que
	\[
		|a^m-b^n|\geq \frac{\max\{a^m,b^n\}}{\max\{m,n\}^C}.
	\]
	En particular, podemos tomar $C=10000\log^* a\log^* b$.
\end{prop}
\begin{proof}
	Sea $k=a^m-b^n$. Por el teorema de Matveev $$-\log|k/a^m|=-\log|1-a^{-m}b^n|\leq \log^*\max\{m,n\}.$$
	Como $\max\{m,n\}\geq 3>e$, reescribiendo el término de la izquierda obtenemos $$m\log a-\log|k|\leq C\log\max\{m,n\}$$ y luego $$|k|\geq \frac{a^m}{\max\{m,n\}^C}.$$ El mismo análisis se puede hacer si dividimos por $b^n$ en vez de $a^m$ al principio y el teorema sigue.
\end{proof}

Como consecuencia, si $a^m-b^n=1$ con $a,b$ fijos, entonces $m$ y $n$ están acotados por una constante computable $C'$.
En 1844 Catalan conjeturó que la ecuación $a^m-b^n=1$ con $a,b>0$ y $m,n>1$ tenía solo por solución: $3^2 - 2^3 = 1$.
La proposición que acabamos de probar muestra que la conjetura de Catalan admite finitos contraejemplos, fijando $a$ y $b$.
Sin embargo, la constante $C$ es ridículamente grande en nuestra proposición y no sirve para probar la conjetura.

En su lugar, una proposición un tanto mejor fue dada por Tijdeman:
\begin{thm}[Tijdeman, 1976]
	Sea $K$ un cuerpo numérico.
	Existen una constante computable $C$ tal que las soluciones de la ecuación diofántica exponencial
	$$ x^m - y^n = 1, \qquad m, n \in \N_{>1}; \; x, y \in \Z_{>0} $$
	satisfacen que $\max\{ m, n, |x|, |y| \} \le C$.
	En consecuencia, posee a lo más finitas soluciones.
\end{thm}
% \todo{Incluír teorema de Tijdeman.}
\begin{proof}
	Cfr.\ \citeauthor{bilu:catalan}~\cite[176\psqq]{bilu:catalan}, Thm.~13.19.
\end{proof}
Éste todavía no nos permite responder la conjetura de Catalan, puesto que las cotas siguen siendo demasiado grandes para computadores;
pero el teorema de Tijdeman sigue siendo útil puesto que tiene validez en contexto de $S$-enteros (vid.\ \citeauthor{brindza87catalan}~\cite{brindza87catalan}).

Respecto a la conjetura original de Catalan, esta fue resuelta por Preda Mih\u ailescu.
Una demostración completa y autocontenida está expuesta en el libro de \citeauthor{bilu:catalan}~\cite{bilu:catalan}.

Para la siguiente aplicación nos planteamos primero la siguiente situación. Ciertamente, el TFA asegura que todo número se puede escribir como producto de primos. Sin embargo, si restringimos los primos a un conjunto finito, podríamos preguntarnos qué números pueden escribirse como producto de ellos. Más precisamente, para todo conjunto finito de primos $S$ tenemos el conjunto de $$\N_S:=\left\{\prod_{p\in S}p^{\alpha_p}\colon \alpha_p\geq 0\right\}.$$

Como $\N_S$ es numerable, enumeraremos por $s_1,s_2,s_3,\ldots$ sus elementos y preguntarnos sobre los saltos entre estos (i.e, $g_n=s_{n+1}-s_n$).

\begin{prop}
	Sea $S=\{p_1,\ldots,p_t\}$ un conjunto finito de primos y sean $s_n$ y $g_n$ como en la discusión anterior. Luego existen constantes computables $c_1,c_2$ que dependen de $S$ y tales que $$g_n\geq \frac{s_n}{c_1(\log s_n)^{c_2}}$$ para todo $n\geq 1$. 
\end{prop}
\begin{proof}
	Sea $s_n=p_1^{a_1}\cdots p_t^{a_t}$ y $s_{n+1}=p_1^{b_1}\cdots p_t^{b_t}$. Luego por Baker multiplicativo obtenemos $$-\log\frac{g_n}{s_n}=-\log\left|\frac{g_n}{s_n}\right|=-\log\left|1-\frac{s_{n+1}}{s_n}\right|=\-\log|1-p_1^{b_1-a_1}\cdots p_t^{b_t-a_t}|\leq C\log^*B$$ donde $B=\max_{1\leq i\leq t}\{|b_i-a_i|\}$.

	Primero notemos que, como $p_i^{a_i}\leq s_n$, entonces $$a_i\leq \frac{\log{s_n}}{\log{p_i}}\leq\frac{\log s_n}{\log 2}.$$
	Además, como $a_{n+1}\leq a_n^2$, tenemos $$b_i\leq \frac{\log s_{n+1}}{\log p_i}\leq\frac{2\log s_{n}}{\log 2}.$$ 

	Esto muestra que $B\leq \frac{2\log s_n}{\log 2}$.

	Juntando esto con la desigualdad dada del teorema de Baker obtenemos $$|g_n|\geq \frac{s_n}{c_1(\log s_n)^{c_2}}.$$
\end{proof}

Por último daremos una aplicación a las estimaciones del teorema~\ref{LMF}.
\begin{prop}
	La ecuación $$x^n-2y^n=1,\quad x,y\in \Z_{\geq 2}$$ no tiene soluciones para $n>6726$.
\end{prop}
\begin{proof}
	Sean $x,y,n$ enteros con $x,y\geq 2$, $n\geq 3$ y $x^n-2y^n=1$. Luego $|1-2(y/x)^n|=1/x^n\leq 1/2$.
	Usando la cota
	\[
		\lvert\log(1+z)\rvert \leq 2|z|, \quad z\in \R,\; |z|\leq \frac{1}{2}
	\]
	con $z=2(y/x)^n-1$ obtenemos
	\[
		\lvert\log 2+n\log(y/x)\rvert \leq 2/x^n=2e^{-n\log x}.
	\]
	Por otro lado el teorema~\ref{LMF} usando $\Lambda=\log 2+n\log(y/x)$ implica
	\begin{align*}
		\log |\Lambda|&\geq -22\left( \max\mathopen{}\left\{ 21, \log\mathopen{}\left( \frac{1}{\log x}+\frac{n}{\log 2} \right)\mathclose{}+0.06\right\}\mathclose{} \right)^2
		\log 2\log x\\
			      &\geq -22\left( \max\mathopen{}\left\{ 21, \log\mathopen{}\left( \frac{n+1}{\log 2} \right)\mathclose{}+0.06 \right\}\mathclose{} \right)^2\log 2\log x\\
			      &\geq -22( \max\{ 21, \log(n+1)+0.43 \} )^2\log 2\log x.
	\end{align*}
	Combinando esta cota inferior con la superior obtenida al principio sigue que
	\[
		n\log x-\log 2\leq 22( \max\{ 21, \log(n+1)+0.43 \} )^2\log 2\log x.
	\]
	Dividiendo por $\log x$ y usando $\log x\geq \log 2$ obtenemos
	\[
		n\leq 22( \max\{ 21, \log(n+1)+0.43 \} )^2\log 2+1.
	\]

	Si el máximo en cuestión es 21, obtenemos $n\leq 22\cdot(21)^2\log 2 +1<6726$.
	Si el máximo es $\log(n+1)+0.43$, entonces obtenemos
	\[
		n\leq \log 2(\log(n+1))^2+0.86\log(n+1)\log 2+0.1849\log 2+1.
	\]
	Con un cálculo sencillo, es fácil ver de que esta desigualdad es imposible para $n>760$. 
\end{proof}

\section{Relación con $abc$}

Como habíamos mencionado antes, la teoría de formas lineales en logaritmos se generaliza a los números $p$-ádicos y en este contexto se obtienen teoremas similares en cuanto a la no anulación efectiva de este tipo de formas. Usando esta teoría, se obtienen los siguientes resultados relacionados al problema $abc$.

\begin{thm}[Stewart-Tijdeman, 1986]
	Existe una constante computable $K$ tal que, para todos los $a,b,c\in \Z$ con $a+b=c$ y $\gcd(a,b,c)=1$ se tiene $$\log c< K\cdot \rad(abc)^{15}.$$
\end{thm}

Con resultados más innovadores de Yu en cuanto a formas lineales en logaritmos $p$-ádicos, se mejoró a lo siguiente.

\begin{thm}[Stewart-Yu,1991]
	Para todo $\varepsilon>0$, existe una constante computable $K_{\varepsilon}$ tal que, para todos los $a,b,c\in \Z$ con $a+b=c$ y $\gcd(a,b,c)=1$ se tiene $$\log c< K_{\varepsilon}\cdot \rad(abc)^{\frac{2}{3}+\varepsilon}.$$   
\end{thm}

Finalmente en 2001, los mismo autores lograron mejorar su resultado:

\begin{thm}[Stewart-Yu, 2001]
	Para todo $\varepsilon>0$, existe una constante computable $K_{\varepsilon}$ tal que, para todos los $a,b,c\in \Z$ con $a+b=c$ y $\gcd(a,b,c)=1$ se tiene
	\[
		\log c< K_{\varepsilon}\cdot \rad(abc)^{\frac{1}{3}+\varepsilon}.
	\]
\end{thm}

Terminaremos esta sección dando un resultado demostrado por Pastén en 2022, donde el exponente de los teoremas anterior baja a $\varepsilon$ cuando asumimos que el $a<c^{1-\eta}$ para algún $\eta$ positivo. En el curso de la demostración usaremos el siguiente lema que incluimos sin demostración:

\begin{lem}
	Sea $\varepsilon>0$. Luego para $n$ suficientemente grande se tiene que $$\omega(n)<(1+\varepsilon)\frac{2\log n}{\log\log n}$$ donde $\omega(n)=\#\{p\text{ primo} : p\mid n\}$.
\end{lem}

\begin{thm}[Pastén, 2022]
	Sean $\eta,\varepsilon>0$. Existe una constante $K_{\eta,\epsilon}>0$ que cumple lo siguiente:
	Dados enteros positivos coprimos $a,b,c$, con $a+b=c$ y $a<c^{1-\eta}$, se tiene $$\log c<K_{\eta,\varepsilon}\cdot \rad(abc)^{\varepsilon}.$$
\end{thm}
\begin{proof}
	Sea $S$ el conjunto de primos que divide a $bc$. Sea $R=\rad{bc}=\prod_{q\in S}q$ y sea $n=\omega(R)=\#S$. 

	Luego, $$\frac{a}{c}=1-\frac{b}{c}=1-\prod_{q\in S}q^{e_q}$$ donde $e_q\in \Z$ para $q\in S$. Sea $B=\max_{q\in S}|e_q|$. 

	Por el teorema de Matveev (v. multiplicativa) y el hecho de que $a<c^{1-\eta}$, obtenemos que $$\eta\log c=-\log\frac{a}{c}=-\log\left|1-\prod_{q\in S}q^{e_q}\right|\leq \kappa^n\log B\prod_{q\in S}\log q.$$

	Afirmamos que $B\leq \log H(b/c)=\log c$. Ciertamente, si escribimos
	\[
		\frac{b}{c}=\frac{\displaystyle\prod_{\substack{q\in S \\ e_q>0}}q^{e_q}}{\displaystyle\prod_{\substack{q\in S \\ e_q<0}}q^{-e_q}},
	\]
	entonces $\log H(b/c)=\sum_{\substack{q\in S\\e_q<0}}|e_q|\log q\geq |e_q|$ para todo $q\in S$ con $e_q<0$. Si $q\in S$ con $e_q>0$, entonces
	\[
		|e_q|=e_q\leq \sum_{\substack{q\in S\\ e_q>0}}e_q\log q=\log b\leq \log c=\log H(b/c).
	\]
	Esto muestra lo afirmado. Usando esto y la desigualdad proveniente de Matveev obtenemos $$\frac{\eta\log c}{\log\log c}\leq \kappa^n\cdot\prod_{q\in S}\log q.$$ Usando la desigualdad ma-mg obtenemos
	\[
		\prod_{q\in S}\log q\leq\left( \frac{1}{n}\sum_{q\in S}\log q \right)^n=\left( \frac{1}{n}\log R \right)^n
	\]
	y entonces se sigue que $$\frac{\eta\log c}{\log\log c}\leq \left( \frac{\kappa}{n}\log R \right)^n.$$

	Cuando $c\gg_{\eta} 1$ (i.e, $c$ es suficientemente grande dependiendo de $\eta$), entonces $$\sqrt{\log c}<\frac{\eta\log c}{\log\log c}$$ y obtenemos $$\sqrt{\log c}<\left( \frac{\kappa}{n}\log R \right)^n.$$

	A continuación obtendremos una desigualdad para el lado derecho. Consideremos la función $f\colon (0,\infty)\to \R$ dada por
	\[
		f(t)=\log\mathopen{}\left( \left( \frac{A}{t} \right)^t \right)\mathclose{} = t\log A-t\log t
	\]
	para $A$ fijo. Luego $f'(t)=\log A-\log t-1$ y entonces $f'(t)\geq 0$ para $1\leq t\leq A/e$. Esto implica que $f$ es creciente en ese tramo.

	Notemos que $f(n)=\left( \frac{\kappa}{n}\log R \right)^n$ con $A=\kappa\log R$. 

	Además, $n=\omega(R)<2\log R/\log\log R$ para $R\gg 1$, y para $R\gg 1$ (aún más grande posiblemente) $n<\kappa\log R/e$.
	Como en dicho tramo la función es creciente, entonces
	\begin{align*}
		f(n)=\left( \frac{\kappa}{n}\log R \right)^n&\leq \left( \frac{\kappa\log\log R}{2} \right)^{\frac{2\log R}{\log\log R}}\\
							    &=\left( \frac{\kappa R}{2} \right)^{\frac{2\log\log\log R}{\log\log R}}
							    \ll_{\varepsilon} R^{\varepsilon},
	\end{align*}
	donde la última desigualdad viene de que el exponente tiende a 0 cuando $R$ crece y la igualdad viene de que $$(\log\log R)^{\log R}=(\log R)^{\log\log\log R}$$ (si aplica $\log$ a ambos números será evidente la igualdad).

	Juntando nuestras desigualdades obtenemos $$\sqrt{\log c}\ll_{\varepsilon} R^{\epsilon}$$ para $c\gg_{\eta} 1$ y $R\gg 1$. Elevando al cuadrado obtenemos lo pedido.
\end{proof}

\section{Ecuaciones de unidades: Introducción y caso 1}

Sea $K$ un cuerpo de característica 0 y $\Gamma\subseteq K^{\times}$ un subgrupo finitamente generado. 

Una ecuación de unidades es una ecuación de la forma $$a_1x_1+\cdots+a_nx_n=1$$ donde los $a_i$ están en $K^{\times}$ y se busca resolver para $x_i\in \Gamma$. 

Lo interesante de estas ecuaciones es que muchas ecuaciones diofánticas se pueden llevar a una de este tipo. Típicamente $K$ es un cuerpo numérico y $\Gamma= \mathcal{O}_K^{\times}$.

Nos concentraremos en ecuaciones de unidad para $n=2$, i.e, la ecuación
\begin{equation}
	ax+by=1,\quad a,b\in K^{\times},\; x,y\in \Gamma.
	\label{eu2}
\end{equation} 

En este contexto, Siegel probó en 1921 la finitud de soluciones para el caso en que $K$ es un cuerpo numérico y $\Gamma=\mathcal{O}_K^{\times}$. Mahler en 1933 probó finitud en el caso $K=\Q$ y $\Gamma$ el conjunto de $S$-unidades (i.e, $S$ es un conjunto finito de primos en $\Z$ y una $S$-unidad es un racional cuyo numerador y denominador solo contiene potencias de los primos en $S$). Para $S$-unidades en cuerpo numérico la finitud fue probada por Parry en 1950. El caso general con $K$ cuerpo de característica 0 y $\Gamma\leq K^{\times}$ finitamente generado fue probado por Lang en 1960. De esta forma, tenemos el siguiente teorema:

\begin{thm}[Siegel-Mahler-Parry-Lang, 1960]
	\label{thm:siegel}
	La ecuación \eqref{eu2} tiene finitas soluciones.
\end{thm}

En 1979, Győry encontró una cota para las alturas de la soluciones de \eqref{eu2} usando los resultados efectivos de Baker sobre formas lineales en logaritmos.

En esta sección probaremos el teorema~\ref{thm:siegel} en el caso $K=\Q$ y en el caso que $K$ es cuerpo numérico y $\Gamma=\mathcal{O}_K^{\times}$.

Lidiemos con el primer caso. Lo primero que veremos es que nos podemos reducir al caso de $S$-unidades. Ciertamente, si $\gamma_1,\ldots,\gamma_r$ son generadores de $\Gamma$, entonces si $S=\{p_1,\ldots,p_t\}$ denota el conjunto de primos que ocurren en las factorizaciones de los numeradores y denominadores de $a,b,\gamma_1,\ldots,\gamma_r$, entonces $a,b,\gamma_1,\ldots,\gamma_r$ son $S$-unidades, i.e, viven dentro de $$\Z_S^{\times}=\{\pm p_1^{e_1}\cdots p_t^{e_t}\mid e_t\in \Z\}.$$

Si $(x,y)$ es solución de \eqref{eu2} (en el caso que estamos cubriendo), entonces $ax$ y $by$ son $S$-unidades también, de modo que nos reducimos a probar la finitud de las soluciones de
\begin{equation}
	\label{eus}
	x+y=1,\quad x,y\in \Z_S^{\times}.
\end{equation}

\begin{thm}
	La ecuación \eqref{eu2} tiene finitas soluciones y el conjunto de soluciones se puede determinar de manera efectiva cuando $K=\Q$.
\end{thm}
\begin{proof}
	Como ya habíamos dicho, basta reducirse al caso de $S$-unidades y considerar la ecuación \eqref{eus}.

	Sean entonces $x,y\in \Z_S^{\times}$ tales que $x+y=1$. Primero escribimos $x=u/w$ $y=v/w$ con $u,v,w$ enteros y $\gcd(u,v,w)=1$. Así, $u+v=w$.

	Luego los enteros $u,v$ y $w$ están compuestos por primos en $S$ y además $\gcd(u,v,w)=1$ y $u+v=w$ implica que $u,v,w$ son coprimos dos a dos.

	Reordenando posiblemente nuestra lista de primos podemos y nuestro enteros $u,v,w$ podemos asumir que $$u=\pm p_1^{b_1}\cdots p_r^{b_r},\quad v=\pm p_{r+1}^{b_{r+1}}\cdots p_s^{b_s},\quad w=\pm p_{s+1}^{b_{s+1}}\cdots p_t^{b_t},$$ con $0\leq r\leq s\leq t$ y $b_i$ enteros no negativos.

	Estamos listos si logramos probar que $B=\max\{b_1,\ldots,b_t\}$ está acotado superiormente por una constante computable.
	Por simetría podemos asumir $B=p_t$. Como
	\[
		-\left( \frac{u}{v} \right)-1=-\left( \frac{w}{v} \right),
	\]
	tenemos que
	\[
		p_t^{-B}=p_t^{-b_t}=|w/v|_{p_t}=\lvert\pm p_1^{b_1}\cdots p_r^{b_r}p_{r+1}^{-b_{r+1}}\cdots p_s^{-b_s}-1\rvert_{p_t}>0.
	\]

	Del teorema~\ref{Yu86} obtenemos que $$|\pm p_1^{b_1}\cdots p_r^{b_r}p_{r+1}^{-b_{r+1}}\cdots p_s^{-b_s}-1|_{p_t}>(eB)^{-C}$$ donde $C$ es computable en términos de los $p_i$.

	Pero luego $$p_t^{-B}>(eB)^{-C}$$ y entonces $$\frac{B}{1+\log B}\leq \frac{C}{\log p_t}.$$
	Esto implica que $B$ está acotado (el lado de la izquierda tiende a infinito cuando $B$ crece).
\end{proof}

\section{Ecuaciones de unidades: Caso 2}
Ahora veremos el caso 2, i.e, consideramos la ecuación \eqref{eu2} con $K$ un cuerpo numérico y $\Gamma=\mathcal{O}_K^{\times}$.

% Para esto recordamos algunos conceptos relacionados al conjunto $\mathcal{O}_K^{\times}$. Sea $d=[K\colon \Q]$. Como $K$ es separable sobre $\Q$, significa que existen $d$ incrustaciones de $K$ en $\C$. Estas vienen en dos sabores, reales y complejos, y los últimos vienen de a pares. Digamos que son $r_1$ incrustaciones reales y $r_2$ pares de incrustaciones complejas, de modo que $r_1+2r_2=d$. 
% Denotamos $\sigma_1,\ldots,\sigma_{r_1}$ las incrustaciones reales y $\sigma_{r_1+i}=\overline{\sigma_{r_1+r_2+i}}$ son los pares de incrustaciones complejas. 

% Luego tenemos el siguiente resultado que describe la estructura de $\mathcal{O}_K^{\times}$.
Para ello, recuérdese el siguiente resultado clásico:
\begin{thm}[de las unidades de Dirichlet]
	% Sea $r=r_1+r_2-1$ y $L\colon \mathcal{O}_K^{\times}\to \R^r$ definido por $\varepsilon\mapsto (\log|\sigma_1(\varepsilon)|,\ldots,\log|\sigma_{r}(\varepsilon)|)$.
	Sea $K$ un cuerpo numérico con $r_1$ lugares reales y $r_2$ lugares imaginarios, donde $r := r_1 + r_2 - 1$.
	La función
	\[
		\operatorname{Log} \colon \mathcal{O}_K^\times \longrightarrow \R^r, \qquad
		\alpha \longmapsto (\log|\sigma_1(\alpha)|, \dots, \log|\sigma_r(\alpha)|)
	\]
	es un homomorfismo de grupos, su núcleo $\ker({\rm Log})$ es el grupo $\mu$ de raíces de la unidad en $K$ 
	e $\Img{\rm Log}$ es un reticulado de rango $r$. En particular, $\mathcal{O}_K^{\times}\cong \Z^r\times\mu$.
\end{thm}
\begin{proof}
	Cfr.\ \citeauthor{neukirch:algebraic}~\cite[42, 358]{neukirch:algebraic}, Thm.~I.7.4 y Prop.\ VI.1.1.
\end{proof}

Esto implica que existen $\varepsilon_1,\ldots,\varepsilon_r\in \mathcal{O}_K^{\times}$ tales que para todo $\varepsilon\in \mathcal{O}_K^{\times}$, $$\varepsilon=\zeta\varepsilon_1^{b_1}\cdots\varepsilon_r^{b_r},\quad \zeta\in U_K,\; b_i\in \Z.$$
Más aún, la matriz
\[
	M = \begin{pmatrix}
		\log|\sigma_1(\varepsilon_1)|  & \cdots & \log|\sigma_1(\varepsilon_r)|\\
		\vdots & \ddots & \vdots \\
		\log|\sigma_r(\varepsilon_1)| & \ldots & \log|\sigma_r(\varepsilon_r)|
	\end{pmatrix} 
\]
es invertible.

Por último, necesitaremos el siguiente lema.

\begin{lem}
	Existe una constante $C>0$ tal que, si $\varepsilon\in\mathcal{O}_K^{\times}$ con $\varepsilon=\zeta\prod_{i=1}^{r}\varepsilon_i^{b_i}$, entonces $$\max_{1\leq i\leq r}|b_i|\leq C\cdot \max_{1\leq i\leq d}\log|\sigma_i(\varepsilon)|.$$
\end{lem}
\begin{proof}
	Sea $v=(b_1\cdots b_r)^t$ el vector columna asociado. Luego $L(\varepsilon)=Mv$, de modo que $v=M^{-1}L(\varepsilon)$.
	Escribiendo $M^{-1}=(a_{ij})$ obtenemos $$b_i=\sum_{j=1}^ra_{ij}\log\sigma_j(\varepsilon)$$ para $1\leq i\leq r$.
	Usando la desigualdad triangular concluimos que
	\begin{equation}
		\max_{1\leq i\leq r}|b_i|\leq \left( \max_{1\leq i\leq r}\sum_{j=1}^r|a_{ij}| \right)\cdot \max_{1\leq i\leq r}\log|\sigma_j(\varepsilon)|.
		\tqedhere
	\end{equation}
\end{proof}

Ahora sí, el teorema.

\begin{thm}
	\label{SM2}
	La ecuación \eqref{eu2} tiene finitas soluciones y estas se pueden determinar de manera efectiva, en el caso en que $K$ es cuerpo numérico y $\Gamma=\mathcal{O}_K^{\times}$.
\end{thm}
\begin{proof}
	Sean $x,y\in \mathcal{O}_K^{\times}$ y $a,b\in K^{\times}$ tales que $$ax+by=1.$$ Escribimos $$x=\zeta_1\varepsilon_1^{a_1}\cdots\varepsilon_r^{a_r}\qquad x=\zeta_2\varepsilon_1^{b_1}\cdots\varepsilon_r^{b_r}$$ con $a_i,b_i\in \Z$ y $\zeta_1,\zeta_2\in U_K$. Luego $$a\zeta_1\varepsilon_1^{a_1}\cdots\varepsilon_r^{a_r}+b\zeta_2\varepsilon_1^{b_1}\cdots\varepsilon_r^{b_r}=1.$$ 
	Daremos cotas inferiores y superiores de
	\[
		\Lambda_i:=|\sigma_i(a)\sigma_i(\zeta)\sigma_i(\varepsilon_1)^{a_1}\cdots\sigma_i(\varepsilon_r)^{a_r}-1|=|\sigma_i(b)\sigma_i(y)|
	\]
	para un valor específico de $i$.

	Sean $i$ y $j$ los índices que minimizan y maximizan resp.\ a los $|\sigma_k(y)|$ para $1\leq k\leq d$. Luego \begin{equation}
		|\sigma_i(y)|^{d-1}|\sigma_j(y)|\leq \prod_{i=d}^r|\sigma_i(y)|=\lvert\galnorm_{K/\Q}(y)\rvert=1.
	\end{equation} Del lema anterior sigue $B\leq C\cdot \log|\sigma_j(y)|$. Luego $e^{B/C}\leq |\sigma_j(y)|$ y entonces $|\sigma_j(y)|^{-\frac{1}{d-1}}\leq e^{-\frac{B}{C(d-1)}}$. Juntando esto con (3) se sigue que $$|\sigma_i(y)|\leq e^{-\frac{B}{C(d-1)}}.$$ Se sigue que $$|\Lambda_i|\leq |\sigma_i(b)|e^{-\frac{B}{C(d-1)}}.$$ Por otro lado, el teorema de Baker implica $$|\Lambda_i|\geq (1+B)^{-C'}$$ para una constante computable $C'$ que depende de $a$, las unidades fundamentales $\varepsilon_1,\ldots,\varepsilon_r$ y la finitas raíces de la unidad en $K$. Deducimos entonces que $$(1+B)^{-C'}\leq |\sigma_i(b)|e^{-\frac{B}{C(d-1)}}$$ y nos entrega una cota computable para $B$. 
\end{proof}

En 1984, Evertse mostró que la cantidad de soluciones de dicha ecuación está acotado por $3\cdot 7^{d+2r}$ donde $d=[K:\Q]$. En 1996, Beukers y Schlickewei dieron la cota uniforma $512^{r+2}$.
\section{Aplicaciones de ecuaciones de unidades}

La primera aplicación será dar finitud a ciertas formas binarias.

\begin{prop}
	Sea $F(x,y)=a_0x^d+a_1x^{d-1}y+\cdots+a_{d-1}xy^{d-1}+a_dy^d\in \Z[x,y]$ una forma binaria irreducible de grado $d\geq 3$ tal que $F(x,1)$ tiene al menos 3 ceros distintos en $\C$. Luego la ecuación $$F(x,y)=m$$ para un $m\in \Z$ tiene finitas soluciones enteras  y estas pueden ser determinadas de manera efectiva.
\end{prop}
\begin{proof}
	Primero nos reducimos al caso $a_0=1$. Ciertamente, $G(x,y):=F(a_0x,y)=a_0^{d-1}F(x,y)$ es una forma binaria con coeficientes enteros y sus soluciones se corresponden con las de la ecuación $G(x,y)=a_0^{d-1}m$. Esto muestra que la reducción es posible.

	Sea $K$ el cuerpo de escisión de $F(x,1)$, de modo que $$F(x,y)=(x-\alpha_1y)(x-\alpha_1y)\cdots(x-\alpha_dy)$$ con $\alpha_1,\ldots,\alpha_d\in \mathcal{O}_K$. Luego la ecuación $F(x,y)=m$ implica que los $(x-\alpha_iy)$ dividen a $m\in \mathcal{O}_K$ y por ende $$(x-\alpha_iy)=\mu_i\beta_i,\quad \mu_i\in \mathcal{O}_K,\;\beta_i\in \mathcal{O}_K^{\times}$$ para todo $1\leq i\leq d$. Como $\galnorm_{K/\Q}(\mu_i)\leq \galnorm_{K/\Q}(m)$ para todo $i$, solo hay finitas elecciones para los $\mu_i$. Veremos que también es el caso para los $\beta_i$.

	Reordenando si es necesario, supongamos que $\alpha_1,\alpha_2$ y $\alpha_3$ son ceros distintos. Mediante combinaciones lineales de los $(x-\alpha_iy)=\mu_i\beta_i$, $i=1,2,3$ podemos eliminar los $x$ e $y$ para obtener:
	$$(\alpha_2-\alpha_3)\mu_1\beta_1+(\alpha_3-\alpha_1)\mu_2\beta_2=(\alpha_1-\alpha_2)\mu_3\beta_3.$$

	Pero entonces $$\left( \frac{(\alpha_2-\alpha_3)\mu_1}{(\alpha_1-\alpha_2)\mu_3} \right)\frac{\beta_1}{\beta_3}+\left( \frac{(\alpha_3-\alpha_1)\mu_2}{(\alpha_1-\alpha_2)\mu_3} \right)\frac{\beta_2}{\beta_3}=1$$ donde $\beta_1/\beta_2,\beta_1/\beta_3\in \mathcal{O}_K^{\times}$ y los coeficientes que acompañan a estos bichos están en $K^{\times}$ (ojo que usamos que los ceros son distintos para poder dividir).

	Por el teorema~\ref{SM2}, existen finitas elecciones para el cocientes $\beta_1/\beta_2$. Usando las relaciones $(x-\alpha_i)=\mu_i\beta_i$ vemos que entonces hay finitas elecciones para el cocientes $(x-\alpha_1y)/(x-\alpha_2y)$. Esto implica finitas elecciones para el cociente $x/y$ y entonces hay finitas soluciones para $F(x,y)=m$ como buscábamos.
\end{proof}

\begin{prop}
	Sea $f(x)\in \Z[x]$ libre de cuadrados y con al menos tres ceros distintos en $\C$. Luego la ecuación $y^2=f(x)$ tiene finitas soluciones enteras.
\end{prop}

\nocite{baker:transcendental, evertse:unit, stewart1991abc, stewart1991abc, stewart1986oesterle}
\printbibliography

\end{document}

