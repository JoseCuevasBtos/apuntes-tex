\chapter{Identificación de errores}
Como \LaTeX{} debe ser compilado está expuesto a errores de escritura.
El objetivo de los errores es advertir al usuario de mala sintaxis, no debe reaccionarse con miedo sino que debe leerse con detenimiento;
usualmente un error indica también la línea que causa el error.

\begin{enumerate}
	\item \verb|Overfull \hbox (...pt too wide):| \\
		Significa que tienes una línea más larga de lo que debería.
		Como \LaTeX{} ocupa la alineación <<justificada>> trata de que todas las líneas midan lo mismo, pero
		basta poner una ecuación al final de la línea para excederse.
		Ésto no es un error sino más bien una advertencia: mi recomendación es arreglarlo con \lstinline|\break|
		o \lstinline|\-|.

	\item \verb|Missing \begin{document}:| \\
		Significa que no ocupaste el commando \verb|\begin{document}|.

	\item \verb|\comando before \documentclass:| \\
		Significa que no ocupaste el comando \verb|\documentclass| u ocupaste otro comando ántes.

	\item \texttt{File `\textit{type}.cls' not found:} \\
		Significa que intentaste escribir \lstinline|\documentclass{¬type¬}| con un formato inexistente o probablemente te equivocaste al escribir.

	\item \texttt{File `\textit{paquete}.sty' not found:} \\
		Significa que trataste de importar dicho paquete que no existe.

	\item \texttt{Option clash for package \textit{paquete}:} \\
		Significa que usaste una opción inexistente para dicho paquete.

	\item \texttt{Missing \$ inserted:} \\
		Significa que intentaste escribir una ecuación matemática sin usar \$ o \$\$.

	\item \verb|\begin{entorno} on input line ... ended by \end{document}|: \\
		Significa que abriste un entorno sin cerrarlo con \verb|\end{entorno}|.

	\item \verb|\begin{document} ended by \end{entorno}:| \\
		Lo contrario al anterior, se te olvido abrir el entorno con \verb|\begin{entorno}|.

	\item \texttt{Undefined control sequence:} \\
		Significa que usaste un comando que no existe, o probablemente que escribiste mal.
		Otra posibilidad común es que usaste un comando de un paquete sin importarlo primero como escribir matemáticas sin \texttt{amsmath}
		o importar una foto sin \texttt{graphicx}. 

	\item \texttt{File ended while scanning use of \lstinline|\¬comando¬|:} \\
		Significa que abriste un comando como \lstinline|\¬comando¬{...| sin cerrarlo con la llave \lstinline|}| faltante.

	\item \texttt{Misplaced alignment tab character \&:} \\
		Significa que usaste \& fuera de un entorno apropiado, como una tabla fuera del entorno \texttt{tabular} o \texttt{align}.

	\item \texttt{File `\textit{foto.jpg}' not found: using draft setting:} \\
		Significa que intentaste incluir un archivo de una foto que no existe.
\end{enumerate}

También al malemplear comandos exclusivos de un determinado paquete te pueden saltar errores del tipo \texttt{Package
\textit{paquete} Error}
% I do not know the key `\textit{opción}' and I am going to ignore it. Perhaps you misspelled it:}
seguido de una breve explicación.
Un ejemplo es una opción inexistente, obsoleta o mal escrita, en cuyo caso el mensaje suele ser \texttt{I do not know
the key `\textit{opción}' and I am going to ignore it. Perhaps you misspelled it.}
