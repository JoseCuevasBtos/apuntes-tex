\chapter{Introducción}
Este texto pretende ser introductorio y en general sólo abarca cosas que creo que un lector cualquiera podría necesitar a la hora de usar \LaTeX{}. Por su naturaleza, sólo se roza la superficie de lo que es posible en este lenguaje, si lo que busca es una documentación más detallada revise el párrafo de \textbf{Más allá...} al final de la sección~\S\ref{sec:setup}.

\section*{Instalación y compilación}
Instalar \LaTeX{} puede ser complicado en Windows y en Mac OS, para ello se recombienda optar por el sitio en línea
\href{https://www.overleaf.com/}{\sffamily\color{newgreen}overleaf};
sin embargo, si usted planea hacer notas extensas (o mejor aún, escribir un libro), entonces es preferible instalar \LaTeX{} en su propio computador,
para ello existen los siguientes métodos según su sistema operativo:
Mik\TeX{} en Windows, Mac\TeX{} en Mac OS y \TeX{}Live en Linux.

Hay varios editores de \LaTeX{} que pueden ser útiles, los más populares, en orden desde más sencillo a más difícil son:
overleaf (en línea), \TeX{}Maker, VSCode, Vim, Emacs.
Naturalmente, los últimos son los más recomendados ya que son los más extensibles, pero son más complicados a menos que ya se empleen con anterioridad,
hay una serie de extensiones en cada caso para VSCode, Vim e Emacs para facilitar su uso en \LaTeX{}.

Sin lugar a dudas el mejor método es trabajar en Linux, no solo la instalación es mejor sino que hay varias cosas recomendables para usuarios
experimentados:
se pueden hacer archivos del tipo \texttt{MakeFile} con todas las instrucciones detalladas,
se puede escribir un ejecutable archivo en \texttt{shell},
o también existe un paquete de nombre \texttt{rubber} que permite compilaciones automáticas e inteligentes de \LaTeX{} escrito en \texttt{Python};
personalmente utilizo \texttt{rubber}, pero aún así dejo algunos ejemplos aquí.

\section*{Historial de versiones}
Fechas en formato \textsc{dd-mm-aa}. 
\begin{description}
	\item[06-08-20] Publicación original como artículo.
	\item[20-10-20] Corrección de errores ortográficos.
	\item[26-06-21] Actualización a formato libro.
\end{description}
