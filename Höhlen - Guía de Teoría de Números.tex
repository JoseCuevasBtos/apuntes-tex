\documentclass[letterpaper,11pt]{article}

\usepackage[utf8]{inputenc}
\usepackage[spanish]{babel}
\usepackage{amsmath, amsfonts}
\usepackage[backend=biber]{biblatex}
\addbibresource{Teoria de Numeros.bib}

\DeclareUnicodeCharacter{0301}{\'{e}}

\newcommand{\Z}{\mathbb{Z}}
\DeclareMathOperator{\mcd}{mcd}
\DeclareMathOperator{\mcm}{mcm}

\title{Guía de Teoría de Números}
\author{Joseph Höhlen}
\date\today

\begin{document}

\maketitle

\begin{enumerate}
\item Demuestre que existen infinitos enteros $n$ tales que $4n^2+1$ es divisible por 5 y por 13 al mismo tiempo.
\item Encontrar todos los elementos del conjunto
$$S=\left\{z\in\Z:z=\frac{x^2-3x+2}{2x+1},\; x\in\Z\right\}.$$
\item Encontrar todos los enteros $n$ tales que si les borramos el último dígito obtenemos un divisor de $n$.
\item
	\begin{enumerate}
	\item Encontrar todos los naturales $n$ tales que $2^n-1$ es divisible por 7.
	\item Probar que no existe natural $n$ tal que $2^n+1$ es divisible por 7.
	\end{enumerate}
\item Encontrar el mayor $x\in\Z$ tal que $23^{6+x}$ divide a $2000!$.
\item (OIM 1969) Pruebe de que existen infinitos naturales $a$ con la siguiente propiedad: el número $z=n^4+a$ no es primo para todo natural $n$.
\item (San Petersburgo 1996) Encontrar todos los $n$ naturales tales que $3^n+5^n$ divide a $3^{n+1}+5^{n+1}$.
\item Encontrar la suma de todos los números pares entre $n^2-n+1$ y $n^2+2+1$.
\item (OIM 1970) Encuentre el conjunto de todos los enteros positivos $n$ para los cuales si partimos el conjunto $\{n,n+1,n+2,n+3,n+4,n+5\}$ en dos subconjuntos, el producto de los elementos de ambos subconjuntos son iguales.
\item (OIM 1998) Encuentre todos los pares $(a,b)$ de enteros positivos tales que $ab^2+b+7$ divide a $a^2b+a+b$.
\end{enumerate}

\nocite{*}
\printbibliography

\end{document}