\documentclass[topologia-analisis.tex]{subfiles}
\begin{document}

\newrefsegment
\chapter{Conexión y la categoría de espacios topológicos}
\label{cha:normed_spaces}

En éste capítulo veremos una introducción a los métodos categoriales.
Más específicamente, comenzaremos dando una breve introducción a otra de las definiciones vitales de la topología: la conexión y sus derivados.
Luego, al hablar del extremo opuesto, los espacios \textit{hereditariamente} disconexos, introduciremos ciertas nociones de límite categoriales
de espacios topológicos.
Después retomaremos un problema de compacidad: las compactificaciones como un tipo de adjunción.
Y finalmente hablaremos de espacios de funciones desde la perspectiva de una adjunción, lo que a mi juicio esclarece la topología que uno le quiere dar
(convergencia puntual, uniforme y compacto-abierto).

\section{Espacios conexos}
\begin{prop}
	Son equivalentes:
	\begin{enumerate}
		\item $X$ no puede expresarse como $X_1\oplus X_2$ con $X_1,X_2$ subespacios de $X$.
		\item $\emptyset, X$ son los únicos conjuntos cerrados y abiertos de $X$.
		\item Si $X = X_1\cup X_2$ con $X_1,X_2$ separados; entonces algunos es vacío.
		\item Para toda aplicación continua $f\colon X \to D(2)$ se cumple que $f[X] = \{0\}$ o $f[X] = \{1\}$.
	\end{enumerate}
\end{prop}
% \begin{proof}
% 	$1) \implies 2)$. Usar proposición~\ref{thm:open_subspace_sum}.
% 	\par
% 	$2) \implies 3)$. Por definición $X_1\cap\overline X_2 = \overline X_1\cap X_2 = \emptyset$. Luego, $\overline X_1 \subseteq X_2^c \subseteq X_1$ y así se obtiene que $\overline X_1 = X_1 = X_2^c$ y que $\overline X_2 = X_2 = X_1^c$. Luego son cerrados y abiertos cuya unión es el espacio, ergo, alguno es vacío.
% 	\par
% 	$3) \implies 4)$. $X = f^{-1}(0) \cup f^{-1}(1)$ donde ambos son cerrados y abiertos disjuntos, luego están separados.
% 	\par
% 	$4) \implies 1)$. Usar proposición~\ref{thm:open_subspace_sum}.
% \end{proof}

\begin{mydefi}[Espacio conexo]\index{conexo}
	Se le dice \strong{conexo} a cualquier espacio que cumple alguna condición del teorema anterior. 
	% Es cualquiera que cumple con alguna de las condiciones del teorema anterior.
	De lo contrario se dice que un espacio es \strong{disconexo}.
\end{mydefi}
Ejemplos triviales de conjuntos conexos lo son $\emptyset$ y los conjuntos singulares.
Como ejercicio argumente por qué $[0,1]\cup[2,3]$ y $\Q$ son subespacios disconexos de $\R$.

\begin{cor}
	Sea $X$ un espacio topológico, $C \subseteq X$ conexo y $X_1, X_2$ conjuntos separados tales que $C \subseteq X_1\cup X_2$.
	Entonces $C \subseteq X_1$ o $C \subseteq X_2$.
\end{cor}
\begin{proof}
	Veamos la contrarrecíproca: si $C \not\subseteq X_1$ y $C \not\subseteq X_2$, entonces $C\cap X_1 \ne \emptyset \ne C\cap X_2$.
	Definamos $A := C \cap X_1$ y $B := C\cap X_2$, nótese que $A\cap B = \emptyset$ y que $A \cup B = C$.
	Además
	$$ \overline A \cap B \subseteq (\overline C\cap\overline X_1) \cap (C\cap X_2) = \emptyset $$
	y análogamente $A \cap \overline B = \emptyset$, por lo que están separados y $C$ es disconexo.
\end{proof}

\begin{thmi}
	Todo subespacio de $\overline{\R}$ (bajo la topología del orden) es conexo syss es un intervalo.
\end{thmi}
\begin{proof}
	$\implies$.
	En $\overline{\R}$ todo subespacio $S$ posee ínfimo y supremo $a, b$;
	luego para probar que es un intervalo basta probar que para todo $a < x < b$ se cumple que $x \in S$.
	De lo contrario $S\cap [-\infty, x)$ y $S\cap(x, \infty]$ son abiertos disjuntos cuya unión es $S$.
	\par
	$\impliedby$.
	Supongamos por contradicción que un intervalo puede representarse como la suma de dos subespacios abiertos no vacíos $U, V$ de tal forma
	que $x\in U$, $y\in V$ y $x < y$. Sea $U' := U\cap[x,y]$, $V' := V\cap[x,y]$ tal que $U'\cup V' = [x,y]$.
	Notemos que $U,V$ son abiertos-cerrados en el subespacio $I$, de forma que $U',V'$ son cerrados.
	Sea $s := \sup(U')$, luego se puede comprobar que $s \in \overline U' \subseteq U$ y que $s \in \overline V' \subseteq V$;
	lo que contradice a la cualidad de subespacios disjuntos.
\end{proof}
Es fácil notar que la topología del orden sobre $\overline{\R}$ genera la topología usual sobre $\R$ como subespacio. Ergo, se concluye el mismo resultado para $\R$.

\begin{thm}
	La imagen continua de conexos es conexa.
\end{thm}

\begin{cor}
	Si $X$ es de Urysohn, conexo y $|X| \ge 2$, entonces $|X| \ge \mathfrak{c}$.
\end{cor}
\begin{proof}
	Sean $x, y \in X$, por ser de Urysohn están competamante separados por lo que existe $f\colon X \to [0, 1]$ tal que
	$f(x) = 0$ y $f(y) = 1$.
	Como $X$ es conexo, entonces $f$ ha de ser suprayectiva, luego $|X| \ge \mathfrak{c}$.
\end{proof}
Nótese que algun axioma de separación fuerte es necesario, ya que el espacio de Sierpiński es un ejemplo sencillo de un espacio $T_0$ conexo con dos puntos
(en general, hay toda una familia de espacios espectrales que incurren en el mismo fenómeno).
Otro ejemplo es $\N$ con la topología cofinita, el cual es $T_1$ y conexo.

\begin{thmi}[Teorema del valor intermedio de Weierstrass]\index{teorema!del valor intermedio}
	Sea $f\colon [a,b] \to \R$, donde $f(a) < f(b)$, entonces para todo $y \in (f(a), f(b))$ existe $c\in (a, b)$ tal que $f(c) = y$.
\end{thmi}
\begin{figure}[!hbt]
	\centering
	\includegraphics{intermediate-value.pdf}
	\caption{Teorema del valor intermedio de Weierstrass.}
\end{figure}
\begin{cor}
	Si $f\colon A\to\R$ es inyectiva y continua con $A\subseteq\R$ un intervalo, entonces $f$ es estrictamente monótona.
\end{cor}

\begin{cor}[Bolzano]
	Sea $f\colon [a,b] \to \R$, donde $f(a)\cdot f(b) < 0$, entonces existe $c\in (a, b)$ tal que $f(c) = 0$.
\end{cor}
\begin{cor}
	Toda función continua de la forma $f\colon [a,b] \to [a,b]$ posee un punto fijo.
\end{cor}

\begin{thm}
	Sean $\{C_i\}_{i\in I}$ una familia de subespacios conexos de $X$, tal que existe algún $i_0 \in I$ tal que $C_{i_0}$ no está separado del resto de $C_i$,
	entonces $\bigcup_{i\in I} C_i$ es conexo.
\end{thm}
\begin{proof}
	Sea $C := \bigcup_{i\in I} C_i = X_1 \cup X_2$ donde $X_1,X_2$ están separados.
	Como $C_{i_0}$ es conexo (por hipótesis) entonces $C_{i_0}$ está contenido en $X_1$ o en $X_2$, supongamos sin perdida de generalidad que
	$C_{i_0} \subseteq X_1$.
	Lo mismo aplica para el resto de $C_j$'s, pero como $C_j$ no está separado de $C_{i_0}$, entonces $C_j \subseteq X_1$ para todo $j \in I$.
	En conclusión $C \subseteq X_1$ y por ende $X_2 = \emptyset$.
\end{proof}

\begin{cor}
	Sean $\{C_i\}_{i\in I}$ una familia de subespacios conexos de $X$, tal que su intersección es no-vacía, entonces $\bigcup_{i\in I} C_i$ es conexo.
\end{cor}
\begin{cor}
	Sea $A$ un conjunto cualquiera tal que $C \subseteq A \subseteq \overline C$ con $C$ conexo, entonces $A$ es conexo.
	En particular la clausura de un conexo es conexa.
\end{cor}
\begin{proof}
	Notemos que para todo $x\in A$ se cumple que $C$ y $\{x\}$ no están separados.
	Luego el conjunto $\{C\}\cup\{\{x\} : x\in A\}$ es una familia de conexos tal que uno no está separado del resto.
\end{proof}

\begin{cor}
	Si $X$ posee un subconjunto denso conexo, entonces $X$ es conexo.
\end{cor}
\begin{cor}
	Si todo par de puntos en $X$ están contenidos en un subconjunto conexo de $X$, entonces $X$ es conexo.
\end{cor}

\begin{thm}
	La topología producto de espacios no vacíos es conexa syss lo son los factores.
\end{thm}
\begin{proof}
	Sea $\{X_i\}_{i\in I}$ tal que $X := \prod_{i\in I}X_i$.
	\par
	$\implies$. Si $X$ es conexo entonces $\pi_j(X) = X_j$ lo es pues la proyección es continua y la imagen de conexos es conexa.
	\par
	$\impliedby$.
	Para $I = \{0,1\}$ basta notar que $X = (X_1\times\{x_2\})\cup(\{x_1\}\times X_2)$ donde ambos son conexos no-disjuntos, por ende, $X$ es conexo.
	Luego se puede aplicar inducción para notar que el producto finitos de conexos es conexo.

	Si los $X_i$ son conexos y no vacíos, entonces basta considerar un $\vec u \in X$, y notar que dado%
	\footnote{Un subconjunto finito de $I$.}
	$J \in [I]^{<\omega}_{\neq\emptyset}$ se define
	$$ \prod_{j\in J} X_j \simeq C_J := \prod_{i\in I} A_i,\quad A_i := \begin{cases}\{\vec u_i\} &i\notin J \\ X_i &i\in J\end{cases}  $$
	Luego la unión de los $C_J$ es un conjunto denso (¿por qué?) de $X$ que es conexo pues es la intersección de conexos
	cuya intersección contiene a $\vec u$, es decir, es no vacía.
\end{proof}
Nótese que aquí se puede criticar de que hace falta AE para extraer el $\vec u\in X$,
pero de darse que el producto sea nulo entonces es también trivialmente conexo.

\begin{cor}
	Los espacios euclídeos, los cubos de Tychonoff y los cubos de Alexandroff son conexos.
	Los cubos de Cantor son siempre disconexos.
\end{cor}

\textbf{Ejemplo (la topología por cajas puede no ser conexa).}
Sea $\prod_{i\in\N}^{\rm Box} \R$.
Sea $A$ el conjunto de las sucesiones acotadas y $B := A^c$ es el conjunto de las sucesiones no-acotadas.
Para probar que $A$ y $B$ son abiertos, basta notar que son entornos de todos sus puntos, para ello si $\vec x := (x_i)_{i=0}^\infty \in A$, entonces $\vec x\in \prod_{i=0}^\infty B_1(x_i) \subseteq A$, y de manera similar si $\vec y := (y_i)_{i=0}^\infty \in B$ entonces $\vec y \in \prod_{i=0}^\infty B_1(y_i) \subseteq B$, en donde, el producto de abiertos es abierto por lo que se prueba que $A$ y $B$ lo son.

\begin{mydefi}
	Se define la \strong{componente conexa}\index{componente conexa} $C(x)$ de un punto $x$ es la unión de todos los subespacios conexos que le contienen.
	La \strong{cuasicomponente}\index{cuasicomponente} $Q(x)$ de un punto $x$ es la intersección de los entornos abiertos cerrados de $x$.
	\nomenclature{$C(x), Q(x)$}{La componente conexa, cuasi-componente de $x$ resp.}
\end{mydefi}
\begin{prop}
	Se cumple que:
	\begin{enumerate}
		\item $C(x)$ es cerrado y es el máximo subespacio conexo que contiene a $x$.
		\item Para todo par de puntos sus componentes conexas (y cuasi-componentes) son iguales o disjuntas.
		\item $Q(x)$ es cerrado y el conjunto de las cuasi-componentes forma una partición estricta del espacio.
		\item Si $x\sim y$ syss $x,y$ están contenidos en un subespacio conexo, entonces $\sim$ es de equivalencia,
			cuyas clases de equivalencia son las componentes conexas del espacio.
		\item $C(x) \subseteq Q(x)$.
	\end{enumerate}
\end{prop}
\begin{proof}
	Probaremos la última: Sea $F$ un entorno abierto-cerrado de $x$, de modo que está separado de $F^c$.
	Luego $C(x)\cap F$ y $C(x)\cap F^c$ están separados, pero como $C(x)$ es conexo, entonces o $C(x)\cap F$ o $C(x)\cap F^c$ es disjunto.
	Pero $x \in C(x)\cap F$, luego $C(x)\cap F^c = \emptyset$, ergo $C(x) \subseteq F$. Como $C(x)$ es una cota inferior y $Q(x)$ es el ínfimo,
	se cumple que $C(x) \subseteq Q(x)$ como se quería probar.
\end{proof}

Un ejemplo es que si consideramos el subespacio $[0,1]\cup[2,3]$ de $\R$ es claro que es disconexo,
pero sus componentes conexas son $[0,1]$ y $[2,3]$ que vendrían a ser las partes conexas del conjunto.

\begin{ex}
	En $\Q$ es fácil comprobar que todo conjunto con al menos dos elementos es disconexo, de modo que las componentes conexas son puntos.
	No obstante, $\Q$ \textit{no} es homeomorfo al espacio discreto $D(\aleph_0)$.
	Para demostrar esto, basta notar que $(0, 1) \cap \Q$ es un conjunto que no es cerrado, mientras que en el espacio discreto todo conjunto es cerrado.

	La confusión de querer descomponer $\Q$ como suma de sus componentes yace de que las componentes son siempre cerradas, pero no siempre abiertas.
	Por ejemplo, son abiertas cuando hay solo finitas componentes.
\end{ex}
% Algo interesante es que las componentes conexas de $\Q$ son todos los conjuntos singulares, luego es fácil notar que es homeomorfo $D(\aleph_0)$.
\begin{prop}
	En un espacio de Hausdorff compacto, las componentes y cuasicomponentes coinciden.
\end{prop}
\begin{proof}
	Por la proposición anterior, basta probar que $C(x) \supseteq Q(x)$, vale decir, que toda cuasicomponente es conexa.
	Denotemos $Q := Q(x)$ y sean $X_1, X_2$ un par de cerrados disjuntos en $Q$ tales que $X_1 \cup X_2 = Q$.
	Supongamos que $x \in X_1$, como el espacio es compacto Hausdorff, entonces es normal y existen un par de abiertos (en $X$) disjuntos $U_1, U_2$
	tales que cada $X_i \subseteq U_i$.
	Así pues, $Q \subseteq U_1 \cup U_2$ y, por compacidad, existen finitos abiertos-cerrados $F_1, F_2, \dots, F_k$ tales que
	$$ Q \subseteq F := \bigcap_{i=1}^{k} F_i \subseteq U \cup V, $$
	donde claramente $F$ es abierto-cerrado. Como $\overline{U \cap F} \subseteq \overline{U} \cap F = \overline{U} \cap (U \cup V) \cap F = U \cap F$,
	vemos que $U \cap F$ también es abierto y cerrado.
	Como $x \in U \cap F$, entonces $Q \subseteq U \cap F$ y
	$$ X_2 \subseteq Q \subseteq U \cap F \subseteq U, $$
	como $X_2 \subseteq V$, entonces ha de darse que $X_2 = \emptyset$ como se quería probar.
\end{proof}

\begin{mydefi}
	Se le llama un \strong{arco}\index{arco} entre $a$ y $b$ a una función continua $\gamma\colon [0,1]\to X$ tal que $\gamma(0) = a$ y $\gamma(1) = b$.
	Un espacio se dice \strong{arcoconexo}\index{espacio!arcoconexo} si entre todo par de puntos existe un arco.

	En un EVT se le dice un \strong{segmento}\index{segmento} entre $a$ y $b$ al conjunto $\{\lambda a + (1-\lambda)b: \lambda\in[0,1]\}$.
	Un subconjunto se dice \strong{convexo}\index{convexo (conjunto)} si contiene todo segmento entre sus puntos.
\end{mydefi}
Es claro que todo segmento es un tipo de arco.

\begin{prop}
	Se cumple:
	\begin{enumerate}
		\item Todo espacio convexo es arcoconexo.
		\item La imagen continua de un arcoconexo es arcoconexa.
		\item La unión de una familia de intersección total no vacía de espacios arcoconexos es arcoconexa.
		% \item La intersección de convexos es convexa.
		\item Todo espacio arcoconexo es conexo.
	\end{enumerate}
\end{prop}
\begin{proof}
	En particular probaremos la última:
	Llamemos $\Img( C([0,1], X) )$ a la familia de las imágenes de todas las funciones continuas de dominio $[0,1]$ y codominio $X$.
	Luego si $X$ es no vacío entonces posee un elemento $x$, luego basta considerar el subconjunto $S$ de $C([0,1], X)$
	que contiene a las funciones que pasan por $x$, como $[0,1]$ es conexo, entonces $\Img(S)$ es una familia de conexos que contienen $x$,
	y por definición, para todo $y\in X$ existe $f\in C([0,1], X)$ tal que $x,y\in\Img f$, luego $\bigcup \Img(S) = X$, luego $X$ es conexo.
\end{proof}

\thmdep{AE}
\begin{thm}
	El producto de espacios arcoconexos no vacíos es arcoconexo.
\end{thm}
\begin{proof}
	Sean $\{X_i\}_{i\in I}$ una familia de arcoconexos, como son no vacíos, por AE $Y := \prod_{i\in I} X_i$ es no vacío.
	Sean $\vec u,\vec v\in Y$, luego si $f_i:[0,1] \to X_i$ es un arco de extremos $u_i$ y $v_i$,
	entonces la diagonal $f\colon [0,1] \to Y$ es continua y es un arco entre $\vec u$ y $\vec v$.
\end{proof}
\thmdep{}

% \begin{mydef}[Envolvente convexa]\index{envolvente convexa}
% 	Dado un subconjunto $A$ de un espacio vectorial $V$, se le llama \strong{envolvente convexa} de $A$, al mínimo conjunto convexo que le contiene, i.e.,
% 	$$ \conv A := \bigcap\{C: A\subseteq C\wedge C\text{ convexo}\}. $$
% \end{mydef}
% \begin{prop}
% 	Dado $A$ subconjunto de un $\korpe$-espacio vectorial, entonces
% 	$$ \conv A = \left\{ \sum_i\alpha_ix_i : \sum_i\alpha_i = 1 \wedge \forall i\;(x_i\in A\wedge \alpha_i>0) \right\} $$
% \end{prop}

\begin{mydef}
	Un espacio se dice \strong{débil-localmente conexo}\index{espacio!débil-localmente conexo} si todo punto posee un entorno conexo.
\end{mydef}
\begin{prop}
	Para un espacio topológico $X$ son equivalentes:
	\begin{enumerate}
		\item $X$ es débil-localmente conexo.
		\item Las componentes conexas de $X$ son abiertas.
		\item $X$ es la suma de espacios topológicos.
	\end{enumerate}
	Más aún, los factores de la suma son únicos salvo isomorfismo y salvo permutación.
\end{prop}

\begin{ex}
	Sea $X$ un espacio con más de tres puntos, con el punto excluído $\eta$.
	Es decir, un abierto de $X$ es cualquier subconjunto que no contiene a $\eta$ o todo $X$.
	Éste espacio es conexo, pues para que $X = U \cup V$, entonces sin perdida de generalidad $\eta \in U$ y luego $U = X$; así que también es
	débil-localmente conexo.
	Pero $V := X \setminus \{ \eta \}$ es un subespacio abierto de $X$ con la topología discreta y con dos puntos, luego es disconexo.

	También podemos elegir que $X$ sea infinito con punto excluido $\eta$, en cuyo caso será compacto y $X \setminus \{ \eta \}$ no lo será,
	por lo que $X$ no es localmente compacto.
\end{ex}
El ejemplo anterior exhibe las falencias de la definición anterior, lo que motiva la siguiente definición.

\begin{mydefi}
	Se dice que un espacio es \strong{localmente conexo}\index{espacio!localmente conexo} (resp. \strong{arcoconexo})\index{espacio!localmente arcoconexo}
	si todo punto posee una base de entornos conexos (resp. arcoconexos).
\end{mydefi}
Así, lo que probamos es que el ejemplo anterior es débil-localmente conexo, pero no localmente conexo.

\begin{thm}
	Un espacio es localmente conexo (resp. localmente arcoconexo) syss todas las componentes conexas (resp. arcoconexas)
	de todo subespacio abierto son abiertas (y, por tanto, cerradas).
\end{thm}
\begin{proof}
	La demostración es la misma reemplazando conexo por arcoconexo donde corresponda.

	$\implies$. Sea $C$ una componente conexa y sea $x\in C$.
	Sea $\mathcal{B}$ una base formada por conexos, como $\mathcal{B}$ es base existe $B \in \mathcal{B}$ tal que $x\in B$, como $B$ es conexo
	y $C$ es el mayor conexo que contiene a $x$ debe darse que $x \in B \subseteq C$, luego $C$ es entorno de $x$.

	$\impliedby$.
	Sea $\mathcal{B}$ la familia de componentes conexas de todo subespacio abierto.
	Sea $U$ un entorno abierto cualquiera de $x$, luego $x \in C_U(x) \subseteq U$ y $C_U(x) \in \mathcal{B}$, por lo que $\mathcal{B}$ es base.
\end{proof}
Nótese que como la clausura de un subespacio conexo es conexo, entonces las componentes conexas siempre son cerradas, pero ese no es el caso para
subespacios arcoconexos; por ello, la observación en paréntesis es interesante para componentes arcoconexas.
\begin{ex}
	Considere $X = \Q$.
	Veremos que las componentes conexas son puntos: para ello, sea $x \in \Q$ y sea $S \subseteq \Q$ que contiene a $x$ y a otro punto $y$.
	Ahora bien, por densidad de los irracionales, existe $z \in \R \setminus \Q$ tal que $z$ está entre $x, y$.
	Así, $(-\infty, z) \cap \Q, (z, \infty) \cap \Q$ son abiertos disjuntos de $\Q$ y al intersectarlos con $S$ prueban que $S$ es disconexo.

	Se nota que las componentes arcoconexas son también puntos, luego todas son cerradas, pero ninguna es abierta.
\end{ex}

\begin{thm}
	Un espacio localmente arcoconexo es conexo syss es arcoconexo.
\end{thm}
\begin{proof}
	Ya vimos que todo arcoconexo es conexo, así que veremos el converso:
	Si el espacio es localmente arcoconexo, entonces sus componentes arcoconexas son abiertas y cerradas (y no vacías), como el espacio es conexo,
	entonces la componente debe igualar al espacio.
\end{proof}

% \begin{exn}[peine del topólogo]
% 	\addnamedexample{Peine del topólogo}{Un espacio que es conexo pero no es arcoconexo ni localmente conexo}
% 	Consideremos
% 	$$ Y := (0,1]\times\{0\} \cup \bigcup_{n \in \N_{\ne 0}} \{1/n\} \times [0,1] $$
% 	que es conexo (¿por qué?) y luego definimos
% 	$$ P := Y \cup \{0\} \times (0,1]. $$
% 	Veamos que \underline{$P$ es conexo:} para ello nótese que $\{0\} \times (0, 1]$ e $Y$ son conexos disjuntos, así que basta ver que no están separados.
% 	Sea $(0, y) \in P$, es decir, $y \in (0, 1]$, luego siempre se cumple que $(s, y) \in B_r\big( (0, y) \big) \cap P$ donde $0 \le s < r$ y por propiedad
% 	arquimediana existe $1/n < r$, luego $(1/n, y) \in Y$ por lo que es conexo.

% 	Sin embargo \underline{$P$ no es localmente conexo:} Nótese que $B_r\big( (0, 1) \big) \cap P$ con $r < 1$ siempre contiene a un segmento
% 	$ S_0 := \{0\} \times (1-r, 1]$ y a otro segmento $ S_n := \{1/n\} \times (f(n),1]$ con $f(n) := \sqrt{r^2 - 1/n^2}$.
% 	Nótese que el siguiente segmento está a distancia horizontal
% 	$$ \frac{1}{n} - \frac{1}{n+1} = \frac{1}{n(n+1)}, \quad \delta_n := \frac{1}{2n(n+1)} $$
% 	Así pues $( 1/n - \delta_n, 1/n + \delta_n ) \times (f(n)/2, 2) \cap P$ es un abierto que contiene a $S_n$, pero no contiene puntos de $S_0$.
% 	Mientras que $(-1/2n, 1/2n) \times ( \frac{1-r}{2}, 2 )$ es un abierto que contiene a $S_0$, pero es disjunto de $S_n$.
% \end{exn}

\section{Límites inversos y espacios disconexos}
\begin{mydef}
	Se dice que un conjunto parcialmente ordenado $(I, \le)$ es un \strong{conjunto dirigido}\index{conjunto!dirigido} si para todo $i, j \in I$
	existe $k \in I$ tal que $i \le k$ y $j \le k$.

	Dado un conjunto ordenado dirigido $(I, \le)$, un \strong{sistema inverso}\index{sistema inverso}%
	\footnote{\citeauthor{engelking:top}~\cite{engelking:top} les llama \textit{sistema inverso}, mientras que
	\citeauthor{bourbaki:TG1}~\cite[I.28]{bourbaki:TG1} les llama \textit{sistema proyectivo}.}
	es una familia $\{ X_i \}_{i\in I}$ de espacios topológicos con funciones continuas $\varphi^i_j \colon X_i \to X_j$ para $i \ge j$, tal que:
	\begin{enumerate}
		\item $\varphi^i_i = \Id_{X_i}$.
		\item $\varphi^i_j \circ \varphi^j_k = \varphi^i_k$ cuando $i \ge j$ y $j \ge k$.
	\end{enumerate}
	De manera compacta, denotaremos que $(X_\bullet, \varphi^\bullet_\bullet, I)$ o que $\{ X_i \}_{i\in I}$ es el sistema inverso.
\end{mydef}
El ejemplo prototípico de un conjunto dirigido es $(\N, \le)$.
Secretamente los sistemas inversos capturan la noción de un diagrama filtrado en $\mathsf{Top}$;
como ésta categoría es completa, entonces posee toda clase de límites inversos, pero estos serán más fáciles de calcular.

\begin{mydef}
	Sea $S := \{ X_i \}_{i\in I}$ un sistema inverso.
	Una \strong{hebra}\index{hebra}%
	\footnote{eng. \textit{thread}.}
	es una tupla $(x_i)_{i\in I}$ de elementos $x_i \in X_i$ tales que $\varphi^i_j(x_i) = x_j$.
	Denotamos por $\invlim_{i\in I} X_i$ al subespacio de todas las hebras en $\prod_{i\in I} X_i$, llamado el \strong{límite inverso}%
	\index{limite@límite!inverso} del sistema $S$.

	También decimos que $S$ es el \strong{límite cofiltrado}\index{limite@límite!cofiltrado}%
	\footnote{Esto se debe a que en la teoría de categorías existe una noción de <<categoría filtrada>> y lo que llamamos un \textit{sistema inverso}
	es un diagrama contravariante filtrado.}
	de los espacios $\{ X_i \}_{i\in I}$; esto será útil para sentencias de la forma <<el límite cofiltrado de $\mathcal{P}$ es $\mathcal{P}$>>.
\end{mydef}

El siguiente ejemplo, fundamental el teoría de números, da una buena imagen:
\begin{exn}\label{ex:p-adics}
	Considere los anillos $A_n := \Z/p^n\Z$.
	Entonces la reducción $\bmod{p^n}$ induce un homomorfismo de anillos $\varphi^{n+1}_n\colon A_{n+1} \to A_n$.
	Para hacer que $(A_\bullet, \varphi^\bullet_\bullet)$ sea un sistema inverso dotamos a cada $A_n$ con la topología discreta
	(que es la única topología Hausdorff posible), y así, denotamos $\Z_p := \liminv_n \Z/p^n\Z$.
	% https://q.uiver.app/#q=WzAsNSxbMiwxLCJcXFovcF4yXFxaIl0sWzMsMSwiXFxaL3BcXFoiXSxbMSwxLCJcXFovcF4zXFxaIl0sWzIsMCwiXFxaX3AiXSxbMCwxLCJcXGNkb3RzIl0sWzIsMCwiIiwyLHsic3R5bGUiOnsiaGVhZCI6eyJuYW1lIjoiZXBpIn19fV0sWzAsMSwiIiwyLHsic3R5bGUiOnsiaGVhZCI6eyJuYW1lIjoiZXBpIn19fV0sWzQsMiwiIiwyLHsic3R5bGUiOnsiaGVhZCI6eyJuYW1lIjoiZXBpIn19fV0sWzMsMl0sWzMsMF0sWzMsMV0sWzMsNF1d
	\[\begin{tikzcd}
		&& {\Z_p} \\
		\cdots & {\Z/p^3\Z} & {\Z/p^2\Z} & {\Z/p\Z}
		\arrow[two heads, from=2-2, to=2-3]
		\arrow[two heads, from=2-3, to=2-4]
		\arrow[two heads, from=2-1, to=2-2]
		\arrow[from=1-3, to=2-2]
		\arrow[from=1-3, to=2-3]
		\arrow[from=1-3, to=2-4]
		\arrow[from=1-3, to=2-1]
	\end{tikzcd}\]
	A los elementos de $\Z_p$ se le denominan \textit{números $p$-ádicos}.%
	\footnote{En realidad, a $\Z_p$ se le otorga un carácter dual. Por un lado ha de ser un espacio topológico, pero también un anillo.}
\end{exn}

\begin{prop}
	Sea $(X_\bullet, \varphi^\bullet_\bullet, I)$ un sistema inverso y sea $X$ su límite inverso.
	\begin{enumerate}
		\item Para cada $i \in I$, existe una función continua $p_i \colon X \to X_i$
			tal que para cada $j \le i$ se cumple que $p_i \circ \varphi^i_j = p_j$.
		\item Si $Y$ es un espacio topológico con una familia de funciones continuas $\{ q_i \colon Y \to X_i \}_{i\in I}$
			tales que $q_i \circ \varphi^i_j = q_j$ para todo $j \le i$,
			entonces existe una única función continua $f \colon Y \to X$ tal que el siguiente diagrama conmuta:
			\begin{center}
				\includegraphics{cats/inverse_system_top.pdf}
			\end{center}
	\end{enumerate}
\end{prop}
\begin{proof}
	\begin{enumerate}
		\item Basta definir $p_i := \pi_i|_X$ donde $\pi_i \colon \prod_{j\in I} X_j \to X_i$ es la proyección a la $i$-ésima coordenada.
		\item Basta definir $f(y) := \big( q_i(y) \big)_{i\in I}$ y notar que la unicidad se sigue de la definición de $X$.
			\qedhere
	\end{enumerate}
\end{proof}
La proposición anterior, que no es más que un ejercicio, pero nos dice que la notación coincide con la interpretación categórica.

\begin{exn}
	Sea $\{ Y_\alpha \}_{\alpha\in S}$ una familia de espacios topológicos.
	Definamos $I$ como la familia de subconjuntos finitos de $S$ con el orden parcial dado por la inclusión.
	Definamos $X_i := \prod_{\alpha \in i} Y_\alpha$ para $i \in I$ y definamos $\varphi^i_j \colon X_i \to X_j$ como la proyección canónica
	(nótese que $X_i$ <<tiene más coordenadas>> que $X_j$, de modo que $\varphi^i_j$ <<borra las coordenadas adicionales>>).
	Mediante la propiedad universal de la proposición anterior es fácil comprobar que
	\begin{equation}
		\invlim_i X_i = \prod_{\alpha \in S} Y_\alpha.
		\tqedhere
	\end{equation}
\end{exn}

\begin{prop}
	El límite cofiltrado de espacios $T_i$ es $T_i$ con $i \le 3.5$.
\end{prop}

\begin{prop}
	El límite $\invlim S$ de un sistema inverso $S = \{ X_i \}_{i\in I}$ de espacios de Hausdorff
	es un cerrado en $\prod_{i\in I} X_i$.
\end{prop}
\begin{proof}
	Para todo $j \le i$ defínamos:
	$$ F_{ij} := \left\{ (x_k)_k \in \prod_{k\in I} X_k : \varphi^i_j(x_i) = x_j \right\}, $$
	nótese que es un cerrado pues los espacios son Hausdorff, y $\invlim S = \bigcap_{i, j} F_{ij}$ de modo que es cerrado.
\end{proof}
\thmdep{AE}
\begin{cor}
	Sea $(X_\bullet, \varphi_\bullet^\bullet, I)$ un sistema inverso de espacios compactos de Hausdorff
	y sea $X := \invlim_{i\in I} X_i$ con $f_i \colon X \to X_i$. Entonces:
	\begin{enumerate}
		\item $X$ es compacto y para todo $i \in I$:
			\begin{equation}
				f_i[X] = \bigcap_{j\ge i} \varphi^j_i[X_j].
				\label{eqn:proj_syst_incl}
			\end{equation}
		\item Si cada $X_i$ es no vacío, entonces $X$ no es vacío.
	\end{enumerate}
	% El límite de un sistema inverso de espacios compactos de Hausdorff (no vacíos) es compacto de Hausdorff (no vacío).
\end{cor}
% \begin{proof}
% 	Lo único no trivial es la igualdad \eqref{eqn:proj_syst_incl}.
% \end{proof}
Por ejemplo, $\Z_p$ debe ser entonces un espacio de Hausdorff compacto.

\begin{cor}
	Sea $(X_\bullet, \varphi_\bullet^\bullet, I)$ un sistema inverso tal que para todo $x_i \in X_i$ y todo $j \le i$,
	la fibra $(\varphi_i^j)^{-1}[\{ x_i \}]$ es compacto y Hausdorff.
	Entonces la relación \eqref{eqn:proj_syst_incl} se satisface y las fibras $p_i^{-1}[\{ x_i \}]$ son compactas Hausdorff.
\end{cor}
\begin{proof}
	Fijemos $i \in I$ y para todo $x_i \in \bigcap_{j\ge i} \varphi^j_i[X_j] \subseteq X_i$, definamos $F_j := (\varphi_i^j)^{-1}[\{ x_i \}]$.
	Para todo $i \le j \le k$ se tiene que $ \varphi^k_j[F_k] \subseteq F_j$ por la transitividad de los índices,
	de modo que $(F_\bullet, \varphi^\bullet_\bullet, \{ j \in I : j \ge i \})$ constituye un sistema inverso cuyo límite es $p_i^{-1}[\{ x_i \}]$.
\end{proof}

\begin{cor}\label{thm:proj_lim_surj_maps}
	Sea $I$ un conjunto dirigido y $(X_\bullet, \varphi_\bullet^\bullet), (Y_\bullet, \psi_\bullet^\bullet)$ un par de sistemas inversos
	de espacios de Hausdorff.
	Sea $\{ f_i \colon X_i \to Y_i \}_{i\in I}$ una familia de funciones continuas tal que para todo $j \le i$, el siguiente diagrama conmuta:
	% https://q.uiver.app/#q=WzAsNCxbMCwwLCJYX2kiXSxbMCwxLCJYX2oiXSxbMSwwLCJZX2kiXSxbMSwxLCJZX2oiXSxbMiwzLCJcXHBzaV5pX2oiXSxbMCwxLCJcXHZhcnBoaV5pX2oiLDJdLFswLDIsImZfaSJdLFsxLDMsImZfaiIsMl1d
	\[\begin{tikzcd}
		{X_i} & {Y_i} \\
		{X_j} & {Y_j}
		\arrow["{\psi^i_j}", from=1-2, to=2-2]
		\arrow["{\varphi^i_j}"', from=1-1, to=2-1]
		\arrow["{f_i}", from=1-1, to=1-2]
		\arrow["{f_j}"', from=2-1, to=2-2]
	\end{tikzcd}\]
	Sea $X := \invlim_i X_i$, $Y := \invlim_i Y_i$ y $f := \invlim_i f_i \colon X \to Y$. Entonces:
	\begin{enumerate}
		\item Si para cada $\vec y := (y_i)_i \in Y$ se cumple que $f_i^{-1}[\{ y_i \}]$ es compacto y no vacío,
			entonces $f^{-1}[\{ \vec y \}]$ es compacto y no vacío.
		\item Si cada $X_i$ es compacto y cada $f_i$ es suprayectivo, entonces $f$ también es suprayectivo y, en consecuente, $Y$ es compacto.
	\end{enumerate}
\end{cor}
\thmdep{}

\begin{mydef}
	Se dice que un espacio es:
	\begin{description}
		\item[Hereditariamente disconexo]\index{hereditariamente disconexo (espacio)}\index{espacio!hereditariamente disconexo}
			Si todo subespacio de cardinalidad ${}>1$ es disconexo.%
			\footnote{Ésta es terminología de \citeauthor{engelking:top}~\cite{engelking:top}.
			Desgraciadamente, la mayoría de autores (e.g., \citeauthor{bourbaki:TG1}~\cite[I.83]{bourbaki:TG1}) opta por \textit{totalmente
			disconexo} o \textit{discontinuo}, la cual es relativamente conflictivo.}
		\item[Cerodimensional]\index{espacio!cerodimensional}\index{cerodimensional (espacio)}
			Si es no vacío, es $T_1$ y todo punto posee una base de entornos abiertos-cerrados.
		\item[Fuertemente cerodimensional]\index{espacio!fuertemente cerodimensional}\index{fuertemente cerodimensional (espacio)}
			Si es no vacío, de Tychonoff y si para todo par de conjuntos $A,B$
			completamente separados existe un abierto-cerrado $C$ tal que $A\subseteq C \subseteq B^c$.
	\end{description}
\end{mydef}
Todo espacio discreto es fuertemente cerodimensional y extremadamente disconexo.
% Como ejercicio, demuestre que $\Q$ también lo es.
\begin{prob}
	Demuestre que $\Q$ y $\R \setminus \Q$ son hereditariamente disconexos y cerodimensionales.
	% pero no son extremadamente disconexos.
\end{prob}

El siguiente corolario es claro:
\begin{cor}
	Las propiedades de ser hereditariamente disconexo o cerodimensional son hereditarias.
\end{cor}
\begin{thm}
	Se cumple que:
	\begin{enumerate}
		\item Un espacio es hereditariamente disconexo syss sus componentes conexas son singulares.
		\item En un espacio cerodimensional las cuasi-componentes son singulares, luego es hereditariamente disconexo.
		\item Todo espacio fuertemente cerodimensional es cerodimensional.
	\end{enumerate}
\end{thm}

\begin{ex}
	El espacio $X := \{0\} \cup \{1/n : n \in \N_{\ne 0}\}$ es hereditariamente disconexo:
	En efecto, sea $C \subseteq X$ conexo no vacío y supongamos que $1/n \in C$, nótese que $\{1/n\}$ es un conjunto abierto-cerrado de $X$,
	luego como $C$ es conexo se ha de cumplir que $C = \{1/n\}$.
	Si $1/n \notin C$ para todo $n$, entonces $C = \{0\}$.
	% También es fácil comprobar que $X$ es extremadamente disconexo.
\end{ex}

\begin{thm}
	Para un espacio de Tychonoff no vacío $X$ son equivalentes:
	\begin{enumerate}
		\item $X$ es fuertemente cerodimensional.
		\item Todo cubrimiento finito por coceros $\{ U_i \}_{i=1}^k$ admite un refinamiento finito por abiertos $\{ V_i \}_{i=1}^m$
			tal que $V_i \cap V_j = \emptyset$ para $i \ne j$.
	\end{enumerate}
\end{thm}
\begin{proof}
	$2 \implies 1$. Sea $f \colon X \to [0, 1]$ tal que $f[A] \subseteq \{ 0 \}$ y $f[B] \subseteq \{ 1 \}$.
	Los conjuntos $U_1 := f^{-1}\big[ (0, 1] \big]$ y $U_2 := f^{-1}\big[ [0, 1) \big]$ constituye un cubrimiento por coceros,
	de modo que admiten un refinamiento por abiertos disjuntos $\{ V_i \}_{i=1}^m$.
	Si elegimos los $V_i$'s de modo que exactamente $V_1, V_2, \dots, V_n$ cortan a $A$ y $V_{n+1}, \dots, V_m$ cortan a $B$,
	entonces $U := V_1 \cup \cdots \cup V_n$ es un abierto-cerrado tal que $A \subseteq U \subseteq B^c$.

	$1 \implies 2$. Basta proceder por inducción sobre $k$.
	El caso base $k = 1$ es trivial.
	Veamos el caso general: sea $\{ U_i \}_{i=1}^k$ un cubrimiento por coceros, y definamos $U_j^\prime := U_j$ para $j < k-1$ y
	$U_{k-1}^\prime := U_{k-1} \cup U_k$.
	Por hipótesis inductiva, existe un refinamiento finito por abiertos y cerrados disjuntos $\{ W_i \}_{i=1}^M$;
	es fácil modificarlo de modo que $M = k-1$ y $W_j \subseteq U_j$ para cada $j < k$.

	Los conjuntos $W_{k-1} \setminus U_{k-1}$ y $W_{k-1} \setminus U_k$ son ceros, de modo que están completamente separados
	por el teorema~\ref{thm:func-closed-perf-sep}.
	Luego, existe un conjunto abierto y cerrado $G$ tal que
	\[
		W_{k-1} \setminus U_{k-1} \subseteq G \subseteq X \setminus (W_{k-1} \setminus U_k) = W_{k-1}^c \cup U_k.
	\]
	Así pues, $V_{k-1} := W_{k-1} \setminus U \subseteq U_{k-1}$ y $V_k := W_{k-1} \cap U \subseteq U_k$ son abiertos y cerrados.
	Rellenando con $V_i := W_i$ para $i < k-1$ se puede verificar que $\{ V_i \}_{i=1}^{k-1}$ satisface lo exigido.
\end{proof}
% \begin{proof}
% \end{proof}

\begin{thm}
	Todo espacio cerodimensional y de Lindelöf (e.g., compacto) es fuertemente cerodimensional.
\end{thm}
\begin{proof}
	Sean $A, B$ un par de cerrados disjuntos.
	Para cada punto $x \in X$ elíjase un abierto y cerrado $x \in W_x \subseteq X$ tal que $A \cap W_x = \emptyset$ o $B \cap W_x = \emptyset$.
	Como el espacio es Lindelöf, sea $\{ W_{x_n} \}_{n\in\N}$ un subcubrimiento numerable, y defínanse los conjuntos
	$$ U_i := W_{x_i} \setminus \bigcup_{j<i} W_{x_j} \qquad i\in\N. $$
	Entonces $\{ U_i \}_{i\in\N}$ es un cubrimiento de $X$ por abiertos y cerrados disjuntos dos a dos, y el conjunto
	$U := \bigcup \{ U_i : A \cap U_i = \emptyset \}$ satisface que $A \subseteq U \subseteq B^c$.
\end{proof}

\begin{cor}
	Todo espacio no vacío numerable y regular es fuertemente cerodimensional.
\end{cor}
\begin{proof}
	Como $X$ es numerable, entonces es normal y es Lindelöf.
	Basta probar que $X$ es cerodimensional, es decir, que $X$ posee una base de abiertos-cerrados.
	Sea $x \in X$ un punto y $V$ un entorno de $x$.
	Por el lema de Urysohn, existe $f \colon X \to [0, 1]$ tal que $f(x) = 0$ y $f[X \setminus V] = \{ 1 \}$.
	Como $X$ es numerable, entonces su imagen también lo es, luego existe $r \in [0, 1] \setminus \Img f$.
	Así, $U := f^{-1}\big[ [0, r) \big] = f^{-1}\big[ [0, 1] \big]$ es un subentorno abierto y cerrado de $V$ que contiene a $x$.
\end{proof}

\begin{thm}
	Todo espacio no vacío, $T_1$, heriditariamente disconexo y localmente compacto es cerodimensional.
\end{thm}
\begin{proof}
	% Como la definición de cerodimensional es local, podemos suponer que el espacio $X$ es compacto.
	Sea $x \in X$ un punto, $V$ un entorno de $x$ y sea $W \subseteq V$ un subentorno de $x$ tal que $\overline{W}$ es compacto.
	Como la componente conexa de $x$ en $\overline{W}$ es $\{ x \}$ y como las componentes conexas y cuasicomponentes coinciden,
	entonces, denotando por $\mathcal{K}$ a la familia de abiertos-cerrados que contienen a $x$ en el subespacio $\overline{W}$,
	se cumple que $\{ x \} = \bigcap \mathcal{K}$.
	Por la propiedad de las intersecciones finitas, tenemos que existen finitos $F_1, \dots, F_k \in \mathcal{K}$
	tales que $x \in U := F_1 \cap F_2 \cap \cdots \cap F_k \subseteq W \subseteq V$.
	Nótese que $U$ es cerrado en $\overline{W}$, luego es cerrado también en $X$, y $U$ es abierto en $W$, luego también es abierto en $X$.
	% tales que $\{ x \} = F_1 \cap F_2$
\end{proof}

\begin{cor}
	Sea $X$ un espacio topológico no vacío localmente compacto y paracompacto (e.g., compacto y Hausdorff). Son equivalentes:
	\begin{enumerate}
		\item $X$ es hereditariamente disconexo.
		\item $X$ es cerodimensional.
		\item $X$ es fuertemente cerodimensional.
	\end{enumerate}
\end{cor}

\begin{thm}
	El producto de espacios no vacíos hereditariamente disconexos (resp. cerodimensionales) es hereditariamente disconexo (resp. cerodimensional).

	En consecuencia, el límite cofiltrado de espacios hereditariamente disconexos (resp. cerodimensionales) es
	hereditariamente disconexo (resp. cerodimensional o vacío).
\end{thm}
% \begin{exn}
% 	Sea $\{ X_j \}_{j\in J}$ una familia de espacios topológicos (no un sistema inverso).
% 	Si definimos $\mathcal{P}$ como la familia de subconjuntos finitos de $J$
% \end{exn}
\thmdep{AE}
\begin{thm}
	Para un espacio topológico $X$ no vacío, son equivalentes:
	\begin{enumerate}
		\item $X$ es el límite cofiltrado de espacios discretos finitos.
		\item $X$ es Hausdorff, compacto y hereditariamente disconexo.
		\item $X$ es Hausdorff, compacto y cerodimensional.
	\end{enumerate}
\end{thm}
\begin{proof}
	Ya vimos que $1 \implies 2 \iff 3$.

	$3 \implies 1$. Sea $X$ Hausdorff, compacto y cerodimensional.
	Sea $\mathcal{R}$ el conjunto de todas las relaciones de equivalencias $R \subseteq X \times X$ que son conjuntos abiertos.
	Para $R \in \mathcal{R}$, nótese que el espacio cociente $X/R$ es finito y discreto; esto pues cada clase $[x]_R$ es abierta
	y $X = \bigcup_{x\in X} [x]$, de modo que solo hay finitas clases de equivalencia.
	% \todo{Revisar el por qué. Claramente es compacto}
	Para $R_1, R_2 \in \mathcal{R}$ denotamos $R_1 \preceq R_2$ syss para todo $x \in X$ se tiene que $[x]_{R_1} \supseteq [x]_{R_2}$
	(o equivalentemente, $R_1 \supseteq R_2$).
	Entonces $(\mathcal{R}, \preceq)$ es un conjunto dirigido, pues para $R_1, R_2 \in \mathcal{R}$ se cumple que $R_i \preceq R_1 \cap R_2$.
	Ahora bien, si $R \preceq S$ entonces $\varphi_{RS} \colon X/R \to X/S$ mandando $\varphi_{RS}([x]_R) = [x]_S$.

	Finalmente, sea $Y := \liminv_{R \in \mathcal{R}} X/R$, entonces como las proyecciones canónicas $\pi_R \colon X \to X/R$ son compatibles
	con el sistema inverso, tenemos una aplicación $\psi \colon X \to Y$ canónica.
	Como las proyecciones son suprayectivas, entonces $\psi$ también por el corolario~\ref{thm:proj_lim_surj_maps}, y como $Y$ es Hausdorff y $X$
	es compacto, entonces basta ver que $\psi$ es inyectiva.
	Sean $x, y \in X$, entonces como $X$ es cerodimensional, existe $U$ abierto y cerrado tal que $x \in U$ e $y \notin U$.
	Definiendo $R$ como la relación que identifica todo el conjunto $U$ y todo el conjunto $X \setminus U$, entonces vemos que
	$\pi_R(x) \ne \pi_X(y)$, de modo que $\psi(x) \ne \psi(y)$ como se quería probar.
\end{proof}
\begin{mydef}
	Se dice que un espacio topológico es \strong{de Stone}\index{espacio!de Stone}%
	\footnote{También llamados \textit{espacios profinitos} o \textit{espacios booleanos}.}
	si satisface alguna de las condiciones del teorema anterior.
\end{mydef}

\begin{cor}
	Un espacio de Stone $X$ es 2AN syss $X = \invlim_{i \in I} X_i$ donde $\{ X_i \}_{i\in I}$ es un sistema inverso cuyos índices $I$ son numerables.
\end{cor}
\begin{proof}
	$\impliedby$. Basta notar que $X$ es un subespacio de $\prod_{i\in I} X_i$, el cual es de Stone y 2AN.

	$\implies$. Sea $\mathcal{R}$ el conjunto de relaciones de equivalencia abiertas en $X$ como en la demostración anterior.
	Entonces, para todo $x \in X$ se nota que su clase $[x] \subseteq X \times X$ es un abierto, y como $X = \bigcup_{x\in X} [x]$,
	entonces tiene que haber finitas clases, cada una siendo cerrada y abierta, luego compacta, de modo que es una unión de finitos elementos de la base.
	Así pues, podemos notar que $\mathcal{R}$ es numerable, lo que basta siguiendo la demostración anterior.
\end{proof}
\thmdep{}

\begin{exn}[el espacio <<escoba>> reducido]
	\addnamedexample{Espacio <<escoba>> reducido}{Un espacio conexo, pero no arcoconexo}
	Sea
	$$ X := \{ (x, x/n) : x\in[0, 1], n\in\N_{\ne 0} \} \cup \{ (1, 0) \}. $$
	\begin{figure}[!hbt]
		\centering
		\includegraphics[scale=1]{deleted_broom.pdf}
		\caption{Espacio <<escoba>> reducida.}%
		\label{fig:deleted_broom}
	\end{figure}

	\begin{enumerate}
		\item \underline{$X$ es conexo:}
			Claramente $X \setminus \{(1, 0)\}$ es conexo, dado que es la unión de conexos (las <<hebras>> $\{(x, x/n) : x\in[0,1]\}$ con un $n$ fijo)
			que tienen un punto en común, el $(0,0)$.
			Luego, basta probar que el conjunto $\{(1, 0)\}$ no está separado del resto,
			para ello nótese que todo entorno $ B_\epsilon\big( (1, 0) \big) \cap X $ contiene un punto de la forma $(1, 1/n)$ para algún $n < \epsilon$,
			luego $(1, 0)$ es un punto de acumulación de $X \setminus \{(1, 0)\}$ y sanseacabó.

		\item \underline{$X$ no es arcoconexo:}
			Sea $p\colon [0, 1] \to X$ un arco tal que $p(0) = (1, 0)$, queremos probar que $p(t) = (1, 0)$ para todo $t$.
			Luego definamos
			$$ C := \{ t\in[0, 1] : p(t) = (1, 0) \}. $$
			Nótese que $C = p^{-1}\big[ \{(1, 0)\} \big]$, luego es cerrado.

			Veamos que también es abierto:
			Sea $t_0 \in C$, como $p$ es continuo, entonces existe un $\delta > 0$ tal que si $|t - t_0| < \delta$, entonces
			$\|p(t) - p(t_0)\| < 1/2$.
			Como $\|(0, 0) - (1, 0)\| = 1 > 1/2$, entonces $p(t) \ne (0, 0)$ y por lo tanto tiene coordenada en $x$ no nula.
			Ahora, definimos la función
			\begin{align*}
				m \colon \R_{>0}\times\R &\longrightarrow \R \\
				(x, y) &\longmapsto y/x
			\end{align*}
			que es claramente continua.
			Notemos que $p\circ m$ toma valores en $Z := \{0\}\cup\{1/n : n>0\}$.
			Sea $I := (t_0 - \delta, t_0 + \delta) \cap [0, 1]$, $I$ es claramente conexo en $[0, 1]$,
			luego $(p\circ m)[I]$ es conexo en $Z$, pero $Z$ es hereditariamente disconexo, luego ha de corresponder a un punto,
			que ha de ser el 0 (pues $(p\circ m)(t_0) = 0$ y $t_0 \in I$).
			Así que $t_0 \in I \subseteq C$, por lo que $C$ es entorno de todos sus puntos y por ende es abierto.

			Finalmente, como $C$ es un abierto y cerrado no vacío de $[0, 1]$, entonces ha de ser $[0, 1]$.
			Luego todo arco desde $(1, 0)$ es constante, así que $X$ no puede ser arcoconexo. \qedhere
	\end{enumerate}
\end{exn}

Otro ejemplo son los siguientes:
\begin{ex}
	Sea
	$$ X := \{(1/n, y) : y\in [0,1]\} \cup \{0\}\times[0,1] \cup [0,1]\times\{0\}, \quad Y := X \setminus \{0\} \times (0, 1) $$
	A $X$ se le conoce como el \strong{peine del topólogo} y a $Y$ como el \textit{peine reducido del topólogo}.
	\addnamedexample{Peine del topólogo}{Un espacio arcoconexo no localmente conexo}

	\begin{figure}[!hbt]
		\centering
		\includegraphics[scale=1]{deleted_comb.pdf}
		\caption{Peine reducido del topólogo.}%
		\label{fig:deleted_comb}
	\end{figure}
	\begin{enumerate}
		\item \underline{$X, Y$ son conexos:}
			Es claro que $X$ lo es, puesto que las <<hebras>> (los conjuntos de la forma $\{(x, y) : y\in[0,1]\}$ para $x = 1/n$ o $x = 0$)
			son conexas, el <<mango>> (el conjunto $[0, 1]\times\{0\}$) también y todas las hebras cortan el mango.
			El problema con $Y$ es que hay un punto, el $(0, 1)$, que no está en una hebra, sin embargo, es fácil notar que todo entorno del $(0, 1)$
			corta al resto del conjunto, que sí es conexo, por lo que $Y$ también lo es.

		\item \underline{$X$ es arcoconexo, pero $Y$ no:}
			Podemos notar que $X = \bigcup_{x = 1/n, x=0} H_x$ donde $H_x := \{x\}\times[0, 1] \cup [0,1]\times\{0\}$ es arcoconexo;
			luego $X$ es la unión de arcoconexos de intersección no vacía, por lo que $X$ es arcoconexo.

			Para demostrar que $Y$ no es arcoconexo, podemos emplear un argumento similar al del espacio <<escoba>> reducido,
			pero empleando dos funciones en lugar de una: una que proyecta sobre la coordenada $y$ para notar que todo arco siempre
			tiene coordenada $y = 1$ y una segunda que proyecta sobre la coordenada $x$
			empleando la disconexión hereditaria de $\{0\} \cup \{1/n: n>0\}$.

		\item \underline{$X, Y$ no son localmente conexos:} 
			Nótese que $B_r\big( (0, 1) \big) \cap P$ con $r < 1$ siempre contiene a un segmento
			$ S_0 := \{0\} \times (1-r, 1]$ y a otro segmento $ S_n := \{1/n\} \times (f(n),1]$ con $f(n) := \sqrt{r^2 - 1/n^2}$.
			Nótese que el siguiente segmento está a distancia horizontal
			$$ \frac{1}{n} - \frac{1}{n+1} = \frac{1}{n(n+1)}, \quad \delta_n := \frac{1}{2n(n+1)} $$
			Así pues $( 1/n - \delta_n, 1/n + \delta_n ) \times (f(n)/2, 2) \cap P$ es un abierto que contiene a $S_n$, pero no contiene puntos de $S_0$.
			Mientras que $(-1/2n, 1/2n) \times ( \frac{1-r}{2}, 2 )$ es un abierto que contiene a $S_0$, pero es disjunto de $S_n$.
		% \item \underline{$X, Y$ no son localmente conexos:} 
		% 	Para ello hay que probar que toda base de entornos de $(0, 1)$ no son arcoconexos.
		% 	Por el argumento anterior, es claro que $(0, 1)$ no posee entornos arcoconexos en $Y$,
		% 	y los entornos arcoconexos de $(0, 1)$ en $X$ siempre deben tener a la hebra completa $\{0\} \times [0, 1]$,
		% 	pero toda base de entornos de $(0, 1)$ posee entornos que no contienen a la hebra completa, ergo, contienen entornos no arcoconexos.
	\end{enumerate}
\end{ex}

Una característica común al espacio escoba reducido y al peine reducido es que son espacios conexos, no arcoconexos,
pero cuyas clausuras (en $\R^2$) son arcoconexas.
Topológicamente ésto puede verse como que no son compactos (demuéstrelo).
Sin embargo, ésto no ocurre para el seno del topólogo, que es conexo, no arcoconexo, y cuya clausura (que es compacta) tampoco es arcoconexa.

\subsection{Objetos proyectivos}
En ésta subsección seguimos el artículo de \citeauthor{gleason58projective}~\cite{gleason58projective}.
% \addtocategory{article}{gleason58projective}
En una primera lectura es opcional, pero los resultados de Gleason han adquirido una nueva luz en la teoría de objetos condensados de Clausen-Scholze.

\begin{mydef}
	Se dice que un espacio $X$ es \strong{extremadamente disconexo}\index{extremadamente disconexo (espacio)}\index{espacio!extremadamente disconexo}
	si es de Hausdorff y la clausura de todo abierto es abierta.
\end{mydef}
\begin{thm}
	Todo espacio extremadamente disconexo es hereditariamente disconexo.
\end{thm}
% El espacio $\{ 1/n : n \in \N \} \cup \{ 0 \}$ es extremadamente disconexo por ejemplificar.
No obstante, la mayoría de ejemplos a los que nos hemos enfrentado no lo son: queda al lector verificar que ni $\Q, \R \setminus \Q$
ni $\Z_p$ son extremadamente disconexos.

La razón, como veremos en ésta sección, tiene que ver con el hecho de que los espacios extremadamente disconexos resultan increíblemente restrictivos.
\begin{mydef}
	Una \strong{categoría de Gleason}\index{categoría!de Gleason} es una subcategoría $\catC \subseteq \mathsf{Haus}$ de espacios de Hausdorff
	tal que:
	\begin{enumerate}
		% \item Dados $X, Y \in \Obj\catC$ toda función $X \to Y$ es continua.
		\item Dado $X \in \Obj\catC$, entonces $X \times D(2) \in \Obj\catC$ y cada proyección $\pi_i \colon X\times D(2) \to X$
			está en $\Morf\catC$.
		\item Sea $X \in \Obj\catC$ y $F \subseteq X$ un subespacio cerrado.
			Entonces $F \hookto X \in \Morf\catC$.
	\end{enumerate}
\end{mydef}
Varias subcategorías son de Gleason.
Claramente $\mathsf{Haus}, \mathsf{KHaus}$ y, más importante, $\mathsf{Stone}$ son de Gleason.

\begin{thm}
	Sea $\catC$ una categoría de Gleason. Entonces los objetos proyectivos de $\catC$ son espacios extremadamente disconexos.
\end{thm}
\begin{proof}
	Sea $X$ un objeto proyectivo de $\catC$ y sea $U \subseteq X$ un abierto.
	Sea $D(2) = \{ 0, 1 \}$ y en $X \times D(2)$ considere el cerrado $Y := (X \setminus U) \times \{ 0 \} \cup \overline{U} \times \{ 1 \}$
	y su inclusión $i \colon Y \to X\times D(2)$.
	Sea $\pi \colon X\times D(2) \to X$ la proyección, de modo que $i\circ\pi \in \Morf\catC$ y es suprayectivo (luego un epimorfismo).
	Como $X$ es proyectivo, tomando $\Id_X \colon X \to X$ vemos que existe $g \colon X \to Y$ tal que el siguiente diagrama conmuta:
	\begin{center}
		\begin{tikzcd}[row sep=large]
			{} & X \dar[equals] \dlar["g"', dashed] \\
			Y \rar["i\circ\pi"', two heads] & X
		\end{tikzcd}
	\end{center}
	Nótese que $i\circ\pi|_{U\times\{ 1 \}} \colon U\times\{ 1 \} \to U$ es biyectivo, de modo que $g(x) = (x, 1)$ para $x \in U$ y,
	por el corolario~\ref{thm:hausdorff_dense_extension} vemos que $g(x) = (x, 1)$ para $x \in \overline{U}$.
	Luego $g^{-1}[ \overline{U}\times\{ 1 \} ] = \overline{U}$, pero $\overline{U}\times\{ 1 \}$ es abierto en $Y$,
	de modo que $\overline{U}$ es abierto.
\end{proof}
% Vale aclarar, un objeto proyectivo $X$ de una categoría $\catA$ es aquel tal que $\Hom_\catA(X, -)$ res

\begin{thm}
	Sean $X$ un espacio extremadamente disconexo. Entonces:
	\begin{enumerate}
		\item Toda sucesión convergente en $X$ es eventualmente constante.
		\item En consecuencia, si $X$ es 1AN, entonces es discreto.
	\end{enumerate}
\end{thm}
\begin{proof}
	\begin{enumerate}
		\item Sea $(x_n)_{n\in\N}$ una sucesión que converge a $y$ y supongamos, por contradicción, que no es eventualmente constante.
			% Pasando a una subsucesión podemos suponer que los términos son todos distintos entre sí y distintos de $y$.

			Construyamos recursivamente una subsucesión de $(x_n)_n$ y una sucesión de abiertos $(U_k)_k$.
			El abierto $U_0$ contiene al primer $x_{\sigma(0)} \ne y$ e $y \notin \overline{U}_0$.
			% Luego, podemos construir recursivamente una sucesión de abiertos $\{ U_j \}_{j\in\N}$ tales que $x_n \in U_n$,
			% los abiertos son disjuntos e $y \notin U_n$ para cada $n$.
			Construido $U_n$, definamos $V := X \setminus ( \overline{U}_1 \cup \cdots \cup \overline{U}_n )$ el cual es un entorno de $y$,
			luego existe un $\sigma(n+1) \in \N$ mayor que $\sigma(n)$ tal que $x_{\sigma(n+1)} \in V$, $x_{\sigma(n+1)} \ne y$
			y sea $W$ un entorno de $x_{\sigma(n+1)}$ tal que $y \notin \overline{W}$.
			Definimos $U_{n+1} := W \cap V$.

			Definamos ahora $G := \bigcup_{n \in \N} U_{2n}$. Como $X$ es extremadamente disconexo, $\overline{G}$ es abierto
			y claramente $y \in \overline{G}$ (pues $\lim_n x_{\sigma(2n)} = y$), así que $\overline{G}$ es entorno de $y$
			y algún $x_{\sigma(2m+1)} \in \overline{G}$. Así que $U_{2m+1} \cap G \ne \emptyset$, lo cual es absurdo pues $U_{2m+1}$
			es disjunto del resto de $U_{2n}$'s.

		\item Basta notar que la clausura de un conjuntos en un espacio 1AN viene determinado por los límites de sucesiones,
			los cuales son eventualmente constantes, luego todo conjunto es cerrado. \qedhere
	\end{enumerate}
\end{proof}

\begin{lem}\label{thm:gleason_technical}
	Sea $\rho \colon X \epicto Y$ una función continua suprayectiva tal que para todo $F \subset X$ cerrado propio se tiene que $\rho[F] \ne Y$
	(e.g., $\rho$ biyectiva).
	Entonces, para todo abierto $U \subseteq X$ tenemos que
	$$ \rho[U] \subseteq \overline{Y \setminus \rho[X \setminus U]}. $$
\end{lem}
\begin{proof}
	Supongamos que $U \ne \emptyset$.
	Sea $y \in \rho[U]$ y sea $V$ un entorno de $y$. Basta ver que $V$ corta a $Y \setminus \rho[X \setminus U] =: S$.
	Como $U \cap \rho^{-1}[V] \subseteq X$ es un abierto no vacío, entonces $Y \ne \rho\big[ X \setminus (U \cap \rho^{-1}[V]) \big]$, así que,
	tómese $y' \in Y \setminus \rho\big[ X \setminus (U \cap \rho^{-1}[V]) \big]$.
	Como $\rho$ es suprayectiva, existe $x' \in X$ tal que $\rho(x') = y'$.
	Como $y' \in Y \setminus \rho[X \setminus U] = S$, necesariamente $x' \in U \cap \rho^{-1}[V]$ y,
	por tanto, $y' = \rho(x') \in \rho\big[ \rho^{-1}[V] \big] = V$, por lo que $y' \in V \cap S$ como se quería ver.
\end{proof}

\begin{lem}
	Si $Y$ es extremadamente disconexo y $U_1, U_2$ son un par de abiertos disjuntos, entonces $\overline{U}_1, \overline{U}_2$ también son disjuntos.
\end{lem}
\begin{proof}
	Como $U_2$ es abierto, entonces $\overline{U}_1, U_2$ son disjuntos.
	Análogamente como $\overline{U}_1$ es abierto, entonces $\overline{U}_1, \overline{U}_2$ son disjuntos.
\end{proof}

\begin{lem}
	Sean $X, Y$ un par de espacios de Hausdorff compactos con $Y$ extremadamente disconexo.
	Si $\rho \colon X \to Y$ es continua suprayectiva y para todo cerrado propio $F \subset Y$ tenemos que $\rho[F] \ne X$,
	entonces $\rho$ es un homeomorfismo.
\end{lem}
\begin{proof}
	Por el lema de la función cerrada, basta verificar que $\rho$ es inyectiva.
	Sean $x_1, x_2 \in X$ tales que $\rho(x_1) = \rho(x_2)$ y sean $U_1, U_2$ abiertos disjuntos con $x_i \in U_i$.
	Ambos $X \setminus U_i$ son compactos, luego $Y \setminus \rho[X \setminus U_i]$ son abiertos y son disjuntos (¿por qué?),
	luego $\overline{Y \setminus \rho[X \setminus U_i]}$ son abiertos disjuntos, pero por el lema~\ref{thm:gleason_technical} vemos que
	\[
		\rho(x_1) = \rho(x_2) \in \overline{Y \setminus \rho[X \setminus U_1]} \cap \overline{Y \setminus \rho[X \setminus U_2]},
	\]
	lo que es absurdo.
\end{proof}

\begin{lem}
	Sea $\pi\colon Z \epicto Y$ una función suprayectiva entre espacios de Hausdorff compactos.
	Entonces $Z$ contiene un compacto $K \subseteq Y$ tal que $\pi[K] = Y$, pero $\pi[F] \ne Y$ para todo $F \subset K$ cerrado propio.
\end{lem}
\begin{proof}
	Es una aplicación del lema de Zorn.
\end{proof}

\begin{thm}
	En las siguientes categorías:
	\begin{enumerate}[(a)]
		\item La subcategoría plena $\mathsf{KHaus}$ de espacios Hausdorff compactos.
		\item La subcategoría plena de espacios de Tychonoff (aquellos que poseen compactificaciones).
		\item La categoría de espacios Hausdorff localmente compactos cuyas flechas son las funciones propias.
	\end{enumerate}
	Los objetos proyectivos son precisamente los espacios extremadamente disconexos.
\end{thm}
\begin{proof}
	Claramente las categorías (a)-(c) son de Gleason, así que basta probar que todo espacio extremadamente disconexo es proyectivo.

	Realizaremos la primera, empleando los lemas anteriores. Para el resto hay que apropiadamente ver que los lemas tienen su análogo.
	Sea $Y \in \Obj\mathsf{KHaus}$ extremadamente disconexo y sean $f\colon A \epicto B$ una función continua suprayectiva con $f \in \Morf\mathsf{KHaus}$
	y $g \colon Y \to B$ una función continua.
	Hay que probar que existe $\bar f \colon Y \to A$ tal que $f = \bar f \circ g$.

	Considere $Z := \{ (y, a) \in Y \times A : g(y) = f(a) \}$ como subespacio de $Y\times A$.
	Como los espacios son Hausdorff, $Z$ es cerrado y, por tanto, compacto.
	Como $f$ es suprayectivo, entonces la proyección $\pi_1 \colon Z \epicto Y$ es suprayectiva.
	Así, existe $K \subseteq Z$ compacto tal que $\pi_1[K] = Y$ y $\pi_1[F] \ne Y$ para todo cerrado propio $F \subset K$.
	Luego $\rho := \pi_1|_K \colon K \to Y$ es un homeomorfismo y podemos definir $\bar f := \rho^{-1} \circ \pi_2$,
	donde $\pi_2 \colon Z \to A$ es la otra proyección.
\end{proof}

\begin{cor}
	Un espacio de Tychonoff es extremadamente disconexo syss su compactificación de \v Cech-Stone lo es.
\end{cor}
\begin{proof}
	Se sigue de la propiedad universal de los objetos proyectivos y la compactificación de \v Cech-Stone
\end{proof}
\begin{ex}
	Considere el subespacio $\N \subseteq \R$ (el cual es discreto), entonces $\beta\N$ es un espacio extremadamente disconexo y compacto.
\end{ex}

\begin{cor}
	Todo espacio de Hausdorff compacto es la imagen continua de un espacio extremadamente disconexo compacto.
	En resumen, la categoría $\mathsf{KHaus}$ tiene suficientes proyectivos.
\end{cor}
\begin{proof}
	Sea $X$ de Hausdorff compacto, entonces $D(X)$, el espacio discreto cuyo conjunto subyacente es $X$, induce una función continua suprayectiva
	$D(X) \epicto X$, donde $D(X)$ es extremadamente disconexo.
	Luego, por propiedad universal, esta función se factoriza por $\beta D(X) \epicto X$ la cual satisface lo exigido.
\end{proof}

\section{Espacios normados y espacios vectoriales topológicos}
Siguiendo a \citeauthor{bourbaki:evt}~\cite{bourbaki:evt} emplearemos la siguiente terminología:
\begin{mydef}
	Un \strong{cuerpo topológico}\index{cuerpo!topológico} $K$ es un cuerpo con una topología
	tal que las funciones ${+} \colon K\times K \to K$, ${\cdot} \colon K\times K \to K$ y $()^{-1} \colon K^\times \to K^\times$
	son continuas.
	Un \strong{cuerpo métrico}%
	\footnote{O \emph{cuerpo valuado}, dependiendo del autor.}
	es un par $(K, |\,|)$, donde $K$ es un cuerpo y $|\,| \colon K \to \R_{\ge 0}$ es una función valor absoluto.
\end{mydef}

En estricto rigor nos interesan los cuerpos métricos, no (topológicamente) discretos y completos (como $\R$ y $\C$),
pero la terminología anterior será de utilidad.
\begin{mydefi}
	Sea $K$ un cuerpo topológico.
	Un \strong{$K$-espacio vectorial topológico}\index{espacio!vectorial topológico (EVT)} (abrev., EVT) $V$ es un $K$-espacio vectorial
	dotado de una topología mediante la cual las aplicaciones ${+} \colon V\times V \to V$ y
	\[
		K\times V \longrightarrow V, \qquad (\lambda, x) \longmapsto \lambda x
	\]
	son continuas.
	Dados $V, W$ un par de $K$-EVTs, se denota por $L(V, W)$ (resp.\ $\dual_K(V, W)$) al conjunto de aplicaciones $K$-lineales (continuas) $V \to W$;
	de no haber ambigüedad se obvía el subíndice <<$K$>>.

	Dado un $K$-EVT $V$, una aplicación de $\dual(V, V)$ se dice un \strong{operador}\index{operador}.
	Los isomorfismos bicontinuos se dicen \strong{isomorfismo topológico}\index{isomorfismo!topológico}.%
	\footnote{Algunos libros le dicen \textit{isomorfismo topológico} a los homeomorfismos, en inglés también suele aparecer ésta definición
	como \textit{toplinear isomorphism}.}
\end{mydefi}
\begin{cor}
	Los $K$-EVTs (como objetos) con las transformaciones lineales continuas (como flechas) forman una categoría denotada $\mathsf{TVS}_K$.
\end{cor}

\begin{prop}
	Sea $K$ un cuerpo topológico, sea $V$ un $K$-EVT y sea $F \le V$ un subespacio vectorial. Entonces:
	\begin{enumerate}
		\item $V$ es de Hausdorff syss el $\Vec 0 \in V$ es un punto cerrado.
		\item $\overline{\{ \Vec 0 \}}$ es un subespacio vectorial y $V/\overline{\{ \Vec 0 \}}$ es,
			con la topología cociente, un $K$-espacio vectorial topológico de Hausdorff.
		\item El cociente de espacios vectoriales $V/F$ con la topología cociente es un $K$-EVT
			(y, en particular, la proyección $V \epicto V/F$ es continua).
		\item El cociente $V/F$ es de Hausdorff syss $F$ es cerrado en $V$, denotado $F \le_f V$.
	\end{enumerate}
\end{prop}

\begin{prop}
	Sea $K$ un cuerpo topológico, sea $V$ un $K$-EVT y $W \le V$ un subespacio vectorial.
	Entonces $\overline W$ también es un subespacio vectorial.
\end{prop}
\begin{proof}
	Hay que ver que $\overline W$ es cerrado bajo suma y producto por escalar.

	Para ello, sea $\vec v \in W$, entonces $\vec v + W \subseteq W \subseteq \overline W$, luego $W \subseteq -\vec v + \overline W$,
	pero como las traslaciones son homeomorfismos, entonces $-\vec v + \overline W$ es cerrado y por ende $\overline W \subseteq -\vec v + \overline W$,
	y sumando $\vec v$ se tiene que $\vec v + \overline W \subseteq \overline W$.
	Como aplica para todo $\vec v \in W$ se tiene que $W + \overline W \subseteq \overline W$.

	Sea ahora, $\vec v \in \overline W$, luego como $\vec v + W \subseteq W + \overline W \subseteq \overline W$, se repiten los pasos
	y se concluye que $\vec v + \overline W \subseteq \overline W$. Es decir, $\overline W$ es cerrado por suma de vectores.

	Sea $\alpha \in K$.
	Si $\alpha = 0$, entonces $\alpha\overline W = \{\Vec 0\} \subseteq W \subseteq \overline W$.
	Si $\alpha \ne 0$, entonces $\alpha W \subseteq W \subseteq \overline W$, luego $W \subseteq (1/\alpha)\overline W$,
	pero $\vec x \mapsto (1/\alpha)\vec x$ es homeomorfismo, entonces $(1/\alpha)\overline W$ es cerrado y así $\overline W \subseteq (1/\alpha)\overline W$,
	y así finalmente $\alpha\overline W \subseteq \overline W$.
\end{proof}
Como ejercicio para el lector otorgue una demostración alternativa al teorema anterior.

\begin{mydef}
	Sea $K$ un cuerpo métrico y sea $V$ un $K$-espacio vectorial.
	Se dice que un conjunto $S \subseteq V$ es \strong{equilibrado}\index{conjunto!equilibrado} si para todo $\alpha \in K$
	con $|\alpha| \le 1$ y todo $\vec x \in S$ se cumple que $\alpha \vec x \in S$ (equivalentemente, $\alpha S \subseteq S$).
\end{mydef}
\begin{lem}
	Sea $K$ un cuerpo métrico y sea $V$ un $K$-espacio vectorial.
	\begin{enumerate}
		\item La intersección de conjuntos equilibrados es equilibrada.
		\item Dado un subconjunto $S \subseteq V$ existe un mínimo subconjunto $E \subseteq V$ que es equilibrado
			y que contiene a $S$ llamado la \strong{envoltura equilibrada}\index{envoltura!equilibrada}.
	\end{enumerate}
\end{lem}
\begin{prop}
	Sea $K$ un cuerpo métrico y sea $V$ un $K$-EVT.
	\begin{enumerate}
		\item La clausura de un conjunto equilibrado es también equilibrada.
		\item Supongamos que $K$ es no discreto y localmente compacto.
			Entonces la envoltura equilibrada de un subconjunto compacto es también compacta.
	\end{enumerate}
\end{prop}
\begin{proof}
	Denotaremos por $B := \overline{B_1(0)} = \{ \lambda \in K : |\lambda| \le 1 \}$.
	\begin{enumerate}
		\item Sea $S \subseteq V$ un subconjunto arbitrario,
			entonces éste es equilibrado syss la imagen de la multiplicación escalar $s\colon B \times S \to V$ cae en $S$.
			Por continuidad, $s[ \,\overline{B \times S}\, ] = s[ B\times \overline{S} ] \subseteq \overline{s[B \times S]} = \overline{S}$.
		\item Como $K$ es localmente compacto y no discreto, uno puede corroborar que $B$ ha de ser compacto (¿por qué?).
			Luego la envoltura equilibrada de un compacto $C \subseteq V$ es la imagen de $B \times C \to V$,
			la cual es compacta. \qedhere
	\end{enumerate}
\end{proof}

\begin{mydef}
	Sea $k$ un cuerpo arbitrario y sea $V$ un $k$-espacio vectorial.
	Se dice que un subconjunto $A \subseteq V$ es \strong{absorvente}\index{conjunto!absorvente} si para todo vector $\vec x \in V$
	existe $\alpha \in k$ tal que $\alpha\vec x \in A$.
\end{mydef}
\begin{prop}
	Sea $K$ un cuerpo métrico no discreto, sea $V$ un $K$-EVT y sea $U$ un entorno del $\Vec 0$.
	Entonces:
	\begin{enumerate}
		\item Dado un escalar no nulo $\alpha\in K^\times$, entonces $\alpha U$ es entorno de $\Vec 0$.
		\item $U$ es un conjunto absorvente.
		\item $U$ contiene un entorno de $\Vec 0$ equilibrado.
	\end{enumerate}
\end{prop}
\begin{proof}
	\begin{enumerate}
		\item Queda al lector.
		\item Por continuidad del producto externo $f(\alpha, \vec v) := \alpha \vec v$ se cumple que $f^{-1}[U]$ es entorno de $(\Vec 0, \vec v)$, luego contiene a un abierto de la forma $B_{2\epsilon}(\Vec 0)\times W$ con $W$ entorno de $\Vec 0$, por ende, para todo $|\alpha| \le \epsilon$ se cumple que $\alpha\vec u\in U$ con $\vec u\in U$ y para un cierto $\epsilon > \Vec 0$.
		\item Siguiendo la construcción anterior, consideremos $W' := W\cap U$ que es entorno de $0$ contenido en $U$ tal que para todo $|\alpha| < 2\epsilon$ se cumple que $\alpha W \subseteq U$.
			Luego existe $|\alpha_0| < \epsilon$, de forma que
			$$ U_0 := \bigcup_{|\alpha| \le 1} \alpha\alpha_0 W' \subseteq U $$
			y $U_0$ es equilibrado. \qedhere
	\end{enumerate}
\end{proof}

\begin{thm}
	Sea $K$ un cuerpo topológico y $V$ un $K$-EVT.
	Para todo subespacio vectorial $U$ con interior no vacío, se cumple que $U=V$.
	En consecuente, toda aplicación lineal abierta entre EVTs debe ser sobreyectiva.
\end{thm}
\begin{proof}
	Como $U$ posee interior no vacío, entonces contiene a un abierto de la forma $\vec u + W$ donde $W$ es un entorno de $\Vec 0$ y $\vec u\in U$,
	como $W$ es vectorial entonces $W\subseteq U$.
	Como $W$ es absorvente, para todo $\vec v\in V$ existe $\alpha\in\K_{\ne 0}$ tal que $\alpha\vec v\in W\subseteq U$, luego
	$\alpha^{-1}(\alpha\vec v) = \vec v \in U$, ergo $U = V$.
\end{proof}

\begin{prop}
	Sea $K$ un cuerpo métrico no discreto y sea $V$ un $K$-EVT Hausdorff de dimensión 1.
	Entonces para todo $\vec x \in V$ no nulo, $a \mapsto a\vec x$ es un isomorfismo topológico de $K$-EVTs.
	En consecuencia, $V \cong K$ (en $\mathsf{TVS}_K$).
\end{prop}
\begin{proof}
	Que $f(a) := a\vec x$ sea un homomorfismo continuo biyectivo es trivial de la definición,
	por lo que falta ver que la inversa es continua.
	Ello podemos verificarlo en el $\Vec 0 \in V$:
	sea $\epsilon > 0$ un número real arbitrario y sea $a_0 \in K$ tal que $0 < |a_0| < \epsilon$, luego elija un entorno equilibrado $U$ del $\Vec 0$
	que no contiene a $a_0\vec x$ (por ser espacio de Hausdorff), luego si $b\vec x \in U$
	necesariamente $|b| < |a_0|$, ya que de lo contrario $|a_0b^{-1}| \le 1$ y $a_0\vec x = (a_0b^{-1})b\vec x \in U$, lo cual es absurdo.
	Así que para todo $\vec y \in U$ se cumple que $|f^{-1}(\vec y)| < \epsilon$.
\end{proof}

Recuérdese que un \emph{hiperplano (afín)} $H$ de un $K$-espacio vectorial $V$ es un subconjunto de la forma $\{ \vec x \in V : f(\vec x) = a \}$,
donde $f \colon V \to K$ es una aplicación lineal no nula y $a \in K$.
\begin{thm}
	Sea $K$ un cuerpo métrico no discreto y sea $V$ un $K$-EVT.
	Un hiperplano afín $H$ definido por $f(\vec x) = a$ es cerrado syss $f \colon V \to K$ es un homomorfismo continuo.
\end{thm}
\begin{proof}
	$\impliedby$. Trivial.

	$\implies$. Podemos suponer, sin perdida de generalidad, que $a = 0$.
	Ahora, $H \le V$ es un subespacio vectorial y el cociente $V/H$ es un $K$-EVT de dimensión 1 y es de Hausdorff (¿por qué?).
	Denotando por $\pi \colon V \to V/H$ a la proyección (que es continua), la propiedad universal de los cocientes podemos factorizar
	\[\begin{tikzcd}[column sep=small]
		V \ar[rr, "f"] \drar["\pi"', two heads] &                                 & K \\
			                                & V/H \urar["\exists!g"', dashed]
	\end{tikzcd}\]
	donde $g$ es un isomorfismo lineal y, por la proposición anterior, también debe ser un homeomorfismo.
	Luego $f$ es continua por composición.
\end{proof}

\begin{thm}
	Sea $K$ un cuerpo métrico completo y no discreto.
	Todo $K$-EVT de Hausdorff $V$ de dimensión finita $n$ es topológicamente isomorfo a $K^n$.
\end{thm}
\begin{proof}
	Procedemos por inducción, donde el caso $n = 1$ ya está probado.
	Si $\vec e_1, \dots, \vec e_n \in V$ es una base ordenada, entonces por hipótesis inductiva $W := K\vec e_1 + \cdots K\vec e_{n-1}$ es isomorfo
	a $K^{n-1}$ como $K$-EVT y, por tanto, es completo y, luego, también es cerrado en $V$.
	Así que $V/W$ es un $K$-EVT de Hausdorff de dimensión 1, por consiguiente es topológicamente isomorfo a $K$ y $V \cong W \times K \cong K^n$.
\end{proof}
\begin{ex}
	Considere a $\Q$ con la métrica usual.
	Nótese que $\R$ es un $\Q$-EVT con la topología usual y, por tanto, $\Q(\sqrt{2}) \subseteq \R$ es un $\Q$-EVT de dimensión 2.
	No obstante, $\Q(\sqrt{2})$ no es topológicamente isomorfo a $\Q^2$: la razón está en que todo par de vectores $\vec u, \vec v$
	linealmente independientes de $\Q^2$ satisfacen que el reticulado $\Z\vec u + \Z\vec v \subseteq \Q^2$ es un subconjunto discreto,
	mientras que $\{ 1, \sqrt{2} \}$ es una base de $\Q(\sqrt{2})$ tal que $\Z + \sqrt{2}\Z \subseteq \Q(\sqrt{2})$ es denso.
\end{ex}

\begin{cor}
	Sea $K$ un cuerpo métrico completo y no discreto, y sea $V$ un $K$-EVT de Hausdorff.
	Todo subespacio vectorial de dimensión finita de $V$ es cerrado.
\end{cor}
\begin{proof}
	Basta notar que un subespacio de dimensión finita $F$ es topológicamente isomorfo a $K^n$, y por tanto es un espacio métrico completo,
	luego es cerrado.
\end{proof}
\begin{cor}
	Sea $K$ un cuerpo métrico completo y no discreto, y sea $E$ un $K$-EVT arbitrario.
	Toda aplicación $K$-lineal $K^n \to E$ es continua.
\end{cor}

% Comenzamos por recordar la siguiente definición:
\begin{mydefi}
	Sea $V$ un $K$-espacio vectorial, donde $K$ es un cuerpo métrico.
	Una aplicación $\|\,\|\colon V \to \R$ se dice una \strong{seminorma}\index{seminorma} si:
	\begin{enumerate}[{SN}1.]
		\item $\|\vec x\| \ge 0$ y $\|\Vec 0\| = 0$.
		\item $\|\lambda\vec x\| = |\lambda| \, \|\vec x\|$.
		\item $\|\vec x + \vec y\| \le \|\vec x\| + \|\vec y\|$.
	\end{enumerate}
	Se dice que $\|\,\|$ es una \strong{norma}\index{norma} si, en lugar de SN1, satisface:
	\begin{enumerate}[{N}1.]
		\item $\|\vec x\| \ge 0$ y $\|\vec x\| = 0$ syss $\vec x = \Vec 0$.
	\end{enumerate}
	Un par $(V, \|\,\|)$ se dice un \strong{$K$-espacio (semi)normado}\index{espacio!normado} si $\|\,\|$ es una\break (semi)norma.
	Nótese que todo espacio (semi)normado es (pseudo)métrico con $d(\vec x, \vec y) := \|\vec x - \vec y\|$ y,
	por tanto, es un espacio topológico.
	Dos seminormas sobre $V$ se dicen \strong{equivalentes} si inducen la misma topología.
\end{mydefi}

\begin{thm}
	Sea $K$ un cuerpo métrico.
	Todo espacio normado es un EVT sobre $K$.
\end{thm}
\begin{proof}
	Denotaremos $E$ al espacio.
	\begin{enumerate}[i)]
		\item 
			\underline{La suma es continua:} 
			Consideremos $E^2$ con la norma $L_1$, luego, sean $(a,b),(c,d)\in E^2$
			$$ \|(a+b)-(c+d)\| \le \|a-c\| + \|b-d\| = \|(a-c,b-d)\|_1 = \|(a,b) - (c,d)\|_1, $$
			ergo, $+$ tiene la propiedad de Lipschitz, por ende es continua.

		\item \underline{El producto escalar es continuo:} 
			Consideremos $K\times E$ con la norma $L_\infty$, luego sean $(\lambda,x),(\lambda',y)\in K\times E$, sea $\epsilon > 0$,
			entonces consideremos
			$$ \delta := \frac{\epsilon}{|\lambda| + \|x\| + 1} > 0, $$
			y digamos que $(\lambda',y)$ está a menos de $\min(\delta, 1)$ de distancia de $(\lambda,x)$, por lo que
			$$ \|(\lambda,x) - (\lambda',y)\|_\infty = \max\{ |\lambda - \lambda'|, \|x - y\| \} < \min(\delta, 1). $$
			Luego, veamos que
			$$ \|\lambda x - \lambda'y\| = \|\lambda(x - y) + (\lambda - \lambda')y\| \le |\lambda|\, \|x - y\| + |\lambda - \lambda'|\,\|y\| $$
			ahora estamos casi listos, el único factor extraño es $\|y\| \le \|x - y\| + \|x\| \le 1 + \|x\|$.
			Ahora acotando los factores convenientes por $\delta$ se obtiene:
			\begin{equation}
				\|\lambda x - \lambda'y\| < |\lambda|\delta + \delta(1 + \|x\|) = \epsilon. \tqedhere
			\end{equation}
	\end{enumerate}
\end{proof}

\begin{thm}
	Sea $K$ un cuerpo métrico no discreto y $X$ un $K$-espacio vectorial.
	Dos normas $\|\,\|_1$ y $\|\,\|_2$ sobre $X$ son equivalentes syss existen $\alpha,\beta > 0$ reales
	tales que para todo $\vec x \in X$ se cumple que
	$$ \|\vec x\|_1 \le \alpha\|\vec x\|_2,\quad \|\vec x\|_2 \le \beta\|\vec x\|_1. $$
\end{thm}
\begin{proof}
	$\impliedby$. Queda al lector.

	$\implies$. La demostración es por contradicción.
	Si las normas son equivalentes, pero las constantes no existen entonces sin perdida de generalidad supondremos que
	$$ \inf_{\vec x \ne \Vec 0} \frac{\|\vec x\|_2}{\|\vec x\|_1} = 0 $$
	de esta forma, definimos:
	\begin{align*}
		f \colon X &\longrightarrow X \\
		\vec x &\longmapsto
		\begin{cases}
			\frac{\vec x}{\|\vec x\|_1}, &\vec x\ne \Vec 0\\
			\Vec 0, &\vec x=\Vec 0
		\end{cases}
	\end{align*}
	notemos que $f$ no es continua en $(X, \|\,\|_1)$ pues si aplicamos $f$ primero y luego $\|\,\|_1$, entonces la función es discontinua en $\Vec 0$,
	pero la norma es continua y la composición de continuas es continua.
	No obstante, $f$ sí es continua en $(X, \|\,\|_2)$, pues lo es en todo $\vec x\ne \Vec 0$ por ser cociente entre continuas,
	y en $\vec x = \Vec 0$ también ya que de no serlo, entonces $f[B_\epsilon(\Vec 0)]$ no sería entorno de $\Vec 0$, luego al aplicarle $\|\,\|_2$ ...
\end{proof}

\begin{thm}
	Sea $K$ un cuerpo métrico, no discreto, completo.
	Las normas sobre un espacio de dimensión finita sobre $K$ son equivalentes.
\end{thm}
\begin{proof}
	En particular probaremos que son todas equivalentes a $\|\vec x\|_\infty := \max\{ |x_1|,\dots,|x_n| \} $.
	Sea $ \|\,\| $ una norma cualquiera, luego como $E$ es de dimensión finita posee una base canónica $\vec e_1,\dots,\vec e_n$ de modo que
	$$ \|\vec x\| = |x_1|\,\|\vec e_1\| + \cdots + |x_n|\,\|\vec e_n\| \le \lambda\|\vec x\|_\infty $$
	donde $\lambda := \|\vec e_1\| + \cdots + \|\vec e_n\|$.

	Para ver la otra desigualdad probaremos por inducción que para todo $n$ y todo $i\le n$ existe $\mu_i>0$
	tal que $|x_i| < \mu_i\|\vec x\|$ con $\vec x \in \korpe^n$, pues,
	luego $\mu := \max\{\mu_i: i\le n\}$ satisface que $\|\vec x\|_\infty \le \mu\|\vec x\|$.

	Para $n = 1$ veamos que $\mu := 1/\|1\|$ cumple lo pedido.
	El caso $E := \korpe^{n+1}$, se demostrará que $\vec e_{n+1}$ es adherente a $F := \korpe^n \times \{0\}$ que es cerrado (por hipótesis inductiva),
	lo que sería absurdo.
	Supongamos que hay un índice que no cumple la hipótesis; sin perdida de generalidad podemos suponer que uno de ellos
	es el $(n+1)$-ésimo (de lo contrario reordenamos índices), es decir que para todo $\mu > 0$ existe $\vec x\in E$ tal que $|x_{n+1}|>\mu\|\vec x\|$.
	Podemos asumir que $x_{n+1} = 1$ pues basta dividir por tal $\vec x$ por $x_{n+1}$.
	Para probar que $\vec e_{n+1}$ es adherente, sea $\epsilon > 0$, luego $\mu := 1/\epsilon > 0$
	y existe $\vec y\in F$ tal que $d(-\vec y, \vec e_{n+1}) = \|\vec y + \vec e_{n+1}\| < 1/\mu = \epsilon$.
	Pero esto es absurdo, lo que completa la demostración.
\end{proof}
\begin{ex}
	\warn
	Para dimensión infinita, el teorema anterior falla.
	Considere $V = \R^{\oplus\N}$, es decir, el espacio de sucesiones $\vec x = (x_n)_{n\in\N}$ en $\R$
	donde todos salvo finitas coordenadas son nulas.
	Aquí podemos considerar las normas
	\[
		\|\vec x\|_1 = \sum_{n\in\N} |x_n|, \qquad \|\vec x\|_2 := \sum_{n=0}^{\infty} \frac{|x_n|}{n+1},
	\]
	y notamos que la sucesión de la base canónica $\vec e_0 := (1, 0, 0, \dots)$, $\vec e_1 := (0, 1, 0, \dots)$ y así sucesivamente
	converge, en la norma $\|\,\|_2$, al vector $\Vec 0$; mientras que diverge en la norma $\|\,\|_1$.
\end{ex}

El ejemplo anterior es <<artificial>>, pero más adelante, cuando tengamos más herramientas analíticas a disposición, el lector podrá corroborar
que hay varios otros ejemplos <<más naturales>> donde hay normas no equivalentes a disposición.
En cierto sentido, esto también da una especie de \emph{riqueza} a la categoría de EVTs.

\begin{lem}
	Sea $K$ un cuerpo topológico, sea $V$ un $K$-espacio vectorial y $(E_i)_{i\in I}$ una familia de $K$-EVTs.
	La topología inicial sobre $V$ definida por una familia de aplicaciones $K$-lineales $\{ f_i \colon V \to E_i \}_{i\in I}$
	dota a $V$ estructura de $K$-EVT.

	El espacio $V$ es de Hausdorff syss para cada $\vec x \in V$ no nulo, existe un índice $i \in I$ y un entorno del origen $U_i \subseteq E_i$
	tal que $f_i(\vec x) \notin U_i$.
	En particular, si cada $E_i$ es de Hausdorff, $V$ también syss $\bigcap_{i\in I} \ker(f_i) = \{ 0 \}$.
\end{lem}
\begin{proof}
	La primera afirmación (que $V$ sea un $K$-EVT) se deduce de la propiedad universal de la topología inicial.
	% (i.e., cada $f_i$ es continuo y $g \colon F \to V$ es continuo syss todas las poscomposiciones $g\circ f_i$ lo son).
	La condición equivalente de ser Hausdorff se deduce de que un $K$-EVT es Hausdorff syss el origen es un punto cerrado y
	\begin{equation}
		\overline{\{ 0 \}} = \bigcap_{i\in I} f_i^{-1}[\overline{\{ 0 \}}].
		\tqedhere
	\end{equation}
\end{proof}
\begin{cor}
	Sea $K$ un cuerpo topológico y sea $V$ un $K$-EVT con la topología inicial definida por
	las funciones lineales continuas $\{ f_i \colon V \to E_i \}_{i\in I}$.
	Sea $X$ un espacio topológico, una función $g \colon X \to V$ es continua syss cada $g\circ f_i$ lo es.
\end{cor}

La siguiente definición será útil más adelante:
\begin{mydef}
	Sea $K$ un cuerpo métrico y sea $V$ un $K$-espacio vectorial.
	Dado un conjunto $\Gamma$ de seminormas sobre $V$, la unión de las topologías inducidas por los elementos de $\Gamma$
	se dice la \strong{topología definida por $\Gamma$} sobre $V$ y es la mínima tal que $V$ es un $K$-EVT y todas las seminormas $p \in \Gamma$
	son continuas.
\end{mydef}
La siguiente proposición es un ejercicio:
\begin{prop}\label{thm:top_def_seminorm_set}
	Sea $K$ un cuerpo métrico y sea $V$ un $K$-EVT cuya topología está definida por un conjunto $\Gamma$ de seminormas.
	\begin{enumerate}
		\item Si $\Gamma$ es finito, entonces $V$ es un espacio seminormado cuya topología está inducida por la seminorma
			\[
				P(\vec x) := \max\{ p(\vec x) : p \in \Gamma \}.
			\]
		\item La clausura $\overline{\{ 0 \}}$ es el conjunto $\vec x \in V$ tales que $p(\vec x) = 0$ para todo $p \in \Gamma$.
		\item Para $p \in \Gamma$, denotemos por $V_p$ al $K$-espacio vectorial $V$ dotado de la topología de la seminorma $p$.
			La diagonal $V \hookto \prod_{p\in \Gamma} V_p$ es un encaje topológico.
		\item Si $V$ es de Hausdorff y $\Gamma$ es numerable, entonces $V$ es metrizable.
	\end{enumerate}
\end{prop}
\begin{proof}
	El inciso 3 se deduce del hecho de que la afirmación es cierta en general para las topologías iniciales.
	El inciso 4 se deduce del tercero, a partir de que el producto numerable de espacios pseudométricos es pseudométrico.
\end{proof}

% \begin{mydefi}[Espacio vectorial topológico]
% 	Se dice que $(V, \tau)$ es un $\korpe$-\strong{espacio vectorial topológico} (abrevaido, EVT)\index{espacio!vectorial topológico (EVT)} si
% 	$V$ es un $\korpe$-espacio vectorial y $\tau$ es una topología sobre $V$ tal que la suma $+\colon V^2 \to V$ y el producto por escalar
% 	$\cdot\colon \korpe\times V\to V$ son funciones continuas.
% 	% \nomenclature{EVT}{Espacio vectorial topológico}
% \end{mydefi}

% \begin{prop}
% 	Un EVT es de Hausdorff syss $\{\Vec 0\}$ es cerrado.
% \end{prop}
% \begin{proof}
% 	Basta notar que el gráfico de la diagonal es preimagen del $\{\Vec 0\}$ bajo la función $f\colon V^2 \to V$ dada por $f(x, y) := x-y$.
% \end{proof}

\begin{thm}
	Sea $K$ un cuerpo métrico, no discreto y localmente compacto.
	Un $K$-espacio normado es localmente compacto syss es de dimensión finita.
\end{thm}
\begin{proof}
	$\implies$. Sea $X$ el espacio. Como $X$ es localmente compacto, entonces $\overline B_1( \Vec 0 )$ es compacto, luego es totalmente acotado
	y existen $\vec x_1, \dots, \vec x_n \in \overline B_1(\Vec 0)$ tales que:
	$$ \bigcup_{i=1}^n B_{1/2}(\vec x_i) \supseteq \overline B_1(\Vec 0). $$
	Definamos $F := \Span\{ \vec x_1, \dots, \vec x_n \}$, veremos que $F = X$.
	Sea $\vec x \in X \setminus F$, entonces sea $r := d(\vec x, F)$, luego nótese que $K := \overline B_{r+1}(\vec x) \cap F$ es un conjunto
	compacto (¿por qué?) y no vacío; por ende la función $d(\vec x, -)$ que es continua, alcanza un mínimo y necesariamente dicho mínimo es $r$;
	vale decir, existe $\vec y \in F$ tal que $r = \|\vec x - \vec y\|$.
	Como $F$ es cerrado, se debe cumplir que $\vec x \ne \vec y$, luego existe $\vec x_i$ tal que
	$$ \left\| \frac{\vec x - \vec y}{\|\vec x - \vec y\|} - \vec x_i \right\| < \frac{1}{2}. $$
	Finalmente, multiplicando todo por $r > 0$ se obtiene que
	$$ d(\vec x, \vec y + r\vec x_i) = \|\vec x - \vec y - r\vec x_i\| < \frac{r}{2}, $$
	pero $\vec y + r\vec x_i \in F$, lo que contradice la definición de $r$.
	\par
	$\impliedby$. Todo espacio normado de dimensión finita es necesariamente $\R^n$ con la topología usual, el cual es localmente compacto.
\end{proof}

\begin{prop}
	Sea $K$ un cuerpo métrico no discreto, sean $N, M$ un par de $K$-espacios normados y sea $f \in L(N, M)$ una aplicación lineal.
	Son equivalentes:
	\begin{enumerate}
		\item $f$ es continua.
		\item $f$ es continua en $\Vec 0$.
		\item $f$ tiene la propiedad de Lipschitz.
		\item $f[\overline{B}_1(\Vec 0)]$ es acotado en $M$.
	\end{enumerate}
\end{prop}
\begin{proof}
	Basta notar que $3 \implies 1 \implies 2 \implies 4 \implies 3$
	(donde $2\implies 4$ deriva que ser continua en $\Vec 0$ implica estar acotada cerca de $\vec 0$, luego se multiplica por un escalar).
\end{proof}
% \begin{mydef}
% 	Sean $E, F$ un par de $\korpe$-EVTs, se denota por $\dual(N, M)$ al conjunto de funciones continuas y lineales desde $E$ a $F$.
% \end{mydef}

En lo sucesivo, como querremos trabajar con espacios de funciones lineales, denotaremos por $|\,|$ a una norma sobre un $K$-espacio vectorial $V$,
para poder denotar por $\|\,\|$ a una norma sobre $\dual(W, V)$.
\begin{prop}
	Sean $E$ un EVT y $F$ un espacio normado sobre un cuerpo métrico $K$, entonces
	\begin{align*}
		\| \, \|_\infty \colon \dual(E, F) &\longrightarrow K \\
		f &\longmapsto \sup\{|f(\vec x)| : |\vec x| = 1 \}
	\end{align*}
	determina una norma.
\end{prop}
En particular, si $f \in \dual(V, V)$ y definimos $C := \|f\|_\infty$, entonces
$$ \forall\vec x\in V \quad | f(\vec x) | \le C |\vec x|. $$

% \begin{prop}
% 	Sean $E, F, G$ un trío de $\korpe$-EVTs. Entonces se cumplen:
% 	\begin{enumerate}
% 		\item $\Id_E \in \dual(E, E)$.
% 		\item Si $f \in \dual(E, F)$ y $g\in \dual(F, G)$, entonces $f\circ g \in \dual(E, G)$.
% 	\end{enumerate}
% 	En consecuencia, los espacios vectoriales topológicos (como objetos) y las funciones lineales continuas entre ellos (como flechas)
% 	constituyen una categoría denotada $\mathsf{TVS}$, la cual es una subcategoría tanto de $\mathsf{Top}$ como $\mathsf{Vect}$.
% \end{prop}
% % El lector debería advertir la potencia y rigidez de un equilibrio entre dichos objetos.

\begin{mydef}
	Un \strong{espacio de Banach}\index{espacio!de Banach} sobre un cuerpo métrico $K$ es un $K$-espacio normado que es completo (como espacio métrico).
\end{mydef}
\begin{prop}
	Sea $K$ un cuerpo métrico, sea $E$ un $K$-EVT y $F$ un $K$-espacio de Banach.
	Entonces $\dual(E, F)$ también es de Banach.
\end{prop}

\subsection{El teorema de Hahn-Banach analítico}
En esta subsección y la siguiente trabajaremos principalmente con $K = \R$.
\begin{mydefi}
	% Un conjunto $A$ de un $\korpe$-EVT se dice:
	% % \strong{(linealmente) acotado} si para todo entorno $U$ de $\Vec 0$ existe $\lambda\in\korpe$ tal que $A \subseteq \lambda U$.
% % 	Un abierto $U$ de un $\K$-EVT (donde $\K$ es métrico) $V$ se dice:
	% \begin{description}
	% 	\item[(Linealmente) acotado]\index{acotado (linealmente)}
	% 		Si para todo entorno $U$ de $\Vec 0$ existe $\lambda\in\korpe$ tal que $A \subseteq \lambda U$.
	% 	\item[Absorvente]\index{absorvente (conjunto)}
	% 		Si para todo $\vec x\in V$ existe $\epsilon > 0$ tal que para todo $|\alpha| \le \epsilon$ se cumple que $\alpha\vec x\in A$.
	% \end{description}
	Sea $V$ un $\R$-espacio vectorial y sea $A \subseteq V$ un subconjunto cualquiera.
	Se dice que $A$ es \strong{convexo}\index{convexo (conjunto)} si para todo $\vec x,\vec y\in A$ y todo $\lambda\in[0,1]$
	se cumple que $\lambda\vec u + (1-\lambda)\vec v \in A$.
	Se dice que $A$ es \strong{absolutamente convexo}\index{absolutamente!convexo (conjunto)} si es convexo y equilibrado.
\end{mydefi}

\begin{prop}
	Sea $V$ un $\R$-espacio normado. Entonces:
	\begin{enumerate}
		\item Las bolas son convexas, y en consecuente, las bolas centradas en el origen son absolutamente convexas.
		\item La intersección de convexos es convexo, en consecuente, la intersección de absolutamente convexos es absolutamente convexo.
	\end{enumerate}
\end{prop}

\begin{mydef}
	Sea $V$ un $\R$-EVT y $A \subseteq V$ un subconjunto.
	Se definen la \strong{envoltura (absolutamente) convexa} de $A$ como
	\begin{align*}
		\conv A &:= \bigcap\{C : A\subseteq C\wedge C\text{ convexo}\},\\
		\aco A  &:= \bigcap\{C : A\subseteq C\wedge C\text{ abs. convexo}\};
	\end{align*}
	resp.
\end{mydef}

\begin{prop}
	Si $A$ es convexo, entonces $\Int A$ y $\overline A$ son convexos.
\end{prop}
\begin{proof}
	Sea $\vec u\in A$ y $\vec v\in\Int A$, luego para todo $0<\lambda<1$ se cumple que $\lambda\vec u + (1-\lambda)\vec v \in \lambda\vec u + (1-\lambda)\Int A$ donde el último es un subconjunto abierto de $A$, es decir, $A$ es entorno de $\lambda\vec u + (1-\lambda)\vec v$, luego pertenece a $\Int A$, ergo es convexo.
	\par
	Sea $f:\R\times V^2 \to V$ la aplicación tal que $f(\lambda,\vec u,\vec v) := \lambda\vec u + (1-\lambda)\vec v$ la cual es continua pues $V$ es un EVT, luego, un conjunto $A$ es convexo syss $f\big[ [0,1]\times A^2 \big] \subseteq A$, luego
	\begin{equation}
		f\left[ [0,1]\times\overline{A}^2 \right] = f\left[ \overline{ [0,1]\times A^2 } \right]
		\subseteq \overline{ f\big[ [0,1]\times A^2 \big] } \subseteq \overline A. \tqedhere
	\end{equation}
\end{proof}

\begin{ex}
	Considere el subespacio $X \subseteq \R^2$ dado por
	$$ X = \{ (x, 1/x) : x \in (0, \infty) \} \cup \{ (0, 0) \}. $$
	Nótese que $X$ es cerrado (¿por qué?), y luego compruebe que
	$$ \conv X = (0, \infty) \times (0, \infty) \cup \{ (0, 0) \}. $$
	De modo que $X$ es cerrado, pero $\conv X$ no lo es.
\end{ex}

% \begin{lem}
% 	Sean $V, W$ un par de $\korpe$-EVTs y $f \in L(V, W)$.
% 	Entonces $f$ es continua syss para todo entorno $U \subseteq W$ de $\Vec 0$, se cumple que $f^{-1}[U] \subseteq V$ es un entorno de $\Vec 0$.
% \end{lem}
% \begin{proof}
% 	$\implies$. Trivial.
% 	\par
% 	$\impliedby$.
% 	Como las traslaciones son homeomorfismos, entonces basta probar que la preimagen de todo entorno abierto $U \subseteq W$ de $\Vec 0$ es abierta.
% 	Sea $\vec v \in f^{-1}[U]$, es decir, $f(\vec v) \in U$; consideremos $g(\vec x) = \vec x + \vec v$ el cual es un homeomorfismo sobre $V$,
% 	luego existe un entorno abierto $U'$ de $\Vec 0$ tal que $f(\vec v) + U' \subseteq U$.
% 	Por hipótesis existe $U''$ entorno abierto de $\Vec 0$ tal que $f[U''] \subseteq U'$, luego
% 	$$ f[ \vec v + U'' ] = f(\vec v) + f[U''] \subseteq f(\vec v) + U' \subseteq U, $$
% 	por lo que $\vec v + U'' \subseteq f^{-1}[U]$ y $f$ es continua.
% \end{proof}

% \begin{thm}
% 	Sea $V$ un $\K$-EVT, entonces $f\in L(V, \K)$ es continua syss el núcleo es cerrado.
% \end{thm}
% \begin{proof}
% 	$\implies$. $\{0\}$ es cerrado en $\K$, luego $f^{-1}[\{0\}] = \ker f$ es cerrado.
% 	\par
% 	$\impliedby$. En primer lugar, como $\dim\K = 1$, entonces se tiene que $f$ es o nula (lo que es claramente continua) o suprayectiva.
% 	Si $f$ es suprayectiva, tenemos que probar, por el lema, que para todo $\epsilon > 0$ se cumple que $f^{-1}[ B_\epsilon(0) ]$ es
% 	un entorno del $\Vec 0$.

% 	Sea $\alpha \in \K$ tal que $0 < |\alpha| < \epsilon$, luego existe $\vec v_0$ tal que $f(\vec v_0) = \alpha$ y concretamente
% 	$f^{-1}[ \{\alpha\} ] = \vec v_0 + \ker f$, el cual es un cerrado del espacio, luego su complemento es un entorno abierto del $\Vec 0$,
% 	el cual contiene a un subentorno abierto equilibrado $U$; vale decir, $\alpha \notin f[U]$.
% 	Sea $\vec u \in U$, si $f(\vec u) \ne 0$, entonces como
% 	$$ f\left( \frac{\alpha}{f(\vec u)} \vec u \right) = \frac{\alpha}{f(\vec u)} f(\vec u) = \alpha, $$
% 	entonces, como $U$ es equilibrado se sigue que $| \alpha/f(\vec u) | > 1$, vale decir, $|f(\vec u)| < |\alpha| < \epsilon$.
% 	Si $f(\vec u) = 0 < \epsilon$.
% 	Así pues, en cualquier caso, $U \subseteq f^{-1}[ B_\epsilon(0) ]$ como se quería probar.
% \end{proof}

\thmdep{AEN}
\begin{thm}
	Sea $K$ un cuerpo métrico, sean $M, N$ un par de $K$-espacios normados con $N$ de Banach y sea $F \le M$ un subespacio vectorial.
	Dado $f \colon F \to N$ una función lineal continua, entonces posee una extensión $\bar f \in \dual(\overline F, N)$
	tal que $\| \bar f \|_\infty = \| f \|_\infty$.
\end{thm}
\begin{proof}
	Sea $\vec x \in \overline F$, entonces podemos elegir $\vec x_n \in F$ tal que $d(\vec x, \vec x_n) < 1/n$, de modo que $\lim_n \vec x_n = \vec x$.
	Luego, nótese que como $\vec x_n$ es convergente en $\overline F$, entonces es de Cauchy en $F$.
	Sea $C := \| f \|_\infty > 0$ y sea $\epsilon > 0$, entonces para $n, m$ suficientemente grande se tiene que
	\[
		\|\vec x_n - \vec x_m\| < \frac\epsilon C \implies \|f(\vec x_n) - f(\vec x_m)\|
		= \|f(\vec x_n - \vec x_m)\| \le C \|\vec x_n - \vec x_m\| < \epsilon.
	\]
	Vale decir, la sucesión de los $f(\vec x_n)$'s es de Cauchy, luego es convergente pues $N$ es completo.
	Luego podemos definir:
	\[
		\bar f(\vec x) = \lim_n f(\vec x_n).
	\]
	Y así para cualquier sucesión que aproxime a todo punto en $\overline F$.
	\begin{enumerate}[i)]
		\item \underline{$\bar f$ está bien definida:}
			Sean $\lim_n \vec x_n = \lim_n \vec y_n = \vec x$.
			Luego, por continuidad de la suma se tiene que $\lim_n \vec x_n - \vec y_n = \Vec 0$,
			lo que equivale a ver que $\lim_n |\vec x_n - \vec y_n| = 0$.
			De aquí se sigue claramente que $\lim_n | f(\vec x_n) - f(\vec y_n) | = 0$.

		\item \underline{$\bar f$ es lineal:}
			Se reduce a aplicar la continuidad de la suma y del producto escalar para poder hacer álgebra con dichos límites.

		\item \underline{$\bar f$ está acotada por la misma cota:}
			Se reduce a ver que
			\begin{equation}
				\lim_n | f(\vec x_n) | \le \lim_n C | \vec x_n | = C\mathopen{}\big\lvert \lim_n \vec x_n \big\rvert\mathclose{}
				= C| \vec x |. \tqedhere
			\end{equation}
	\end{enumerate}
\end{proof}
\thmdep{}

\thmdep{AE}
Los siguientes teoremas se demuestran usando el axioma de elección, pero está probado que basta con asumir el teorema del ultrafiltro.
\begin{thmi}[Teorema de Hahn-Banach]\index{teorema!de Hahn-Banach}
	Sea $V$ un $\R$-espacio vectorial con una función $p \colon V \to \R$ tal que\footnotemark{}
	\begin{equation}
		\forall \lambda > 0 \qquad p(\lambda\vec u) = \lambda p(\vec u), \quad
		p(\vec u + \vec v) \le p(\vec u) + p(\vec v).
		\label{eqn:hahn_ban_axioms}
	\end{equation}
	Sea $W \le V$ y $f \in L(W, \R)$ un funcional tal que $f(\vec x) \le p(\vec x)$ para todo $\vec x \in W$.
	Entonces existe $\bar f \in L(V, \R)$ tal que $\bar f|_W = f$ y que $\bar f(\vec u) \le p(\vec u)$ para todo $\vec u\in V$.
\end{thmi}
\footnotetext{A las funciones que satisfacen esta condición se les conoce como \textit{funcionales convexos}, \textit{funcionales de Minkowski}
o \textit{medidores} en otras lenguas (eng.\ \textit{gauge}).}
\begin{proof}
	Sea $W\subseteq W_1\subset V$ un subespacio vectorial y $f_1:W_1\to \R$ cumpliendo las condiciones del enunciado,
	luego si $\vec w\in V\setminus W_1$, entonces sea $W_2 := W_1\oplus\sangle{\vec w}$, veremos que existe $f_2\colon W_2\to\R$
	que también cumple las condiciones.
	\par
	Sean $\vec u_1,\vec v_1\in W_1$, entonces
	\begin{align*}
		f_1(\vec u_1) - f_1(\vec v_1) &= f_1(\vec u_1 - \vec v_1) \le p(\vec u_1 - \vec v_1) \\
		&= p(\vec u_1 + \vec w - (\vec v_1 + \vec w)) \le p(\vec u_1 + \vec w) + p(-\vec v_1 - \vec w),
	\end{align*}
	reordenando nos queda que
	$$ -p(-\vec v_1 - \vec w) - f(\vec v_1) \le p(\vec u_1 + \vec w) - f_1(\vec u_1) $$
	como desigualdad real, luego, se concluye facilmente que
	$$ \sup_{\vec v_1\in W_1} \left( -p(-\vec v_1 - \vec w) - f(\vec v_1) \right)
	\le \inf_{\vec u_1\in W_1} \left( p(\vec u_1 + \vec w) - f_1(\vec u_1) \right). $$
	Sea $ k $ un real cualquiera entre los dos valores de la desigualdad superior y definamos $f_2(\vec v + \alpha\vec w) := f_1(\vec v) + \alpha k$
	donde $\vec v\in W_1$.
	Claramente $f_2$ es un funcional y por definición
	$$ \gamma \le p\left( \frac{\vec v}{\alpha} + \vec w \right) - f_1\left( \frac{\vec v}{\alpha} \right) $$
	luego se concluye facilmente que $f_2(\vec v + \alpha\vec w) \le p(\vec v + \alpha\vec w)$ como se quería probar.
	\par
	Ahora invocamos el AE en forma del lema de Zorn construyendo la familia de funcionales que cumplen lo del enunciado
	para encontrar un funcional maximal que sería el que buscamos.
\end{proof}
Nótese que el AE o TUF sólo se requiere si $V$ es de dimensión infinita.

\begin{thmi}[Teorema de Hahn-Banach analítico]
	Si $V$ es un $\K$-espacio vectorial (con $\K\in\{\R,\C\}$),
	$W$ un subespacio vectorial, $f\in L(W,\K)$ un funcional y $p\colon V\to\R$ una seminorma tal que para todo $\vec w\in W$
	se cumple que $|f(\vec w)| \le p(\vec w)$, entonces existe $\bar f\colon V\to\K$ tal que $\bar f|_W = f$ y que
	$|\bar f(\vec u)| \le p(\vec u)$ para todo $\vec u\in V$.
\end{thmi}
\begin{proof}
	Si $\K = \R$, toda seminorma cumple los requisitos de la versión anterior del teorema de Hahn-Banach,
	luego $f$ admite una extensión $\bar f$, pero $\bar f(-\vec v) \le p(-\vec v) = p(\vec v)$, luego $|f(\vec v)| \le p(\vec v)$ como se quería probar.
	\par
	Si $\K = \C$, entonces comenzamos por considerar a $V$ como un $\R$-espacio vectorial,
	de modo que $g := \Re(f) \in L(W,\R)$ por lo que admite una extensión $\bar g$ por el inciso anterior,
	de modo que probaremos que $\bar f(\vec v) := \bar g(\vec v) - \ui\bar g(\ui\vec v)$ cumple lo pedido.
	En primer lugar nótese que $\Re f(\ui\vec v) = \Re( \ui f(\vec v) ) = -\Im f(\vec v)$, por lo que $\bar f|_W = f$.
\end{proof}
\thmdep{}

Vamos a presentar un par de corolarios, para los cuales emplearemos el concepto de dual.
Dado un $\R$-EVT $V$ denotaremos por $V^* := \dual(V, \R)$ y $V^\wedge := L(V, \R)$.

Veamos un par de consecuencias:
\begin{cor}
	Sea $N$ un $\R$-espacio normado.
	Para cada $\vec x_0 \in N$, existe un funcional $f \in N^*$ tal que
	\[
		\|f\| = |\vec x_0|, \qquad f(\vec x_0) = |\vec x_0|^2.
	\]
\end{cor}
\begin{proof}
	Basta extender el funcional $\tilde f(t\vec x_0) := t|\vec x_0|^2$ definido en el subespacio vectorial $F := \R\vec x_0 \le N$.
\end{proof}
\begin{cor}
	Sea $N$ un $\R$-espacio normado.
	Para todo $\vec x \in N$, se cumple que
	\[
		|\vec x| = \sup\{ |f(\vec x)| : f \in N^*, \|f\| = 1 \} = \max\{ |f(\vec x)| : f \in N^*, \|f\| = 1 \}.
	\]
\end{cor}

\subsection{El teorema de Hahn-Banach geométrico}
Éste es uno de los resultados centrales del análisis funcional.

\begin{mydef}
	Un \strong{espacio localmente convexo}\index{espacio!localmente convexo} es un $\R$-EVT que posee una base de entornos convexos.
	Un \strong{espacio de Fréchet}\index{espacio!de Fréchet} es un espacio localmente convexo, metrizable y completo.
\end{mydef}
\begin{cor}
	Todo espacio de Banach es de Fréchet.
\end{cor}

El siguiente lema es fundamental:
\begin{lem}\label{thm:GHB_seminorm_lemma}
	Sea $V$ un $\R$-EVT y sea $U \subseteq E$ un entorno abierto convexo del origen.
	Defínase
	\[
		p_U \colon V \longrightarrow \R, \qquad \vec x \longmapsto \inf\{ \alpha > 0 : \alpha^{-1}\vec x \in U \},
	\]
	entonces satisface \eqref{eqn:hahn_ban_axioms}, además,
	\[
		U = \{ \vec x \in V : p(\vec x) < 1 \}
	\]
	y si $V$ es seminormado, entonces existe $M > 0$ tal que $0 \le p(\vec x) \le M|\vec x|$ para todo $\vec x \in V$.
\end{lem}
\begin{proof}
	Es claro que $p(\lambda\vec x) = \lambda p(\vec x)$ para $\lambda > 0$.
	Sea $r > 0$ tal que $B_r(\Vec 0) \subseteq U$, entonces para todo $\vec x \in V$ se cumple que $p(\vec x) \le \frac{1}{r}|\vec x|$,
	por lo que $M = 1/r$ sirve.

	Sea $\vec x \in U$, existe $\epsilon > 0$ suficiente pequeño tal que $(1 + \epsilon)\vec x \in U$, de modo que $p(\vec x) \le \frac{1}{1+\epsilon} < 1$.
	Recíprocamente, si $p(\vec x) < 1$, entonces existe $0 < \alpha < 1$ con $\alpha^{-1}\vec x \in U$, luego
	\[
		\vec x = \alpha(\alpha^{-1}\vec x) + (1 - \alpha)\Vec 0 \in U.
	\]
	Finalmente, para ver que se satisface \eqref{eqn:hahn_ban_axioms}, sea $\vec u, \vec v \in V$ y sea $\epsilon > 0$.
	% que $p(\vec u + \vec v) \le p(\vec u) + p(\vec v)$
	Luego $\vec x := \frac{1}{p(\vec u) + \epsilon} \vec u, \vec y := \frac{1}{p(\vec v) + \epsilon} \vec v \in U$, de modo que con
	$t := \frac{p(\vec u) + \epsilon}{p(\vec u) + p(\vec v) + 2\epsilon} \in (0, 1)$ se cumple que
	\[
		\frac{\vec u + \vec v}{p(\vec u) + p(\vec v) + 2\epsilon} = t\vec x + (1 - t)\vec y \in U,
	\]
	por lo que $p(\vec u + \vec v) \le p(\vec u) + p(\vec v)$.
\end{proof}

\begin{prop}
	Sea $V$ un espacio localmente convexo, entonces posee una base de entornos del origen que son absolutamente convexos.
\end{prop}
\begin{cor}
	Sea $V$ un $\R$-EVT.
	Entonces $V$ es localmente convexo syss existe una familia de seminormas que define su topología.
\end{cor}
\begin{proof}
	$\impliedby$.
	Es trivial de verificar que, si $p \colon V \to \R$ es una seminorma
	entonces la bola $B^p_r(0) = \{ \vec x \in V : p(\vec x) < r \}$ es un entorno absolutamente convexo del origen
	y que los abiertos son intersecciones finitas de los conjuntos anteriores.

	$\implies$.
	Si $\mathcal{B}$ es una base de entornos simétricos absolutamente convexos del origen,
	entonces la topología de $V$ está generada por las seminormas $\{ p_U : U \in \mathcal{B} \}$, donde $p_U$ es la aplicación definida en
	el lema~\ref{thm:GHB_seminorm_lemma}.
\end{proof}
% \begin{cor}
% 	Sea $V$ un espacio localmente convexo. Son equivalentes:
% 	\begin{enumerate}
% 		\item $V$ es (pseudo)metrizable.
% 		\item Existe una familia numerable de seminormas que inducen su topología.
% 		\item $V$ es 1AN o, equivalentemente, existe una base numerable de entornos del origen.
% 	\end{enumerate}
% \end{cor}

Ahora bien, como la continuidad en la topología inicial se define por poscomposición,
el corolario anterior nos permite verificar continuidad en cada seminorma por separado.
\begin{mydef}
	Sea $V$ un $\R$-espacio vectorial.
	Dado un epimorfismo lineal $f \colon V \to \R$ y un $\gamma \in \R$,
	se dice que el hiperplano afín $H = \{ \vec x \in V : f(\vec x) = \gamma \}$ \strong{separa} a un par de subconjuntos prefijados $A, B \subseteq V$
	si (quizá tras intercambiar $A$ y $B$) se cumple que
	\[
		\forall \vec a\in A, \vec b \in B, \qquad f(\vec a) \le \gamma \le f(\vec b).
	\]
	Se dice que $H$ \strong{separa estrictamente} a $A$ y $B$ si existe $\epsilon > 0$ tal que
	\[
		\forall \vec a\in A, \vec b \in B, \qquad f(\vec a) \le \gamma - \epsilon < \gamma + \epsilon \le f(\vec b).
	\]
	% Dado un $f \in \dual(V, \R)$, entonces un conjunto de la forma $H_c := f^{-1}[ \{c\} ]$ se dice un \strong{hiperplano} de $V$.
	% Nótese que si $f(\vec v_0) = c$, entonces $H_c = \vec v_0 + H_0 = \vec v_0 + \ker f$.
	% Los conjuntos de la forma $f^{-1}\big[ [c, \infty) \big]$ y $f^{-1}\big[ (-\infty, c] \big]$ se dicen \strong{semi-espacios cerrados}.

	% Dado un punto $\vec x_0$ y un cerrado $S$, se dice que un hiperplano \strong{separa} a $S$ y $\vec x_0$ si están completamente contenidos en los
	% dos semi-espacios cerrados asociados a $H$.
\end{mydef}

\addtocounter{thmi}{1}
\begin{slem}
	Sea $V$ un espacio localmente convexo, sea $U \subseteq V$ un abierto convexo no vacío y sea $\vec x_0 \in V \setminus U$.
	Entonces existe $f \in V^*$ tal que $f(\vec u) \le f(\vec x_0)$ para todo $\vec u \in U$ y,
	en particular, el hiperplano cerrado $\{ \vec x : f(\vec x) = f(\vec x_0) \}$ separa a $U$ y al punto $\vec x_0$.
\end{slem}
\begin{proof}
	Apliquemos una traslación por $\vec y_0$ de modo que $U$ contenga al origen.
	Considere la función $p := p_U$ definida en el lema~\ref{thm:GHB_seminorm_lemma},
	considere el subespacio vectorial $F := \R\vec x_0 \le V$ y el funcional
	\[
		\tilde f \colon F \longrightarrow \R, \qquad t\vec x_0 \longmapsto t.
	\]
	Ya vimos que $\tilde f(\vec y) \le p(\vec y)$ para todo $\vec y \in F$, así que el teorema de Hahn-Banach nos permite extender $\tilde f$
	a un funcional $f \in L(V, \R)$ con $|f(\vec x)| \le p(\vec x)$.
	Finalmente, omo $p$ es continuo (¿por qué?) y $|f|$ está acotado por $p$, se comprueba que $f$ también es continuo.
	% Finalmente, como $p$ está acotado, vemos que $\|f\|_\infty \le M$ (donde $M$ es la constante del lema anterior),
	% por lo que $f$ es un funcional continuo.
\end{proof}
El lector puede notar que la norma está solamente empleada para probar que $f$ es continua, mediante la cota de $p$, pero que virtualmente
no tiene mayores usos.
En efecto, el lema anterior (y el teorema siguiente) son válidos para $\R$-EVTs en general, probando el lema~\ref{thm:GHB_seminorm_lemma}
con ayuda de los espacios vectoriales topológicos \emph{ordenados} (vid.\ \citeauthor{bourbaki:evt}~\cite[II.22\psqq]{bourbaki:evt}, \S II.3).
\addtocounter{thmi}{-1}

% \begin{lem}
% 	Sea $V$ un $\R$-EVT, $S \subseteq V$ un cerrado convexo y $\vec x_0 \notin S$;
% 	entonces existe un hiperplano afín cerrado que separa a $S$ y a $\vec x_0$.
% \end{lem}
% \begin{proof}
% 	Primero haremos el caso en que $V$ es de dimensión finita:
% 	Consideremos la aplicación $\vec v \mapsto \|\vec v - \vec x_0\|$, la cual alcanza un mínimo $\vec q \in S$
% 	tal que $\vec n := \vec q - \vec x_0$ tiene distancia mínima.
% 	Consideremos la aplicación lineal $f(\vec v) := \vec v\cdot\vec n$, donde $\cdot$ es el producto interno de $\R^n$,
% 	queremos probar que el semi-espacio $f^{-1}\big[ [\vec q\cdot\vec n, +\infty) \big]$ contiene completamente a $S$.
% 	% complementario claramente contiene a $\vec x_0$.
% 	Sea $\vec p \in S$ arbitrario, entonces para todo $t \in [0, 1]$ se cumple que
% 	$$ \|\vec q - \vec x_0\| \le \|\vec q + t(\vec p - \vec q) - \vec x_0\| = \| (\vec q - \vec x_0) + t(\vec p - \vec q) \| $$
% 	Luego elevando al cuadrado se obtiene que
% 	$$ 0 \le 2\vec n(\vec p - \vec q) + t(\vec p - \vec q)^2, $$
% 	y considerando el límite cuando $t \to 0$, se concluye que
% 	$$ \vec n\cdot\vec p \ge \vec n\cdot\vec q = \vec n^2 + \vec n\cdot \vec p > \vec n\cdot\vec p. $$
% 	Lo que concluye que están contenidos en semi-espacios distintos.
% 	\todo{Completar caso general, \cite[p.~86]{lang:analysis}.}
% \end{proof}

\begin{thmi}[Teorema de Hahn-Banach geométrico]\index{teorema!de Hahn-Banach!geométrico}
	Sea $V$ un espacio localmente convexo y sean $A, B \subseteq V$ un par de conjuntos convexos no vacíos disjuntos.
	\begin{enumerate}
		\item Si $A$ es abierto o $B$ es abierto, entonces existe un hiperplano cerrado que los separa.
		\item Si $V$ es normado, $A$ es cerrado y $B$ es compacto, entonces existe un hiperplano cerrado que los separa estrictamente.
	\end{enumerate}
\end{thmi}
\begin{proof}
	\begin{enumerate}
		\item Sea
			\[
				C := A - B = \{ \vec a - \vec b : \vec a \in A, \vec b \in B \},
			\]
			el cual es abierto (pues $C = \bigcup_{\vec y \in B} (A - \vec y)$), es convexo (¿por qué?) y $\Vec 0 \notin C$.
			Así, por el lema anterior, existe un funcional $f \in V^*$ tal que $f(\vec z) < 0$ para todo $\vec z \in C$, o equivalentemente,
			\[
				\forall \vec a\in A, \vec b \in B, \qquad f(\vec a) < f(\vec b).
			\]
			Así pues, sea $\gamma \in \R$ tal que
			\[
				\sup_{\vec a \in A} f(\vec a) \le \lambda \le \inf_{\vec b \in B} f(\vec b),
			\]
			se verifica que el hiperplano cerrado $\{ \vec x \in V : f(\vec x) = \lambda \}$ separa a $A$ y a $B$.
		\item Igual que antes, sea $C := A - B$, el cual es convexo, $\Vec 0 \notin C$ y es cerrado (¿por qué?).
			Así que existe $r > 0$ tal que $B_r(\Vec 0) \cap C = \emptyset$ y, por el inciso anterior, existe un hiperplano $H$
			que separa a la bola $B_r(\Vec 0)$ y a $C$.
			Queda al lector verificar que una traslación de $H$ es precisamente quién separa estrictamente a $A$ y a $B$. \qedhere
	\end{enumerate}
\end{proof}
\begin{cor}
	Sea $V$ un $\R$-espacio normado y sea $F \le V$ tal que $\overline{F} \ne V$.
	Entonces existe un funcional $f \in V^*$ no nulo que se anula en $F$.
\end{cor}
\begin{proof}
	Sea $A := \overline{F}$ y $B := \{ \vec x_0 \}$ para algún $\vec x_0 \notin \overline{A}$.
	Por el teorema geométrico de Hahn-Banach, existe un hiperplano cerrado $\{ \vec x \in V : f(\vec x) = \gamma \}$
	que separa estrictamente a $A$ y a $B$, es decir
	\[
		\forall \vec y \in F \qquad f(\vec y) < \gamma < f(\vec x_0).
	\]
	Ahora bien, para todo $\lambda \in \R$ se cumple que $f(\lambda\vec y) = \lambda f(\vec y) < \gamma$,
	ya que $\overline{F} \le V$, por lo que necesariamente $f(\vec y) = 0$ para $\vec y \in \overline{F} \supseteq F$.
\end{proof}

El enunciado del teorema geométrico de Hahn-Banach es <<óptimo>>:
\begin{exn}
	En $\R^2$ podemos considerar (ver fig.~\ref{fig:top/geometric_HB})
	\[
		A := \left\{ (x, y) : x > 0, y > \frac{1}{x} \right\}, \qquad B := \{ (x, y) : x \le 0 \}.
	\]
	Aquí $A$ y $B$ son convexos (¿por qué?), y el hiperplano (cerrado) $H = \{ (x, y) : x = 0 \}$ los separa,
	pero no existe un hiperplano que los separa estrictamente (¡demuéstrelo!).
	\begin{figure}[!hbtp]
		\centering
		\includegraphics{top/geometric_HB.pdf}
		\caption{}%
		\label{fig:top/geometric_HB}
	\end{figure}
\end{exn}
\begin{ex}
	Sea $E$ un $\R$-EVT localmente convexo metrizable que no es completo.
	Podemos considerar a $E$ como subespacio de su compleción $\widehat{E}$.
	Sea $y_0 \in \widehat{E} \setminus E$, y considere los subconjuntos $A := E$ y $B := \{ y_0 \}$.
	Claramente $A$ y $B$ son convexos, pero $A$ no es cerrado, y no existe ningún hiperplano cerrado que los separa ya que $A$ es denso.
\end{ex}

\begin{mydef}
	Sea $V$ un $\R$-espacio vectorial y $S \subseteq V$ convexo.
	Un punto $\vec x \in V$ se dice un \strong{extremo} si $\vec x = t\vec y_1 + (1 - t)\vec y_2$ con
	$t \in [0, 1]$ y $\vec y_1, \vec y_2 \in S$ implica que $\vec y_1 = \vec y_2 = \vec x$.
\end{mydef}

\thmdep{AE}
\begin{thm}
	Sea $V$ un $\R$-EVT.
	Entonces todo subconjunto compacto convexo no vacío $S$ de $V$ posee extremos.
\end{thm}
\begin{proof}
	Sea $ \mathcal{F} $ la familia de subconjuntos $K$ de $S$ que sean convexos, compactos, no vacíos,
	tales que si $\vec x \in K$ y $\vec y_1,\vec y_2 \in S$ satisfacen que $\vec x = t\vec y_1 + (1 - t)\vec y_2$
	para algún $t \in [0, 1]$, entonces $\vec y_1, \vec y_2 \in K$.

	Sea $\{ K_i \}_{i\in I} \subseteq \mathcal{F}$ es una $\subseteq$-cadena (descendiente), entonces
	$$ K := \bigcap_{i\in I} K_i \in \mathcal{F}, $$
	para ello nótese que como los $K_i$'s son de Hausdorff y compactos, entonces, fijando $j \in I$
	se cumple que todo $i\in I$ tal que $K_i \subseteq K_j$ debe cumplir que $K_i$ sea un subespacio convexo cerrado de $K_j$.
	Luego $K$ es una intersección de cerrados, así que es cerrado; por compacidad tampoco es vacío puesto que la familia tiene la PIF,
	luego $K$ es compacto y ya hemos visto que la intersección de convexos es convexa.
	Además es fácil ver que $K$ también posee la última propiedad, luego $K \in \mathcal{F}$.
	Finalmente por el lema de Zorn (versión descendiente) se cumple que $\mathcal{F}$ posee un $\subseteq$-minimal $S_0$.

	Hay que probar que $S_0$ consta de un sólo elemento, que sería un extremo de $S$.
	Para ello, veremos que para todo $\lambda \in \dual(E, \R)$ se satisface que $\lambda[S_0]$ consta de un solo punto.
	Como $S_0$ es compacto, entonces $\lambda[S_0]$ también y luego podemos elegir el máximo $c$ y ver que $S_0 \cap \lambda^{-1}[ \{c\} ] \in \mathcal{F}$:
	Para ello, claramente $S_0$ es compacto y convexo, y para la última propiedad veamos que si
	$$ \vec x = t\vec y_1 + (1 - t)\vec y_2 \in S_0 \cap \lambda^{-1}[ \{c\} ] $$
	luego aplicando $\lambda$ a ambos lados se tiene que necesariamente $\lambda(\vec y_1) = \lambda(\vec y_2) = c$, por definición de máximo.
	Finalmente, por minimalidad de $S_0$, entonces $\lambda[S_0] = \{c\}$.
	\par
	Queda al lector comprobar que $\dual(E, \R)$ separa puntos: vale decir, que para todo par de puntos distintos podemos encontrar un funcional
	continuo que toma valores distintos en dos puntos distintos.
\end{proof}

\begin{cor}
	Sea $V$ un $\R$-EVT, sea $\emptyset \ne K \subseteq V$ compacto y convexo, y sea $\lambda \in \dual(V, \R)$.
	Si $c$ es un extremo de $\lambda[K]$, entonces $K \cap \lambda^{-1}[ \{c\} ]$ posee un extremo de $K$.
\end{cor}

\begin{thm}[de Krein-Milman]%
	% \index{teorema!de Krein-Milman}
	Sea $V$ un $\R$-EVT y $\emptyset \ne K \subseteq V$ compacto y convexo.
	Entonces si $S$ es el conjunto de los extremos de $K$, entonces $K = \overline{\conv S}$.
\end{thm}
\begin{proof}
	Llamemos $K' := \overline{\conv S}$, claramente $K' \subseteq K$.
	Como $K'$ es cerrado en $K$ que es compacto, entonces $K'$ es compacto y es claramente convexo por definición,
	luego si $\vec x_0 \in K \setminus K'$, entonces existe $\lambda \in \dual(V, \R)$ que separa a $\vec x_0$ y $K'$,
	por lo que, sin perdida de generalidad supondremos que $\lambda(\vec x_0) > \lambda(\vec v)$ para todo $\vec v\in K'$.
	No obstante, si $c$ es el máximo de $\lambda[K]$, entonces existe un extremo en $K \cap \lambda^{-1}[ \{c\} ]$,
	luego existe que $\vec y \in K'$ tal que $\lambda(\vec y) = c \ge \lambda(\vec x_0)$; lo que es absurdo.
\end{proof}
% Se ha comprobado el teo
\thmdep{}

\section{Topologías débiles y dualidad}
\begin{mydef}
	Sea $k$ un cuerpo y sean $V, W$ un par de $k$-espacios vectoriales.
	Dada una forma bilineal $\beta \colon V\times W \to k$, se dice que $V, W$ \emph{están en dualidad} (respecto a $\beta$).
	Se dice que la dualidad es \strong{separante en $V$} (resp.\ \emph{en $W$}) si para todo $\vec x \in V$ (resp.\ $\vec y \in W$) no nulo
	la aplicación lineal $\beta(\vec x, -) \colon W \to k$ (resp.\ $\beta(-, \vec y)$) es no nula.
\end{mydef}
\begin{ex}
	Sea $V$ un $k$-espacio vectorial arbitrario.
	Dado un subespacio vectorial $F \le V^\wedge$, se cumple que naturalmente $V, F$ están en dualidad con
	\[
		\langle -, - \rangle \colon V\times F \longrightarrow k, \qquad (\vec x, f) \longmapsto f(\vec x).
	\]
	Es claro que esta dualidad siempre es separante en $F$.
\end{ex}

\begin{cor}
	Sean $V, W$ un par de espacios vectoriales en dualidad sobre un cuerpo $k$.
	La dualidad es separante en $V$ syss la aplicación $k$-lineal
	\[
		V \longrightarrow W^\wedge, \qquad \vec x \longmapsto \beta(\vec x, -)
	\]
	es inyectiva.
\end{cor}

\begin{mydefi}
	Sea $K$ un cuerpo topológico y sean $E, F$ un par de $K$-espacios vectoriales en dualidad.
	La \strong{topología $\sigma(E, F)$} sobre $E$ es la topología inicial inducida por las funciones $\beta(-, \vec y_0)$,
	donde $\vec y_0$ recorre los elementos de $F$.
	Ocasionalmente, escribiremos $E_\sigma$ (o $F_\sigma$) para enfatizar la topología.

	Dado un $K$-EVT $V$, la \strong{topología débil} sobre $V$ es la topología $\sigma(V, V^*)$,
	mientras que la \strong{topología débil*}\index{topología!debil@débil(*)} sobre $V^*$ es la topología $\sigma(V^*, V)$;
	donde la forma bilineal es la canónica.
	% \[
	% 	\langle -, - \rangle \colon V \times V' \longrightarrow K, \qquad (\vec x, f) \longmapsto f(\vec x)
	% \]
\end{mydefi}

\begin{prop}
	Sea $K$ un cuerpo topológico de Hausdorff y $E, F$ un par de $K$-espacios vectoriales en dualidad.
	Se cumple que $E_\sigma$ (resp.\ $F_\sigma$) es un espacio de Hausdorff syss la dualidad es separante en $E$ (resp.\ $F$).
\end{prop}
\begin{proof}
	Se sigue de la proposición~\ref{thm:top_def_seminorm_set}.
\end{proof}
\begin{cor}
	Sea $K$ un cuerpo topológico de Hausdorff y sea $V$ un $K$-EVT.
	El dual topológico $V^*_\sigma$ es Hausdorff en la topología débil*.
\end{cor}

Se recomienda al lector intentar lo siguiente:
\begin{prob}
	Sea $V$ un $\R$-espacio normado.
	Demuestre que $V_\sigma$ (es decir, $V$ con la topología débil) es Hausdorff.
\end{prob}
% \begin{hint}
% 	Emplee el teorema de Hahn-Banach geométrico.
% \end{hint}

\begin{prop}
	Sean $E, F$ un par de $K$-espacio vectorial en dualidad, separante en $E$ y $F$.
	Dado un conjunto de elementos linealmente independientes $\vec y_1, \dots, \vec y_n \in F$,
	existen $\vec x_j \in E$ tales que $\beta(\vec x_i, \vec y_j) = \delta_{ij}$.
\end{prop}
\begin{proof}
	Procedemos por inducción, donde el caso $n = 1$ se sigue de que $\beta(-, \vec y_1)$ no es la función nula.
	Así, supongamos que existen $\overline{\vec x}_1, \dots, \overline{\vec x}_{n-1}$ tales que $\beta(\overline{\vec x}_i, \vec y_j) = \delta_{ij}$.
	Sean
	\[
		S_n := \sum_{i=1}^{n-1} K\overline{\vec x}_i, \qquad T_n := \bigcap_{i=1}^{n-1} \ker\beta(-, \vec y_i).
	\]
	Entonces como $E \cong S_n \oplus T_n$ (algebraicamente),
	ahora bien, $\beta(-, \vec y_n)$ no es la función nula sobre $T_n$, pues de lo contrario, $\vec y_n$ sería linealmente dependiente
	al resto y, por tanto, existe $\vec x_n \in T_n$ tal que $\beta(\vec x_n, \vec y_n) = 1$.
	Así $\vec x_j := \overline{\vec x}_j - \beta(\overline{\vec x}_j, \vec y_n)\vec x_n$ satisface lo exigido.
\end{proof}

\begin{prop}
	Sea $K$ un cuerpo topológico y sean $E, F$ un par de $K$-espacios vectoriales en dualidad.
	Un funcional $f \in L(E, K)$ es continuo en la topología $\sigma(E, F)$ syss es de la forma $f = \beta(-, \vec y_0)$
	para algún $\vec y_0 \in F$.
	Además, si la dualidad es separante en $E$, entonces el $\vec y$ es único.
\end{prop}

\begin{thm}
	Sea $K$ un cuerpo topológico y sea $V$ un $K$-espacio normado.
	Entonces
	\[
		\iota := \ev_- \colon V \longrightarrow V^{**}, \qquad \vec x \longmapsto (\ell \mapsto \ell(\vec x))
	\]
	es una isometría.
	Además, si $V$ es de Banach, entonces $\Img\iota \le_f V^{**}$.
\end{thm}

\section*{Notas históricas y referencias}
La teoría de espacios conexos es bastante antigua, más incluso que la exploración de los axiomas de separación, y data de \textbf{Camille Jordan},
conocido por su famoso teorema de las curvas planares a finales de siglo \textsc{xix}.
Por el contrario, la teoría de espacios \emph{disconexos} es más reciente y puede trazarse a la década de 1930;
una de las mayores inspiraciones para los límites inversos de espacios topológicos y los espacios hereditariamente disconexos son
ciertos espacios de números $p$-ádicos.

Los espacios vectoriales topológicos están estudiados en varias fuentes, aunque la mayoría se restringe al caso de $\R$ y $\C$.
No obstante, la gran mayoría de analistas saben que la teoría se puede llevar a cabo en mayor generalidad,
\citeauthor{bourbaki:evt}~\cite{bourbaki:evt} es de las pocas fuentes que sí lo hacen.

\printbibliography[segment=\therefsegment, check=onlynew, notcategory=history, notcategory=historical, notcategory=other]
% \bibbycategory[segment=\therefsegment, check=onlynew]

\end{document}
