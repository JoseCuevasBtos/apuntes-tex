\documentclass[topologia-analisis.tex]{subfiles}
\begin{document}

\chapter{Espacios topológicos con estructura}

\section{Espacios (pseudo)métricos}
\begin{mydefi}[Espacio métrico]\index{espacio!métrico}
	Sea $X$ un espacio y $d : X\times X \to [0, +\infty)$ una función tal que para todos $x,y\in X$ satisfaga:
	\begin{enumerate}
		\item $d(x, x) = 0$.
		\item $d(x, y) = d(y, x)$ (simetría).
		\item $d(x, z) \le d(x, y) + d(y, z)$ (desigualdad triangular).
	\end{enumerate}
	Entonces $d$ se dice una \strong{pseudométrica}.
	Si $x \ne y$, pero $d(x, y) = 0$, entonces decimos que $x,y$ son \strong{indistinguibles}.
	Una pseudométrica es una \strong{métrica} si no posee puntos indistinguibles.
	Un par $(X, d)$ donde $d$ es una (pseudo)métrica se dice un espacio (pseudo)métrico.

	Se definen:
	$$ B_r(x) := \{ y\in X : d(x, y) < r \}, \quad \overline B_r(x) := \{ y\in X : d(x, y) \le r \} $$
	donde a la primera se le dice \strong{bola abierta} centrada en $x$ de radio $r$ y a la segunda \strong{bola cerrada}.
	La topología estandar a la que dotaremos a un espacio pseudométrico, es aquella que tiene por subbase a las bolas abiertas.
\end{mydefi}

Bajo ésta definición es claro que las bolas abiertas son abiertas.
Veamos más:
\begin{prop}
	En un espacio pseudométrico, las bolas cerradas son efectivamente conjuntos cerrados.
\end{prop}
\begin{proof}
	Vamos a probar que $\overline B_r(x) =: F$ es cerrado.
	Para ello hay que ver que $F^c$ sea abierto.
	Sea $y \in F^c$, esto quiere decir que $d(x, y) > r$, luego sea $s := d(x, y) - r > 0$, entonces veremos que $B_s(y) \subseteq F^c$.
	Sea $z \in B_s(y)$, por definición se cumple que $d(y, z) < s$, luego:
	\begin{equation}
		d(x, z) \ge d(x, y) - d(y, z) > r - s = d(x, y) - ( d(x, y) - r ) = r. \tqedhere
	\end{equation}
\end{proof}
Estaríamos tentados a decir que $ \overline{ B_r(x) } = \overline B_r(x) $, pero ésta igualdad no siempre se da.

\begin{exn}[$D_2$]
	\addnamedexample{$D_2$}{Un espacio donde $\overline{ B_1(x) } \ne \overline B_1(x)$}
	Consideremos a $D_2 = \{a, b\}$ [el espacio discreto de dos elementos], cuya topología viene inducida por la métrica discreta
	$$ d(x, y) =
	\begin{cases}
		0, &x=y \\
		1, &x\ne y
	\end{cases} $$
	Luego sea $B_1(a) = \{ y : d(a, y) < 1 \} = \{a\}$ y notemos que $B_1(a)$ es cerrado en éste espacio (pues todo conjunto es abierto y cerrado en el
	espacio discreto), luego
	$$ \overline{B_1(a)} = \{a\} \ne \{a, b\} = \overline B_1(a). $$
\end{exn}

\begin{mydef}
	Se dice que un espacio pseudométrico es \strong{métricamente denso} si para todo par de puntos distintos y distinguibles $x, y$ con $r := d(x, y)$,
	y para todo real $s > 0$ existe un tercer punto $z$ tal que $0 < d(x, z) < s$ y $d(z, y) < r$ (ver fig.~\ref{fig:metrically-dense}).
\end{mydef}
\begin{figure}[!hbt]
	\centering
	\includegraphics[scale=1]{metrically-dense.pdf}
	\caption{Representación de ser métricamente denso.}%
	\label{fig:metrically-dense}
\end{figure}

\begin{thm}
	Dado un espacio pseudiométrico $X$, son equivalentes:
	\begin{enumerate}
		\item $X$ es métricamente denso.
		\item $\overline{ B_r(x) } = \overline B_r(x)$ para todo $x\in X$ y todo $r > 0$.
		\item $\Fr B_r(x) = \{ y : d(x, y) = r \}$.
	\end{enumerate}
\end{thm}
\begin{proof}
	Es claro que $(2) \iff (3)$.
	\par
	$(1) \implies (2)$.
	Ya vimos que $F := \overline B_r(x)$ es cerrado, así que basta ver que todo punto de $F$ es adherente a $B_r(x)$.
	Sea $y \in F$, si $x,y$ son indistinguibles entonces es claro que $y$ es adherente, sino sea $s > 0$,
	por definición de métricamente denso existe $z$ tal que $z \in B_s(y) \cap B_r(x)$, es decir, $y$ es adherente a $B_r(x)$.
	\par
	El recíproco queda de ejercicio al lector.
\end{proof}

\begin{thmi}
	Sean $X, Y$ espacios pseudométricos, entonces son equivalentes:
	\begin{enumerate}
		\item $f \colon X \to Y$ es continua.
		\item Para todo $x \in X$ y todo $\epsilon > 0$ existe un $\delta > 0$ tal que $0 < d(x, y) < \delta \implies d( f(x), f(y) ) < \epsilon$.
	\end{enumerate}
\end{thmi}
\begin{hint}
	Basta notar que los abiertos básicos son bolas abiertas.
\end{hint}

\begin{lem}
	Dado $X$ un espacio pseudométrico, $x,y\in X$ y $A \subseteq X$ no vacío, entonces se satisface:
	$$ | d(x, A) - d(y, A) | \le d(x, y) $$
\end{lem}
\begin{proof}
	Sea $a \in A$, por definición y por desigualdad triangular se cumple
	$$ d(x, A) \le d(x, a) \le d(x, y) + d(y, a) $$
	y como la desigualdad de los extremos se cumple para todo $a\in A$ se tiene que $d(x, A) - d(x, y)$ es una cota inferior de $d(y, a)$, por lo que,
	por definición de ínfimo se tiene que
	$$ d(x, A) - d(x, y) \le d(y, A) \iff d(x, A) - d(y, A) \le d(x, y) $$
	y análogamente se prueba que $d(y, A) - d(x, A) \le d(x, y)$.
\end{proof}
\begin{thm}
	Sea $X$ un espacio pseudométrico y $A \subseteq X$ no vacío, entonces la función $d(-, A) \colon X \to [0, \infty)$ es continua.
\end{thm}

Sin embargo, nótese que la propiedad de <<ser métrico>> no es estrictamente topológica, ya que no es intrínseca a la topología misma del espacio,
por ello se introducen las dos siguientes definiciones:
\begin{mydefi}
	Un espacio topológico $X$ se dice \strong{metrizable} si existe una métrica $d$ tal que induce la misma topología.
	% Además, $X$ se dice \strong{topológicamente completo}\index{completo (topológicamente)} si es metrizable, y existe una métrica $d$
	% que es completa e induce su topología
\end{mydefi}

\begin{thm}
	Todo espacio metrizable es perfectamente normal.
\end{thm}

\begin{prop}
	Si un espacio es metrizable, entonces su topología está inducida por una métrica acotada.
\end{prop}
\begin{proof}
	Sea $X$ un espacio topológico y $d$ la métrica que induce su topología.
	Denótese $\underline{d}$ a la métrica dada por:
	$$ \underline{d}(x, y) := \min\{1, d(x, y)\}. $$
	Es fácil notar que efectivamente es una métrica, y $ \mathcal{B} := \{ B(x, 1/n) : x\in X, n\in\N_{\ne 0} \} $ es una base para ambas topologías,
	como las bases inducen una única topología se concluye que concuerdan.
\end{proof}
En general denotaremos $\underline{d}$ a la métrica dada en la demostración.

\begin{thm}
	La suma de espacios métricos es métrico.
\end{thm}
\begin{proof}
	Sea $Y := \bigoplus_{i\in I} X_i$ y sea $d_i$ la métrica sobre $X_i$ resp.
	Entonces definimos $d$ sobre $Y$ como:
	$$ d(x, y) :=
	\begin{cases}
		\underline{d}_i(x, y) & \exists i\in I : x,y\in X_i \\
		1		     & \text{en otro caso}
	\end{cases} $$
	Luego es claro que la base inducida por $d$ son abiertos de cada $X_i$ o $Y$ entero, lo que induce la topología de la suma.
\end{proof}

\begin{thm}
	El producto a lo más numerable de espacios métricos es métrico.
\end{thm}
\begin{proof}
	Sean $(X_i, d)$ espacios métricos e $Y$ su producto, veamos dos casos:
	\begin{enumerate}[a)]
		\item \underline{Producto finito:}
			Es claro que
			$$ d(\vec x, \vec y) := \sum_{i=1}^n d_i(x_i, y_i) $$
			es una métrica que induce la topología producto.

		\item \underline{Producto numerable:}
			Acá hacemos algo parecido y definimos
			$$ d(\vec x, \vec y) := \sum_{i=1}^n \frac{1}{2^i} \underline{d}_i(x_i, y_i) $$
			que notemos está acotada por $ \frac{1}{2} + \frac{1}{4} + \frac{1}{8} + \cdots = 1 $ e induce la topología producto. \qedhere
	\end{enumerate}
\end{proof}
Cabe destacar que el caso infinito y general de la suma y producto de espacios metrizables depende del AE para elegir una métrica,
pero usualmente trabajaremos con espacios donde la métrica venga elegida de antemano de modo que no habrá necesidad de invocarlo.

\begin{cor}
	El cubo de Hilbert $[0,1]^{\N}$ es metrizable.
\end{cor}

\thmdep{DE}
\begin{thmi}[Teorema de metrización de Urysohn]\index{teorema!de metrización de Urysohn}
	% Todo espacio regular 2AN es metrizable.
	Un espacio 2AN es metrizable syss es regular.
\end{thmi}
\begin{proof}
	$\implies$. Nótese que
	$$ \rm metrizable \implies \text{perfectamente normal} \implies regular. $$
	$\impliedby$. Se cumple que
	$$ \rm regular + 2AN \overset{ \text{(\ref{thm:regular-2AN-normal})} }{ \implies } normal \implies Tychonoff $$
	Y sabemos que todo espacio de Tychonoff 2AN está encajado en el cubo de Hilbert (\ref{thm:tychonoff-universal}) que es metrizable.
\end{proof}

\begin{thm}
	Un espacio de Hausdorff compacto es metrizable syss es 2AN.
\end{thm}
\begin{proof}
	$\impliedby$. Se cumple que
	$$ \rm Hausdorff + compacto \overset{ \text{(\ref{thm:hausdorff-compact-normal})} }{ \implies } normal \implies regular $$
	y todo espacio regular 2AN es metrizable por el teorema de metrización de Urysohn.
	\par
	$\implies$. Si $X$ es compacto entonces es de Lindelöf y si es de Lindelöf y métrico entonces es 2AN.
\end{proof}
\thmdep{}

Y un teorema curioso:
\begin{thm}[Alexandroff-Hausdorff]\index{teorema!de Alexandroff-Hausdorff}
	Sea $X$ un espacio metrizable compacto.
	Entonces existe una función $f\colon C \to X$, donde $C$ es el conjunto de Cantor, que es continua y suprayectiva.
\end{thm}
\begin{proof}
	En primer lugar, recuerde que $C := \{ 0, 1 \}^\N$, de modo que se comprueba que $C$ es homeomorfo a $C^n$ para todo $n \in \N_{\ne 0}$
	y $C^{\aleph_0}$.
	Comencemos por probar algo más débil: hay una función continua suprayectiva $g\colon C \to [0, 1]$.
	Los elementos de $C$ pueden verse como sucesiones $\vec x := (x_1, x_2, x_3, \dots) \in C$ donde $x_i \in \{ 0, 1 \}$, luego construimos
	$$ g(\vec x) := \sum_{i=1}^{\infty} \frac{x_i}{2^i}, $$
	y notamos que, por representación en base 2, se cumple que es suprayectiva.
	Para la continuidad, notemos que todo $y \in [0, 1]$ está contenido en algún intervalo de la forma $\left( \frac{k}{2^m}, \frac{k+1}{2^{m+1}} \right)$.
	Existe alguna representación $\vec x = (x_1, x_2, \dots, x_n, 0, 0, \dots)$ tal que $g(\vec x) = \frac{k}{2^m}$.
	Luego la preimagen corresponde al conjunto
	$$ \{ (x_1, x_2, \dots, x_n) \} \times \{ 0, 1 \}^\N \setminus \{ (x_1, x_2, \dots, x_n, 1, 1, \dots) \}, $$
	el cual sí es abierto (el primer conjunto es claramente abierto y los puntos son cerrados en $C$ pues es Hausdorff).

	Luego, consideramos la función $g\times g\colon C^2 \to [0, 1]^2$ la cual es continua y suprayectiva, y recordamos que $C^2 \approx C$.
	Y así, podemos construir en general una función $h := g^{\aleph_0} \colon C \to [0, 1]^{\aleph_0} =: H$ el cual es cubo de Hilbert.
	Todo espacio métrico compacto es de Tychonoff y 2AN, luego está encajado en el cubo de Hilbert digamos por $\iota\colon X \to H$.
	Como $X$ es compacto, entonces $\iota[X]$ es cerrado en $H$.

	Ahora probaremos que para todo $A \subseteq C$ cerrado se cumple que existe $f\colon C \to A$ continua con $f|_A = \Id_A$.
	Construiremos $f_n\colon C \to \{ 0, 1 \}^n$ por recursión como prosigue:
	Sea $\vec x \in C$, definimos $f_1(\vec x) := x_1$ si existe $\vec a \in A$ con $a_1 = x_1$, y definimos $f_1(\vec x) := 1 - x_1$ en otro caso.
	Si tenemos definido $f_n(\vec x)$ definiremos $f_{n+1}(\vec x) := (f_n(\vec x), x_{n+1})$ si existe $\vec a \in A$ tal que
	$(a_1, a_2, \dots, a_{n+1}) = (f_n(\vec x), x_{n+1})$ y $f_{n+1}(\vec x) := (f_n(\vec x), 1 - x_{n+1})$ en otro caso.

	Así, podemos ver que para todo $n$ y todo $\vec x$ se cumple que $f_n(\vec x)$ coincide con las primeras $n$-ésimas coordenadas
	de algún $\vec a \in A$.
	Definimos $f(\vec x) := (\pi_1 f_1(\vec x), \pi_2 f_2(\vec x), \dots)$ y vemos que es una función $f\colon C \to A$ tal que $f|_A = \Id_A$;
	falta comprobar la continuidad de $f$.
	Para ello nótese que $f$ es continua syss $f\circ\pi_n\colon C \to \{ 0, 1 \}$ lo es para todo $n\in\N_{\ne 0}$,
	pero $\pi_n(f(\vec x)) = \pi_n(f_n(\vec x))$, el cual sólo depende de las $n$-ésimas primeras coordenadas $(x_1, \dots, x_n)$,
	luego es claro ver que es continua.

	Finalmente definamos $A := h^{-1}[ \iota X ]$ el cual es un cerrado en $C$ por continuidad de $h$ y se cumple que existe
	$f\colon C \to A$ continua y suprayectiva, así que $f\circ h\circ\iota^{-1} \colon C \to X$ es continua y suprayectiva.
\end{proof}

% \begin{mydef}
%	 Un espacio es \strong{de Peano}\index{espacio!de Peano} si es metrizable, compacto, conexo y localmente conexo.
% \end{mydef}
% \begin{thm}
%	 Todo espacio de Peano es arcoconexo.
% \end{thm}
% \begin{proof}

% \end{proof}

\begin{lem}
	Sea $X, Y$ un par de espacios métricos con $Y$ completo, sea $A \subseteq X$ un subconjunto y $f \colon A \to Y$ una función continua.
	Existe $A \subseteq A^* \subseteq X$ que es un conjunto $G_\delta$ y existe una extensión continua $f^* \colon A^* \to Y$ de $f$.
\end{lem}
\begin{proof}
	Definamos la \emph{oscilación} de una función $g \colon A \to Y$ cualquiera en un abierto $U \subseteq X$ que corta a $A$ como
	\[
		\operatorname{osc}(g, U) := \diam g[U\cap A] = \sup\{ d(g(x), g(y)) : x, y \in U \cap A \} \le \infty.
	\]
	Para $x \in \overline{A}$ definamos
	\[
		\operatorname{osc}(g, x) := \inf\{ \operatorname{osc}(g, U) : U \text{ es un entorno de } x \}.
	\]
	Así, $A^* := \{ x \in \overline{A} : \operatorname{osc}(f, x) = 0 \}$ es un conjunto $G_\delta$, ya que
	\[
		B_n := \{ x \in \overline{A} : \operatorname{osc}(g, x) < 1/n \}
	\]
	es un abierto (¿por qué?) y claramente $A^* = \bigcap_{n=1}^\infty B_n$.
	Más aún, es fácil probar que una función $g$ es continua syss $\operatorname{osc}(g, x) = 0$ para todo $x \in A$,
	por lo que $A^* \supseteq A$ es un $G_\delta$ y $f$ se extiende continuamente a $A^*$.
	La completitud es necesaria para construir dicha extensión.
\end{proof}
\begin{thm}[Lavrentieff]
	Sean $X, Y$ un par de espacios métricos completos, sean $A \subseteq X, B \subseteq Y$ un par de subespacios y sea $f \colon A \to B$ un homeomorfismo.
	Entonces existen $A \subseteq A^* \subseteq X, B \subseteq B^* \subseteq Y$ subconjuntos $G_\delta$ tal que $f$ se extiende
	a un homeomorfismo $f^* \colon A^* \to B^*$.
\end{thm}
\begin{proof}
	Por el lema anterior, existen extensiones continuas $f^* \colon A_0 \to Y$ y $(f^{-1})^* \colon B_0 \to X$,
	donde $A \subseteq A_0 \subseteq X$ y $B \subseteq B_0 \subseteq Y$ son subconjuntos $G_\delta$.
	Definamos
	\[
		A^* = (f^*)^{-1}[B_0] = \{ x \in A_0 : f^*(x) \in B_0 \},
		\quad B^* = \{ y \in B_0 : (f^{-1})^*(y) \in A_0 \}
	\]
	los cuales son $G_\delta$ por ser preimágenes continuas de $G_\delta$.

	Veamos que $f^*[A^*] = B^*$ y que es un homeomorfismo.
	Para ello, nótese que la composición $f^* \circ (f^{-1})^* \colon A^* \to X$ es continua y se restringe a la identidad en el denso $A$,
	de modo que ha de ser la función identidad $\Id_{A^*} \colon A^* \to A^*$ como se quiere ver.
	Análogamente, $(f^{-1})^* \circ f^* = \Id_{B^*}$.
\end{proof}

\begin{thm}
	Todo $G_\delta$ en un espacio métrico $X$ es homeomorfo a un cerrado de $X \times \R^\omega$.
\end{thm}
\begin{cor}\label{thm:Gd_sub_compl_is_compl}
	Sea $X$ un espacio métrico y $A \subseteq X$.
	\begin{enumerate}
		\item Si $X$ es completamente metrizable y $A$ es un $G_\delta$, entonces $A$ también es completamente metrizable.
		\item Si $A$ es completamente metrizable, entonces es un $G_\delta$ de $X$.
	\end{enumerate}
\end{cor}

\section{Espacios de Baire y espacios \v Cech-completos}\label{sec:cech_complete}
\begin{mydefi}
	Un espacio $X$ se dice \strong{completamente metrizable} si existe una métrica $d$ sobre $X$ que es completa y que induce su topología.
\end{mydefi}
Ojo que inicialmente el lector pensará que un espacio métrico, como $\Q$, no es topológicamente completo puesto que su métrica no lo es;
sin embargo, la definición implica que \textit{alguna} métrica sea completa, así que el argumento no es tan directo.
Aún así, $\Q$ no es completamente metrizable, pero necesitamos caracterizar mejor la noción de lo que significa \textit{topológicamente}
la cualidad de ser <<completamente metrizable>>.

Un ejemplo es que el intervalo $(0, 1)$ no es completo con la métrica usual, sin embargo, si es completamente metrizable puesto que es homeomorfo a $\R$.

\begin{prop}
	Todo espacio compacto y metrizable es completamente metrizable.
\end{prop}
El recíproco no es cierto, basta notar que $\R$ es completamente metrizable y no compacto.

\begin{mydef}
	Sea $A \subseteq X$, se le dice un conjunto:
	\begin{description}
		\item[Diseminado]\index{diseminado (conjunto)} Si $\Int(\overline A) = \emptyset$.
		\item[Primera categoría] Si es la unión numerable de diseminados.
		\item[Segunda categoría] Si no es de primera categoría.
	\end{description}
\end{mydef}
La idea es que los conjuntos diseminados son <<pequeños>>, por ejemplo, en $\R$ los conjuntos diseminados son cosas como los conjuntos finitos,
$\Z$, $\{1/n : n>0\}$, etc. Un conjunto abierto no vacío nunca es diseminado.
$\Q$ no es diseminado, pero sí es de primera categoría y además también tiene interior vacío.

\begin{prop}
	Sea $A \subseteq X$, entonces son equivalentes:
	\begin{enumerate}
		\item Ser diseminado.
		\item $X \setminus \overline A$ es un abierto denso.
		\item Existe $F$ cerrado tal que $A \subseteq F$ y $\Int F = \emptyset$.
		\item Todo abierto no vacío $U$ de $X$ contiene otro abierto no vacío $V$ tal que $A \cap V = \emptyset$.
	\end{enumerate}
\end{prop}
\begin{proof}
	Es claro que $1 \iff 2 \iff 3$.

	$1\implies 4$. Sea $U$, como $X \setminus \overline A$ es un abierto denso, entonces $V := U\cap(X\setminus\overline A)$ es un abierto
	no vacío tal que $A\cap V = \emptyset$.

	$4\implies 2$. Nótese que si $V \subseteq U$ satisface que $A\cap V = \emptyset$, entonces
	\[
		\overline{A} \cap V \subseteq \overline{A\cap V} = \overline\emptyset = \emptyset.
	\]
	Luego $\emptyset \ne V \subseteq U \cap (X \setminus \overline A)$, por lo que $X \setminus \overline A$ es denso y es claramente abierto.
\end{proof}

\begin{prop}
	Se cumplen las siguientes:
	\begin{enumerate}
		\item Todo abierto no vacío es no diseminado.
		\item $\emptyset$ es de primera categoría.
		\item Todo subconjunto de un conjunto de primera categoría es también de primera categoría.
		\item La unión numerable de conjuntos de primera categoría es también de primera categoría.
	\end{enumerate}
\end{prop}
Sin embargo, nos gustaría extender la condición 1 por <<Todo abierto no vacío es de segunda categoría>>, ésto es la base de la noción de ésta sección:

\addtocounter{thmi}{1}
\begin{slem}
	En un espacio topológico son equivalentes:
	\begin{enumerate}
		\item Todo abierto no vacío es de segunda categoría.
		\item Todo conjunto de primera categoría tiene interior vacío.
		\item Toda unión numerable de cerrados de interior vacío tiene interior vacío.
		\item Toda intersección numerable de abiertos densos es también densa.
	\end{enumerate}
\end{slem}
% \begin{proof}
%	 Es claro que $1\iff 2\implies 3\iff 4$.
% \end{proof}
\addtocounter{thmi}{-1}

\begin{mydefi}
	Un espacio se dice \strong{de Baire}\index{espacio!de Baire} si cumple cualquiera de las condiciones del lema anterior.
\end{mydefi}
Claramente todo espacio de Baire es de segunda categoría, pero el converso no es cierto:

\begin{exn}
	\addnamedexample{$\R \amalg \Q$}{Un espacio de segunda categoría que no es de Baire}
	Consideremos a $X := \R \amalg \Q = (\R\times\{0\}) \cup (\Q\times\{1\})$.
	\par
	Nótese que $X$ es de segunda categoría ya que $\R\times\{0\} \subseteq X$ es de segunda categoría.
	Sin embargo, $X$ no es de Baire:
	Nótese que $X \setminus \{(q, 1)\}$ es un abierto denso para todo $q \in \Q$, sin embargo,
	$ \R\times\{0\} = \bigcap_{q \in \Q} X \setminus \{(q, 1)\} $ es un abierto en $X$ que no es denso.
\end{exn}

\subsection{Teorema de categorías de Baire y elección}
Al igual que en la sección \S\ref{sec:compactness-choice} aquí nos dedicaremos a enunciar equivalencias fuertes con el axioma de elección.
Si al lector le desinteresa le basta con revisar el teorema~\ref{thm:baire_cat_thm}.
La ventaja es que todos los teoremas tienen más o menos la misma demostración, pero con consciencia de los distintos grados de AE admitidos en cada caso.

\begin{thm}
	Todo espacio completamente pseudometrizable y separable es de Baire.
\end{thm}
\begin{proof}
	Sea $d$ una métrica completa que induce la topología sobre $X$.
	Sea $\{p_n : n\in\N\}$ un conjunto denso de $X$, $\{D_n\}_{n\in\N}$ una familia de abiertos densos y $U_0$ un abierto no vacío de $X$.
	Como cada $D_n$ es abierto y denso, $U_0\cap D_n$ es abierto y no vacío.

	Construiremos la siguiente sucesión por recursión:
	Sea $y_1$ el $x_i$ de menor índice tal que $y_1 \in U_0\cap D_0$ y $\epsilon_1 := 1/m$, donde $m$ es el menor índice tal que
	$$ y_1 \in U_1 := B_{\epsilon_0}(y_0) \subseteq \overline{ B_{\epsilon_0}(y_0) } \subseteq U_0\cap D_0. $$
	Y sea $y_{n+1}$ el $x_i$ de menor índice tal que $y_{n+1} \in U_n\cap D_n$ y $\epsilon_{n+1} := 1/m$, donde $m$ es el menor índice tal que
	$\epsilon_{n+1} < \epsilon_n$ e $y_{n+1} \in U_{n+1} := B_{\epsilon_n}(y_0) \subseteq \overline{U}_{n+1} \subseteq U_n\cap D_n$.

	Finalmente, $(y_n)_{n=1}^\infty$ es de Cauchy y, por completitud, converge a un límite $L$ tal que
	\begin{equation}
		L\in \bigcap_{n=1}^\infty \overline{U_n} \subseteq U_0 \cap \bigcap_{n\in\N} D_n. \tqedhere
	\end{equation}
\end{proof}

\begin{ex}[Un espacio metrizable que no es completamente metrizable]
	Nótese que $\Q$ es metrizable, pero es un espacio de primera categoría, así que no es de Baire, luego no es completamente metrizable.
\end{ex}

Como señalé, casi todas las demostraciones son idénticas salvo los detalles que admiten modificaciones:
\begin{thm}\label{thm:easy_baire}
	Todo espacio numerablemente compacto y pseudometrizable es de Baire.
\end{thm}
\begin{proof}
	Sea $d$ una métrica compatible para $X$, $\{D_n\}_{n\in\N}$ y $U_0$ como en el teorema anterior.

	Ahora la sucesión no es de puntos sino sólo de abiertos y de índices:
	Sea $k_0$ el menor índice $m$ tal que existe $x$ tal que $\overline{B(x, 1/m)} \subseteq U_0 \cap D_0$ y sea
	$$ U_1 := \bigcup \{ B_{1/k_0}(x) : \overline{B(x, 1/k)} \subseteq U_0\cap D_0 \}, $$
	y así sucesivamente.

	Ésta sucesión es $\subseteq$-decreciente, luego notamos que si $\bigcap_{n=1}^\infty U_n = \emptyset$, entonces
	$ \bigcup_{n=1}^\infty \overline{U_n}^c = X $, es decir, se tiene un cubrimiento numerable por abiertos de $X$,
	pero como es numerablemente compacto se daría que algún $\overline{U_N}^c = X$, es decir, que $\overline{U_N} = \emptyset$,
	lo que es absurdo.
\end{proof}

\begin{thm}
	Son equivalentes:
	\begin{enumerate}
		\item \textbf{El axioma de elecciones numerables}.
		\item Todo espacio completamente metrizable y totalmente acotado es de Baire.
		\item Todo espacio completamente metrizable y 2AN es de Baire.
	\end{enumerate}
\end{thm}
\begin{proof}
	$1\implies 2$. Basta notar que AEN implica que totalmente acotado y completo sea equivalente a compacto.

	$2\implies 3$. Basta comprobar que todo espacio 2AN es totalmente acotado.

	$3\implies 1$. ...
	\todo{Completar demostración.}
\end{proof}

\begin{thmi}\label{thm:baire_cat_thm}
	Son equivalentes:
	\begin{enumerate}
		\item \textbf{El axioma de dependientes elecciones}.
		\item \textbf{El teorema de Baire:}\index{teorema!de Baire} Todo espacio completamente pseudometrizable es de Baire.
		\item Las dos condiciones:
			\begin{enumerate}[(a)]
				\item Todo espacio compacto de Hausdorff es de Baire.
				\item El producto numerable de espacios compactos de Hausdorff es compacto.
			\end{enumerate}
		\item El producto numerable de espacios compactos de Hausdorff es de Baire.
		\item El producto numerable de espacios discretos es de Baire.
		\item $X^\N$ es de Baire para todo espacio discreto $X$.
		\item $(\alpha X)^\N$ es de Baire para todo espacio discreto $X$.
	\end{enumerate}
\end{thmi}
\begin{proof}
	$1\implies 2$. Ejercicio para el lector (\textsc{Pista:} Vea el teorema~\ref{thm:easy_baire}).
	\par
	$1\implies 3$. La condición (b) es el teorema~\ref{thm:countable_prod_compact}.
	% \todoref{Insertar referencia.}
	La condición (a) es el teorema de pseudometrizable numerablemente compacto, pero empleando el DE para elegir correctamente los abiertos
	de la sucesión.

	$2\implies 5$. Sean $\vec x := (x_n)_{n\in\N}, \vec y := (y_n)_{n\in\N} \in \prod_{i\in I} X_i$, entonces la métrica
	$$ d( \vec x, \vec y ) :=
	\begin{cases}
		0, & \vec x = \vec y \\
		2^{-\{\min n : x_n \ne y_n\}}, & \vec x \ne \vec y \\
	\end{cases} $$
	induce la topología producto en dicho espacio y es Cauchy completa (¿por qué?).

	$5 \implies 6$. Trivial.
	\par
	$6 \implies 1$. Sea $(X, \rho)$ tal que para todo $x \in X$ el conjunto $\{ y : x\orho y \}$ es no vacío.
	Considere $Y := X^\N$ como el producto de $X$ tomado como espacio discreto y definamos:
	$$ D_n := \{ (x_n)_n \in Y : \exists m\in\N x_n\orho x_m \}, $$
	queremos ver que los $D_n$'s son densos y abiertos.
	Primero definamos $Z := \{ z\in X : \exists x \; x\orho z \}$ (nótese que $Z$ no es el espacio entero \textit{a priori}).
	Entonces
	$$ \vec x \in \prod_{i\in\N} W_i \subseteq D_n, \quad
	W_i :=
	\begin{cases}
		Z\cup\{x_n\}, &i = n \\
		X, &i \ne n
	\end{cases} $$
	Y por la definición de $\orho$ es claro que todo $D_n$ corta a todo abierto.

	Ahora, como $Y$ es de Baire, existe $\vec x \in \bigcap_{n\in\N} D_n$, es decir,
	$$ \forall n\in\N\; \exists m\in\N \; x_n \orho x_m $$
	luego podemos construir $y_n$ por recursión de tal manera que $y_0\orho y_1\orho y_2\orho\dots$,
	empleando la minimalidad del índice sobre la sucesión $\vec x$.

	$3\implies 4\implies 7$. Trivial.
	\par
	$7\implies 1$. Sigua la misma demostración que $6\implies 1$.
\end{proof}

% \thmdep{DE}
% \begin{thmi}[Teorema de Baire]\label{thm:baire_cat_thm}\index{teorema!de Baire}
%	 Todo espacio completamente metrizable es un espacio de Baire.
% \end{thmi}
% \begin{proof}
%	 Elijamos una métrica completa sobre $X$, vamos a probar que la intersección numerable de abiertos densos es también densa;
%	 así pues, sean $\{D_n\}_{n\in\N}$ abiertos densos y sea $U_0$ un abierto no vacío arbitrario en $X$.
%	 Queremos ver que $U_0$ y la intersección de los $D_n$'s se cortan, es decir, que $U_0$ corta a todo $D_n$.

%	 Como $D_0$ es abierto y denso, $U_0\cap D_0$ es abierto no vacío;
%	 elijamos $x_1 \in U_0\cap D_0$ y $\epsilon_1 < 1$ tal que $U_1 := B_{\epsilon_1}(x) \subseteq \overline{ B_{\epsilon_1}(x) } \subseteq U_0\cap D_0$.
%	 Luego procedemos recursivamente eligiendo $x_n \in U_n\cap D_n$ y $\epsilon_n < 1/n$ tales que
%	 $$ U_{n+1} := B_{\epsilon_n}(x) \subseteq \overline{ B_{\epsilon_n}(x) } \subseteq U_n\cap D_n \subseteq U_0\cap D_n. $$
%	 Así pues, $(x_n)_{n=1}^\infty$ es una sucesión de Cauchy, luego converge a $L$ y claramente
%	 \begin{equation}
%		 L \in \bigcap_{n=1}^\infty \overline{ U_n } \subseteq U_0\cap \bigcap_{n\in\N} D_n. \tqedhere
%	 \end{equation}
% \end{proof}
% \thmdep{}

\subsection{Consecuencias del teorema de Baire}
\thmdep{DE}
\begin{thm}[de Banach-Steinhaus]\index{teorema!de Banach-Steinhaus}
	Sea $X$ un espacio de Banach e $Y$ un espacio normado.
	Sea $\{T_i \colon X \to Y\}_{i\in I}$ una familia de operadores continuos, son equivalentes:
	\begin{enumerate}
		\item Para todo $\vec x \in X$ el conjunto $\{T_i\vec x : i\in I\}$ está acotado.
		\item El conjunto $\{ \|T_i\| : i\in I\}$ está acotado.
	\end{enumerate}
\end{thm}
\begin{proof}
	$2 \implies 1$. Trivial.
	\par
	$1 \implies 2$. Definamos los conjuntos
	$$ E_n := \{ \vec x \in X : \forall i\in I \quad \|T_i\vec x\| \le n \} $$
	los cuales son claramente cerrados.
	Por hipótesis de 1 se satisface que $X = \bigcup_{n\in\N} E_n$.
	Luego, por el teorema de categorías de Baire, algún $E_n$ debe de tener interior no vacío.
	Es fácil ver que $E_n - E_n \subseteq E_{2n}$, luego $E_{2n}$ debe de contener alguna bola centrada en $\Vec 0$.
	Finalmente por principio arquimediano, algún $E_N$ debe contener a la bola $\overline B_1(\Vec 0)$, por ende, el conjunto de las normas está acotado
	por dicho $N$ como se quería probar.
\end{proof}
\begin{cor}
	Sea $E$ un espacio de Banach y sea $S \subseteq E$.
	Supongamos que para todo $f \in E^*$, el conjunto $f[S] \subseteq \R$ es acotado, entonces $S$ es acotado.
\end{cor}

\begin{thmi}[Teorema de la función abierta]
	Sean $E, F$ un par de espacios de Banach y sea $T \colon E \to F$ una función lineal continua.
	Son equivalentes:
	\begin{enumerate}
		\item $T$ es sobreyectiva.
		\item $T$ es una aplicación abierta.
		\item Para todo $f \in F$ existe $u \in U$ tal que $Tu = f$.
		\item Existe una constante $C > 0$ tal que para todo $f \in F$ existe $u \in U$ con $Tu = f$ y $\|u\|_E \le C\,\|f\|_F$.
	\end{enumerate}
\end{thmi}
\begin{proof}
	Es claro que $4 \implies 3 \iff 1$.

	$2 \implies 4$. Sea $V := T[B_1^E(\Vec 0)]$ el cual es un entorno del origen $\Vec 0 \in F$, por lo que existe $\delta > 0$
	tal que $V \supseteq B_{2\delta}^F(\Vec 0) \supseteq \overline{B}_\delta^F(\Vec 0)$.
	Luego para todo $f \in F$ no nulo, vemos que $\tilde f := \delta/\|f\| \cdot f \in V$, por lo que existe $\tilde u \in E$ con $\|\tilde u\| < 1$
	tal que $T\tilde u = \tilde f$ o equivalentemente
	\[
		u := \frac{\|f\|}{\delta}\cdot\tilde u \qquad Tu = f, \quad \|u\| \le \frac{1}{\delta}\|f\|.
	\]
	$4 \implies 2$. Sea $U \subseteq E$ abierto.
	Para $f_0 \in T[U]$ existe $u_0 \in U$ con $Tu_0 = f_0$.
	Por continuidad, para todo $r > 0$ se cumple que
	\[
		\forall y \in B_r^F(\Vec 0) \quad \exists u \in B_{Cr}^E(\Vec 0) : Tu = y.
	\]
	Como $U$ es abierto y $T$ es lineal, existe $r$ suficientemente pequeño tal que $u_0 \in B_{Cr}^E(\Vec 0) \subseteq U$
	de modo que $T[U] \supseteq B_r^F(\Vec 0)$.

	$3 \implies 4$. Sean
	\[
		F_n := \{ f \in F : \exists u : Tu = f, \; \|u\|_E \le n\|f\|_F \}.
	\]
	Por hipótesis, $F = \bigcup_{n\in\N} F_n$ y, como $F$ es de Banach, entonces es un espacio de Baire, por lo que los $F_n$'s no pueden
	ser diseminados, vale decir, $\Int(\overline{F}_N) \ne \emptyset$ para algún $N$.
	Así, para todo $f_0 \in F$, para todo $\delta$ suficientemente pequeño, $\epsilon \in (0, 1)$ y todo $f \in B_\delta(f_0)$
	existe $u \in E$ tal que
	\[
		\|u\| \le N\|Tu\|, \qquad \|f - Tu\| < \epsilon.
	\]
	En particular, para $f = f_0$ y $\epsilon = 1$ existe $u_0$ tal que
	\[
		\|u_0\| \le N\|Tu_0\| \le N(\epsilon + \delta + \|f_0\|) \qquad \|f_0 - Tu_0\| < 1.
	\]
	Por linealidad, podemos recursivamente construir $u_m$ con $\delta = 1/m$ y $\epsilon = 1/m2^n$ tal que
	\begin{align*}
		\|u_{n+1}\| &\le N\left( \frac{1/m}{2^n} + \frac{1}{m} + \| {\textstyle f_0 - T\left( \sum_{i=0}^{n} u_n \right)} \|\right), \\
		\|({\textstyle f_0 - T\left( \sum_{i=0}^{n} u_n \right)}) - Tu_{n+1} \| &< \frac{1}{m2^{n+1}}.
	\end{align*}
	Luego $\bar u := \sum_{n=1}^{\infty} u_n$ converge por completitud, $\|\bar u\| \le N\|T\bar u\|$ y $T\bar u = f_0$ por continuidad de $T$.
\end{proof}
\begin{cor}
	Toda isomorfismo lineal continuo entre espacios de Banach es un homeomorfismo.
\end{cor}
\begin{cor}
	Sea $E$ un espacio vectorial y sean $\|\,\|_1, \|\,\|_2$ un par de normas sobre $E$ con las cuales es completo.
	Si existe $C > 0$ tal que
	\[
		\forall \vec x \in E \qquad \|\vec x\|_1 \le C\|\vec x\|_2,
	\]
	entonces las normas son equivalentes.
\end{cor}

\begin{thmi}[Teorema del gráfico cerrado]\index{teorema!del gráfico cerrado}
	Sean $E, F$ un par de espacios de Banach y sea $T \colon E \to F$ una función lineal.
	Si su gráfico
	\[
		\Gamma(T) = \{ (\vec x, T(\vec x)) : \vec x \in E \} \subseteq E\times F
	\]
	es cerrado, entonces $T$ es continuo.
\end{thmi}
\begin{proof}
	Denotaremos por $\|\,\|_E, \|\,\|_F$ las normas de $E, F$ resp.
	Mediante la biyección del gráfico $\vec x \mapsto (\vec x, T\vec x)$ tenemos un isomorfismo lineal $E \simeq \Gamma(T)$,
	y la norma que define la topología subespacio de $\Gamma(T)$ es $\|\vec x\|_{\Gamma(T)} := \|\vec x\|_E + \|T\vec x\|_F$.
	Como $\Gamma(T)$ es cerrado, esta es una norma con la cual $E$ también es completo.
	Finalmente, como claramente $\|\,\|_E \le \|\,\|_{\Gamma(T)}$ concluimos por el corolario anterior.
\end{proof}
\thmdep{}

\begin{advanced}
\subsection{Espacios \v Cech-completos}
La siguiente es una noción puramente topológica que generaliza a los espacios completamente metrizables.

\addtocounter{thmi}{1}
\begin{slem}
	En un espacio de Tychonoff $X$ son equivalentes:
	\begin{enumerate}
		\item En toda compactificación $cX$ el resto $cX \setminus c[X]$ es un conjunto $F_\sigma$.
		\item En la compactificación de \v Cech-Stone el resto es un conjunto $F_\sigma$.
		\item En alguna compactificación el resto es un conjunto $F_\sigma$.
	\end{enumerate}
\end{slem}
\begin{proof}
	Es claro que $1 \implies 2\implies 3$.

	$3\implies 2$. Sea $cX$ una compactificación donde el resto $R := cX \setminus c[X]$ es un $F_\sigma$.
	Luego, por maximalidad sea $f\colon \beta X \to cX$ tal que $\beta\circ f = c$.
	Nótese que $f^{-1}[ R^c ] = f^{-1}[ c[X] ] = \beta[X] \subseteq \beta X$, de modo que $f^{-1}[R] = \beta X \setminus \beta[X]$
	que es un $F_\sigma$ por ser la preimagen continua de un $F_\sigma$.

	$2\implies 1$.
	Sea $cX$ una compactificación arbitraria. Por maximalidad existe $f\colon \beta X \to cX$ tal que $\beta\circ f = c$.
	Sea $\beta X \setminus \beta[X] = \bigcup_{n\in\N} F_n$, con $F_n$ cerrados.
	Luego, como $cX \setminus c[X] = f[ \beta X \setminus \beta[X] ] = \bigcup_{n\in\N} f[ F_n ]$ y $f[ F_n ]$ es cerrado,
	entonces $cX \setminus c[X]$ es un conjunto $F_\sigma$.
\end{proof}
\addtocounter{thmi}{-1}

\begin{mydef}
	Un espacio $X$ es \strong{\v Cech-completo}\index{espacio!Cech-completo@\v Cech-completo} si es de Tychonoff y satisface cualquiera
	de las condiciones del lema anterior.
\end{mydef}
\begin{cor}
	Todo espacio localmente compacto de Tychonoff es \v Cech-completo.
\end{cor}
\begin{cor}
	Todo subespacio cerrado o $G_\delta$ de un espacio \v Cech-completo es también \v Cech-completo.
\end{cor}

\begin{thm}
	En un espacio $X$ de Tychonoff, son equivalentes:
	\begin{enumerate}
		\item $X$ es \v Cech-completo.
		\item Existe una familia numerable $\{ \mathcal{C}_i \}_{i\in\N}$ de cubrimientos por abiertos de $X$
			tal que si $ \mathcal{F} $ es una familia de cerrados con la PIF y tal que para todo $i\in\N$
			existe un $F_i \in \mathcal{F}$ y un $A_{i, j} \in \mathcal{C}_i$ con $F_i \subseteq A_{i, j}$;
			entonces $\bigcap \mathcal{F} \ne \emptyset$.
	\end{enumerate}
\end{thm}
\begin{proof}
	$2 \implies 1$.
	Sea $K$ una compatificación de $X$ y, para cada $U \in \mathcal{C}_i$ sea $U^* \subseteq K$ abierto tal que $U = U^* \cap X$.
	Claramente
	\[
		X \subseteq \bigcap_{i\in\N} \bigcup_{U\in \mathcal{C}_i} U \subseteq \bigcap_{i\in\N} \bigcup_{U\in \mathcal{C}_i} U^* =: G.
	\]
	Vamos a probar la inclusión recíproca.
	Sea $x \in G$ y sea $\mathcal{F}_x = \{ \overline{V} \cap X : V \subseteq X \text{ es entorno de } x \}$.
	Es claro que $\mathcal{F}_x$ es una familia de cerrados en $X$ con la PIF y que para cada índice $i$, existe $U \in \mathcal{C}_i$ tal que $x \in U^*$.
	Como $K$ es regular, existe $V$ tal que $x \in V \subseteq \overline{V} \subseteq U^*$, de modo que $\overline{V} \cap X \subseteq U^* \cap X = U$.
	Así, por hipótesis, $\bigcap \mathcal{F}_x \ne \emptyset \subseteq X$ y, como $\bigcap \mathcal{F}_x = \overline{\{ x \}} = \{ x \}$
	concluimos que $x \in X$. Así, $X$ es un conjunto $G_\delta$ en $K$.

	$1 \implies 2$.
	Sea $K$ una compactificación de $X$ y supongamos que $X = \bigcap_{i\in \N} U_i$, donde $U_i \subseteq K$ son abiertos.
	Sea
	\[
		\mathcal{C}_i = \{ V \cap X : V \subseteq K\text{ es abierto y } \overline{V} \subseteq U_i \},
	\]
	como $K$ es regular, cada $\mathcal{C}_i$ es un cubrimiento por abiertos de $X$.

	Sea $\mathcal{F}$ una familia de cerrados en $X$ con la PIF con las hipótesis del enunciado.
	Como $K$ es compacto, existe $x \in \bigcap_{A \in \mathcal{F}} \overline{A}$, donde $\overline{()}$ denota la clausura en $K$.
	Así, por hipótesis, para cada $i\in \N$ existen $F_i \in \mathcal{F}$ y $V_i \in \mathcal{C}_i$
	tales que $F_i \subseteq V_i \subseteq \overline{V}_i \subseteq U_i$.
	Por lo tanto,
	\[
		x \in \bigcap_{i\in\N} U_i = X,
	\]
	y, luego, $x \in \bigcap_{A \in \mathcal{F}} (\overline{A} \cap X) = \bigcap \mathcal{F}$.
\end{proof}

Como corolario obtenemos lo siguiente:
\begin{thm}
	Un espacio metrizable es \v Cech-completo syss es completamente metrizable.
\end{thm}
\begin{proof}
	$\impliedby$.
	Si $X$ es métrico y completo, entonces $\mathcal{C}_n := \{ B_{1/n}(x) : x \in X \}$ satisface las hipótesis del teorema anterior.
	% (ya que, para $n$ suficientemente grande, los elementos).

	$\implies$.
	Supongamos que $X$ es metrizable y \v Cech-completo, sea $\widehat{X}$ una compleción de $X$ y $c\widehat{X}$ una compactificación.
	Luego, $c \widehat{X}$ es una compactificación de $X$, por lo que $X$ es un subconjunto $G_\delta$ de $c\widehat{X}$ y,
	por el corolario~\ref{thm:Gd_sub_compl_is_compl} concluímos.
\end{proof}

\begin{prop}
	El producto numerable de espacios \v Cech-completos es también \v Cech-completo.
\end{prop}

\begin{thmi}[Teorema de categorías de Baire]
	Todo espacio \v Cech-completo es de Baire.
\end{thmi}
\todo{Escribir demostración (ver \cite[196-198]{engelking:top}).}
\end{advanced}

\section{Espacios uniformes}
El estudio de los espacios métricos demuestra que conforman una clase extremadamente distinguida de espacios en la topología general (énfasis en lo último,
nótese que en la topología diferencial se estudian las \textit{variedades diferenciales} y no es difícil ver que todas ellas son espacios métricos, y claramente
su estudio no es menos difícil por ello). No obstante, es particularmente fuerte exigir, además de una noción de cercanía, un número real asociado.
Ésto asemeja la teoría de distancias sobre geometrías de Hilbert, en donde, los axiomas no exigen una medida natural entre puntos, sino que sólo exigen
una manera natural de comparar distancias (contrástese con la geometría de Birkhoff).
La idea general está en que si exigimos una idea de cercanía global, pero sin necesidad de un real asociado, obtenemos varios de los resultados previos
y además ampliamos arduamente nuestra clase de espacios topológicos: de hecho, una de las ventajas que ganamos con los espacios uniformes son un lenguaje
con el cual definir topológicamente ciertos tipos de espacios de funciones e incluso estructuras sobre grupos topológicos.

Volvamos al caso de los espacios métricos, la idea que queremos rescatar es la siguiente:
En un espacio métrico $X$ dado un radio $r > 0$, se satisface que los conjuntos de la forma $\{ (x, y) \in X\times X : d(x, y) < r \}$ caracterizan la métrica;
en particular, satisfacen que contienen a la diagonal y son simétricos.

Un subconjunto de $X \times X$ es (al menos formalmente) una relación sobre $X$, de modo que dados $A, B \subseteq X \times X$ entonces --como relaciones--
poseen inversa y composición, denotaremos:
\begin{gather*}
	-A := A^{-1} = \{ (x, y) : (y, x) \in A \}, \\
	A + B := A\circ B = \{ (x, z) : \exists y\in X \; (x, y) \in A, (y, z) \in B \}.
\end{gather*}
Como la composición de relaciones es asociativa, se tiene que $(A + B) + C = A + (B + C)$.
Ojo que entre subconjuntos arbitrarios de $X\times X$ se satisface que <<$+$>> no es una operación conmutativa.
Luego podemos definir, por recursión:
$$ 1\cdot A := A, \qquad (n+1)\cdot A := n\cdot A + A. $$
\begin{mydef}
	Se dice que un conjunto $V \subseteq X\times X$ es una \strong{banda}\index{banda} de $X$ si:
	\begin{enumerate}[{B}1.]
		\item $\triangle \subseteq V$ (reflexividad).
		\item $V = -V$ (simetría).
	\end{enumerate}
	(Es decir, es una relación reflexiva y simétrica).
	Denotamos por $\mathcal{D}_X$ la familia de todas las bandas sobre $X$.

	Dados dos puntos $x, y \in X$ y una banda $V$ denotamos $d(x, y) < V$ si $(x, y) \in V$; de lo contrario denotaremos que $d(x, y) \ge V$.
	Dado un conjunto $A \subseteq X$ decimos que su diámetro es menor que $V$, denotado $d(A) < V$, si
	$$ \forall x,y \in A \; d(x, y) < V \iff A\times A \subseteq V. $$
	Finalmente, dado un punto $x \in X$ y una banda $V$, definimos la bola de centro $x$ y radio $V$ como:
	$$ B_V(x) := \{ y \in X : d(x, y) < V \}, $$
	en particular $d\big( B_V(x) \big) < 2V$.
\end{mydef}
Se pueden comprobar las siguientes propiedades, para $V, V_1, V_2$ bandas:
\begin{enumerate}
	\item $d(x, x) < V$.
	\item $d(x, y) < V$ syss $d(y, x) < V$.
	\item Si $d(x, y) < V_1$ y $d(y, z) < V_2$, entonces $d(x, z) < V_1 + V_2$.
\end{enumerate}
Ésto justifica el paralelo con los espacios métricos.

\begin{mydefi}
	Una \strong{uniformidad}\index{uniformidad} $\mathcal{U}$ sobre un conjunto $X$ es un subconjunto $\mathcal{U} \subseteq \mathcal{D}_X$ tal que:
	\begin{enumerate}[{U}1.]
		\item Si $V \in \mathcal{U}$ y $V \subseteq W \in \mathcal{D}_X$, entonces $W \in \mathcal{U}$.
		\item Si $V_1, V_2 \in \mathcal{U}$, entonces $V_1\cap V_2 \in \mathcal{U}$.
		\item Para todo $V \in \mathcal{U}$ existe $W \in \mathcal{U}$ tal que $2W \subseteq V$.
	\end{enumerate}
	Más aún, una uniformidad se dice \strong{de Hausdorff}\index{uniformidad!de Hausdorff} si satisface:
	\begin{enumerate}[{U}4.*]
		\item $\bigcap \mathcal{U} = \triangle$.
	\end{enumerate}
	Un par $(X, \mathcal{U})$, donde $\mathcal{U}$ es una uniformidad sobre $X$, se dice un \strong{espacio uniforme}\index{espacio!uniforme}.
	Cuando no haya ambigüedad sobre los signos obviaremos la uniformidad.
	Cuando hablemos de una banda en un espacio uniforme se asumirá que nos referimos a un elemento de su uniformidad.
\end{mydefi}
\begin{ex}
	Sea $X$ un conjunto cualquiera.
	\begin{itemize}
		\item $\mathcal{U} := \mathcal{D}_X$ es una uniformidad sobre $X$ conocida como la \strong{uniformidad discreta}.
		\item $\mathcal{U} := \{X\times X\}$ es una uniformidad sobre $X$ conocida como la \strong{uniformidad indiscreta}.
	\end{itemize}
\end{ex}

\begin{prop}
	Sea $(X, \mathcal{U})$ un espacio uniforme y definamos $\tau \subseteq \P X$ así:
	Un subconjunto $G \subseteq X$ está en $\tau$ syss para todo $x \in G$ existe una banda $V \in \mathcal{U}$ tal que $B_V(x) \subseteq G$.
	Entonces $\tau$ es una topología a la que llamamos la \strong{inducida por la uniformidad}.
\end{prop}
\begin{proof}
	Es claro que $\emptyset, X \in \tau$ y que la unión de elementos de $\tau$ está en $\tau$.

	Sean $G_1, G_2 \in \tau$ y sea $x \in G_1\cap G_2$. Existen $V, W \in \mathcal{U}$ tales que $B_V(x) \subseteq G_1$ y $B_W(x) \subseteq G_2$,
	luego, como $V\cap W \in \mathcal{U}$ por U2., es fácil notar que $B_{V\cap W}(x) \subseteq G_1\cap G_2$.
\end{proof}
Desde ahora en adelante, siempre estudiemos los espacios uniformes con su topología inducida.
Es claro que la unifomidad (in)discreta da lugar a la topología (in)discreta sobre el espacio.

\begin{prop}
	Sea $X$ un espacio uniforme y sea $A \subseteq X$.
	Entonces un punto $x \in A$ es interior syss existe una banda $V$ tal que $B_V(x) \subseteq A$.
\end{prop}

Al igual que con los espacios topológicos, es difícil dar una topología así como así, por lo cual es menester una idea de base:
\begin{mydef}
	Una \strong{base}\index{base!(espacio uniforme)} de un espacio uniforme $(X, \mathcal{U})$ es una subfamilia $\mathcal{B} \subseteq \mathcal{U}$
	tal que para todo $V \in \mathcal{U}$ existe un $W \in \mathcal{B}$ tal que $W \subseteq V$.
\end{mydef}

\begin{prop}
	Una familia de bandas $\mathcal{B}$ que es base de un espacio uniforme $X$ satisface:
	\begin{enumerate}[{BU}1.]
		\item Si $V_1, V_2 \in \mathcal{B}$ entonces existe $W \in \mathcal{B}$ tal que $W \subseteq V_1\cap V_2$.
		\item Para todo $V \in \mathcal{B}$ existe $W \in \mathcal{B}$ tal que $2W \subseteq V$.
	\end{enumerate}
	Y conversamente, toda familia de bandas que cumple las condiciones anteriores determina una única uniformidad
	$$ \mathcal{U} = \{ V \in \mathcal{D}_X : \exists W \in \mathcal{B} \; W \subseteq V \} $$
	de la cual es base.
	Más aún, la uniformidad es de Hausdorff syss $\bigcap \mathcal{B} = \triangle$.
\end{prop}

\begin{ex}
	Todo espacio pseudométrico es uniforme:
	En efecto, sea $(X, d)$ un espacio pseudométrico, luego para todo $n \in \N_{\ne 0}$ podemos definir:
	$$ V_n := \left\{ (x, y) \in X\times X : d(x, y) < \tfrac{1}{n} \right\}, $$
	así pues $\mathcal{B} := \{V_n : n\in \N_{\ne 0}\}$ es una base de una uniformidad $\mathcal{U}$ que llamamos \strong{uniformidad inducida por la métrica}.
	Se puede demostrar (¡hágalo!) que $\mathcal{U}$ induce la misma topología que la pseudométrica.
\end{ex}

Consideremos a $\R$ como espacio uniforme con la uniformidad del ejemplo anterior.
Luego, nótese que el conjunto
$$ V := \{ (x, y) \in \R\times\R : |x - y| \le 1 \} $$
es una banda y que $B_V(x) = [x-1, x+1]$.
Ésto comprueba que dada una banda $V$ arbitraria, puede darse que $B_V(x)$ no sea un abierto, no obstante:
\begin{prop}
	Sea $X$ un espacio uniforme.
	Dada una banda $V$ y un punto $x \in X$, el conjunto $B_V(x)$ es un entorno de $x$ y, en particular, $\Int B_V(x)$ es un abierto que contiene a $x$.
\end{prop}
Ésto, sin embargo, puede arreglarse.

\begin{prop}
	Sea $X$ un espacio uniforme.
	Dado un conjunto $A$ se cumple que un punto $x$ es adherente a $A$ si para toda banda $V$ se cumple que $A \cap B_V(x) \ne \emptyset$.
\end{prop}
\begin{cor}
	Sea $X$ un espacio uniforme.
	Sea $A \subseteq X$ y $V$ una banda tales que $d(A) < V$, entonces $d( \overline A ) < 3V$.
\end{cor}

Ahora, la razón detrás del nombre de uniformidad de Hausdorff:
\begin{thm}
	Un espacio uniforme es $T_0$ syss su uniformidad es de Hausdorff, en cuyo caso es un espacio topológico de Hausdorff.
\end{thm}
\begin{proof}
	$\implies$. Sean $x, y \in X$ distintos, luego existe un entorno $A$ de $x$ tal que $y \notin A$.
	Luego existe una banda $V$ tal que $d(x, y) \ge V$, es decir, para todo par de puntos existe una banda que \textit{no} les contiene, y luego se comprueba
	que la uniformidad es de Hausdorff.
	\par
	$\impliedby$. Si $\bigcap \mathcal{U} = \triangle$, entonces para todo par de puntos distintos $x, y \in X$ existe una banda $V$ tal que
	$d(x, y) \ge V$; luego por U3. existe una banda $W$ tal que $2W \subseteq V$, por lo que $d(x, y) \ge 2W$ y luego $B_W(x), B_W(y)$ son entornos disjuntos
	de $x, y$ (pues si $d(x, z) < W$ y $d(y, z) < W$, entonces $d(x, y) < 2W$, contradicción) y el espacio es de Hausdorff.
\end{proof}

\begin{mydefi}
	Sean $(X, \mathcal{U})$ e $(Y, \mathcal{V})$ un par de espacios uniformes.
	Una aplicación $f\colon (X, \mathcal{U}) \to (Y, \mathcal{V})$ se dice \strong{uniformemente continua}\index{uniformemente continua (función)}
	si para todo $V \in \mathcal{V}$ existe una banda $U \in \mathcal{U}$ tal que $d(x, y) < U$ implica que $d\big( f(x), f(y) \big) < V$.
\end{mydefi}
\begin{prop}
	Dados $X, Y, Z$ espacios uniformes. Entonces:
	\begin{enumerate}
		\item $\Id_X \colon X \to X$ es uniformemente continua.
		\item Si $f \colon X \to Y$ y $g\colon Y \to Z$ son uniformemente continuas, entonces $f\circ g \colon X \to Z$ también lo es.
			En consecuencia, los espacios uniformes (como objetos) y las funciones uniformemente continuas (como flechas) constituyen
			una categoría, denotada $\mathsf{Unif}$.
		\item Toda aplicación uniformemente continua es continua.
			En consecuencia, $\mathsf{Unif} \subseteq \mathsf{Top}$.
			% \item Las aplicaciones continuas son uniformemente continuas.
	\end{enumerate}
\end{prop}
Categorialmente, $\mathsf{Unif}$ no es una subcategoría plena; de modo que existen objetos isomorfos en $\mathsf{Top}$ que no son isomorfos en $\mathsf{Unif}$.
Por ello:
\begin{mydef}
	Sean $X, Y$ un par de espacios uniformes.
	Se dice que una aplicación $f\colon X \to Y$ es un \strong{homeomorfismo uniforme}\index{homeomorfismo!uniforme} si es biyectiva, uniformemente continua
	y si $f^{-1}$ es uniformemente continua (i.e., es un isomorfismo en $\mathsf{Unif}$).
	Si existe un homeomorfismo uniforme entre $X, Y$, entonces se dicen \strong{uniformemente homeomorfos}.
	\par
	Se dice que una aplicación $f\colon X \to Y$ es un \strong{encaje uniforme} si $f\colon X \to f[X]$ es un homeomorfismo uniforme,
	en cuyo caso $X$ se dice \strong{uniformemente encajado} en $Y$.
\end{mydef}
% En particular, la inclusión $\iota\colon S \to X$ de un subespacio $S \subseteq X$ es un encaje uniforme.

\begin{thm}
	Sea $X$ un espacio uniforme e $Y \subseteq X$. Entonces:
	$$ \mathcal{U}_Y := \{ V \cap (Y\times Y) : V \in \mathcal{U} \} $$
	es una uniformidad sobre $Y$ que induce la topología subespacio.
	Más aún, la aplicación $\iota\colon (Y, \mathcal{U}_Y) \to (X, \mathcal{U})$ es un encaje uniforme.
\end{thm}

Una de las grandes (¡y buenas!) diferencias con los espacios métricos, es que con los métricos sólo puedes hacer producto de a lo más numerables y conservar la
estructura métrica; no obstante, con espacios uniformes el producto se admite arbitrario:
\begin{thm}
	Sean $\{ (X_i, \mathcal{U}_i) \}_{i\in I}$ espacios uniformes y denotemos $X := \prod_{i\in I} X_i$.
	Así pues, sea $\mathcal{B}$ la familia de los conjuntos de la forma
	$$ \{ (\vec x, \vec y) \in X\times X : \forall i \in J \; d(x_i, y_i) < V_i \} $$
	donde $J \subseteq I$ es finito, y $V_i \in \mathcal{U}_i$.
	Entonces $\mathcal{B}$ es base de una uniformidad en $X$ que induce la topología producto.
\end{thm}
\begin{proof}
	Que $\mathcal{B}$ sea base de una uniformidad $\mathcal{U}$ queda de ejericio al lector.
	Que ambas topologías coinciden lo veremos por doble inclusión:
	Sea $A$ un entorno de $\vec x \in X$ en el sentido de la topología producto, entonces posee un subentorno básico $B$,
	es decir, que existe $J \subseteq I$ finito tal que $B = \prod_{i\in I} B_i$ donde $B_i = X_i$ para todo $i\notin J$, y $x_i \in B_i$ para todo $i \in J$;
	de modo que $B_{V_i}(x_i) \subseteq B_i$ para alguna banda $V_i$ de $X_i$ para todo $i \in J$.
	Luego, definiendo $V_i := X_i$ para todo $i \notin J$ se cumple que $W := \prod_{i\in I} V_i$ es una banda de la base $\mathcal{B}$ y de que
	$B_W(\vec x) \subseteq B$, de modo que $B$ es un entorno en la topología dada por la uniformidad.
	Así mismo es fácil ver que todo abierto en la topología inducida por la uniformidad lo es en la topología producto.
\end{proof}
\begin{prop}
	Sean $(X_i)_{i\in I}$ una familia de espacios uniformes no vacíos. Entonces:
	\begin{enumerate}
		\item Las proyecciones $ \pi_j\colon \prod_{i\in I} X_i \to X_j $ son uniformemente continuas.
		\item Sea $(f_i\colon Y \to X_i)_{i\in I}$ una familia de funciones uniformemente continuas.
			Entonces la diagonal $f := \Diag_{i\in I} f_i \colon Y \to \prod_{i\in I} X_i$ es la única función uniformemente continua
			tal que el siguiente diagrama conmuta para todo $j \in I$:
			\begin{center}
				\begin{tikzcd}[row sep=large]
					{}					& \prod_{i\in I} X_i \dar["\pi_j", two heads] \\
					Y \rar["f_j"'] \urar["\exists!f", dashed] & X_j
				\end{tikzcd}
			\end{center}
			En consecuencia, $\prod_{i\in I} X_i$ es un producto categorial en $\mathsf{Unif}$.

		\item Una aplicación $f\colon Y \to \prod_{i\in I} X_i$ es uniformemente continua syss $f\circ\pi_i$ lo es para todo $i \in I$.
	\end{enumerate}
\end{prop}

\begin{prop}
	Sea $X$ un espacio uniforme, $x \in X$ y $V$ una banda en $X$.
	Si $V$ es abierto (resp. cerrado) en $X\times X$, entonces $B_V(x)$ es abierto (resp. cerrado).
\end{prop}
\begin{proof}
	Basta notar que $\iota\colon y \mapsto (x, y)$ es continua, luego $B_V(x) = \iota^{-1}[V]$ y se sigue el enunciado.
\end{proof}

\begin{thm}\label{thm:entourage_neighborhood}
	Sea $X$ un espacio uniforme y $A \subseteq X\times X$.
	Si $V$ es una banda de $X$, entonces $V + A + V$ es un entorno de $A$ y
	$$ \overline A = \bigcap_{V \in \mathcal{U}} (V + A + V). $$
\end{thm}
\begin{proof}
	Sea $(x, y) \in (V + A + V)$. Por definición existen $p, q \in X$ tales que $(x, p) \in V, (p, q) \in A, (q, y) \in V$, o equivalentemente,
	tales que $(x, y) \in B_V(p) \times B_V(q)$. Nótese que
	$$ A \subseteq \bigcup_{(p, q) \in A} (B_V(p) \times B_V(q)) \subseteq V + A + V, $$
	luego $A$ está contenido en la unión de los interiores de $( B_V(p) \times B_V(q) )$ de modo que efectivamente $V + A + V$ es un entorno de $A$.
	\par
	Para la segunda parte nótese que $(x, y) \in (V + A + V)$ syss existe $(p, q) \in A \cap \big( B_V(x) \times B_V(y) \big)$,
	por ende, $(x, y) \in V+A+V$ para todo $V$ syss todo entorno de $(x, y)$ corta a $A$ syss es adherente a $A$.
\end{proof}

\begin{mydef}
	Sea $X$ un espacio uniforme, $A \subseteq X$ y $V$ una banda en $X$.
	Se define el \strong{entorno uniforme} de $A$ como
	$$ V[A] := \bigcup_{a\in A} B_V(a). $$
\end{mydef}
Al igual que con las bolas de radio $V$, los entornos uniformes sí son entornos del conjunto $A$, pero no necesariamente son abiertos.

\begin{prop}
	Sea $X$ un espacio uniforme y $A \subseteq X$. Entonces
	$$ \overline A = \bigcap_{V \in \mathcal{U}} V[A]. $$
\end{prop}
\begin{proof}
	Sea $V$ una banda, luego es fácil ver que
	$$ V + (A\times A) + V = V[A] \times V[A], $$
	y luego, aplicando el teorema anterior vemos que
	$$ \overline A \times \overline A = \overline{A \times A} = \bigcap_{V \in \mathcal{U}} (V + (A\times A) + V)
	= \bigcap_{V \in \mathcal{U}} (V[A] \times V[A]), $$
	que es lo que se quería probar.
\end{proof}

\begin{thm}\label{thm:closed_bands_are_base}
	Sea $X$ un espacio uniforme, los interiores (resp. las clausuras) de las bandas en $X \times X$ son una base de su uniformidad.
	En particular, las bandas abiertas (resp. cerradas) son una base de su uniformidad.
\end{thm}
\begin{proof}
	Sea $V$ una banda en $X$, luego existe otra banda $W$ tal que $3W \subseteq V$ (¿por qué?).y el teorema~\ref{thm:entourage_neighborhood} nos dice que
	$W + W + W$ es un entorno de $W$, luego $W \subseteq \Int V \subseteq V$; y de aquí se concluye el caso para los interiores de las bandas.
	Para las clausuras, la proposición anterior nos da que $\overline W \subseteq W + W + W \subseteq V$.
\end{proof}

\begin{thm}
	Todo espacio uniforme de Hausdorff es regular.
\end{thm}
\begin{proof}
	Si $X$ es uniforme de Hausdorff, entonces es un espacio topológico de Hausdorff, luego es $T_1$.
	Por el teorema~\ref{thm:separation_properties} inciso 6 para probar que $X$ es regular basta probar que todo punto posee una base de entornos cerrados,
	lo que se sigue pues si $W$ es una banda cerrada, entonces $B_W(x)$ es un entorno cerrado de $X$ y por el teorema anterior, las bolas cerradas centradas
	en $X$ forman una base de sus entornos.
\end{proof}

\begin{mydef}
	Sea $X$ un espacio uniforme.
	Una función $\rho\colon X \times X \to \R$ se dice una \strong{pseudométrica uniforme}\index{pseudométrica!uniforme} si es una pseudométrica
	y además es una función uniformemente continua.
\end{mydef}

\begin{prop}\label{thm:uniform_pseudometric}
	Sea $X$ un espacio uniforme.
	Una pseudométrica $\rho\colon X \times X \to \R$ es uniforme syss para todo $\epsilon > 0$ existe una banda $V$ en $X$ tal que
	$$ d(x, y) < V \implies \rho(x, y) < \epsilon. $$
\end{prop}
\begin{proof}
	$\impliedby$. $\rho$ es una función uniformemente continua syss para todo $\epsilon > 0$ existe una banda $W$ en $X \times X$ tal que
	$$ d\big( (x, y), (x', y') \big) < W \implies |\rho(x, y) - \rho(x', y')| < \epsilon. $$ 
	Ahora bien, por hipótesis, sea $V$ una banda en $X$ tal que $d(x, y) < V \implies \rho(x, y) < \epsilon/2$ y definamos $W := V \times V$,
	que es banda en $X \times X$, luego $d\big( (x, y), (x', y') \big) < V\times V$ syss $d(x, x') < V$ y $d(y, y') < V$, lo que implica que
	$$ \epsilon = \frac{\epsilon}{2} + \frac{\epsilon}{2} \ge \rho(x, x') + \rho(y, y') \ge |\rho(x, y) - \rho(x', y')|. $$.
	$\implies$. Si $\rho$ es uniformemente continua, entonces para todo $\epsilon > 0$ existe una banda $V$ en $X$ tal que si $d(x, x') < V$ y
	$d(y, y') < V$, entonces $|\rho(x, y) - \rho(x', y')| < \epsilon$. En particular, tomando $y = y' = x'$ entonces se obtiene que
	$d(x, y) < V \implies \rho(x, y) < \epsilon$ como se quería probar.
\end{proof}

\begin{thm}\label{thm:uniformity_pseudometric}
	Sea $X$ un espacio uniforme.
	Dada una sucesión de bandas $(V_n)_{n\in\N}$ tal que $V_0 = X \times X$ y que $3V_{i+1} \subseteq V_i$, entonces existe una pseudométrica
	$\rho\colon X\times X \to \R$ tal que para todo $i\ge 1$:
	$$ \left\{ (x, y) : \rho(x, y) < \frac{1}{2^i} \right\} \subseteq V_i \subseteq \left\{ (x, y) : \rho(x, y) \le \frac{1}{2^i} \right\}. $$
	Más aún, dicha pseudométrica es uniforme.
\end{thm}
\begin{proof}
	Sean $x, y \in X$ y definamos $\rho(x, y)$ como el ínfimo de los valores
	$$ 1/2^{i_1} + \cdots 1/2^{i_k} < 1/2^i $$
	donde $d(x_{j-1}, x_j) < V_{i_j}$, $x_0 := x$ y $x_k := y$.
	Queda al lector comprobar que $\rho$ es efectivamente una pseudométrica sobre $X$ y es claro que $V_i \subseteq \{ (x, y) : \rho(x, y) \le 1/2^i \}$.

	Aún queda probar que si $\rho(x, y) < 1/2^i$, entonces $d(x, y) < V_i$, i.e., si existe
	$x_0, x_1, \dots, x_k$ con
	$$ 1/2^{i_1} + \cdots 1/2^{i_k} < 1/2^i, $$
	$d(x_{j-1}, x_j) < V_{i_j}$, $x_0 := x$ y $x_k := y$; entonces $d(x, y) < V_i$.
	Procedemos por inducción (fuerte) sobre $k$:
	El caso base $k = 1$ es claro pues $d(x, y) < V_{i_1}$ con $1/2^{i_1} < 1/2^i$, entonces $i_1 > i$ y $d(x, y) < V_{i_1} < V_i$.

	Para el caso inductivo podemos suponer que $k > 1$ y así tenemos que $1/2^{i_1} < 1/2^{i+1}$ o que $1/2^{i_k} < 1/2^{i+1}$; sin perdida de generalidad
	suponemos la primera. Definamos $n$ como el mayor natural $\le k-1$ tal que
	$$ 1/2^{i_1} + 1/2^{i_2} + \cdots + 1/2^{i_n} < 1/2^{i+1}. $$
	Si $n < k-1$, entonces $1/2^{i_1} + \cdots + 1/2^{i_n} + 1/2^{i_{n+1}} \ge 1/2^{i+1}$, de modo que
	$$ 1/2^{i_{n+2}} + \cdots + 1/2^{i_k} < 1/2^{i+1}. $$
	Así pues, por hipótesis inductiva se cumple que $d(x_0, x_n) < V_{i+1}$ y que $d(x_{n+1}, x_k) < V_{i+1}$.
	Además, como $1/2^{i_{n+1}} < 1/2^i$, entonces $i_{n+1} \ge i+1$, entonces $d(x_{i_n}, x_{i_{n+1}}) < V_{i_{n+1}} \subseteq V_{i+1}$;
	por lo que $d(x_0, x_k) < 3V_{i+1} \subseteq V_i$ como se quería probar.
\end{proof}

\thmdep{DE}
\begin{cor}
	Sea $X$ un espacio uniforme y $V$ una banda de $X$.
	Entonces, existe una pseudométrica uniforme $\rho$ sobre $X$ tal que
	$$ \{ (x, y) : \rho(x, y) < 1 \} \subseteq V. $$
\end{cor}
\begin{proof}
	Basta emplear el teorema anterior considerando una sucesión inducida por $V_0 := X\times X$ y $V_1 := V$.
\end{proof}

\begin{cor}
	Todo espacio uniforme $T_0$ es de Tychonoff.
\end{cor}
\begin{proof}
	Sean $x \in X$ un punto que no pertenece a un cerrado $F \subseteq X$ no vacío.
	Luego existe una banda $V$ tal que $B_V(x) \subseteq F^c$ y por el corolario anterior, existe una métrica $\rho$
	tal que $\rho(x, y) \ge 1$ si $d(x, y) \ge V$; luego $f(y) := \min\{1, \rho(x, y)\}$ es una función continua que separa a $x$ y a $F$.
\end{proof}
% Ejercicio para el lector: ¿dónde se empleó la hipótesis de ser uniformidad \textit{de Hausdorff}?
\thmdep{}

\begin{thm}
	Sea $X$ un conjunto y $P$ una familia de pseudométricas sobre $X$. Entonces los conjuntos de la forma:
	$$ V(F, \epsilon) := \left\{ (x, y) : \max_{\rho \in P} \rho(x, y) < \epsilon \right\}, $$
	donde $\epsilon > 0$ y $F \subseteq P$ finito, conforman una uniformidad en $X$ bajo la cual todas las pseudométricas de $P$ son uniformes.
	Ésta uniformidad es de Hausdorff syss para todo par de puntos distinto $x, y$ existe $\rho \in P$ tal que $\rho(x, y) > 0$.
\end{thm}
\begin{proof}
	Es claro que los conjuntos $V(F, \epsilon)$ son (vistos como relaciones sobre $X$) reflexivos y simétricos.
	La propiedad U2 se cumple pues
	$$ V\big( F_1 \cup F_2, \min\{ \epsilon_1, \epsilon_2 \} \big) \subseteq V(F_1, \epsilon_1) \cap V(F_2, \epsilon_2). $$
	Y la propiedad U3 se cumple pues $2V(F, \epsilon/2) \subseteq V(F, \epsilon)$ (por la desigualdad triangular de las pseudométricas).
\end{proof}
\begin{mydef}
	Dado un conjunto $X$ y una familia $P$ de pseudométricas sobre $X$, se le llama la \strong{unifomidad inducida por las pseudométricas $P$}%
	\index{unifomidad!inducida por las pseudométricas} a la uniformidad definida en el teorema anterior.
\end{mydef}

\begin{prop}
	Sea $X$ un espacio topológico y $P$ una familia de pseudométricas continuas sobre $X$ tales que para todo punto $x \in X$ y todo cerrado $F \subseteq X$
	con $x \notin F$ se cumple que existe $\rho \in P$ tal que $\rho(x, F) > 0$.
	Entonces la uniformidad inducida por las pseudométricas $P$ induce la misma topología original en $X$.
\end{prop}

\begin{thmi}
	Sea $X$ un espacio de Tychonoff. Entonces existe una uniformidad de Hausdorff $\mathcal{U}$ que induce su topología.
\end{thmi}
\begin{proof}
	Sea $f \in C(X)$, i.e., una función $f\colon X \to \R$ continua.
	Entonces determina una pseudométrica $\rho_f(x, y) := |f(x) - f(y)|$ que es continua (¿por qué?).
	Luego, sea $P$ la familia de todas las pseudométricas de dicha forma, y también sea $P^* \subseteq P$ la familia de pseudométricas de la forma $\rho_f$,
	donde $f\colon X \to \R$ es continua y acotada.
	Finalmente, la proposición anterior nos dice que tanto $P$ como $P^*$ inducen una uniformidad que induce la topología original sobre $X$.
\end{proof}
No deja de ser curioso que DE solo sea necesario para ver que el converso se cumple.

\begin{thmi}
	Un espacio uniforme (resp. uniforme de Hausdorff) $X$ es pseudometrizable (resp. metrizable) syss existe una base de su uniformidad numerable.
\end{thmi}
\begin{proof}
	$\implies$. Ejercicio para el lector.

	$\impliedby$. Sea $\{U_n\}_{n=0}^\infty$ una base numerable de la uniformidad. Luego podemos extraer de la base una sucesión de bandas $(V_n)_{n\in\N}$
	definiendo $V_0 := X\times X$ tal que $3V_{i+1} \subseteq V_i$ y por ser base eventualmente posee bandas más pequeñas que toda banda de la uniformidad.
	Luego la pseudométrica del teorema~\ref{thm:uniformity_pseudometric} nos induce la topología.

	Para ver la propiedad de Hausdorff basta notar que si $\rho(x, y) = 0$ debe darse que $d(x, y) < V_n$ para todo $n\in\N$,
	luego $x = y$ pues $\bigcap_{n\in\N} V_n = \triangle$ dado que la uniformidad es de Hausdorff.
\end{proof}

\section{Espacios de funciones}
\label{sec:function_spaces}
La sección anterior nos otorgó un fuerte y elegante lenguaje con el cual construir nuevos tipos de espacios topológicos. Tal vez una de las mejores razones
para aprender de espacios uniformes es que permiten aclarar varios detalles sobre los distintos tipos de espacios de funciones en topología.
En ésta sección nos enfocaremos en tres tipos de espacios, o mejor dicho, de convergencia sobre ellos: la convergencia puntual, la convergencia uniforme
y la convergencia casi-uniforme; lo que generalizaremos en uno de los espacios más formidables, la topología compacto-abierto.
\addtocategory{other}{brandsma:functional}

% \begin{mydefi}[Continuidad uniforme]
%	 Sea $f\colon X\to Y$ con $X,Y$ métricos, entonces se dice que $f$ es \strong{uniformemente continua}\index{uniformemente!continua (función)}
%	 si para todo $\epsilon > 0$ existe un $\delta > 0$ tal que para todo $x, y \in X$ se cumple que $d(x, y) < \delta \implies d(f(x), f(y)) < \epsilon$.
% \end{mydefi}

% \begin{prop}
%	 Se cumple:
%	 \begin{enumerate}
%		 \item Toda aplicación uniformemente continua es continua.
%		 \item Las funciones de Lipschitz son uniformemente continuas. En particular, la identidad lo es.
%		 \item La composición de uniformemente continuas es uniformemente continua.
%			 En consecuencia, los espacios topológicos (como objetos) y las funciones uniformemente continuas (como flechas)
%			 conforman una categoría denotada $\mathsf{UnifTop}$.
%		 \item Las funciones constantes son uniformemente continuas.
%	 \end{enumerate}
% \end{prop}

% Recuérdese que un espacio de Hausdorff es de Tychonoff syss admite alguna compactificación.
\begin{thm}
	Un espacio de Hausdorff compacto $K$ posee una única uniformidad que induce su topología y que tiene por base a las
	bandas abiertas de $K \times K$.
\end{thm}
\begin{proof}
	El que posea (al menos) una se sigue del teorema anterior, así que, sea $\mathcal{U}$ una uniformidad que induce la topología en $K$.
	Sea $V \subseteq K \times K$ una banda (en el sentido absoluto) que es un subconjunto abierto.
	El teorema~\ref{thm:closed_bands_are_base} implica que las bandas cerradas $\mathcal{B \subseteq U}$ forman una base y, como $K$ es de Hausdorff,
	$\bigcap \mathcal{B} = \triangle \subseteq W$.
	Tomando complementos se comprueba que existen finitas bandas cerradas $V_1, \dots, V_n \in \mathcal{B}$ tales que $\bigcap_{i=1}^{n} V_i \subseteq W$,
	de modo que $W \in \mathcal{U}$.
	Así, vemos que todas las bandas abiertas (en sentido absoluto) de $K\times K$ son bandas abiertas de $\mathcal{U}$ y,
	por el teorema~\ref{thm:closed_bands_are_base}, son base como se quería probar.
\end{proof}

\begin{cor}
	Sea $X$ un espacio de Hausdorff compacto e $Y$ un espacio uniforme.
	Toda función continua $f \colon X \to Y$ es uniformemente continua.
\end{cor}
\begin{proof}
	Sea $V \subseteq Y\times Y$ una banda abierta, entonces $(f\times f)^{-1}[V] \subseteq X\times X$ es una banda abierta de $X$,
	pero en la prueba del teorema anterior vimos que toda banda abierta pertenece a la uniformidad de $X$, por lo que ganamos.
\end{proof}
% \begin{thm}
%	 Si $X,Y$ son métricos y $X$ es compacto, entonces toda aplicación $f\colon X\to Y$ continua es uniformemente continua.
% \end{thm}
% \begin{proof}
%	 Sea $\epsilon > 0$.
%	 Nótese que para todo $x \in X$ existe $\delta(x) > 0$ tal que $d(x, y) < \delta \implies d(f(x), f(y)) < \epsilon/2$.
%	 Supongamos que construimos la el conjunto de pares ordenados $(x, \delta(x))$ que satisfacen la condición anterior, luego
%	 si a cada par le asociamos con la bola abierta $B_{\delta(x)/2}(x)$, entonces forman un cubrimiento abierto de $X$ y por compacidad
%	 admite un subcubrimiento finito $\{ B_{\delta_1/2}(x_1), \dots, B_{\delta_n/2}(x_n) \}$.

%	 Definamos $\delta := \min\{ \delta_1,\dots,\delta_n \}/2$.
%	 Si $d(x, y) < \delta$, entonces existe $i$ tal que $x \in B_{\delta_i/2}(x_i)$, de modo que
%	 $$ d(y, x_i) \le d(y, x) + d(x, x_i) < \frac{\delta_i}{2} + \frac{\delta_i}{2} = \delta_i $$
%	 De modo que $x, y \in B_{\delta}(x_i)$ lo que, por construcción significa que
%	 \begin{equation}
%		 d(f(x), f(y)) \le d( f(x), f(x_i) ) + d( f(x_i), f(y) ) < \epsilon. \tqedhere
%	 \end{equation}
% \end{proof}

\begin{mydef}
	Sean $X$ un conjunto arbitrario e $Y$ un espacio topológico, entonces se dice que una sucesión de funciones $(f_i\colon X \to Y)_{i\in\N}$
	converge puntualmente a otra $f$ si para todo $x\in X$ se cumple que $\lim_n f_n(x) = f(x)$, en cuyo caso simplemente escribiremos $f = \lim_n f_n$.

	En particular, este comportamiento se replica en la topología producto $\prod_{x\in X} Y = Y^X$,
	por ello a éste último le decimos la \strong{topología de convergencia puntual}.
\end{mydef}

\begin{ex}[Límite puntual de funciones continuas que no es continua]
	Consideremos $[0,1]^\R$ donde $f_n(x) = x^n$, es fácil probar que el límite puntual $f = \lim_n f_n$ es la función $f(x) = \sfloor{x}$ que es discontinua,
	es decir, el límite puntual de continuas puede no ser continua.
	Por éste y muchos otros ejemplos se puede notar que la topología de convergencia puntual otorga poca información en términos de funciones.
\end{ex}

\begin{mydef}
	Sean $X$ un conjunto arbitrario y sea $Y$ un espacio uniforme.
	Se define la \strong{uniformidad de convergencia uniforme}\index{uniformidad!de convergencia uniforme} sobre $Y^X$ como aquella que tiene por base a
	$$ \widetilde{V} := \{ (f, g) \in Y^X : \forall x\in X \quad d(f(x), g(x)) < V \}, $$
	donde $V$ recorre todas las bandas de $Y$.
	Denotaremos por $\Func_u(X, Y)$ al conjunto $Y^X$ con esta uniformidad.
\end{mydef}
% \begin{mydefi}\index{convergencia!uniforme}
%	 Sean $X$ un conjunto arbitrario y sea $Y$ un espacio uniforme, entonces se dice que una sucesión de funciones $(f_i\colon X \to Y)_{i\in\N}$ converge
%	 \strong{uniformemente} a otra $f$ si para toda banda $U$ sobre $Y$ existe $n_0\in\N$ tal que para todo $n\ge n_0$ y todo $x\in X$ se cumple que
%	 $d(f(x), f_n(x)) < U$ para todo $x\in X$.
%	 En cuyo caso denotaremos $f_n \unifto f$.
% \end{mydefi}

\begin{prop}
	Sea $X$ un conjunto e $Y$ un espacio métrico.
	Entonces $\Func_u(X, Y)$ es metrizable.
\end{prop}

\begin{thmi}[Teorema del límite uniforme]\index{teorema!del límite uniforme}
	Sean $X$ un espacio topológico e $Y$ un espacio uniforme.
	El conjunto de funciones $f \in Y^X$ continuas en un punto $x_0 \in X$ es cerrado en $\Func_u(X, Y)$.
	En consecuencia, $C(X, Y)$ es cerrado en $\Func_u(X, Y)$.
\end{thmi}
\begin{proof}
	...
\end{proof}

% \begin{thm}
%	 Sea $X$ un conjunto e $Y$ un espacio uniforme.
%	 Dado $A \subseteq Y^X$ definimos $c(A)$ como el conjunto de las funciones $f \in Y^X$ tales que existe una sucesión $(f_i)_i$ en $A$
%	 tal que $f_n \unifto f$.
%	 \par
%	 La función $c$ cumple las propiedades de una clausura de Kuratowski,
%	 luego podemos definir la topología inducida por $c$ como la \strong{topología de la continuidad uniforme en $Y^X$}.
% \end{thm}
% \begin{proof}
%	 Por la definición los dos primeros axiomas de la clausura de Kuratowski se cumplen. También notemos que por definición de $c$ es fácil
%	 notar que $A \subseteq B$ implica $c(A) \subseteq c(B)$.

%	 Veamos que \underline{$c(A) = c(c(A))$:}
%	 Como $A \subseteq c(A)$, es claro que $c(A) \subseteq c(c(A))$, luego basta probar la otra contención.
%	 Sea $f \in c(c(A))$, por definición, existe $(f_k)_{k\in\N}$ en $c(A)$ tal que $f = \lim_k f_k$.
%	 Vamos a construir una sub-sucesión $f_{n_k}$ tal que para todo $k\in\N_{\ne 0}$ se cumple que $n_k$ es el primer índice tal que
%	 $$ \forall x\in X\;\left( d(f(x), f_{n_k}(x)) < \frac{1}{2k} \right) $$
%	 (que se permite por buen orden de $\N$).

%	 Análogamente, cada $f_j = \lim_k g_{j,k}$ con $g_{j,k}\in A$ y definimos $\tau_j$ como la sucesión tal que para todo $k\in\N$ se cumple que
%	 $$ \forall x\in X\;\left( d(f_j(x), g_{j,\tau_j(k)}(x)) < \frac{1}{2k} \right). $$
%	 Finalmente definimos la sucesión $h_k := g_{n_k,\tau_{n_k}(k)}$ sobre $A$ y notamos que
%	 $$ \forall x\in X\;\left( d(f(x), h_k(x)) \le d(f(x), f_{n_k}(x)) + d(f_{n_k}(x), g_{n_k,\tau_{n_k}(k)}(x)) < \frac{1}{k} \right), $$
%	 y por propiedad arquimediana se cumple que $f = \lim_k h_k$.

%	 Veamos que \underline{$c(A\cup B) = c(A)\cup c(B)$.}
%	 Por contención es claro que $c(A) \cup c(B) \subseteq c(A\cup B)$, luego basta probar la otra contención.
%	 Sea $f \in c(A\cup B)$, luego existe una sucesión $(f_i)_{i\in\N}$ en $A\cup B$ tal que $f = \lim_i f_i$,
%	 luego esta sucesión debe tener infinitos términos en $A$ o $B$, digamos que es en $A$, luego sea $f_{n_i}$ la subsucesión de funciones sobre $A$.
%	 Como $f = \lim_i f_{n_i}$, entonces $f \in c(A)$.
% \end{proof}
% % Como la clausura cumple los criterios de Kuratowski, entonces determina una única topología la que le decimos la \strong{de continuidad uniforme en $M^X$}.
% Cabe destacar que la topología de convergencia uniforme es más débil que la topología de convergencia puntual.

% \begin{thmi}[Teorema del límite uniforme]\index{teorema!del límite uniforme}
%	 El conjunto de funciones continuas $C(X, M)$ es cerrado en la topología de convergencia uniforme,
%	 luego toda sucesión de funciones continuas es uniformemente convergente a una continua.
%	 % Si una sucesión $(f_i)_{i\in\N}$ de funciones continuas desde $X$ a $M$ es uniformemente continua a $f$, entonces $f$ es continua.
%	 % De hecho, $C(X, M)$ es un conjunto cerrado.
% \end{thmi}
% \begin{proof}
%	 En concreto se demostrará que toda función adherente a $C(X, M)$ es continua.
%	 En efecto sea $g$ adherente a $C(X, M)$, para ver que es continua hemos de probar que para todo $x\in X$ y $\epsilon>0$ se cumple que $f^{-1}[B_\epsilon(x)]$ es entorno de $x$.
%	 Sea $f$ una función continua perteneciente al entorno $B_{\epsilon/3}(g)$ de $g$, luego por definición, $U_x \subseteq f^{-1}[B_{\epsilon/3}(x)]$, probaremos que $U_x \subseteq g^{-1}[B_\epsilon(x)]$, para lo cual, sea $y\in U_x$, luego
%	 $$ d(g(x), g(y)) \le d(g(x), f(x)) + d(f(x), f(y)) + d(f(y), g(y)) < \epsilon, $$
%	 que es lo que se quería probar.
%	 % Por definición, esto significa que para todo $x_0\in X$ se cumple que $f^{-1}[V]$ es un entorno de $x_0$ cuando $V$ es entorno de $f(x_0)$. Notemos que como las bolas abiertas son base de espacios métricos, se cumple que todo entorno $V$ de $f(x_0)$ contiene a un conjunto de la forma $B_\epsilon(f(x_0))$. Es decir, que $f$ sea continua en $x_0$ equivale a decir que para todo $\epsilon > 0$ existe un entorno $U$ de $x_0$ tal que para todo $x\in U$ se cumple que $d(f(x_0), f(x)) < \epsilon$, así que trataremos de probar eso.
%	 % \par
%	 % Sea $\epsilon > 0$, por definición de uniformidad continua, existe $N\in\N$ tal que para todo $i\ge N$ se cumple que $d(f(x_0), f_i(x_0)) < \epsilon/3$.
%	 % Como $f_N$ es continua, existe un entorno $U$ de $x_0$ tal que para todo $x\in U$ se cumple que $d(f_N(x_0), f_N(x)) < \epsilon/3$.
%	 % Finalmente, para todo $x\in U$ se cumple que
%	 % $$ d(f(x_0), f(x)) \le d(f(x_0), f_N(x_0)) + d(f_N(x_0), f_N(x)) + d(f_N(x), f(x)) < \epsilon. $$
%	 % Lo que prueba la continuidad de $f$.
% \end{proof}

\begin{ex}[Una sucesión de funciones discontinuas que convergen uniformemente a una continua]
	Sean $f_n \colon \R \to \R$ dada por
	$$ f_n(x) = \frac{1}{n}\chi_{\Q}(x) =
	\begin{cases}
		\frac{1}{n}, & x\in\Q \\
		0, & x\notin\Q \\
	\end{cases} $$
	Claramente éstas funciones son discontinuas y convergen uniformemente a la función nula que es continua.
\end{ex}

\begin{mydef}
	Sea $X$ un conjunto, $\Sigma \subseteq \P X$ un cubrimiento de $X$ e $Y$ un espacio uniforme.
	Se denota por $\Func_\Sigma(X, Y)$ al conjunto $Y^X$ con la uniformidad que tiene por base a las intersecciones finitas de bandas
	$$ \widetilde{V}_A := \{ (f, g) : \forall x\in A \quad d(f(x), g(x)) < V \}, $$
	donde $V$ recorre las bandas de $Y$, y $A$ recorre los elementos de $\Sigma$.

	Sea $X$ un espacio topológico y $\Sigma$ es la familia de subconjuntos compactos de $X$, denotamos $\Func_c(X, Y) := \Func_\Sigma(X, Y)$.
	Esta se conoce como la \strong{uniformidad de convergencia uniforme sobre compactos}\index{uniformidad!de convergencia uniforme sobre compactos}.
\end{mydef}
% Nótese que $\Sigma$ siempre incluye los puntos $\{ x \}$, así

\begin{prop}
	Sea $X$ un conjunto, $\Sigma \subseteq \P X$ un cubrimiento de $X$ e $Y$ un espacio uniforme.
	Se cumplen:
	\begin{enumerate}
		\item Si $A \in \Sigma$, la restricción $\rho_A \colon \Func_\Sigma(X, Y) \to \Func_u(A, Y)$ es uniformemente continua.
		\item En consecuencia, para todo $x \in X$ la aplicación de evaluación
			\begin{align*}
				\ev_x \colon \Func_\Sigma(X, Y) &\longrightarrow Y \\
				f &\longmapsto f(x)
			\end{align*}
			es uniformemente continua.
		\item Si $Y$ es de Hausdorff, entonces $\Func_\Sigma(X, Y)$ también.
	\end{enumerate}
\end{prop}
\begin{proof}
	\begin{enumerate}
		\item Basta notar que, dada la banda $\widetilde{V}$ sobre $\Func_u(A, Y)$, su preimagen $(\rho_A\times \rho_A)^{-1}[\widetilde{V}]
			= \widetilde{V}_A$ es una banda de $\Func_\Sigma(A, Y)$.
		\item Sea $A \in \Sigma$ tal que $x \in A$.
			Basta notar que el siguiente diagrama conmuta (en $\mathsf{Unif}$):
			\begin{center}
				\begin{tikzcd}
					\Func_\Sigma(X, Y) \dar["\ev_x"'] \rar["\rho_A"] & \Func_u(A, Y) \dar["\ev_x"] \\
					Y \rar[equals]				   & Y
				\end{tikzcd}
			\end{center}
		\item Ejercicio para el lector. \qedhere
	\end{enumerate}
\end{proof}

\begin{thm}
	Sea $X$ un espacio topológico, e $Y$ un espacio uniforme.
	La topología de convergencia uniforme en compactos sobre $C_c(X, Y)$ tiene por subbase a los conjuntos
	$$ T(K; U) := \{ f \in C(X, Y) : f[K] \subseteq U \}, $$
	donde $K$ recorre los subespacios compactos de $X$ y $U$ recorre los abiertos de $Y$.
\end{thm}

\begin{mydef}
	Sea $X, Y$ un par de espacios topológicos.
	Se denota por $C_c(X, Y)$ al espacio de funciones continuas $C(X, Y)$ con la topología inducida por la subbase de conjuntos
	$$ T(K; U) := \{ f \in C(X, Y) : f[K] \subseteq U \}, $$
	donde $K$ recorre los subespacios compactos de $X$ y $U$ recorre los abiertos de $Y$.
	Esta se llama la \strong{topología compacto-abierto}\index{topología!compacto-abierto}.
\end{mydef}

\begin{thm}
	Sea $X, Y, Z$ un trío de espacios topológicos. Se cumplen:
	\begin{enumerate}
		\item Si $X$ es de Hausdorff localmente compacto, la aplicación de evaluación
			\[
				\ev \colon C_c(X, Y) \times X \longrightarrow Y, \qquad \ev(f, x) := f(x)
			\]
			es continua.
		\item Si $Y$ es de Hausdorff localmente compacto, la composición $-\circ- \colon C_c(X, Y) \times C_c(Y, Z) \to C_c(X, Z)$ es continua.
	\end{enumerate}
\end{thm}

\begin{lem}
	Sean $X, Y$ un par de espacios de Hausdorff, sea $\mathcal{S}$ una subbase de $Y$ y sea $\mathcal{K}$ una familia de subespacios compactos de $X$
	tal que para todo compacto $K \subseteq X$ y abierto $U \subseteq X$ que satisfagan que $K \subseteq U$ existen $K_1, \dots, K_n \in \mathcal{K}$
	de modo que $K \subseteq K_1 \cup \cdots \cup K_n \subseteq U$.
	Entonces la familia de los $T(K; V)$ con $K \in \mathcal{K}, V \in \mathcal{S}$ es una subbase de $C_c(X, Y)$.
\end{lem}
\begin{thm}
	Sean $X, Y, Z$ un trío de espacios de Hausdorff.
	Entonces la aplicación
	\begin{center}
		\begin{tikzcd}[sep=large]
			C_c(X \times Y, Z) \rar[hook] & C_c(X, C_c(Y, Z))
		\end{tikzcd}
	\end{center}
	dada por $\Phi(f)(x)(y) := f(x, y)$ es continua y es un encaje topológico.
	Si $Y$ es localmente compacto, entonces $\Phi$ es de hecho un homeomorfismo.
\end{thm}

\thmdep{DE}
\begin{lem}
	Si $X$ es normal, $A\subseteq X$ cerrado y $g \in C(A)$ es tal que $|g(a)| \le c$ para todo $a\in A$, entonces existe $h \in C(X)$ tal que
	\begin{enumerate}
		\item $|h(x)| \le \frac{1}{3}c$ para todo $x\in X$.
		\item $|h(a) - g(a)| \le \frac{2}{3}c$ para todo $a\in A$.
	\end{enumerate}
\end{lem}
\begin{proof}
	Sean
	$$ A_+ := \left\{ a\in A: g(a) \ge \frac{1}{3}c \right\},\quad A_- := \left\{ a\in A: g(a) \le -\frac{1}{3}c \right\}. $$
	que son claramente cerrados y disjuntos en $A$, que es cerrado, luego son cerrados y disjuntos en $X$.
	Como $X$ es normal, existe una función de Urysohn $h\colon X \to \left[ -\frac{1}{3}c, \frac{1}{3}c \right]$
	tal que $h[A_+] = \frac{1}{3}c$ y $h[A_-] = -\frac{1}{3}c$.
	$h$ satisface todos los requerimientos.
\end{proof}

\begin{thm}[de extensión de Tietze-Urysohn]
	Si $X$ es normal, $A\subseteq X$ cerrado y $f \in C(A)$, entonces $f$ es continuamente extensible a $X$.
	Más aún si $|f(a)| \le c$ (resp. $<c$) para todo $a\in A$, entonces la extensión $\bar f$ cumple que $\bar f(x) \le c$ (resp. $<c$) para todo $x\in X$.
	% Toda función continua $f: C\to\R$ (o a $[0,1]$) con $C$ cerrado en $X$ normal es continuamente extensible a $X$.
\end{thm}
\begin{proof}
	Contemplamos tres casos:
	\begin{enumerate}[a)]
		\item \underline{$|f(a)|\le c$ para todo $a\in A$:}
			\\
			Por el lema anterior existe $g_0$ tal que $|f(a) - g_0(a)| \le \frac{2}{3}c$ y $|g_0(x)| \le \frac{1}{3}c$.
			Luego aplicamos el lema anterior sobre $f - g_0$ para obtener $g_1$ tal que $|f(a) - g_0(a) - g_1(a)| \le \frac{2}{3}\cdot \frac{2}{3}c$ y $|g_1(x) \le \frac{1}{3}\cdot \frac{2}{3}c$.
			Y así, para obtener inductivamente\footnote{Aquí se aplica también DE.} una función $g_{n+1}$ tal que $|f(a) - g_0(a) - \cdots - g_n(a)| \le \frac{2}{3}\cdot \left(\frac{2}{3}\right)^n c$ y que $|g_n(x)| \le \frac{1}{3}\cdot \left(\frac{2}{3}\right)^n c$.
			\\
			Ahora se define $F \in C(X)$ tal que
			$$ F(x) := \sum_{n=0}^\infty g_n(x), $$
			que por el teorema del sandwich satisface que $F|_A = f$, mientras que
			$$ |F(x)| \le \sum_{n=0}^\infty |g_n(x)| \le \frac{1}{3}c \cdot \frac{1}{1 - \frac{2}{3}} = c. $$
			La continuidad se hereda del hecho de que las sumas parciales son continuas y convergen uniformemente a $F$.

		\item \underline{$|f(a)|<c$ para todo $a\in A$:}
			Aplicando el inciso anterior se construye $F$ tal que $|F(x)| \le c$ para todo $x\in X$, luego sea $B := F^{-1}[\{\pm c\}]$, claramente $B$ es cerrado y disjunto de $A$, luego por el lema de Urysohn existe una función continua $\phi:X \to [0,1]$ tal que $\phi[B] = 0$ y $\phi[A] = 1$, luego $\bar f := F\cdot\phi$ es una extensión continua de $f$ y cumple la desigualdad.

		\item \underline{$f$ no es necesariamente acotado:}
			Basta notar que $(-c, c)$ es homeomorfo a $(0,1)$ que es homeomorfo a $\R$. \qedhere
	\end{enumerate}
\end{proof}
\thmdep{}

\begin{prop}
	Si $Y$ es métrico, entonces el conjunto de funciones acotadas $(Y^X)^*$ es métrico con
	$$ d(f,g) := \sup\{ d(f(x), g(x)) : x\in X\} = \diam(\Img f\cup\Img g), $$
	a la que llamamos \strong{métrica de Cebyshev}\index{metrica@métrica!de Cebyshev}.
	En consecuente si $Y$ es de completo, $(Y^X)^*$ también.
\end{prop}
Notemos que si $X$ es compacto e $Y$ es métrico, entonces $C(X, Y)^* = C(X, Y)$.
Si $Y$ es normado, entonces $(Y^X)^*$ también con $\|f\|_\infty := \sup( \Img(\|f\|) )$, a la que usualmente se le dice \strong{norma de Chebyshev}.

En particular, la norma de Chebyshev induce la topología de convergencia uniforme sobre $(Y^X)^*$.

\begin{prop}[teorema de Dini]
	Sea $X$ compacto y $(f_i)_{i\in I}$ una sucesión de $C(X, \R)$ tal que para todo $i\in\N$ y $x\in X$ se cumpla que $f_i(x) \le f_{i+1}(x)$.
	Si existe $f\in\Func(X, \R)$ tal que para todo $x\in X$ se da que $f(x) = \lim_i f_i(x)$, entonces la convergencia es uniforme.
\end{prop}
\begin{proof}
	Sea $\epsilon > 0$.
	Definamos $F_i := \{x : |f(x) - f_i(x)| \ge \epsilon\}$, luego $F_i$ son cerrados pues $F_i^c$ son abiertos (¿por qué?),
	además $F_0 \supseteq F_1 \supseteq F_2 \supseteq \cdots$.
	Por construcción se cumple que $\bigcap_{i=0}^n F_i = \emptyset$, luego no pueden tener la PIF,
	por ende hay algún $F_i$ que es vacío, que es lo que se quería probar.
\end{proof}

En esta sección usaremos la noción de polinomios funcionales, éstos se asemejan a los comunes pero sus variables son funciones continuas,
e.g. $ \frac12 f(x)^3 + g(x)\cdot h(x) + 5 $. En general obviaremos el <<$(x)$>>, por comodidad.

\begin{lem}
	Sea $f \in C\big( X, [0,1] \big)$, entonces existe una sucesión de polinomios funcionales tal que convergen uniformemente a $\sqrt{f}$.
\end{lem}
\begin{proof}
	Vamos a definir recursivamente la sucesión como
	$$ g_0 := 0, \quad g_{i+1} := g_i + \frac{1}{2}(f - g_i^2). $$
	Ahora queremos probar convergencia por teorema de Dini, así que hemos de probar que $g_i(x) \le \sqrt{f}(x)$ para todo $i$ y $x$, lo haremos por inducción, el caso $i = 0$ es trivial y
	$$ \sqrt{f} - g_{i+1} = \sqrt{f} - g_i - \frac{1}{2}(f - g_i^2) = (\sqrt{f} - g_i) \left( 1 - \frac{1}{2}(\sqrt{f} + g_i) \right) $$
	dado que $g_i \le \sqrt{f} \le 1$, se da que
	$$ \sqrt{f} - g_{i+1} \ge (\sqrt{f} - g_i)\left( 1 - \frac{1}{2}2\sqrt{f} \right) \ge 0 $$
	lo que completa la demostración de la desigualdad.
	Usando que $g_i \le \sqrt{f}$ y la definición de $g_i$ se comprueba que son uniformemente crecientes, luego queda al lector probar que en el límite los $g_i$ se van a $\sqrt{f}$.
\end{proof}

Vemos que el teorema se aplica para el intervalo $[0,1]$, pero si $f\in C(X, \R)$ como $X$ es compacto, la imagen ha de ser compacta, luego posee un máximo global $M$ y $\frac 1M f(x)$ está en el intervalo $[0,1]$ y así podemos conseguir la raíz en casos generales.

\begin{mydef}
	Se dice que una familia $\{f_i\}_{i\in I}$ de funciones de dominio $X$ \strong{separa puntos} si para todo par de puntos distintos $x,y\in X$ existe alguna $f_i$ tal que $f_i(x) \ne f_i(y)$.
\end{mydef}

\begin{thmi}[Teorema de Stone-Weierstrass]\index{teorema!de Stone-Weierstrass}
	Todo anillo de funciones continuas $A$ que separan puntos y que contiene a las constantes sobre un compacto $X$ es denso en $C(X, \R)$.
\end{thmi}
\begin{proof}
	Probaremos que $\overline{A} = C(X, \R)$.
	\par
	Primero notemos que $\min(f, g)(x) := \min(f(x), g(x))$ pertenece a $\overline{A}$, pues
	$$ \min(f,g) = \frac{1}{2}(f+g-|f-g|),\quad \max(f,g) = \frac{1}{2}(f+g+|f-g|), $$
	donde $|f| = \sqrt{f^2}$ y vimos que la raíz de toda función positiva-valorada de $A$ pertenece a $\overline{A}$.
	\par
	Sea $f\in C(X, \R)$, queremos probar que para todo $\epsilon > 0$ existe $f_\epsilon \in \overline{A}$ tal que para todo $x\in X$ se cumple que
	$$ |f(x) - f_\epsilon(x)| < \epsilon. $$
	Por construcción para todo $a,b\in X$ existe $h_{a,b} \in A$ tal que $h_{a,b}(a) \ne h_{a,b}(b)$, luego se define $g_{a,b}(x) := \frac{h(x) - h(a)}{h(b) - h(a)} \in A$ tal que $g_{a,b}(a) = 0$ y $g_{a,b}(b) = 1$. Luego, definimos
	$$ f_{a,b}(x) := (f(b) - f(a))g_{a,b}(x) + f(a) \in A, $$
	que cumple que $f_{a,b}(a) = f(a)$ y $f_{a,b}(b) = f(b)$. Luego, definimos
	$$ U_{a,b} := \{x: f_{a,b}(x) < f(x) + \epsilon\},\quad V_{a,b} := \{x: f_{a,b}(x) > f(x) - \epsilon\}, $$
	los cuales resultan ser entornos de $a$ y de $b$.
	Luego $(U_{a,b})_{a\in X}$ es un cubrimiento abierto de $X$, por ende, posee un subcubrimiento abierto finito $(U_{a_i,b})_{i=1}^n$.
	Notemos que $f_b := \min(f_{a_i,b})_{i=1}^n \in \overline{A}$ que cumple que $ f_b(x) < f(x) + \epsilon $ para todo $x\in X$ y que $f_b(x) > f(x) - \epsilon$ para todo $x \in V_b := \bigcap_{i=1}^n V_{a_i, b}$ donde $V_b$ es entorno de $b$.
	Por lo tanto $(V_b)_{b\in X}$ es un cubrimiento abierto de $X$, por ende, posee un subcubrimiento abierto finito $(V_{b_i})_{i=1}^m$.
	Finalmente, $f_\epsilon := \max(f_{b_i})_{i=1}^m$ cumple lo pedido y pertenece a $\overline{A}$.
	\par
	Como $\overline{A}$ es cerrado y todo entorno de toda función continua corta a $A$, luego toda función continua es adherente a $\overline{A}$, en conclusión $\overline{A} = C(X,\R)$ como se quería probar.
\end{proof}

Algunos le agregan a $X$ la condición de ser de Hausdorff, de modo que sea un espacio normal y, en consecuencia de Urysohn. Sin embargo, en el enunciado admitimos que simplemente basta que sepamos que $A$ contiene tales funciones.
La condición de ser de Hausdorff (sumado a DE) simplemente probaría que tal subanillo propio siempre existe, sin embargo, si nosotros proponemos, por ejemplo, el anillo de polinomios no haría falta elección para aplicar el resultado anterior.

\begin{cor}
	Toda función $f: [a,b] \to \R$ puede ser uniformemente aproximada por polinomios.
\end{cor}
Otro ejemplo interesante es aproximar mediante polinomios de la función exponencial, pues $\exp(x)^n = \exp(nx)$, esto en algún momento se puede generalizar mucho más y en el cuerpo de los números complejos en un resultado conocido como el teorema de Fourier.

\begin{thmi}[Teorema de Stone-Weierstrass]
	Todo anillo de funciones continuas $A$ que es cerrado por conjugados, que separa puntos y contiene a las constantes sobre un compacto $X$ es denso en $C(X,\C)$.
\end{thmi}
\begin{proof}
	Para ello definimos $A_{\Re} := \{\Re(f) : f\in A\}$ y $A_{\Im} := \{\Im(f) : f\in A\}$ y vemos que cumplan la hipótesis de la versión real.
	Claramente separan puntos, son cerrados por suma, contienen constantes y para ver que son cerradas por producto vamos a ver que es cerrado bajo producto supongamos que sean $f,g\in A$ y definamos $f_1 := \Re(f)$ y $f_2 := \Im(f)$, y análogo para $g$.
	Como $A$ es cerrado por conjugados, vemos que $f_1 = \frac{f + \overline f}{2} \in A$, de modo que es fácil ver que $f_1g_1 \in A_{\Re}$.
	Por un argumento similar es fácil ver que $f_2\in A$, de modo que $f_2g_2\in A$ y luego $\imaginary f_2g_2\in A$ luego $f_2g_2\in A_{\Im}$.
	\\
	Finalmente, si $f\in C(X,\C)$, entonces $f_1,f_2\in C(X)$.
	Luego, por Stone-Weierstrass real se cumple que para todo $\epsilon > 0$ existen $g_\epsilon,h_\epsilon\in\overline{A_{\Re}}$ tales que para todo $x\in X$ se cumple:
	$$ \max\{ |f_1(x) - g_\epsilon(x)|, |f_2(x) - h_\epsilon(x)| \} < \frac{\epsilon}{2} $$
	finalmente, por desigualdad triangular $f_\epsilon := g_\epsilon + \imaginary h_\epsilon\in \overline A$ está a menos de $\epsilon$ de $f$, como se quería probar.
\end{proof}

\begin{cor}
	Si $\D := \overline{ B_1(0) }$, entonces
	\nomenclature{$\D$}{$ = \overline{ B_1(0) } = \{ z\in\C : |z| \le 1 \}$}
	$$ A := \left\{ \sum_{k = -n}^n \alpha_n\exp(inz) : n\in\N\wedge\forall i\;(\alpha_i\in\C) \right\} $$
	es denso en $C(\D, \C)$.
\end{cor}

\begin{thm}
	Sea $X$ un espacio compacto, sea $R := C(X; \R)$ considerado como espacio normado y sea $\mathfrak{a}$ un ideal cerrado en $R$.
	Denotemos
	$$ Z := \{ x\in X : \forall f\in \mathfrak{a} \; f(x) = 0 \}, $$
	si $g \in R$ se anula en $Z$, entonces $g \in \mathfrak{a}$.
\end{thm}
\begin{proof}
	Sea $\epsilon > 0$, definamos $U := g^{-1}\big[ (-1, \epsilon) \big] \subseteq X$ que es abierto y contiene a $Z$,
	luego $S := U^c$ es cerrado y, por tanto, compacto.
	Todo $y \in S$ no está en $Z$, luego existe $h_y \in \mathfrak{a}$ tal que $h_y(y) \ne 0$ y de hecho,
	$f_y$ no se anula en un entorno abierto $V_y$ de $y$.
	Los conjuntos $V_y$'s forman un cubrimiento abierto de $S$ y como $S$ es compacto posee un subcubrimiento finito:
	$$ S \subseteq V_{y_1} \cup \cdots \cup V_{y_n}. $$
	Y a dichos $V_{y_i}$'s tiene un $h_{y_i}$ asociado; luego
	$$ h := h^2_{y_1} + \cdots + h^2_{y_n} \in \mathfrak{a} $$
	y además no se anula en $S$.
	Como $S$ es compacto, $h$ alcanza un mínimo $a > 0$ en $S$.
	Definamos la función
	$$ H_n := \frac{nh}{1 + nh}. $$
	Nótese que $1 + nh$ es estrictamente positiva y continua en $X$, luego posee inversa continua,
	y $h \in \mathfrak{a}$, así que $H_n \in \mathfrak{a}$ también.
	Nótese que $fH_n \in \mathfrak{a}$ y que $H_n$ converge a 1 en $S$.
\end{proof}

\begin{mydef}\index{equicontinuidad}
	Si $M$ es métrico, se dice que un conjunto $A\subseteq M^X$ es
	\begin{description}
		\item[Equicontinuo] Si para todo $x_0\in X$ y $\epsilon > 0$ existe un entorno $U$ de $x_0$ tal que para todo $f\in A$
			se cumple que $f[U] \subseteq B_\epsilon(f(x_0))$.
		\item[Uniformemente equicontinuo] Si $X$ es métrico y para todo $\epsilon > 0$ existe un $\delta > 0$ tal que para todo $x_0\in X$
			y $f\in A$ se cumple que $d(x, x_0) < \delta \implies d(f(x), f(x_0)) < \epsilon$.
	\end{description}
\end{mydef}
Claramente si $X$ es métrico todo conjunto uniformemente equicontinuo es equicontinuo, pero de hecho:

\begin{prop}
	Si $X$ es métrico compacto, todo conjunto equicontinuo en $M^X$ es uniformemente equicontinuo.
\end{prop}
\begin{hint}
	La demostración ocupa el mismo truco que para probar que continuo $\implies$ uniformemente continuo en dominio compacto.
\end{hint}

\thmdep{AEN}
\begin{thmi}[Teorema de Ascoli-Arzelà]\index{teorema!de Ascoli-Arzelà}
	Sea $X$ métrico compacto e $Y$ métrico completo.
	$F$ es relativamente compacto en la topología de convergencia uniforme syss cumple las siguientes condiciones:
	\begin{itemize}
		\item $F$ es uniformemente equicontinuo.
		\item Para todo $x\in X$ el conjunto $\{f(x): f\in F\}$ es relativamente compacto.
	\end{itemize}
\end{thmi}
\begin{proof}
	$\impliedby$. Si $X$ es compacto, entonces es totalmente acotado, por lo que para todo $n\in\N_{\ne 0}$ se puede definir $D_n$ como un conjunto finito $\{x_1,\dots,x_k\}$ tal que $X = \bigcup_{i=1}^k B_{1/n}(x_i)$, y por ende, $D := \bigcup_{n\in\N_{\ne 0}} D_n$.
	Sea $F$ equicontinuo y uniformemente acotado, vamos a probar que es secuencialmente compacto.
	\par
	Sea $(f_i)_{i\in\N}$ una sucesión en $A$, de él se puede extraer una sucesión como $(f_i(d_0))_{i\in\N}$ está contenido en un compacto, entonces posee una subsucesión convergente, sea $(f_{\sigma(0,i)})_{i\in\N}$ dicha subsucesión.
	Asimismo, de ésta se puede extraer la subsucesión $(f_{\sigma(1,i)})_{i\in\N}$ de modo que $(f_{\sigma(1,i)}(d_1))_{i\in\N}$ converja y así.
	Defínase $g_i := f_{\sigma(i,i)}$.
	\par
	Sea $\epsilon > 0$.
	Por la equicontinuidad existe $\delta > 0$ tal que para todo $n\in\N$ se cumple que $d(x, y) < \delta$ implica $d(g_n(x), g_n(y)) < \epsilon/3$.
	Sea $m > 1/\delta$ un natural, por construcción existe $n_0 \in \N_{\ne 0}$ tal que para todo $p,q\ge n_0$ y todo $d\in D_m$ se cumple que $d(g_p(d), g_q(d)) < \epsilon/3$.
	Finalmente, para todo $x\in X$ existe $d\in D_m$ tal que $d(x, d) < 1/m < \delta$ que cumple que para todo $p,q\ge n_0$ pasa
	$$ d(g_p(x), g_q(x)) \le d(g_p(x), g_p(d)) + d(g_p(d), g_q(d)) + d(g_q(d), g_q(x)) < \epsilon. $$
	Como se quería probar.
	\par
	$\implies$. Si $F$ es relativamente compacto, entonces es totalmente acotado, por ende, dado $\epsilon > 0$ existen $f_1,\dots,f_k\in F$ tales que $F \subseteq \bigcup_{i=1}^k B_{\epsilon/3}(f_i)$.
	Luego, dado $x\in X$ se cumple que
	$$ F(x) := \{f(x): f\in F\} \subseteq \bigcup_{i=1}^k B_{\epsilon/3}(f_i(x)) $$
	luego $F(x)$ es totalmente acotado, y su clasurura también y es cerrado, ergo es completa, en conclusión $\overline{F(x)}$ ha de ser compacto.
	\par
	Sea $\epsilon > 0$, hemos de probar que $F$ es equicontinuo, como $X$ es compacto, los $f_i$s son uniformemente continuos, por ende existe $\delta > 0$ tal que para todo $x,y\in X$ se cumple $d(x,y)<\delta$ implica $d(f_i(x), f_i(y)) < \epsilon/3$.
	Por el cubrimiento para todo $f\in F$ existe $f_i$ tal que $d(f, f_i) < \epsilon/3$, por lo que
	\begin{equation}
		d(f(x), f(y)) \le d(f(x), f_i(x)) + d(f_i(x), f_i(y)) + d(f_i(y), f(y)) < \epsilon. \tqedhere
	\end{equation}
\end{proof}
\thmdep{}

\section{Espacios ordenados}
\begin{mydefi}
	Sea $(X, \le)$ un conjunto linealmente ordenado, se define la topología inducida por el orden como la que tiene por base
	a los intervalos abiertos del conjunto.
	\par
	Se dice que $C \subseteq X$ es \strong{convexo}\index{convexo} si para todo $x, y\in C$ se cumple que $[x, y] \subseteq C$.
\end{mydefi}
Queremos ver que los espacios ordenados son completamente normales, para ello veremos algo más fuerte.

\begin{mydef}
	Un espacio $X$ que es $T_1$ se dice \strong{monótonamente normal}\index{espacio!monótonamente normal} si existe una función $G$ que a cada par
	$(A, B)$ de cerrados disjuntos se cumple que $G(A, B)$ es un abierto tal que
	\begin{enumerate}
		\item $A \subseteq G(A, B) \subseteq \overline{G(A, B)} \subseteq B^c$.
		\item Si $A \subseteq A'$ y $B' \subseteq B$ entonces $G(A, B) \subseteq G(A', B')$ (monotonía).
	\end{enumerate}
	En éste caso $G$ se dice un \strong{operador monótono de normalidad}.
\end{mydef}

\begin{prop}
	Si $X$ es monótonamente normal entonces posee un operador monótono de normalidad $G_0$ tal que
	$G_0(A, B) \cap G_0(B, A) = \emptyset$.
\end{prop}
\begin{proof}
	Basta definir $G_0(A, B) := G(A, B) \setminus \overline{G(B, A)}$.
\end{proof}

\begin{thm}
	Sea $X$ un espacio que es $T_1$ es monótonamente normal syss existe un operador $H$ tal que para
	todo par $(p, U)$ donde $p \in U \subseteq X$ y $U$ es abierto se cumple que $H(p, U)$ es un abierto tal que:
	\begin{enumerate}
		\item $p \in H(p, U) \subseteq U$.
		\item Si $H(p, U) \cap H(q, V) \ne \emptyset$, entonces $p \in V$ o $q \in U$.
	\end{enumerate}
\end{thm}
\begin{proof}
	$\implies$.
	Sea $G$ un operador monótono de normalidad tal que $G(A, B) \cap G(B, A) = \emptyset$.
	Luego sea $H(p, U) := G(\{p\}, U^c)$, entonces claramente $H$ cumple la propiedad 1.
	Si $p \notin V$ y $q \notin U$, entonces aplicando la monotonía se cumple
	$$ H(p, U) \cap H(q, V) = G(\{p\}, U^c) \cap G(\{q\}, V^c) \subseteq G(\{p\}, \{q\}) \cap G(\{q\}, \{p\}) = \emptyset. $$

	$\impliedby$. Sean $A, B$ cerrados disjuntos, entonces definamos
	$$ G(A, B) := \bigcup \{ H(a, U) : a\in A \wedge U \subseteq B^c \} $$
	que es abierto por ser unión de abiertos.
	Nótese que para todo $a\in A$ se cumple que $a \in H(a, B^c) \subseteq G(A, B)$, es decir, $A \subseteq G(A, B)$.
	Sea $b\in B$, entonces $b \in V := H(b, A^c)$, luego por la propiedad 2. de $H$ se deduce que $V \cap G(A, B) = \emptyset$ y que
	$$ G(A, B) \subseteq V^c \implies \overline{ G(A, B) } \subseteq V^c $$
	por minimalidad de la clausura y porque $V^c$ es cerrado.
	En definitiva $b \notin \overline{ G(A, B) }$ como se quería ver.
\end{proof}

\begin{prop}
	Sea $X$ un espacio que es $T_1$ y $H$ un operador como el teorema anterior pero sólo aplicable para $(p, B)$ donde $B$ es un abierto perteneciente
	a una base $\mathcal{B}$, entonces existe un operador $\overline H$ que funciona para todo par $(p, U)$ y en consecencia $X$ es monótonamente normal.
\end{prop}
\begin{proof}
	Basta definir:
	$$ \overline H(p, U) := \bigcup \{ H(p, B) : B\subseteq U \} $$
	que claramente cumple con las hipótesis exigidas.
\end{proof}

\begin{thmi}
	Ser monótonamente normal es hereditario.
	En consecuencia, todo espacio monótonamente normal es completamente normal.
\end{thmi}
\begin{proof}
	Sea $Y \subseteq X$ un subespacio y sea $H$ el operador descrito en el teorema ante-anterior.
	Para $U$ abierto en $Y$ definamos:
	$$ U^* := \bigcup\{ V : V\text{ es abierto en }Y \wedge V\cap Y = U \} $$
	Y definamos $H_Y(p, U) := H(p, U^*) \cap Y$ que cumple con las hipótesis exigidas.
\end{proof}

\thmdep{AE}
\begin{thm}
	Todo espacio ordenado es monótonamente normal y, en consecuencia, es completamente normal.
\end{thm}
\begin{proof}
	Sea $X$ un espacio ordenado y consideremos:
	$$ X^* := \{ (-n, 0) : n\in\N \} \cup X \times \{1\} \cup \{ (+n, 2) : n\in\N \} $$
	bajo el orden lexicográfico.
	Es decir, $X^*$ es también un conjunto ordenado que extiende a $X$, pero que no posee ni máximo ni mínimo.
	Por AE sea $\preceq$ un buen orden sobre $X^*$ de modo que $E( A ) := \min_\preceq(A)$ es una función de elección sobre $X^*$.
	Veremos que $X^*$ es monótonamente normal construyendo un operador $H$ sobre una base.

	Para ello consideremos un par $\big( p, (a, b) \big)$ y definamos:
	$$ x(p, a) :=
	\begin{cases}
		a,		   &(a, p) =   \emptyset \\
		E\big( (a, p) \big), &(a, p) \ne \emptyset
	\end{cases},
	\quad y(p, b) :=
	\begin{cases}
		b,		   &(p, b) =   \emptyset \\
		E\big( (p, b) \big), &(p, b) \ne \emptyset
	\end{cases} $$
	Luego definimos $H\big(p, (a, b)\big) = \big( x(p, a), y(p, b) \big)$ que es efectivamente abierto.
	Claramente $p \in H\big(p, (a, b)\big) \subseteq (a, b)$.

	Sean $p,q,(a,b),(c,d)$ tales que se cumpla que $H\big( p, (a, b) \big) \cap H\big( q, (c, d) \big) \ne \emptyset$ que contiene a un elemento $z$.
	Sin perdida de generalidad supongamos que $x(p, a) \le x(q, c)$.
	Queremos probar que $p \in (c, d)$ o que $q \in (a, b)$.
	Vemos dos posibles casos:
	\begin{enumerate}[a)]
		\item \underline{$y(q, d) \le y(p, b)$:}
			Luego $q \in \big( x(q, c), y(q, d) \big) \subseteq \big( x(p, a), y(p, b) \big) \subseteq (a, b)$.

		\item \underline{$y(q, d) > y(p, b)$:}
			Éste caso se hará por contradicción.
			Primero recordar que la situación es la siguiente:
			$$ x(p, a) \le x(q, c) < z < y(p, b) < y(q, d). $$
			Luego si $p > x(q, c)$, entonces $p \in \big( x(q, c), y(q, d) \big) \subseteq (c, d)$;
			así que suponemos que $p \le x(q, c)$.
			Similarmente si $q < y(p, b)$, entonces $q \in \big( x(p, a), y(p, b) \big) \subseteq (a, b)$;
			así que suponemos que $q \ge y(p, b)$. Es decir:

			Pero si $p = x(q, c)$, considerando que $c \le x(q, c) < z < y(p, b) \le q$ entonces $z \in (c, q) \ne \emptyset$
			así que $p \in (c, q) \subseteq (c, d)$.
			Y si $q = y(p, b)$, considerando que $p < x(q, c) < z < y(p, b) \le b$ entonces $z \in (p, b) \ne \emptyset$
			así que $q \in (p, b) \subseteq (a, b)$.
			Es decir, la situación es la siguiente:
			$$ x(p, a) < p < x(q, c) < z < y(p, b) < q < y(q, d). $$
			Pero entonces $x(q, c) \in (p, z) \subseteq (p, b)$ y por definición $y(p, b) := \min_\preceq(p, b)$, es decir, $y(p, b) \preceq x(q, c)$.
			Y similarmente $y(p, b) \in (z, q) \subseteq (c, q)$ y por definición $x(q, c) := \min_\preceq(c, q)$, es decir, $x(q, c) \preceq y(p, b)$.
			En conclusión, $x(q, c) = y(p, b)$, lo que contradice la existencia de $z$. \qedhere
	\end{enumerate}
\end{proof}
\thmdep{}

\begin{mydef}
	Se dice que un conjunto ordenado es \strong{orden-completo} si todo subconjunto acotado posee ínfimo.
\end{mydef}

\begin{thmi}
	Un espacio ordenado no vacío es compacto syss es orden-completo y posee extremos (mínimo y máximo).
\end{thmi}
\begin{proof}
	$\implies$.
	Veamos las dos condiciones:
	\begin{enumerate}[i)]
		\item \underline{$X$ posee extremos:} De lo contrario entonces $\{ (-\infty, x) \}_{x\in X}$ sería un cubrimiento por abiertos
			que no posee subcubrimiento finito (pues todo conjunto finito posee máximo).
			Análogamente para el mínimo.

		\item \underline{$X$ es completo:} De lo contrario habría un corte $(A, B)$ tal que $A,B$ son convexos, todo elemento de $A$ es menor que
			todo elemento de $B$ y ni $A$ ni $B$ poseen mínimo ni máximo. En cuyo caso se daría que
			$$ A = \bigcup_{a\in A} (-\infty, a), \quad B = \bigcup_{b\in B} (b, \infty) $$
			lo que prueba que $A,B$ son abiertos y aquí se forma un cubrimiento por abiertos de $X$ que no posee subcubrimiento finito.
	\end{enumerate}

	$\impliedby$.
	Sea $X$ completo y con mínimo $m$ y máximo $M$.
	Si $m = M$, entonces $X = \{m\}$ que es compacto.
	Sino, fijemos un cubrimiento por abiertos $\mathcal{C}$ de $X$, y sea
	$$ S := \{ y \in X : [m, y] \text{ tiene subcubrimiento finito} \}; $$
	nótese que $S$ no es vacío pues $m \in S$ y está acotado así que admite un supremo $s$.
	\underline{Veamos que $s \in S$}, en efecto existe un entorno $U_s$ de $s$ en $\mathcal{C}$, de modo que $s \in (a, b) \subseteq U$.
	Luego $a < s$, es decir, $a \in S$ así que existe un subcubrimiento $\mathcal{C}_a$ finito de $[m, a]$ y $\mathcal{C}_a \cup \{U_s\}$ es
	un subcubrimiento finito de $[m, s]$.
	Similarmente sea $U_b$ un entorno de $b$ en $\mathcal{C}$, luego $\mathcal{C}_a \cup \{ U_s, U_b \}$ es un subcubrimiento finito
	de $[m, b]$, por lo que si $s \ne M$, entonces $b$ contradiciría la maximalidad de $s$, es decir, necesariamente $s = M$, lo que prueba que $X$
	admite un subcubrimiento finito y que es compacto.
\end{proof}

De ésto inmediatamente sigue una generalización del teorema de Heine-Borel:
\begin{thm}
	Un subespacio de un espacio ordenado completo es compacto syss es cerrado y acotado.
\end{thm}

\begin{exn}\label{ex:omega_1_ord_space}
	% \addnamedexample{$[0, \omega_1]$}{Un espacio que es completamente normal, pero no perfectamente normal.
	% Un espacio de Lindelöf que no es hereditariamente de Lindelöf}
	% \addnamedexample{$[0, \omega_1)$}{Un espacio que es 1AN pero no es ni 2AN, ni es separable.
	% Un espacio que es secuencialmente compacto, pero no es ni compacto ni de Lindelöf}
	Sea $X := [0, \omega_1]$ y $Y := [0, \omega_1)$.
	\begin{enumerate}
		\item Como $X$ e $Y$ son espacios ordenados, entonces son completamente normales, sin embargo,
			nótese que $\{ \omega_1 \}$ es un cerrado que no es un conjunto $G_\delta$ pues tendría que ser la intersección de numerables cerrados
			que podemos suponer de la forma $\{ [\alpha_n, \omega_1] \}_{n\in\N}$ donde $\lim_{n \to \omega_0} \alpha_n = \omega_1$,
			pero $\cf\omega_1 \ne \omega_0$ por lo que la sucesión $(\alpha_n)_{n\in\N}$ no existe;
			en consecuencia $X$ no es perfectamente normal.

		\item Por el mismo argumento, $X$ no es 1AN pues $\omega_1$ no posee una base de entornos numerable.
			Sin embargo, $Y$ sí es 1AN pues el único punto con una base de entornos no numerable de $X$ es $\omega_1$,
			como consecuencia, $X$ no es metrizable.
			$X$, $Y$ no son separables pues para toda sucesión numerable $(\alpha_n)_{n\in\N}$ de puntos siempre
			existe un intervalo $(\beta, \omega_1)$ que es un abierto que no contiene ningún $\alpha_n$.
			Una modificación de éste argumento también permite ver que ni $X$ ni $Y$ son 2AN.

		\item $X$ es compacto pues es orden-completo, por el mismo argumento $Y$ no es compacto.
			Sin embargo, tanto $X$ como $Y$ son secuencialmente compactos.
			Luego $Y$ no es metrizable (pues todo espacio métrico y secuencialmente compacto es compacto).
			Además, tanto $X$ como $Y$ son numerablemente compactos,
			pero como $Y$ no es compacto, entonces $Y$ no es de Lindelöf.
			Luego $X$ es de Lindelöf, pero no hereditariamente de Lindelöf.
	\end{enumerate}
\end{exn}

Éste es otro ejemplo que emplea la noción de cofinalidad:
\begin{exn}[tabla de Tychonoff]\label{ex:tychonoff_plank}
	\addnamedexample{Tabla de Tychonoff}{Un espacio que es normal, pero no completamente normal}
	Sea $T := [0, \omega_1] \times [0, \omega_0]$ con la topología usual, llamado la \strong{tabla de Tychonoff}
	y $T_\infty := T \setminus \{\omega_1, \omega_0\}$ con la topología subespacio, llamado la \strong{tabla reducida de Tychonoff}.
	\begin{itemize}
		\item $T$ es normal:
			Puesto que $[0, \omega_1]$ y $[0, \omega_0]$ son espacios de Hausdorff compactos,
			luego $T$ también es de Hausdorff compacto, por ende, es normal.

		\item $T$ no es completamente normal, dado que $T_\infty$ no es normal:
			Sean $A := \{\omega_1\} \times [0, \omega_0)$ y $B := [0, \omega_1) \times \{\omega_0\}$.
			Es fácil comprobar que $A, B$ son cerrados y claramente disjuntos; sin embargo veamos que no poseen entornos disjuntos.
			En efecto, sea $U$ un entorno de $A$.
			Entonces para todo $(\omega_1, n) \in U$ necesariamente existe $\alpha_n < \omega_1$ tal que
			$(\alpha_n, \omega_1] \times \{n\} \subseteq U$.
			Sea $\gamma := \lim_{n\to\omega} \alpha_n < \omega_1$ (puesto que $\omega_1$ es un ordinal regular),
			de modo que $(\gamma, \omega_1] \times [0, \omega_0) \subseteq U$.
			Pero claramente todo entorno de $(\gamma + 1, \omega_0) \in B$ corta a $U$, por la caracterización anterior,
			por lo que $T_\infty$ no es normal.
	\end{itemize}
\end{exn}

\begin{exn}[tabla de Alexandroff]\label{ex:alexandroff_plank}
	\addnamedexample{Tabla de Alexandroff}{Un espacio que es de Urysohn, pero no es ni regular ni de Tychonoff}
	Sea $X := [0, \omega_1] \times [-1, 1]$ y $p := (\omega_1, 0)$ con la topología cuya base son los abiertos en la topología producto usual
	junto a los conjuntos de la forma
	$$ U(\alpha, n) := \{p\} \cup (\alpha, \omega_1]\times(0, 1/n). $$
	Nótese que la topología en $X$ es, por definición, más fuerte que la topología producto, que es de Tychonoff, por lo que $X$ es al menos Urysohn.
	Pero no es regular, dado que $C := (0, \omega_1) \times \{0\}$ es un cerrado (nótese que su complemento es trivialmente un entorno de todo punto
	distinto de $p$, y para $p$ basta considerar que $p \in U(0, 1) \subseteq C^c$) y todo entorno de $C$ corta a todo entorno de $p$.
	Como $X$ no es regular, tampoco es de Tychonoff.
\end{exn}

\begin{exn}[sacacorchos de Tychonoff]\label{ex:tychonoff_corkscrew}
	\addnamedexample{Sacacorchos de Tychonoff}{Un espacio regular pero no de Urysohn}
	\addnamedexample{Sacacorchos reducido de Tychonoff}{Un espacio regular y de Urysohn que no es de Tychonoff}
	La construcción del espacio a continuación es complicada, así que recomendamos leerlo con detalle:
	En primer lugar sea $T$ la tabla reducida de Tychonoff, vale decir, el subespacio $[0, \omega_1] \times [0, \omega_0] \setminus \{(\omega_1, \omega_0)\}$.
	Luego definamos $P_k := T \times \{k\}$ e $Y := \bigcup_{k\in\Z} P_k$.
	Pictóricamente cada $P_i$ es el cuadrante del sacacorchos de Tychonoff, de modo que por definición diremos que
	$P_{4k}$ es el primer cuadrante, $P_{4k + 1}$ es el segundo, $P_{4k + 2}$ el tercero y $P_{4k + 3}$ el cuarto.
	De momento $Y$ posee únicamente la topología producto, pero ahora definiremos $X$ como un espacio cociente de $Y$ bajo la siguiente relación
	de equivalencia $\sim$ dada por que:
	\begin{itemize}
		\item Dos puntos que sean iguales.
		\item $(\Omega, n, 2k) \sim (\Omega, n, 2k+1)$ con $k \in \Z$ y $n \in \N$.
		\item $(\alpha, \omega, 2k+1) \sim (\alpha, \omega, 2k+2)$ con $k \in \Z$ y $0 \le \alpha < \Omega$.
			% \item $(\alpha, \omega, 4k+3) \sim (\alpha, \omega, 4k+4)$ con $k \in \Z$ y $0 \le \alpha < \Omega$.
	\end{itemize}
	Denotamos por $L_k := P_{4k} \cup P_{4k+1} \cup P_{4k+2} \cup P_{4k+3}$.
	A ésto le sumamos dos puntos $a_+$ y $a_-$ que son los que están <<al final arriba>> y <<al final abajo>> del sacacorchos,
	y tal que la base de entornos de $a_+$ es $\bigcup_{k \ge n} L_k$ para todo $n\in\Z$ y $\bigcup_{k \le n} L_k$ para $a_-$.
	Denotamos por $Z := X \setminus \{a_-\}$.
	% Primero definamos $A_\gamma := (-1, -2, \dots, \gamma, \dots, 2, 1)$ para todo ordinal $\gamma > 0$.
	% En particular sea $P := A_{\omega_1} \times A_{\omega_0}$ y sea $P^* := P \setminus \{(\omega_1, \omega_0)\}$.
	% Luego
	% ...
	\todo{Completar construcción, ver \cite[pp. 109--111]{steen:counterexamples}.}
\end{exn}

% \section{Grupos topológicos}
% \begin{mydef}[Grupos topológicos]\index{grupo!topológico}
%	 Se dice que un grupo $(G, \cdot)$ dotado de una topología es un \strong{grupo topológico} si las aplicaciones
%	 $$ \cdot : G\times G \to G, \quad ()^{-1} : G \to G $$
%	 son continuas.
%	 \par
%	 Un \strong{isomorfismo topológico}\index{isomorfismo!topológico} entre dos grupos topológicos es un isomorfismo de grupos que es también
%	 un homeomorfismo de espacios topológicos.
% \end{mydef}

% \textbf{Ejemplos.} Son grupos topológicos:
% \begin{enumerate}
%	 \item Cualquier grupo $G$ con la topología discreta, en cuyo caso se dice un \strong{grupo discreto}.
%	 \item La topología estándar en $(\R, +)$.
%	 \item La topología producto estándar en $(\R^n, +)$.
%		 Por ende, la topología estándar en $(\C, +)$.
%	 \item La topología subespacio en $(\R^\times, \cdot)$ y $(\C^\times, \cdot)$.
% \end{enumerate}

% \begin{prop}
%	 Un grupo con una topología es un grupo topológico syss la aplicación $(x, y) \mapsto xy^{-1}$ es continua.
% \end{prop}
% \begin{proof}
%	 $\implies$. Trivial.\\
%	 $\impliedby$. Claramente $x \mapsto (1, x)$ es continua (pues lo es en cada coordenada) de modo que
%	 $x\mapsto (1, x) \mapsto x^{-1}$ muestra ser continua.
%	 Por ello $(x, y) \mapsto (x, y^{-1}) \mapsto x(y^{-1})^{-1} = xy$ es continua.
% \end{proof}
% A las aplicaciones de la forma $x \mapsto ax$ (resp. $xa$) en un grupo topológico les decimos \strong{traslación izquierda (resp. derecha) por $a$}.
% Claramente las traslaciones son continuas.

% \begin{prop}
%	 $A$ es abierto en $G$ syss para todo $g\in G$ se cumple que $gA$ es abierto.
% \end{prop}

% \begin{thm}
%	 En un grupo topológico $G$ son equivalentes:
%	 \begin{enumerate}
%		 \item $G$ es de Hausdorff.
%		 \item $\{1\}$ es cerrado.
%		 \item Si $ \mathcal{B} $ es una base de entornos de 1, entonces $\{1\} = \bigcap \mathcal{B}$.
%	 \end{enumerate}
% \end{thm}
% \begin{proof}
%	 Claramente $(1)\implies(2)$ y si $\{1\}$ es cerrado, entonces con $f(x, y) := xy^{-1}$ vemos que $f^{-1}[ \{1\} ] = \{(x, x) : x\in G\}$ la diagonal
%	 es cerrada, es decir, $G$ es de Hausdorff.
%	 \par
%	 Nuevamente se sigue que $(1)\implies(3)$.
%	 Veamos que $(3) \implies (2)$: notemos que para todo $x\in G$ existe un entorno $U$ de 1, tal que $x^{-1} \notin U$, luego $1 \notin xU$ donde $xU$
%	 es un entorno de $x$, ergo $\{1\}$ es cerrado.
% \end{proof}

% \begin{prop}
%	 Se cumple:
%	 \begin{enumerate}
%		 \item Si $H \le G$ (es subgrupo), entonces $\overline H \le G$.
%		 \item Si $H \nsle G$ (es subgrupo normal), entonces $\overline H \nsle G$.
%	 \end{enumerate}
% \end{prop}

\end{document}
