\documentclass[teoria-numeros.tex]{subfiles}
\begin{document}

\chapter{Geometría de los números}
Éste capítulo pretende explorar varios tópicos en simultáneo.
En primera instancia y tal como señala el título, queremos introducir la técnica de la \textit{geometría de los números} de Minkowski,
la cual es hoy estándar en cualquier libro moderno de teoría de números.
Ésta disciplina ha tomado varias formas con el pasar del tiempo, pero nuestro enfoque pretende tocar también la introducción y utilización
de los adèles e idèles como diccionario local-global; ello con inspiración de \citeauthor{cassels67global}~\cite{cassels67global}.
Finalmente, una versión adélica del teorema de Minkowski también pavimenta el camino para presentar distintas mejoras al clásico lema de Siegel,
principalmente siguiendo a \citeauthor{bombieri:heights}~\cite{bombieri:heights}.

\section{Cuerpos localmente compactos y adèles}
Gran parte de la combinación entre cuerpos y topología ya la detallamos en el capítulo anterior,
pero aquí conviene revisar las aplicaciones de una topología pura y no solo de usos de las métricas.
\nocite{weil:basic}

Recuérdese lo siguiente:
\begin{thmi}\label{thm:exist_uniq_Haar}
	Sea $G$ un grupo (topológico) localmente compacto.
	Entonces posee una medida de Borel no nula que es invariante por la izquierda, vale decir,
	tal que para toda función $f \in C_c(G; \C)$ continua de soporte compacto
	y todo $y \in G$ tenemos:
	$$ \int_{G} f(y\cdot x) \, \ud\mu(x) = \int_{G} f(x) \, \ud \mu(x). $$
	Más aún, si $\nu$ satisface lo anterior, entonces $\nu = c\mu$ para algún $c \in \R_{>0}$.
\end{thmi}
\begin{mydef}
	Sea $G$ un grupo topológico.
	Una medida de Borel no nula invariante por la izquierda se dice una \strong{medida de Haar (izquierda)}\index{medida!de Haar}.
\end{mydef}

\begin{cor}
	Sea $G$ un grupo localmente compacto y $H \nsle G$ un subgrupo normal cerrado.
	Sean $\mu, \nu$ medidas de Haar sobre $G, H$ resp.
	Entonces existe una única medida de Haar $\lambda$ sobre el cociente $G/H$ tal que
	$$ \forall f\in C_c(G; \C) \quad \int_{G} f(x) \, \ud \mu(x) = \int_{G/H} \left( \int_{H} f(xy) \, \ud \nu(x) \right) \ud \lambda(\pi(y)). $$
\end{cor}
% \begin{proof}
%	 Para $f \in C_c(G; \C)$ sea
%	 \begin{align*}
%		 \Phi_f \colon G/H &\longrightarrow \C \\
%		 yH &\longmapsto \int_{H} f(xy) \, \ud \nu(x),
%	 \end{align*}
%	 la cual es una función continua y de soporte compacto.
%	 Como $G/H$ es localmente compacto, sea $\lambda$...

%	 Por tanto, por representación de Riesz existe una única medida de Borel $\mu$ sobre $G$ tal que
%	 $$ \int_{G} f(x) \, \ud \mu(x) = \int_{G/H}  \, \ud  $$
% \end{proof}
El corolario anterior debería interpretarse geométricamente como una generalización del teorema de Fubini.

En general, la mayoría de libros suele citar los resultados generales de medidas de Haar porque uno está obligado a reconstruirlas explícitamente
en los debidos contextos.

\newcommand{\Mod}{\operatorname{Mod}}
\begin{mydef}
	Sea $G$ un grupo localmente compacto y $\sigma \colon G \to G$ un automorfismo en $\mathsf{TopGrp}$.%
	\footnote{Vale decir, $\sigma$ es un isomorfismo de grupos y un homeomorfismo de espacios topológicos.}
	Definimos el \strong{módulo}\index{modulo!módulo (automorfismo)} de $\sigma$ como el cociente $\Mod_G\sigma := \mu( \sigma[X] )/\mu(X)$
	donde $\mu$ es una medida de Haar sobre $G$ y $X$ es un conjunto medible con $0 < \mu(X) < \infty$.
	Obviaremos el subíndice <<$G$>> de no haber ambigüedad.
\end{mydef}
Nótese que como las medidas de Haar son únicas salvo constante, el módulo es independiente de la elección de $\mu$.

\begin{prop}
	Sea $G$ un grupo localmente compacto, y sean $\sigma, \tau$ automorfismos de $G$.
	Entonces:
	\begin{enumerate}
		\item Para todo conjunto medible $S$ y toda función medible $f \in \mathscr{L}(G; \R)$ se tiene:
			\begin{align*}
				\mu( \sigma[S] ) &= \Mod\sigma \cdot \mu(S), \\
				\int_{S} f(\sigma^{-1}(x)) \, \ud\mu(x) &= (\Mod\sigma) \cdot \int_{\sigma[S]} f(x) \, \ud\mu(x).
			\end{align*}
		\item $\Mod(\sigma\circ\tau) = \Mod\sigma \cdot \Mod\tau$.
		\item Sea $H \nsle G$ un subgrupo normal cerrado y $\sigma$-invariante.
			Denotemos $\overline{\sigma} \in \Aut(G/H)$ el automorfismo inducido, entonces:
			$$ \Mod_G\sigma = \Mod_{G/H}(\overline{\sigma}) \cdot \Mod_H(\sigma|_H). $$
	\end{enumerate}
\end{prop}

Esto podemos emplearlo para el contexto de cuerpos topológicos:%
\footnote{Más generalmente, \citeauthor{weil:basic}~\cite{weil:basic} considera el caso de espacios vectoriales topológicos
sobre un anillo de división topológico. El lector interesado puede generalizar lo siguiente.}
\begin{mydef}
	Sea $K$ un cuerpo localmente compacto y sea $a \in K$.
	Se define $\Mod_K(a)$ como el módulo del automorfismo $x \mapsto ax$ (sobre el grupo $(K, +)$) si $a \ne 0$ y $\Mod_K(0) := 0$.
\end{mydef}
\begin{prop}
	Sea $K$ un cuerpo localmente compacto.
	Entonces\break $\Mod_K \colon K \to \R$ es una función continua y $\Mod_K(ab) = \Mod_K(a) \cdot \Mod_K(b)$.
\end{prop}
\begin{proof}
	Sea $\mu$ una medida de Haar (aditiva) sobre $K$ y $X$ un entorno compacto del 0.
	Para todo $a \in K$ y todo $\epsilon > 0$, existe un entorno abierto $U \supseteq a\cdot X$ tal que $\mu(U) \le \mu(aX) + \epsilon$.
	Sea $W$ un entorno (abierto) de $a$ tal que $WX \subseteq U$, entonces para todo $b \in W$ tenemos que
	$$ \Mod_K(b) \le \Mod_K(a) + \mu(X)^{-1}\epsilon. $$
	Empleando que $\Mod_K(a^{-1})^{-1} = \Mod_K(a)$ (¿por qué?), es fácil concluir una desigualdad recíproca similar y ver que es continua.

	La última afirmación se sigue de la proposición anterior.
\end{proof}

\begin{thm}
	Sea $K$ un cuerpo localmente compacto y no discreto.
	Para todo $\epsilon > 0$, el conjunto $\overline{B}_\epsilon := \{ a \in K : \Mod_K(a) \le \epsilon \}$ es compacto.
\end{thm}
\begin{proof}
	Sea $C$ un compacto con $0 \in \Int C$ y sea $U$ un entorno tal que $U\cdot C \subseteq C$.
	Como $K$ no es discreto, entonces existe $r \in U \cap C$ tal que $0 < \Mod_K(r) < 1$ y, por inducción, $r^n \in C$ para $n \in \N$.
	Como $C$ es compacto, sea $s$ un punto de acumulación de $\{ r^n \}_{n\in\N}$ y, aplicando $\Mod_K(-)$, vemos que $\Mod_K(s) = 0$,
	por lo que $s = 0$ (¿por qué?).

	Sea $a \in \overline{B}_\epsilon$. Como $r^n a \to 0$, sea $m$ el mínimo natural tal que $r^m a \in C$.
	Si $a \notin C$, entonces $r^{m-1} a \notin C$ y, por tanto, $r^m a \in C \setminus rC$.
	Sea $S := \overline{C \setminus rC}$, entonces $S \subseteq C$ es compacto y $0 \notin S$, así que $\delta := \inf\{ \Mod_K(x) : x \in S \} > 0$.
	Sea $N$ el máximo entero tal que $\Mod_K(r)^N \le \delta / \epsilon$, entonces como $a \in \overline{B}_\epsilon \setminus C$ tenemos que
	$$ \Mod_K(r)^N \epsilon \le \delta \le \Mod_K(r^m a) = \Mod_K(r)^m \Mod_K(a) \le \Mod_K(r)^m \epsilon. $$
	Por tanto, $N \ge m$.
	Esto prueba que $\overline{B}_\epsilon \subseteq \bigcup_{j=0}^{N} r^{-j} C$ y es fácil ver que $\overline{B}_\epsilon$ es cerrado,
	por lo que es compacto.
\end{proof}

Por el teorema anterior, el lector podría sospechar que $\Mod_K(-)$ se comporta como un valor absoluto, pero solo hasta cierto punto.
Es fácil probar que para $\alpha \in \C^\times$ tenemos que $\Mod_\C(\alpha) = |\alpha|^2$, la cual no es una función valor absoluto (¿por qué?).

\begin{cor}
	Sea $K$ un cuerpo localmente compacto y no discreto.
	Entonces:
	\begin{enumerate}
		\item La familia $\{ \overline{B}_r \}_{r > 0}$ es una base de entornos cerrados del 0.
		\item $K$ es un espacio 1AN y un grupo métrico (posee una métrica que induce su topología invariante bajo $+$).
		\item Se tiene que $\lim_n a^n = 0$ syss $\Mod_K(a) < 1$.
		\item Todo subcuerpo discreto de $K$ es finito.
	\end{enumerate}
\end{cor}
\begin{proof}
	\begin{enumerate}
		\item Sea $C$ un compacto con $0 \in \Int C$.
			Sea $\epsilon > \sup\{ \Mod_K(x) : x \in C \}$, de modo que $C \subseteq \overline{B}_\epsilon$.
			Defínase $S := \overline{\overline{B}_\epsilon \setminus C}$ y sea $\delta := \inf\{ \Mod_K(x) : x \in S \} \le \epsilon$.
			Finalmente, elíjase $0 < \gamma < \delta$ tal que $\overline{B}_\gamma \subseteq \overline{B}_\delta$ con $B_\gamma \cap S = \emptyset$;
			de modo que $B_\gamma \subseteq C$.

		\item Esto es una aplicanción del teorema de Birkhoff-Kakutani.
		\item Basta considerar la familia $\{ \overline{B}_{r^n} \}_{n \in \N}$ con $r := \Mod_K(a)$.
		\item Sea $L \subseteq K$ discreto.
			Para todo $a \in L^\times$, nótese que $\Mod_K(a) \le 1$, puesto que de lo contrario el conjunto $\{ a^{-n} \}_{n\in\N}$
			tiene un punto de acumulación (el 0).
			Así, $L$ es un subespacio discreto del compacto $\overline{B}_1$ y, por lo tanto, es finito. \qedhere
	\end{enumerate}
\end{proof}

\begin{thm}
	Sea $K$ un cuerpo localmente compacto y no discreto.
	Sea $V$ un $K$-espacio vectorial topológico y $U \le V$ un subespacio con base $\{ v_1, \dots, v_n \}$.
	Entonces la aplicación
	$$ (a_1, \dots, a_n) \longmapsto \sum_{i=1}^{n} a_i v_i $$
	es un isomorfismo y un homeomorfismo.
	Más aún, $U$ es localmente compacto (como espacio) y es cerrado en $V$.
\end{thm}
\begin{cor}
	Sea $K$ un cuerpo localmente compacto y no discreto.
	Entonces todo $K$-espacio vectorial de dimensión finita posee una única topología de modo que sea un $K$-espacio vectorial topológico.
\end{cor}
\begin{cor}
	Sea $K$ un cuerpo localmente compacto, y sea $V$ un $K$-espacio vectorial localmente compacto y no discreto.
	Entonces $V$ tiene dimensión finita $d$ y $\Mod_V(a) = \Mod_K(a)^d$.
\end{cor}

\begin{cor}
	Sea $K$ un cuerpo localmente compacto no discreto y sea $A \colon V \to V$ un endomorfismo sobre un espacio vectorial de dimensión finita.
	Entonces $\Mod_V(A) = \Mod_K(\det A)$.
\end{cor}
% \begin{hint}
%	 Siga la demostración para $K = \R$ mediante la descomposición de Gauss de matrices.
% \end{hint}
\warn
Esta demostración depende de la conmutatividad de $K$.

\begin{prop}
	Sea $K$ un cuerpo localmente compacto no discreto.
	La aplicación $\Mod_K \colon K^\times \to \R_{> 0}$ induce un homomorfismo abierto en un subgrupo cerrado de $(\R_{> 0}, \cdot)$.
\end{prop}
\begin{thm}
	Sea $K$ un cuerpo localmente compacto no discreto.
	Entonces existe una constante $C > 0$ tal que para todo $a, b \in K$:
	$$ \Mod_K(a + b) \le C \max\{ \Mod_K(a), \Mod_K(b) \}. $$
	Además, $C$ está dado por
	$$ C := \sup\{ \Mod_K(1 + a) : \Mod_K(a) \le 1 \}, $$
	y si $C \le 1$, entonces $\Gamma := \Img( \Mod_K|_{K^\times} ) \subseteq \R_{> 0}$ es discreto.
\end{thm}

\subsection{Adèles}
\begin{mydef}
	Sea $K$ un cuerpo global y sea $S \supseteq M_K^\infty$ un conjunto finito de lugares de $K$ que contenga los lugares infinitos.
	Se define el siguiente anillo:
	$$ K_\A(S) := \prod_{v \in S} K_v \times \prod_{v \notin S} \mathfrak{o}_v, $$
	con la suma y multiplicación coordenada a coordenada.
	Esta descripción hace que $K_\A(S)$ sea un anillo (topológico) localmente compacto.%
	\footnote{En efecto, cada $\mathfrak{o}_v$ es compacto, luego el producto lo es por el teorema de Tychonoff.
	Cada $K_v$ es localmente compacto y solo tomamos un producto finito de ellos.}
	De manera conjuntista, $K_\A(S)$ corresponde al conjunto de adèles (cfr. def.~\ref{def:adele}) $\vec a$
	tales que $|a|_v \le 1$ para todo $v \notin S$.

	Se define el anillo de adèles como:%
	\footnote{Esta es notación de \citeauthor{weil:basic}~\cite[59]{weil:basic} y \citeauthor{bombieri:heights}~\cite[604]{bombieri:heights}.
	La notación estándar es $\A_K$, pero optamos por $K_\A$ para evitar confusión con el espacio afín.}
	$$ K_\A := \bigcup_{\substack{M_K^\infty \subseteq S \subseteq M_K \\ |S| < \infty}} K_\A(S), $$
	con la topología en la cual cada $K_\A(S)$ es un subanillo abierto.
	% De manera conjuntista, $K_\A$ es el conjunto de todos los adèles.
\end{mydef}

Tenemos la inclusión natural $K \hookto K_\A$ dada por la función diagonal $a \mapsto (a)_{v \in M_K}$.
Esto vuelve a $K$ un subgrupo (aditivo) de $K_\A$
% \begin{lem}
%	 El subconjunto
%	 $$ \Omega := \prod_{v \mid \infty} K_v \times \prod_{v \nmid \infty} \mathfrak{o}_v \subseteq K_\A, $$
%	 es un dominio fundamental para $K_\A/K$, es decir, cada clase de equivalencia en $K_\A/K$ tiene exactamente un representante en $\Omega$.
% \end{lem}

\begin{thm}
	Sea $K$ un cuerpo global. Entonces:
	\begin{enumerate}
		\item El subconjunto $K \subseteq K_\A$ es un subgrupo (aditivo) cerrado discreto.
		\item Si $\car K = 0$, sea $\{ \omega_1, \dots, \omega_n \}$ una $\Z$-base de $\mathcal{O}_K$ y sea
			$$ \Omega_\infty := \left\{ \vec a \in \prod_{v\mid \infty} K_v \cong K \otimes_\Q \R :
			\vec a = \sum_{j=1}^{n} \omega_j \otimes r_j \quad r_j \in [0, 1) \right\}, $$
			(donde empleamos el teorema~\ref{thm:place_separable_ext}).

			Si $\car K \ne 0$, entonces $\Omega_\infty := \{ 0 \}$.
			El subconjunto
			$$ \Omega := \Omega_\infty \times \prod_{v \nmid \infty} \mathfrak{o}_v \subseteq K_\A(M_K^\infty), $$
			es un dominio fundamental para $K_\A/K$, es decir, cada clase de equivalencia en $K_\A/K$ tiene exactamente un representante en $\Omega$.
		\item En consecuencia, $K_\A/K$ es un grupo compacto.
	\end{enumerate}
\end{thm}
\begin{proof}
	\begin{enumerate}
		\item Empleando que las traslaciones son homeomorfismos, basta ver que el 0 está aislado.
			Elíjase un lugar $w \in M_K$, entonces 
			$$ U := \{ a \in K_w : |a|_w < 1 \} \times \prod_{v \ne w} \{ a \in K_v : |a|_v \le 1 \} $$
			es un entorno del 0 y, por la fórmula del producto, $K \cap U = \{ 0 \}$.
		\item Primero veamos la existencia de representantes.
			Sea $\vec a \in K_\A$, entonces el conjunto de lugares (finitos) $S$ tales que $|a|_v > 1$ es finito.
			Considerando el conjunto $\{ a_v : v \in S \}$, por el teorema de aproximación, tenemos que existe $b \in K$
			tal que $|a_v - b|_v < 1$ para todo $v \in S$.
			Reemplazando $\vec a$ con $\vec a - b$ podemos suponer que $\vec a \in K_\A(M_K^\infty)$.
			Defínase
			$$ (a_v)_{v\mid\infty} = \sum_{j=1}^{n} \omega_j \otimes r_j $$
			con $r_j \in \R$.
			Luego, existen $c_j \in \Z$ tales que $0 \le r_j - c_j < 1$ para cada $j$, de modo que $\vec a - \sum_{j=1}^{n} \omega_j \otimes c_j$
			es un representante de $\vec a$ en $\Omega$.

			Sean $\vec a, \vec b \in \Omega \subseteq K_\A(M_K^\infty)$ tales que $\vec c := \vec a - \vec b \in K$.
			Denotemos $(a_v)_{v\mid\infty} = \sum_{j=1}^{n} \omega_j \otimes a_j, (b_v)_{v\mid\infty} = \sum_{j=1}^{n} \omega_j \otimes b_j$.
			Como $\vec c \in K_\A(M_K^\infty)$, entonces $c_v \in \mathfrak{o}_v$ para todo lugar finito, de modo que $c \in \mathcal{O}_K$ y,
			por lo tanto,
			$$ c = \sum_{j=1}^{n} \omega_j \otimes (a_j - b_j) $$
			satisface que $a_j - b_j \in (-1, 1)$ sea entero, por tanto $a_j = b_j$. \qedhere
	\end{enumerate}
\end{proof}

\begin{mydef}
	Sea $K$ un cuerpo numérico y sea $v \in M_K$.
	Definimos la normalización $\beta_v$ de la medida de Haar sobre $K_v$:
	\begin{enumerate}[(a)]
		\item Si $K_v = \R$, entonces $\beta_v$ es la medida de Lebesgue usual.
		\item Si $K_v = \C$, entonces $\beta_v$ es el doble de la medida de Lebesgue usual.
		\item Si $K_v \supseteq \Q_p$, entonces
			$$ \beta_v(\mathfrak{o}_v) = |\disc(K_v/\Q_p)|_p^{1/2}. $$
	\end{enumerate}
\end{mydef}
Sobre $K_\A(S)$ definimos:
$$ \beta_S := \prod_{v \in S} \beta_v \times \prod_{v \notin S} \beta_v|_{\mathfrak{o}_v}. $$
Los cuales son compatibles entre sí y se pegan en una medida de Haar $\beta$ sobre el anillo de adèles $K_\A$.

\begin{prop}
	Sea $K$ un cuerpo numérico.
	La medida de Haar sobre $K_\A/K$ satisface que $\beta_{K_\A/K}(K_\A/K) = 1$.
\end{prop}
\begin{proof}
	Como $\Omega$ (definido como antes) es un dominio fundamental, basta verificar que $\beta(\Omega) = 1$.
	Por definición:
	$$ \beta(\Omega) := \left(\prod_{v\mid\infty} \beta_v\right)(\Omega_\infty) \cdot \prod_{p} \prod_{v\mid p} |\disc(K_v/\Q_p)|_p^{1/2}. $$
	Nótese que para un lugar $u$ sobre $\Q$ tenemos que
	$$ |\disc(K/\Q)|_u := \prod_{v \mid u} |\disc(K_v/\Q_u)|_v, $$
	y por la fórmula del producto, concluimos que
	$$ \prod_{p} \prod_{v \mid p} |\disc(K_v/\Q_p)|_p^{1/2} = |\disc(K/\Q)|^{-1/2}. $$

	Ahora estudiemos la parte arquimediana.
	En primer lugar, nótense que hay $r + 2s$ encajes complejos $\sigma \colon K \to \C$,
	donde $r$ son encajes reales y $2s$ son encajes imaginarios (vienen de a pares por la conjugación compleja).
	Denotando $\overline{()}$ para la conjugación compleja, ordenemos los monomorfismos así:
	$$ \sigma_1, \dots, \sigma_r, \sigma_{r+1}, \dots, \sigma_{r+s}, \sigma_{r+s+1} := \overline{\sigma_{r+s}}, \dots, \sigma_{r+2s} = \overline{\sigma_{r+s}} $$
	donde los $\sigma_i$'s son los reales para $1 \le i \le r$.

	Sea $d := [K : \Q]$, entonces $K \otimes_\Q \R \cong \prod_{v \mid \infty} K_v \cong \R^d$ y $d = r + 2s$.
	Para $\alpha \in K$ definimos el siguiente vector
	$$ \vec v(\alpha) := ( \sigma_1(\alpha), \dots, \sigma_r(\alpha),
	\Re\sigma_{r+1}(\alpha), \Im\sigma_{r+1}(\alpha), \dots, \Re\sigma_{r+s}(\alpha), \Im\sigma_{r+s}(\alpha)). $$
	Sea $\omega_1, \dots, \omega_d$ una $\Z$-base de $\mathcal{O}_K$; entonces la medida de Lebesgue de $\Omega_\infty$ en $\R^d$ está dada por
	$$ \big| \det[\vec v(\omega_1), \dots, \vec v(\omega_n)] \big| = 2^{-s} \big| \det[\sigma_i(\omega_j)]_{ij} \big|, $$
	donde en el lado derecho se emplean todos los encajes complejos.
	Nótese que el $2^{-s}$ sale del hecho de la matriz de la derecha tiene entradas del tipo $\sigma_{r+s}(\omega) = \Re\sigma_{r+s}(\omega) +
	\Im\sigma_{r+s}(\omega)$ y del tipo $\overline{\sigma}_{r+s}(\omega) = \Re\sigma_{r+s}(\omega) - \Im\sigma_{r+s}(\omega)$;
	al sumar ambas filas obtendremos $2\Re\sigma_{r+s}\omega, \Im\sigma_{r+s}\omega$ por lo que sale un <<2>> por cada encaje imaginario.

	Finalmente, basta notar que $\big| \det[\sigma_i(\omega_j)]_{ij} \big| = \sqrt{| \disc(K/\Q) |}$ para concluir, ya que el factor $2^{-s}$
	se cancela con nuestras normalizaciones de $\beta_\C$.
	% \todo{Concluir demostración (cfr. \cite[606]{bombieri:heights}).}
\end{proof}
\begin{cor}\label{thm:adele_principal_descomposition}
	Sea $K$ un cuerpo global.
	Existen reales $\delta_v > 0$ para cada lugar $v \in M_K$ con $\delta_v = 1$ para todos salvo finitos $v$'s,
	de modo que el subconjunto $W \subseteq K_\A$ de adèles $\vec\gamma \in W$ que satisfacen $|\gamma_v|_v \le \delta_v$ posee la siguiente propiedad:
	todo adèle $\vec\alpha \in K_\A$ puede descomponerse como
	$$ \vec\alpha = \beta + \vec\gamma, \qquad \beta \in K, \vec\gamma \in W. $$
\end{cor}
\begin{proof}
	Esto se sigue de que el cociente $K_\A/K$ tenga medida de Haar finita, así que el $W$ del enunciado no es más que un conjunto de representantes
	del cociente.
\end{proof}

\section{Los teoremas de Minkowski}
\begin{lem}
	Sea $K$ un cuerpo numérico y $v \in M_K$ un lugar finito.
	Fijado un natural $n$ denótese $E_v := K_v^n$.
	Para un $\mathfrak{o}_v$-submódulo $\Lambda$ de $E_v$ son equivalentes:
	\begin{enumerate}
		\item $\Lambda$ es abierto y compacto dentro de $E_v$.
		\item $\Lambda$ es un $\mathfrak{o}_v$-submódulo finitamente generado y $K_v \langle \Lambda \rangle = E_v$.
	\end{enumerate}
\end{lem}

\begin{lem}
	Sea $\Lambda \le \R^n$ un subgrupo (aditivo).
	Son equivalentes:
	\begin{enumerate}
		\item $\Lambda$ es discreto y $\R^n/\Lambda$ es compacto (como grupo topológico).
		\item $\Lambda$ es discreto y $\R\langle \Lambda \rangle = \R^n$.
		\item $\Lambda$ es un $\Z$-módulo libre y posee una $\Z$-base que es también una $\R$-base de $\R^n$.
	\end{enumerate}
\end{lem}
\begin{mydef}
	Sea $K$ un cuerpo numérico y $v \in M_K$ un lugar (sin especificar).
	Fijado un natural $n$, denótese $E_v := K_v^n$.
	Un subgrupo $\Lambda \le E_v$ se dice un \strong{$K_v$-reticulado}\index{reticulado} si:
	\begin{enumerate}[(a)]
		\item Cuando $v$ es un lugar finito, $\Lambda$ es abierto y compacto dentro de $E_v$.
		\item Cuando $v$ es un lugar al infinito, $\Lambda$ es discreto y $E_v/\Lambda$ es compacto.
	\end{enumerate}

	Sea $A$ un dominio íntegro con $K := \Frac A$.
	Un \strong{$A$-reticulado} dentro de $K^n$ es un $A$-submódulo libre $\Lambda \le K^n$ tal que $K\langle \Lambda \rangle = K^n$.
\end{mydef}

\begin{prop}
	Sea $K$ un cuerpo numérico y fijemos un $K$-espacio vectorial $E := K^n$.
	Se cumplen:
	\begin{enumerate}
		\item Si $\Lambda$ es un $\mathcal{O}_K$-reticulado en $E$,
			entonces $\Lambda_v := \overline{\Lambda} \subseteq E_v$ es un $K_v$-reticulado para todo lugar finito $v$ y
			además $\Lambda_v = \mathfrak{o}_v^n$ solo para finitos lugares $v$.
		\item Si $\{ \Lambda_v \}_{v\in M_K^0}$ es una familia de $K_v$-reticulados en $E_v$ tales que $\Lambda_v = \mathfrak{o}_v^n$
			solo para finitos lugares; entonces existe un único $\mathcal{O}_K$-reticulado $\Lambda$ tal que
			$\overline{\Lambda} = \Lambda_v \subseteq E_v$.
			Además, dicho $\Lambda$ viene dado por:
			$$ \Lambda = \bigcap_{v\in M_K \setminus M_K^\infty} (\Lambda_v \cap E). $$
	\end{enumerate}
\end{prop}

Veamos una especie de resultado inverso.
Si consideramos los lugares infinitos en vez de los finitos, notamos primero que $M_K^\infty$ es un conjunto finito,
por lo que $E_\infty := \prod_{v\in M_K^\infty} E_v$ es un $\R$-espacio vectorial de dimensión finita.
\begin{prop}
	La imagen $\Lambda_\infty$ de un $K$-reticulado $\Lambda$ bajo el encaje diagonal $E \hookto E_\infty$ es un $\R$-reticulado.
\end{prop}

\begin{mydef}
	Sea $K$ un cuerpo numérico y $\vec x \in K_\A$ un adèle.
	Dado $\lambda \in \R$ se denota
	\[
		(\lambda\vec x)_v :=
		\begin{cases}
			\lambda x_v, & v\mid\infty, \\
			\phantom{\lambda}x_v, & v\nmid\infty.
		\end{cases}
	\]
	Dada una tupla de adèles $\vec y := ({\vec y}^1, \dots, {\vec y}^N) \in K_\A^N$, se denota $\lambda\vec y := (\lambda{\vec y}^1, \dots, \lambda{\vec y}^N)$.
	Para un lugar $v \mid \infty$, un subconjunto $S_v \subseteq E_v$ se dice \strong{simétrico}\index{simétrico (conjunto)} si $-S_v = S_v$.

	Para cada lugar arquimediano $v \mid \infty$, sean $S_v \subseteq E_v$ abiertos no vacíos, acotados, convexos, simétricos y
	sea $\Lambda$ un $\mathcal{O}_K$-reticulado en $E$.
	Se denota
	$$ S_\A^\Lambda := \prod_{v \mid \infty} S_v \times \prod_{v \nmid \infty} \Lambda_v \subseteq E_\A. $$
	Para $1 \le n \le N$ se define el \strong{$n$-ésimo mínimo sucesivo}\index{minimo@mínimo!sucesivo} como
	$$ \lambda_n := \inf\{ t > 0 : tS \text{ contiene $n$ vectores $K$-linealmente independientes de } \Lambda \}. $$
	Se denota por $\mu_n$ al supremo de los $\mu \ge 0$ tales que para todo $\vec x, \vec y \in \mu S_\A^\Lambda$ con $\vec x - \vec y \in E$,
	las últimas $N - n + 1$ coordenadas de $\vec x$ e $\vec y$ coinciden.
\end{mydef}
Es claro que
$$ 0 < \lambda_1 \le \cdots \le \lambda_N < \infty, \qquad 0 \le \mu_1 \le \cdots \le \mu_N < \infty. $$

\newcommand{\Vol}{\operatorname{Vol}}
\begin{lem}
	Sea $K$ un cuerpo numérico y $N \ge 1$.
	Considere el homomorfismo
	\begin{align*}
		\Phi_n \colon E_\A = K_\A^N &\longrightarrow (K_\A/K)^n \times K_\A^{N-n} \\
		\vec x &\longmapsto (\overline{\vec x}_1, \dots, \overline{\vec x}_n, \vec x_{n+1}, \dots, \vec x_N).
	\end{align*}
	Para cada $v \in M_K^\infty$, sean $T_v \subseteq E_v$ abiertos no vacíos, acotados, convexos y simétricos,
	sea $\Lambda$ un $\mathcal{O}_K$-reticulado en $E$ y sea $T := T_\A^\Lambda$.
	Para todo $\mu \ge 1$ real, tenemos que
	\[
		\Vol( \Phi_n[\mu T] ) \ge \mu^{d(N - n)} \Vol( \Phi_n[T] ).
	\]
\end{lem}
Aquí, $\Vol$ es cualquier medida de Haar sobre el codominio $(K_\A/K)^n \times K_\A^{N-n}$.
El teorema también podría reformularse adecuadamente en términos del módulo de la función $\vec x \mapsto \Phi_n(\mu\vec x)$.
% Hay una elección canónica construida a inicios de ésta sección que puede considerar
\begin{proof}
	Supongamos primero que $n = N$.
	Como $\Phi_n$ es un homomorfismo y las medidas de Haar son invariantes bajo traslaciones,
	para cada $\vec y$ podemos suponer que $T - \vec y$ contiene al origen y, por convexidad, $\mu(T - \vec y) \supseteq T - \vec y$.
	Así que $\Vol( \Phi_n[\mu T] ) \ge \Vol( \Phi_n[T] )$.

	Supongamos que $n < N$, e identifiquemos $K_\A^N = K_\A^n \times K_\A^{N-n}$.
	Dado $\vec y \in K_\A^{N-n}$ denotése
	$$ T(\vec y) := \{ \vec w \in K_\A^n : (\vec w, \vec y) \in T \}. $$
	Denotando por $\overline{\vec w}, \vec y$ las variables en $(K_\A/K)^n$ y en $K_\A^{N-n}$ resp., por el teorema de Fubini tenemos que
	$$ \Vol(\Phi_n[\mu T]) = \int_{K_\A^{N-n}} \ud\vec y \int_{\Phi_n[(\mu T)(\vec y)]} \ud\overline{\vec w}. $$
	Con el cambio de variables $\vec y = \mu\vec z$ obtenemos que
	$$ \Vol(\Phi_n[\mu T]) = \mu^{d(N-n)}\int_{K_\A^{N-n}} \Vol\big( \Phi_n[(\mu T)(\mu\vec z)] \big) \, \ud\vec z, $$
	pero $(\mu T)(\mu\vec z) = \mu\cdot T(\vec z)$; así que, aplicando el caso $N = n$, obtenemos que
	$$ \Vol\big( \Phi_n[(\mu T)(\mu\vec z)] \big) \ge \Vol( \Phi_n[T(\vec z)] ). $$
	Así concluimos pues
	\begin{equation}
		\Vol(\Phi_n[\mu T]) \ge \mu^{d(N - n)} \int_{K_\A^{N-n}} \Vol( \Phi_n[T(\vec z)] ) \, \ud\vec z = \mu^{d(N-n)} \Vol( \Phi_n[T] ).
		\tqedhere
	\end{equation}
\end{proof}

\begin{thm}[Davenport-Estermann]
	Sea $K$ un cuerpo numérico de grado $d$ y $N \ge 1$.
	Para cada $v \in M_K^\infty$, sean $T_v \subseteq E_v$ abiertos no vacíos, acotados, convexos y simétricos,
	y sea $\Lambda$ un $\mathcal{O}_K$-reticulado en $E$.
	Entonces
	$$ (\mu_1 \cdots \mu_N)^d \Vol(T_\A^\Lambda) \le 1. $$
\end{thm}
Aquí <<$\Vol$>> es realmente la medida producto de $\beta$ sobre el espacio de adèles $K_\A$.
\begin{proof}
	Queremos aplicar el lema anterior de manera inductiva, para lo que denotamos por
	\begin{align*}
		\Phi^N_n \colon K_\A^N &\longrightarrow (K_\A/K)^n \times K_\A^{N-n}, \\
		\Psi_n := \Id_{(K_\A/K)^n} \times \Phi^{N-n}_1 \colon (K_\A/K)^n \times K_\A^{N-n} &\longrightarrow (K_\A/K)^{n+1} \times K_\A^{N-n-1}.
	\end{align*}
	De modo que $\Phi^N_{n+1} = \Phi_n^N \circ \Psi_n$, abreviaremos $\Phi^N_n$ por $\Phi_n$.

	Veamos que $\Psi_n$ es inyectiva en $\Phi_n[\mu_{n+1}T]$.
	Sean $\vec x, \vec y \in \mu_{n+1}T$ tales que $\Phi_{n+1}(\vec x) = \Phi_{n+1}(\vec y)$.
	Esto significa que $\vec x_j = \vec y_j$ para todo $j > n+1$ y que $\vec x_j - \vec y_j \in K$ para todo $j \le n+1$;
	en particular, $\vec x - \vec y \in K^N$.
	Así, por definición de $\mu_{n+1}$ concluimos que $\vec x_{n+1} = \vec y_{n+1}$, vale decir, que $\Phi_n(\vec x) = \Phi_n(\vec y_n)$.

	Ahora, empleando las medidas de Haar normalizadas y la unicidad del teorema~\ref{thm:exist_uniq_Haar} concluimos que
	$$ \Vol( \Phi_n[\mu_{n+1}T] ) = \Vol( \Phi_{n+1}[\mu_{n+1}T] ), $$
	y, aplicando el lema anterior con $T^\prime = \mu_nT$ y $\mu' := \mu_{n+1}/\mu_n$, obtenemos que
	\begin{align*}
		\Vol( \Phi_N[\mu_N T] ) &= \Vol\left( \Phi_{N-1}\left[ \frac{\mu_N}{\mu_{N-1}} \mu_{N-1} T \right] \right) \ge
		\left( \frac{ \mu_N}{\mu_{N-1}} \right)^d \Vol( \Phi_{N-1}[ \mu_{N-1}T ] ) \\
					&\ge \prod_{n=1}^{N-1} \left( \frac{\mu_{n+1}}{ \mu_n} \right)^{d(N-n)} \Vol( \Phi_1[ \mu_1T ] )
					\ge (\mu_1 \cdots \mu_N)^d \Vol(T),
	\end{align*}
	donde en el último paso empleamos el lema como $\Vol( \Phi_0[ \mu_1T ] ) \ge \mu_1^{dN} \Vol(T)$.
	Finalmente, notemos que $\Phi_n$ llega al cociente $(K_\A/K)^N$ que es compacto, por lo que su medida de Haar total normalizada es 1.
\end{proof}

\begin{thm}[Minkowski-McFeat]\index{teorema!de Minkowski-McFeat}
	Sea $K$ un cuerpo numérico de grado $d$ y $N \ge 1$.
	Para cada $v \in M_K^\infty$, sean $S_v \subseteq E_v$ abiertos no vacíos, acotados, convexos y simétricos,
	y sea $\Lambda$ un $\mathcal{O}_K$-reticulado en $E$.
	Denotando $S := \prod_{v\mid\infty} S_v \times \prod_{v\nmid\infty} \Lambda_v \subseteq E_\A$ se tiene
	$$ (\lambda_1 \cdots \lambda_N)^d \Vol(S) \le 2^{dN}. $$
	Más aún, si para cada lugar imaginario $v \in M_K^\infty$ se cumple que $S_v$ es $\C$-simétrico (i.e.\ para todo $|\zeta| = 1$
	se satisface que $\zeta\cdot S_v = S_v$), entonces tenemos la siguiente cota:
	\begin{equation}
		\frac{2^{dN} \pi^{sN}}{(N!)^r \big( (2N)! \big)^s} |\disc(K/\Q)|^{-N/2} \le ( \lambda_1 \cdots \lambda_N )^d \Vol S,
		\label{eqn:strong_minkowski}
	\end{equation}
	donde $r, s$ denotan la cantidad de lugares reales e imaginarios de $K$ resp.
\end{thm}
\begin{proof}
	Sea $\gamma \in \GL_N(K_\A)$ una matriz invertible que determina un automorfismo lineal de $E_\A$.
	Se puede verificar que
	$$ \prod_{v\in M_K} \|\det \gamma\|_v = 1, $$
	de modo que no afecta el volumen de $S$ y, así, suponer que $\lambda_n S$ contiene a la base canónica $\vec e_1, \dots, \vec e_n$ de $E_\A$.

	Aplicando el teorema de Davenport-Estermann basta probar que $\mu_n \ge \frac{1}{2} \lambda_n$.
	Procedemos por inducción sobre $n$.
	Sean $\vec x, \vec y \in \frac{1}{2} \lambda_n S$ tales que $\vec x - \vec y \in E$ y, como $S$ es convexo y simétrico, entonces
	$$ \vec x - \vec y = \frac{1}{2}(2\vec x) + \frac{1}{2}(-2\vec y) \in \lambda_n S. $$
	Si $n = 1$ y $\lambda > \lambda_1$, entonces $\lambda S \cap E$ contiene a $\vec e_1$, es discreto y relativamente compacto,
	así que $\overline{ \lambda_1 S} \subseteq E_\A$ contiene a $\vec e_1$, de modo que $\lambda_1 S$ no y $\vec x = \vec y$,
	comprobando que $\mu_1 \ge \frac{1}{2} \lambda_1$.
	Si $n \ge 2$ entonces, por hipótesis inductiva, podemos suponer que $\lambda_n > \lambda_{n-1}$.
	Así $\overline{\lambda_{n-1}S} \subseteq \lambda_n S$ y, como $\vec e_1, \dots, \vec e_{n-1}, \vec x - \vec y \in \lambda_n S$ y,
	por tanto, $\vec x - \vec y$ deben ser combinación lineal de $\vec e_1, \dots, \vec e_{n-1}$; comprobando así que $\vec x_j = \vec y_j$ para $j \ge n$.

	Probaremos ahora el <<más aún>>.
	Para cada lugar al infinito $v \in M_K^\infty$ defínase
	$$ S_v^\prime := \left\{ \vec t \in E_v : \sum_{j=1}^{N} \lambda_j |t_j|_v < 1 \right\}. $$
	Así, por simetría (incluída la $\C$-simetría cuando $v$ es imaginario) concluimos que $S_v^\prime \subseteq S_v$.
	Sea
	\[
		S' := \prod_{v \in M_K^\infty} S_v^\prime \times \prod_{v \in M_K^0} \mathfrak{o}_v^N.
	\]
	Es claro que $S' \subseteq S$, por lo que $\Vol(S') \subseteq \Vol S$; finalmente solo calculamos las medidas de Haar:
	\begin{equation*}
		\beta_v^N( S' ) =
		\begin{cases}
			\dfrac{2^N}{N!} (\lambda_1 \cdots \lambda_N)^{-1}, & v \text{ es un lugar real}, \\
			\dfrac{(4\pi)^N}{(2N)!} (\lambda_1 \cdots \lambda_N)^{-2}, & v \text{ es un lugar imaginario}, \\
			|\disc(K_v/\Q_p)|^{N/2}_p, & v \in M_K^0.
		\end{cases}
		% \label{eqn:}
	\end{equation*}
	Donde los dos primeros cálculos son ejercicios para el lector (calcular volumenes de esferas y cubos salvo transformación lineal)
	y el último es por definición de (la normalización) $\beta_v$.
	También, recuérdese que $r + 2s = d$.
\end{proof}
% Naturalmente los nombres <<primer>> y <<segundo>> teorema de Minkowski se corresponden a una tradición,
% y a la nomenclatura general en libros de geometría de los números.
% Paradójicamente demostramos el segundo teorema antes, siguiendo a \citeauthor{bombieri:heights}~\cite{bombieri:heights}, \S C.2.

Aplicando el teorema anterior con $K = \Q$, donde $M_K^\infty$ solo consiste del lugar arquimediano usual <<$\infty$>> obtenemos:
\begin{thmi}[Teoremas de Minkowski]\index{teorema!de Minkowski}
	Sea $N \ge 1$ un natural.
	Sea $S_\infty \subseteq \R^N$ un abierto no vacío, acotado, convexo y simétrico, y sea $\Lambda_\infty \subseteq \R^N$ un $\R$-reticulado
	(i.e.\ un subgrupo aditivo discreto generado por una $\R$-base).
	\begin{description}
		\item[\thmstyle Primero]\index{teorema!de Minkowski!primero} Supongamos que
			$$ \Vol(S_\infty) > 2^N \Vol(\Lambda_\infty), $$
			entonces $S_\infty$ contiene un punto de $\Lambda_\infty$ distinto del origen.
			Aquí $\Vol(\Lambda_\infty)$ denota la medida de Lebesgue de un dominio fundamental de $\R^N$
			respecto a la acción por traslación de $\Lambda_\infty$.\footnotemark
		\item[\thmstyle Segundo] Definiendo $\lambda_n^\prime$ con $1 \le n \le N$ como el ínfimo $t > 0$ tal que $tS_\infty$
			posee $n$ vectores $\R$-linealmente independientes en $\Lambda_\infty$, obtenemos que
			$$ \lambda_1^\prime \cdots \lambda_N^\prime \Vol(S_\infty) \le 2^N \Vol(\Lambda_\infty). $$
	\end{description}
\end{thmi}
\footnotetext{%
	Esta es la notación de \citeauthor{bombieri:heights}~\cite[615]{bombieri:heights},
	mientras que \citeauthor{clark:geometry_num}~\cite[7]{clark:geometry_num} emplea \textit{covolúmen}.
}
% El primer teorema de Minkowski, pese a su sencilleza, posee una infinitud de aplicaciones.

% \subsection{Más geometría de números clásica}
% El <<caso clásico>> denota, en este libro, el caso $K = \Q$ con $M_K^\infty = \{ \infty \}$.
% Además de los dos teoremas de Minkowski ya demostrados, uno puede conseguir una serie de resultados de convexidad más finos.
% Seguimos a \citeauthor{schmidt80diophantine}~\cite[80\psqq]{schmidt80diophantine}, Ch.~IV.

% \begin{mydef}
%	 En un espacio euclídeo $\R^n$ entendemos por un \strong{elipsoíde}\index{elipsoíde} a la imagen de una bola cerrada mediante
%	 una transformación ($\R$-)lineal.
% \end{mydef}
% \begin{thm}[Jordan]
%	 Sea $K \subseteq \R^n$ un cuerpo convexo simétrico.
%	 Existe un elipsoíde $E$ tal que $E \subseteq K \subseteq \sqrt{n} E$.
% \end{thm}
% % \begin{thm}[de Blichfeldt]
% %	 Sea $P$ el punto medio de una bola abierta $K \subseteq \R^n$, sea $\Lambda \subseteq \R^n$ un reticulado y supongamos que
% %	 $$ \Vol(K) \ge \frac{n + 2}{2} \cdot 2^{n/2} \cdot \Vol(\Lambda). $$
% %	 Entonces $K$ contiene 
% % \end{thm}

\section{Aplicaciones}
\subsection{Finitud del grupo de clases y el grupo de $S$-unidades}
\begin{lem}\label{thm:adelic_constant}
	Sea $K$ un cuerpo global, existe una constante real $C > 0$ con la siguiente propiedad:
	si $\vec a \in K_\A$ es un adèle tal que $\prod_{v\in M_k} |a_v|_v > C$,
	entonces existe un adèle principal $\beta \in K \subseteq K_\A$ no nulo tal que
	\[
		\forall v\in M_k \qquad |\beta|_v \le |a_v|_v.
	\]
\end{lem}
\begin{proof}
	Sea $c \ge 0$ la medida de Haar (en $K_\A$) del conjunto de adèles $\vec\gamma \in K_\A$ tales que
	\[
		\begin{cases}
			|\gamma_v|_v \le 1/2, & v \mid \infty, \\
			|\gamma_v|_v \le 1, & v \nmid \infty.
		\end{cases}
	\]
	Esta constante $c$ satisface que $0 < c < \infty$ (pues solo hay finitos lugares arquimedianos), entonces veamos que $C := 1/c$ sirve.

	El conjunto $T$ de adèles $\vec\gamma \in K_\A$ tales que
	\[
		\begin{cases}
			|\gamma_v|_v \le \tfrac{1}{2}|\alpha_v|_v, & v \mid \infty, \\
			|\gamma_v|_v \le |\alpha_v|_v, & v \nmid \infty,
		\end{cases}
	\]
	tiene medida de Haar $c \prod_{v\in M_K} |\alpha_v|_v > c\cdot C = 1$, de modo que al pasar al cociente $K_\A \to K_\A / K$
	hay al menos dos puntos $\vec\gamma, \vec\delta \in T$ con la misma imagen (pues la medida de Haar del cociente es 1), por lo que
	$\beta := \vec\gamma - \vec\delta \in K \subseteq K_\A$ satisface
	\begin{equation}
		|\beta|_v = |\gamma_v - \delta_v|_v \le |\alpha_v|_v.
		\tqedhere
	\end{equation}
\end{proof}
% Dicho sea de paso, la constante en el lema anterior puede hacerse explícita.

\begin{cor}
	Sea $K$ un cuerpo global y fijemos un lugar $v_0 \in M_K$.
	Sean $\delta_v > 0$ reales para cada $v \ne v_0$, donde $\delta_v = 1$ para todos salvo finitos $v$'s.
	Entonces existe $\beta \in K^\times$ tal que
	$$ \forall v\ne v_0 \qquad |\beta_v| \le \delta_v. $$
\end{cor}

\begin{thm}[de aproximación fuerte]\index{teorema!de aproximación!fuerte (adèles)}
	Sea $K$ un cuerpo global y fijemos un lugar $v_0 \in M_K$.
	Sea $\mathcal{A}$ la proyección del anillo de adèles $K_\A$ que borra la coordenada $v_0$.
	Entonces la imagen de $K$ es densa en $\mathcal{A}$.
\end{thm}
Otra manera de leer éste resultado es que dado un adèle cualquiera, éste se puede aproximar tanto como se quiera por un adèle principal
salvo por una coordenada.
\begin{proof}
	El enunciado equivale al siguiente:
	sea $S \subseteq M_K$ un conjunto finito de lugares con $v_0 \notin S$, sean $\epsilon > 0$ y $\alpha_v \in K_v$ para cada $v \in S$;
	entonces existe un adèle principal $\beta \in K$ tal que
	\[
		\forall v\in S, w \notin S \cup \{ v_0 \}, \qquad |\alpha_v - \beta_v|_v \le \epsilon, \quad |\beta_w|_w \le 1.
	\]
	Ahora bien, por el corolario~\ref{thm:adele_principal_descomposition} existen $\delta_v$'s y un conjunto $W \subseteq K_\A$ de los adèles
	con $|\theta_v|_v \le \delta_v$, de modo que todo adèle $\vec\varphi \in K_\A$ se descompone
	\begin{equation}
		\vec\varphi = \vec\theta + \gamma, \qquad \vec\theta \in W, \gamma \in K.
		\label{eqn:approximating_adele_desc}
	\end{equation}
	Ahora bien, por el corolario anterior existe $\lambda \in K^\times$ tal que
	\[
		\forall v\in S, w \notin S \cup \{ v_0 \}, \qquad |\lambda|_v \le \delta_v^{-1}\epsilon, \quad |\lambda|_w \le \delta_w^{-1}.
	\]
	Así, escogiendo $\vec\varphi := \lambda^{-1}\vec\alpha$ y multiplicando la igualdad \eqref{eqn:approximating_adele_desc} por $\lambda$ obtenemos que
	$$ \vec\alpha = \vec\psi + \beta, \qquad \vec\psi \in \lambda W, \beta \in K, $$
	que es precisamente lo que se quería probar.
\end{proof}

\begin{mydef}
	Sea $K$ un cuerpo global.
	Los elementos del grupo multiplicativo del anillo de adèles $K_\A^\times$ se denominan \strong{idèles}\index{idele@idèle}, esto equivale a dar un adèle
	$\vec\alpha \in K_\A$ cuyas coordenadas son todas no nulas y tal que $|\alpha_v|_v = 1$ para todos salvo finitos lugares $v$'s.
	Los elementos $\beta \in K^\times \subseteq K_\A^\times$ se denominan \strong{idèles principales}\index{idele@idèle!principal}.

	El conjunto de idèles se denota $I_K$ y lo dotamos de la topología inicial inducida por la función:
	$$ I_K \to K_\A \times K_\A, \qquad \vec x \mapsto (\vec x, \vec x^{-1}), $$
	la cual convierte a $I_K$ en un grupo topológico.
	Sobre el grupo de idèles tenemos el siguiente homomorfismo multiplicativo:
	$$ |\,| \colon I_K \to (\R_{> 0}, \cdot), \qquad \vec x \mapsto \prod_{v\in M_K} |x_v|_v, $$
	denotaremos por $I_K^0$ a su núcleo.

	El \strong{grupo de clases de idèles}\index{grupo!de clases!de idèles} de $K$ es el cociente topológico $\Cl_K := I_K / K^\times$.
	La fórmula del producto se traduce en que $K^\times \subseteq I_K^0$, de modo que el homomorfismo se factoriza en $|\,|\colon \Cl_K \to \R_{>0}$;
	y denotamos por $\Cl_K^0 = I_K^0/K^\times$ a su núcleo.
\end{mydef}
\warn
La topología sobre $I_K$ \underline{no} es la topología subespacio de $K_\A$ y, \textit{a priori}, $\Cl_K$ no es un grupo topológico
(aunque sí es un grupo y sí posee una topología).

\begin{prop}
	Sea $K$ un cuerpo global.
	El subgrupo de idèles principales $K^\times \le I_K$ forma un subgrupo discreto (en la topología de $I_K$).
	En consecuente, $K^\times \le_f I_K$ es un subgrupo cerrado y $\Cl_K$ es un grupo topológico.
\end{prop}
\begin{proof}
	Basta factorizar
	\[\begin{tikzcd}[row sep=0pt]
		K^\times \rar & K^\times \times K^\times \rar["i", hook] & K_\A \times K_\A \\
		a \rar[mapsto] & (a, a^{-1})
	\end{tikzcd}\]
	donde $i$ es la inclusión canónica.
	Como $K^\times$ es discreto en la topología de $K_\A$, esta factorización muestra que $K^\times$ también lo es en la topología de $I_K$.
\end{proof}

\addtocounter{thmi}{1}
% \begin{slem}
%	 Sea $K$ un cuerpo global.
%	 El homomorfismo $|\,| \colon I_K \to \R_{> 0}$ es continuo.
% \end{slem}

\begin{slem}
	Sea $K$ un cuerpo global.
	El conjunto $I_K^0$ es cerrado en $K_\A$ y su $K_\A$-topología subespacio coincide con su $I_K$-topología subespacio.
\end{slem}
\begin{proof}
	Sea $\vec a \in K_\A$ tal que $\vec a \notin I_K^0$, queremos encontrar un $K_\A$-entorno de $\vec a$.
	Veamos por casos:
	\begin{enumerate}[(a)]
		\item $\prod_{v\in M_K} |a_v|_v < 1$:
			Sea $S \subseteq M_K$ el subconjunto de lugares $v \in M_K$ tales que $|a_v|_v > 1$, el cual es finito.
			Entonces el conjunto $W$ de adèles $\vec\gamma \in K_\A$ tales que
			\begin{equation}
				\forall v \in S, w \notin S, \qquad |\gamma_v - \alpha_v|_v < \epsilon, \; |\gamma_w|_w \le 1
				\label{eqn:adele_neighborhood}
			\end{equation}
			es $K_\A$-abierto.
			Agregando lugares a $S$ y eligiendo $\epsilon > 0$ suficientemente pequeño, podemos asegurar que $W$ no corte a $I_K^0$.

		\item $\prod_{v\in M_K} |a_v|_v =: C > 1$:\footnotemark{}
			\footnotetext{¿Por qué habría de existir este producto infinito?}
			Existe un subconjunto finito $S \subseteq M_K$ y un número $\epsilon > 0$ tales que:
			\begin{itemize}
				\item Si $v \in M_K$ es tal que $|a_v|_v > 1$, entonces $v \in S$.
				\item Si $v \notin S$ y $\vec\gamma \in K_\A$ satisface $|\gamma_v|_v < 1$, entonces $|\gamma_v|_v \le \frac{1}{2}C$.
				\item Si $v \in S$ y $\vec\gamma \in K_\A$ satisface $|\gamma_v - \alpha_v|_v < \epsilon$,
					entonces $1 < \prod_{v\in S} |\gamma_v|_v < 2C$.
			\end{itemize}
			Entonces el conjunto $W$ de adèles $\vec\gamma \in K_\A$ tales que satisfacen \eqref{eqn:adele_neighborhood}
			% $$ \forall v \in S, w \notin S, \qquad |\gamma_v - \alpha_v|_v < \epsilon, \; |\gamma_w|_w \le 1 $$
			es un $K_\A$-entorno de $\vec a$.
	\end{enumerate}
	Así, sabemos que $I_K^0$ es cerrado en $K_\A$.

	Sea $W \subseteq I_K^0$ un conjunto que contiene a un idèle $\vec a \in I_K^0$.
	Si $W$ es $K_\A$-abierto, entonces contiene a un $K_\A$-entorno de $\vec a$ cuyos elementos $\vec\gamma$ satisfacen \eqref{eqn:adele_neighborhood}
	% \begin{equation}
	%	 \forall v \in S, w \notin S \qquad |\gamma_v - \alpha_v|_v < \epsilon, \; |\gamma_w|_w \le 1.
	%	 \label{eqn:}
	% \end{equation}
	para algún $S \subseteq M_K$ finito y $\epsilon > 0$.
	Esto contiene al $I_K$-entorno de $\vec a$ cuyos elementos $\vec\gamma$ satisfacen
	\begin{equation}
		\forall v \in S, w \notin S \qquad |\gamma_v - \alpha_v|_v < \epsilon, \; |\gamma_w|_w = 1.
		\label{eqn:idele_neighborhood}
	\end{equation}

	Supongamos que $W$ es $I_K$-abierto, es decir, contiene a la intersección con un $I_K$-entorno $H$ de $\vec a$ cuyos elementos $\vec\gamma$
	satisfacen \eqref{eqn:idele_neighborhood} para un conjunto $S$ que contiene a los lugares arquimedianos $M_K^\infty$ y los lugares $v$'s
	tales que $|a_v|_v \ne 1$.
	Si achicamos $\epsilon > 0$ lo suficiente, entonces para todo $\vec\gamma \in H$ se cumplirá que
	$$ 2^{-1} < \prod_{v\in M_K} |\gamma_v|_v < 2. $$
	La intersección $H \cap I_K^0$ es la misma que la del $K_\A$-entorno dado por \eqref{eqn:adele_neighborhood}, de modo que es $K_\A$-abierto.
\end{proof}
\addtocounter{thmi}{-1}

\begin{thm}
	Sea $K$ un cuerpo global.
	El grupo $\Cl_K^0$ es compacto.
\end{thm}
\begin{proof}
	Por el lema anterior, basta probar que existe un subconjunto compacto $W \subseteq K_\A$ tal que la restricción de la proyección
	$W \cap I_K^0 \to \Cl_K^0$ siga siendo sobreyectiva.
	Sea $C > 0$ la constante dada en el lema~\ref{thm:adelic_constant}, y sea $\vec\alpha \in I_K$ un idèle con $\prod_{v\in M_K} |\alpha_v|_v > C$;
	sea $W$ el conjunto de los $\vec\gamma \in K_\A$ tales que
	$$ \forall v\in M_K \qquad |\gamma_v|_v \le |\alpha_v|_v. $$
	Claramente $W$ es compacto y para todo idèle $\vec\beta \in I_K^0$ existe un idèle principal $\lambda \in K^\times$ tal que
	$$ \forall v\in M_K \qquad | \lambda|_v \le |\beta_v^{-1}\alpha_v|_v, $$
	es decir, $\lambda\vec\beta \in W$ como se quería ver.
\end{proof}

\begin{cor}
	Sea $K$ un cuerpo global.
	El grupo de clases de ideales $\Cl(\mathcal{O}_K)$ del anillo de enteros $\mathcal{O}_K$ es finito.
\end{cor}
\begin{proof}
	Sea $I_{\mathcal{O}_K}$ el grupo de ideales fraccionarios de $\mathcal{O}_K$.
	Recuérdese que, por el primer teorema de Ostrowski, existe una biyección entre lugares finitos e ideales primos de $\mathcal{O}_K$;
	con ella construyamos el siguiente homomorfismo:
	\[
		I_K^0 \longrightarrow I_{\mathcal{O}_K}, \qquad
		\vec a \longmapsto \prod_{\mathfrak{p} \in \Spec\mathcal{O}_K} \mathfrak{p}^{ \nu_{\mathfrak{p}} (a_{\mathfrak{p}}) }.
	\]
	Dotando a $I_{\mathcal{O}_K}$ de la topología discreta, vemos que la función es continua y es fácil ver que es sobreyectiva.
	La imagen de los idèles principales $K^\times$ cae en el subgrupo de ideales fraccionarios principales $P_{\mathcal{O}_K}$,
	de modo que tomando cocientes a ambos lados obtenemos un homomorfismo continuo sobreyectivo $\Cl_K^0 \to \Cl(\mathcal{O}_K)$ y,
	como $\Cl_K^0$ es compacto y $\Cl(\mathcal{O}_K)$ es discreto, concluimos que $\Cl(\mathcal{O}_K)$ es finito.
\end{proof}
El corolario anterior \textit{per se} ya es una bonita consecuencia de las nociones de adèles e idèles,
pero combinándolo con técnicas de la geometría de los números podemos mejorar bastante.

\begin{mydef}
	Sea $k$ un cuerpo con un conjunto de lugares $M$ (que supondremos es $M_k$ cuando $k$ sea global),
	y sea $S \subset M$ un subconjunto de lugares que contiene a los arquimedianos.
	Se denota
	\[
		\mathfrak{o}_{S, k} := \bigcap_{v \in M \setminus S} \mathfrak{o}_v = \{ a \in k : \forall v \notin S \quad |a|_v \le 1 \}.
	\]
	De no haber ambigüedad sobre los signos, obviaremos el subíndice <<$k$>>.
	Como los lugares $v \notin S$ son no arquimedianos, la desigualdad ultramétrica prueba que $\mathfrak{o}_S$ es un anillo.
	Los elementos de su grupo de unidades se llaman \strong{$S$-unidades} de $k$:
	$$ U_{S, k} := \mathfrak{o}_{S, k}^\times = \{ a \in k : \forall v \notin S \quad |a|_v = 1 \}. $$
\end{mydef}
\begin{ex}
	Sea $K$ un cuerpo numérico.
	\begin{itemize}
		\item Las $M_K^\infty$-unidades de $K$ son precisamente las unidades de su anillo de enteros $\mathcal{O}_K$.
		\item Sea $\mathfrak{p} = \pi\mathcal{O}_K \in \Spec(\mathcal{O}_K)$.
			Las $M_K^\infty \cup \{ \mathfrak{p} \}$-unidades son productos de unidades de $\mathcal{O}_K$ con potencias enteras de $\pi$.
	\end{itemize}
\end{ex}

\begin{thmi}[Teorema de las unidades de Dirichlet-Chevalley-Hasse]\index{teorema!de las unidades!de Dirichlet-Chevalley-Hasse}
	Sea $K$ un cuerpo numérico y sea $M_K^\infty \subseteq S \subseteq M_K$ un conjunto finito de lugares.
	Entonces el grupo de $S$-unidades de $K$ es el producto directo entre un grupo finito (que corresponde a las raíces de la unidad de $K$)
	y a un grupo abeliano libre de rango $|S| - 1$.
	En particular, $U_{K, S}$ es finitamente generado.
	% En particular, si $K$ posee $r$ lugares reales y $2s$ lugares complejos
\end{thmi}
\begin{proof}
	Consideremos el homomorfismo
	$$ \lambda\colon U_{K, S} \longrightarrow \R^{|S|}, \qquad a \longmapsto \log|a|_v. $$
	El núcleo de $\lambda$ serían los elementos $a \in K^\times$ tales que $|a|_v = 1$ para todos los lugares;
	o equivalentemente, serían números algebraicos de altura nula, pero un teorema de Kronecker nos dice que las raíces de la unidad son las únicas
	que satisfacen aquello.
	La imagen de $\lambda$ cae en el $\R$-subespacio vectorial
	$$ \left\{ \vec x \in \R^{|S|} : \sum_{v\in S} x_v = 0 \right\} $$
	el cual tiene ($\R$-)dimensión $s - 1$.
	Más aún, generan $\R$-linealmente dicho espacio, puesto que la imagen de $I_S^0$ lo hace.

	Finalmente, la imagen $\lambda[U_{K, S}]$ es discreta, puesto que para toda tupla de constantes $0 < c_v < C_v$ para cada $v \in S$, el conjunto
	de $S$-unidades $\eta \in U_{K, S}$ que satisfacen
	$$ \forall v \in S \qquad c_v \le |\eta|_v \le C_v $$
	es finito, ya que es la intersección entre un conjunto compacto del grupo de idèles $I_K$ y el subgrupo discreto $K^\times$.
\end{proof}

% \subsection{Aplicación: el grupo de clases}
% Para esta sección seguimos a \citeauthor{lang:diophantine}~\cite{lang:diophantine}.
% \begin{mydef}
%	 Sea $K$ un cuerpo y $M_K$ un conjunto propio de lugares de $K$.
%	 Un \strong{$M_K$-divisor (multiplicativo)}\index{mdivisor@$M$-divisor} es una función $\mathfrak{d} \colon M_K \to (0, \infty) \subseteq \R$
%	 que satisface lo siguiente:
%	 \begin{enumerate}
%		 \item $\mathfrak{d}(v) = 1$ para todos salvo finitos lugares $v \in M_K$.
%		 \item Para todo lugar finito $v \in M_K^0$ existe $\alpha \in K^\times$ tal que $\mathfrak{d}(v) = |\alpha|_v$.
%	 \end{enumerate}
%	 Se dice que $\mathfrak{d}$ es \strong{principal}\index{mdivisor@$M$-divisor!principal} si existe $\alpha \in K^\times$
%	 tal que $\mathfrak{d}(v) = |\alpha|_v$ para cada $v \in M_K$.

%	 Sea $L/K$ una extensión finita.
%	 Recuérdese que definimos $M_L$ como el conjunto de valores absolutos en $L$ que extienden a los de $M_K$.
%	 Si $\mathfrak{d}$ es un $M_K$-divisor, entonces también es un $M_L$-divisor con $\mathfrak{d}(w) := \mathfrak{d}(v)$ cuando $w \mid v$.

%	 Se dice que un $M$-divisor es \strong{finito}\index{mdivisor@$M$-divisor!finito} si 
% \end{mydef}
% En otras palabras, un $M$-divisor es la imagen de un idèle.

\subsection{Aplicaciones: sumas de cuadrados}
% Veamos algunas aplicaciones:
\begin{thm}
	Un primo $p \equiv 1 \pmod 4$ es suma de dos cuadrados.
\end{thm}
\begin{proof}
	% {Teo.~\ref{thm:primes_two_square}}
	Por el criterio de Euler sabemos que $-1$ es un residuo cuadrático si $p \equiv 1 \pmod 4$, de modo que $1 + m^2 \equiv 0 \pmod p$
	para algún $m \in \Z$.
	Consideremos $\vec u = (1, m)$ y $\vec v = (0, p)$ los cuales son linealmente independientes y generan el reticulado $\Lambda = \vec u\Z + \vec v\Z$,
	cuyos paralelepípedos fundamentales tienen área $p$.
	Sea $\vec w = a\vec u + b\vec v \in \Lambda$, nótese que
	$$ \|\vec w\|^2 = a^2 + (am + bp)^2 \equiv a^2 + (am)^2 = a^2(1 + m^2) \equiv 0 \pmod p. $$
	Considere $X = B_{\sqrt{2p}}(\Vec 0)$, la bola de radio $\sqrt{2p}$.
	Luego $\mu(X) = 2\pi p > 4p$ por lo que posee un punto $\vec w$ de $\Lambda$ tal que $p \mid \|w\|^2$ y $\|\vec w\|^2 < 2p$,
	luego $\|\vec w\|^2 = p$ y $p$ es la suma de dos cuadrados.
\end{proof}

Otra aplicación es la siguiente, propuesta por \citeauthor{ankeny:squares}~\cite{ankeny:squares}:
\begin{thm}
	Todo número que no es de la forma $4^a(8n + 7)$ con $a, n \in \N$ es una suma de tres cuadrados.
	% Recíprocamente las sumas de tres cuadrados son exactamente los números de esa forma.
\end{thm}
\begin{proof}
	Es fácil notar que basta verlo para $a = 0$. Sea $m = p_1\cdots p_r$ libre de cuadrados, veamos la demostración por casos:
	\begin{enumerate}[(a)]
		\item \underline{Caso $m\equiv 3 \pmod 8$:} 
			En primer lugar, por el teorema de Dirichlet, podemos encontrar un primo $q$ tal que
			$$ \forall i \; \leg{-2q}{p_i} = 1, \qquad q \equiv 1 \pmod 4. $$
			Luego, elaborando los símbolos de Jacobi se obtiene que
			\begin{align*}
				1 &= \prod_{i=1}^{r} \left( \frac{-2q}{p_i} \right)
				= \prod_{i=1}^{r} \left(\frac{-2}{p_i}\right) \left(\frac{q}{p_i}\right) \\
				  &= \left(\frac{-2}{m}\right) \prod_{i=1}^{r} \left(\frac{p_i}{q}\right)
				  = \left(\frac{-2}{m}\right) \left(\frac{m}{q}\right)
				  = \left(\frac{-2}{m}\right) \left(\frac{-m}{q}\right) = \left(\frac{-m}{q}\right).
			\end{align*}
			En consecuente, existe un $b > 0$ impar tal que $b^2 \equiv -m \pmod q$, o equivalentemente, existe $\bar h$ tal que
			\begin{equation}
				b^2 - \bar hq = -m,
				\label{eq:three_sqr_jacobi_1}
			\end{equation}
			analizando la misma fórmula mód 4 se obtiene que $1 - \bar h q \equiv 1 \pmod 4$, de lo que se deduce que $\bar h$ es múltiplo
			de 4 y $4h = \bar h$.

			Como $-2q$ es un residuo cuadrático mód $m$, entonces $(-2q)^{-1}$ también lo es y, por tanto, existe un entero $t$
			tal que $t^2 \cdot (-2q) \equiv 1 \pmod m$.
			Luego considere
			\begin{align*}
				R(x, y, z) &:= 2tqx + tby + mz \\
				S(x, y, z) &:= \sqrt{2q}x + \frac{b}{\sqrt{2q}}y \\
				T(x, y, z) &:= \sqrt{\frac{m}{2q}} y
			\end{align*}
			las cuales son transformaciones lineales y es fácil notar que el conjunto $C$ de los $(x, y, z)$'s tales que $R^2 + S^2 + T^2 < 2m$
			es convexo.
			En coordenadas $(R, S, T)$, el conjunto tiene medida $\frac{4}{3}\pi(2m)^{3/2}$ y el determinante de las transformaciones lineales
			descritas es $m^{3/2}$ de modo que la medida de $C$, en coordenadas $(x, y, z)$, es $ \frac{1}{3} 2^{7/2}\pi \approx 11.84 > 8$;
			por lo cual, tomando el reticulado $\Lambda = \vec e_1\Z + \vec e_2\Z$ vemos que $C$ contiene un punto de coordenadas enteras
			$(x_1, y_1, z_1)$ con imágenes $R_1, S_1, T_1$.
			% De hecho, el conjunto es una transformación lineal de la bola unitaria abierta y un cálculo te da que la medida de dicha esfera
			% es $ \frac{4}{3}\pi (2m)^{3/2}$.
			Nótese que dicha solución satisface lo siguiente:
			\begin{align*}
				R_1^2 + S_1^2 + T_1^2 &= (2tqx_1 + tby_1 + mz_1)^2 + \left( \sqrt{2q}x_1 + \frac{b}{\sqrt{2q}}y_1 \right)^2 \\
						      &\qquad {} + \left( \sqrt{\frac{m}{2q}} y_1 \right)^2 \\
						      &\equiv t^2(2qx_1 + by_1)^2 + \frac{1}{\sqrt{2q}}(2qx_1 + by_1)^2 \equiv 0 \pmod m,
			\end{align*}
			por la definición de $t$.
			Más aún,
			\begin{align*}
				R_1^2 + S_1^2 + T_1^2 &= R_1^2 + \left( \sqrt{2q}x_1 + \frac{b}{\sqrt{2q}}y_1 \right)^2 + \left( \sqrt{\frac{m}{2q}} y_1 \right)^2 \\
						      &= R_1^2 + \frac{1}{2q}\big( (2qx_1 + by_1)^2 + my_1^2 \big) \\
						      &= R_1^2 + 2 \underbrace{(qx_1^2 + bx_1y_1 + hy_1^2)}_{=: v},
			\end{align*}
			luego $v \in \Z$ y como $R_1^2 + S_1^2 + T_1^2 \in \Z$ concluimos que $R_1 \in \Z$.
			Lo anterior se reduce a ver que $m \mid R_1^2 + 2v$, pero como $R_1^2 + 2v < 2m$, entonces necesariamente
			\begin{equation}
				R_1^2 + 2v = m.
				\label{eq:three_sqr_id}
			\end{equation}

			Sea $p$ un primo impar tal que $\nu_p(v) = 2n+1$ es impar (posiblemente puede no existir).
			% \todo{¿Por qué existe dicho $p$?}
			\begin{enumerate}[i)]
				\item \underline{Si $p \nmid m$:}
					entonces como $m \equiv R_1^2 \pmod p$ se tiene que
					$$ \left( \frac{m}{p} \right) = +1, $$
					es fácil notar que
					$$ 4qv = (2qx_1 + by_1)^2 + my_1^2, $$
					si $p \mid q$ entonces, por \eqref{eq:three_sqr_jacobi_1}, se cumple que $ \left( \frac{-m}{p} \right) = 1 $.

					Si $p \nmid q$ entonces, la ecuación anterior se traduce en que
					$$ p^{2n+1} \parallel e^2 + mf^2, $$
					de lo que se concluye que $ \left( \frac{-m}{p} \right) = 1 $, ésto debido a que la potencia del $p$ es impar.
					% \todo{¿Por qué ésto se sigue de lo anterior?}

					En cualquier caso $\left( \frac{-m}{p} \right) = 1$, combinado al hecho de que $\left( \frac{m}{p} \right) = +1$, se
					concluye que $\left( \frac{-1}{p} \right) = 1$ y $p \equiv 1 \pmod 4$.

				\item \underline{Si $p \mid m$:}
					Entonces por \eqref{eq:three_sqr_id} se concluye que $p \mid R_1$ y notando que
					$$ m = R_1^2 + \frac{1}{2q}\big( (2qx_1 + by_1)^2 + my_1^2 \big), $$
					notamos que $p \mid 2qx_1 + by_1$.
					Como $m$ está libre de cuadrados, entonces $m/p \not\equiv 0 \pmod p$ y la ecuación anterior se
					reescribe a
					$$ \frac{1}{2q} \frac{m}{p} y_1^2 \equiv \frac{m}{p} \pmod p \iff y_1^2 \equiv 2q \pod p
					\implies \left( \frac{2q}{m} \right) = 1, $$
					pero recordemos que $\left( \frac{-2q}{m} \right) = 1$ de modo que $\left( \frac{-1}{p} \right) = 1$ y $p \equiv 1 \pmod 4$.
			\end{enumerate}
			En consecuencia, todos los primos impares que dividen a $v$ con valuación impar son $\equiv 1 \pmod 4$
			Luego, $2v$ es una suma de dos cuadrados y como $m = R_1^2 + 2v$, entonces $m$ es suma de tres cuadrados.

		\item \underline{Caso $m \equiv 1, 2, 5, 6 \pmod 8$:} 
			Nuevamente, por el teorema de Dirichlet podemos escoger un primo $q$ tal que para todo $p_i$ divisor primo impar de $m$
			se cumpla que $ \left( \frac{-q}{p_i} \right) = 1 $, que $q \equiv 1 \pmod 4$ y que, si $m$ es par, entonces
			$$ m = 2\bar m, \qquad \left(\frac{-2}{q}\right) = (-1)^{ \frac{\bar m - 1}{2} }, \qquad (-q)t^2 \equiv 1 \pmod{p_i}, $$
			siguiendo el mismo despeje se concluye que $ \left(\frac{-m}{q}\right) = 1$ con lo que
			$$ b^2 - qh = -m, $$
			y definiendo
			\begin{align*}
				R(x, y, z) &:= tqx + tby + mz, \\
				S(x, y, z) &:= \sqrt{q}x + \frac{b}{\sqrt{q}}y, \\
				T(s, y, z) &:= \sqrt{ \frac{m}{q} } y.
			\end{align*}
			Luego procedemos de manera análoga al caso anterior. \qedhere
			% \todo{Completar demostración de suma de tres cuadrados.}
	\end{enumerate}
\end{proof}

\begin{mydef}
	Los números triangulares son aquellos de la forma
	$$ \frac{n(n+1)}{2} $$
	para algún $n \in \N$.
\end{mydef}
\begin{thm}[Eureka]\label{thm:three_triangles}
	Todo número natural puede escribirse como suma de tres números triangulares.
\end{thm}
El nombre se debe a que cuando Gauss probó el teorema escribió:
\begin{center}
	\textgreek{ΕΥΡΗΚΑ!} \quad $\rm num = \triangle + \triangle + \triangle$.
\end{center}
\begin{proof}
	Sea $n$ un natural.
	Nótese que $8n + 3 \equiv 3 \pod 8$ de modo que el teorema anterior nos dice que se puede escribir como suma de tres cuadrados:
	$$ x_1^2 + x_2^2 + x_3^2 = 8n + 3 \equiv 3 \pmod 8. $$
	Los cuadrados módulo 8 son $0, 1$ y $4$; de modo que necesariamente $x_i^2 \equiv 1 \pmod 8$.
	Luego, se sigue cada $x_i$ es impar y de la forma $2m_i + 1$, y finalmente
	\begin{align}
		\sum_{i=1}^{3} \frac{m_i(m_i + 1)}{2} &= \frac{1}{8} \sum_{i=1}^{3} (4m_i^2 + 4m_i + 1 - 1) \notag \\
						      &= \frac{1}{8} \left( -3 + \sum_{i=1}^{3} (2m_i + 1)^2 \right) = \frac{1}{8}(-3 + 8n + 3) = n.
						      \tqedhere
	\end{align}
\end{proof}

\section{El lema de Siegel}
El lema de Siegel da una cota efectiva y útil para una solución de un sistema de ecuaciones lineales en $\Z$ en términos de las entradas.
Si en lugar de <<en $\Z$>> contemplásemos las ecuaciones viviendo posiblemente en un anillo de $S$-enteros nos percatamos que da una cota
de una solución de un sistema de matrices en términos de \textit{alturas}.
Si finalmente, un sistema de ecuaciones lo vemos como una subvariedad lineal de un espacio proyectivo, entonces el <<sistema de ecuaciones lineales>>
es ahora un punto en una variedad grassmanniana y la solución es un punto en la subvariedad lineal.

\subsection{Alturas en las variedades grassmannianas}
Esta sección pretende tanto dar un ejemplo más concreto de alturas, como servir de introducción a la teoría de Arakelov.
Aquí $\Q$ es siempre un cuerpo global con su conjunto estándar de valores absolutos $M_\Q$,
y $M_{\algcl\Q}$ es el conjunto de lugares en $\algcl\Q$ que extienden a lugares de $M_\Q$ sin normalizar.

\newcommand{\Ar}{\mathrm{Ar}}
\begin{mydef}
	Sea $\vec x \in \A^n(\algcl\Q)$.
	Dado un lugar $v \in M_{\algcl\Q}$ se define su altura multiplicativa como
	\[
		H_v(\vec x) =
		\begin{cases}
			\max_j |x_j|_v, & v \nmid \infty, \\
			\left( \sum_{j=1}^{n} |x_j|_v^2 \right)^{1/2}, & v \mid \infty,
		\end{cases}
	\]
	(de modo que si $\vec x \in \A^n(\Q)$ y $v \mid p$, entonces coincide con $H_p(\vec x)$.)
	Dado un cuerpo numérico $K \supseteq \algcl\Q$, un punto $\vec x \in \A^n(K)$ y dados $v \mid w \mid p$ (en la torre $\algcl\Q/K/\Q$)
	se define
	$$ H_w(\vec x) := H_v(\vec x)^{[K_w : \Q_p] / [K : \Q]}. $$
	Finalmente, dado un punto $P \in \PP^n(K)$ que en una carta afín viene representado por $\vec x \in \A^n(K)$, se define
	$$ h_\Ar(P) := \sum_{w\in M_K} \log H_w(\vec x), $$
	y se define $H_\Ar(P) := \exp h_\Ar(P)$.
\end{mydef}
\begin{obs}
	La altura global de Arakelov $h_\Ar$ es la altura global inducida por el haz inversible $\mathscr{O}_{\PP^n}(1)$
	con la $M$-métrica de Fubini-Study
	(i.e.,\ si $u \nmid \infty$ entonces la $u$-métrica es la trivial y si $u \mid \infty$, entonces es la $u$-métrica de Fubini-Study).
\end{obs}

\begin{mydef}
	Sea $F$ un cuerpo numérico y fijemos $K := \algcl F$.
	Dado un $L$-subespacio lineal $W \subseteq K^n$ de dimensión $m \le n$, entonces $W$ se corresponde canónicamente al punto cerrado
	$$ P_w := \left[ \bigwedge^m W \right] \in \PP_K\left( \left( \bigwedge^m K^n \right)^\vee \right) \cong \Grass_{m, n} $$
	de modo que denotamos $h_\Ar(W) := h_\Ar(P_W)$.
	Si $A \in \Mat_{m\times n}(K)$ es una matriz de orden $m\times n$ de rango $m \le n$,
	denotaremos por $h_\Ar(A)$ a la altura del subespacio generado por sus filas.

	Más generalmente, si $A \in \Mat_{m\times n}(K)$ es una matriz (posiblemente con $m > n$) de rango $r$,
	denotamos por $h_\Ar^{\rm fila}(A) := h_\Ar(\bigwedge^r W)$, donde $W$ es el $K$-subespacio lineal de $K^n$ generado por las filas de $A$
	y $\bigwedge^r W$ es visto como un punto en el espacio proyectivo $\PP_K( \bigwedge^r K^n )$.
\end{mydef}
\begin{obs}
	Podemos explicitar el cálculo de $h_\Ar$ sobre matrices.
	Sea $K := \algcl\Q$ y sea $A \in \Mat_{m\times n}(K)$ una matriz de rango $m \le n$ y definamos
	$$ \mathcal{J} := \{ I \subseteq \{ 1, \dots, n \} \text{ de cardinalidad } m \}. $$
	Denotemos por $A_I$ la submatriz de $A$ de orden $m\times m$ constituida por las filas de $I \in \mathcal{J}$.
	Entonces el punto asociado a $A$ en $\PP( \bigwedge^m K^n )$ tiene por coordenadas $\det(A_I)$.
	Para $u \in M_{\algcl\Q}$ tenemos que
	\[
		H_u(A) =
		\begin{cases}
			\max_{I \in \mathcal{J}} \{ |\det(A_I)|_u \}, & u \nmid \infty, \\
			\left( \sum_{I\in \mathcal{J}} |\det(A_I)|_u^2 \right)^{1/2}, & u \mid \infty.
		\end{cases}
	\]
	Sea $F \subseteq K$ un cuerpo numérico tal que $A \in \Mat_{m\times n}(F)$.
	Dado $w$ en $M_F$ tal que $u \mid w$ en $\algcl\Q$ y tal que $w \mid p$ para $p \in M_\Q$, definimos
	$$ H_w(A) := H_u(A)^{ [F_w : \Q_p] / [F : \Q] }. $$
	Finalmente, se satisface que
	$$ h_\Ar(A) = \sum_{w \in M_F} \log H_w(A). $$
	De esto es claro que, dado $G \in \GL_n(\algcl\Q)$, se satisface que $h_\Ar(AG) = h_\Ar(A)$.
\end{obs}

\begin{prop}\label{thm:arakelov_binet_formula}
	Sea $A \in \Mat_{m\times n}(\algcl\Q)$ una matriz de rango $m \le n$ y sea $u \in M_{\algcl\Q}^\infty$ un lugar arquimediano.
	Entonces $H_u(A) = |\det(A A^*)|_u^{1/2}$, donde $A^* = \overline{A}^t$ es la conjugada de la traspuesta de $A$.
\end{prop}
\begin{proof}
	Sin perdida de generalidad podemos suponer que $u$ es un lugar complejo y que $A \in \Mat_{m\times n}(\C)$.
	El enunciado ahora se reduce a la clásica fórmula de Binet:
	$$ \det(A^* A) = \sum_{I \in \mathcal{J}} |\det(A_I)|^2. $$
	Identifiquemos a $A$ con su transformación lineal $L \colon \C^m \to \C^n$ en base canónica, con adjunta $L^*$.
	Por funtorialidad
	$$ \bigwedge^m(L) \circ \bigwedge^m(L^*) = \bigwedge^m(L\circ L^*). $$
	% Bajo las identificaciones
	Relativo a la base canónica, $\bigwedge^m(L)$ (resp.\ $\bigwedge^m(L^*)$) posee una única fila (resp.\ columna) cuyas entradas
	son $\det(A_I)$ (resp.\ $\overline{\det(A_I)}$), donde $I \in \mathcal{J}$.
	En consecuencia, $\bigwedge^m(L\circ L^*)$ es una única matriz con entrada $\det(A A^*)$.
\end{proof}
\begin{cor}\label{thm:arklv_can_hgt_mat}
	Sea $K$ un cuerpo numérico y $A \in \Mat_{m\times n}(K)$ una matriz de rango $r > 0$.
	Denotando por $H(A)$ a la altura canónica de la matriz vista como un punto en $\PP^{nm - 1}_K$, entonces
	\[
		H_\Ar^{\rm fila}(A) \le \big( \sqrt{n} H(A) \big)^r.
		% \qquad h_\Ar^{\rm fila}(A) \le r h(A) + \frac{r}{2}\log n.
	\]
\end{cor}
\begin{proof}
	Esto significa que hay $r$ columnas de $A$ que son $K$-linealmente independientes, por lo que existe una submatriz $A' \in \Mat_{r\times n}(K)$
	tal que $H_\Ar^{\rm fila}(A) = H_\Ar(A')$; haciendo la sustitución podemos sustituir $A$ con $A'$ y $m$ con $r$.
	Separemos a $A$ en dos submatrices complementarias $B, C$ de ordenes $m_1\times n$ y $m_2\times n$ resp.
	Entonces para $v \in M_K^0$ se comprueba por definición que
	$$ H_v(A) \le H_v(B) \, H_v(C), $$
	empleando la desigualdad triangular (¿por qué?).
	Si $v \in M_K$ es arquimediano, entonces empleando la proposición anterior tenemos que $H_u(A) = |\det(A\,A^*)|_u^{1/2}$,
	y $A\,A^*$ se puede expandir como matriz por bloques a lo que se reduce a la desigualdad de Fischer:
	\[
		\det
		\begin{vmatrix}
			B\,B^* & C\,B^* \\
			B\,C^* & C\,C^*
		\end{vmatrix} 
		\le \det(B\,B^*) \det(C\,C^*).
	\]
	Así pues, $H_v(A) \le H_v(B) \, H_v(C)$ también se satisface en este caso.
	Luego, tendremos que $h_\Ar(A) \le h_\Ar(B) + h_\Ar(C)$.
\end{proof}

\begin{prop}
	Sea $K := \algcl\Q$, sea $W$ un $K$-subespacio vectorial no nulo de $K^n$ y sea $W^\perp$ su anulador en $(K^n)^\vee$.
	Entonces $h_\Ar(W^\perp) = h_\Ar(W)$.
\end{prop}
\begin{proof}
	Sea $V := K^n$, todo elemento $\vec x \in \bigwedge^m(V)$ determina una transformación lineal $\psi(\vec x) \colon \vec y \mapsto \vec x \wedge \vec y$,
	desde $\bigwedge^{n-m}(V) \to \bigwedge^n(V)$, es decir, determina un elemento $\varphi(\vec x) \in \bigwedge^n(V) \otimes \bigwedge^{n-m}(V^\vee)$.
	El homomorfismo
	$$ \varphi \colon \bigwedge^m(V) \longrightarrow \bigwedge^n(V) \otimes \bigwedge^{n-m}(V^\vee) $$
	es un isomorfismo que manda cada elemento de la base canónica de $\bigwedge^m(V)$ a un elemento
	de la base canónica de $\bigwedge^n(V) \otimes \bigwedge^{n-m}(V^\vee)$ multiplicado por $\pm 1$.
	Mediante este isomorfismo, la recta $\bigwedge^m(W)$ se manda a la recta $\bigwedge^n(V) \otimes \bigwedge^{n-m}(W^\perp)$;
	por lo que las coordenadas de $\left[ \bigwedge^m(W) \right] \in \PP\left( \bigwedge^m(V) \right)$ son, salvo signo,
	las coordenadas de $\left[ \bigwedge^{n-m}(W^\perp) \right] \in \PP\left( \bigwedge^{m-n}(V^\vee) \right)$.
\end{proof}
\begin{cor}\label{thm:arak_hgt_duality}
	Sea $K$ un cuerpo numérico y $A \in \Mat_{m\times n}(K)$ una matriz de rango $m$.
	La altura de Arakelov del espacio de soluciones $\vec y\cdot A = \Vec 0$ es $h_\Ar(A)$.
\end{cor}

\subsection{Variaciones sobre un tema de Siegel}
\begin{lem}[Siegel, 1929]\index{lema!de Siegel}
	Sea $A = [a_{ij}]_{ij} \in \Mat_{N\times M}(\Z)$ una matriz no nula, donde $N > M$.
	Sea $B > 0$ tal que cada $a_{ij} \le B$, entonces existe $\vec x \in \Z^N$ no nulo tal que $\vec x \cdot A = \Vec 0$ con
	$$ \max_i |x_i| \le \lfloor (NB)^{ \frac{M}{N - M} } \rfloor. $$
\end{lem}
\begin{proof}
	Sin perdida de generalidad supongamos que $A$ no posee columnas nulas.
	Para un natural $k > 0$ (sin fijar aún) definamos
	$$ T := \{ \vec x \in \Z^N : \forall i \quad 0 \le x_i \le k \}, $$
	el cual posee $(k+1)^N$ puntos.
	Sean $S_m^+, S_m^-$ la suma de las entradas positivas y negativas resp.\ de la $m$-ésima columna de $A$.
	Así, para $\vec x \in T$ definiendo $\vec y := \vec x\cdot A$ vemos que
	$$ kS_m^- \le y_m \le kS_m^+. $$
	Sea entonces
	$$ T' := \{ \vec y \in \Z^M : \forall m \quad kS_m^- \le y_m \le kS_m^+ \}. $$
	Sea $B_m := \max_i |a_{i,m}|$, entonces $S_m^+ - S_m^- \le NB_m$ de modo que $T'$ posee $\prod_{m=1}^{M} (NkB_m + 1)$ elementos.
	Queremos aplicar el principio del palomar, es decir, buscamos un $k$ tal que
	$$ \prod_{m=1}^{M} (NkB_m + 1) < (k + 1)^N. $$
	Así, elíjase $k$ como la parte entera de $\prod_{m=1}^{M} (NB_m)^{\frac{1}{N - M}}$ y, empleando que $NkB_m + 1 < NB_m(k + 1)$,
	se concluye que la desigualdad está satisfecha.

	Finalmente por el principio del palomar existen $\vec y, \vec z \in T$ tales que $\vec y Z = \vec z A$ y, por tanto, $\vec x := \vec y - \vec z$
	satisface que $\vec x\cdot A = \Vec 0$ y $\max_i |x_i| \le k$ como se quería.
\end{proof}
La condición $N > M$ asegura que siempre $\vec x \cdot A = \Vec 0$ admita soluciones enteras no triviales.
La parte interesante del lema es, por supuesto, la cota para una solución.

Es tangible la conexión entre el lema de Siegel y los problemas atacados en la teoría de Minkowski;
así que procedemos a emplear el teorema de Minkowski-McFeat para obtener una reformulación y mejora del lema de Siegel
% más aún, las cotas parecen tener una reformulación mucho más natural en términos de alturas.
% Procedemos a cumplir ambas exigencias.
en forma de una desigualdad de puntos y grassmannianos.

Primero fijemos la siguiente notación:
\begin{sit}\label{sit:bombieri_vaaler_siegel}
	Sea $K/\Q$ un cuerpo numérico de grado $d$, sean $1 \le m \le n$ enteros y sea $A \in \Mat_{m\times n}(K)$ una matriz de rango $m$.
	Para cada $v \in M_K$ denotemos por $Q_v^n \subseteq K_v^n$ el cubo unitario de volumen 1, más precisamente:
	\[
		Q_v^n :=
		\begin{cases}
			\max_{j=1}^n\{ \|x_j\|_v \} < \frac{1}{2}, & K_v = \R, \\
			\max_{j=1}^n\{ \|x_j\|_v \} < \frac{1}{2\pi}, & K_v = \C, \\
			\max_{j=1}^n\{ \|x_j\|_v \} \le 1, & v \nmid \infty.
		\end{cases}
	\]
	También para cada $v \in M_K$ denotamos
	$$ S_v := \{ \vec y \in K_v^m : \vec y\cdot A \in Q_v^n \} \subseteq K_v^m. $$
	Y construimos
	$$ \Lambda := \{ \vec x \in K^m : \forall v \in M_K^0 \; \vec x \in S_v \}. $$
\end{sit}
\begin{obs}
	En la situación anterior,
	recordando que las transformaciones lineales $L\colon \vec y \mapsto \vec y\cdot A$ son continuas, tenemos que $S_v$ es la preimagen de $Q_v^n$
	bajo $L$, por lo que cada $S_v$ es inmediatamente un abierto no vacío y acotado.
	Si $v$ es arquimediano entonces también es fácil verificar que $S_v$ es convexo y simétrico.
\end{obs}

\begin{lem}
	En la situación~\ref{sit:bombieri_vaaler_siegel} para $v \in M_K^\infty$ se satisface que
	$$ \beta_v(S_v) \ge \|\det(A\,A^*)\|_v^{-1/2}, $$
	donde $A^* = \overline{A}^t$ es la matriz conjugada traspuesta de $A$.
\end{lem}

\begin{lem}
	En la situación~\ref{sit:bombieri_vaaler_siegel}, para $v \in M_K^0$ tal que $v \mid p$ con $p \in M_\Q^0$, se satisface que
	$$ \beta_v(S_v) = |D_{K_v/\Q_p}|_p^{m/2} \left( \max_I\{ \|\det(A_I)\|_u \} \right)^{-1}, $$
	donde $I$ recorre los subconjuntos de $\{ 1, \dots, n \}$ de cardinalidad $m$ y $A_I$ es la submatriz de $m\times m$ de $A$
	con las columnas de índice en $I$.
\end{lem}

\begin{prop}
	En la situación~\ref{sit:bombieri_vaaler_siegel}, existe una $K$-base $\vec x_1, \dots, \vec x_m \in \A^n(K)$ de la imagen de $A$ tal que
	$$ \prod_{j=1}^{m} H(\vec x_j) \le \left( \frac{2}{\pi} \right)^{ms/d} |D_{K/\Q}|_\infty^{m/2d} H_\Ar(A), $$
	donde $s$ es la cantidad de lugares complejos de $K$.
\end{prop}
\begin{proof}
	Agrupando los lemas anteriores y la proposición~\ref{thm:arakelov_binet_formula} se obtiene que
	\[
		\prod_{v\in M_K} \beta_v(S_v) \ge H_\Ar(A)^{-d} \cdot \prod_{p \in M_\Q^0} \prod_{v\mid p} |D_{K_v/\Q_p}|_p^{m/2}
		= H_\Ar(A)^{-d}\cdot |D_{K/\Q}|_\infty^{-m/2},
	\]
	por lo que, el teorema de Minkowski-McFeat nos dice que
	$$ \lambda_1 \cdots \lambda_m \le 2^m |D_{K/\Q}|_\infty^{m/2d} H_\Ar(A). $$
	Ahora queremos estimar los mínimos sucesivos respecto al reticulado $\Lambda$.
	Si $\vec y \in K^m$ es un punto en $\lambda S \cap \Lambda$ para algún $\lambda > 0$, definamos $\vec x := \vec y \cdot A$.
	Por definición de $S := \prod_{v\mid \infty} S_v$ tenemos que $\max_j \|x_j\|_v < \lambda/2$ si $K_v = \R$,
	que $\max_j \|x_j\|_v < \lambda^2/(2\pi)$ si $K_v = \C$ y $\max_j \|x_j\|_v \le 1$ si $v \nmid \infty$; así pues
	$$ H(\vec y\cdot A) < \frac{\lambda}{2} \left( \frac{2}{\pi} \right)^{s/d}. $$
	Por definición de mínimos sucesivos, existen vectores $K$-linealmente independientes $\vec y_1, \dots, \vec y_m \in K^m$
	tales que cada $\vec y_j \in \lambda_j \overline{S}$ con $1 \le j \le m$, por lo que definiendo $\vec x_j := \vec y_j\cdot A$
	vemos que se satisface el enunciado.
\end{proof}

\begin{thm}[lema de Siegel-Bombieri-Vaaler]\index{lema!de Siegel-Bombieri-Vaaler}
	Sea $K$ un cuerpo numérico de grado $d$ y sea $A \in \Mat_{m\times n}(K)$ una matriz de rango $m \le n$.
	Entonces, el núcleo $\ker A$ posee una $K$-base $\vec x_1, \dots, \vec x_{n-m} \in \mathfrak{o}_K^m$ tal que
	$$ \prod_{\ell=1}^{n-m} H(\vec x_\ell) \le |D_{K/\Q}|^{ \frac{n-m}{2d} } H_\Ar(A). $$
\end{thm}
\begin{proof}
	Sea $A'$ una matriz de orden $(n-m)\times m$ cuyas filas forman una base de $\ker A$.
	Claramente $A'$ tiene rango $n-m$ y $\Im(A') = \ker(A)$, de modo que $H_\Ar(A') = H_\Ar(A)$ por el corolario~\ref{thm:arak_hgt_duality}.
	Ahora bien, aplicando la proposición anterior a $A'$ vemos que existe una $K$-base $\vec x_1, \dots, \vec x_{n-m}$ de $\ker A$
	tal que
	$$ \prod_{\ell=1}^{n-m} H(\vec x_\ell) \le \left( \frac{2}{\pi} \right)^{ms/d} |D_{K/\Q}|^{m/2d} H_\Ar(A), $$
	y finalmente concluimos puesto que $2 < \pi$.
\end{proof}

Mediante la cota del corolario~\ref{thm:arklv_can_hgt_mat} obtenemos la siguiente consecuencia:
\begin{cor}
	Sea $K$ un cuerpo numérico de grado $d$ y sea $A \in \Mat_{m\times n}(K)$ una matriz de rango $m$.
	Entonces $\ker A$ posee una $K$-base $\vec x_1, \dots, \vec x_{n-m} \in \mathfrak{o}_K^m$ tal que
	$$ \prod_{\ell=1}^{n-m} H(\vec x_\ell) \le |D_{K/\Q}|^{ \frac{n-m}{2d} } \big( \sqrt{N}H(A) \big)^m. $$
	En consecuencia, existe un punto $\vec x \in \mathfrak{o}_K^m$ tal que $\vec x\cdot A = \Vec 0$ con
	$$ H(\vec x) \le |D_{K/\Q}|^{1/2d} \big( \sqrt{N}H(A) \big)^{ \frac{m}{n-m} }. $$
\end{cor}
Aquí vemos que la versión de Bombieri-Vaaler mejora el lema de Siegel original cambiando el <<$N$>> por <<$\sqrt{N}$>>.

\begin{thmi}[Lema de Siegel relativo]\index{lema!de Siegel relativo}
	Sea $K$ un cuerpo numérico de grado $d$, sea $F/K$ una extensión finita de grado $r$, sea $A \in \Mat_{m\times n}(F)$ una matriz con entradas en $F$
	y supongamos que $rm < n$.
	Entonces existen $n - rm$ vectores $K$-linealmente independientes $\vec x_1, \dots, \break \vec x_{n-rm} \in \mathfrak{o}_K^m$ tales que
	\begin{align*}
		\forall \ell \quad \vec x_\ell\cdot A &= \Vec 0 \\
		\intertext{y}
		\prod_{\ell=1}^{n-rm} H(\vec x_\ell) &\le |D_{K/\Q}|^{ \frac{n-rm}{2d} } \prod_{j=1}^{m} H_\Ar(A_{j,*})^r.
	\end{align*}
\end{thmi}

\section*{Notas históricas}
El nombre <<teorema de Minkowski-McFeat>> es no estándar.
En realidad, ésta es una reformulación adélica (original de \citeauthor{mcfeat:geometry}~\cite{mcfeat:geometry})
de lo que se conoce en el folclore como el <<segundo teorema de Minkowski>>.
Los teoremas que se siguen de él corresponden a la formulación original del alemán \textbf{Hermann Minkowski} (1864-1909)
publicados en el libro póstumo \cite{minkowski1896geometrie}~(\citeyear{minkowski1896geometrie}).
% Los teoremas 
% que corresponde a los enunciados originales de Minkowski.

El lema de Siegel fue demostrado originalmente en \cite{siegel29diophantischer}~(\citeyear{siegel29diophantischer}).
El lema de Bombieri-Vaaler fue probado en \cite{bombieri83siegel}~(\citeyear{bombieri83siegel}).
\addtocategory{historical}{minkowski1896geometrie, siegel29diophantischer, bombieri83siegel}

Ahora procedemos a hacer un breve recuento de la historia de los adèles, según \citeauthor{roquette:riemann}~\cite[191\psq]{roquette:riemann}.
En primer lugar, Claude Chevalley introdujo la noción de los idèles como una herramienta para la teoría de cuerpos de clase;
el primer registro fue la carta a Edmund Hasse del 20 de junio de 1935, y más tarde se dieron a conocer en el encuentro anual de la Sociedad Matemática Alemana
el 12 de septiembre de 1938.
Inspirados en Chevalley, André Weil introdujo los adèles (bajo el nombre de \textit{diferenciales}) en un artículo de 1938 publicado en la revista Crelle
acerca de una demostración del teorema de Riemann-Roch;
en simultáneo, Emil Artin y George Whaples introdujeron la noción de adèle (bajo el nombre de \textit{vector de valuación}) en \cite{artin:prod_form}.
Los nombres \textit{idèle} y \textit{adèle} tomaron fuerza gracias a algunas reseñas de Hasse, y al libro de \citeauthor{weil:basic}~\cite{weil:basic}.
Estos también fueron una herramienta fundamental en la tesis de \citeauthor{tate67fourier}~\cite{tate67fourier} \citeyear{tate67fourier}.
\addtocategory{history}{roquette:riemann}
\addtocategory{historical}{tate67fourier}

\printbibliography[segment=\therefsegment, check=onlynew, notcategory=history, notcategory=historical, notcategory=other]
\bibbycategory[segment=\therefsegment, check=onlynew]

\end{document}
