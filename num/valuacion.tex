\documentclass[teoria-numeros.tex]{subfiles}
\begin{document}

\chapter{Teoría de valuación}

\section{Valores absolutos y cuerpos métricos}
\begin{mydefi}
	Sea $k$ un cuerpo, una función $| \, |\colon k \to \R$ se dice una aplicación \strong{valor absoluto}\index{valor absoluto} si:
	\begin{enumerate}[{VA}1.]
		\item $|x| > 0$ para todo $x \ne 0$ y $|0| = 0$.
		\item $|xy| = |x| \, |y|$.
		\item $|x + y| \le |x| + |y|$ (desigualdad triangular)\index{desigualdad!triangular}.
	\end{enumerate}
	Si además, satisface que $|x + y| \le \max\{|x|, |y|\}$ (desigualdad ultramétrica)\index{desigualdad!ultramétrica},
	entonces $|\,|$ se dice un \strong{valor absoluto no arquimediano}\index{valor absoluto!no arquimediano} y
	de lo contrario se dice \strong{arquimediano}.
	% En general, denotamos por $|\,|$ a los valores absolutos, y por $v$ a las valuaciones.
	% Reservamos subíndices de haber ambigüedad.

	Todo valor absoluto induce una métrica $d(x, y) := |x - y|$ sobre $k$, y por ende, una topología.
	Por ello, se dice que el par $(k, |\,|)$ es un \strong{cuerpo métrico}\index{cuerpo!métrico} si $|\,|$ es un valor absoluto sobre $k$;
	de no haber ambigüedad sobre los signos obviaremos el valor absoluto.
	Se dice que $k$ es un \strong{cuerpo métrico arquimediano}\index{cuerpo!métrico!arquimediano} si $|\,|$ es arquimediano
	y que $k$ es un \strong{cuerpo ultramétrico}\index{cuerpo!ultramétrico} de lo contrario.
	Se dice que dos valores absolutos son \strong{equivalentes}\index{equivalentes (valores absolutos)} si inducen la misma topología sobre $k$.
\end{mydefi}

\begin{ex}
	\begin{itemize}
		\item Sea $k$ un cuerpo arbitrario. Entonces $|\,| \colon k \to \R $ dado por
			$$ |x| = \chi_{k^\times}(x) =
			\begin{cases}
				0, & x = 0 \\
				1, & x \ne 0
			\end{cases} $$
			es un valor absoluto no aquimediano, llamado el valor absoluto \strong{trivial}.
			Nótese que el valor absoluto induce la topología discreta.
			\textit{Ojo}, la expresión <<cuerpo métrico discreto>> la emplearemos con otros fines.

		\item Los valores absoluto estándar sobre $\R$ y $\C$ son efectivamente funciones <<valor absoluto>>
			e inducen las topologías usuales resp.; les denotaremos $| \, |_\infty$ para diferenciarles de otros valores absoluto.
			\nomenclature{$|\,|_\infty$}{Valor absoluto usual sobre $\R$ o sobre $\C$}
	\end{itemize}
\end{ex}
% Nótese que para todo valor absoluto se cumple que
% $$ |1| = |1^2| = |(-1)^2| = |1|^2 \implies |1| = |(-1)| = 1. $$
% Y además $|-x| = |x|$ y $|x^{-1}| = |x|^{-1}$ para todo $x \ne 0$.

\begin{cor}
	Sea $k$ un cuerpo métrico. Entonces:
	\begin{enumerate}
		\item $|1| = 1$.
		\item $|a^n| = |a|^n$ para todo $n \in \Z$.
		\item $|-1| = 1$, por ende, $|-a| = |a|$.
		\item Para todo $n \in \N$ se cumple que $|n| \le n$.
		\item Si $k$ es finito, entonces $|\,|$ es trivial.
	\end{enumerate}
\end{cor}

Será necesario comprobar lo siguiente:
\begin{thm}
	Sea $k$ un cuerpo métrico.
	Las funciones:
	$$ +\colon k\times k \to k, \quad \cdot\colon k\times k \to k, \quad ()^{-1} \colon k^\times \to k^\times $$
	son continuas.
	Equivalentemente, $k$ es un cuerpo topológico.
	Además, como es un espacio métrico, la función $|\,|\colon k \to \R$ es continua.
\end{thm}
\begin{proof}
	En el enunciado y la demostración $k\times k$ denota el producto como espacios topológicos.
	Es sabida que dicha topología es la misma que aquella inducida por la métrica $L^2$ (y cualquier métrica $L^p$ con $p \in [1, \infty]$).
	En particular fijaremos la métrica $L^\infty$, en donde:
	$$ d\big( (a_1, b_1), (a_2, b_2) \big) = \max\{ |a_1 - a_2|, |b_1 - b_2| \} < \delta $$

	Sean $a, b \in k$, demostrar que $+$ es continua equivale a ver que para todo $\epsilon > 0$ existe $\delta > 0$ tal que
	$$ d\big( (a_1, b_1), (a_2, b_2) \big) < \delta \implies |(a_1 + b_1) - (a_2 + b_2)| < \epsilon. $$
	Así pues, basta elegir $\delta = \epsilon/2$.

	Para el producto, sea $(a_1, b_1) \in k$ un punto arbitrario y sea $M := \max\{ |a_1|, |b_1|, 1 \} > 0$.
	Luego elegimos $\delta := \min\{ \frac{\epsilon}{2M + 1}, 1 \}$, y vemos que
	\begin{align*}
		| a_1 b_1 - a_2 b_2 | &= | a_1b_1 - a_2b_1 + a_2b_1 - a_2b_2 | \le |a_1 - a_2| \, |b_1| + |a_2| \, |b_1 - b_2| \\
				      &< \delta( |b_1| + |a_2| ) \le \delta( |b_1| + |a_2| + \delta ) \\
				      &< \frac{\epsilon}{2M + 1}(M + M + 1) = \epsilon.
	\end{align*}
	Finalmente, para la inversa, sea $a \in k^\times$, luego $|a| > 0$. Elegimos $\delta := \min \{ \frac{|a|^2}{2}\epsilon, \frac{|a|}{2} \} > 0$ y
	notamos que si $|a - b| < \delta$, entonces $|b| > |a| - \delta \ge |a|/2$ y
	\begin{equation}
		\left| \frac{1}{a} - \frac{1}{b} \right| = \frac{|a - b|}{|a| \, |b|} < \frac{\delta}{|a|\, |a|/2} \le \epsilon.
		\tqedhere
	\end{equation}
\end{proof}

\begin{prop}\label{thm:valuation_equiv}
	Sean $|\,|_1$, $|\,|_2$ dos valores absolutos no triviales sobre $k$.
	Entonces las siguientes afirmaciones son equivalentes:
	\begin{enumerate}
		\item $|\,|_1$ y $|\,|_2$ son valores absolutos equivalentes.
		\item $|x|_1 < 1$ implica $|x|_2 < 1$ para todo $x \in k$.
		\item Existe un $\lambda > 0$ real tal que $|x|_1 = |x|_2^\lambda$ para todo $x \in k$.
	\end{enumerate}
\end{prop}
\begin{proof}
	$1 \implies 2$.
	Si $|x|_1 < 1$, entonces $\lim_n |x|_1^n = 0$, por lo que $\lim_n x = 0$.
	Como las topologías son la misma, la convergencia se da para ambos valores absolutos, luego $\lim_n |x|_2^n = 0$, por lo que $|x|_2 < 1$.

	$2 \implies 3$.
	Nótese que si $|x|_1 < 1$ implica $|x|_2 < 1$, luego si $|x|_1 > 1$ elijamos $x^{-1}$ luego $|x^{-1}|_1 < 1$ implica $|x^{-1}|_2 < 1$.
	Como $|\,|_1$ y $|\,|_2$ son no triviales, elijamos $x_0$ tal que $a := |x_0|_1 > 1$ y $b := |x_0|_2 > 1$.
	Sea
	$$ \lambda := \frac{\ln b}{\ln a} > 0, $$
	y claramente se satisface que $|x_0|_1 = |x_0|_2^\lambda$.
	Sea $x \in k^\times$, entonces sea $\alpha > 0$ tal que $|x|_1 = |x_0|_1^\alpha$.
	Luego sean $m, n$ enteros tales que $m/n > \alpha$, luego
	$$ |x|_1 < |x_0|_1^{m/n} \iff |x^n/x_0^m|_1 < 1 \iff |x^n/x_0^m|_2 < 1 \iff |x|_2 < |x_0|_2^{m/n} $$
	como ello aplica para todo racional, entonces $|x|_2 \le |x_0|_2^\alpha$.
	De manera análoga se comprueba que $|x|_2 \ge |x_0|_2^\alpha$.
	Finalmente se estable que $|x|_1 = |x_0|_1^\alpha = |x_0|_2^{\lambda\alpha} = |x|_2^\lambda$.
	\par
	$3 \implies 1$. Trivial.
\end{proof}

\begin{prop}
	Un cuerpo métrico $k$ es ultramétrico syss para todo $n \in \Z$ se cumple que $|n| \le 1$.
\end{prop}
\begin{proof}
	$\implies$.
	Basta notar que
	$$ |n| = |\underbrace{1 + 1 + \cdots + 1}_n| \le \max\{|1|, \dots, |1|\} = 1 $$
	para $n \ge 0$, y emplear que $|-n| = |n|$ para $n < 0$.

	$\impliedby$.
	Sean $a, b \in k$ arbitrarios, entonces
	\begin{align*}
		|(a + b)^n| &= \left| a^n + \binom{n}{1}a^{n-1}b + \cdots + \binom{n}{n-1}ab^{n-1} + b^n \right| \\
			    &\le |a|^n + |a|^{n-1} |b| + \cdots + |b|^n \le (n + 1) \max\{ |a|^n, |b|^n \}.
	\end{align*}
	Luego aplicando raíces (reales) a los valores absolutos se obtiene que $|a + b| \le \sqrt[n]{n + 1} \max\{|a|, |b|\}$ para todo $n \in \N_{\ne 0}$,
	pero $\lim_n \sqrt[n]{n+1} = 1$, lo que comprueba que $|a + b| \le \max\{|a|, |b|\}$.
\end{proof}

\begin{cor}
	Todo cuerpo métrico de característica no nula es ultramétrico.
\end{cor}

\begin{cor}
	El único valor absoluto (salvo equivalencia) sobre $\Fp$ es el trivial.
\end{cor}
\begin{proof}
	Sea $a \in \Fp^\times$, entonces como $\Fp$ es ultramétrico, $|a| \le 1$, pero $a^{-1} \equiv n \pmod p$ para algún $n \in \N$,
	así que $|a^{-1}| \le 1$ y luego $|a| = 1$.
\end{proof}

\begin{cor}\label{thm:archimedian_over_subflds}
	Si un valor absoluto $|\,|$ sobre $k$ es no arquimediano sobre algún subcuerpo (e.g., es trivial), entonces es no arquimediano en todo $k$.
\end{cor}

\begin{exn}
	Sea $p \in \Z$ primo.
	Se define $\nu_p(a)$, la \strong{valuación $p$-ádica}\index{valuación!padica@$p$-ádica} de $a \in \Z$, como el máximo $n \in \N$ tal que $p^n \mid a$;
	se extiende a que $\nu_p(0) := \infty$.
	Ésto lo podemos extender a $\Q$ definiendo que $\nu_p(a/b) := \nu_p(a) - \nu_p(b)$ (¿por qué está bien definido?).
	Finalmente, eligiendo un real $\lambda \in (0, 1)$ podemos definir:
	$$ \left| \frac{a}{b} \right|_p := \lambda^{\nu_p(a/b)}, $$
	\nomenclature{$\nu_p(a), |a|_p$}{Valuación y valor absoluto $p$-ádico}
	y verificamos que efectivamente sea un valor absoluto no arquimediano.

	La condición VA1 es clara.
	Además, para enteros se verifica que $\nu_p(ab) = \nu_p(a) + \nu_p(b)$ de lo que se concluye VA2.
	Nótese lo siguiente, $n = \nu_p(a/b)$ syss $\frac{a}{b} = p^n \frac{u}{v}$ donde $p \nmid uv$, de éste modo si $n := \nu_p(a / b) \le \nu_p(c/d) =: m$
	$$ \frac{a}{b} + \frac{c}{d} = p^n \frac{u}{v} + p^m \frac{w}{z} = p^n \frac{uz + p^{m-n}wv}{vz}, $$
	donde claramente $p \nmid uz + p^{m-n}wv$ y $p\nmid vz$.
	Con ésto se comprueba que $\nu_p(a/b + u/v) = \min\{ \nu_p(a/b), \nu_p(c/d) \}$, ésto se traduce en la desigualdad ultramétrica.

	La observación de cierre es que la elección de $\lambda$ para el valor absoluto no afecta en nada, en el sentido de que otorga valores absolutos
	equivalentes. Por ello, el estándar es escoger $\lambda = 1/p$ de modo que $| a/b |_p = p^{-\nu_p(a/b)}$.
\end{exn}
Más generalmente, si $A$ es un DFU y $K := \Frac A$, el ejemplo anterior nos permite construir valores absoluto mediante los elementos irreducibles de $A$.

\begin{mydefi}
	Un dominio íntegro $A$ se dice un \strong{anillo de valuación}\index{anillo!de valuación} si para todo $a \in \Frac(A)^\times$
	se cumple que $a \in A$ o $a^{-1} \in A$.
\end{mydefi}
\begin{ex}
	\begin{itemize}
		\item Todo cuerpo es un anillo de valuación.
		\item $\Z$ no es de valuación, puesto que $2/3 \in \Q$ satisface que $2/3 \notin \Z$ y $3/2 \notin \Z$.
		\item Considere la localización $\Z_{(p)}$, proponemos que es un anillo de valuación.
			En efecto, $\Frac( \Z_{(p)} ) = \Q$ y para toda fracción reducida $a/b \in \Q$ se cumple que o bien $p\nmid b$, en cuyo caso
			$a/b \in \Z_{(p)}$, o bien $p \mid b$ y $p \nmid a$, en cuyo caso $b/a \in \Z_{(p)}$.
	\end{itemize}
\end{ex}
El nombre <<anillo de valuación>> sugiere que todo anillo de valuación desciende una valor absoluto en $K$, pero ésto podría no ser cierto.
Para ello, vemos que la complicación está en que las funciones hasta $\R$ son demasiado \textit{concretas} y también demasiado \textit{rígidas},
mientras que buscamos una definición más \textit{abstracta} que sí nos permita establecer una analogía con los valores absolutos.
Ésta definición, es la de \textit{valuación}, propuesta por W. Krull que estudiamos en el capítulo 12 de \cite{Alg}.

El último ejemplo es parte de algo más general:
\begin{prop}
	Sea $(k, v)$ un cuerpo ultramétrico. Definamos:
	$$ \mathfrak{o} := \{ a \in k : |a| \le 1 \}, \qquad \mathfrak{m} := \{ a \in k : |a| < 1 \}, $$
	entonces $(\mathfrak{o}, \mathfrak{m})$ es un anillo local de valuación, y su cuerpo de residuos $\kk(v) := \mathfrak{o}/\mathfrak{m}$ se le dice el
	\strong{cuerpo de restos (de clases)}\index{cuerpo!de restos de clases} de $(k, v)$.
\end{prop}
\begin{proof}
	El que $\mathfrak{o}$ sea un dominio íntegro deriva de que $1 \in \mathfrak{o}$, no posee divisores de cero por estar contenido en $k$,
	es cerrado bajo multiplicación por VA2 y es cerrado bajo adición por la desigualdad ultramétrica.

	El hecho de que $\mathfrak{m}$ sea su ideal maximal se deriva de que, las razones anteriores demuestran que $\mathfrak{m}$ es ideal de $\mathfrak{o}$
	y si $a \in \mathfrak{m}$, entonces $a^{-1} \in k$ tiene valor absoluto $|a|^{-1} > 1$, luego $a^{-1} \notin \mathfrak{o}$.
\end{proof}

Con ésto podemos dar un primer teorema de clasificación:
\begin{thmi}[Primer teorema de Ostrowski]\index{teorema!de Ostrowski!I}
	Todo valor absoluto no trivial sobre $\Q$ es (salvo equivalencia):
	\begin{enumerate}
		\item $| \, |_\infty$ si $|\,|$ es arquimediano.
		\item Un valor $p$-ádico $|\,|_p$ si es no arquimediano.
	\end{enumerate}
\end{thmi}
\begin{proof}
	\begin{enumerate}
		\item Sea $n > 1$ entero. Entonces nótese que todo $m$ posee una única expansión en base $n$:
			$$ m = m_t n^t + \cdots + m_1n + m_0, \qquad 0\le m_i \le n-1, m_t \ne 0 $$
			donde $n^t \le m < n^{t+1}$.
			Luego por desigualdad triangular, empleando que $|m_i| \le n-1$, se concluye que $|m| \le (t + 1)(n - 1)r^t$.
			% Nótese que como $|\,|$ es arquimediano, entonces $|m| > 1$ para algún $m$, luego $|m|^j > 1$ para todo $j$.
			Como $m^j < n^{j(t+1)}$, luego
			$$ |m|^j = |m^j| \le j(t + 1)(n-1)r^{jt} \implies r > \frac{\sqrt[t]{m}}{\sqrt[jt]{ j(t+1)(n-1) }}, $$
			aplicando límites se concluye que $r \ge \sqrt[t]{|m|}$; como algún $m$ tiene valor absoluto $|m| > 1$ (por ser arquimediano),
			entonces $r > 1$ y como $t = \lfloor \log_n(m) \rfloor = \log_n(m)$ se concluye que
			$$ |m| \le r^t \le r^{\log_n(m)}. $$
			Cambiando $m$ entero a $x$ racional, vemos que la misma cota aplica, luego si $|x|_\infty < 1$, entonces $\log_n(x) < 0$ y
			$|x| < 1$ lo que prueba que los valores absolutos son equivalentes.

		\item Supongamos que $| \, |$ es un valor absoluto no trivial sobre $\Q$.
			Por la proposición anterior, sea $(\mathfrak{o}, \mathfrak{m})$ el anillo de valuación asociado a $|\,|$.
			Nótese que, como $|n| \le 1$ para todo $n \in \Z$, concluimos que $\Z \subseteq \mathfrak{o}$.
			Luego $\Z \cap \mathfrak{m}$ es un ideal primo de $\Z$, digamos $(p)$; luego $A \supseteq \Z_{(p)}$.
			Como $| \, |$ es no trivial, entonces $\mathfrak{o} \ne \Q$ y $\Z_{(p)} \ne \Q$ por lo que $p \ne 0$.
			Si $a \in \Z$ es tal que $p \nmid a$, entonces $a \notin \mathfrak{m}$ y necesariamente $|a| = 1$.
			Luego si $p \nmid ab$, entonces $| p^n \frac{a}{b} | = |p|^n$ y de ahí se concluye que $| \, |$ es equivalente a $| \, |_p$. \qedhere
	\end{enumerate}
\end{proof}

Y hay también otro caso:
\begin{thmi}
	Sea $t$ trascendente sobre $\Fp$.
	Entonces todos los valores absoluto no triviales sobre $\Fp(t)$ son (salvo equivalencia):
	\begin{enumerate}
		\item $|\,|_q$ para algún polinomio $q(t)$ irreducible.
		\item $|\,|_\infty$ dado por $|f(t)/g(t)|_\infty = c^{\deg g - \deg f}$, donde $c \in (0, 1)$ es arbitrario.
	\end{enumerate}
\end{thmi}
\begin{proof}
	Fijemos un valor absoluto $|\,|$ sobre $\Fp(t) =: K$; nótese que si lo restringimos a $\Fp$ necesariamente dará el valor absoluto trivial.

	Supongamos que $|t| \le 1$, entonces por desigualdad ultramétrica $|f(t)| \le 1$ para todo $f(x) \in \Fp{}[x]$
	y elijamos $q(t)$ como un polinomio de grado minimal tal que $|p(t)| < 1$.
	Es fácil probar que $q(t)$ es irreducible (¿por qué?).
	Sea $g(t)$ otro polinqmio tal que $|g(t)| < 1$, entonces, por algoritmo de la división
	$$ g(t) = q(t) \cdot h(t) + r(t), $$
	donde $\deg r < \deg q$ o $r = 0$.
	Despejando, tenemos que $r(t) = g(t) - q(t)h(t)$ es también tal que $|r(t)| < 1$, de modo que, por definición de $q(t)$,
	se ha de cumplir que $r(t) = 0$.

	Sea $\phi(t) \in K$, luego $\phi(t) = q(t)^\alpha \cdot \psi(t)$, donde ni el numerador ni el denominador de $\psi(t)$ son divisibles por $q(t)$,
	de modo que $|\psi(t)| = 1$ y $|\phi(t)| = |q(t)|^\alpha$, y elegir $c := |q(t)| \in (0, 1)$.
	% Debido a que $\Fp(t)$ es un DFU, podemos libremente definir $\nu_{p}(\phi) = \alpha$ y podemos normalizar el valor absoluto $|\,|_p$
	% de modo que $\| p(t) \| = c^{-\deg p}$ donde $c > 1$ es una constante fija.

	Por otro lado, supongamos que $|t| > 1$, entonces definamos $y := 1/t$.
	Nótese que $\Fp(t) = \Fp(y)$ y que $|y| =: c < 1$, de modo que, éste ya es el polinomio de grado minimal tal que $|f(y)| < 1$.
	% Ésta valuación la denotamos por $p_\infty$ y consideremos la normalización $\|y\|_{p_\infty} = c^{-1}$.
	Sea $\phi(t) = g(t)/h(t)$ donde $g(t), h(t)$ son polinomios de grados $m, n$ resp., entonces
	$$ \phi(t) = y^{n-m} \frac{g_1(y)}{h_1(y)}, $$
	donde $g_1(y), h_1(y)$ son polinomios (respecto a $y$) que no son divisibles por $y$, de modo que $| \phi(t) | = c^{n-m}$.
\end{proof}
Al igual que con los valores absoluto $p$-ádicos, existen infinitas elecciones, por ello hablaremos de las normalizaciones así:
$$ \| q(t) \|_q := \left| \frac{\Fp{}[t]}{(q(t))} \right|^{-1} = p^{-\deg q}, \qquad \|f(t)\|_\infty = p^{\deg f}, $$
(en la primera igualdad, el $|\,|$ denota cardinalidad).
A veces al teorema anterior se le llama \textit{teorema de Ostrowski para $k(t)$.} 

Si $k$ es un cuerpo métrico, entonces podemos importar las siguientes definiciones de la topología/análisis:
\begin{mydef}
	Sea $k$ un cuerpo métrico.
	Sea $(a_n)_{n \in \N} \subseteq k$ una sucesión.
	Se dice que $(a_n)_n$ converge a un límite $L \in k$, denotado $\lim_n a_n = L$, si para todo
	$$ \forall \epsilon > 0 \exists N \in \N \forall n > N \quad |a_n - L| < \epsilon. $$
	Se dice que $(a_n)_n$ es una \strong{sucesión fundamental}\index{sucesión!fundamental} si es una sucesión de Cauchy, i.e., si para todo $\epsilon > 0$
	existe $N \in \N$ tal que
	$$ \forall n, m > N \quad |a_n - a_m| < \epsilon. $$
	Un cuerpo con valor absoluto se dice \strong{completo}\index{cuerpo!métrico!completo} si toda sucesión fundamental es convergente.

	Se dice que un cuerpo $K$ con un valor absoluto $\| \, \|$ es una \strong{compleción}\index{compleción} de $k$ si:
	\begin{enumerate}
		\item $K$ es completo.
		\item $K/k$ es una extensión de cuerpos y $\|x\| = |x|$ para todo $x \in k$.
		\item $k$ es (topológicamente) denso en $K$, i.e., para todo $x \in K$ y todo $\epsilon > 0$ existe $y \in k$ tal que $\|x - y\| < \epsilon$.
	\end{enumerate}
\end{mydef}
Luego veremos otra definición categorial de \textit{compleción} que nos servirá para fines del álgebra conmutativa,
pero ésta definición es suficiente por el momento.

\begin{thm}\label{thm:metric_fld_completion}
	Todo cuerpo métrico $k$ posee una compleción.
	Más aún, sus compleciones son isométricamente isomorfas.%
	\footnote{Es decir, existe un isomorfismo de cuerpos que además preserva distancias o, en este caso, que respeta el valor absoluto;
		en particular, es también un homeomorfismo.}
\end{thm}
\begin{proof}
	Como $k$ es un espacio métrico, entonces posee una compleción $K$ (como espacio) y podemos definir $\| \alpha \| := d(\alpha, 0)$ para $\alpha \in K$,
	donde $d$ es la métrica en $K$ que extiende a la métrica de $k$.

	Los elementos de $K$ son límites de sucesiones fundamentales en $k$, así pues definimos $\alpha := \lim_n a_n$ y $\beta := \lim_n b_n$.
	Luego $\alpha + \beta := \lim_n (a_n + b_n)$, $\alpha\cdot\beta := \lim_n (a_n\cdot b_n)$ y si $\alpha \ne 0$ y $(a_n)_n$ no se anula,
	entonces se comprueba que $\alpha^{-1} = \lim_n a_n^{-1}$ (esta última es una igualdad, no una definición).
	Todo ésto se puede comprobar a partir de que la suma, el producto y la inversa es continua en un cuerpo topológico y, en particular, lo es
	en un cuerpo métrico.

	Finalmente, si $(K_1, |\,|_1)$ y $(K_2, |\,|_2)$ son compleciones de $k$, con los encajes de cuerpos topológicos $f\colon k \to K_1$ y $g\colon k \to K_2$
	(que son encajes topológicos y monomorfismos de cuerpos),
	entonces todos los elementos de $K_1$ son de la forma $\lim_n f(a_n)$ para alguna sucesión fundamental $(a_n)_n \in k$.
	Luego definimos:
	\begin{align*}
		\varphi \colon K_1 &\longrightarrow K_2 \\
		\lim_n f(a_n) &\longmapsto \lim_n g(a_n),
	\end{align*}
	el cual está bien definido y es un homeomorfismo.%
	\footnote{De hecho, ésta misma función es la que se emplea para probar que la compleción de un espacio métrico es única salvo homeomorfismo.}
	Para ver que además es un homomorfismo de cuerpos, basta recordar que $f$ y $g$ lo son, y que la suma, producto e inverso son continuas.
\end{proof}

\begin{cor}
	Sea $\varphi\colon k \to L$ un encaje de cuerpos topológicos, donde $k$ es métrico y $L$ es métrico completo.
	Entonces, $\overline{\varphi[k]}$ (la clausura topológica) es una compleción de $k$.
\end{cor}

\begin{cor}
	$\R$ sólo posee un valor absoluto arquimediano salvo equivalencia, $| \, |_\infty$.
\end{cor}
\begin{proof}
	Basta notar que $\R$ es la compleción de $(\Q, |\,|_\infty)$ y de que todo valor absoluto en $\R$ se restringe a $\Q$.
\end{proof}

\begin{lem}
	Sean $|\,|_1, \dots, |\,|_n$ valores absoluto sobre $k$ tales que ningún par es equivalente.
	Existe $a \in k$ tal que
	$$ |a|_1 > 1, \quad \forall i\ne 1 \quad |a|_i < 1. $$
\end{lem}
\begin{proof}
	Lo probaremos por inducción sobre $n$.
	El caso base $n = 2$ esta dado puesto que, por la proposición~\ref{thm:valuation_equiv}, podemos encontrar $b, c \in k$ tales que
	$$ |b|_1 < 1, \; |b|_2 \ge 1, \quad |c|_1 \ge 1, \; |c|_2 < 1. $$
	Luego elegimos $a = bc^{-1}$ y notamos que $|a|_1 < 1$ y $|a|_2 > 1$ como se quería.

	Para el caso $n+1$, por hipótesis inductiva y por lo anterior podemos encontrar $b, c \in k$ tales que
	\begin{gather*}
		|b|_1 > 1, \quad |b|_2 < 1, \; \cdots, \; |b|_n < 1 \\
		|c|_1 > 1, \qquad |c|_{n+1} < 1.
	\end{gather*}
	\begin{enumerate}[(a)]
		\item \underline{Si $|b|_{n+1} \le 1$:}
			Entonces elegimos $a := b^r c$ donde $r \in \N$ no está fijo.
			Nótese que $|a|_1 > 1$ y $|a|_i = |b|_i^r \, |c|$ para $1 < i \le n$, luego como $|b|_i^r \to 0$ podemos elegir $r$ suficientemente grande
			de modo que $|a|_i < 1$ para $1 < i \le n$ y es claro que $|a|_{n+1} < 1$.

		\item \underline{Si $|b|_{n+1} > 1$:}
			Entonces elegimos
			$$ a := \frac{b^r}{1 + b^r}c, $$
			con $r \in \N$ sin fijar.
			Como $|b|_i^r \to 0$, entonces se comprueba que $|a|_i < 1$ para todo $1 < i \le n$ si $r$ es suficientemente grande.
			Para $j = 1$ o $j = n+1$, nótese que $|b|_j > 1$ luego nótese que
			$$ |b|_j^r - 1 \le |1 + b^r|_j \le 1 + |b|_j^r $$
			por desigualdad triangular, luego
			$$ 1 = \lim_r \frac{|b|_j^r}{|b|_j^r - 1} \ge \lim_r \frac{|b|_j^r}{|1 + b^r|_j} \ge \lim_r \frac{|b|_j^r}{|b|_j^r + 1} = 1, $$
			con lo cual, por teorema del sandwich, el límite de al medio converge a 1, luego para un $r$ suficientemente grande se cumple
			que $|a|_j$ está <<cerca>> de $|c|_j$ y luego $|a|_1 > 1$ y $|a|_n < 1$ como se quería probar. \qedhere
	\end{enumerate}
\end{proof}

\begin{thm}[de aproximación]\index{teorema!de aproximación}
	Sean $|\,|_1, \dots, |\,|_n$ valores absoluto sobre $k$ tales que ningún par es equivalente.
	Sean $a_1, \dots, a_n \in k$ y sea $\epsilon > 0$, entonces existe $a \in k$ tal que
	$$ \forall 1\le i \le n \quad |a - a_i|_i < \epsilon. $$
\end{thm}
\begin{proof}
	Empleando el lema, podemos obtener $b_i \in k$ tal que $|b_i|_i > 1$ y $|b_i|_j < 1$ para $j \ne i$.
	Luego nótese que
	$$ \lim_r \left| \frac{b_i^r}{1 + b_i^r} \right|_i = 1, \qquad \lim_r \left| \frac{b_i^r}{1 + b_i^r} \right|_j = 0, \; j \ne i. $$
	Luego $a_ib_i^r/(1 + b_i^r)$ tendrá valor absoluto en $|\,|_i$ cercano a $|a_i|_i$ y valor absoluto en $|\,|_j$ cercano a 0.
	Definimos
	$$ a := \sum_{i=1}^{n} \frac{a_i b_i^r}{1 + b_i^r}, $$
	el cual, con $r$ suficientemente grande, satisface las hipótesis requeridas.
\end{proof}
Nótese que hay un paralelo entre el teorema chino del resto y el teorema de aproximación:
En efecto, podemos considerar a los valores absolutos como los $p$-ádicos y los $a_i$'s como residuos mód $p_i$, luego el $a \in \Q$ dado
satisface que $|a - a_i|_i < p^{-n_i}$, o equivalentemente, satisface que $\nu_{p_i}(a - a_i) \ge n_i$

\begin{cor}\label{thm:finite_places_not_prod_form}
	Sean $|\,|_1, \dots, |\,|_n$ valores absoluto no triviales sobre $k$ tales que ningún par es equivalente, y sean $\eta_j \in \R$.
	Una relación del estilo
	$$ \prod_{j=1}^{n} |a|_j^{\eta_j} = 1, $$
	para todo $a \in k^\times$ se da syss cada $\eta_j = 0$.
\end{cor}
\begin{proof}
	Sin perdida de generalidad supongamos que $\eta_1 \ne 0$.
	Sea $a_1$ tal que $|a_1|_1^{\eta_1}$ es suficientemente grande,
	luego podemos elegir $a_j = 1$ para $j\ne 1$ y, por el teorema de aproximación con un $\epsilon$ suficientemente pequeño podemos obtener un $\beta \in k$
	tal que $|\beta|_1^{\eta_1}$ es suficientemente grande y $|\beta|_j^{\eta_j} \approx 1$ para $j \ne 1$, de modo que la relación falle.
\end{proof}

Ahora, a por una sorpresa:
\begin{prop}[fórmula del producto]
	Para todo $x \in \Q^\times$ se satisface que
	$$ |x|_\infty \cdot |x|_2 \cdot |x|_3 \cdots = |x|_\infty \cdot \prod_{p} |x|_p = 1, $$
	donde $p$ recorre todos los primos de $\Z$.
	(Nótese que el producto converge pues $|x|_p = 1$ para todos salvo finitos $p$'s.)
\end{prop}
Demostrar ésta proposición no es más que un ejercicio, pero lo interesante es que por el teorema de Ostrowski estamos tomando un producto sobre todos
los valores absoluto de $\Q$ salvo equivalencia, y que por el corolario anterior, ésto sería imposible de sólo consistir de finitos factores.

En particular, (y ésto nunca lo he visto mencionado en otra parte) tenemos otra demostración de la infinitud de los primos.

También hicimos el caso de $\Fp(t)$ porque obtenemos el mismo resultado:
\begin{prop}[fórmula del producto]
	Para todo $f(t) \in \Fp(t)^\times$ se satisface que
	$$ \|f\|_\infty \cdot \prod_{q(t)} \|f\|_q = 1, $$
	donde $q(t)$ recorre todos los irreducibles de $\Fp{}[t]$ salvo asociados.
	% (Nótese que el producto converge pues $|x|_p = 1$ para todos salvo finitos $p$'s.)
\end{prop}

\subsection{Segundo teorema de Ostrowski}
El primer teorema de Ostrowski nos clasifica los valores absolutos sobre $\Q$, pero hay una segunda versión, también atribuida a Ostrowski,
% \cite[Teo.~1.1, pág.~33]{cassels:local_fields}
que nos clasifica los valores absolutos arquimedianos, simplificando enormemente su estudio.
En cierta manera, éste teorema nos dará otro indicio de unicidad para $\R$.

\begin{lem}
	El único valor absoluto arquimediano $|\,|$ sobre $\C$ es (salvo equivalencia) $|\,|_\infty$.
\end{lem}
\begin{proof}
	Sea $\zeta = a + b\ui$ con $a, b \in \R$.
	Como $\ui^4 = 1$, entonces $|\ui| = 1$.
	Además, sabemos que $|\,|$ en $\R$ es equivalente a $|\,|_\infty$, luego $|a| = |a|_\infty^\lambda$ para algún $\lambda > 0$ real.
	Luego:
	$$ |\zeta| = |a + b\ui| \le |a| + |b| = |a|_\infty^\lambda + |b|_\infty^\lambda \le 2|\zeta|_\infty^\lambda. $$
	Luego elijamos $\alpha, \beta$ y por el teorema de aproximación existe $\gamma$ tal que $|\alpha - \gamma|, |\beta - \gamma|_\infty < \epsilon$,
	pero nótese que
	$$ |\alpha - \gamma|_\infty \ge \left( \frac{|\alpha - \gamma|}{2} \right)^{1/\lambda}
	\ge \left( \frac{|\alpha - \beta| - \epsilon}{2} \right)^{1/\lambda}, $$
	lo cual, eligiendo $|\alpha - \beta|$ y $\epsilon$ apropiadamente, conlleva a una contradicción.
\end{proof}

\begin{lem}
	Sea $k$ un cuerpo métrico completo.
	Supongamos que $t^2 + 1$ es irreducible en $k[t]$, entonces existe $\Delta > 0$ real tal que
	$$ \forall a,b\in k \quad |a^2 + b^2| \ge \Delta \max\{ |a|^2, |b|^2 \}. $$
\end{lem}
\begin{proof}
	Definamos
	$$ \Delta := \frac{|4|}{1 + |4|}. $$
	Probaremos la contrarrecíproca: supongamos que existe $c_1 \in k$ tal que
	$$ \delta_1 := |c_1^2 + 1| < \Delta < 1, $$
	entonces construiremos una sucesión fundamental que converja a una solución del polinomio.
	Para ello definamos $c_2 := c_1 + h_1$, luego
	$$ c_2^2 + 1 = c_1^2 + 1 + 2c_1h_1 + h_1^2, $$
	por lo que elegimos
	$$ h_1 := -\frac{(c_1^2 + 1)}{2c_1}, $$
	con lo que se comprueba que
	$$ \delta_2 := |c_2^2 + 1| = |h_1|^2 = \frac{|c_1^2 + 1|^2}{|4|\,|c_1|^2} \le \delta_1 \theta, $$
	donde
	$$ \theta := \frac{\delta_1}{|4|(1 - \delta_1)} < 1, $$
	donde empleamos que $|c_1| \ge 1 - \delta_1 > 0$ por desigualdad triangular; nótese que $\delta_2 < \delta_1$.
	Definiendo por recursión $c_n$ y $h_n$ del mismo modo, vemos que, por inducción sobre $n$ se cumple
	$$ \delta_{n+1} = |c_{n+1}^2 + 1| = |h_n|^2 \le \delta_n \theta \le \delta_1 \theta^n. $$
	Finalmente, basta notar que $(c_n)_n$ es una sucesión fundamental:
	$$ |c_{n+1} - c_n|^2 = |h_n|^2 \le \delta_1 \theta^n. $$
	Y así, $c^* := \lim_n c_n$ existe y satisface que $|{c^*}^2 + 1| = \lim_n |c_n^2 + 1| = 0$.
\end{proof}

\begin{lem}
	Sea $k$ un cuerpo y $|\,|\colon k \to [0, \infty)$ una función tal que:
	\begin{enumerate}
		\item $|a| = 0$ syss $a = 0$.
		\item $|ab| = |a| \, |b|$.
		\item Si $|a| \le 1$, entonces existe $C > 0$ tal que $|1 + a| \le C$.
	\end{enumerate}
	Entonces existe un $\lambda > 0$ tal que $|\,|^\lambda$ es un valor absoluto.
\end{lem}
\begin{proof}
	Es claro que $|\,|^\lambda$ satisface VA1 y VA2, sólo falta probar la desigualdad triangular.
	Sin perdida de generalidad, podemos suponer que $C > 1$ y así, elegimos $\lambda$ de modo que $C = 2$.

	Ahora, hay que probar la desigualdad triangular.
	En primer lugar, se comprueba que $|a + b| \le 2\max\{ |a|, |b| \}$.
	De modo que, por inducción, se comprueba que
	$$ \left| \sum_{i=1}^{2^r} a_i \right| \le 2^r \max_i\{ |a_i| \}, $$
	así que eligiendo $r$ de modo que $n \le 2^r < 2n$ vemos que
	$$ \left| \sum_{i=1}^{n} a_i \right| \le 2^r \max_i\{ |a_i| \} \le 2n\max_i \{ |a_i| \}. $$
	Empleando la desigualdad anterior con cada $a_i = 1$, tenemos $|n| \le 2n$.

	Finalmente, sean $a, b \in k$ arbitrarios y sea $n > 0$ natural, por el teorema del binomio:
	\begin{align*}
		|a + b|^n &= \left| \sum_{i=0}^{n} \binom{n}{i} a^i b^{n-i} \right|
		\le 2(n+1) \max_i \left\{ {\textstyle \left| \binom{n}{i} \right| } \, |a|^i \, |b|^{n-i} \right\} \\
			  &< 4(n+1) \max_i \left\{ {\textstyle \binom{n}{i}} \, |a|^i \, |b|^{n-i} \right\} \le 4(n+1) (|a| + |b|)^n,
	\end{align*}
	tomando raíces $n$-ésimas a ambos lados y considerando el límite cuando $n \to \infty$ se comprueba la desigualdad triangular.
\end{proof}

\begin{lem}
	Sea $k$ un cuerpo métrico completo.
	Supongamos que $t^2 + 1$ es irreducible en $k[t]$, entonces $|\,|$ posee una extensión a un valor absoluto en $k(\sqrt{-1})$.
\end{lem}
\begin{proof}
	Sea $\ui := \sqrt{-1}$ y definamos en $k(\ui)$:
	$$ \| a + \ui b \| := \sqrt{|a^2 + b^2|}. $$
	Es fácil notar que $\|\,\|$ extiende a $k$.
	Y es fácil también comprobar que satisface VA1 y VA2.
	Vamos a comprobar la desigualdad triangular: sean $\alpha, \beta \in k(\ui)$, la desigualdad es trivial si alguno fuese nulo, así que en caso contrario,
	elijamos $0 \ne \|\alpha\| \ge \|\beta\|$, entonces vemos que
	$$ \|\alpha + \beta\| \le \|\alpha\| + \|\beta\| \iff \left\| 1 + \frac{\beta}{\alpha} \right\| \le 1 + \left\| \frac{\beta}{\alpha} \right\|, $$
	donde $\gamma := \beta / \alpha$ satisface que $\|\gamma\| \le 1$.
	Así, supongamos que $\|a + \ui b\|^2 \le 1$, entonces, por el lema anterior, se cumple que $|a|, |b| \le \Delta^{-1/2}$.
	Luego vemos que
	\begin{align*}
		\|1 + (a + \ui b)\|^2 &= |(1 + a)^2 + b^2| \le 1 + |2|\,|a| + |a|^2 + |b|^2 \\
				      &\le 1 + |2| \Delta^{-1/2} + 2\Delta^{-1} =: C^2.
	\end{align*}
	Con lo que se cumple lo pedido.
	% \todo{Revisar conclusión, \cite[pág.~37]{cassels:local_fields}.}
\end{proof}

\begin{thmi}[Segundo teorema de Ostrowski]\index{teorema!de Ostrowski!II}
	Sea $k$ un cuerpo métrico arquimediano.
	Entonces existe un monomorfismo de cuerpos $\sigma\colon k \to \C$ y una constante real $\lambda > 0$ tales que
	$$ |x| = |\sigma(x)|_\infty^\lambda. $$
	Más aún, si $k$ es completo, entonces $k$ es isométricamente isomorfo a $\R$ o a $\C$.
\end{thmi}
\begin{proof}
	En particular, probaremos lo siguiente:
	\begin{displayquote}
		Si $k$ es un cuerpo métrico arquimediano completo e $\ui = \sqrt{-1} \in k$, entonces $k$ es isométricamente isomorfo a $\C$.
	\end{displayquote}
	Como $k$ es arquimediano, entonces $\car k = 0$ y contiene a $\Q$.
	Como es completo, entonces contiene a $\R$ y como contiene a $\ui$, entonces contiene a $\C$.
	Es claro que la restricción de $|\,|$ en $\C$ es equivalente al valor absoluto usual $|\,|_\infty$.

	Sea $\alpha \in k$ arbitrario.
	Luego $z \mapsto |\alpha - z|$ es una función continua de dominio $\C$ y codominio $\R$ que, veremos, alcanza un mínimo.
	Nótese que $|\alpha - z| \ge |z|_\infty^\lambda - |\alpha|$, de modo que para un radio $R > 0$ suficientemente grande
	se cumple que si $|z| > R$ entonces $|\alpha - z| \ge |\alpha|$, luego, por compacidad de la bola de radio $R$, el mínimo se alcanza en su interior,
	digamos en el complejo $b \in \C$ y definimos $\beta := \alpha - b$.
	Si $\alpha \notin \C$, entonces $\beta \ne 0$ y $|\beta| > 0$.
	Nótese que
	$$ 0 < |\beta| = \inf_{z \in \C} |\alpha - z|. $$
	Sea $c \in \C$ tal que $0 < |c| < |\beta|$. Por la propiedad superior se cumple que $|\beta - c| \ge |\beta|$.
	Notemos que
	$$ \frac{\beta^n - c^n}{\beta - c} = \prod_{\substack{\zeta^n = 1 \\ \zeta\ne 1}} (\beta - \zeta c), $$
	como $\C$ contiene a todas las raíces de la unidad, entonces $\zeta c \in \C$ y $|\beta - \zeta c| \ge |\beta|$.
	Aplicando $|\,|$ a ambos lados se obtiene que:
	$$ \frac{|\beta - c|}{|\beta|} \le \frac{|\beta^n - c^n|}{|\beta|^n} = | 1 - (c/\beta)^n | = 1 + |c/\beta|^n, $$
	el cual converge a 1 para $n$ suficientemente grande.
	Luego $|\beta - c| \le |\beta|$ y por antisimetría de $\le$ se concluye igualdad, es decir, en el complejo $b - c$ también se alcanza el mínimo.

	Sustituyendo $\beta$ por $\beta - c$ y repitiendo el proceso notamos que el mínimo siempre se alcanza en $b - nc$ para todo $n \in \N$, pero
	$$ |n| \, |c| \le |\beta| + |\beta - nc| = 2|\beta|. $$
	Luego, como $|n| > 1$ para algún $n$ y claramente también para sus potencias, se concluye que $|c| = 0$ lo cual es absurdo por elección de $|c|$.
	En conclusión, necesariamente $\alpha \in \C$.
\end{proof}

% Todo valor absoluto no arquimediano sobre un cuerpo induce una $\R$-valuación:
% En efecto, sea $0 < r < 1$ un real arbitrario, entonces $v(x) := \log_r|x|$ es una valuación.
% La propiedad V1 puede considerarse como por definición, la propiedad V2 se reduce a que el logaritmo convierte productos en sumas y la propiedad V3 es una
% traducción de la desigualdad ultramétrica.

% En el caso de los valores absolutos no arquimedianos habíamos construido los objetos
% $$ \mathfrak{o} = \{ x \in k : |x| \le 1 \}, \qquad \mathfrak{m} = \{ x \in k : |x| < 1 \}, $$
% cuyos análogos en el mundo de las valuaciones son
% $$ \mathfrak{o} = \{ x \in k : v(x) \ge 0 \}, \qquad \mathfrak{m} = \{ x \in k : v(x) > 0 \}. $$

\subsection{Valuaciones y dominios de valuación discreta}
Como señalamos anteriormente, no estudiaremos en detalle las valuaciones en éste libro, pero aún así veremos instancias de ellas:
\begin{mydef}
	Una ($\R$-)\strong{valuación}\index{valuación} sobre un anillo $A$ es una función $v\colon A \to \R\cup\{ \infty \}$ tal que:
	\begin{enumerate}[{V}1.]
		\item $v(a) = \infty$ syss $a = 0$.
		\item $v(ab) = v(a) + v(b)$ para todo $a, b\in A$.
		\item $v(a + b) = \min\{ v(a), v(b) \}$ para todo $a, b\in A$.
	\end{enumerate}
\end{mydef}

Hay una gran familia de ejemplos de valuaciones que será de nuestro interés, para lo cual requerimos el siguiente resultado de álgebra conmutativa
(cfr. \cite{Alg}, cor.~11.30):
\begin{thm}[de las intersecciones de Krull]
	Sea $A$ un dominio íntegro noetheriano y sea $\mathfrak{a} \nsl A$. Entonces:
	$$ \bigcap_{n\in\N} \mathfrak{a}^n = 0. $$
\end{thm}

\begin{prop}
	Sea $A$ un dominio íntegro noetheriano y sea $\mathfrak{p} \nsl A$ un ideal primo.
	Definamos $\nu_{\mathfrak{p}}(a) := n$ como el natural tal que $a \in \mathfrak{p}^n$ pero $a \notin \mathfrak{p}^{n+1}$.
	Entonces $\nu_{\mathfrak{p}}\colon A \to \Z\cup\{ \infty \}$ es una $\R$-valuación,
	llamada la \strong{valuación $\mathfrak{p}$-ádica}\index{valuación!padica@$\mathfrak{p}$-ádica}.
\end{prop}
Así, las valuaciones $p$-ádicas que habíamos introducido en el texto son un caso particular de la construcción anterior.

\begin{thm}
	Sea $k$ un cuerpo.
	\begin{enumerate}
		\item Si $|\,|$ es un valor absoluto no arquimediano sobre $k$, entonces para todo $r \in (0, 1)$ real se cumple que
			$v(a) := \log_r|a|$ es una $\R$-valuación.
		\item Recíprocamente, si $v$ es una $\R$-valuación sobre $k$, entonces para todo $r \in (0, 1)$ real se cumple que $|a| := r^{\phi(v(a))}$
			es un valor absoluto no arquimediano.
	\end{enumerate}
	En ambos casos, el anillo de valuación del valor absoluto $|\,|$ y de la valuación $v$ coinciden.
\end{thm}

% \begin{thm}
% 	Sea $A$ un subanillo de un cuerpo $k$. Son equivalentes:
% 	\begin{enumerate}
% 		\item Existe una valuación $v$ sobre $k$ tal que $A = \{a \in k : v(a) \ge 0\}$.
% 		\item Si $a\in k$ entonces $a \in A$ o $a^{-1} \in A$.
% 		\item $A$ es local, $k = \Frac(A)$ y todo ideal finitamente generado de $A$ es principal.
% 		\item $A$ es local y todo subanillo $A \subset B \subseteq k$ contiene algún $b \in A$ que es inversible en $B$ pero no en $A$.
% 		\item Para todo subanillo $A \subseteq B \subseteq k$ existe $\mathfrak{p} \nsl A$ primo tal que $B = A_{\mathfrak{p}}$.
% 	\end{enumerate}
% \end{thm}
% \begin{proof}
% 	$1 \implies 2$. Es claro.

% 	$2 \implies 1$. Sean $a, b \in K$.
% 	Si $aA \not\subseteq bA$, entonces $b^{-1}a \notin A$, luego $ba^{-1} \in A$ y $bA \subseteq aA$.
% 	Así, definamos $G := \{ aA : a \in k^\times \}$ y notemos que $G$ es un grupo abeliano mediante cuya operación es $(aA)(bA) := (ab)A$;
% 	para distinguir la notación aditiva emplearemos corchetes en los elementos, de modo que:
% 	$$ [aA] + [bB] := [(ab)A], \quad 0 := [1A], \quad -[aA] = [a^{-1}A]. $$
% 	Así $(G, +, \subseteq)$ es un grupo abeliano ordenado.
% 	Finalmente es claro que $v(a) := [aA]$ es una $G$-valuación en $k$ que satisface lo exigido.

% 	$3 \implies 2$. Sean $a, b \in A$ no nulos y sea $c \in A$ tal que $aA + bA = cA$ (como suma de ideales).
% 	Sean $r := a/c$ y $s := b/c$ elementos de $k$ tal que $rA + sA = A$, es decir, $rA, sA \subseteq A$ y por ende son ideales de $A$;
% 	y como $A$ es local, debe cumplirse que alguno de los elementos sea inversible (de lo contrario $rA + sA \subseteq \mathfrak{m}$),
% 	digamos que $r$ lo es, vale decir, $r^{-1} = c/a \in A$ y $sr^{-1} = b/a \in A$ (el otro caso implica que $rs^{-1} = a/b \in A$).

% 	$2 \implies 3$. Es claro.
% 	% \todo{Revisar que en un anillo de valuación todo ideal finitamente generado sea principal.}

% 	$2 \implies 5$. Sabemos que $B$ es también un anillo de valuación con $\Frac(B) = k$, y por ende tiene un único ideal maximal $\mathfrak{q}$.
% 	Sea $B' := A_{\mathfrak{q} \cap A}$ de modo que $A \subseteq B' \subseteq B$ y $B'$ es de valuación.
% 	Sea $b \in B \setminus B'$, luego $b^{-1} \in B' \subseteq B$ así que $b$ es inversible en $B$, pero $b^{-1}$ no lo es en $B'$.
% 	Luego $b^{-1}$ pertenece al ideal maximal de $B'$ que es $(\mathfrak{q} \cap A)B' = \mathfrak{q} \cap B'$ y $b^{-1} \in \mathfrak{q}$,
% 	pero los elementos de $\mathfrak{q}$ no son inversibles en $B$ lo cual es absurdo.

% 	$5 \implies 4$. Trivial.

% 	$4 \implies 2$. Sea $B$ la clausura íntegra de $A$ en $k$.
% 	Nótese que $B = A$, de lo contrario existe $b \in A$ que es inversible en $B$ y no en $A$;
% 	así que tomamos $\mathfrak{p} \nsl A$ primo que contenga a $b$ y, por el teorema del ascenso, lo levantamos a un primo $\mathfrak{q} \nsl B$ tal
% 	que $\mathfrak{p} = \mathfrak{q} \cap A$, pero $b \in \mathfrak{q}$ y $(1) = (b) = \mathfrak{q} \ne B$ lo que sería absurdo.
% 	Luego $A$ es íntegramente cerrado. 

% 	Sean $a, b\in A$ tales que $b$ es no nulo y $a/b \notin A$.
% 	Luego $C := A[a/b]$ contiene a un elemento $d \in A$ inversible en $C$ que no lo es en $A$.
% 	Por definición:
% 	$$ d^{-1} = c_0(a/b)^m + c_1(a/b)^{m-1} + \cdots + c_0, \qquad c_i \in A, $$
% 	luego $b^m = dc_0a^m + dc_1a^{m-1} b + \cdots + dc_mb^m$. Como $d \notin A^\times$, entonces está en el maximal y $1 - dc_m \in A^\times$.
% 	Definiendo $d' := d(1 - dc_m)$ y dividiendo por $a^m$ se obtiene que
% 	$$ (b/a)^m = d'c_0 + d'c_1(b/a) + \cdots + d'c_{m-1}(b/a)^{m-1}, $$
% 	donde $d'c_i \in A$, de modo que $b/a$ es entero en $A$ y $b/a \in A$.
% \end{proof}

\begin{mydef}
	Sea $k$ un cuerpo métrico.
	Decimos que un valor absoluto es \strong{discreto}\index{cuerpo!métrico!discreto},
	cuando el grupo multiplicativo $\{ |a| : a \in k^\times \} \subseteq \R^\times$ es discreto como subespacio (con la topología usual).
\end{mydef}
Como el grupo multiplicativo de un cuerpo métrico es claramente un grupo topológico, entonces basta notar que el neutro 1 está aislado, vale decir,
que existe un $\delta > 0$ tal que
$$ 1 - \delta < |a| < 1 + \delta \implies |a| = 1. $$

\begin{prop}
	El valor absoluto de un cuerpo es discreto syss en su anillo de valuación $(\mathfrak{o}, \mathfrak{m})$ se cumple que $\mathfrak{m}$ es principal.
\end{prop}
\begin{proof}
	$\impliedby$. Si $\mathfrak{m} = (\pi)$ es principal, entonces
	$$ |a| < 1 \implies a \in \mathfrak{m} \implies \exists b \in \mathfrak{o} : a = \pi b \implies |a| \le |\pi|. $$
	Por otro lado, si $|a| > 1$, entonces $|a^{-1}| < 1$ y $|a| \ge |\pi|^{-1}$. Concluimos pues $|\pi| < 1$ (por estar en $\mathfrak{m}$).
	\par
	$\implies$. Si $|\,|$ es discreto, entonces el conjunto
	$$ \{|a| : |a| < 1\} $$
	alcanza su máximo, digamos en $\pi \in \mathfrak{m}$. Luego si $|a| < 1$, entonces $|\pi^{-1}a| \le 1$ así que $b = \pi^{-1}a \in \mathfrak{o}$
	y luego $a = b\pi$ con lo que comprobamos que $\mathfrak{m} = (\pi)$.
\end{proof}
% Como $(\mathfrak{o}, \mathfrak{m})$ es un DIP, entonces $\kdim A = 1$;
% así pues, podemos hablar equivalentemente de o valores absolutos discretos o valuaciones discretas.
% Más aún, si $(\mathfrak{o}, \mathfrak{m})$ es un anillo local que es DIP, entonces 

\begin{mydefi}
	Se dice que un anillo $A$ es un \strong{dominio de valuación discreta}\index{dominio!de valuación discreta} si existe un valor absoluto discreto
	$|\,|$ sobre $\Frac A$ tal que $A$ es el anillo de valuación de $|\,|$.
	% Se dice que un anillo de valuación $(A, \mathfrak{m})$ es un \strong{dominio de valuación discreta}\index{dominio!de valuación discreta}
	% si existe una valuación discreta $v$ sobre $K := \Frac A$ tal que $A = \{ a \in K : v(a) \ge 0 \}$.
\end{mydefi}
El nombre proviene de que un anillo es de valuación discreta si existe una valuación $v$ con valores en $\Z$ que es positiva y
tal que los elementos del maximal son aquellos de valuación no nula.
% En una valuación discreta $v$ se cumple que $v(1) = 0$, y en su anillo de valuación, $v(x) = 0$ syss $x$ es inversible.
% Con ésto se puede concluir lo siguiente:
\begin{cor}
	Sea $A$ un dominio de valuación discreta $v$.
	Entonces:
	\begin{enumerate}
		\item Todos los ideales impropios de $A$ son de la forma:
			$$ \mathfrak{a}_n = \{ x \in A : v(x) \ge n \}. $$
		\item $A$ es noetheriano.
		\item $A$ es local y su único ideal maximal es $\mathfrak{a}_1$.
	\end{enumerate}
\end{cor}
\begin{proof}
	Es claro que de la primera se sigue el resto.
	Supongamos que $v(x) = v(y)$, entonces supongamos que $xy^{-1} \in A$ (por ser anillo de valuación), luego $v(xy^{-1}) = 0$,
	por lo que es inversible y luego $(x) = (y)$.
	Si $\mathfrak{b} \nsl A$ es un ideal impropio, entonces $v[ \mathfrak{b} ] \subseteq \N$, por lo que posee un mínimo $n$
	y un $x \in \mathfrak{b}$ con $v(x) = n$, luego es fácil concluir que $\mathfrak{b} = (x) = \mathfrak{a}_n$.
\end{proof}

\begin{mydefi}
	Si $(A, \mathfrak{m})$ es un dominio de valuación discreta y $\mathfrak{m} = (\pi)$, entonces decimos que $\pi$ es
	un \strong{uniformizador}\index{uniformizador} de $A$.
\end{mydefi}
\begin{figure}[!hbt]
	\centering
	\includegraphics[scale=1]{alg/metric_fld.pdf}
	\caption{Cuerpo métrico discreto.}%
	\label{fig:alg/metric_fld}
\end{figure}

\section{Análisis ultramétrico}
La siguiente proposición es fundamental y la emplearemos a lo largo de casi toda la teoría:
\begin{prop}\label{thm:bigger_wins}
	Sea $k$ un cuerpo ultramétrico.
	Si $a, b\in k$ son tales que $|a| < |b|$, entonces $|a + b| = |b|$.
\end{prop}
\begin{proof}
	Claramente $|a + b| \le |b|$ y nótese que
	\begin{equation}
		|b| = |(a + b) + (-a)| \le \max\{ |a|, |a + b| \}.
		\tqedhere
	\end{equation}
\end{proof}

\begin{mydef}
	Sea $k$ un cuerpo métrico y sea $(a_n)_{n\in\N} \subseteq k$ una sucesión. Decimos que la suma formal $\sum_{n\in\N} a_n$ converge a un valor $S$,
	si la sucesión de las sumas parciales:
	$$ S_n := \sum_{i=0}^{n} a_n $$
	converge a $S$; en cuyo caso anotaremos que $S = \sum_{n=0}^{\infty} a_n$.
\end{mydef}

\begin{prop}\label{thm:series_convergence_ultrametric}
	Sea $k$ un cuerpo ultramétrico completo. La serie $\sum_{n\in\N} a_n$ converge syss $\lim_n a_n = 0$.
\end{prop}
\begin{proof}
	$\implies$. Basta notar que
	$$ \lim_n a_n = \lim_n (s_n - s_{n-1}) = {\textstyle (\lim_n S_n) - (\lim_n S_{n-1}) } = S - S = 0. $$
	$\impliedby$. Basta notar que, por desigualdad ultramétrica, se cumple que para $m > n$
	$$ |S_m - S_n| = |a_{n+1} + a_{n+2} + \cdots + a_m| \le \max\{|a_j| : n < j \le m\}, $$
	el cual, eligiendo $n$ suficientemente grande, podemos acotar por un $\epsilon > 0$ arbitrario, y así, la sucesión $(S_n)_n$ es fundamental y converge.
\end{proof}
Ésto genera un símil con las series de $\C$, pero demuestra por qué la propiedad de ser \textit{no arquimediano} es mucho más potente (¡y útil!) en éstos casos.
Otro ejemplo es que casi siempre podemos intercambiar sumas:

\begin{prop}
	Sea $k$ un cuerpo ultramétrico completo, $(a_{ij})_{i,j} \in k$ una sucesión (doble).
	Si para todo $\epsilon > 0$ existe un $N$ tal que $|a_{ij}| < \epsilon$ si $\max\{ i, j \} \ge N$,
	entonces las series inducidas
	$$ \sum_{i=0}^{\infty} \sum_{j=0}^{\infty} a_{ij}, \qquad \sum_{j=0}^{\infty} \sum_{i=0}^{\infty} a_{ij}, $$
	convergen y coinciden.
\end{prop}
\begin{proof}
	Sea $\epsilon > 0$ arbitrario y $N$ como en la hipótesis.
	Luego, nótese que $|a_{ij}| < \epsilon$ para todo $i \ge N$, de modo que la serie $\left| \sum_{j=0}^{\infty} a_{ij} \right| < \epsilon$
	converge para todo $i \ge N$ y en el resto de índices también, luego es claro que la serie doble también.

	Finalmente, nótese que
	$$ \left| \sum_{i=0}^{N} \sum_{j=0}^{N} a_{ij} - \sum_{i=0}^{\infty} \sum_{j=0}^{\infty} a_{ij} \right| < \epsilon, $$
	y como el término de la izquierda es una suma finita, así que podemos reordenarla y así es fácil concluir que ambas sumas coinciden.
\end{proof}

\begin{mydef}
	Sea $k$ un cuerpo métrico, dada una serie formal
	$$ f(x) := f_0 + f_1 x + f_2 x^2 + \cdots \in k[[x]], $$
	se define su \strong{radio de convergencia}\index{radio!de convergencia} como:
	$$ R := \frac{1}{\limsup_n |f_n|^{1/n}} \in [0, \infty] $$
	(con el convenio usual de que si $\limsup_n |f_n|^{1/n} = 0$, entonces $R = \infty$).
	Definimos su \strong{dominio de convergencia}\index{dominio!de convergencia} como el conjunto $\mathcal{D} \subseteq k$ en donde la serie
	determinada por $f(x)$ converge.
\end{mydef}
\begin{prop}
	Sea $k$ un cuerpo ultramétrico completo, y sea $f(x) \in k[[x]]$ una serie formal con radio de convergencia $R$
	y dominio de convergencia $\mathcal{D}$.
	Entonces:
	\begin{enumerate}[(a)]
		\item Si $R = 0$, entonces $\mathcal{D} = \{ 0 \}$.
		\item Si $R = \infty$, entonces $\mathcal{D} = k$.
		\item Si $0 < R < \infty$ y $|f_n|R^n \to 0$, entonces
			$$ \mathcal{D} = \overline{B}_R(0) = \{ a \in k : |a| \le R \}. $$
		\item Si $0 < R < \infty$ y $|f_n|R^n \not\to 0$, entonces
			$$ \mathcal{D} = B_R(0) = \{ a \in k : |a| < R \}. $$
	\end{enumerate}
\end{prop}

\begin{lem}
	Sea $k$ un cuerpo ultramétrico completo.
	Sea $f(x) \in k[[x]]$ una serie formal con dominio de convergencia $\mathcal{D}$ y sea $c \in \mathcal{D}$.
	Para $m \in \N$ definamos
	\begin{equation}
		g_m := \sum_{n\ge m} \binom{n}{m} f_n c^{n-m},
		\label{eq:formal_ser_shift}
	\end{equation}
	entonces la serie formal $g(x) := g_0 + g_1x + g_2x^2 + \cdots \in k[[x]]$ tiene el mismo dominio de convergencia $\mathcal{D}$
	y para todo $b \in \mathcal{D}$ se cumple que $f(b + c) = g(b)$.
\end{lem}
\begin{proof}
	Nótese que $|g_m| \le \sup_{n\ge m}\{ |f_n c^{n-m}| \}$ y, por hipótesis, $f_n c^{n-m} \to 0$ para $m$ fijo;
	de modo que aplicando la proposición anterior es fácil verificar que los dominios de convergencia coinciden.
	Para $b \in \mathcal{D}$, vemos que
	$$ f(b + c) = \sum_{n=0}^{\infty} f_n(b + c)^n = \sum_{n=0}^{\infty} \sum_{m=0}^{n} \binom{n}{m} f_n c^{n-m} b^m, $$
	y aplicando intercambio de series dobles se concluye el enunciado.
\end{proof}
\begin{cor}
	Sea $k$ un cuerpo ultramétrico completo, y sea $f(x) \in k[[x]]$ una serie formal con dominio de convergencia $\mathcal{D}$.
	Entonces $f\colon \mathcal{D} \to k$ (como función) es continua.
\end{cor}
\begin{proof}
	Es fácil comprobar que toda serie formal (como función) es continua en el 0, y el lema anterior permite ver que en otro punto de $\mathcal{D}$
	es también una serie formal centrada en 0 y, por ende, continua.
\end{proof}

\begin{mydef}
	Sea $k$ un cuerpo ultramétrico.
	Un subconjunto $S \subseteq k$ se dice un \strong{conjunto de representantes de restos}\index{conjunto!de representantes de restos} si para toda
	clase de equivalencia $C$ de su cuerpo de restos de clases $\mathfrak{o}/\mathfrak{m}$ se cumple que existe exactamente un elemento de $C$ en $S$.
\end{mydef}

\begin{prop}\label{thm:power_series_exp}
	Sea $k$ un cuerpo ultramétrico discreto.
	Sea $(\mathfrak{o}, \mathfrak{m})$ su anillo de valuación y sea $(\pi) = \mathfrak{m}$.
	Si $S$ es un conjunto de representantes de restos y $(\pi_j)_{j\in\N}$ es una sucesión tal que cada $|\pi_j| = |\pi|^j$,
	entonces todo elemento $a \in \mathfrak{o}$ se escribe de forma única como
	$$ a = \sum_{n=0}^{\infty} a_n\pi_n, \qquad a_i \in S. $$
\end{prop}
\begin{proof}
	Como acotación, nótese que si $k$ es completo entonces las series de esa forma siempre convergen
	por la proposición~\ref{thm:series_convergence_ultrametric}, de modo que hay una correspondencia biunívoca.

	Sea $a \in \mathfrak{o}$, entonces considere la clase de equivalencia $[a] \in \mathfrak{o}/(\pi)$, nótese que, por definición de $S$, existe un
	único $a_0 \in S$ tal que $a_0 \equiv a/\pi_0 \pmod\pi$, vale decir, tal que $a - a_0\pi_0 = b_0\pi_1$ para un único $b_0 \in \mathfrak{o}$.
	Análogamente existe un único $a_1 \in S$ tal que $a_1 \equiv b_0 \pmod\pi$, vale decir, tal que $(b_0 - a_1)\pi_1 = b_1 \pi_2$ para un único
	$b_1 \in \mathfrak{o}$; luego notamos que $a = a_0\pi_0 + a_1\pi_1 + b_1\pi_2$.
	Así procedemos definiendo por recursión $(a_n)_n \subseteq S$ y $(b_n)_n \in \mathfrak{o}$.
	Luego, comprobamos que
	$$ \left| a - \sum_{n=0}^{N} a_n\pi_n \right| = |b_N\pi_N| \le |\pi|^N, $$
	donde como $|\pi| < 1$, vemos que para $N \to \infty$ el valor absoluto converge a 0, o equivalentemente, $a = \sum_{n=0}^{\infty} a_n\pi_n$.

	Para comprobar unicidad empleamos un método similar.
	Si $\sum_{n=0}^{\infty} c_n\pi^n = \sum_{n=0}^{\infty} a_n\pi^n$, entonces vemos que $c_0 \equiv a_0 \pmod \pi$ con lo que $c_0 = a_0$ por definición
	de conjunto de representantes de restos, y así vamos inductivamente comprobando.
\end{proof}
Tradicionalmente se emplea la sucesión $\pi_j := \pi^j$, pero ésta ligera generalización será útil más adelante.
Nótese que en un cuerpo ultramétrico discreto $k$ tenemos que $k = \Frac \mathfrak{o} = \mathfrak{o}[\pi^{-1}]$, de modo que todo $a \in k$
es tal que $\pi^j a \in \mathfrak{o}$ para algún $j \ge 0$, luego todo $a \in k$ admite una expansión
$$ a = \sum_{n=-j}^{\infty} a_j \pi^j = a_{-j}\pi^{-j} + \cdots + a_{-1}\pi^{-1} + a_0 + a_1\pi + \cdots $$
Series de éste estilo se dicen \textit{series de Laurent}, surgen naturalmente en el análisis complejo y fueron parte de la motivación para estudiar $\Q_p$.

\begin{thm}
	Sea $k$ un cuerpo ultramétrico discreto, completo y que su cuerpo de restos de clases es finito.
	Entonces $\mathfrak{o}$ es (topológicamente) compacto.
\end{thm}
\begin{proof}
	Para espacios métricos, ser compacto equivale a ser secuencialmente compacto (cf. \cite[Teo.~3.55]{Top}).
	Así pues, debemos comprobar que dada una sucesión $(a_j)_{j\in\N} \subseteq \mathfrak{o}$, entonces posee una subsucesión convergente.
	Sea $S$ un conjunto de representantes de restos, el cual es finito por hipótesis; por la proposición anterior se cumple que
	$$ a_j = \sum_{n=0}^{\infty} a_{jn} \pi^n, \qquad a_{j,n} \in S. $$
	Nótese que para cada $n$ hay una sucesión $(a_{j,n})_j$ de puntos en $S$, luego necesariamente hay un valor que se repite infinitamente,
	para $j = 0$ escogemos una subsucesión $\sigma(j, 0)$ tal que $a_{\sigma(j, 0), 0}$ es constante.
	Similarmente, para $j = 1$ podemos extraer una subsucesión $\sigma(j, 1)$ de $\sigma(j, 0)$ tal que $a_{\sigma(j, 0), 0}$ y $a_{\sigma(j, 1), 1}$ son
	ambas constantes.
	Y esto lo podemos hacer para todo $j$, y finalmente definimos $\eta(j) := \sigma(j, j)$, la cual fija a la coordenada $n$-ésima para todo $j \ge n$,
	luego notamos que claramente converge.
\end{proof}

\begin{cor}
	Sea $k$ un cuerpo ultramétrico. Son equivalentes:
	\begin{enumerate}
		\item $k$ es localmente compacto.
		\item $k$ es completo, discreto y su cuerpo de restos de clases es finito.
	\end{enumerate}
\end{cor}
Note que si uno quiere extender el corolario a cuerpos métricos arquimedianos, entoces ser localmente compacto equivale a ser completo por el teorema de Ostrowski.

\begin{thm}
	Todo cuerpo ultramétrico es hereditariamente disconexo.
\end{thm}
\begin{proof}
	Sea $(k, |\,|)$ un cuerpo ultramétrico.
	Si $|\,|$ es trivial, entonces $k$ es un espacio discreto así que se satisface.
	Si $|\,|$ no es trivial, basta probar que posee una base de la topología formada por conjuntos abiertos y cerrados;
	en particular, probaremos que las bolas abiertas son siempre cerradas.

	Sea $\epsilon > 0$ y $a \in k$.
	Sea $b \in B_\epsilon(a)^c$, entonces $B_\epsilon(b) \cap B_\epsilon(a) = \emptyset$ pues si $c \in B_\epsilon(b)$ entonces
	$$ |c - b| < \epsilon \le |b - a| \implies |c - a| = |b - a| \ge \epsilon. $$
	Así pues, $B_\epsilon(b) \subseteq B_\epsilon(a)^c$, por lo que $B_\epsilon(a)^c$ es abierto y $B_\epsilon(a)$ es cerrado.
\end{proof}

\subsection{El teorema de Strassmann y aplicaciones}
\begin{thmi}[Teorema de Strassmann]\index{teorema!de Strassmann}
	Sea $k$ un cuerpo ultramétrico completo, y sea $f(x) \in k[[x]]$ una serie formal no nula.
	Supongamos que $f_n \to 0$ (o equivalentemente, tal que $f$ converge en $\mathfrak{o}$), entonces $f(a) = 0$ para finitos $a \in \mathfrak{o}$'s.
	Más aún, hay a lo más $N$ raíces $a \in \mathfrak{o}$, donde $N$ es el natural tal que $|f_N| > |f_n|$ para todo $n > N$.
\end{thmi}
\begin{proof}
	Lo haremos por inducción sobre $N$.

	Si $N = 0$, entonces no puede darse que $f(a) = 0$ para algún $a \in \mathfrak{o}$, pues
	$$ f_0 = - \sum_{n=1}^{\infty} f_na^n, $$
	pero
	$$ \left| \sum_{n=1}^{\infty} f_na^n \right| \le \max_{n\ge 1} |f_n a^n| \le \max_{n\ge 1} |f_n| < |f_0|. $$
	Si $N > 0$, entonces sea $a \in \mathfrak{o}$ tal que $f(a) = 0$ y sea $b \in \mathfrak{o}$, entonces
	$$ f(b) = f(b) - f(a) = \sum_{n=1}^{\infty} f_n(b^n - a^n) = (b - a) \sum_{n=1}^{\infty} \sum_{j < n} f_n b^j a^{n-j-1}, $$
	luego, intercambiando sumatorias y definiendo:
	$$ g(x) := \sum_{j=0}^{\infty} g_j x^j, \qquad g_j := \sum_{r=0}^{\infty} f_{j+1+r} a^r; $$
	se tiene que $f(b) = (b - a) g(b)$.
	En particular se puede verificar que:
	\begin{itemize}
		\item $|g_{N-1}| = |f_N|$.
		\item $|g_j| \le |f_N|$ para todo $j$ y $|g_j| < |f_N|$ para $j > N - 1$.
	\end{itemize}
	Luego, por hipótesis inductiva, $g$ posee a lo más $N - 1$ ceros en $\mathfrak{o}$ y así, $f$ posee a lo más $N$ ceros en $\mathfrak{o}$.
\end{proof}

\begin{cor}
	Sea $k$ un cuerpo ultramétrico completo.
	Dos series formales $f(x), g(x) \in k[[x]]$ que convergen en $\mathfrak{o}$ son iguales syss $f(a) = g(a)$ para infinitos $a \in \mathfrak{o}$.
\end{cor}
\begin{cor}
	Sea $k$ un cuerpo ultramétrico completo de $\car k = 0$.
	Sea $f(x) \in k[[x]]$ una serie formal que converge en $\mathfrak{o}$ tal que $f(x + d) = f(x)$ para algún $d \in \mathfrak{o}_{\ne 0}$,
	entonces $f$ es constante.
\end{cor}
\begin{proof}
	Basta notar que $f(x) - f(0)$ tiene infinitos ceros en $x = nd \in \mathfrak{o}$ para todo $n \in \Z$.
\end{proof}

Un ejemplo del cómo emplear el teorema de Strassmann radica en las sucesiones por recursión lineal.

Un ejemplo es la siguiente aplicación detallada en \citeauthor{alter75diophantine}~\cite{alter75diophantine}:
\begin{prob}
	Este problema tiene por objetivo resolver $x^2 + 11z^2 = 3^n$.
	\begin{enumerate}
		\item Considere la sucesión
			$$ a_0 := 0, \qquad a_1 := 1, \qquad a_{n+2} := a_{n+1} - 3a_n, $$
			demuestre que
			$$ a_n = \frac{1}{\sqrt{-11}}(r^n - r'^n). $$
		\item Demuestre que $a_{n+m} = a_{n+1}a_m - 3a_na_{m-1}$.
		\item Demuestre que $(a_n; a_m) = \pm a_{(n; m)}$.
		\item Concluya que la ecuación diofántica $a_n = t$ con $t \in \Z$ tiene a lo más una solución para un $t$ fijo,
			salvo si $t = 1$, en cuyo caso, admite tres soluciones ($n \in \{ 1, 2, 5 \}$).
		\item Emplee lo anterior para concluir que $x^2 + 11z^2 = 3^n$ tiene a lo más una solución con $x \ge 0$.
	\end{enumerate}
\end{prob}

\addtocategory{article}{alter75diophantine}

\subsection{Lema de Hensel y anillos henselianos}
El llamado lema de Hensel es una de las herramientas más importantes en teoría de cuerpos de clases y ha probado ser de extrema utilidad.
En primer lugar introducimos un glosario de las distintas versiones en las que se puede encontrar el lema de Hensel:
\begin{thm}
	Sea $(A, \mathfrak{m}, k)$ un anillo local y fijemos $|\,| := |\,|_{\mathfrak{m}}$ el valor absoluto $\mathfrak{m}$-ádico.
	Son equivalentes:
	\begin{enumerate}
		\item Sea $f(x) \in A[x]$ mónico.
			Si existe $a_0$ tal que $f(a_0) \equiv 0 \pmod{\mathfrak{m}}$,
			entonces existe $a \equiv a_0 \pmod{\mathfrak{m}}$ tal que $f(a) = 0$.
		\item Sea $f(x) \in A[x]$ mónico.
			Si existe $a_0$ tal que $|f(a_0)| < 1$ y $|f'(a_0)| = 1$,
			entonces existe $a \equiv a_0 \pmod{\mathfrak{m}}$ tal que $f(a) = 0$.
		\item Sea $f(x) \in A[x]$ mónico.
			Si existe $a_0$ tal que $|f(b)| < |f'(b)|^2$,
			entonces existe un único $a \in A$ tal que $f(a) = 0$ y $|a - b| < |f'(b)|$.
		\item Sea $f(x) \in A[x]$ mónico.
			Si $f \equiv g_0h_0 \pmod{\mathfrak{m}}$ con $g_0$ mónico y $g_0, h_0 \in k[x]$ coprimos (en $k[x]$),
			entonces existen $g, h \in A[x]$ tales que $f = gh$, $g \equiv g_0$ y $h \equiv h_0 \pmod{\mathfrak{m}}$.
	\end{enumerate}
\end{thm}
% \begin{proof}
% 	...
% \end{proof}

\begin{mydef}
	Un anillo local $(A, \mathfrak{m}, k)$ que satisface lo anterior, se dice un \strong{anillo henseliano}\index{anillo!henseliano}.
	Un cuerpo métrico $(K, |\,|)$ se dice \strong{henseliano}\index{cuerpo!henseliano} si o bien es arquimediano y completo, o bien es ultramétrico
	y su anillo de valuación es henseliano.
\end{mydef}

\begin{thmi}[Lema de Hensel]\index{lema!de Hensel}
	Todo cuerpo métrico completo $k$ es henseliano.
	% Sea $k$ un cuerpo ultramétrico completo, con anillo de valuación $\mathfrak{o}$ y sea $f(x) \in \mathfrak{o}[x]$.
	% Suponga que existe un $a_0 \in \mathfrak{o}$ tal que $|f(a_0)| < |f'(a_0)|^2$, donde $f'(x)$ es el polinomio derivado de $f(x)$.
	% Entonces $f$ posee una raíz en $\mathfrak{o}$.
	% Más aún, existe un único $a \in \mathfrak{o}$ tal que:
	% $$ f(a) = 0, \qquad |a - a_0| < \frac{|f(a_0)|}{|f'(a_0)|}. $$
\end{thmi}
\begin{proof}
	Sean $f_j(x) \in \mathfrak{o}[x]$ polinomios tales que
	$$ f(x + y) = f(x) + f_1(x)y + f_2(x)y^2 + \cdots \in \mathfrak{o}[x, y], $$
	donde los $f_i$'s vendrán dados por expandir un binomio de Newton en cada monomio original.
	Se puede comprobar que $f_1(x) = f'(x)$.
	Luego, por el enunciado, existe $b_0 \in \mathfrak{o}$ tal que
	$$ f(a_0) + b_0 f_1(a_0) = 0, $$
	luego, definamos $a_1 := a_0 + b_0$ y notemos que por desigualdad ultramétrica
	$$ |f(a_1)| = |f(a_0 + b_0)| \le \max_{j\ge 2} |f_j(a_0) b_0^j|, $$
	como $f_j(a_0) \in \mathfrak{o}$ entonces $|f_j(a_0)| \le 1$, luego
	$$ |f(a_1)| \le |b_0|^2 = \frac{|f(a_0)|^2}{|f'(a_0)|^2} < |f(a_0)|, $$
	además de que $|b_0| < |f'(a_0)|$.
	Del mismo modo se nota que
	$$ |f'(a_1) - f'(a_0)| \le |b_0| < |f'(a_0)|. $$
	Luego se cumple que $|f'(a_1)| = |f'(a_0)|$.
	Ahora podemos volver a elegir un $b_1$ con las mismas propiedades, en particular, notando que $|f(a_1)| < |f(a_0)| \le |f'(a_0)|^2 = |f'(a_1)|^2$,
	y así recursivamente comprobamos que
	$$ |f(a_{n+1})| \le |b_n|^2 = \frac{|f(a_n)|^2}{|f'(a_n)|^2} = \frac{|f(a_n)|^2}{|f'(a_0)|^2}, $$
	como $|f'(a_0)|^2$ es solo una constante, entonces vemos que $|f(a_n)| \to 0$ y luego, por la igualdad superior, $b_n \to 0$, de modo que $(a_n)_n$
	es una sucesión fundamental que converge a una raíz de $f$ (¿por qué está en $\mathfrak{o}$?).
\end{proof}

Veamos algunas aplicaciones del lema de Hensel:
\begin{thm}
	Sea $p \in \Z$ primo:
	\begin{itemize}
		\item Si $p \ne 2$:
			Sea $b \in \Z_p$ tal que $|b| = 1$ (i.e., $p \nmid b$).
			Supongamos que $b$ es un residuo cuadrático módulo $p$, entonces $b$ es un cuadrado en $\Z_p$.
			% existe $a_0 \in \Z_p$ tal que
			% $|a_0^2 - b| < 1$ (i.e., $b$ es residuo cuadrático módulo $p$).
			% Entonces existe algún $a \in \Z_p$ tal que $b = a^2$.
		\item Si $p = 2$:
			Sea $b \in \Z_2$ tal que $b \equiv 1 \pmod 8$, entonces existe algún $a \in \Z_p$ tal que $b = a^2$.
	\end{itemize}
\end{thm}
\begin{proof}
	En ambos casos se emplea el lema con $f(x) := x^2 - b$. Nótese que $f'(x) = 2x$.
	El ser residuo cuadrático equivale a que existe $a_0$ con $a_0^2 \equiv b \pmod p$, de modo que $|f(a_0)| < 1 = |2a_0|^2 = |f'(a_0)|^2$ (pues $|2| = 1$).
	Para el segundo caso evaluamos en $a_0 = 1$ y se tiene que $|f(1)| \le 2^{-3} < 2^{-2} = |2|^2$.
\end{proof}
Nótese que la condición de que $b \equiv 1 \pmod 8$ nos dice que $b$ es un residuo cuadrático módulo $2^n$ para todo $n > 0$.

\begin{cor}
	Sea $p \in \Z$ primo:
	\begin{itemize}
		\item Si $p \ne 2$:
			El grupo $\Q_p^\times / (\Q_p^\times)^2 \cong \Z/2\Z \times \Z/2\Z$,
			y representantes de las clases laterales son $1, p, c, cp$ con $c$ un residuo no cuadrático módulo $p$.
			En consecuencia, $\Q_p$ tiene exactamente 3 extensiones cuadráticas.
		\item Si $p = 2$:
			El grupo $\Q_2^\times / (\Q_2^\times)^2 \cong (\Z/2\Z)^3$,
			y representantes de las clases laterales son los generados por $-1, 5, 2$.
			En consecuencia, $\Q_2$ tiene exactamente 7 extensiones cuadráticas.
	\end{itemize}
\end{cor}
\begin{proof}
	Sea $x$ un número no nulo libre de cuadrados en $\Z_p$ con $p \ne 2$.
	Aplicamos reducción módulo $p$: si $x \not\equiv 0 \pmod p$, entonces $x$ es una unidad y existe $y \in \Z_p$ tal que
	$xy = c$, luego $y = c/x$ es un residuo cuadrático no nulo módulo $p$ (pues es división de dos residuos no cuadradáticos) y por lo tanto
	$[c] = [x] \in \Q_p^\times / (\Q_p^\times)^2$.
	Si $x \equiv 0 \pmod p$, entonces como $p^2 \nmid x$ se tiene que $x/p \not\equiv 0 \pmod p$ luego, o bien $x/p$ es un cuadrado
	o no, i.e., o bien $[x] = [p]$ o bien $[x] = [cp] \in \Q_p^\times / (\Q_p^\times)^2$.

	Para $p = 2$ siga un procedimiento similar.
	Otra manera de verlo es que $\Q_2^\times / (\Q_2^\times)^2$ es un grupo abeliano de ocho elementos y tal que $[x^2] = 1$ para todo elemento del grupo.
\end{proof}

\begin{prop}
	Sea $p \in \Z$ primo:
	\begin{itemize}
		\item Si $p \ne 3$:
			Sea $b \in \Z_p$ con $|b| = 1$.
			Supongamos que $b$ un residuo cúbico módulo $p$, entonces $b$ es un cubo en $\Z_p$.
		\item Si $p = 3$:
			Sea $b \in \Z_3$ con $|b| = 1$.
			Se cumple que $b \equiv \pm 1 \pmod 9$ syss $b$ es un cubo en $\Z_3$.
	\end{itemize}
\end{prop}
\begin{proof}
	En éste caso empleamos el lema de Hensel con $f(x) := x^3 - b$.
	Veamos el caso de $p = 3$. La condición de que $b \equiv \pm 1 \pod 9$ se traduce en que existe $e \in \{ 0, \pm 1 \}$
	tal que $b \equiv \pm (1 + 3e)^3 \pmod{27}$.
	Luego con $a_0 := \pm(1 + 3e)^3$ vemos que $|f(a_0)| \le 3^{-3}$ y que $|f'(a_0)| = |3| \, |a_0^2| = 3^{-1}$.
\end{proof}

Aquí vemos un ejemplo del <<comportamiento local>> de los $p$-ádicos.
Para hablar de ecuaciones diofantinas conviene admitir la siguiente terminología:
\begin{mydef}
	Se dice que $\Q$ es un \strong{cuerpo global} y que sus compleciones $\R, \Q_p$ son \strong{cuerpos locales}\index{cuerpo!local}.
\end{mydef}
Como los cuerpos locales contienen al cuerpo global, vemos la siguiente observación a modo de eslógan:
\begin{displayquote}
	\itshape
	La existencia de soluciones globales implica la existencia de soluciones locales.
\end{displayquote}
Una pregunta interesante sería tener una especie de recíproco, vale decir, ¿existen soluciones globales si existen soluciones locales en todas partes?
La respuesta en general es que no:

\begin{prob}
	La ecuación diofántica:
	$$ (x^2 - 2)(x^2 - 17)(x^2 - 34) = 0 $$
	tiene soluciones locales en todas partes, pero no soluciones globales.
\end{prob}
\begin{proof}
	Es claro que la ecuación no posee soluciones globales.
	Por otro lado, sabemos que $2$ es un cuadrado en $\Q_{17}$ y que $17$ es un cuadrado en $\Q_2$ pues $17 \equiv 1 \pmod 8$.
	Finalmente, nótese que en $\Z_p$ con $p \notin \{ 2, 17 \}$ siempre se cumple que $2, 17, 34$ son inversibles (tienen $|\,| = 1$)
	y siempre alguno es un cuadrado módulo $p$ (basta expresarlos como potencias de una raíz primitiva).
\end{proof}

\section{Extensiones del valor absoluto}
En ésta sección veremos como dada una extensión de cuerpos $L/k$ con $k$ (ultra)métrico, podemos asignarle un valor absoluto a $L$
que extienda al de $k$.
Para ello, contemplaremos dos posibilidades.

\subsection{Extensiones trascendentes}%
\label{sec:val_trasc_ext}

\begin{thm}\label{thm:abs_val_over_kx}
	Sea $(k, |\,|)$ un cuerpo ultramétrico y sea $c > 0$ arbitrario.
	Para $f(x) = f_0 + f_1x + \cdots + f_nx^n \in k[x]$ defínase
	$$ \|f\| = \|f\|_c := \max_j\{ |f_j| c^j \}. $$
	Para $h(x) = f(x)/g(x) \in k(x)$ extendamos $\|h\| := \|f\| / \|g\|$.
	Entonces $\|\,\|$ es un valor absoluto sobre $k(x)$ que extiende a $|\,|$.
	\nomenclature{$\|\,\|_c$}{Valor absoluto sobre $k(x)$ dado por $\| \sum_{j=0}^{n} a_jx^j \| := \max_j\{ |a_j|c^j \} $}
\end{thm}
\begin{proof}
	Primero veamos algunos axiomas para polinomios $f(x) \in k[x]$.
	Evidentemente $\|f\| = 0$ syss $f = 0$, y la desigualdad ultramétrica se comprueba,
	pues si $f(x) := \sum_{j\ge 0} f_jx^j, g(x) := \sum_{j\ge 0} g_jx^j$, entonces
	$$ \|f+g\| = \max_j\{ |f_j + g_j|c^j \} \le \max_j\{ \max\{ |f_j|, |g_j| \}c^j \} = \max\{ \|f\|, \|g\| \}. $$
	Claramente, $\|fg\| \le \|f\| \, \|g\|$, veamos que se alcanza igualdad:
	Sean $I, J \ge 0$ los naturales tales que
	\begin{align*}
		\|f_I x^I\| &= \|f\|, & \forall i \le I \quad \|f_ix^i\| &< \|f\|, \\
		\|g_J x^J\| &= \|g\|, & \forall j \le J \quad \|g_jx^j\| &< \|g\|,
	\end{align*}
	El coeficiente de $x^{I+J}$ en $f\cdot g$ es $ \sum_{i+j=I+J} f_ig_j $.
	Separamos por casos:
	\begin{enumerate}[(a)]
		\item \underline{Si $i < I$:} Entonces $\|f_ix^i\| < \|f\|$, o equivalentemente, $|f_i| < c^{-i} \|f\|$ y
			$\|g_jx^j\| \le \|g\| \iff |g_j| \le c^{-j}\|g\|$. Luego:
			\begin{equation}
				|f_i g_j| < c^{-i-j}\|f\| \, \|g\| \iff \|f_ig_j x^{I+J}\| < \|f\| \, \|g\|.
				\label{eq:trascendence_av_ineq}
			\end{equation}

		\item \underline{Si $j < J$:} Se concluye \eqref{eq:trascendence_av_ineq} análogamente.
		\item \underline{Si $(i, j) = (I, J)$:} Entonces $|f_I| = c^{-I}\|f\|$ y $|g_J| = c^{-J}\|g\|$, por lo que
			$$ | f_I g_J | = c^{-I-J} \|f\|\,\|g\| \iff \|f_Ig_J x^{I+J}\| = \|f\| \, \|g\|. $$
	\end{enumerate}
	En conclusión:
	$$ \left\| \sum_{i+j = I+J} f_ig_j x^{I+J} \right\| = \|f\| \, \|g\|, $$
	de modo que $\|f\cdot g\| \ge \|f\| \, \|g\|$, y luego $\|fg\| = \|f\|\,\|g\|$ (VA2).

	Por VA2 se puede comprobar que la extensión a $k(x)$ está bien definida y es fácil ver que es un valor absoluto que extiende a $|\,|$.
\end{proof}

\begin{cor}
	Sea $(k, |\,|)$ un cuerpo ultramétrico.
	Sea $\vec x := (x_1, \dots, x_n)$ una tupla de indeterminadas, y sea $\vec c := (c_1, \dots, c_n)$ una tupla de reales $>0$.
	Para $f(\vec x) = \sum_{\alpha} f_\alpha \vec x^\alpha \in k[\vec x]$, en notación multiíndice, defínase
	$$ \|f\|_{\vec c} := \max_\alpha \{ |f_\alpha| \vec c^\alpha \}. $$
	Y para $h := f/g \in k(\vec x)$ extendamos $\|h\|_{\vec c} := \|f\|/\|g\|$.
	Entonces $\|\,\|_{\vec c}$ es un valor absoluto sobre $k(\vec x)$ que extiende a $|\,|$.
\end{cor}

\begin{prop}\label{thm:metric_division_alg}
	Sea $k$ un cuerpo ultramétrico, $c > 0$ arbitrario y sea $\|\,\| := \|\,\|_c$.
	Sean $R(x) \in k[x]$, y $G(x) = \sum_{j=0}^{m} G_jx^j \in k[x]$ no nulo tal que $\|G\| = \|G_mx^m\|$.
	Sean $L(x), M(x) \in k[x]$ tales que
	$$ R(x) = L(x)G(x) + M(x), \qquad M = 0 \vee \deg M < m. $$
	Entonces $\|L\|\,\|G\| \le \|R\|$ y $\|M\| \le \|R\|$.
\end{prop}
\begin{proof}
	Sea $R(x) = R_0 + R_1x + \cdots + R_nx^n$ con $R_n \ne 0$ y sea $L(x) = L_0 + \cdots + L_{n-m}x^{n-m}$.
	Los coeficientes de $L(x)$ son tales que satisfacen el sistema de ecuaciones lineales
	$$ G_m L_{n-m-j} + G_{m-1} L_{n-m-j+1} + \cdots + G_{m-j}L_{n-m} = R_{n-j}, $$
	donde $j \le \min\{ m, n-m \}$.
	Empleando el hecho de que $\|G\| = \|G_mx^m\|$ se prueba, por inducción, que
	$$ \| L_{n-m-j}x^{n-m-j} \| \, \|G\| \le \|R\|, $$
	de lo que se comprueba que $\|L\| \, \|G\| \le \|R\|$.
	Finalmente, $M = R - LG$ implica que $\|M\| \le \|R\|$ (por desigualdad ultramétrica).
\end{proof}

\begin{thm}[lema de Gauss]
	Sea $k$ un cuerpo ultramétrico y sea $\vec x := (x_1, \dots, x_n)$ una tupla de indeterminadas.
	Si $f(\vec x) \in \mathfrak{o}[\vec x]$ es un producto de polinomios no constantes en $k[\vec x]$,
	entonces también es un producto de polinomios no constantes en $\mathfrak{o}[\vec x]$.
\end{thm}
\begin{proof}
	Considere la extensión $\|\,\| := \|\,\|_{(1, \dots, 1)}$ en $k(\vec x)$.
	Nótese que el grupo de valores de $\|\,\|$ coincide con el de $|\,|$ y también:
	$$ \mathfrak{o}[\vec x] = \{ f \in k[\vec x] : \|f\| \le 1 \}. $$
	Si $f = gh$ para algunos $g, h\in k[\vec x]$ no constantes, entonces existe $a \in k$ tal que $|a| = \|g\|$,
	luego sustituyendo $g$ por $a^{-1}g$ podemos suponer que $\|g\| = 1$ y así:
	$$ 1 \ge \|f\| = \|g\| \, \|h\| = \|h\|, $$
	luego, $g, h \in \mathfrak{o}[\vec x]$ como se quería probar.
\end{proof}
\begin{cor}
	Sea $k$ un cuerpo ultramétrico.
	Si $f \in \mathfrak{o}[\vec x]$ es irreducible en $\mathfrak{o}[\vec x]$, entonces también lo es en $k[\vec x]$.
\end{cor}
Éste teorema lo probamos en más generalidad con el nombre de \textit{criterio de irreducibilidad de Gauss} en \cite[Teo.~2.86]{Alg}.
Gauss originalmente formuló su criterio en el siguiente contexto:

\begin{prop}[lema de Gauss]
	Sea $f(\vec x) \in \Z[\vec x]$.
	Si $f$ es un producto de polinomios no constantes en $\Q[\vec x]$, entonces lo es en $\Z[\vec x]$.
\end{prop}
\begin{proof}
	Sea $f = gh$ para algunos $g, h \in \Q[\vec x]$ no constantes.
	Luego, $g, h \in \Z_p[\vec x]$ para todo primo $p \notin S$, donde $S$ es un conjunto finito (¿cuál conjunto?).
	Nótese que $\bigcap_{p} \Z_p = \Z$, por lo que si $S = \emptyset$, entonces está probado.

	Si $S \ne \emptyset$, entonces para todo $p \in S$ la demostración del lema de Gauss nos da que existe $m(p) \in \Z$
	tal que $p^{m(p)}g \in \Z_p[\vec x]$ y $p^{-m(p)}h \in \Z_p[\vec x]$.
	Así, definiendo
	$$ r := \prod_{p\in S} p^{m(p)}, $$
	se obtiene que $rg, r^{-1}h \in \Z_p[\vec x]$ para todo $p$ primo; por lo que $rg, r^{-1}h \in \Z[\vec x]$.
\end{proof}

\begin{mydef}
	Sea $(\mathfrak{o}, \mathfrak{m})$ un anillo de valuación.
	Un polinomio $f(x) := a_0 + a_1x + \cdots + a_nx^n \in \mathfrak{o}[x]$ se dice \strong{de Eisenstein}\index{polinomio!de Eisenstein} si
	$$ a_n \in \mathfrak{o}^\times = \mathfrak{o \setminus m}, \qquad \forall j < n \quad a_j \in \mathfrak{m}, \qquad a_0 \notin \mathfrak{m}^2. $$
\end{mydef}

\begin{thm}[criterio de irreducibilidad de Eisenstein]
	Sea $k$ un cuerpo ultramétrico discreto de anillo de valuación $\mathfrak{o}$.
	Todo polinomio de Eisenstein en $\mathfrak{o}[x]$ es irreducible en $k[x]$.
	% Sea $k$ un cuerpo ultramétrico discreto y sea $\pi$ un uniformizador de $\mathfrak{o}$.
	% Sea $f(x) = \sum_{j=0}^{n} a_jx^j \in \mathfrak{o}[x]$ tal que
	% $$ |a_0| = |\pi|, \qquad \forall j < n \quad |a_j| < 1, \qquad |a_n| = 1. $$
	% Entonces $f$ es irreducible en $k[x]$.
\end{thm}

\subsection*{Polígonos de Newton}
Sea $k$ un cuerpo ultramétrico y sea $f(x) = a_0 + a_1x + \dots + a_nx^n \in k[x]$ con $a_0 \ne 0 \ne a_n$.
Para obtener su polinomio de Newton, graficaremos los puntos:
$$ P(j) := (j, -\log|a_j|) \in \R^2. $$
y consideraremos su clausura convexa $\Pi(f)$ a la que llamamos el \strong{polígono de Newton}\index{polígono!de Newton}.
Dicho polígono está compuesto por varios segmentos $\lambda_1, \dots, \lambda_s$ de pendientes crecientes,
donde $\lambda_i$ une los puntos $P(m_{i-1}), P(m_i)$ y su pendiente es:
$$ \gamma_i := \frac{-\log|a_{m_i}| + \log|a_{m_{i-1}}|}{m_i - m_{i-1}}. $$
donde $0 = m_0 < m_1 < \cdots < m_s = n$ y $\gamma_1 < \cdots < \gamma_s$.
En éste sentido, diremos que $f$ es de \strong{tipo}\index{tipo}:
\begin{equation}
	(m_1, \gamma_1; m_2 - m_1, \gamma_2; \dots; m_s - m_{s-1}, \gamma_s)
	\label{eq:newton_type_pol}
\end{equation}
Un polinomio del tipo $(n, \gamma)$ se dice un \strong{polinomio Newton-puro}\index{polinomio!Newton-puro}.%
\footnote{La terminología \textit{polinomio de tipo ...} o \textit{Newton-puro} son originales
de \citeauthor{cassels:local_fields}~\cite[100]{cassels:local_fields}, quien señala que son expresiones no estándar.}

\begin{figure}[!hbtp]
	\centering
	\includegraphics{num-th/newton_polygon.pdf}
	\caption{Polígono de Newton de $f(x) = 4x^5 - \frac 14 x^4 + 7x^3 + 5x^2 - 6x - 2$.}%
	\label{fig:newton-polygon}
\end{figure}

\begin{lem}\label{lem:newton_norms}
	Sea $k$ un cuerpo ultramétrico.
	Sea $f(x) = a_0 + a_1x + \cdots + a_nx^n \in k[x]$ con $a_0 \ne 0 \ne a_n$ un polinomio de tipo \eqref{eq:newton_type_pol}.
	Definiendo $c = \exp(\gamma_s)$ y $\|\,\| := \|\,\|_c$.
	Entonces $\|f\| = \|a_jx^j\|$ para $j \in \{ m_{s-1}, m_s \}$ y
	$$ \left\| f - \sum_{m_{s-1} \le j \le m_s} a_jx^j \right\| < \|f\|. $$
\end{lem}
\begin{proof}
	Basta notar que si $\ell < j \le n$, entonces la condición de que $\|a_jx^j\| \ge \|a_\ell x^\ell\|$ equivale a que
	$$ \frac{-\log|a_j| + \log|a_\ell|}{j - \ell} \le \gamma_s, $$
	donde el lado izquierdo representa la pendiente del segmento que une los puntos $P(\ell)$ y $P(j)$ del polígono de Newton de $f$.
	Finalmente aplicamos la elección de $\gamma_s$ para concluir.
\end{proof}
% \todo{Revisar éste lema.}
Ejercicio para el lector: ¿por qué requerimos que $k$ sea ultramétrico?

\begin{lem}
	Sea $k$ un cuerpo ultramétrico.
	Sean $f, g \in k[x]$ dos polinomios Newton-puros de pendiente común $\gamma$, entonces $f\cdot g$ es Newton-puro de pendiente $\gamma$.
\end{lem}
\begin{proof}
	Fijemos $c := e^\gamma$ y $\|\,\| := \|\,\|_c$.
	Si $f(x) = a_0 + \cdots + a_nx^n$ y $g(x) = b_0 + \cdots + b_mx^m$ con $a_n \ne 0 \ne b_m$, entonces por el lema anterior, vemos que
	$$ \|f\| = |a_0| = \|a_nx^n\|, \qquad \|g\| = |b_0| = \|b_mx^m\|, $$
	luego $\|fg\| = |a_0b_0| = \|a_nb_m x^{n+m}\|$ y concluimos por el lema anterior que $fg$ es Newton-puro, y es fácil verificar que la pendiente
	es también la misma.
\end{proof}

\begin{lem}\label{lem:adding_newton_types}
	Sea $k$ un cuerpo ultramétrico.
	Sea $f \in k[x]$ un polinomio de tipo \eqref{eq:newton_type_pol} y $g \in k(x)$ un polinomio Newton-puro de tipo $(N, \gamma)$.
	Si $\gamma_s < \gamma$, entonces $f\cdot g$ es de tipo
	$$ (m_1, \gamma_1; m_2 - m_1, \gamma_2; \dots; m_s - m_{s-1}, \gamma_s; N, \gamma). $$
\end{lem}
\begin{proof}
	Sea $c := e^{\gamma_s}$ y $d := e^{\gamma}$, aquí trabajaremos con ambos valores absolutos por separado.
	Como $\gamma > \gamma_s$ vemos que $\|g(x) - b_0\|_c < \|g(x)\|_c$. Luego, por el lema~\ref{lem:newton_norms},
	$$ \left\| f\cdot g - b_0 \sum_{m_{s-1} \le j \le m_s} a_jx^j \right\|_c < \|f\cdot g\|_c. $$
	Similarmente, $\|f(x) - a_nx^n\|_d < \|f(x)\|_d$, luego
	$$ \left\| f\cdot g - a_nx^n g(x) \right\|_d < \|f\cdot g\|_d. $$
	Se puede verificar que las condiciones del lema~\ref{lem:newton_norms} implican que $fg$ es del tipo descrito.
	\todo{Verificar conclusión.}
\end{proof}

\begin{lem}
	Sea $k$ un cuerpo ultramétrico henseliano (e.g., completo) y $\|\,\| = \|\,\|_c$.
	Sea $f(x) = \sum_{j=0}^{n} a_jx^j \in k[x]$ con $a_n \ne 0$.
	Si existe $m$ con $0 < m < n$ tal que
	$$ \|f\| = \|a_nx^n\|, \qquad \forall j > m\quad \|a_jx^j\| < \|f\|. $$
	Entonces $f = gh$ donde $g, h \in k[x]$ tienen grados $m, n-m$ resp.
\end{lem}
\begin{proof}
	...
	\todo{Completar demostración, vid. \cite[103-104]{cassels:local_fields}.}
\end{proof}

\begin{cor}\label{thm:irred_implies_pure}
	Sea $k$ un cuerpo ultramétrico henseliano.
	Si $f(x) \in k[x]$ es irreducible, entonces es Newton-puro.
\end{cor}
\begin{proof}
	Basta notar que si $f$ no es Newton-puro, entonces podemos encontrar un $m$ como en el lema anterior y así ver que es reducible.
\end{proof}

\begin{thm}[del polígono de Newton]\label{thm:newton_polygon}
	Sea $k$ un cuerpo ultramétrico henseliano.
	Sea $f(x) \in k[x]$ un polinomio de tipo \eqref{eq:newton_type_pol}, entonces se puede escribir como
	$$ f(x) = g_1(x) \cdots g_s(x), $$
	donde cada $g_j \in k[x]$ es Newton-puro de tipo $(m_j - m_{j-1}, \gamma_j)$.
\end{thm}
\begin{proof}
	Nótese que $f = \prod_{j=1}^{n} h_j$, donde cada $h_j$ es irreducible y, por lo tanto, Newton-puro.
	Podemos definir los $g_\ell$'s como los productos de $h_j$'s que tienen la misma pendiente $\delta_\ell$, de modo que $g_\ell$
	es de tipo $(q_\ell, \delta_\ell)$; lo que dará que $f$ se escribe como producto de polinomios Newton-puros de distintas pendientes.
	Empleando repetidas veces el lema~\ref{lem:adding_newton_types} se obtiene que el producto de $g_\ell$, quizá tras reordenar, es
	$$ (q_1, \delta_1; q_2, \delta_2; \dots; q_r, \delta_r), $$
	y por ello, vemos que $r = s$, $q_j = m_j - m_{j-1}$ y $\delta_j = \gamma_j$.
\end{proof}

\subsection{Extensiones algebraicas}
\begin{mydef}
	Sea $k$ un cuerpo métrico y sea $V$ un $k$-espacio vectorial.
	Una función $\|\,\|\colon V \to [0, \infty)$ se dice una \strong{norma}\index{norma} si para todo $\vec u, \vec v\in V$ y todo $c \in k$ se cumple que:
	\begin{enumerate}[{EN}1., ref=EN\arabic*, leftmargin=*]
		\item $\|\vec v\| = 0$ syss $\vec v = \Vec 0$.
		\item $\|c \vec v\| = |c| \, \|\vec v\|$.
		\item\label{ax:norm_triangle} $\|\vec u + \vec v\| \le \|\vec u\| + \|\vec v\|$.
	\end{enumerate}
	Además, se dice que la norma $\|\,\|$ es \strong{ultramétrica} si, en lugar de \ref{ax:norm_triangle}, satisface:
	\begin{enumerate}[resume*]
		\item $\|\vec u + \vec v\| \le \max\{ \|\vec u\|, \|\vec v\| \}$.
	\end{enumerate}
	Un par $(V, \|\,\|)$ se dice un \strong{$k$-espacio normado}\index{espacio!normado}, obviaremos los índices de no haber ambigüedad sobre los signos.
	Nótese que $d(\vec u, \vec v) := \|\vec u - \vec v\|$ determina una métrica sobre $V$, luego una topología.
	Dos normas $\|\,\|_1, \|\,\|_2$ sobre $V$ se dicen \strong{equivalentes}\index{equivalentes!(normas)} si inducen la misma topología sobre $V$.
\end{mydef}

Comenzamos con unos resultados clásico de análisis:
\begin{lem}
	Sea $k$ un cuerpo métrico no trivial y $V$ un $k$-espacio vectorial.
	Dos normas $\|\,\|_1, \|\,\|_2$ sobre $V$ son equivalentes syss existen constantes $c_1, c_2 > 0$ tales que para todo $\vec a \in V$ se cumpla
	$$ c_1\|\vec a\|_1 \le \|\vec a\|_2 \le c_2\|\vec a\|_1. $$
\end{lem}
\begin{proof}
	Supondremos que $V \ne 0$.

	$\impliedby$. Sea $U$ un abierto relativo a $\|\,\|_2$, probaremos que es abierto relativo a $\|\,\|_1$ y por simetría veremos que los abiertos coinciden.
	Sea $\vec u \in U$, luego existe un $\epsilon > 0$ tal que
	$$ \{ \vec v : \| \vec v - \vec u \|_2 < \epsilon \} \subseteq U, $$
	Luego, claramente
	$$ \vec u \in \left\{ \vec v : \|\vec v - \vec u\|_1 < \frac{\epsilon}{c_1} \right\} \subseteq U, $$
	por lo que $U$ es abierto respecto a $\|\,\|_1$.

	$\implies$. Como las normas son equivalentes, entonces existe $r > 0$ tal que
	$$ \{ \vec v \in V : \|\vec v\|_1 < r \} \subseteq \{ \vec v \in V : \|\vec v\|_2 < 1 \}, $$
	como $k$ no es trivial, existe $\alpha \in k$ tal que $|\alpha| > 1$ de modo que $\lim_n |\alpha|^{-n} = 0$ y $\lim_n |\alpha|^n = \infty$.
	Luego, para todo $\vec v \in V$ existe un $n \in \Z$ tal que
	$$ |\alpha|^n \le \frac{1}{r}\|\vec v\|_1 < |\alpha|^{n+1} \implies \left\| \frac{1}{\alpha^{n+1}} \vec v \right\|_1 < r, $$
	de modo que $\| \vec v / \alpha^{n+1} \|_2 < 1$, ergo
	$$ \| \vec v \|_2 < |\alpha|^{n+1} = |\alpha| \, |\alpha|^n \le \frac{|\alpha|}{r} \|\vec v\|_1, $$
	de modo que $c_1 := |\alpha|/r$ funciona.
	Análogamente se construye $c_2$.
\end{proof}

\addtocounter{thmi}{1}
\begin{slem}
	Sea $k$ un cuerpo métrico y $V$ un $k$-espacio vectorial con base $\{ \vec e_1, \dots, \vec e_n \}$.
	Entonces, la función
	$$ \| a_1\vec e_1 + \cdots + a_n\vec e_n \|_\infty := \max_i\{ |a_i| \}, $$
	es una norma y $(V, \|\,\|_\infty)$ es completo si $k$ lo es.
\end{slem}
\addtocounter{thmi}{-1}

\begin{thm}\label{thm:normed_space_over_compl}
	Sea $k$ un cuerpo métrico no trivial completo y sea $V$ un $k$-espacio vectorial de dimensión finita.
	Todas las normas sobre $V$ son equivalentes y bajo todas $V$ resulta ser (un espacio métrico) completo.
\end{thm}
\begin{proof}
	Fijemos una base $\{ \vec e_1, \dots, \vec e_n \}$, y sea $\|\,\|_\infty$ la norma descrita en el lema anterior.
	Sea $\|\,\|$ otra norma de $V$, y definamos $c_2 := n\max_i \{ \|\vec e_i\| \}$; por desigualdad triangular se comprueba que
	$$ \|a_1\vec e_1 + \cdots + a_n\vec e_n\| \le |a_1| \, \|\vec e_1\| + \cdots + |a_n| \, \|\vec e_n\|
	\le c_2 \|a_1\vec e_1 + \cdots + a_n\vec e_n\|_\infty. $$
	Probaremos el teorema por inducción sobre $n$, el caso $n = 1$ es claro.

	Para el caso inductivo, supondremos que no existe $c_1 > 0$ tal que $c_1 \|\vec v\|_\infty \le \|\vec v\|$ para todo $\vec v \in V$,
	de modo que escogiendo $c_1 = 1/m$ existe $\vec v_m \in V$ tal que $\|\vec v_m\| < \frac{1}{m} \|\vec v_m\|_\infty$.
	Ahora bien, tenemos infinitos $\vec v_m$'s y como solo consta de finitas coordenadas hay, por palomar, al menos una que es la que tiene máximo
	valor absoluto infinitas veces, digamos que $\vec e_n$.
	Además podemos normalizar la sucesión de modo que obtenemos una sucesión $\{ \vec v_{\sigma m} \}_m$ tal que:
	\begin{itemize}
		\item $\|\vec v_{\sigma m}\|_\infty = 1$.
		\item $\vec v_{\sigma m} = a_1\vec e_1 + \cdots + a_n\vec e_n$ donde $a_n = 1$.
		\item $\|\vec v_{\sigma m}\| < 1/\sigma(m)$.
	\end{itemize}
	Así $\lim_m \|\vec v_{\sigma m}\| = 0$.
	Definamos $W := \Span_k\{ \vec e_1, \dots, \vec e_{n-1} \}$ el cual tiene dimensión menor y sean $\vec w_m := \vec v_{\sigma m} - \vec e_n \in W$.
	Luego $\lim_m \|\vec w + \vec e_n\| = 0$ y
	$$ \| \vec w_m - \vec w_M \| = \| (\vec w_m + \vec e_n) - (\vec w_M + \vec e_n) \| \le \| \vec w_m + \vec e_n \| + \| \vec w_M + \vec e_n \| \to 0, $$
	de modo que $( \vec w_m )_{m\in\N}$ es una sucesión de Cauchy respecto a la norma $\|\,\|$ contenida en $W$.

	En $W$, por hipótesis inductiva, todas las normas coinciden y son completas, de modo que $\vec w_m$ converge a un valor $\vec w \in W$ y así:
	$$ \|\vec w + \vec e_n\| = \|(\vec w - \vec w_m) + (\vec w_m + \vec e_n)\| \le \|\vec w - \vec w_m\| + \|\vec w_m + \vec e_n\| \to 0, $$
	por lo que $\vec w = -\vec e_n$, pero $-\vec e_n \notin W$ lo que es absurdo.
\end{proof}

\begin{cor}
	Sea $(k, |\,|)$ un cuerpo métrico completo y sea $L/k$ una extensión finita de cuerpos.
	Existe a lo más un valor absoluto $\|\,\|$ sobre $L$ (salvo equivalencia) que extiende a $|\,|$.
	Más aún, $L$ también es un completo respecto a ésta métrica.
\end{cor}
\begin{proof}
	Basta notar que un valor absoluto sobre $L$ es una norma de $L$ como $k$-espacio vectorial.
\end{proof}

\begin{thmi}\label{thm:complete_metric_fld_algcl_metric}
	Sea $(k, |\,|)$ un cuerpo métrico completo no trivial y sea $L/k$ una extensión finita de cuerpos.
	Existe exactamente un valor absolutos $\|\,\|$ sobre $L$ (salvo equivalencia) que extiende a $|\,|$.
	En consecuencia, existe exactamente un valor absoluto sobre $\algcl k$ que extiende a $|\,|$.
\end{thmi}
\begin{proof}
	Si $k = \R$ o $\C$, entonces es claro así que nos enfocaremos en el caso ultramétrico.

	Considere la norma $\galnorm_{L/k}\colon L \to k$, y defínase $\|\alpha\| := |\galnorm_{L/k}(\alpha)|^{1/n}$ donde $n := [L : k]$.
	Si $a \in k$, entonces $\galnorm_{L/k}(a) = a^n$, así que vemos que $\|\,\|$ extiende a $|\,|$.
	Si $\alpha \in L$ es no nulo, entonces $\galnorm_{L/k}(\alpha) = \prod_{\sigma} \sigma(\alpha)$, donde $\sigma$ recorre todos los $k$-monomorfismos
	de $L$ en su clasura normal, luego $\galnorm_{L/k}(\alpha) \ne 0$ y, por ello, $\|\alpha\| \ne 0$ (VA1).
	Como $\galnorm_{L/k}$ es multiplicativa, entonces $\|\,\|$ también (VA2).

	Falta ver la desigualdad ultramétrica:
	Sea $\|\alpha\| \le 1$, sea $f(t) = a_0 + \cdots + a_{n-1}t^{n-1} + t^n \in k[t]$ el polinomio minimal de $\alpha$ y sea $F(t) :=
	\psi_{\alpha, L/k}(t) = c_0 + \cdots + c_{m-1}t^{m-1} + t^m \in k[t]$ el polinomio característico de $\alpha$.
	Sabemos que $\galnorm_{L/k}(\alpha) = \pm c_0$ (cfr. \cite[def.~4.63]{Alg} y \cite[prop.~3.67]{Alg}) y también sabemos que el polinomio característico
	es alguna potencia natural de $f(t)$, de modo que $c_0 = a_0^r$ para algún $r\ge 1$.
	Si:
	$$ |a_0|^{r/n} = |c_0|^{1/n} = |\galnorm_{K/k}(\alpha)|^{1/n} = \|\alpha\| \le 1, $$
	entonces $|a_0| \le 1$.
	Como $f(t)$ es irreducible, entonces es Newton-puro (corolario~\ref{thm:irred_implies_pure}) y como $\|f(t)\|_c = |a_0|$ (lema~\ref{lem:newton_norms}),
	entonces vemos que $\|f(t)\| \le 1$, por lo que $f(t) \in \mathfrak{o}[t]$ y luego $F(t) \in \mathfrak{o}[t]$.
	Ahora bien, nótese que
	$$ \|1 + \alpha\| = |\galnorm_{L/k}(1 + \alpha)|^{1/n} = |(-1)^n F(-1)|^{1/n} \le \|F(t)\|_1 \le 1. $$
	Aquí empleamos el hecho de que $\galnorm_{L/k}(x - \alpha) = F(x)$ para todo $x \in k$, ¿por qué ésto es cierto?

	Finalmente, la última desigualdad comprueba la desigualdad ultramétrica y, por tanto, completa la demostración.
\end{proof}
Por la unicidad, denotaremos a éste valor absoluto $|\,|$.
Nótese que $|\alpha| := |\galnorm_{L/K}(\alpha)|^{1/n}$ siempre determina un valor absoluto sobre $L$, independiente de si $K$ es completo o no;
no obstante la completitud de $K$ se exige si se desea unicidad.

\begin{exn}
	\addexample{Una extensión finita $L/K$ de cuerpos con $K$ métrico, donde $L$ admite más de una extensión del valor absoluto}
	Considere $(\Q, |\,|_\infty)$ el cual no es completo y considere la extensión finita $L := \Q(\sqrt{2})/\Q$.
	Considere al elemento $\alpha$ que es raíz de $(x - 1)^2 - 2$.
	Éste elemento, tradicionalmente lo elegimos como $\sqrt{2} + 1 > 0$, donde $|\sqrt{2} + 1| \approx 2.41$;
	pero también podríamos elegirlo como $-\sqrt{2} + 1 < 0$, donde $\lvert-\sqrt{2} + 1\rvert \approx 0.41$.

	Otra manera de entender éste ejemplo es que $2 \in \R$ posee dos raíces, una positiva y otra negativa, y ambas son \textit{algebraicamente indistinguibles};
	el ordenamiento de $\R$ induce un valor absoluto sobre $L \subseteq \R$, así pues ambas elecciones (de que $\sqrt{2}$ sea positivo o no) inducen dos
	valores absolutos no equivalentes.
\end{exn}

De la construcción se sigue:
\begin{cor}\label{thm:conjs_share_norm}
	Sea $k$ un cuerpo ultramétrico completo.
	Todo $k$-conjugado de un elemento $k$-algebraico comparte su valor absoluto.
\end{cor}

\begin{cor}[lema de Krasner]\label{thm:krasner_lemma}
	Sea $k$ un cuerpo ultramétrico completo.
	Sea $\alpha \in \algcl k$ y sea $\beta$ un $k$-conjugado distinto de $\alpha$.
	Entonces, para todo $a \in k$ se cumple que $|a - \alpha| \ge |\alpha - \beta|$.
	
	% En consecuencia, si $\alpha$ es $k$-separable y existen $\beta \in \algcl k$ y un $\alpha'$ un $k$-conjugado de $\alpha$
	% tales que $|\alpha - \beta| < |\alpha - \alpha'|$, entonces $\alpha \in k(\beta)$.
\end{cor}
\begin{proof}
	De lo contrario, existiría algún $a \in k$ tal que
	$$ |a - \alpha| < |\alpha - \beta| = |a - \beta|, $$
	pero $a - \alpha, a - \beta$ son $k$-conjugados por lo que la desigualdad es absurda.
\end{proof}

Nótese que la prueba de la unicidad es universal al caso arquimediano y ultramétrico, pero la prueba de existencia es exclusiva del caso ultramétrico,
aunque mediante la clasificación de cuerpos métricos arquimedianos por el teorema de Ostrowski también podemos contemplar ese caso.

\subsection{Compleción de un cuerpo algebraicamente cerrado}
Recuento sobre lo que hemos hecho: dado un cuerpo métrico, podemos completarlo, y dado un cuerpo métrico completo podemos tomar su clausura algebraica que
también será métrica. Ahora, bien podría darse (y se dá) que dicha clausura algebraica no sea completa, lo que veremos en ésta sección.

\begin{thm}
	Sea $k$ un cuerpo ultramétrico completo y sea $\alpha \in \sepcl k$. Defínase
	$$ r := \min_{\beta\ne\alpha} |\beta - \alpha|, $$
	donde $\beta$ recorre los $k$-conjugados de $\alpha$.
	Sea $\gamma \in B_r(\alpha) \subseteq \algcl k$, entonces $k(\alpha) \subseteq k(\gamma)$.
\end{thm}
\begin{proof}
	Sean $f(x) \in k[x]$ y $\phi(x) \in k(\gamma)[x]$ el polinomio minimal de $\alpha$ sobre $k$ y sobre $k(\gamma)$ resp.
	Luego $\phi(x) \mid f(x)$. Sea $\sigma \in \Gal( k(\alpha, \gamma) / k(\gamma) )$, de modo que $\sigma(\gamma - \alpha) = \gamma - \sigma(\alpha)$.
	Como los conjugados comparten norma por el corolario~\ref{thm:conjs_share_norm}, vemos que
	$$ |\gamma - \sigma(\alpha)| = |\gamma - \alpha| < r, $$
	luego, por desigualdad ultramétrica, $|\alpha - \sigma(\alpha)| \le |\gamma - \alpha| < r$,
	de modo que por definición de $r$ se sigue que $\sigma(\alpha) = \alpha$.
	Así pues, como $\alpha$ es separable, vemos que $k(\alpha, \gamma) = k(\gamma)$ como se quería probar.
\end{proof}

% \begin{mydef}
% 	Sea $k$ un cuerpo ultramétrico completo.
% 	Sean $f(x), g(x) \in k[x]$ polinomios con raíces $\alpha_1, \dots, \alpha_n$ y $\beta_1, \dots, \beta_m$ en $\algcl k$ resp.
% 	Defínase $\epsilon := \min_{i,j} \{ |\alpha_i - \beta_j| \}$.
% 	Si $\|f(x) - g(x)\|_1 < \epsilon$, entonces se dice que $f, g$ están \strong{suficientemente cerca}\index{suficientemente cerca (polinomios)}.
% \end{mydef}

En el siguiente teorema, la expresión <<suficientemente cerca>> significa que existe algún $\epsilon > 0$ tal que si $\|f(x) - g(x)\| < \epsilon$,
entonces se cumple lo enunciado.
\begin{thm}
	Sea $k$ un cuerpo ultramétrico completo.
	Sean $f(x), g(x) \in k[x]$ mónicos de grado común $n$ y fijemos la norma $\|\,\| := \|\,\|_1$ sobre $k[x]$.
	Para $f, g$ suficientemente cerca se cumplen las siguientes:
	\begin{enumerate}
		\item Si $f(x)$ es irreducible en $k$ y separable en $\algcl k$, entonces $g(x)$ también y genera el mismo cuerpo de escisión.%
			\footnote{El cuerpo de escisión de un polinomio $h(x) \in k[x]$ es la mínima extensión $L/k$ en la cual $h$ se escinde,
			es decir, contiene a todas sus raíces (cfr. \cite[def.~4.19]{Alg}).}
		\item Si una raíz $\alpha \in \algcl k$ en $f(x)$ tiene multiplicidad $m$, entonces la cantidad de raíces $\beta_i$ de $g(x)$
			tales que $k(\beta_i) \subseteq k(\alpha)$ es (contando multiplicidad) $m$.
		\item Si $f(x)$ es irreducible en $k$, entonces $g(x)$ también.
	\end{enumerate}
\end{thm}
\begin{proof}
	Digamos que $\|f(x) - g(x)\| < \epsilon$ para un $\epsilon$ que determinaremos en cada inciso,
	y denotemos $\alpha_1, \dots, \alpha_n$ y $\beta_1, \dots, \beta_n$ las raíces (contando multiplicidad) de $f(x), g(x)$ en $\algcl k$ resp.
	\begin{enumerate}
		\item Supongamos que $\|f(x)\| \le C$ (sin fijar) y sea $|\gamma| > C$,
			si $f(x) = \sum_{j=0}^{n} c_j x^j$ entonces $|\gamma|^n > |c_j\gamma^j|$ para $j < n$, de modo que $\gamma$ no puede ser raíz de $f(x)$.
			Es decir, cada $|\alpha_j| \le C$.

			Si elegimos $C := \max\{ \|f\|, \|g\|, 1 \}$, vemos que cada $|\beta_j| \le C$ y, por ende,
			para cualquier $\beta_j$ vemos también que
			$$ |f(\beta_j)| = |f(\beta_j) - g(\beta_j)| \le \epsilon C^n. $$
			Y por desigualdad ultramétrica vemos que
			$$ |\beta_j - \alpha_1| \cdots |\beta_j - \alpha_n| \le \epsilon C^n, $$
			luego, al menos un $\alpha_i$, digamos $\alpha_1$, satisface que
			$$ |\beta_j - \alpha_1| \le C \sqrt[n]{\epsilon}. $$
			Ahora fijemos
			$$ \epsilon := \left( \min_{i\ne j} \{ |\alpha_i - \alpha_j| \} \right)^n > 0, $$
			de modo que, como $f$ es irreducible, vemos que los $\alpha_j$'s son $k$-conjugados,
			luego por el teorema anterior comprobamos que $k(\beta_j) \supseteq k(\alpha_1)$.

			Como $f, g$ son del mismo grado, entonces concluimos que $g$ es irreducible (pues genera una extensión de cuerpos de su mismo grado)
			y es separable (pues los cuerpos coinciden).

		\item De lo contrario, para cualquier $n > 0$ podríamos encontrar otro polinomio $g_n$ tal que $\|f - g_n\| < 1/n$
			y donde $g_n$ posee raíces $\beta_{n, j}$ de multiplicidad $\mu_j$ tal que $\lim_n \beta_{n, j} = \alpha_j$.
			Así, por completitud de $k$, vemos que $\lim_n g_n = f$ y así tenemos que
			$$ (x - \alpha_1)^{\eta_1} \cdots (x - \alpha_n)^{\eta_n} = f(x) = \lim_n g_n(x) = (x - \alpha_1)^{\mu_1} \cdots (x - \alpha_n)^{\mu_n}, $$
			donde $\eta_j \ne \mu_j$ para algún $j$ lo que contradice la factorización única de $\algcl{k}[x]$.

		\item Al igual que en el inciso anterior construimos una sucesión de Cauchy que converja a $f$ donde cada término se factoriza en dos polinomios
			cuyas raíces convergen a algunas raíces de $f$ y llegamos a una contradicción pues sus límites prueban que $f$ es reducible. \qedhere
	\end{enumerate}
\end{proof}

\begin{thm}\label{thm:compl_algcl_is_complete}
	Sea $K$ un cuerpo ultramétrico algebraicamente cerrado, entonces $\hat K$ también es algebraicamente cerrado.
\end{thm}
\begin{proof}
	La demostración prosigue dos pasos:
	\begin{enumerate}[(i)]
		\item \underline{$\hat K$ es perfecto:}
			Es claro si $\car K = 0$ (cfr. \cite[teo.~4.34]{Alg}), así que supondremos que $p := \car K > 0$.
			Basta probar que el endomorfismo de Frobenius es suprayectivo (también, cfr. \cite[teo.~4.34]{Alg}):
			sea $\alpha \in \hat K$, luego $\alpha = \lim_n \beta_n$ para $(\beta_n)_n \subseteq K$;
			como $K$ es algebraicamente cerrado, entonces vemos que $\lim_n \beta_n^{1/p} = \alpha^{1/p} \in \hat K$.

		\item Como $\hat K$ es perfecto, entonces basta probar que todo polinomio irreducible, mónico, separable $f(x) \in \hat K[x]$ es lineal.
			Como $K[x]$ es denso en $\hat K[x]$, podemos encontrar $g(x) \in K[x]$ suficientemente cerca de $f(x)$ de modo que $g$
			sea separable, irreducible y genere el mismo cuerpo de escisión.
			Como $K$ es algebraicamente cerrado, entonces $g$ es lineal y así $f$ también debe serlo. \qedhere
	\end{enumerate}
\end{proof}

\section{Principio local-global}
Ya hablamos en determinado momento del principio local-global,
en ésta sección procedemos a probar que es válido para cierto tipo de ecuaciones diofánticas.
Recuérdese que una \strong{forma cuadrática} es un polinomio homogéneo de grado 2.

\begin{thm}
	Sea $f(x_1, x_2, x_3) \in \Q[\vec x]$ una forma cuadrática y supongamos que en todo $\Q_p$ existe una solución no trivial de $F(\vec x) = 0$.
	Entonces existe una solución no trivial en $\Q$.
\end{thm}
\begin{proof}
	Tras una transformación lineal podemos suponer, sin perdida de generalidad, que
	\[
		f(\vec x) = a_1x_1^2 + a_2x_2^2 + a_3x_3^2.
	\]
	Si algún $a_j \in \Q$ fuese nulo, entonces es claro que hay una solución no trivial en $\Q$, así que supondremos que $a_1a_2a_3 \ne 0$.
	Así mismo, limpiando denominadores, podemos suponer que cada $a_j \in \Z$.
\end{proof}

\section*{Notas históricas}
La genesis de la teoría de valuaciones está en el llamado \textit{análisis $p$-ádico}, lo cual fue una invención de \textbf{Kurt Hensel}
en \cite{hensel:begrundung} (\citeyear{hensel:begrundung}).
\citet{hensel:arithmetik} demostró que $\Q_p$, definido formalmente como series formales sobre $x = p$, es un cuerpo.

La definición de \textit{cuerpo con valor absoluto} (o \textit{cuerpo métrico}, como lo empleamos aquí) fue formulada por vez primera por
el hungaro \textbf{Josef Kürschák}, presentado en el Congreso Internacional de Matemáticos de Cambridge y publicado en \cite{kurschak:korpertheorie}
(\citeyear{kurschak:korpertheorie}).
El resultado de Kürschák era que todo cuerpo métrico se extiende (con su valor absoluto) a un cuerpo métrico completo y algebraicamente cerrado.
El método de Kürschák es equivalente al nuestro:
\begin{enumerate}[i)]
	\item Todo cuerpo métrico admite una compleción a un cuerpo métrico (cfr. teorema~\ref{thm:metric_fld_completion}).
	\item Un cuerpo métrico completo admite una extensión a un cuerpo métrico algebraicamente cerrado
		(cfr. teorema~\ref{thm:complete_metric_fld_algcl_metric}).
	\item La compleción de un cuerpo métrico algebraicamente cerrado es también algebraicamente cerrada
		(cfr. teorema~\ref{thm:compl_algcl_is_complete}).
\end{enumerate}
El paso \textsc{i)} lo sigue de los argumentos clásicos de análisis, en particular citando a Cantor; el paso \textsc{ii)} emplea la construcción de la clausura
algebraica de Steinitz y el paso \textsc{iii)} sigue la demostración del teorema fundamental del álgebra dada por Weierstrass.
Además, Kürschák afirma (sin demostración) que los cuerpos ultramétricos satisfacen el lema de Hensel.

Los dos teoremas de Ostrowski fueron descubiertos por el ruso \textbf{Alexander Ostrowski} en \cite{ostrowski:funktionalgleichung}
(abril 1916, publ. \citeyear{ostrowski:funktionalgleichung}); los nombres \textit{primero} y \textit{segundo} son añadido propio.
Si cambiamos cuerpo métrico arquimediano por $\R$-álgebra normada%
\footnote{Una $k$-álgebra normada es una $k$-álgebra $(A, \|\,\|)$ que es un $k$-espacio normado
y en donde $\|\alpha\beta\| = \|\alpha\|\,\|\beta\|$ para todo $\alpha, \beta\in A$.}
completa entonces la lista se amplía a $\R, \C$ y $\HH$, donde los últimos son los cuaterniones y forman una $\R$-álgebra de división no conmutativa.
% Nótese que los octoniones no entran en la lista.

En otro artículo \cite{ostrowski:fragen} (\citeyear{ostrowski:fragen}) y en una simplificación tardía \cite{ostrowski:perfekte} (\citeyear{ostrowski:perfekte}),
Ostrowski reanuda un estudio sobre los trabajos de Kürschák que incluye una demostración de que una extensión finita de un cuerpo métrico completo
es también completa (cfr. teorema~\ref{thm:complete_metric_fld_algcl_metric}) siguiendo la estrategia de espacios normados y, mediante su demostración,
probando que hay una \textit{única} extensión del valor absoluto a la clausura algebraica (algo que Kürschák no probó).
En el mismo trabajo vemos que Ostrowski emplea el llamado \textit{lema de Krasner} (cfr. corolario~\ref{thm:krasner_lemma}).
Finalmente, respondió a la duda de Kürschák respecto a cuando es la clausura algebraica de un cuerpo métrico completa.

Los resultados desarrollados por Hensel, Kürschák y Ostrowski fueron recopilados y reordenados en simplicidad por el checo Karel Rychlík,
primero publicados en checo en \cite{rychlik:beitrag} (\citeyear{rychlik:beitrag}) con poca repercusión, y más tarde en alemán en
\cite{rychlik:bewertungstheorie} (\citeyear{rychlik:bewertungstheorie});
de ahí que algunas versiones del lema de Hensel sean llamadas <<lema de Rychlík>>.
Como las demostraciones de éste lema siguen el llamado \textit{método de Newton} para encontrar ceros de funciones continuas, también hay versiones
del lema de Hensel llamados <<lema de Newton>>.

El teorema de aproximación fue demostrado en toda su generalidad por \citet{artin:prod_form}.
El teorema, restringido a valores absolutos no arquimedianos, también fue empleado en un manuscrito de Ostrowski \cite{ostrowski:arthmetischen}
escrito cerca de 1916, pero publicado en 1935.

Los polígonos de Newton fueron introducidos por Newton en su correspondencia con Oldenburg (13 de junio de 1676,
vid. \cite[20-47]{newton:correspondence_ii}, y 24 de octubre de 1676, vid. \cite[162-164]{newton:correspondence_ii});
una buena exposición moderna se encuentra en la sección \S8.3 de \citeauthor{brieskorn:curves}~\cite[370-454]{brieskorn:curves}.
La aplicación del polígono de Newton a la teoría de valuación, particularmente el teorema~\ref{thm:newton_polygon},
es originaria por \citet{ostrowski:dirichletschen}, aunque fue inicialmente estudiado por \citet{rella:polynombereichen}.
\addtocategory{historical}{
	hensel:begrundung, hensel:arithmetik, kurschak:korpertheorie,
	ostrowski:funktionalgleichung, rychlik:beitrag, rychlik:bewertungstheorie, ostrowski:arthmetischen,
	ostrowski:dirichletschen, rella:polynombereichen, ostrowski:fragen, ostrowski:perfekte,
}
\addtocategory{history}{brieskorn:curves, newton:correspondence_ii}

\printbibliography[segment=\therefsegment, check=onlynew, notcategory=history, notcategory=historical, notcategory=other]
\bibbycategory[segment=\therefsegment, check=onlynew]

\end{document}
