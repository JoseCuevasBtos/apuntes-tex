\documentclass[teoria-numeros.tex]{subfiles}
\begin{document}

\chapter{Formas modulares}
En la sección sobre curvas elípticas complejas se realizó lo siguiente:
\begin{enumerate}
	\item Vimos que una curva elíptica $E(\C)$ puede, en ciertos casos, verse como un cociente analítico $\C/\Lambda$,
		para algún reticulado $\Lambda$.
	\item A un punto $z \in \C$ con $\Im(z) > 0$ podemos asignarle el reticulado $\Lambda = \Z + \tau\Z$, los cuales son homotéticos a todos los reticulados
		posibles dentro de $\C$.
		Bajo esta asociación, podemos asignarle $z \mapsto j(E_{\Z + \tau\Z})$ la función que da el invariante $j$ de la curva elíptica asociada.
	\item Como el invariante $j$ es efectivamente el mismo para curvas elípticas isomorfas,
		entonces dados dos $\tau_1, \tau_2$ con $\Im(\tau_1), \Im(\tau_2) > 0$ que generen el mismo reticulado, tenemos que $j(\tau_1) = j(\tau_2)$.
		Esto induce a que $j$, visto como una función compleja, posea una inusual cantidad de simetrías.
\end{enumerate}
El objetivo de éste capítulo es perseguir estas observaciones y construir sobre éste tema.
Formalmente, éste capítulo no depende de aquél de curvas elípticas, sino que solo se sirve del último como inspiración.

\section{Acciones sobre el semiplano superior}
% \subsection{Acciones topológicas y analíticas}
Comenzaremos con ciertas propiedades generales de las acciones topo\-lógicas.
\nocite{miyake:modular}

\begin{mydef}
	Sea $G$ un grupo topológico, y sea $X$ un espacio topológico (resp.\ una variedad analítica).
	Una \strong{acción topológica}\index{acción!topológica} (resp.\ \strong{acción analítica}\index{acción!analítica}) de $G$ sobre $X$
	es una aplicación
	\[
		a \colon X \times G \longrightarrow X, \qquad (x, g) \longmapsto x\cdot g,
	\]
	que satisface lo siguiente:
	\begin{enumerate}[{AT}1., leftmargin=*, ref=AT\arabic*]
		\item\label{ax:topological_actions_are_top}
			Para cada $g_0 \in G$, la endofunción $x \mapsto x\cdot g_0$ sobre $X$ es continua (resp.\ holomorfa).
		\item Para cada $g, h \in G$ y $x \in X$ se cumple que $(x\cdot g)\cdot h = x\cdot(gh)$.
		\item Para cada $x \in X$ se cumple que $x\cdot 1 = x$.
	\end{enumerate}
	También se dice que $G$ \strong{actúa topológicamente} (resp.\ \strong{analíticamente}) sobre $X$.
	% Es decir, es una acción donde 
\end{mydef}
Se siguen las propiedades típicas de acciones como que, por ejemplo, por \ref{ax:topological_actions_are_top}, las traslaciones $x \mapsto x\cdot g_0$
son homeomorfismos (resp.\ biholomorfismos) cuando $X$ es un espacio topológico (resp.\ variedad analítica).
También se preservan definiciones como las del estabilizador, órbitas y las definiciones de acción fiel y transitiva.
Nótese que toda acción analítica es topológica.

La generalización del \textquote{teorema de órbita-estabilizador} en este contexto es el siguiente:
\begin{thm}\label{thm:topological_orbit_stab}
	Sea $G \acts X$ una acción topológica transitiva, donde $G$ es un grupo localmente compacto 2AN y $X$ es de Hausdorff localmente compacto.
	Entonces, para cada punto $x \in X$ la aplicación
	$$ \Phi\colon (G/\Stab_x)^+ \longrightarrow X, \qquad \Stab_x g \longmapsto x\cdot g $$
	determina un homeomorfismo entre el espacio topológico cociente de clases laterales derechas $G / \Stab_x$ y $X$.
\end{thm}
\begin{proof}
	Es claro que $\Phi$ es una biyección, por lo que basta probar que es una aplicación continua y abierta.
	La continuidad es clara, mientras que \textquote{ser abierta} equivale a que para todo abierto $U \subseteq G$ la imagen $xU \subseteq X$ sea abierta:
	dado un punto $x\cdot g \in xU$, existe un entorno $V \subseteq G$ del $1 \in G$ que tiene clausura compacta $K$, es simétrico (i.e., $K^{-1} = K$)
	y tal que $K^2 g \subseteq U$.
	Como $G$ posee una base numerable, existe una sucesión $(g_n)_{n\in\N}$ en $G$ tal que $G = \bigcup_{n\in\N} Kg_n$ y defínase $W_n := xKg_n$,
	de modo que $X = \bigcup_{n\in\N} W_n$.
	Como $X$ es de Hausdorff, cada $W_n$ es un subconjunto compacto cerrado.
	Supongamos, por contradicción, que cada $W_n$ tiene interior vacío.
	Como $X$ es regular, podemos construir recursivamente abiertos $U_n \subseteq X$ no vacíos de clausura compacta tales que
	\[
		\overline{U}_n \subseteq U_{n-1} \setminus W_{n-1}, \qquad n \ge 2.
	\]
	En particular, $\overline{U}_1 \supseteq \overline{U}_2 \supseteq \cdots$
	Nótese que, por compacidad de $\overline{U}_1$, se cumple que $\bigcap_{n\in\N} \overline{U}_n \ne \emptyset$;
	pero esto es absurdo ya que $\bigcap_{n\in\N} \overline{U}_n$ no corta a ningún $W_m$ y $X = \bigcup_{m\in\N} W_m$.
	Así, algún $W_n$ tiene interior no vacío y basta con trasladar.
\end{proof}

\begin{mydef}
	Sea $a\colon G \acts X$ una acción topológica.
	Se dice que $a$ es una \strong{acción propiamente discontinua}\index{acción!propiamente discontinua} si todo par de puntos distintos $x \ne y \in X$
	posee entornos $U, V \subseteq X$ resp., tales que
	\[
		|\{ g \in G : U\cdot g \cap V \ne \emptyset \}| < \infty.
	\]
\end{mydef}
\begin{cor}
	Si $a\colon G \acts X$ es una acción topológica sobre un espacio $X$ localmente compacto,
	entonces $a$ es propiamente discontinua syss para todo par de subconjuntos compactos $A, B \subseteq X$ se cumple que
	\[
		|\{ g \in G : A\cdot g \cap B \ne \emptyset \}| < \infty.
	\]
\end{cor}
\begin{proof}
	Como todo cubrimiento abierto de un compacto admite un subcubrimiento finito, vemos \textquote{${\implies}$}.
	El recíproco \textquote{$\impliedby$} es precisamente por definición de \textquote{localmente compacto}.
\end{proof}

\addtocounter{thmi}{1}
\begin{slem}
	Si $G$ es un grupo topológico y $\Gamma \le G$ es un subgrupo discreto (i.e., cuya topología subespacio es la discreta),
	entonces $\Gamma \le_f G$ es un subgrupo cerrado sin puntos de acumulación.
\end{slem}
\begin{proof}
	Por definición de <<discreto>>, existe un entorno $U \subseteq G$ del $1 \in G$ tal que $U \cap \Gamma = \{ 1 \}$;
	como $G$ es un grupo topológico, existe un subentorno simétrico $V$ del $1$ tal que $V^{-1}V \subseteq U$.
	En consecuencia, dados dos elementos distintos $\alpha \ne \beta \in \Gamma$ se cumple que $\alpha V \cap \beta V = \emptyset$.
	
	Dado un elemento adherente $g \in \overline{\Gamma}$ y un $\alpha \in gV^{-1} \cap \Gamma$, vemos que $gV^{-1} \cap \Gamma = \{ \alpha \}$,
	de modo que $g = \alpha \in \Gamma$.
\end{proof}
\addtocounter{thmi}{-1}

\begin{thm}\label{thm:proper_discont_action}
	Sea $G \acts X$ una acción topológica transitiva y sea $\Gamma \le G$ un subgrupo.
	Supongamos que:
	\begin{enumerate}[(i)]
		\item $G$ es un grupo localmente compacto y 2AN.
		\item $X$ es Hausdorff localmente compacto.
		\item Todos los estabilizadores de $G$ son subgrupos compactos.
	\end{enumerate}
	Entonces son equivalentes:
	\begin{enumerate}
		\item El subgrupo $\Gamma$ es discreto.
		\item La acción (por restricción) $\Gamma \acts X$ es propiamente discontinua.
	\end{enumerate}
\end{thm}
\begin{proof}
	$1 \implies 2$.
	Sea $x \in X$ un punto y sean $A, B \subseteq X$ un par de compactos.
	Sean
	\[
		M := \{ g \in G : gx \in A \}, \qquad N := \{ g \in G : gx \in B \},
	\]
	los cuales son cerrados (¿por qué?) y, como $G \acts X$ es transitiva, satisfacen $M\cdot x = A$ y $N\cdot x = B$.

	Ya que $G$ es localmente compacto, sea $\{ U_i \}_i$ un cubrimiento por abiertos de $M$ tal que cada $\overline{U}_i$ es compacto;
	aplicando ${\cdot x}$, vemos que $A = M\cdot x \subseteq \bigcup_{i} U_i\cdot x$ y, por compacidad, debe existir un subcubrimiento finito
	$\{ U_j \}_{j=1}^n$, de modo que $M \subseteq \bigcup_{j=1}^n \overline{U}_i\cdot\Stab_x$ (esto es aplicando preimágenes mediante $g \mapsto gx$),
	donde el conjunto de la derecha es compacto.
	Así, $M$, y análogamente $N$, son compactos.
	Finalmente $NM^{-1}$ es compacto, por lo que
	\[
		\{ \gamma \in \Gamma : \gamma A \cap B \} = \Gamma \cap NM^{-1}
	\]
	es compacto y discreto, ergo finito.

	$2 \implies 1$.
	Sea $1 \in V \subseteq G$ un entorno del neutro con $\overline{V}$ compacto.
	Para todo punto $x \in X$ se cumple que $\Gamma \cap V \subseteq \{ \gamma \in \Gamma : \gamma x \in \overline{V}x \}$,
	el cual es finito ya que $\{ x \}, \overline{V}x \subseteq X$ son compactos.
	Así, podemos achicar $V$ de modo que $\Gamma \cap V = \{ 1 \}$ es un abierto de $\Gamma$.
\end{proof}

\addtocounter{thmi}{1}
El siguiente lema es expandir las definiciones correspondientes:
\begin{slem}\label{thm:lemma_for_Haus_quot}
	Sea $G \acts X$ una acción topológica.
	Supongamos que todo par de puntos $x, y \in X$ poseen entornos $x \in U \subseteq X$, $y \in V \subseteq X$ tales que $gU \cap V = \emptyset$
	para todo $g \in G$ tal que $gx \ne y$.
	Entonces el espacio topológico cociente $G \coquot X$ es de Hausdorff.
\end{slem}
\addtocounter{thmi}{-1}

\begin{prop}\label{thm:prp_disct_Haus_quot}
	Si $G \acts X$ es una acción propiamente discontinua sobre un espacio de Hausdorff $X$.
	Entonces $G \coquot X$ es de Hausdorff.
\end{prop}
\begin{proof}
	Sean $x \ne y \in X$ y sean $U_0, V_0$ entornos de $x, y$ resp., tales que el conjunto $\mathcal{G} := \{ g \in \Gamma : gU_0 \cap V_0 \}$ es finito
	y sean $\{ \gamma_1, \dots, \gamma_m \} = \mathcal{G}$ sus elementos.
	Cambiando el orden de los índices podemos suponer que existe un índice $\ell$ tal que
	\[
		\forall 1 \le i \le \ell < j \le n \quad \gamma_i x = y, \; \gamma_j x \ne y.
	\]
	Para cada $j > \ell$ podemos escoger entornos disjuntos $W_j, V_j$ de $\gamma_j x, y$ resp., y definir
	\[
		U := U_0 \cap \bigcap_{j=\ell+1}^m \gamma_j^{-1}W_j, \qquad V := V_0 \cap \bigcap_{j=\ell+1}^{m} V_j.
	\]
	De éste modo, $U$ y $V$ satisfacen las hipótesis del lema anterior.
\end{proof}

En éste capítulo, trabajaremos principalmente con dos abiertos distinguidos del plano complejo $\C$ que son el
\strong{semiplano superior} y el \strong{disco unitario} resp.:
\[
	\mathfrak{H} := \{ z \in \C : \Im z > 0 \}, \qquad \D := \{ z \in \C : |z| < 1 \}.
\]
Tanto a $\mathfrak{H}$ como $\D$ los vamos a dotar de estructura de variedad analítica.

% \begin{mydef}
% 	Dada una matriz
% 	\[
% 		\gamma :=
% 		\begin{bmatrix}
% 			a & b \\
% 			c & d
% 		\end{bmatrix}
% 		\in \GL_2(\R)
% 	\]
% 	y un punto $z \in \C$ definiremos temporalmente $j(\gamma, z) := cz + d \in \C$, de modo que se tiene la siguiente relación
% 	\[
% 		\gamma\cdot
% 		\begin{bmatrix}
% 		z \\ 1
% 		\end{bmatrix} =
% 		\begin{bmatrix}
% 		az + b \\ cz + d
% 		\end{bmatrix} =:
% 		j(\gamma, z)
% 		\begin{bmatrix}
% 		\gamma\cdot z \\ 1
% 		\end{bmatrix},
% 	\]
% 	donde esto define el símbolo $\gamma\cdot z$, lo cual determina una acción $\GL_2(\R) \acts \C$.
% \end{mydef}
\begin{mydef}
	Sobre la esfera de Riemann $\PP^1(\C)$, podemos considerar las \strong{transformaciones de Möbius} asociadas a matrices inversibles
	\[
		\gamma :=
		\begin{bmatrix}
			a & b \\
			c & d
		\end{bmatrix}
		\in \GL_2(\R), \qquad \gamma\cdot z := \frac{az + b}{cz + d}.
	\]
	Las cuales determinan una acción analítica $\GL_2(\R) \acts \PP^1(\C)$.
\end{mydef}

\addtocounter{thmi}{1}
\begin{slem}
	Dado $\gamma \in \GL_2(\R)$ y un punto $z \in \C$, se satisface que
	\begin{equation}
		\Im(\gamma\cdot z) = \frac{\det(\gamma)\Im z}{|cz + d|^2}.
		\label{eqn:im_part_of_gl2_act}
	\end{equation}
	En particular, $\mathfrak{H}$ es $\GL_2^+(\R)$-estable.
\end{slem}

El teorema del mapeo de Riemann nos dice que:
\begin{slem}
	Las variedades $\mathfrak{H}$ y $\D$ son biholomorfas.
\end{slem}
\addtocounter{thmi}{-1}
Recuérdese que
\[
	\GL_2^+(\R) := \{ \gamma \in \GL_2(\R) : \det\gamma > 0 \} \supseteq \SL_2(\R).
\]
Dentro de $\GL_2^+(\R)$ tenemos a todos los elementos de la forma $\lambda I_2$ para $\lambda \in \R^\times$, donde $I_2$ es la matriz identidad;
a estos elementos les llamaremos \strong{escalares}\index{escalares (elementos de $\GL_2(\R)$)}
y esto determina un monomorfismo canónico $\R^\times \hookto \GL_2^+(\R)$.
También será útil el grupo simétrico especial
\[
	\SO_2(\R) := \left\{ 
		\begin{bmatrix}
			\phantom{-}\cos\theta & \sin\theta \\
			-\sin\theta & \cos\theta
		\end{bmatrix}
		: 0 \le \theta < 2\pi
	\right\} \subseteq \SL_2(\R).
\]

\begin{thm}
	Se cumplen:
	\begin{enumerate}
		\item Para todo $z \in \mathfrak{H}$ existe un $\gamma \in \SL_2(\R)$ tal que $\gamma\cdot\ui = z$.
		\item El homomorfismo $\iota$ induce un isomorfismo
			\[
				\GL_2^+(\R) / \R^\times \cong \SL_2(\R) / \{ \pm 1 \} \cong \Aut_{\mathsf{An}}(\mathfrak{H}).
			\]
		\item El grupo simétrico ortogonal $\SO_2(\R) = \{ \gamma \in \SL_2(\R) : \gamma\cdot\ui = \ui \}$ es el estabilizador del $\ui$.
	\end{enumerate}
\end{thm}
De esto extraemos dos consecuencias:
\begin{cor}
	El espacio de clases laterales (derechas) $\SL_2(\R)/\SO_2(\R)$ es homeomorfo a $\mathfrak{H}$.
\end{cor}
\begin{proof}
	Se sigue de aplicar el inciso 3 del teorema anterior al\break teorema~\ref{thm:topological_orbit_stab}.
\end{proof}
\begin{cor}
	La acción (canónica) de un subgrupo $\Gamma \le \Aut_{\mathsf{An}}(\mathfrak{H})$ es propiamente discontinua sobre $\mathfrak{H}$
	syss $\Gamma$ es un subgrupo discreto de $\SL_2(\R)$.
\end{cor}
\begin{proof}
	Se sigue de aplicar el inciso 2 del teorema anterior al\break teorema~\ref{thm:proper_discont_action}.
\end{proof}

Para la siguiente definición, nótese que si $\gamma \in \GL_2(\R)$ es un elemento escalar, entonces fija a todo $\mathfrak{H}$.
\begin{mydef}
	Sea $\gamma
	% := \begin{bsmallmatrix}
	% 	a & b \\
	% 	c & d
	% \end{bsmallmatrix}
	\in \GL_2(\R) \setminus \R^\times$ una matriz inversible no escalar, entonces su polinomio característico es
	\[
		\psi(t) = \psi_\gamma(t) := t^2 - \tr(\gamma) \, t + \det\gamma \in \R[t].
	\]
	el cual tiene discriminante $\delta := \tr(\gamma)^2 - 4\det\gamma$.
	Decimos que $\gamma$ es un elemento \strong{elíptico}\index{elíptico!(elemento de $\GL_2(\R)$)}
	(resp.\ \strong{parabólico}\index{parabólico!(elemento de $\GL_2(\R)$)}, \strong{hiperbólico}\index{hiperbólico!(elemento de $\GL_2(\R)$)})
	si $\delta < 0$ (resp.\ $\delta = 0$, $\delta > 0$).

	Dado un subgrupo $\Gamma \le \GL_2^+(\R)$ y un par de puntos $x, y \in \PP^1(\C)$,
	denotaremos por $\Gamma_x := \Stab_x \cap \Gamma$ al estabilizador de $x$ respecto a la acción de $\Gamma$
	y por $\Gamma_{x, y} := \Gamma_x \cap \Gamma_y$.
	Si $x \in \R \cup \{ \infty \}$, entonces denotaremos por $\Gamma_x^{(p)}$ al subgrupo de estabilizadores parabólicos en $\Gamma_x$.
\end{mydef}
Como las raíces del polinomio característico corresponden a valores propios, vemos lo siguiente:
\begin{cor}
	Sea $\gamma \in \GL_2(\R) \setminus \R^\times$ no escalar y considere la acción analítica $\GL_2(\R) \acts \PP^1(\C)$.
	\begin{enumerate}
		\item $\gamma$ es elíptico syss el endomorfismo de $\gamma$ (sobre $\PP^1(\C)$) tiene exactamente dos puntos fijos $z_0$ y $\overline{z}_0$,
			con $z_0 \in \mathfrak{H}$.
		\item $\gamma$ es  parabólico syss $\gamma$ tiene exactamente  un punto fijo      $z \in \R \cup \{ \infty \}$.
		\item $\gamma$ es hiperbólico syss $\gamma$ tiene exactamente dos punto fijos, ambos en $\R \cup \{ \infty \}$.
	\end{enumerate}
\end{cor}
\begin{ex}
	En $\SL_2\R$ hay dos matrices distinguidas con las que trabajaremos bastante,
	llamadas la \strong{matriz de traslación} y \strong{matriz de salto} resp.:
	\[
		T :=
		\begin{bmatrix}
			1 & 1 \\
			0 & 1
		\end{bmatrix}, \qquad
		S :=
		\begin{bmatrix}
			0 & -1 \\
			1 & 0
		\end{bmatrix}.
	\]
	Calculando el polinomio característico, es fácil comprobar que $T$ es parabólica y que $S$ es elíptica;
	pero podemos ser más específicos y notar que el único punto fijo de $T$ es $\infty$, mientras que $S$ fija a los puntos $\pm\ui$.
\end{ex}

\begin{prop}\label{thm:stab_calculations}
	Se cumplen:
	\begin{align*}
		\GL_2^+(\R)_\ui &= \R^\times \cdot \SO_2\R, \\
		\GL_2^+(\R)_\infty &= \left\{
			\begin{bmatrix}
				a & b \\
				0 & d
			\end{bmatrix} : a, d \in \R^\times, \; b \in \R, \; ad > 0
		\right\}, \\
		\GL_2^+(\R)_\infty^{(p)} &= \left\{
			\begin{bmatrix}
				a & b \\
				0 & a
			\end{bmatrix} : a \in \R^\times, \; b \in \R
		\right\}, \\
		\GL_2^+(\R)_{\infty, 0} &= \left\{
			\begin{bmatrix}
				a & 0 \\
				0 & d
			\end{bmatrix} : a, d \in \R^\times, \; ad > 0
		\right\}.
	\end{align*}
	Además, dado $z \in \mathfrak{H}$ se cumple que $\GL_2^+(\R)_z$ es conjugado de $\GL_2^+(\R)_\ui$
	y dados $x, y \in \R\cup\{ \infty \}$ se cumple que $\GL_2^+(\R)_x, \GL_2^+(\R)_x^{(p)}$ y $\GL_2^+(\R)_{x, y}$
	son conjugados de $\GL_2^+(\R)_\infty, \GL_2^+(\R)_\infty^{(p)}$ y $\GL_2^+(\R)_{\infty, 0}$ resp., por un elemento en $\SL_2\R$.
\end{prop}
\begin{proof}
	Las primeras igualdades son un mero ejercicio.
	La conjugación se deduce de que, dado $z \in \mathfrak{H}$, existe $\gamma \in \SL_2\R$ tal que $\gamma\cdot z = \ui$;
	y dados $x, y \in \R \cup \{ \infty \}$, existe $\alpha \in \SL_2\R$ tal que $\alpha\cdot x = \infty$ y $\alpha\cdot y = 0$.
\end{proof}

\begin{mydefi}
	Un \strong{grupo fuchsiano}\index{grupo!fuchsiano} $\Gamma$ es un subgrupo discreto de $\SL_2(\R)$,
	siempre dotado de su acción canónica $\Gamma \acts \mathfrak{H}$; se denota
	$$ Z(\Gamma) := \Gamma \cap \{ \pm 1 \}. $$
	Un punto $z \in \mathfrak{H} \cup \R \cup \{ \infty \}$ se dice \strong{elíptico}\index{elíptico!(elemento de $\mathfrak{H}^*$)}
	(resp.\ \strong{parabólico}\index{parabólico!(elemento de $\mathfrak{H}^*$)}, \strong{hiperbólico}\index{hiperbólico!(elemento de $\mathfrak{H}^*$)})
	relativo a $\Gamma$ si existe un elemento $\gamma$ elíptico (resp.\ parabólico, hiperbólico) de $\Gamma$ tal que $z$ es un punto fijo de $\gamma$.
	Los puntos parabólicos respecto a $\Gamma$ se dicen \strong{cúspides}\index{cúspide (grupo fuchsiano)} de $\Gamma$.
	% Dados dos puntos $z, w \in \mathfrak{H} \cup \R \cup \{ \infty \}$, denotaremos por $\Gamma_z$ el estabilizador de $z$ por $\Gamma \acts \mathfrak{H}$
	% y por $\Gamma_{z, w} := \Gamma_z \cap \Gamma_w$.
\end{mydefi}
El siguiente ejemplo es central en la teoría:
\begin{ex}
	$\SL_2\Z$ es un grupo fuchsiano.
	Esto se puede comprobar, por ejemplo, así:
	la topología de $\SL_2(\R)$ es la subespacio de $\Mat_2(\R) \approx \R^4$ y
	$\SL_2\Z$ es un subespacio del subgrupo discreto $\Z^4 \le_f \R^4$.
\end{ex}

\begin{thm}
	Sea $\Gamma$ un grupo fuchsiano.
	\begin{enumerate}
		\item Dado un punto elíptico $z \in \mathfrak{H}$   de $\Gamma$, entonces $\Gamma_z$ es un grupo cíclico finito.
		\item Dada una cúspide $x \in \R \cup \{ \infty \}$ de $\Gamma$, entonces $\Gamma_x \subseteq \SL_2(\R)_x^{(p)}$ y
			\[
				\Gamma_x / Z(\Gamma) \cong \Z.
			\]
			Además, dado $\sigma \in \SL_2(\R)$ tal que $\sigma \cdot x = \infty$, se cumple que
			\[
				\sigma \Gamma_x \sigma^{-1}\cdot \{ \pm 1 \} = \left\{ \pm
					\begin{bmatrix}
						1 & h \\
						0 & 1
					\end{bmatrix}^m : m \in \Z, h > 0
				\right\}.
			\]
		\item Si $\Gamma_{x, y} \ne Z(\Gamma)$ para $x \ne y \in \R \cup \{ \infty \}$, se verifica que
			\[
				\Gamma_{x,y} / Z(\Gamma) \cong \Z.
			\]
	\end{enumerate}
\end{thm}
\begin{proof}
	\begin{enumerate}
		\item Como $\SL_2\R$ actúa transitivamente en $\mathfrak{H}$,
			vemos que $\SL_2(\R)_z$ es conjugado al estabilizador $\SL_2(\R)_\ui = \SO_2(\R)$, el cual es un grupo abeliano compacto
			ya que es topológicamente isomorfo a $\SS^1 \le \C^\times$.
			Luego $\Gamma_z = \Gamma \cap \SL_2(\R)_z$ es compacto y discreto, ergo, finito y todo subgrupo finito de $\SS^1$ es cíclico.
		\item Tomando conjugados podemos suponer que $x = \infty$ es una cúspide de $\Gamma$.
			Así, debe existir $\gamma =
			\begin{bsmallmatrix}
				1 & h \\
				0 & 1
			\end{bsmallmatrix} \in \Gamma_\infty$ con $h \ne 0$ (por la prop.~\ref{thm:stab_calculations}).

			Supongamos que existe $\alpha \in \Gamma_x$ que no es ni escalar ni parabólico,
			entonces es de la forma $\alpha = 
			\begin{bsmallmatrix}
				a & b \\
				0 & a^{-1}
			\end{bsmallmatrix}$ con $a \ne \pm 1$.
			Sustituyendo quizá $\alpha$ por $\alpha^{-1}$ podemos suponer que $|a| < 1$ y, por lo tanto,
			\[
				\alpha^n \gamma \alpha^{-n} =
				\begin{bmatrix}
					1 & a^{2n}h \\
					0 & 1
				\end{bmatrix} \in \Gamma
			\]
			para todo $n \in \N$, pero estos se acumulan en $I_2 \in \Gamma$, lo que contradice que $\Gamma$ sea discreto.

			Finalmente, tenemos que
			\[
				\Gamma_\infty \le \SL_2(\R)^{(p)}_\infty = \left\{ \pm
					\begin{bmatrix}
						1 & b \\
						0 & 1
					\end{bmatrix} : b \in \R 
				\right\},
			\]
			y los únicos subgrupos discretos de $\SL_2(\R)^{(p)}_\infty$ son los de la forma del enunciado.

		\item Tras conjugar nos reducimos al caso de $\Gamma_{\infty, 0}$ y notamos que
			\[
				\Gamma_{\infty, 0} \le \SL_2(\R)_{\infty, 0} = \left\{
					\begin{bmatrix}
						a & 0 \\
						0 & a^{-1}
					\end{bmatrix} : a \in \R^\times
				\right\},
			\]
			donde el grupo de la derecha es topológicamente isomorfo a $\R^\times$.
			Luego, notamos que los únicos subgrupos discretos no triviales de $\R^\times / \{ \pm 1 \}$ son
			los cíclicos infinitos $\{ a^m : m \in \Z \}$.
			\qedhere
	\end{enumerate}
\end{proof}

\begin{cor}
	Si $\Gamma$ es un grupo fuchsiano y $\Gamma' \le \Gamma$ es un subgrupo de índice finito,
	entonces las cúspides de $\Gamma'$ son exactamente las misma de $\Gamma$.
\end{cor}
\begin{proof}
	En el teorema anterior concluimos que $x \in \R\cup\{ \infty \}$ es una cúspide syss su estabilizador $\Gamma_x$ es infinito,
	lo cual no cambia pasando a un subgrupo de índice finito.
\end{proof}

\begin{mydef}
	Sea $\Gamma$ un grupo fuchsiano.
	El conjunto de cúspides de $\Gamma$ lo denotaremos $P_\Gamma$ y denotaremos también:
	\[
		\mathfrak{H}^* = \mathfrak{H}^*_\Gamma := \mathfrak{H} \cup P_\Gamma.
	\]
	Dado $r > 0$ real, definimos
	\[
		U_r := \{ z \in \mathfrak{H} : \Im z > r \}, \qquad U_r^* := U_r \cup \{ \infty \}.
	\]
	Dotamos a $\mathfrak{H}^*$ de la siguiente topología: para un punto $z \in \mathfrak{H}$ consideramos una base de entornos usual de $\mathfrak{H}$
	y para una cúspide $x \in P_\Gamma$ consideramos la base de entornos de la forma $\sigma^{-1}U_r^*$, donde $r > 0$ y $\sigma$ recorre los elementos
	de $\SL_2(\R)$ tales que $\sigma\cdot x = \infty$.
\end{mydef}
\begin{prop}
	Para todo grupo fuchsiano $\Gamma$, el espacio $\Gamma \coquot \mathfrak{H}^*$ es de Hausdorff.
\end{prop}

\begin{mydef}
	% Denotemos por $\pi \colon \mathfrak{H}^* \epicto \Gamma \coquot \mathfrak{H}^*$ la identificación.
	Un elemento de $\Gamma \coquot \mathfrak{H}^*$ se dice un \strong{punto elíptico} (resp.\ una \strong{cúspide})
	si es la imagen de un punto elíptico (resp.\ una cúspide).
	Los puntos que no son ni elípticos ni cúspides se dicen \strong{ordinarios}\index{punto!ordinario}\index{ordinario (punto)}.

	Un grupo fuchsiano $\Gamma$ se dice \strong{de primer tipo}\index{grupo!fuchsiano!de primer tipo} si $\Gamma \coquot \mathfrak{H}^*$ es compacto.
\end{mydef}
\begin{thm}
	Si $\Gamma$ es un grupo fuchsiano de primer tipo, entonces $\Gamma \coquot \mathfrak{H}^*$ contiene solo finitos puntos elípticos y cúspides.
\end{thm}

\subsection*{$\Gamma \coquot \mathfrak{H}^*$ como superficie de Riemann}
Recuérdese que una \strong{variedad analítica} de dimensión $n$ (sobre $\C$) es una variedad topológica de dimensión $2n$
dotada de una \emph{estructura analítica}, vale decir, de un atlas maximal cuyos mapas de transición son biholomorfos.
Una \emph{superficie de Riemann} es una curva analítica compacta.

Que $\Gamma \coquot \mathfrak{H}^*$ sea una variedad topológica de dimensión 2 es un ejercicio para el lector, de modo que nos preocuparemos
de explicitar un atlas (que se extenderá a una estructura analítica).
Denotemos por $\varphi \colon \mathfrak{H}^* \epicto \Gamma \coquot \mathfrak{H}^*$ la proyección canónica,
para un punto $z \in \mathfrak{H}^*$ fijo denotaremos por $U_z \subseteq \mathfrak{H}^*$ un entorno abierto tal que
\[
	\Gamma_z = \{ \gamma \in \Gamma : \gamma\cdot U \cap U = \emptyset \}.
\]
Esto induce una inyección canónica $\Gamma_z \coquot U_z \hookto \Gamma \coquot \mathfrak{H}^*$,
mediante la cual $\Gamma_z \coquot U_z$ es un entorno abierto de $z$.
\begin{itemize}
	\item Si $z$ es un punto ordinario, entonces $\Gamma_z = Z(\Gamma)$ y $U_z \epicto \Gamma_z \coquot U_z$ es un homeomorfismo;
		de modo que declaramos que $(\Gamma_z \coquot U_z, \varphi^{-1})$ es una carta de $z$.
	\item Si $z$ es un punto elíptico...
\end{itemize}

\section{Formas automorfas}
\begin{mydef}
	Sea $f \colon \mathfrak{H} \to \PP^1(\C)$ una función meromorfa y sea $k \in \Z$ un entero.
	Dado $\gamma :=
	\begin{bsmallmatrix}
		a & b \\
		c & d
	\end{bsmallmatrix} \in \GL_2^+(\R)$, se define
	\[
		(f|_k \gamma)(z) := \frac{\det(\gamma)^{k/2} f(\gamma\cdot z)}{cz + d}.
	\]
\end{mydef}
\begin{cor}
	Sea $f \colon \mathfrak{H} \to \PP^1(\C)$ una función meromorfa y sea $k \in \Z$ un entero. Entonces:
	\begin{enumerate}
		\item Para todo $\alpha, \beta \in \GL_2^+(\R)$ se cumple que $f|_k \alpha \beta = (f|_k \alpha)|_k \beta$.
		\item Para una matriz escalar $\gamma = \lambda I_2 \in \GL_2^+(\R)$ (con $\lambda \in \R^\times$),
			se cumple que $f|_k \gamma = (\sign\gamma)^k f$.
	\end{enumerate}
\end{cor}

\begin{mydefi}
	Sea $\Gamma$ un grupo fuchsiano.
	Se dice que una función meromorfa $f \colon \mathfrak{H} \to \PP^1(\C)$ es una \strong{forma automorfa}\index{forma!automorfa} de peso $k$
	con respecto a $\Gamma$ o una \strong{$\Gamma$-forma automorfa} de peso $k$ si
	$$ \forall \gamma \in \Gamma \qquad f|_k \gamma = f. $$
	Denotaremos por $\Omega_k(\Gamma)$ al conjunto de $\Gamma$-formas automorfas de peso $k$.
\end{mydefi}
\begin{cor}
	Sea $\Gamma$ un grupo fuchsiano y sea $k \in \Z$. Entonces:
	\begin{enumerate}
		\item $\Omega_k(\Gamma)$ es un $\C$-subespacio vectorial de $C^{\rm an}(\mathfrak{H}, \PP^1(\C))$.
		\item Si $\Gamma' \subseteq \Gamma$ es otro grupo fuchsiano, entonces $\Omega_k(\Gamma') \supseteq \Omega_k(\Gamma)$.
		\item Dados $f \in \Omega_k(\Gamma)$ y $g \in \Omega_l(\Gamma)$ (con $l \in \Z$),
			entonces $f\cdot g \in \Omega_{k+l}(\Gamma)$.
		\item Si $k$ es impar y $-1 \in \Gamma$, entonces $\Omega_k(\Gamma) = \{ 0 \}$.
	\end{enumerate}
\end{cor}

\printbibliography[segment=\therefsegment, check=onlynew, notcategory=historical]
% \bibbycategory[segment=\therefsegment, check=onlynew]

\end{document}
