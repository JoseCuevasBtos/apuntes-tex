\documentclass[teoria-numeros.tex]{subfiles}
\begin{document}

\chapter{Fracciones continuas}
Ya habíamos visto en el primer capítulo la idea de la representación de números por bases, pero ésto tiene un par de problemas:
El primero es que elegir algo como el número 10 es considerablemente arbitrario, podríamos haber elegido 2 ó cualquier otro en éste sentido.
El segundo problema es que, como vimos, las representaciones no son únicas, pues $0.\overline{9} = 1$, y ésto aplica para cualquier base empleando el último
dígito, por ejemplo, en base 5 se cumple que $( 0.\overline{4} )_5 = 1$ y así, por ende, podríamos sospechar que, si en lugar de emplear una base finita
utilizaramos una <<base infinita>>, dado que no hay un último dígito el problema se vería resuelto.

\section{Introducción}
\nocite{khinchin:fractions}
\begin{mydefi}[Fracción continua]\index{fracción continua}
	Dadas dos sucesiones $(a_n)_{n=0}, \break (b_n)_{n = 1}^\infty \in \C$ se define la siguiente sucesión $(c_n)_{n=0}^\infty \in \C$:
	$$ c_0 := a_0, \quad c_1 := a_0 + \dfrac{b_1}{a_1}, \quad c_2 := a_0 + \dfrac{b_1}{a_1 + \dfrac{b_2}{a_2}}, \; \cdots $$
	a la sucesión $(c_n)_{n\in\N}$ se le dicen \strong{aproximantes}.%
	\footnote{Khinchin sugiere los términos \textit{convergentes} y \textit{aproximantes}, preferimos el segundo para evitar confusiones,
	aunque otros autores en español tienden a optar por el primero.} 
	También añadimos un número $\infty$ (sin signo) que satisface lo siguiente:
	$$ \frac{z}{0} = \infty, \qquad \frac{z}{\infty} = 0, \qquad a \pm \infty = \infty $$
	Finalmente, si $c_n$ converge a un límite $L$ se admiten las siguientes notaciones:
	$$ L = a_0 + \dfrac{b_1}{a_1 + \dfrac{b_2}{a_2 + \dfrac{b_3}{a_3 + \cdots}}} = a_0 + \frac{b_1}{a_1} {{}\atop +} \frac{b_2}{a_2} {{}\atop + \cdots}
	= a_0 + \ksum_{n=1}^\infty \frac{b_n}{a_n} $$
	\nomenclature{$a_0 + \ksum_{n=1}^\infty \frac{b_n}{a_n}$}{Notación de Gauss de fracción continua,
	$\displaystyle {}=a_0 + \frac{b_1}{a_1 + \frac{b_2}{a_2 + \cdots}} = a_0 + \frac{b_1}{a_1} {{}\atop +} \frac{b_2}{a_2} {{}\atop + \cdots}$}
	La última es notación de Gauss y favoreceremos ésta.
	% Además, en varios casos la sucesión $b_n$ tendrá valor constante 1, y ahí se admite la siguiente notación:
	% $$ a_0 + \ksum_{n=1}^\infty \frac{1}{a_n} = [a_0; a_1, a_2, a_3, \dots]. $$
	% A las fracciones continuas de ésta forma, donde todos los $(a_n)_{n\in\N}$ son naturales no nulos (y donde se admite que la sucesión sea finita)
	% se les dicen \textit{fracciones continuas simples}\index{fracción continua!simple}.
	% \nomenclature{$[a_0; a_1, a_2, \dots]$}{Notación de Khinchin de fracción continua, ${} = a_0 + \ksum_{n=1}^\infty \frac{1}{a_n}$}
\end{mydefi}

Las notaciones están hechas para abreviar espacio, pero no dicen mucho sobre la sucesión real, pues notemos que a pesar de su parecido con la suma,
aquí el orden sí importa y deben hacerse desde derecha a izquierda, por ejemplo:
\begin{align*}
	\frac{1}{2} {\atop +} \left( \frac{1}{3} {\atop +} \frac{1}{5} \right) &= \frac{1}{2} {\atop +} \frac{1}{3 + \frac{1}{5}} \\
									       &= \frac{1}{2} {\atop +} \frac{5}{16} = \frac{1}{2 + \frac{5}{16}} = \frac{16}{37}
\end{align*}
mientras que
\begin{align*}
	\left( \frac{1}{2} {\atop +} \frac{1}{3} \right) {\atop +} \frac{1}{5} &= \frac{1}{2 + \frac{1}{3}} {\atop +} \frac{1}{5} \\
									       &= \frac{3}{7} {\atop +} \frac{1}{5} = \frac{3}{7 + \frac{1}{5}} = \frac{15}{36}.
\end{align*}

\begin{thm}
	Dadas las sucesiones $(a_n)_{n=0}^\infty, (b_n)_{n=1}^\infty$, entonces sus aproximantes $(c_n)_{n=0}^\infty$ se pueden
	calcular mediante la fórmula $c_n = p_n/q_n$, donde éstos se calcular recursivamente por
	\begin{align*}
		p_{-1} = 1, \quad p_0 = a_0, &\qquad q_{-1} = 0, \quad q_0 = 1 \\
		p_n = a_np_{n-1} + b_np_{n-2}, &\qquad q_n = a_nq_{n-1} + b_nq_{n-2}
	\end{align*}
\end{thm}
\begin{proof}
	La demostración es por inducción:
	$$ c_0 = a_0 = \frac{a_0}{1} $$
	Y suponiendo que aplica para toda sucesión entonces se tiene que
	\begin{align*}
		c_{n+1} &= a_0 + \frac{b_1}{a_1} {\atop {}+\cdots+{}} \frac{b_n}{a_n} {\atop +} \frac{b_{n+1}}{a_{n+1}} \\
			&= a_0 + \frac{b_1}{a_1} {\atop {}+\cdots+{}} \frac{b_n}{a_n + \frac{b_{n+1}}{a_{n+1}}} \\
			&= a_0 + \frac{b_1}{a_1} {\atop {}+\cdots+{}} \frac{b_na_{n+1}}{a_na_{n+1} + b_{n+1}} = \frac{p_{n+1}}{q_{n+1}} = \frac{p^\prime_n}{q^\prime_n}.
	\end{align*}
	donde $p^\prime_n, q^\prime_n$ representan los valores dados por las nuevas sucesiones en donde el último término está cambiado.
	Por inducción se tiene que
	\begin{align*}
		p^\prime_n &= a^\prime_n p_{n-1} + b^\prime_n p_{n-2} = (a_n a_{n+1} + b_{n+1})p_{n-1} + b_na_{n+1}p_{n-2} \\
			   &= ( a_np_{n-1} + b_np_{n-2} ) a_{n+1} - b_np_{n-2}a_{n+1} + b_{n+1}p_{n-1} + b_na_{n+1}p_{n-2} \\
			   &= a_{n+1}p_n + b_{n+1}p_{n-1} = p_{n+1}.
	\end{align*}
	y reemplazando las $p$'s por $q$'s nos queda la misma igualdad para $q_{n+1}$.
\end{proof}

\begin{thm}
	Dadas las sucesiones $(a_n)_{n=0}^\infty, (b_n)_{n=1}^\infty$ y la sucesión $(r_n)_{n=0}^\infty \in \C_{\ne 0}$ con $r_0 = 1$, entonces
	$$ a_0 + \ksum_{n=1}^N \frac{b_n}{a_n} = r_0a_0 + \ksum_{n=1}^N \frac{r_{n-1}r_n b_n}{r_na_n}. $$
	de modo que una converge syss la otra lo hace y comparten límite.
\end{thm}
\begin{hint}
	Basta probar por inducción que sus aproximantes son de la forma $c_n = r_np_n / r_nq_n$.
\end{hint}

\begin{thmi}[Fracción continua de Euler]\index{fracción continua!de Euler}
	Sea $(t_n)_{n=0}^\infty \in \C_{\ne 0}$, entonces
	$$ \sum_{n=0}^N t_0\cdots t_n = \frac{t_0}{1} {\atop +} \frac{-t_1}{1 + t_1} {\atop {}+\cdots+{}} \frac{-t_n}{1 + t_n}, $$
	de modo que una converge syss la otra lo hace y comparten límite.
	Más aún se cumple que
	$$ 1 + \sum_{n=0}^N t_0\cdots t_n = \frac{1}{1} {\atop +} \frac{-t_0}{1 + t_0} {\atop +} \frac{-t_1}{1 + t_1} {\atop {}+\cdots}
	= \frac{1}{1} {\atop +} \ksum_{n=0}^N \frac{-t_n}{1 + t_n}. $$
\end{thmi}
\begin{proof}
	Vamos a probarlo por inducción sobre $N$.
	El caso $N = 0$ es trivial y
	$$ t_0 + t_0t_1 = t_0(1 + t_1) = \frac{t_0}{ \dfrac{1 + t_1 - t_1}{1 + t_1} } = \frac{t_0}{1 + \dfrac{-t_1}{1 + t_1}}
	= \frac{t_0}{1} {\atop +} \frac{-t_1}{1 + t_1}. $$
	Si asumimos que se cumple para $N$, notemos que
	$$ \sum_{n=0}^{N+1} t_0\cdots t_n = \sum_{n=0}^{N-1} t_0\cdots t_n + t_0\cdots t_N(1 + t_{N+1}). $$
	Si $t_{N+1} = -1$, entonces $a_{N+1} = 1 + t_{N+1} = 0$ y
	$$ c_{N+1} = \frac{a_{N+1}p_N + b_{N+1}p_{N-1}}{a_{N+1}q_N + b_{N+1}q_{N-1}} = \frac{p_{N-1}}{q_{N-1}} = c_{N-1}. $$
	Si $t_{N+1} \ne -1$, entonces $a_{N+1} \ne 0$ y aplicamos la hipótesis de inducción a la sucesión $t_0, \dots, t_{N-1}, t^\prime_N$, donde
	$t^\prime_N = t_N(1 + t_{N+1})$ y notamos que
	$$ \sum_{n=0}^{N+1} t_0\cdots t_n = \frac{t_0}{1} {\atop +} \frac{-t_1}{1 + t_1} {\atop {}+\cdots+{}} \frac{-t^\prime_N}{1 + t^\prime_N} $$
	donde
	\begin{align*}
		\frac{-t^\prime_N}{1 + t^\prime_N} &= \frac{-t_N(1 + t_{N+1})}{1 + t_N(1 + t_{N+1})} \\
						   &= \frac{-t_N}{1 + t_N + \frac{1}{1 + t_{N+1}} - 1} = \frac{-t_N}{1 + t_N} {\atop +} \frac{-t_{N+1}}{1 + t_{N+1}}.
	\end{align*}
	Para la segunda identidad basta notar que
	$$ 1 + \frac{t_0}{1 + F} = \frac{1 + F + t_0}{1 + F} = \frac{1}{1 - \dfrac{t_0}{1 + t_0 + F}} = \frac{1}{1} {\atop +} \frac{-t_0}{1 + t_0 + F} $$
	que aplica para todo $F$ (incluyendo $F = -1$ o $F = \infty$), así que basta reemplazar por el resto de términos para obtener la identidad buscada.
\end{proof}

\begin{ex}[ $\displaystyle \exp(x) = \frac{1}{1} {\atop +} \frac{-x/1}{1 + x/1} {\atop +} \frac{-x/2}{1 + x/2} {\atop +} \frac{-x/3}{1 + x/3} {\atop {}+\cdots} $ ]
	Nótese que ya habíamos visto que
	$$ \exp(x) = 1 + \sum_{n=1}^\infty \frac{x^n}{n!} $$
	dónde cada término en la sumatoria puede verse como la multiplicación de $x/n$, de modo que por la fracción continua de Euler se cumple que
	$$ \exp(x) = \frac{1}{ \displaystyle 1 + \ksum_{n=1}^\infty \frac{-x/n}{1 + x/n} }, \qquad \forall x\in\C_{\ne 0} $$
	También podemos emplear $-x$ para obtener y elevar ambos lados a $(-1)$ para tener que:
	$$ \exp(x) = 1 + \ksum_{n=1}^\infty \frac{x/n}{1 - x/n} $$
	Luego podemos ajustar según el teorema por $r_n = n$:
	$$ \exp(x) = 1 + \frac{x}{1 - x} {\atop +} \frac{x}{2 - x} {\atop +} \frac{2x}{3 - x} {\atop +} \frac{3x}{4 - x} {\atop {}+\cdots} $$
\end{ex}

\section{Fracciones continuas simples}
\begin{mydefi}
	Dada una sucesión $(a_n)_{n=0}$ posiblemente finita (por ello obviamos el punto de final) de números naturales no nulos, donde $a_0$ se permite como
	cualquier entero (incluyendo al cero) se denota
	$$ [a_0; a_1, a_2, \dots] = a_0 + \ksum_{n=1} \frac{1}{a_n} $$
	lo que se considera una \strong{fracción continua simple}\index{fracción continua!simple}.
	Claramente el punto-coma sirve para denotar una idea de parte entera.
\end{mydefi}
De momento las fracciones continuas simples son sólo un subconjunto de las generalizadas, pero veremos que son mucho más.

En primer lugar una consecuencia del primer teorema que vimos:
\begin{prop}
	Sea $[a_0; a_1, a_2, \dots]$, entonces sus aproximantes $(c_n)_{n=0}$ satisfacen que $c_n = p_n/q_n$ donde $p_n,q_n$ son enteros determinados
	recursivamente por:
	\begin{align*}
		p_{-1} = 1, \quad p_0 = a_0, &\qquad q_{-1} = 0, \quad q_0 = 1 \\
		p_n = a_np_{n-1} + p_{n-2}, &\qquad q_n = a_nq_{n-1} + q_{n-2}
	\end{align*}
\end{prop}

\begin{prop}\label{thm:approximants_closeness}
	Si $(c_n)_{n=0}$ son los aproximantes de $[a_0; a_1, a_2, \dots]$ y $c_n = p_n/q_n$, entonces se comprueba que $p_nq_{n+1} - p_{n+1}q_n = (-1)^{n+1}$,
	o equivalentemente,
	$$ c_n - c_{n+1} = \frac{(-1)^{n+1}}{q_nq_{n+1}}. $$
\end{prop}
\begin{proof}
	Naturalmente es por inducción, en el caso $n = 0$ se reduce a ver que
	$$ p_0q_1 - p_1q_0 = a_0a_1 - ( a_1a_0 + 1 ) = -1. $$
	Y el caso $n+1$ se da pues
	\begin{align}
		p_{n+1}q_{n+2} - p_{n+2}q_{n+1} &= p_{n+1}(a_{n+2}q_{n+1} - q_n) + (a_{n+2}p_{n+1} - p_n)q_{n+1} \notag \\
						&= p_{n+1}q_n - p_nq_{n+1} = (-1) (-1)^{n+1}. \tqedhere
	\end{align}
\end{proof}

\begin{cor}
	Toda fracción continua simple infinita converge.
\end{cor}

\begin{prop}\label{thm:real_simple_cont_frac_rep}
	Todo número real $x$ admite una representación como fracción continua simple:
	$$ x = [a_0; a_1, a_2, a_3, \dots] $$
	donde $t_0 := x$ y $a_0 = \sfloor{x}$ y el resto de dígitos se construye por recursión:
	$$ t_{n+1} :=  \frac{1}{t_n - a_n}, \quad a_{n+1} = \sfloor{t_{n+1}} $$
	donde se subentiende que la fracción termina en $a_n$ si $a_n = t_n$.
\end{prop}
Los términos $t_n$ se ven como desplazar la fracción continua:
$$ t_0 = [a_0; a_1, a_2, \dots], \quad t_1 := [a_1; a_2, a_3, \dots], \quad t_2 = [a_2; a_3, a_4 \dots], \quad \dots $$
% \begin{proof}
%	 Sea 
% \end{proof}

\begin{lem}
	Dada una fracción continua simple cuyos aproximantes son $(p_n/q_n)_{n=0}^\infty$, entonces la sucesión $q_n$ es estrictamente creciente.
\end{lem}

\begin{thmi}
	Un número es irracional syss tiene una fracción continua simple infinita asociada.
\end{thmi}
\begin{proof}
	$\implies$. Por contrarrecíproca queda que si un número tiene una fracción continua simple finita entonces es racional lo que es trivial.
	\par
	$\impliedby$.
	Sea $\alpha := [a_0; a_1, a_2, \dots]$ y supongamos por contradicción que converge a un número racional $p/q$ (siendo $p,q$ coprimos).
	Si $c_n = (p_n/q_n) \to p/q$ entonces por el lema existe $N$ tal que para todo $n \ge N$ se cumple que $q_n > q$.
	Luego se cumple que
	\begin{align*}
		| \alpha - c_n | &\le |c_{n+1} - c_n| = \frac{1}{q_nq_{n+1}} < \frac{1}{q_nq} \\
		| \alpha - c_n | &= \left| \frac{p}{q} - \frac{p_n}{q_n} \right| = \frac{|pq_n - p_nq|}{q_nq} \ge \frac{1}{q_nq}.
	\end{align*}
	donde $|pq_n - p_nq| \ge 1$ pues $c_n \ne \alpha$.
\end{proof}

\begin{thm}
	Un número irracional posee una única fracción continua simple asociada.
\end{thm}
\begin{proof}
	Sean $[a_0; a_1, \dots] = [b_0; b_1, \dots] =: x$ dos expansiones en fracción continua simple de un mismo número irracional $x$,
	luego $x$ no es entero y de hecho $a_0 < x < a_0 + 1$, luego
	$$ a_0 = \lfloor [a_0; a_1, a_2, \dots] \rfloor = \lfloor [b_0; b_1, b_2, \dots] \rfloor = b_0, $$
	y luego notamos que $\frac{1}{x - a_0} = [a_1; a_2, \dots] = [b_1; b_2, \dots]$ y empleamos el mismo truco inductivamente,
	notando que nunca puede ser entero porque $x$ es irracional, para concluir que $a_n = b_n$ para todo $n$.
\end{proof}
De éste modo podemos hablar de \textit{los} aproximantes de una fracción continua simple.
Ésto será útil pues claramente los aproximantes nos otorgan buenas aproximaciones racionales para números irracionales,
pero el recíproco también es cierto (cfr. teorema~\ref{thm:approximants_satisfy_approx}).

Uno puede mejorar el teorema anterior un poco.
En expansión base $b \ge 2$, veíamos que un número tenía una expansión periódica asociada si era racional;
vamos a responder el qué signfica tener expansión periódica, pero veamos un par de lemas primero.
% el siguiente teorema de \citeauthor{lehmer00continued}~\cite{lehmer00continued} responde la misma pregunta para fracciones continuas:
\begin{lem}
	Sea $\alpha$ un real irracional con $\alpha = [a_0; a_1, a_2, \dots]$ y sean $t_n$ como en la proposición~\ref{thm:real_simple_cont_frac_rep}.
	% y defínase
	% $$ \alpha^\prime_n := [a_n; a_{n+1}, a_{n+2}, \dots]. $$
	Entonces, para todo $n$ se tiene lo siguiente:
	$$ \alpha = t_0, \qquad \alpha = \frac{t_1 a_0 + 1}{t_1}, \qquad \alpha = \frac{t_n p_{n-1} + p_{n-2}}{t_n q_{n-1} + q_{n-2}}. $$
\end{lem}
\begin{hint}
	Aplique inducción.
\end{hint}

\begin{mydef}
	Una fracción continua simple $\alpha = [a_0; a_1, a_2, \dots]$ se dice \strong{eventualmente periódica}\index{eventualmente periódica (fracción continua)}
	si existen $n_0 \in \N$ y $m > 0$ tal que $a_{n + m} = a_n$ para todo $n \ge n_0$.
	Si $n_0 = 0$, decimos que (la fracción continua simple de) $\alpha$ es \strong{puramente periódica}\index{puramente periódica (fracción continua)}.
\end{mydef}
\begin{thm}
	Para un número irracional $\alpha$ son equivalentes:
	\begin{enumerate}
		\item Su fracción continua simple es eventualmente periódica.
		\item $\alpha$ es algebraico cuadrático, vale decir, es raíz de un polinomio cuadrático.
	\end{enumerate}
\end{thm}
\begin{proof}
	$1 \implies 2$.
	Sea
	\[
		\alpha = [b_0; b_1, \dots, b_m, a_1, a_2, \dots, a_n, a_1, \dots].
	\]
	Definiendo $\beta := [a_1, a_2, \dots]$ nótese que $t_n = \beta$ por hipótesis, de modo que el lema anterior nos da que
	$$ \beta = \frac{ \beta p_{n-1} + p_{n-2} }{ \beta q_{n-1} + q_{n-2} } \iff q_{n-1} \beta^2 + (q_{n-2} - p_{n-1}) \beta - p_{n-2} = 0, $$
	% la periodicidad nos da
	% $$ \beta = \frac{1}{a_1} {\atop +} \frac{1}{a_2} {\atop + \cdots +} \frac{1}{a_n} {\atop +} \frac{\beta}{1}. $$
	% Recuérdese que la notación no es conmutativa y se debe ir desarrollando de derecha a izquierda.
	% Con esa advertencia en mente, es fácil comprobar que en desarrollo sale $f(\beta)/g(\beta)$ donde $f, g \in \Z[x]$ son polinomios de grado 1 con coeficientes enteros,
	% Así que
	% $$ \beta = \frac{u + v \beta}{w + z\beta} \iff z \beta^2 + (w - v) \beta - u = 0, $$
	donde cada $p_i, q_i \in \Z$ son aproximantes, lo que prueba que $\beta$ es cuadrático.

	Realizando un procedimiento similar podemos despejar a $\alpha$ como $\alpha = f(\beta)/g(\beta)$ donde los polinomios son lineales,
	de modo que $\alpha \in \Q(\beta)$ y, siendo irracional, necesariamente es cuadrático también.

	$2 \implies 1$.
	Sea $\alpha$ con polinomio minimal $ax^2 + bx + c = a(x - \alpha)(x - \beta) \in \Z[x]$.
	Defínase
	$$ g(x, y) := ax^2 + bxy + cy^2 = (x, y) \cdot
	\begin{bmatrix}
		a & b/2 \\
		b/2 & c
	\end{bmatrix} \cdot
	\begin{pmatrix}
		x \\ y
	\end{pmatrix}. $$
	En notación de la proposición~\ref{thm:real_simple_cont_frac_rep} con $\kappa := (q_nt_{n+1} - q_{n-1})^{-1} \ne 0$,
	tenemos que
	$$ \begin{pmatrix}
		\alpha \\ 1
	\end{pmatrix} = \kappa \cdot
	\begin{bmatrix}
		p_n & p_{n-1} \\
		q_n & q_{n-1}
	\end{bmatrix} \cdot
	\begin{pmatrix}
		t_{n+1} \\ 1
	\end{pmatrix} =:
	\kappa \cdot M \cdot
	\begin{pmatrix}
		t_{n+1} \\ 1
	\end{pmatrix}. $$
	Luego si definimos la siguiente matriz
	$$ \begin{bmatrix}
		A_n & B_n/2 \\
		B_n/2 & C_n
	\end{bmatrix} := M^t \cdot
	\begin{bmatrix}
		a & b/2 \\
		b/2 & c
	\end{bmatrix} \cdot M, $$
	vemos que su determinante es el mismo que del polinomio $f(x)$, pues $|\det M| = 1$ por la proposición~\ref{thm:approximants_closeness}.

	Además:
	\begin{align*}
		A_nt_{n+1}^2 + B_nt_{n+1}^2 + C_n &= (t_{n+1}, 1) \cdot
		\begin{bmatrix}
			A_n & B_n/2 \\
			B_n/2 & C_n
			\end{bmatrix} \cdot \begin{pmatrix}
			t_{n+1} \\ 1
		\end{pmatrix} \\
						  &= \kappa^2 (\alpha, 1) \cdot
						  \begin{pmatrix}
							  a & b/2 \\
							  b/2 & c
							  \end{pmatrix} \cdot \begin{pmatrix}
							  \alpha \\ 1
						  \end{pmatrix} = \kappa^2 g(\alpha, 1) = 0.
	\end{align*}
	Luego $g_n(x) := A_nx^2 + B_nx + C_n$ tiene una raíz $t_{n+1}$.
	Ahora bien, $A_n = f(p_n, q_n)$ (¿por qué?) y, por lo tanto, $C_n = A_{n-1}$.
	Nótese que
	$$ \left| \beta - \frac{p_n}{q_n} \right| \le |\beta - \alpha| + \left| \alpha - \frac{p_n}{q_n} \right| \le
	|\beta - \alpha| + \frac{1}{q_n^2} \le |\beta - \alpha| + 1, $$
	de modo que para todo $n \ge 1$ se cumple que
	$$ |A_n| = |f(p_n, q_n)| = aq_n^2 \left| \alpha - \frac{p_n}{q_n} \right| \, \left| \beta - \frac{p_n}{q_n} \right| \le a(|\beta - \alpha| + 1)
	a + \sqrt{d}, $$
	donde $a|\beta - \alpha| = \sqrt{d}$ por la fórmula cuadrática.

	Como los $A_n$'s y $C_n$'s son enteros esto nos a lo sumo finitas soluciones, y como $B_n^2 = d + 4A_nC_n$ esto nos da finitas posibilidades para $B_n$
	también.
	Así, hay finitos polinomios $g_n$'s, cada uno con dos raíces, por lo tanto, hay alguna raíz que se repite infinitas veces, digamos $t_m = t_n$
	con $m < n$. Pero la expansión de $t_m = [a_m; a_{m+1}, \dots, a_{n-1}, t_n]$, así que la fracción continua simple de $t_m$ es periódica.
\end{proof}

% Así resolvimos cuando un irracional tiene fracción continua simple eventualment
\begin{prop}
	Sea $\alpha \in \Q(\sqrt{d})$ un número real cuadrático y denotemos por $\overline{()}$ la conjugación.
	Entonces la fracción continua simple de $\alpha$ es puramente periódica syss $\alpha > 1 > -\overline{\alpha} > 0$.
\end{prop}
\begin{proof}
	$\implies$. Escribamos $\alpha = [ \overline{a_0; a_1, \dots, a_m} ]$, donde la barra denota que $a_{m+1} = a_0$ y así (¡no confundir con conjugación!).
	Por definición de fracción continua simple tenemos que $\lfloor \alpha \rfloor = a_0 = a_{m+1} > 0$, por lo que $\alpha > 1$.
	Empleando el lema obtenemos que
	$$ \alpha = \frac{\alpha p_{m-1} + p_{m-2}}{\alpha q_{m-1} + q_{m-2}}, $$
	de modo que definiendo $f(x) := q_{m-1}x^2 + (q_{m-2} - p_{m-1})x - p_{m-2}$, tenemos que $f(\alpha) = 0 = f(\overline{\alpha})$.
	Basta notar que $f(0) = -p_{m-2} < 0$ y que
	$$ f(-1) = (q_{m-1} - q_{m-2}) + (p_{m-1} - p_{m-2}) > 0, $$
	de modo que, por el teorema del valor intermedio, $\overline{\alpha} \in (-1, 0)$ como se quería probar.

	$\impliedby$. Denotemos por $\beta := -\overline{\alpha}$, por $\alpha_n$ el real tal que $\alpha = [a_0; \dots, a_{n-1}, \alpha]$
	y $\beta_n$ lo mismo con $\beta$.
	Nótese que $\alpha_n > a_n \ge 1$ y veamos que $0 < \beta_n < 1$ por inducción:
	el caso base $n = 0$ es obvio y tomando conjugados vemos que
	$$ \alpha_n = a_n + \frac{1}{\alpha_{n+1}} \iff -\beta_n = a_n - \frac{1}{\beta_{n+1}}. $$
	Luego, por hipótesis inductiva, $a_n$ es un entero en $( 1/\beta_{n+1} - 1, 1/\beta_{n+1} )$, luego $a_n = \lfloor 1/\beta_{n+1} \rfloor$,
	de modo que $1/\beta_{n+1} > 1$ como se quería ver.

	Ahora, como la fracción continua de $\alpha$ es eventualmente periódica, tenemos que $\alpha_n = \alpha_m$ para algunos $n < m$,
	donde elegimos $n > 0$ minimal.
	Tomando conjugados tenemos que $\beta_n = \beta_m$ y, por tanto, $a_{n-1} = \lfloor 1/\beta_n \rfloor = \lfloor 1/\beta_m \rfloor = a_{m-1}$,
	de lo que se sigue que $\alpha_{n-1} = \alpha_{m-1}$ lo que contradice la minimalidad de $n$.
	Es decir, necesariamente $n = 0$.
\end{proof}

Un teorema de \citeauthor{lehmer00continued}~\cite{lehmer00continued} dice que además, la fracción es capicúa (i.e., se lee igual al derecho que al revés)
\addtocategory{article}{lehmer00continued}

\subsection*{Expansión de $e$}
Veamos que $e$ es irracional deduciendo que su fracción continua simple es
$$ e = [2; 1, 2, 1, 1, 4, 1, 1, 6, 1, \dots] = [1; 0, 1, 1, 2, 1, 1, 4, 1, \dots]; $$
éste es el método descrito por \citeauthor{cohn:expansion_of_e}~\cite{cohn:expansion_of_e}.
Sabemos, por el teorema principal, que el patrón en las aproximantes de la fracción continua de la derecha es
\begin{align*}
	p_{3n}   &= p_{3n-1} + p_{3n-2}, &\quad q_{3n}   &= q_{3n-1} + q_{3n-2} \\
	p_{3n+1} &= 2np_{3n} + p_{3n-1}, &\quad q_{3n+1} &= 2nq_{3n} + q_{3n-1} \\
	p_{3n+2} &= p_{3n+1} + p_{3n},   &\quad q_{3n+2} &= q_{3n+1} + q_{3n}
\end{align*}

Definamos las siguientes integrales:
\begin{align*}
	A_n &:= \int_0^1 e^x \frac{x^n (x - 1)^n}{n!} \, \ud x \\
	B_n &:= \int_0^1 e^x \frac{x^{n+1} (x - 1)^n}{n!} \, \ud x \\
	C_n &:= \int_0^1 e^x \frac{x^n (x - 1)^{n+1}}{n!} \, \ud x
\end{align*}

\begin{lem}
	Para todo $n\ge 0$ se cumple que
	$$ A_n = q_{3n}e - p_{3n}, \; B_n = p_{3n+1} - q_{3n+1}e, \; C_n = p_{3n+2} - q_{3n+2}e. $$
\end{lem}
\begin{proof}
	Aplicaremos inducción sobre $n$:
	\underline{Caso $n = 0$:}
	Notemos que $p_0/q_0 = 1/1, p_1/q_1 = 1/0$ y que $p_2/q_2 = 2/1$ así que para aplicar inducción basta probar que
	$$ A_0 = \int_0^1 e^x \, \ud x = e - 1, \quad B_0 = \int_0^1 xe^x \, \ud x = [ xe^x ]_0^1 - \int_0^1 e^x \, \ud x = e - (e - 1) = 1. $$
	y el caso restante saldrá de un caso probado más abajo.
	\underline{Caso $n+1$:} Comenzaremos por ver un par de identidades
	\begin{align*}
		B_n &= \int_0^1 x^{n+1}e^x \frac{(x-1)^n}{n!} \, \ud x \\
		    &= \left[ x^{n+1}e^x \frac{(x-1)^{n+1}}{(n+1)!} \right]_0^1 - \int_0^1 \frac{(x-1)^{n+1}}{(n+1)!} ( (n+1)x^ne^x + x^{n+1}e^x ) \, \ud x \\
		    &= -\int_0^1 \frac{e^xx^n(x-1)^{n+1}}{n!} \, \ud x - \int_0^1 \frac{e^x x^{n+1}(x-1)^{n+1}}{(n+1)!} \, \ud x = -C_n - A_{n+1}. \\
		B_n + C_n &= \int_0^1 \frac{e^x x^n (x-1)^n}{n!}(2x - 1) \, \ud x = 2B_n - A_n,
	\end{align*}
	las cuales se reordenan a $A_{n+1} = -B_n - C_n$ y $C_n = B_n - A_n$
	(nótese que de la última se deduce que $C_0 = B_0 - A_0 = 2 - e$).
	La relación $B_n = -2nA_n + C_{n-1}$ sale de que
	% \begin{align*}
	%	 \frac{\ud}{\ud x}\left( e^x\frac{x^n(x - 1)^{n+1}}{n!} \right) &= \frac{\ud}{\ud x} \left( e^x \frac{x^{n+1}(x - 1)^n - x^n(x - 1)^n}{n!} \right) \\
	%	 &= e^x\frac{x^{n+1}(x - 1)^n - x^n(x - 1)^n}{n!} + e^x \frac{(n+1)x^n(x - 1)^n + nx^{n+1}(x - 1)^{n-1} - nx^{n-1}(x-1)^n - nx^n(x-1)^{n-1}}{n!}
	% \end{align*}
	$$ \frac{\ud}{\ud x}\left( e^x\frac{x^n(x - 1)^{n+1}}{n!} \right)
	= {\color{red} e^x \frac{x^n(x - 1)^{n+1}}{n!} } + {\color{blue} e^x \frac{x^{n-1}(x - 1)^{n+1}}{(n-1)!} } + e^x \frac{(n+1)x^n(x - 1)^n}{n!} $$
	Nótese que al término en azul podemos factorizar un término $(x - 1)$ para obtener
	$$ e^x \frac{x^{n-1}(x - 1)^{n+1}}{(n-1)!} = -e^x\frac{x^{n-1}(x - 1)^n}{(n-1)!} + e^x \frac{n x^n (x - 1)^n}{n!} $$
	Lo mismo se puede hacer en el término en rojo lo que da
	$$ e^x \frac{x^n(x - 1)^{n+1}}{n!} = e^x \frac{x^{n+1}(x-1)^n}{n!} - e^x\frac{x^n(x-1)^n}{n!} $$
	Reemplazando todo en la ecuación original nos queda
	$$ \frac{\ud}{\ud x}\left( e^x\frac{x^n(x - 1)^{n+1}}{n!} \right)
	= e^x \frac{x^{n+1}(x-1)^n}{n!} + (2n) e^x \frac{x^n(x - 1)^n}{n!} -e^x\frac{x^{n-1}(x - 1)^n}{(n-1)!} $$
	Integrando a ambos lados nos queda la relación restante.
	Con ellas se puede deducir el caso inductivo.
\end{proof}

\begin{thm}
	\addexample{$e = [2; 1, 2, 1, 1, 4, 1, 1, 6, 1, 1, 8, 1, \dots]$}
	Se cumple que
	$$ e = [1; 0, 1, 1, 2, 1, 1, 4, 1, 1, 6, 1, \dots] = [2; 1, 2, 1, 1, 4, 1, 1, 6, 1, \dots]. $$
	En consecuencia, $e$ es irracional.
\end{thm}
\begin{hint}
	Basta comprobar que $A_n, B_n, C_n \to 0$.
\end{hint}
La proposición~\ref{thm:irrational_numbers_thr_approx} da una demostración más sencilla de la irracionalidad de $e$, pero
nuestro teorema da otra clase de información interesante.
Una consecuencia es que $e$ tampoco es algebraico cuadrático, pues su fracción continua simple no es periódica.

\section{Aproximaciones diofánticas y la ecuación de Pell}%
\label{sec:basic_diophantine_approx}

\begin{thmi}[Teorema de aproximación de Dirichlet]\index{teorema!de aproximación!de Dirichlet}
	Sea $\alpha$ un número irracional y $M > 0$. Entonces existen $u, v \in \Z$ con $0 < v \le M$ tales que
	$$ \left| \alpha - \frac{u}{v} \right| < \frac{1}{vM} \le \frac{1}{v^2}. $$
\end{thmi}
\begin{proof}
	Considere los $M + 1$ números $\{ 0\cdot\alpha, 1\cdot\alpha, \dots, M\alpha \}$, y la aplicación $f(x) := x - \lfloor x \rfloor $
	que otorga la parte fraccionaria o decimal de los números, de modo que $f(j\alpha) \in [0, 1)$.
	Ahora particionemos el conjunto $[0, 1)$ en $M$ subintervalos:
	$$ \left[ \tfrac{0}{M}, \tfrac{1}{M} \right), \quad \left[ \tfrac{1}{M}, \tfrac{2}{M} \right), \quad \cdots,
	\quad \left[ \tfrac{M-1}{M}, \tfrac{M}{M} \right). $$
	Por el principio del palomar, se cumple que existen $0 \le i < j \le M$ distintos con $f(i\alpha), f(j\alpha)$ en el mismo subintervalo.
	Luego
	$$ \frac{1}{M} > |f(j\alpha) - f(i\alpha)| = | (j - i)\alpha - \lfloor j\alpha \rfloor + \lfloor i\alpha \rfloor |, $$
	definiendo $v := (j - i)$ y $u := \lfloor i\alpha \rfloor - \lfloor j\alpha \rfloor$ se verifica el enunciado.
\end{proof}
Nótese que el teorema anterior no solo nos da la existencia, sino que además deben ser infinitas estas fracciones buenas $u/v$.
En efecto si fijamos una aproximación $u/v$, entonces definimos $\epsilon := |v\alpha - u| > 0$ y eligiendo $M$ tal que $1/M < \epsilon$,
tenemos que la proxima aproximación $u'/v'$ ha de tener un mayor denominador.

A consecuencia del teorema de Dirichlet podemos redemostrar:
\begin{cor}
	Sea $n > 0$ entero tal que $-1$ es un residuo cuadrático módulo $n$.
	Entonces $n$ es suma de dos cuadrados.
\end{cor}
\begin{proof}
	Sea $r^2 \equiv -1 \pmod{n}$, luego por el teorema de aproximación de Dirichlet existe $v < \sqrt{n} = M$ y $u \in \Z$ tales que
	$$ \left| -\frac{r}{n} - \frac{u}{v} \right| < \frac{1}{v\sqrt{n}}. $$
	Multiplicando por $vn$ vemos que $a := rv + un$ tiene $|a| < \sqrt{n}$.
	Nótese que $a \equiv rv \pod n$, de modo que $a^2 + v^2 \equiv r^2v^2 + v^2 \equiv 0 \pod n$
	y $0 < a^2 + v^2 < n + n = 2n$, por lo que necesariamente $a^2 + v^2 = n$.
\end{proof}

A consecuencia tenemos el siguiente criterio sencillo de irracionalidad:
\begin{cor}
	Sea $\alpha$ un número real.
	Entonces $\alpha$ es irracional syss para todo $\epsilon > 0$ existen $u/v \in \Q$ tales que $0 < |\alpha v - u| < \epsilon$.
\end{cor}
\begin{proof}
	$\impliedby$. Por contrarrecíproca, si $\alpha = a/b$ es racional, entonces $va - ub$ es siempre un entero y por tanto
	$|va - ub| \ge 1 > 0$. Luego
	$$ 0 < |v \alpha - u| \ge \frac{1}{b}. $$
	$\implies$. Basta elegir $1/M < \epsilon$ en el teorema de aproximación de Dirichlet.
\end{proof}

Empleando esto, demos ejemplos de números irracionales:
\begin{prop}\label{thm:irrational_numbers_thr_approx}
	Sea $(g_n)_n$ una sucesión creciente y no acotada de números naturales con $g_1 \ge 1$, y sea $(z_n)_n$ una sucesión de 0s y 1s
	tal que $z_n = 1$ para infinitos $n$'s.
	Entonces el número
	$$ \alpha := \sum_{n=1}^{\infty} \frac{z_n}{g_1g_2 \cdots g_n} = \frac{z_1}{g_1} + \frac{z_2}{g_1g_2} + \frac{z_3}{g_1g_2g_3} + \cdots $$
	es irracional.
	En particular, $e = \sum_{n=1}^{\infty} \frac{1}{n!}$ es irracional.
\end{prop}
\begin{proof}
	Definamos los enteros
	$$ G_N := g_1 g_2 \cdots g_N, \qquad F_N := G_N \sum_{n=1}^{N} \frac{z_n}{G_n}. $$
	Entonces
	\begin{align}
		0 < \left| \alpha - \frac{F_N}{G_N} \right| &= \frac{1}{G_N} \sum_{n = N+1}^{\infty} \frac{z_n}{g_{N+1} \cdots g_n} \notag \\
							    &\le \frac{1}{G_N}\left( \frac{1}{g_{N+1}} + \frac{1}{g_{N+1}^2}
							    + \cdots \right) = \frac{1}{G_N(g_{N+1} - 1)}. \label{eqn:eeerational}
	\end{align}
	Como $g_{N+1} \to \infty$, entonces para todo $\epsilon$ eventualmente tendremos que $g_{N+1} - 1 > 1/\epsilon$ y aplicamos el corolario anterior.
\end{proof}

Hay varias modificaciones al teorema anterior e incluiremos algunas:
\begin{prop}
	Sean $\alpha_1, \dots, \alpha_r$ números reales y $M > 0$ un entero.
	Entonces existen $u_1, \dots, u_r \in \Z$ con $0 < v \le M^r$ tales que
	$$ \forall 1 \le j \le r \quad \left| \alpha_j - \frac{u_j}{v} \right| < \frac{1}{vM^r} \le \frac{1}{v^{r+1}}. $$
\end{prop}
Intercambiando el rol del numerador y denominador, Dirichlet probó:
\begin{prop}
	Sean $\alpha_1, \dots, \alpha_r$ números reales y $M > 0$ un entero.
	Entonces existen $u \in \Z$ y $v_1, \dots, v_r \in \Z$ no todos nulos donde cada $|v_j| \le M^{1/r}$ tales que
	$$ \left| \sum_{j=1}^{r} \alpha_jv_j - u \right| < \frac{1}{M}. $$
\end{prop}

Finalmente, Kronecker dio una generalización simultánea:
\begin{thm}[de aproximación de Dirichlet multidimensional]
	Sean $\alpha_{11}, \alpha_{12}, \dots, \alpha_{m,r}$ números reales donde $m, r > 0$ son enteros fijos, y sean $M > 0$ otro entero.
	Entonces existen $m$ enteros $u_1, \dots, u_m$ y $r$ enteros $v_1, \dots, v_r$ (no todos nulos) con cada $|v_j| < M^{m/r}$
	tales que
	$$ \forall 1 \le j \le m \quad \left| \sum_{k=1}^{r} \alpha_{jk} v_k - u_j \right| < \frac{1}{M}. $$
\end{thm}
\begin{proof}
	La $m$-tupla
	$$ \left( \sum_{k=1}^{r} \alpha_{1k}v_k - \left\lfloor \sum_{k=1}^{r} \alpha_{1k}v_k \right\rfloor, \dots, \sum_{k=1}^{r} \alpha_{mk}v_k - \left\lfloor \sum_{k=m}^{r} \alpha_{1k}v_k \right\rfloor \right) $$
	yace en el cubo $[0, 1)^m$ que se descompone en $M^m$ subcubos.

	Para poder aplicar el principio del palomar deben haber más $r$-tuplas $(v_1, \dots, v_r)$ que subcubos.
	Si decidimos que $0 \le v_j \le P$, entonces hay $(P+1)^r$ tuplas y queremos que $(P+1)^r > M^m$ o equivalentemente, $P + 1 > M^{m/r}$,
	por lo que $P := \lfloor M^{m/r} \rfloor$ basta.
	Finalmente, concluimos como en el teorema original de Dirichlet.
\end{proof}

El teorema de aproximación de Dirichlet nos dice que podemos aproximar bien a los números irracionales, pero no nos otorga, de buenas a primeras,
un método eficiente para encontrar buenas aproximaciones.

De momento ya sabemos que la expansión en fracción continua simple de un irracional $\alpha$ es única, lo que nos permite hablar sin ambigüedad de \textit{los}
aproximantes de $\alpha$.
Podemos caracterizarles parcialmente por lo siguiente:
\begin{thmi}[Criterio de Legendre]\index{criterio!de Legendre}
	Sea $\alpha \in \R$ un número irracional y sea $p, q \in \Z$ enteros tales que
	$$ \alpha - \frac{p}{q} = \frac{\theta}{q^2}, \qquad |\theta| < 1. $$
	Sean $p/q = [a_0; a_1, \dots, a_n]$ y supongamos que $\sign\theta = (-1)^n$.
	Entonces $p/q$ es un aproximante de $\alpha$ syss
	$$ |\theta| \le \frac{q_n}{q_n + q_{n-1}}, $$
	(donde $q_{n-1}$ es aproximante $p/q$.)
\end{thmi}
\begin{proof}
	Definamos $\beta$ tal que
	$$ \alpha = \frac{\beta p_n + p_{n-1}}{\beta q_n + q_{n-1}}, $$
	de modo que
	$$ \frac{\theta}{q_n^2} = \frac{\beta p_n + p_{n-1}}{\beta q_n + q_{n-1}} - \frac{p_n}{q_n} = \frac{(-1)^n}{q_n(\beta q_n + q_{n-1})}, $$
	por lo que
	$$ |\theta| = \frac{q_n}{\beta q_n + q_{n-1}}. $$
	Despejando tenemos que $\beta = (q_n - |\theta| q_{n-1})/|\theta| q_{n-1}$ y como $0 < |\theta| < 1$ vemos que $\beta > 0$.

	Si $\beta \ge 1$, entonces $\alpha = [a_0; a_1, \dots, a_n, \beta]$ lo que prueba que $p_n/q_n$ son aproximantes.
	Si $0 < \beta < 1$, entonces $\lfloor a_n + 1/\beta \rfloor =: a_n + c$ con $c > 0$, de modo que $\alpha = [a_0; a_1, \dots, a_n + c, \dots]$,
	por lo que $[a_0; a_1, \dots, a_n]$ no es un aproximante de $\alpha$.
	Esto prueba la equivalencia del enunciado.
\end{proof}

\begin{thm}\label{thm:approximants_satisfy_approx}
	Sean $\alpha \in \R$ un número irracional.
	\begin{enumerate}
		\item Sean $(c_n)_{n\in\N}$ los aproximantes de $\alpha$.
			Entonces para todo $n$ (al menos) uno de los dos aproximantes $p/q \in \{ c_n, c_{n+1} \}$ satisface que
			\begin{equation}
				\left| \alpha - \frac{p}{q} \right| < \frac{1}{2q^2}.
				\label{eqn:continued_fract_approx}
			\end{equation}
			En consecuencia, hay infinitas aproximaciones racionales que satisfacen la desigualdad \eqref{eqn:continued_fract_approx}.

		\item Sean $p, q$ enteros coprimos que satisfacen \eqref{eqn:continued_fract_approx}, entonces $p/q$ es un aproximante de $\alpha$.
	\end{enumerate}
\end{thm}
\begin{proof}
	\begin{enumerate}
		\item Por la proposición~\ref{thm:approximants_closeness}, tenemos que si $c_n$ es un aproximante de $\alpha$, se cumple que
			$|c_n - c_{n+1}| = 1/q_nq_{n+1}$, pero como $\alpha$ está entre $c_n$ y $c_{n+1}$ concluimos que
			$$ |\alpha - c_n| < \frac{1}{q_nq_{n+1}} < \frac{1}{q_n^2}, $$
			puesto que $q_n$ es una sucesión estrictamente creciente.
			Ésto se puede mejorar aún más notando que
			$$ |\alpha - c_n| + |\alpha - c_{n+1}| = |c_n - c_{n+1}| = \frac{1}{q_n q_{n+1}}, $$
			y ahora, por la desigualdad media geométrica-media cuadrática (cf. \cite[Teo.~6.68]{Top}) sabemos que $xy < (x^2 + y^2)/2$ si $0 < x < y$,
			luego se concluye que
			$$ |\alpha - c_n| + |\alpha - c_{n+1}| = \frac{1}{q_n q_{n+1}} < \frac{1}{2q_n^2} + \frac{1}{2q_{n+1}^2}. $$
			Luego, al menos alguno de las dos restas, digamos $c_n$, satisface $|\alpha - c_n| < 1/(2q_n^2)$.
		\item Empleando que $q_{n-1} < q_n$ para los aproximantes de una fracción continua, tenemos que
			$$ \frac{q_n}{q_n + q_{n-1}} > \frac{q}{q + q} = \frac{1}{2}. $$
			De modo que podemos aplicar el criterio de Legendre. \qedhere
			% \item Como los aproximantes $p_n,q_n$ de $\alpha$, son tales que $q_n$ es creciente, entonces podemos elegir un $n$ tal que $0 < q < q_{n+1}$.
			%	 Probaremos que $|q\alpha - p| \ge |q_n\alpha - p_n|$, i.e., que $p_n/q_n$ es la mejor aproximación posible.
			%	 Consideremos el sistema de ecuaciones racionales
			%	 \begin{align*}
			%		 p &= up_n + vp_{n+1}, \\
			%		 q &= uq_n + vq_{n+1}
			%	 \end{align*}
			%	 como viene dado por una matriz a coeficientes enteros de determinante $p_nq_{n+1} - p_{n+1}q_n = (-1)^{n+1} \in \Z^\times$ posee
			%	 las soluciones enteras:
			%	 $$ u = (-1)^{n+1}(q_{n+1}p - p_{n+1}q), \qquad v = (-1)^{n+1}(p_nq - q_np). $$
			%	 Como $0 < q < q_{n+1}$, volviendo a la ecuación original vemos que $u \ne 0$ y si $v \ne 0$ entonces $u, v$ tienen signos opuestos,
			%	 de lo que se deduce que
			%	 \begin{align*}
			%		 |q\alpha - p| &= |(uq_n + vq_{n+1})\alpha - (up_n + vp_{n+1})| \\
			%			       &= |u(q_n\alpha - p_n) + v(q_{n+1}\alpha - p_{n+1})| \ge |q_n\alpha - p_n|.
			%	 \end{align*}
			%	 Fijemos $n$ tal que $q_n \le q < q_{n+1}$ y nótese que
			%	 \begin{align*}
			%		 \left| \frac{p}{q} - \frac{p_n}{q_n} \right| &\le \left| \alpha - \frac{p}{q} \right| + \left| \alpha - \frac{p_n}{q_n} \right|
			%		 = \frac{|q\alpha - p|}{q} + \frac{|q_n\alpha - p_n|}{q_n} \\
			%							      &\le \left( \frac{1}{q} + \frac{1}{q_n} \right) |\alpha q - p|
			%							      < \frac{2}{q_n} \cdot \frac{1}{2q} = \frac{1}{qq_n},
			%	 \end{align*}
			%	 finalmente, multiplicando a ambos lados por $qq_n$ se obtiene que $|pq_n - qp_n| < 1$ debe ser nulo, de modo que $p/q = p_n/q_n$. \qedhere
	\end{enumerate}
\end{proof}

\begin{thm}[Hurwitz]
	Sea $\alpha$ un real irracional y sean $(c_n)_{n\in\N}$ sus aproximantes.
	Entonces para todo $n$, uno de los aproximantes $p/q \in \{ c_n, c_{n+1}, c_{n+2} \}$ satisface que
	$$ \left| \alpha - \frac{p}{q} \right| < \frac{1}{\sqrt{5}q^2}. $$
	En consecuencia, infinitas fracciones reducidas satisfacen la desigualdad anterior.
\end{thm}
\begin{proof}
	Sean $p_n, q_n$ los numeradores y denominadores correspondientes de $c_n$.
	Con la notación del lema anterior es fácil comprobar que
	$$ \left| \alpha - \frac{p_n}{q_n} \right| = \frac{1}{q_n( \alpha^\prime_{n+1}q_n + q_{n-1} )} = \frac{1}{q_n^2(\alpha^\prime_{n+1} + \beta_{n+1})}, $$
	donde $\beta_{n+1} := q_{n-1}/q_n$.
	Basta probar que $\alpha_j^\prime + \beta_j > \sqrt{5}$ para algún $j \in \{ n, n+1, n+2 \}$.

	Supongamos que no se satisface para $j \in \{ n, n+1 \}$.
	Como $\alpha^\prime_n = a_n + 1/\alpha^\prime_{n+1}$ y
	$$ \frac{1}{\beta_{n+1}} = \frac{q_n}{q_{n-1}} = \frac{a_{n}q_{n-1} + q_{n-2}}{q_{n-1}} = a_n + \beta_n, $$
	de modo que
	$$ \frac{1}{\alpha^\prime_{n+1}} + \frac{1}{\beta_{n+1}} = \alpha^\prime_n + \beta_n \le \sqrt{5}. $$
	Luego, concluimos que
	$$ 1 = \frac{1}{\alpha_{n+1}^\prime} \alpha_{n+1}^\prime \le \left( \sqrt{5} - \frac{1}{\beta_{n+1}} \right)( \sqrt{5} - \beta_{n+1} ) =
	5 - \sqrt{5}\left( \frac{1}{\beta_{n+1}} + \beta_{n+1} \right) + 1, $$
	lo que nos da que $\frac{1}{\beta_{n+1}} + \beta_{n+1} < \sqrt{5}$, donde la desigualdad estricta sale del hecho de que $\beta_{n+1}$ es racional.
	Como $\beta_{n+1} < 1$, se deduce de que $\beta_{n+1} > \frac{1}{2}( \sqrt{5} - 1 )$.

	Análogamente vemos que $\beta_{n+2} > \frac{1}{2}( \sqrt{5} - 1 )$, por lo que
	$$ a_{n+1} = \frac{1}{\beta_{n+2}} - \beta_{n+1} < \sqrt{5} - (\beta_{n+2} + \beta_{n+1}) < \sqrt{5} - (\sqrt{5} - 1) = 1, $$
	lo cual es absurdo.
\end{proof}

La constante en el teorema de Hurwitz es aguda:
\begin{prop}
	Para todo $A > \sqrt{5}$ existe un $\alpha$ real irracional para el que la desigualdad $|\alpha - p/q| < 1/Aq^2$ solo admite finitas soluciones.
\end{prop}
\begin{proof}
	Elijamos $\alpha := \frac{1}{2}( \sqrt{5} - 1 )$ y supongamos que
	$$ \alpha = \frac{p}{q} + \frac{\delta}{q^2}, \qquad |\delta| < \frac{1}{A} < \frac{1}{\sqrt{5}}. $$
	Multiplicando por $q$ y reordenando obtenemos
	$$ \frac{\delta}{q} - \frac{1}{2} \sqrt{5}q = \frac{\delta}{q} - \left( \frac{1}{2}q + p + \frac{\delta}{q} \right) = -\frac{1}{2}q - p. $$
	Luego, tomando cuadrados tenemos que
	$$ \frac{\delta^2}{q^2} - \sqrt{5}\delta = \left( \frac{\delta}{q} - \frac{1}{2} \sqrt{5}q \right)^2 - \frac{5q^2}{4} = pq + p^2 - q^2. $$
	Como $|\delta| < 1/\sqrt{5}$ vemos que para $q$ suficientemente grande el lado izquierdo tiene valor absoluto $<1$,
	de modo que $pq + p^2 - q^2 = 0$, o equivalentemente $(2p + q)^2 = 5q^2$ lo cual es absurdo pues $\sqrt{5}$ es irracional.
\end{proof}

\subsection{El teorema y los números de Liouville}
\begin{thm}[de aproximación de Liouville]\index{teorema!de aproximación!de Liouville}
	Sea $\alpha \in \R \setminus \Q$ un número algebraico de grado $d$.
	Entonces existe $C > 0$ tal que toda fracción $u/v \in \Q$ satisface que
	$$ \left| \alpha - \frac{u}{v} \right| \ge \frac{C}{v^d}. $$
\end{thm}
\begin{proof}
	Sea $f(x) = c_d x^d + \cdots + c_1 x + c_0 \in \Z[x]$ el polinomio minimal de $\alpha$ (es decir, los coeficientes son coprimos y
	el $c_d$ <<limpia denominadores>>). Entonces $f(x) = (x - \alpha)g(x)$, donde $g(x) \in \Q[x]$.
	Ahora bien, la función $|g(x)|$ alcanza un máximo $M$ en el compacto $[\alpha - 1, \alpha + 1]$ y, por tanto,
	para toda fracción $u/v \in \Q$ se tiene que
	$$ |f(u/v)| = \left| \alpha - \frac{u}{v} \right| \cdot |g(u/v)| \le M\left| \alpha - \frac{u}{v} \right|, $$
	y nótese que $v^d |f(u/v)|$ es un entero, de modo que $|f(u/v)| \ge 1/v^d$ pues $f(x)$ no tiene raíces racionales.
	Así, tomando $C := \min\{ 1, 1/M \}$ tenemos el enunciado.
\end{proof}
\begin{cor}
	Sea $\alpha$ un irracional algebraico de grado $d$.
	Entonces existen a lo más finitas fracciones $u/v \in \Q$ tales que
	$$ \left| \alpha - \frac{u}{v} \right| < \frac{1}{v^{d+1}}. $$
\end{cor}
\begin{proof}
	Basta notar que de existir infinitas, el denominador no está acotado y escoger $q > 1/C$, donde $C$ es la constante del teorema anterior.
\end{proof}

Liouville empleó su teorema para dar el primer ejemplo de un número trascendente:
\begin{exn}\label{ex:liouville_number}
	El número
	$$ \lambda := \sum_{n=1}^{\infty} \frac{1}{10^{n!}} = 0.110001 00000 00000 00000 001... $$
	es trascendente.

	En efecto, sea $c_n := \sum_{j=1}^{n} 10^{-j!}$ los cuales satisfacen que $\lim_n c_n = \lambda$.
	Luego, nótese que, con $v_n := 10^{n!}$, tenemos que $u_n := v_nc_n$ es entero y ésta fracción satisface que
	\begin{align*}
		\left| \lambda - \frac{u_n}{v_n} \right| &= \frac{1}{10^{(n+1)!}} + \frac{1}{10^{(n+2)!}} + \cdots < \sum_{j=0}^{\infty} \frac{1}{10^{(n+1)! + j}} \\
							 &= \frac{1}{10^{(n+1)!}} \cdot \frac{1}{1 - 1/10} < \frac{1}{10^{(n+1)!}} < \frac{1}{v_n^n}.
	\end{align*}
	Ahora sí, $\lambda$ es irracional (pues su expansión decimal no es periódica) y, si fuese algebraico, tendría un grado $d$ que acote las
	aproximaciones, pero claramente $u_n/v_n$ son infinitas aproximaciones <<buenas>> lo que es absurdo.
\end{exn}
\begin{mydef}
	Se dice que un real irracional $\alpha$ es un \strong{número de Liouville}\index{numero@número!de Liouville} si para cada $j \ge 1$ natural
	existe una fracción reducida $u/v$ tal que $|\alpha - u/v| < 1/v^j$.
	Se denota por $\mathcal{L}$ el conjunto de los números de Liouville.
\end{mydef}

El mismo argumento prueba que los números de Liouville son todos trascendentes, aunque el recíproco es falso.
Exhibir un contraejemplo no es fácil, pero podemos dar una prueba indirecta.

\begin{mydef}
	Sea $\varphi \colon \N \to (0, 1]$ una función y sea $\alpha$ un irracional.
	Se dice que $\varphi$ es \strong{aproximante para $\alpha$}\index{función!aproximante} si existen infinitas fracciones $u/v \in \Q$
	tales que $|\alpha v - u| < \varphi(v)$.
	Denotamos por $A(\varphi)$ el conjunto de reales para los cuales $\varphi$ es aproximante.
\end{mydef}

\begin{thm}[Khinchin]
	Sea $\varphi \colon \N \to (0, 1]$ una función.
	\begin{enumerate}
		\item El conjunto $A(\varphi)$ es no numerable.
		\item El conjunto $A(\varphi)$ es denso.
		\item Si $\sum_{q=1}^{\infty} \varphi(q) < \infty$, entonces $A(\varphi)$ tiene medida (de Lebesgue) nula.
	\end{enumerate}
\end{thm}
\begin{proof}
	\begin{enumerate}
		\item Sea $Z' := \Func(\N, \{ 0, 1 \})$, es decir, el conjunto de las sucesiones binarias.
			Definamos $Z \subseteq Z'$ como las sucesiones que tienen infinitos 1's.
			Sea $2 < g_1 \le g_2 \le g_3 \le \cdots$ una sucesión de naturales creciente tal que $\lim_n g_n = \infty$,
			entonces definimos la función
			\begin{align*}
				\psi_{\vec g} \colon Z &\longrightarrow \R \\
				\vec z = (z_n)_n &\longmapsto \sum_{n=1}^{\infty} \frac{z_n}{g_1 g_2 \cdots g_n}.
			\end{align*}
			Si fijamos $\vec z \in Z$ podemos definir las aproximaciones
			$$ G_N := q_1 q_2 \cdots q_N, \qquad F_N := G_N \sum_{n=1}^{N} \frac{z_n}{G_n}. $$
			Entonces, la desigualdad \eqref{eqn:eeerational} nos da que
			\[
				0 < \left| \psi_{\vec q}(\vec z) - \frac{F_N}{G_N} \right| \le \frac{1}{G_N(q_{N+1} - 1)}.
			\]
			Ahora bien, si rellenamos con $z_{N+1} = z_{N+2} = \cdots = z_{N + r - 1} = 0$ podemos mejorar la desigualdad por
			$\frac{1}{G_N(g_{N + r} - 1)}$, por lo que basta elegir el $r$ suficientemente grande de modo que $\frac{1}{g_{N+r} - 1} < \varphi(G_N)$.

			Finalmente, es fácil notar que cada elección de $\vec g := (g_n)_{n \in \N}$ nos da un valor distinto y siempre nos otorga al menos
			un elemento de $A(\varphi)$ (dado por rellenar con suficientes ceros).
			Luego tenemos una inyección desde el conjunto de $\vec q$'s hasta $A(\varphi)$ y un mero argumento de cardinalidad nos dice que
			hay $\mathfrak{c}$ distintos $\vec q$'s.

		\item Basta notar que, con la notación anterior, a un elemento $\psi_{\vec q}(\vec z) \in A(\varphi)$ podemos sumarle racionales de la forma $a/G_N$,
			donde $a \in \Z$ y $N \in \N$ varían, para obtener un elemento de $A(\varphi)$ y este conjunto de racionales es denso.
			También, el lector puede realizar el ejercicio de probar que $A(\varphi) + \Q = A(\varphi)$ en general y emplear que $A(\varphi) \ne
			\emptyset$ por nuestra demostración anterior.

		\item Fijemos $N, g \in \N$.
			Si $\Alpha \in A( \varphi) \cap [-g, g]$, entonces ha de existir una fracción $u/v$ con $v \ge N$
			tal que $|\alpha - u/v| < \varphi(v)/v$, es decir,
			$$ \alpha \in \left( \frac{u - \varphi(v)}{v}, \frac{u + \varphi(v)}{v} \right). $$
			En consecuencia, $|u| < |\alpha|v + \varphi(v) \le 1 + gv$, por lo que
			$$ A(\varphi) \cap [-g, g] \subseteq \bigcap_{N=1}^\infty \bigcup_{v=N}^{\infty} \bigcup_{|u| \le 1 + gv} \left( \frac{u - \varphi(v)}{v},
			\frac{u + \varphi(v)}{v} \right), $$
			por lo que, denotando la medida de Lebesgue por $\mu$, tenemos que
			\begin{align*}
				\mu\big( A(\varphi) \cap [-g, g] \big) &\le \sum_{v=N}^{\infty} \sum_{|u| \le 1 + gv} 2 \frac{\varphi(v)}{v}
				= \sum_{v=N}^{\infty} 2(3 + 2gv) \frac{\varphi(v)}{v} \\
								       &\le 16g \sum_{v=N}^{\infty} \varphi(v),
			\end{align*}
			lo cual con $N \to \infty$ converge a 0. \qedhere
	\end{enumerate}
\end{proof}

\begin{cor}
	El conjunto $\mathcal{L} \subseteq \R$ es denso, no numerable y de medida nula.
\end{cor}
\begin{proof}
	Basta notar que los elementos de $\mathcal{L}$ pertenecen todos a cada $A(\varphi_j)$ donde $\varphi_j(v) := \frac{1}{v^j}$,
	por lo que $\mathcal{L}$ tiene medida nula.

	Nótese que al intersectar densos no numerables podemos perder susodichas propiedades, por lo que hay que ser cautelosos.
	Replicando el mismo argumento del ejemplo~\ref{ex:liouville_number}, pero con otra base, vemos que el número
	\[
		\sum_{n=1}^{\infty} \frac{a_n}{b^{n!}}
	\]
	con $0 \le a_n < b$ y $b \ge 2$ fijo son trascendentes.
	Incluso con $b = 2$ obtenemos no numerables irracionales dados por la libertad de elegir la sucesión $(a_n)_{n \in \N}$.
	Además, es claro que $\mathcal{L} + \Q = \mathcal{L}$, de modo que es denso.
\end{proof}

Y finalmente un resultado divertido expuesto en \cite{erdos62liouville}:
\begin{thm}[Erd\H os]
	$\mathcal{L + L} = \R$.
\end{thm}
\begin{proof}
	Sea $\alpha \in \R$.
	Como $\Q + \mathcal{L} = \mathcal{L}$, vemos que si $\alpha - \lfloor \alpha \rfloor = \beta + \gamma$, con $\beta, \gamma \in \mathcal{L}$,
	entonces $\alpha = \beta + (\gamma + \lfloor \alpha \rfloor)$ satisface que $\gamma + \lfloor \alpha \rfloor \in \mathcal{L}$.
	Así que, sea $\alpha$ dado por expansión en base 2:
	$$ \alpha = \sum_{n=1}^{\infty} \frac{a_n}{2^n}, $$
	con $0 \le a_n < 2$.
	Definimos
	$$ \beta := \sum_{n=1}^{\infty} \frac{b_n}{2^n}, \qquad \gamma := \sum_{n=1}^{\infty} \frac{c_n}{2^n}, $$
	donde, fijado $m \ge 1$ tal que $m! \le n < (m+1)!$, definimos $b_n = a_n$ y $c_n = 0$ si $m$ es par,
	y $b_n = 0, c_n = a_n$ si $m$ es impar.
	Queda al lector verificar que $\beta, \gamma$ son números de Liouville.
\end{proof}
\nocite{hlawka:number, hua:number, granville:masterclass}

\subsection{La irracionalidad de $\zeta(2)$ y $\zeta(3)$}
En ésta sección seguimos el artículo de \citeauthor{beukers79irrationality}~\cite{beukers79irrationality}.

En las siguientes demostraciones emplearemos $V(n) := \mcm\{ 1, 2, \dots, n \}$.

\begin{prop}
	El número
	$$ \zeta(2) := \frac{1}{1^2} + \frac{1}{2^2} + \frac{1}{3^2} + \cdots = \sum_{k=1}^{\infty} \frac{1}{n^2}. $$
	es irracional.
\end{prop}
\begin{proof}
	Para un complejo $|z| < 1$ tenemos que la serie geométrica
	$$ \frac{z^r}{1 - z} = z^r + z^{r+1} + z^{r+2} + \cdots $$
	Si tenemos $x, y \in (0, 1)$, entonces $z = xy$ nos permite deducir lo siguiente:
	\begin{align*}
		I_{rr} := \int_{0}^{1} \int_{0}^{1} \frac{x^ry^r}{1 - xy} \, \ud x \, \ud y &=
		\int_{0}^{1} \left( \frac{y^r}{r+1} + \frac{y^{r+1}}{r+2} + \cdots \right) \, \ud y \\
											    &= \frac{1}{(r+1)^2} + \frac{1}{(r+2)^2} + \cdots,
	\end{align*}
	de modo que
	$$ \zeta(2) = \frac{1}{1^2} + \frac{1}{2^2} + \cdots + \frac{1}{r^2} + I_{rr}.  $$
	o despejando, tenemos que $I_{rr} = \frac{\alpha \zeta(2) + \beta}{V(r)^2}$, donde $\alpha, \beta \in \Z$.

	Si ahora definimos
	$$ P_n(x) := \frac{1}{n!} \frac{\ud^n}{\ud x^n}\big( x^n(1 - x)^n \big), $$
	entonces se puede notar que $P_n(x)$ es un polinomio con coeficientes enteros (¿por qué?), y luego
	$$ \int_{0}^{1} \int_{0}^{1} \frac{(1 - y)^n P_n(x)}{1 - xy} \, \ud x \, \ud y = \frac{a_n \zeta(2) + b_n}{V(n)^2}, $$
	donde $a_n, b_n$ son enteros. Si integramos parcialmente respecto a $x$ en el lado derecho tenemos
	$$ \frac{a_n \zeta(2) + b_n}{V(n)^2} = (-1)^n \int_{0}^{1} \int_{0}^{1} \frac{y^n(1 - y)^n x^n(1 - x)^n}{(1 - xy)^{n+1}} \, \ud x \, \ud y. $$
	Vamos a acotar el integrando. Para ello, vamos a calcular el máximo valor en el intervalo $x \in [0, 1]$ buscando valores críticos, de lo que
	$$ \frac{y(1 - y) x(1 - x)}{1 - xy} \le \left( \frac{\sqrt{5} - 1}{2} \right)^5. $$
	Así, tenemos que
	\begin{align*}
		0 &< \left| \int_{0}^{1} \int_{0}^{1} \frac{y^n(1 - y)^n x^n(1 - x)^n}{(1 - xy)^{n+1}} \, \ud x \, \ud y \right| \\
		  &= \frac{|a_n \zeta(2) + b_n|}{V(n)^2} \le \left( \frac{\sqrt{5} - 1}{2} \right)^{5n} \int_{0}^{1} \int_{0}^{1} \frac{\ud x \, \ud y}{1 - xy} \\
		  &= \left( \frac{\sqrt{5} - 1}{2} \right)^{5n} \zeta(2).
	\end{align*}
	Así que, basta probar que
	\[
		\lim_n V(n)^2 \left( \frac{\sqrt{5} - 1}{2} \right)^{5n} = 0.
	\]
	Ahora bien, nótese que $V(n) \le n^{\pi(n)}$ y empleando las aproximaciones de Chebychev tenemos que
	$$ \pi(n) \le (\log 3) \cdot \frac{n}{\log n}, $$
	por lo que podemos concluir pues
	\begin{equation}
		V(n)^2 \left( \frac{\sqrt{5} - 1}{2} \right)^{5n} \le 3^n \cdot \left( \frac{\sqrt{5} - 1}{2} \right)^{5n} \approx 0.2705^n.
		\tqedhere
	\end{equation}
\end{proof}

\begin{cor}
	Hay infinitos primos.
\end{cor}
\begin{proof}
	De lo contrario, empleando el producto de Euler
	$$ \zeta(2) = \prod_{p} \frac{1}{1 - 1/p^2}, $$
	notamos que a la derecha tenemos un producto finito de racionales y, por tanto, el lado izquierdo es racional.
\end{proof}
\begin{cor}
	$\pi^2$ es irracional.
\end{cor}

Vamos a emplear un método similar para concluir lo siguiente:
\begin{thm}[Apéry]
	El número $\zeta(3)$ es irracional.
\end{thm}
\begin{proof}
	Nótese que
	\begin{align*}
		\int_{0}^{1} \int_{0}^{1} \frac{\log(xy)}{1 - xy}(xy)^r \, \ud x \, \ud y &= \int_{0}^{1} \int_{0}^{1} \sum_{j=0}^{\infty} (\log x + \log y)x^{r+j} y^{r+j} \, \ud x \, \ud y \\
											  &= \int_{0}^{1} \sum_{j=0}^{\infty} \left( -\frac{y^{r+j}}{(r+j+1)^2} +
											  \frac{(\log y) y^{r+j}}{r+j+1} \right)\, \ud y \\
											  &= -2 \sum_{j=0}^{\infty} \frac{1}{(r+j+1)^3}
											  = -2\left( \zeta(3) - 1 - \frac{1}{2^3} - \cdots - \frac{1}{r^3} \right).
	\end{align*}
	y análogamente si $r > s$ es fácil comprobar que
	$$ I_{rs} := \int_{0}^{1} \int_{0}^{1} \frac{\log(xy)}{1 - xy}x^r y^s \, \ud x \, \ud y =
	\frac{-1}{r - s}\left( \frac{1}{(s+1)^2} + \cdots + \frac{1}{r^2} \right), $$
	de modo que, $I_{rs}$ es siempre un número racional si $\zeta(3)$ es racional.

	Nuevamente, definiendo
	$$ P_n(x) := \frac{1}{n!} \frac{\ud^n}{\ud x^n}\big( x^n(1 - x)^n \big), $$
	entonces se puede notar que
	$$ \int_{0}^{1} \int_{0}^{1} \frac{-\log(xy)}{1 - xy} P_n(x) P_n(y) \, \ud x \, \ud y = \frac{a_n \zeta(3) + b_n}{V(n)^3}, $$
	donde $a_n, b_n$ son enteros.

	Empleando la siguiente identidad
	$$ \frac{-\log(xy)}{1 - xy} = \int_{0}^{1} \frac{1}{1 - (1 - xy)z} \, \ud z, $$
	tenemos que
	\begin{align*}
		\frac{a_n \zeta(3) + b_n}{V(n)^3} &= \int_{0}^{1} \int_{0}^{1} \int_{0}^{1} \frac{P_n(x) P_n(y)}{1 - (1-xy)z} \, \ud x \, \ud y \, \ud z, \\
		\shortintertext{ahora integramos $n$ veces respecto a $x$ y obtenemos que}
						  &= \int_{0}^{1} \int_{0}^{1} \int_{0}^{1} \frac{(xyz)^n (1 - x)^n P_n(y)}{(1 - (1-xy)z)^{n+1}}
						  \, \ud x \, \ud y \, \ud z.
	\end{align*}
	Mediante la sustitución
	$$ w := \frac{1 - z}{1 - (1-xy)z} \iff 1 - w = \frac{xyz}{1 - (1-xy)z}, $$
	podemos reescribir
	\begin{align*}
		\frac{a_n \zeta(3) + b_n}{V(n)^3} &= \int_{0}^{1} \int_{0}^{1} \int_{0}^{1} (1 - x)^n (1 - w)^n \frac{P_n(y)}{1 - (1-xy)w} \, \ud x \, \ud y \, \ud w, \\
		\shortintertext{ahora integramos $n$ veces respecto a $y$ y obtenemos que}
						  &= \int_{0}^{1} \int_{0}^{1} \int_{0}^{1} \frac{x^n(1-x)^n y^n(1-y)^n w^n(1-w)^n}{(1 - (1-xy)w)}
						  \, \ud x \, \ud y \, \ud w.
	\end{align*}
	Nuevamente procedemos a acotar el integrando:
	$$ \frac{x(1 - x) y(1 - y) w(1 - w)}{1 - (1 - xy)w} \le (\sqrt{2} - 1)^4, $$
	y entonces
	\begin{align*}
		\frac{a_n \zeta(3) + b_n}{V(n)^3} &= (\sqrt{2} - 1)^{4n} \int_{0}^{1} \int_{0}^{1} \int_{0}^{1} \frac{1}{1 - (1 - xy)w} \, \ud x \, \ud y \, \ud w \\
						  &= (\sqrt{2} - 1)^{4n} \int_{0}^{1} \int_{0}^{1} \frac{-\log(xy)}{1 - xy} \, \ud x \, \ud y
						  = 2 \zeta(3) \cdot (\sqrt{2} - 1)^{4n},
	\end{align*}
	reordenando tenemos que $0 < |a_n \zeta(3) + b_n| = 2 \zeta(3) V(n)^3 (\sqrt{2} - 1)^{4n}$.
	Empleando la misma desigualdad de Chebyshev, vemos que
	\begin{equation}
		2 V(n)^3 (\sqrt{2} - 1)^{4n} = 2 \cdot 27^{n} \cdot (\sqrt{2} - 1)^{4n} \approx 2(0.7948)^n \to 0.
		\tqedhere
	\end{equation}
\end{proof}

El lector notará los fuertes paralelos entre la demostración para $\zeta(2)$ y $\zeta(3)$, ¿admitirá una generalización para $\zeta(4)$ y $\zeta(5)$?
La respuesta es que no, o al menos no el método empleado; ya en la última demostración vimos que $27(\sqrt{2} - 1)^4 \approx 0.7948 < 1$;
pero ya en casos superiores éste factor se pasa del 1.
\begin{con}
	Se creen:
	\begin{enumerate}
		\item Los valores $\zeta(5), \zeta(7), \zeta(9), \dots$ son irracionales.
		\item Los valores $\zeta(n)$ son trascendentes para todo $n > 1$ entero.
		\item Los valores $\pi, \zeta(3), \zeta(5), \zeta(7), \dots$ son ($\Q$-)algebraicamente independientes.
	\end{enumerate}
\end{con}
Claramente $3 \implies 2 \implies 1$.
Hay gente que cree que, similar al caso par, $\zeta(2n+1)$ se puede escribir como $\pi^{2n+1}$ con sus respectivos errores;
no obstante, una conjetura de Grothendieck implica la propiedad 3, por lo que se opta por ésta última.

Indudablemente es curioso que además de $\zeta(3)$ se desconozca la irracionalidad de otros valores $\zeta(2n+1)$.

% Casi no hay avances en las conjeturas anteriores, salvo un par de detalles:
% (1), para $\zeta(2n+1)$ se sabe que estos valores toman infinitas veces valores irracionales, pero no se ha d

\subsection{La ecuación de Pell}
Ahora veremos cómo emplear fracciones continuas para calcular eficientemente la ecuación de Pell.
En ésta sección fijaremos un natural $D > 0$ que no es un cuadrado y nos enfocaremos en la ecuación de Pell:
\begin{equation}
	x^2 - Dy^2 = 1,
	\label{eq:Pell_equation}
\end{equation}
y diremos que las \strong{soluciones triviales} son las de la forma $(\pm 1, 0)$.
En primer lugar queremos encontrar soluciones no triviales de la ecuación de Pell:
\begin{lem}
	Existe un entero $k$ con $1 \le |k| \le 1 + 2\sqrt{D}$ tal que la ecuación diofántica $x^2 - Dy^2 = k$ tiene infinitas soluciones enteras.
\end{lem}
\begin{proof}
	Como $\sqrt{D}$ es irracional, por el teorema de Dirichlet, existen infinitos $u/v \in \Q$ con $(u; v) = 1$ y $1 \le v$ tales que
	$$ \left| \sqrt{D} - \frac{u}{v} \right| < \frac{1}{v^2}, $$
	o equivalentemente, $|u - v\sqrt{D}| < 1/v$. Nótese que
	\begin{align*}
		|u^2 - v^2D| &= |u - v\sqrt{D}| \, |u + v\sqrt{D}| < \frac{1}{v}(|u - v\sqrt{D}| + 2v\sqrt{D}) \\
			     &< \frac{1}{v^2} + 2\sqrt{D} \le 1 + 2\sqrt{D},
	\end{align*}
	donde empleamos desigualdad triangular.
	Finalmente, concluimos por una variación del principio de palomar.
\end{proof}

\begin{thm}
	La ecuación de Pell siempre posee soluciones no triviales.
\end{thm}
\begin{proof}
	Por el lema anterior, elijamos $k$ de modo que la ecuación diofántica $x^2 - Dy^2 = k$ posee infinitas soluciones.
	Consideramos dos soluciones distintas $(a_1, b_1), (a_2, b_2)$ positivas con
	$$ a_1 \equiv a_2, \qquad b_1 \equiv b_2 \pmod k, $$
	las cuales siempre existen puesto que hay finitas posibilidades de congruencias e infinitas soluciones.
	Definamos:
	$$ A + B\sqrt{D} := (a_1 - b_1\sqrt{D})(a_2 + b_2\sqrt{D}) = (a_1a_2 - b_1b_2D) + (a_1b_2 - a_2b_1)\sqrt{D}, $$
	de modo que $A^2 - B^2D = \galnorm(A + B\sqrt{D}) = k^2$.

	Afirmamos que $B \ne 0$: de lo contrario, existiría $\lambda \in \Q_{\ne 0}$ tal que $a_1 = \lambda a_2$, $b_1 = \lambda b_2$ y tal que
	$$ k = a_2^2 - b_2^2 D = \lambda^2(a_1^2 - b_1^2 D) = \lambda^2k, $$
	por lo que $\lambda = \pm 1$.
	Como las soluciones son positivas, entonces $\lambda = 1$, pero eso es absurdo pues elegimos las soluciones distintas.

	Además nótese que como $a_1 \equiv a_2$ y $b_1 \equiv b_2 \pod k$ se cumple que
	$$ A \equiv a_1^2 - b_1^2D = k \equiv 0, \qquad B \equiv a_1b_1 - a_1b_1 = 0\pmod k. $$
	luego existen $u, v$ enteros tales que $A = ku, B = kv$ y finalmente:
	$$ k^2 = A^2 - B^2D = k^2(u^2 - v^2D) \iff 1 = u^2 - v^2D, $$
	donde $v \ne 0$ pues $B \ne 0$ por lo que es una solución no trivial.
\end{proof}

Está claro que las soluciones de la ecuación de Pell $(a, b)$ son exactamente los números algebraicos $a + b\sqrt{D} \in \Z[\sqrt{D}]$ de norma 1.
De modo que podemos definir
$$ \mathcal{P}_D := \{ a + b\sqrt{D} \in \Z[\sqrt{D}] : a^2 - b^2D = 1 \} $$
el cual es un grupo con la multiplicación.
\begin{thm}
	$\mathcal{P}_D \cong \Z \oplus \Z/2\Z$.
\end{thm}
\begin{proof}
	Nótese que $\mathcal{P}_D \subseteq \Z[\omega]^\times$ el cual es el anillo de enteros de $\Q(\sqrt{D})$.
	Una solución no trivial de $\mathcal{P}_D$ es también una unidad no trivial de $\Z[\omega]$,
	luego tiene una unidad fundamental $\eta > 1$ (teo.~\ref{thm:fundamental_units}).
	Hay dos posibilidades, o bien la unidad fundamental es tal que $\galnorm(\eta) = -1$, en cuyo caso, los elementos de $\mathcal{P}_D$
	son exactamente las potencias pares de $\eta$, i.e., potencias de $\epsilon := \eta^2$;
	o bien la unidad fundamental ya es de norma 1, en cuyo caso, las soluciones de Pell son ya las potencias de $\epsilon := \eta$.
	En ambos casos se obtiene que $\mathcal{P}_D = \{ \pm \epsilon^n : n \in \N \} \cong \Z \oplus \Z/2\Z$.
\end{proof}

\begin{thm}
	Sea $d > 0$ libre de cuadrados y sea $\omega \in \Z[\sqrt{d}]$ con coordenadas positivas tal que $\galnorm(\omega) = 1$.
	Dada una solución de la ecuación de Pell generalizada $x^2 - dy^2 = n$ existen $a, b, k \in \Z$
	tales que $x + y\sqrt{d} = (a + b\sqrt{d})\omega^k$ con
	$$ |a| \le \frac{ \sqrt{|n|}(\sqrt{\omega} + 1/\sqrt{\omega}) }{2}, \qquad |b| \le \frac{ \sqrt{|n|}(\sqrt{\omega} + 1/\sqrt{\omega}) }{2\sqrt{d}}. $$
\end{thm}
\begin{proof}
	Para $\alpha \in \Z[\sqrt{d}]_{\ne 0}$ definamos $L(\alpha) = (\log|\alpha|, \log |\overline{\alpha}|) \in \R^2$.
	Claramente para $ \alpha, \beta$ arbitrarios se tiene $L(\alpha \beta) = L(\alpha) + L(\beta)$.
	Nótese que $L(2) = (\log 2)(1, 1)$ y $L(\omega) = (\log\omega) (1, -1)$ son $\R$-linealmente independientes, de modo que
	$$ L(x + y\sqrt{d}) = c_1 (1, 1) + c_2 L(\omega) = (c_1 + c_2^\prime, c_1 - c_2^\prime), $$
	para algunos $c_1, c_2 \in \R$ y donde $c_2^\prime = c_2\log \omega$.
	Sumando coordenadas obtenemos que
	$$ c_1 = \frac{\log|x + y\sqrt{d}| + \log|x - y\sqrt{d}|}{2} = \frac{\log|n|}{2}. $$
	Ahora bien, elijamos $k$ el entero más cercano a $c_2$ de modo que $\delta := k - c_2 \in (-1/2, 1/2)$, luego tendremos que
	$$ L(x + y\sqrt{d}) = \frac{\log|n|}{2}(1, 1) + L(\omega^k) + \delta L(\omega), $$
	y definimos $a + b\sqrt{d} := (x + y\sqrt{d}) \omega^{-k}$, donde
	$$ L(a + b\sqrt{d}) = ( \log( \sqrt{|n|} ) + \delta\log\omega, \log( \sqrt{|n|} ) - \delta\log\omega ). $$
	Y, por tanto, alguno de los dos números $|a + b\sqrt{d}|, |a - b\sqrt{d}|$ está acotado por $\sqrt{|n|}\omega^{1/2} = \sqrt{|n| \omega}$
	y el otro por $\sqrt{|n|} \omega^0 = \sqrt{|n|}$.
	Así tenemos
	$$ |a| = \frac{|(a + b\sqrt{d}) + (a - b\sqrt{d})|}{2}, \qquad |a| = \frac{|(a + b\sqrt{d}) - (a - b\sqrt{d})|}{2\sqrt{d}}. $$
	Definamos $s := \max\{ |a + b\sqrt{d}|, |a - b\sqrt{d}| \}$, entonces ambos números son $s, |n|/s$ con algún orden.
	Además vimos que alguno de los dos números es $> \sqrt{|n|}$ y el otro es $< \sqrt{|n| \omega}$; como $s$ es el máximo,
	tenemos que $\sqrt{|n|} < s < \sqrt{|n|}\omega$.
	Un cálculo de las derivadas permite deducir que la función $t \mapsto t + |n|/t$ es creciente para $t \ge \sqrt{|n|}$, de modo que
	$$ |a| \le \frac{1}{2}\left( s + \frac{|n|}{s} \right) \le \frac{1}{2}\left( \sqrt{|n| \omega} + \frac{|n|}{\sqrt{|n| \omega}} \right)
	= \frac{\sqrt{|n|}( \sqrt{\omega} + 1/\sqrt{\omega} )}{2}. $$
	Y similarmente con $|b \sqrt{d}|$ por lo que dividiendo por $\sqrt{d}$ da las cotas del enunciado.
\end{proof}
\addtocategory{article}{conrad:pell}
% La ecuación de Pell generalizada podría no tener soluciones en general.

Ahora que hemos asegurado la existencia de soluciones no triviales de la ecuación de Pell, veamos cómo encontrarlas.
\begin{thm}
	Sea $D > 0$ libre de cuadrados, $|n| < \sqrt{D}$ entero y $x, y > 0$ enteros tales que $x^2 - Dy^2 = n$.
	Entonces $x/y$ ocurren como aproximantes en la fracción continua simple de $\sqrt{D}$.
\end{thm}
\begin{proof}
	Considere primero el caso con $n > 0$.
	Como $x - y\sqrt{D} > 0$, entonces vemos que $x + y\sqrt{D} > 2y \sqrt{D}$ y, por tanto
	$$ 0 < x - y \sqrt{D} = \frac{n}{x + y \sqrt{D}} < \frac{\sqrt{D}}{2y \sqrt{D}} = \frac{1}{2y}, $$
	luego dividiendo ambos lados por $y$ tenemos que $| x/y - \sqrt{D} | < 1/(2y^2)$, por lo que es un aproximante
	por el teorema~\ref{thm:approximants_satisfy_approx}.

	Si $n < 0$, entonces $Dy^2 - x^2 = |n|$ luego
	$$ y^2 - \frac{1}{D}x^2 = \frac{|n|}{D} > 0 \implies y > \frac{x}{\sqrt{D}}. $$
	Se sigue que
	$$ y - \frac{x}{\sqrt{D}} = \frac{|n|}{D(y + x/\sqrt{D})} < \frac{|n|}{2D(x / \sqrt{D})} < \frac{1}{2x}, $$
	dividiendo por $x$ llegamos a la misma conclusión.
\end{proof}
% ...
% \todo{Insertar ejemplo con fracciones continuas.}

% Esto nos otorga un método algorítmico para resolver la ecuación de Pell.
\begin{prob}
	Resuelva (con ayuda de un ordenador) la ecuación generalizada de Pell $x^2 - 103y^2 = 2$.
\end{prob}
\begin{sol}
	En primer lugar calcularemos la fracción continua simple de $\sqrt{103}$.
	Empleando que $\lfloor \sqrt{103} \rfloor = 10$, podemos calcular que $a_1$ viene dado por la parte entera de
	$$ t_1 := \frac{1}{\sqrt{103} - 10} = \frac{1}{\sqrt{103} - 10} \cdot \frac{\sqrt{103} + 10}{\sqrt{103} + 10} = \frac{1}{3}( \sqrt{103} + 10 ), $$
	como $10 < \sqrt{103} < 11$ vemos que $20/3 < t_1 < 21/3 = 7$, así que $a_1 = 6$.

	Reiterando el proceso podemos obtener la siguiente tabla:
	\[\begin{array}{r|*{8}{r}}
		n & -1 & 0 & 1 & 2 & 3 & 4 & 5 & 6 \\
		\hline
		a_n & {} & 10 & 6 & 1 & 2 & 1 & 1 & 9 \\
		p_n & 1 & 10 & 61 & 71 & 203 & 274 & 477 & 4567 \\
		q_n & 0 & 1 & 6 & 7 & 20 & 27 & 47 & 450 \\
		p_n^2 - 103 q_n^2 & {} & -3 & 13 & -6 & 9 & -11 & 2 & -11
	\end{array}\]
	Y prosiguiendo obtendremos dos piezas fundamentales de información:
	que $(477, 47)$ es la solución minimal de $x^2 - 103 y^2 = 2$ y que la solución fundamental de $x^2 - 103 y^2 = 1$
	se corresponde a $\omega := 227528 + 22419 \sqrt{103}$.
	Nótese que por el teorema anterior, ésta solución debía aparecer como aproximante y, de haber más,
	también deberían aparecer.

	Finalmente, un cálculo nos da que:
	$$ \frac{ \sqrt{2}( \sqrt{\omega} + 1/\sqrt{\omega} ) }{2} = 477 $$
	por lo que todas las soluciones de la Pell generalizada son de la forma:
	\begin{equation}
		\pm (477 + 47 \sqrt{103}) \cdot (227528 + 22419 \sqrt{103})^k.
		\tqedhere
	\end{equation}
\end{sol}

\subsection{La ecuación de Markoff}
La ecuación de Pell terminó por <<resolverse>> dando un algoritmo para iterar entre todas las soluciones,
el siguiente es otro ejemplo en el mismo espíritu:
\begin{thm}
	Considere la \strong{ecuación de Markoff}\index{ecuación!de Markoff} dada por
	\begin{equation}
		x^2 + y^2 + z^2 = 3xyz.
		\label{eqn:markoff}
	\end{equation}
	Entonces:
	\begin{enumerate}
		\item Dada una solución $(a, b, c)$ de \eqref{eqn:markoff}, entonces $(a, b, 3ab - c)$ es otra solución.
		\item Todas las soluciones positivas no triviales a la ecuación de Markoff están generadas por $(1, 1, 1)$ empleando la regla del inciso anterior,
			reordenando coordenadas o cambiando signo a dos coordenadas.
	\end{enumerate}
\end{thm}
\begin{proof}
	\begin{enumerate}
		\item Es un mero cálculo.
			% \begin{align*}
			%	 a^2 + b^2 + (3ab - c)^2 &= a^2 + b^2 + 9a^2b^2 - 6abc + c^2 \\
			%				 &= -3abc + 9a^2b^2 = 3ab(3ab - c). \tqedhere
			% \end{align*}
		\item Por casos:
			\begin{enumerate}[(a)]
				\item \underline{Si $x = y = z$:} entonces $3x^2 = 3x^3$ implica que $x \in \{ 0, 1 \}$ y descartamos la trivial.
				\item \underline{Si $x = y \ne z$:} entonces $2x^2 + z^2 = 3x^2z$ implica que $x^2 \mid z^2$ y, por tanto, $x \mid z$.
					Sea $z = wx$, luego tenemos la ecuación $2 + w^2 = 3wx$.
					Nótese que, en consecuencia, $w \mid 2$ y como $z \ne x$, necesariamente $w = 2$ lo que nos da
					la solución $(1, 1, 2)$ dada por aplicar la regla a $(1, 1, 1)$.
				\item \underline{Si $x < y < z$:}
					Basta probar que $3xy - z < z$ para poder aplicar un argumento inductivo.

					Mirando la ecuación de Markoff en $\Z[x, y][z]$ podemos obtener la solución
					$$ 2z_\pm = 3xy \pm \sqrt{ 9x^2y^2 - 4(x^2 + y^2) }. $$
					De darse el caso $z_-$ nótese que
					$$ 8x^2y^2 - 4x^2 - 4y^2 = 4x^2(y^2 - 1) + 4y^2(x^2 - 1) > 0, $$
					de modo que
					$$ x^2y^2 < 9x^2y^2 - 4(x^2 + y^2) \iff 2z_- < 3xy - xy = 2xy. $$
					No obstante, como $z$ es el mayor, vemos que $3xyz_- = x^2 + y^2 + z_-^2 < 3z_-^2$ por lo que $xy < z_-$
					lo cual es absurdo.

					Así que se da el caso $z_+$ y vemos que $3xy - z_+ = z_- < z_+$ como se quería probar. \qedhere
			\end{enumerate}
	\end{enumerate}
\end{proof}

\begin{mydef}
	Una solución $(a, b, c)$ de \eqref{eqn:markoff} con coordenadas estrictamente positivas se dice una \strong{terna de Markoff}\index{terna!de Markoff}.
	Un número $x$ que pertenece a alguna terna de Markoff se dice un \strong{número de Markoff}\index{numero@número!de Markoff}.
\end{mydef}
Hay una variedad de razones por las cuales los matemáticos tienen interés en las ternas y números de Markoff.
De partida, el teorema anterior permite construir una estructura a raíz de las ternas (desordenadas) de Markoff, llamado un árbol de Markoff.
Otra razón es que el mismo Markoff demostró que las ternas de Markoff tenían relación a un cierto tipo de formas cuadráticas.

\begin{thm}[Frobenius]
	Se cumplen:
	\begin{enumerate}
		\item Los números de una terna de Markoff son coprimos dos a dos.
		\item Todo número de Markoff impar es $\equiv 1 \pmod 4$.
		\item Todo número de Markoff   par es $\equiv 2 \pmod 8$.
	\end{enumerate}
\end{thm}
\begin{proof}
	\begin{enumerate}
		\item Sea $(a, b, c)$ una terna de Markoff.
			Nótese que si $d \in \N$ es tal que divide a dos de tres números, entonces por la ecuación \eqref{eqn:markoff}
			vemos que divide al restante, de modo que $(a; b) = (a; c) = (b; c) = (a; b; c)$.
			Finalmente, aplicando el teorema anterior, notamos que
			$$ (a; b; c) = (a; b; 3ab - c) = \cdots = (1; 1; 1) = 1. $$
		\item[2. y 3.] Nótese que como $c(3ab - c) = a^2 + b^2$, entonces $c$ no puede ser múltiplo de 4.
			Más aún, para cada $p \mid c$, tenemos que $a^2 + b^2 \equiv 0 \pod p$, donde $p \nmid ab$,
			de modo que $-1$ es un residuo cuadrático módulo $p$ y, por tanto, $p = 2$ o $p \equiv 1 \pmod 4$.
			Así todo factor primo impar $p$ de $c$ satisface $p \equiv 1\pmod 4$ y de aparecer el 2, lo hace una única vez.  \qedhere
	\end{enumerate}
\end{proof}

El siguiente teorema es original de \citeauthor{zhang2007markoff}~\cite{zhang2007markoff}:
\addtocategory{article}{zhang2007markoff}
\begin{thm}[Y.~Zhang]
	Un número de Markoff par $c$ satisface $c \equiv 2 \pmod{32}$.
\end{thm}
\begin{proof}
	Sea $(a, b, c)$ una terna de Markoff donde $c$ es par y $a, b$ son impares.
	Entonces $(a - b)/2$ es par y $c/2 \equiv 1 \pod 4$ por el teorema de Frobenius, luego
	$$ \left( \frac{a - b}{2} \right)^2 + \left( \frac{c}{2} \right)^2 = ab\cdot \frac{3c - 2}{4}. $$
	Como $c$ es coprimo con $a$ y $b$, y $c/2$ es coprimo con $(3c - 2)/4$ (¿por qué?);
	de modo que $c/2$ es coprimo con $ab(3c - 2)/4$ y, en consecuente, con $(a - b)/2$.
	Ahora bien, $ab(3c - 2)/4$ es un impar y suma de dos cuadrados coprimos, luego todos sus factores primos son $\equiv 1 \pod 4$
	(teorema~\ref{thm:primes_two_square} y siguiente), de modo que $(3c - 2)/4 \equiv 1 \pod 4$, o equivalentemente, $c \equiv 2 \pod{16}$.

	Así $3c + 2 \equiv 8 \pod{16}$ implica que $(3c + 2)/8$ es impar y un mero cálculo da
	$$ \left( \frac{a + b}{2} \right)^2 + \left( \frac{c}{2} \right)^2 = 2ab\cdot \frac{3c + 2}{8}. $$
	Aplicamos el mismo procedimiento verificando que $c/2$ es coprimo con $(3c + 2)/8$, de lo que se sigue que $(3c + 2)/8 \equiv 1 \pod 4$ y,
	en consecuente, $c \equiv 2 \pod{32}$.
\end{proof}
El teorema es agudo pues $(1, 1, 2)$ y $(1, 13, 34)$ son ternas de Markoff, de modo que $2, 34$ son pares de Markoff.

La gran pregunta sobre la ecuación de Markoff, aún abierta es la siguiente:
\begin{con}[de unicidad de Markoff]
	Sean $(a_1, b_1, c_1), (a_2, b_2, \break c_2)$ un par de ternas de Markoff donde $a_i \le b_i \le c_i$ para $i \in \{ 1, 2 \}$.
	Si $c_1 = c_2$, entonces $a_1 = a_2$ y $b_1 = b_2$.
\end{con}
Esta conjetura lleva más de un siglo sin resolución y, como otros problemas diofánticos ya estudiados, lleva un historial de demostraciones fallidas.
Diremos que el número de Markoff $c$ en la conjetura es \strong{único}\index{numero@número!de Markoff!único} si la conjetura es cierta para $c$.

\begin{lem}
	Sea $m \in \{ p^n, 2p^n \}$ para algún $n \ge 1$ y $p$ primo impar.
	Entonces, para todo $r$ coprimo con $m$, la ecuación $x^2 + r = 0 \pmod m$ tiene a lo más una única solución con $0 < x < m/2$.
\end{lem}
\begin{hint}
	Se sigue de que dicho $m$ tiene raíces primitivas por el teorema~\ref{thm:numbers_with_prim_root}.
\end{hint}
% \begin{proof}
%	 Sea $A := \Z/m\Z$.
%	 Si $m = p^n$, entonces $A^\times = U_{p^n}$ es cíclico y, por tanto, $A$ posee una raíz primitiva.
%	 Si $m = 2p^n$, entonces por el teorema chino del resto $A \cong \Z/2\Z \times \Z/p^n\Z$, de modo que $A^\times \cong U_2 \times U_{p^n} = U_{p^n}$
%	 y se concluye lo mismo.
%	 Claramente una solución de la ecuación es un elemento de $A$ y es fácil comprobar entonces que tiene, o bien cero o bien dos soluciones.
%	 De tener dos, una solución $m/2 \le y < m$ es tal que $0 < m - y \le m/2$ y $x = m-y$ es una solución.
% \end{proof}

\begin{thm}
	Un número de Markoff $c$ es único si $3c + 2$ o $3c - 2$ es de la forma $p^n, 4p^n$ u $8p^n$
	para algún $p$ primo impar y $n \ge 1$.
\end{thm}
\begin{proof}
	Sean $(a, b, c)$ y $(\tilde a, \tilde b, c)$ dos ternas de Markoff, donde $a \le b \le c$ y $\tilde a \le \tilde b \le c$.
	Dividimos por subcasos:
	\begin{enumerate}[(a)]
		\item \underline{Si $c$ es impar:} 
			Si $3c - 2 = p^n =: m$, entonces $(b - a)^2 + c^2 = abm \equiv 0 \pmod{m}$.
			Como $(c; 3c - 2) = 1$ y como
			$$ 0 < b-a < \frac{c}{2} - 1 < \frac{3c - 2}{2} = \frac{m}{2}, $$
			entonces aplicamos el lema anterior para concluir que $b - a = \tilde b - \tilde a$.
			Igualando
			\[
				ab(3c - 2) = (b - a)^2 + c^2 = (\tilde b - \tilde a)^2 + c^2 = 3\tilde a\tilde b c = \tilde a\tilde b(3c - 2),
			\]
			Así, $ab = \tilde a\tilde b =: d$; luego vemos que $\{ a, b \}$ y $\{ \tilde a, \tilde b \}$ son soluciones
			de $(d/x - x)^2 + c^2 = dc$.

			Si $3c + 2 = p^n =: m$, entonces empleamos la ecuación $(b + a)^2 + c^2 = abm$ de forma análoga.

		\item \underline{Si $c$ es par:}
			Por el teorema de Y.~Zhang tenemos que $3c - 2 \equiv 4 \pod{32}$ y $3c + 2 \equiv 8 \pod{32}$.
			Así que, si $3c - 2 = 4p^n$, empleamos la ecuación
			$$ \left( \frac{b - a}{2} \right)^2 + \left( \frac{c}{2} \right)^2 = abp^n \equiv 0 \pmod{p^n}. $$
			Y si $3c + 2 = 8p^n$, empleamos la ecuación
			$$ \left( \frac{b + a}{2} \right)^2 + \left( \frac{c}{2} \right)^2 = 2abp^n \equiv 0 \pmod{2p^n} $$
			ambas de forma análoga al caso de $c$ impar. \qedhere
	\end{enumerate}
\end{proof}

% Antes de proseguir haremos una traducción útil.
% Sea $(a, b, c)$ una terna de Markoff con $a \le b \le c$.
% Sea $u > 0$ el mínimo natural tal que $ua \equiv \pm b \pmod c$ (puesto que los números en la terna son coprimos).

% \begin{thm}[Baragar-Button-Schmutz]
%	 Un número de Markoff es único si es de la forma $p^n$ o $2p^n$ para algún $p$ primo impar y $n \ge 1$.
% \end{thm}

% Aquí expondremos un caso <<sencillo>> de cuando 

%\section{El problema de Waring}
%En 1770, el matemático inglés \textbf{Edward Waring} escribió para sí en sus \textit{Meditationes Algebraicæ}:
%\begin{displayquote}
%	Todo entero es igual a la suma de no más de 9 cubos.
%	Además, todo entero es la suma de no más de 19 potencias cuartas y \textit{así sucesivamente...}\footnotemark{}
%\end{displayquote}
%\footnotetext{\itshape Every integer is equal to the sum of not more than 9 cubes. Also every integer is the sum of not more than 19 fourth powers, and so on...}
%Esta última frase, <<así sucesivamente...>> ha sido interpretado como el \strong{problema de Waring}\index{problema!de Waring}:
%dado un natural $k\ge 1$ existe un número $g \ge 1$ tal que todo entero $n\in\Z$ se puede escribir como suma de $g$ potencias $k$-ésimas, es decir, tal que
%\begin{equation}
%	\exists a_i\in\Z \qquad n = a_1^k + a_2^k + \cdots + a_g^k.
%	\label{eqn:waring}
%\end{equation}
%En ésta sección presentaremos una prueba elemental para este hecho.
%En primer lugar, obsérvese que los números más problemáticos para ésta afirmación son valores <<pequeños>> de $a$, ya que habrían pocas potencias con las que jugar;
%por ejemplo $n = b^k - 1$ con $b$ <<pequeño>>.
%Esto obstaculiza las cotas que podamos dar para $g$, por lo que es conveniente hacer las siguientes dos definiciones:
%\begin{mydef}
%	Sea $k \ge 1$ entero.
%	Denotamos por $g(k)$ al mínimo cardinal (\textit{a priori}, posiblemente $\infty$) tal que para todo $n \in \Z$ se satisfaga \eqref{eqn:waring}.
%	Denotamos por $G(k)$ al mínimo cardinal tal que para todo $n$ \textit{suficientemente grande} se satisface \eqref{eqn:waring}.
%\end{mydef}
%Si $g(k) < \infty$, entonces las consideraciones previas sugieren que $G(k)$ es bastante menor que $g(k)$ para $k$ suficientemente grande.
%Nótese que, en los capítulos previos hemos ya deducido que $g(2) = G(2) = 4$ (teorema de Lagrange).

%Ahora vamos a introducir las siguientes definiciones:
%\begin{mydef}
%	Sea $k \ge 1$ entero.
%	Para $n$ suficientemente grande, sean:
%	$$ N := \lfloor n^{1/k} \rfloor, \qquad v := 1/100, \qquad P := N^v. $$
%	Dados $1 \le a \le q \le P$ con $b, q$ coprimos, se definen los \strong{arcos mayores}\index{arco!mayor}:
%	$$ \mathfrak{M}(q, a) := \{ \alpha \in \R : |\alpha - a/q| \le N^{v - k} \}. $$
%	Denotamos por $\mathscr{U} := (N^{v - k}, 1 + N^{v - k}]$ una traslación del intervalo semiabierto unitario y se definen los \strong{arcos menores}:
%	\index{arco!menor} 
%	$$ \mathfrak{m}(q, a) := \mathscr{U} \setminus \mathfrak{M}(q, a). $$
%	De no haber ambigüedad sobre la elección del par $a, q$, obviaremos los paréntesis en los arcos.
%\end{mydef}
%Nótese que, pese a llamarse <<arcos>>, en realidad son intervalos.

%% Dado un conjunto $\mathcal{A} = \{ a_m : m\in\N \}$ de naturales, podemos definir la serie de potencias formal
%% \[
%%	 F(z) := \sum_{m=1}^{\infty} z^{a_m}, \qquad |z| < 1,
%% \]
%% de modo que, para $s \ge 1$ entero:
%% $$ F(z)^s = \sum_{m_1=1}^{\infty} \cdots \sum_{m_s=1}^{\infty} z^{a_{m_1} + \cdots  + a_{m_s}} = \sum_{n=0}^{\infty} R_s(n) z^n, $$
%% donde $R_s(n)$ es la cantidad de representaciones de $n$ como suma de $s$ elementos de $\mathcal{A}$.
%Vamos a introducir una técnica de conteo, llamada las \textit{sumas trigonómetricas}.
%Denotaremos $e(\alpha) := \exp(2\pi \ui \alpha)$ y, dado un conjunto finito $b_1 < \cdots < b_N$ de naturales, definiremos
%$$ f(\alpha) := \sum_{j=1}^{N} e(\alpha b_j), $$
%de modo que, dado $s \ge 1$ entero, se satisface que
%$$ f(\alpha)^s = \sum_{m=0}^{sb_N} R_s(m)e(\alpha m); $$
%donde $R_s(m)$ denota la cantidad de maneras de escribir $m$ como suma de $s$ elementos de $b_j$'s.
%Ahora bien, la teoría clásica de Fourier dice que
%$$ \int_{0}^{1} e(\alpha m) \, \ud \alpha = \int_{0}^1 \exp(2\pi m \ui \cdot t) \, \ud t =
%\begin{cases}
%	1, & m = 0, \\
%	0, & m \ne 0.
%\end{cases} $$
%De modo que
%\[
%	\int_{0}^{1} f(\alpha)^s e(-\alpha n) \, \ud \alpha = R_s(n).
%\]
%Aplicando todo lo anterior para $b_j := j^k$ con $N = \lfloor n^{1/k} \rfloor$, obtenemos lo siguiente:
%\[
%	R_s(n) = \int_{\mathfrak{M}} f(\alpha)^s e(-\alpha n) \, \ud \alpha + \int_{\mathfrak{m}} f(\alpha)^s e(-\alpha n) \, \ud \alpha.
%\]
%% Primero, defínase  y, con $N$ definido como antes, y definamos
%% $$ f(\alpha) := \sum_{m=1}^{N} e(\alpha m^k). $$
%% Por definición, los $\alpha m^k$'s van a ser números entre 0 y 1 distintos.

\section*{Notas históricas}
El resultado de que $\zeta(3)$ sea irracional fue original de \citet{apery79irrationalite} y
fue rápidamente sustituído por el método de las integrales de \citeauthor{beukers79irrationality}~\cite{beukers79irrationality}.
Se contaba que cuando Apéry presentó su solución original en los Días Aritméticos (\textit{Journées Arithmétiques}) de Luminy, los estudiantes estaban
inquietos con la pizarra que se llenaba de relaciones crípticas entre fracciones continuas, y que al acabar era como si la respuesta se revelase por milagro.
Indudablemente no eran los estudiantes los únicos desorientados, también así lo estaban los colegas, por lo que la demostración con integrales de Beukers
rapidamente tomó terreno y, además, también alumbro el por qué la demostración de Apéry solo aplicaba en los casos de $\zeta(2)$ y $\zeta(3)$.

El nombre <<ecuaciones de Pell>> es desafortunado y se debe al propio L. Euler quien le acredita las ecuaciones a John Pell;
según Weil, esto podría deberse a que en el texto de \textit{Álgebra} de Wallis, el nombre de Pell aparece en repetidas ocasiones, pero no en relación
a las ecuaciones que llevan su nombre.
Cuando $D$ es relativamente pequeño, se pueden encontrar ingeniosamente soluciones a la ecuación de Pell y, tras descubrir la operación de grupo,
generar más soluciones;
por ello, la ecuación de Pell debería atribuirse mejor al indio Brahamgupta quién resolvió la ecuación $x^2 - 92y^2 = 1$ y descubrió completamente la
ley de grupo en las soluciones.
Otro gran aporte de la matemática hindú es el famoso método \textit{chakravala} que hemos omitido aquí por ser menos eficiente que las aproximaciones
por fracción continua, pero que resultó uno de los más eficaces algoritmos en la historia; su autoría se debate entre Bhaskara II (c. 1150)
o Shankar Shukla.

Además de los indios, los griegos también aportaron a la ecuación, Teón de Esmirna (c. 130 a.C.) descubrió la ley de grupo para la ecuación $x^2 - 2y^2 = \pm 1$
y Diofanto resolvió las ecuaciones con $D \in \{ 26, 30 \}$.
Una situación divertida que vincula a los griegos con las ecuaciones de Pell es el \textit{problema del ganado de Arquimedes} que plantea
calcular el número del ganado del dios Helios mediante una serie de restricciones que eventualmente se reducen a resolver una ecuación de Pell
con $D = 2^2 \cdot 609 \cdot 7766 \cdot 4657^2$;
el problema tardó siglos en resolverse y no se pudo sino mediante computador, pero esto no se corresponde tanto en su dificultad, sino en que los
números elegidos son demasiado grandes como para efectuar las cuentas a mano.%
\footnote{De hecho, se cree que en cierto modo Aristóteles sabía que, tras los despejes apropiados, se escondía una ecuación de Pell astronómica y
que intencionalmente propuso el problema en respuesta a otros trabajos griegos con números grandes.}

Otros matemáticos emplean la terminología <<ecuación de Pell-Fermat>> que es más apropiada, pero
preservamos la terminología anticuada por ser más común.
Fermat redescubrió el problema en 1657. 
El recuento histórico de la ecuación de Pell es de \citeauthor{williams:pell}~\cite{williams:pell}.
\addtocategory{historical}{apery79irrationalite, beukers79irrationality}
\addtocategory{history}{williams:pell}

\printbibliography[segment=\therefsegment, check=onlynew, notcategory=history, notcategory=historical, notcategory=other]
\bibbycategory[segment=\therefsegment, check=onlynew]

\end{document}
