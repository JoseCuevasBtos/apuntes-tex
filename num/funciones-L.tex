\documentclass[teoria-numeros.tex]{subfiles}
\begin{document}

\chapter{Funciones $L$}
Éste capítulo tiene como propósito final presentar la tesis de John Tate.
Ya hemos visto antes la construcción de las series formales de Dirichlet, de la función dseta de Riemann y de las series $L$ de Dirichlet;
en primer lugar, pretendemos generalizar la construcción para cuerpos numéricos $K$.
% donde tenemos las siguientes analogías:
% \begin{tabular}{cc}
% $\Q$ & Cuerpo global \\
% $n \ge 0$ & $\numnorm\mathfrak{a}$ \\
% Clases coprimas módulo $n$ & Ideales coprimos a $\mathfrak{a}$  \\
% Función dseta de Riemann & Función dseta de Dedekind \\
% Caracter de Dirichlet & Grössencharakter
% \end{tabular}
Parte del trabajo requerido involucra poder desarrollar análisis funcional $p$-ádico y, más específicamente, análisis adélico.

% \section{Divisores de Arakelov y el grupo de clases de rayos}
% Sea $K$ un cuerpo numérico,
% un \textit{divisor de Weil} de $\mathcal{O}_K$ es una suma formal $\sum_{\mathfrak{p}} a_{\mathfrak{p}} [\mathfrak{p}]$ finita
% con $a_{\mathfrak{p}} \in \Z$ y donde los corchetes denotan que vemos a $\mathfrak{p}$ como un generador de $\Div$ y no como subgrupo aditivo de $\mathcal{O}_K$.
% Así pues, es exactamente lo mismo que un ideal fraccionario de $\mathcal{O}_K$ pero en notación aditiva.
% \begin{mydef}
% 	Un \strong{divisor de Arakelov al infinito} de $\Spec(\mathcal{O}_K)$ es una suma formal
% 	$$ \sum_{v \in M_K^\infty} \lambda_v [v], \qquad \lambda_v \in \R, $$
% 	y definimos el grupo de \strong{divisores de Arakelov} como
% 	$$ \Div_{\rm Ar}(\mathcal{O}_K) = \Div(\mathcal{O}_K) \oplus \Div_\infty(\mathcal{O}_K) = \Div(\mathcal{O}_K) \oplus \R^{\oplus M_K^\infty}. $$
% 	% Un \strong{ideal de Arakelov}\index{ideal!de Arakelov} de $\mathcal{O}_K$ es un producto formal de un ideal fraccionario $\mathfrak{a} \subseteq K$
% 	% y un \textit{ideal de Arakelov al infinito} que corresponde al producto formal
% 	% $$ \prod_{\mathfrak{p} \mid \infty} \mathfrak{p}^{e_{\mathfrak{p}}}, $$
% 	% donde $e_{\mathfrak{p}} \in [0, \infty) \subseteq \R$.
% \end{mydef}

\section{Series de Dirichlet}
Volvemos al contexto del capítulo~\ref{ch:arithmetic_func}.
\begin{mydef}
	Sea $f \colon \N \to \C$ una función aritmética, definimos su \strong{serie de Dirichlet} asociada como la función
	\[
		D(f; s) := \sum_{n=1}^{\infty} \frac{f(n)}{n^s},
	\]
	donde la serie converja absolutamente.
\end{mydef}
\begin{prop}
	Sean $f$ y $g$ un par de funciones aritméticas con $h := f * g$ (la convolución de Dirichlet)
	y supongamos que las series de $D(f; s)$ y $D(g; s)$ convergen absolutamente en un punto $s$.
	Entonces $D(h; s)$ converge absolutamente en $s$ y
	\[
		D(h; s) = D(f; s)D(g; s).
	\]
\end{prop}

Sea $a_n \colon \N \to \C$ una función aritmética (en notación de sucesión) y definamos
\[
	A(t) := \sum_{n \le e^t} a_n,
\]
de modo que $x \mapsto A(\log x)$ es la función suma de $a_n$.
Así, la serie de Dirichlet en $s \in \C$ es la transformada de Laplace-Stieltjes
\[
	\{ \mathcal{L}^* A \}(s) := \lim_{h \to 0^+} \int_{h}^{\infty} e^{-ts} \, \ud A(t) = \sum_{n=1}^{\infty} \frac{a_n}{n^s},
\]
donde recuerde que $\ud A(t)$ es la medida de Riemann-Stieltjes inducida por $A(t)$.
Diremos que dicha integral \emph{converge absolutamente}\index{converge absolutamente (integral)} si la integral
\[
	\int_{0^+}^{\infty} |e^{-ts}| \, |\ud A(t)| < \infty
\]
existe, donde $|\ud A(t)|$ denota la medida de variación total de $\ud A(t)$
(cfr.\ \citeauthor{lang:analysis}~\cite[196\psq]{lang:analysis}).

Para el siguiente resultado, denotaremos $\mathcal{V} \subseteq \Func(\R; \C)$ el conjunto de funciones de variación acotada en todo compacto.
\begin{thm}\label{thm:conv_abscissa}
	Sea $A \in \mathcal{V}$ y sea $F(s) := \{ \mathcal{L}^* A \}(s)$ su transformada de Laplace-Stieltjes.
	\begin{enumerate}
		\item Si la integral que determina a $F$ converge en complejo $s_0$, entonces converge para todo complejo con $\Re(s) > \Re(s_0)$
			y la convergencia es uniforme en los sectores de la forma
			\[
				S(\theta) := \{ s \in \C : |\arg(s - s_0)| \le \theta \},
			\]
			donde $\theta \in (0, \pi/2)$, ver fig.~\ref{fig:num/LapStie}.
			\begin{figure}[!hbtp]
				\centering
				\includegraphics{num/LapStie.pdf}
				\caption{}%
				\label{fig:num/LapStie}
			\end{figure}
		\item Si la integral de $F$ converge absolutamente en $s_0$, entonces converge absoluta y uniformemente
			en el semiplano derecho $\{ s \in \C : \Re(s) \ge \Re(s_0) \}$.
		\item La función $F$ es holomorfa en todo dominio de convergencia de la integral y
			\[
				F^{(j)}(s) = \int_{0^-}^{\infty} (-t)^j e^{ts} \, \ud A(s).
			\]
	\end{enumerate}
\end{thm}
\begin{proof}
	\begin{enumerate}
		\item Por geometría básica, $s \in S(\theta)$ syss
			\[
				|s - s_0| \le \frac{\Re(s - s_0)}{\cos\theta}.
			\]
			% Que la integral converja equivale a que para todo $\epsilon > 0$ existe $x_0 \in \R$ suficientemente grande
			% tal que para todo $y \ge x \ge x_0$ se cumple que
			% \[
			% 	\forall s \in S(\theta) \qquad \left\lvert \int_{x}^{y} e^{-ts} \, \ud A(t) \right\rvert \le \epsilon.
			% \]
			Definamos, para $u \ge 0$ real,
			\[
				g(u) := \int_{0^+}^{u} e^{-ts_0} \, \ud A(t),
			\]
			por definición de convergencia existe $x_0$ suficientemente grande tal que para todos $v \ge u \ge x_0$ se cumple que
			\[
				|g(v) - g(u)| \le \frac{\epsilon}{2}\cos\theta.
			\]
			Luego para $y \ge x \ge x_0$ y $s \in S(\theta) \setminus \{ s_0 \}$ se cumple que
			\begin{align*}
				\left\lvert \int_{x}^{y} e^{-ts} \, \ud A(t) \right\rvert
				&= \left\lvert \int_{x}^{y} e^{-u(s - s_0)} \, \ud( g(u) - g(x) ) \right\rvert \\
				&= \begin{multlined}[t]
					\Big\lvert e^{-y(s - s_0)}( g(u) - g(x) ) \\
					{} + (s - s_0)\int_{x}^{y} e^{-u(s - s_0)}( g(u) - g(x) ) \, \ud u \Big\rvert
				\end{multlined} \\
				&\le \frac{\epsilon}{2}\cos\theta + |s - s_0|\cdot \frac{\epsilon}{2}\cos\theta \int_{x}^{y} e^{-u\Re(s - s_0)} \, \ud u \\
				&\le \frac{\epsilon}{2}\cos\theta \left( 1 + \frac{|s - s_0|}{\Re(s - s_0)} \right) \le \frac{\epsilon}{2}(\cos\theta + 1)
				\le \epsilon.
			\end{align*}
		\item Basta notar que
			\[
				\int_{x}^{y} |e^{-ts}| \, |\ud A(t)| \le \int_{x}^{y} |e^{-ts_0}| \, |\ud A(t)|.
			\]
		\item La analiticidad se sigue de las anteriores aplicando el criterio de convergencia de Weierstrass para límites uniformes
			de funciones holomorfas (cfr.\ \citeauthor{simon:complex}~\cite[82]{simon:complex}, Thm.~3.1.5).

			Para calcular explícitamente las derivadas, hagamos la expansión en series de Taylor de $e^{-ts}$:
			\[
				\int_{0^-}^{x} e^{-ts} \, \ud A(t) = \sum_{n=0}^{\infty} \frac{s^n}{n!} \int_{0^-}^{x} (-t)^n \, \ud A(t),
			\]
			y el lado derecho es una suma de funciones enteras cuyas derivadas son
			\begin{equation}
				\sum_{n=j}^{\infty} \frac{s^{n-j}}{(n-j)!} \int_{0^-}^{x} (-t)^n \, \ud A(t) =
				\int_{0^-}^{x} (-t)^j e^{-ts} \, \ud A(t).
				\tqedhere
			\end{equation}
	\end{enumerate}
\end{proof}

\begin{mydef}
	Sea $A \in \mathcal{V}$.
	Definimos $\sigma_c := \sigma_c(A)$ (resp.\ $\sigma_a := \sigma_a(A)$), llamados la \strong{abscisa de convergencia}
	(resp.\ \strong{convergencia absoluta}) como el ínfimo de $\Re(s)$ para los complejos $s \in \C$ tales que
	la transformada de Laplace-Stieltjes $\{ \mathcal{L}^*A \}(s)$ converge condicionalmente (resp.\ absolutamente).
	Por convención, $\sigma_c, \sigma_a \in \R \cup \{ \pm\infty \}$.
\end{mydef}

\begin{prop}
	Sea $A \in \mathcal{V}$.
	Supongamos que $F(s) := \{ \mathcal{L}^*A \}(s)$ tiene abscisa de convergencia $\sigma_c$ y que admite una continuación analítica $\tilde F(s)$
	definida en un punto con $\Re(s) = \sigma_c$.
	Entonces
	\[
		\tilde F(s) = \int_{0^-}^{\infty} e^{-ts} \, \ud A(t).
	\]
\end{prop}
\begin{proof}
	Por uniformidad \smash{$\displaystyle F(s) = \lim_{\delta\to 0^+} F(s + \delta)$} y por continuidad
	\begin{equation}
		\tilde F(s) = \lim_{\delta\to 0^+} \tilde F(s + \delta) = \lim_{\delta\to 0^+} F(s + \delta) = F(s).
		\tqedhere
	\end{equation}
\end{proof}

\begin{thm}
	Sea $f$ una función aritmética y sea $F(s) := D(f; s)$ su serie de Dirichlet. Entonces $\sigma_c \le \sigma_a \le \sigma_c + 1$.
\end{thm}
\begin{proof}
	Sea $\epsilon > 0$. De la convergencia condicional de $\sum_{n=1}^{\infty} |f(n)|/n^{\sigma_c + \epsilon}$ se sigue que
	$f(n) \ll_\epsilon n^{\sigma_c + \epsilon}$, por lo que
	\[
		\frac{f(n)}{n^{\sigma_c+1+2\epsilon}} \ll_\epsilon \frac{1}{n^{1+\epsilon}},
	\]
	cuya serie converge absolutamente, así que $\sigma_a \le \sigma_c + 1 + 2\epsilon$ y con $\epsilon \to 0^+$ se concluye lo pedido.
\end{proof}

\begin{thm}[Phragmén-Landau]
	Sea $A \in \mathcal{V}$ creciente y sea $F(s) := \{ \mathcal{L}^*A \}(s)$.
	Entonces el punto $s = \sigma_c \in \R \subseteq \C$ es una singularidad para (la continuación analítica de) $F(s)$.
\end{thm}
\begin{proof}
	Procedemos por contradicción.
	Si $F$ admite una continuación analítica en un entorno de $\sigma_c$ entonces existen reales $\sigma > \sigma_c$ y $r > \sigma - \sigma_c$ tales que
	la serie de Taylor en $\sigma$
	\[
		F(s) = \sum_{n=1}^{\infty} \frac{F^{(n)}(\sigma)}{n!} (s - \sigma)^n
	\]
	converge en la bola $B_r(\sigma)$.
	El inciso 3 del teorema~\ref{thm:conv_abscissa} nos da una fórmula para las derivadas:
	\begin{align*}
		F(s) &= \sum_{n=1}^{\infty} \frac{(s - \sigma)^n}{n!} \int_{0^-}^{\infty} (-t)^n e^{-\sigma t} \, \ud A(t) \\
		     &= \sum_{n=1}^{\infty} \frac{1}{n!} \int_{0^-}^{\infty} t^n(\sigma - s)^n e^{-\sigma t} \, \ud A(t),
	\intertext{y como el integrando y la medida $\ud A(t)$ son positivas, entonces podemos intercambiar sumar e integral}
		     &= \int_{0^-}^{\infty} e^{-\sigma t} \sum_{n=0}^{\infty} \frac{t^n(\sigma - s)^n}{n!} \, \ud A(t) \\
		     &= \int_{0^-}^{\infty} e^{-\sigma t} e^{t(\sigma - s)} \, \ud A(t) = \int_{0^-}^{\infty} e^{-st} \, \ud A(t).
	\end{align*}
	Así vemos que la integral \emph{per se} converge en la bola $B_r(\sigma)$ lo cual es absurdo tomando $s$ con $\Re(s) < \sigma_c$.
\end{proof}

\begin{thm}
	Sea $A \in \mathcal{V}$, sea $F(s) := \{ \mathcal{L}^*A \}(s)$ y sea $\sigma_c$ su abscisa de convergencia.
	\begin{enumerate}
		\item Si $A(x) \ll e^{\delta x}$ para algún $\delta \in \R$, entonces $\sigma_c \le \delta$.
		\item Si la integral que define a $F$ converge en $s_0$ con $\sigma_0 := \Re(s_0) > 0$, entonces $A(x) = o(e^{\sigma_0 x})$.
		\item Si la integral que define a $F$ converge en $s_0$ con $\sigma_0 := \Re(s_0) < 0$, entonces existe $\alpha \in \R$ tal que
			\[
				A(x) = \alpha + o(e^{\sigma_0 x}).
			\]
	\end{enumerate}
\end{thm}
\begin{proof}
	\begin{enumerate}
		\item Basta aplicar integración por partes
			\[
				\int_{0^-}^x e^{-ts} \, \ud A(t) = A(x)e^{-xs} - A(0) + s\int_{0}^{x} e^{-ts}A(t) \, \ud t,
			\]
			por lo que converge para $\Re(s) \ge \delta$.
		\item Por hipótesis
			\[
				B(x) := \int_{0^-}^{x} e^{-ts_0} \, \ud A(t) = F(s_0) + o(1).
			\]
			Se sigue que
			\begin{align*}
				A(x) &= \int_{0^-}^{x} e^{s_0 t} \, \ud B(t) = e^{s_0x} B(x) - s_0 \int_{0}^{x} e^{s_0t}B(t) \, \ud t \\
				     &= s_0\int_{0}^{x} \big( B(x) - B(t) \big) e^{s_0t} \, \ud t + B(x),
			\end{align*}
			y concluimos puesto que $B(x) - B(t) = o(1)$ cuando $x \to \infty$.
		\item Por el inciso 1 del teorema~\ref{thm:conv_abscissa}, sabemos que $F(s)$ converge para $s = 0$, así que $\alpha := F(0)$.
			Luego
			\begin{align*}
				\alpha - A(x) &= \int_{x}^{\infty} e^{s_0t} \, \ud B(t) = -e^{s_0x}B(x) - s_0 \int_{x}^{\infty} e^{s_0 t}B(t) \, \ud t \\
					      &= s_0 \int_{x}^{\infty} \big( B(x) - B(t) \big) e^{s_0t} \, \ud t \\
					      &= s_0 \int_{x}^{\infty} o(e^{\Re(s_0)t}) \, \ud t
					      = o(e^{\Re(s_0)x}).
					      \tqedhere
			\end{align*}
	\end{enumerate}
\end{proof}
\begin{cor}
	Sea $\chi$ un caracter de Dirichlet módulo $n$ no principal.
	Entonces su función $L(\chi, s) := D(\chi, s)$ tiene $\sigma_c = 0$ y $\sigma_a = 1$.
\end{cor}
\begin{proof}
	Que $\sigma_a = 1$ es trivial de que, para $m$ coprimo a $n$, siempre $|\chi(m)| = 1$.
	Que $\sigma_c = 0$ se sigue del hecho de que $\sum_{(b; n) = 1} \chi(b) = 0$, por lo que su función suma está acotada.
\end{proof}

\begin{thm}
	Sea $A \in \mathcal{V}$ y defínase \smash{$\displaystyle\kappa := \limsup_{x \to \infty} x^{-1}\log|A(x)|$}.
	\begin{enumerate}
		\item Si $\kappa \ne 0$, entonces $\sigma_c = \kappa$ para $\{ \mathcal{L}^*A \}(s)$.
		\item Si $\kappa = 0$, entonces o bien \smash{$\displaystyle\lim_{x \to \infty} A(x)$} no existe y $\sigma_c = 0$,
			o bien \smash{$\displaystyle\alpha := \lim_{x \to \infty} A(x) \in \R$} y
			\[
				\sigma_c = \limsup_{x \to \infty} x^{-1}\log|A(x) - \alpha| \le 0.
			\]
	\end{enumerate}
\end{thm}
\begin{proof}
	Nótese que, en cualquier caso, dado $\epsilon > 0$, se cumple que $A(x) \ll_\epsilon \exp((\kappa + \epsilon)x)$, por lo que,
	aplicando el teorema anterior, se sigue que $\sigma_c \le \kappa$.
	\begin{enumerate}[(a)]
		\item Supongamos que $\kappa > 0$, entonces $F(s)$ diverge para $0 < \Re(s) < \kappa$ ya que de lo contrario $A(x) = o(e^{\Re(s)x})$,
			lo que contradice la definición de $\kappa$.
			Así, $\sigma_c = \kappa$.
		\item Si $\kappa < 0$, entonces $A(x) \to 0$ y, nuevamente, el inciso 3 del teorema anterior implica que $A(x) = o(e^{\sigma x})$
			para todo $\sigma > \sigma_c$.
			Luego por definición del $\kappa$ vemos que para $\Re(s) < \sigma_c$ no se satisface que $A(x) = o(e^{\Re(s)x})$.
		\item Si $\kappa = 0$ y $\lim_{x \to \infty} A(x)$ no existe, entonces $F(s)$ diverge en $s = 0$.
			Si $\alpha := \lim_{x \to \infty} A(x) \in \R$, entonces $A(x) = \alpha + o(1)$ y, definiendo $\xi$ como el ínfimo
			de los reales que satisfacen que $A(x) = \alpha + o(e^{\xi x})$ vemos, por el inciso 3 del teorema anterior,
			que $\sigma_c \ge \xi$.
			Por otro lado, es fácil probar que $\sigma_c \le \xi$ haciendo integración por partes, lo que concluye el resultado.
			\qedhere
	\end{enumerate}
\end{proof}

\section{Caracteres de Hecke}
Un \strong{caracter} (usual, o de grupos) sobre un grupo (topológico) $G$ es
un homomorfismo (continuo) $G \to \SS^1 = \{ z \in \C : |z| = 1 \}$.
Cuando agreguemos alguna clase de apellido, trabajaremos con funciones que en muchos casos \textit{no} serán caracteres.

\begin{mydef}[Hecke]
	Sea $K$ un cuerpo numérico y sea $\mathfrak{a} \nsle \mathcal{O}_K$ un ideal entero,
	denotaremos por $J^{\mathfrak{a}}$ el conjunto de ideales fraccionarios de $\mathcal{O}_K$ que son coprimos con $\mathfrak{a}$
	(i.e., el subgrupo generado por los primos $\mathfrak{p \nmid a}$).
	Un \strong{Größencharakter}\index{grossencharakter@Größencharakter}\footnotemark{} módulo $\mathfrak{a}$
	es un caracter de grupos $\chi \colon J^{\mathfrak{a}} \to \SS^1$ tal que existen un par de caracteres continuos
	\[
		\chi_{\rm fin} \colon (\mathcal{O}_K / \mathfrak{a})^\times \to \SS^1, \qquad
		\chi_{\infty}  \colon \left( \prod_{v \mid \infty} K_v \right)^\times \to \SS^1,
	\]
	tales que
	$$ \chi(b \mathcal{O}_K) = \chi_{\rm fin}(b \mod{\mathfrak{a}}) \chi_\infty(b), $$
	para todo $b$ coprimo a $\mathfrak{a}$.
	Aquí a $\mathcal{O}_K / \mathfrak{a}$ lo dotamos de la topología discreta,
	y $\prod_{v\mid\infty} K_v$ es un producto finito de $\R$'s y $\C$'s, dotado de la topología producto.
\end{mydef}
\footnotetext{Del de., \textit{caracter grande}. Plural: \textit{Größencharaktere}.}
\begin{cor}
	Sea $K$ un cuerpo numérico, sea $\mathfrak{a} \nsle \mathcal{O}_K$ un ideal entero y sea $\chi$ un Größencharakter módulo $\mathfrak{a}$.
	Entonces los caracteres continuos $\chi_{\rm fin}$ y $\chi_\infty$ son únicos.
\end{cor}

\begin{prop}
	Sea $K$ un cuerpo numérico, sean $\mathfrak{a \subseteq b} \nsle \mathcal{O}_K$ un par de ideales enteros,
	y sea $\chi$ un Größencharakter módulo $\mathfrak{a}$.
	Son equivalentes:
	\begin{enumerate}
		\item $\chi$ es la restricción de un Größencharakter $\chi' \colon J^{\mathfrak{b}} \to \SS^1$ módulo $\mathfrak{b}$.
		\item $\chi_{\rm fin}$ se factoriza por
			\[\begin{tikzcd}[column sep=small]
				(\mathcal{O}_K/\mathfrak{a})^\times \ar[rr, "\chi_{\rm fin}"] \drar[two heads] &                                                               & \SS^1 \\
												               & (\mathcal{O}_K/\mathfrak{b})^\times \urar["\exists"', dashed]
			\end{tikzcd}\]
	\end{enumerate}
\end{prop}
\begin{mydef}
	Sea $\chi$ un Größencharakter módulo $\mathfrak{a} \nsle \mathcal{O}_K$.
	Se dice que $\chi$ es \strong{primitivo}\index{primitivo} si no existe un ideal entero $\mathfrak{b \mid a}$
	y un Größencharakter $\chi'$ módulo $\mathfrak{b}$ de modo que $\chi$ sea restricción suya.
	El $\subseteq$-mayor ideal $\mathfrak{b}$ para el cual esto se satisface se llama el \strong{conductor} de $\chi$ y se denota $\mathfrak{f}(\chi)$.
\end{mydef}

\begin{ex}
	Sea $K$ un cuerpo numérico y $\mathfrak{a} \nsle \mathcal{O}_K$ un ideal entero.
	Siempre tenemos el Größencharakter trivial $\chi_0(\mathfrak{b}) = 1$ de conductor $\mathfrak{f}(\chi_0) = (1)$.
\end{ex}

\printbibliography[segment=\therefsegment, check=onlynew, notcategory=historical, notcategory=other]

\end{document}
