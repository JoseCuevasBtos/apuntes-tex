\documentclass[teoria-numeros.tex]{subfiles}
\begin{document}

\chapter{Funciones aritméticas}
\label{ch:arithmetic_func}

\section{Funciones multiplicativas}
\nocite{hua:number}
\begin{mydef}
	Se dice que una función $f$ es \strong{aritmética}\index{función!aritmética} cuando su dominio es $\N_{\ne 0}$ y cuando su codominio es $\C$
	(puede ser un subconjunto de $\C$, como $\Z$ o $\R$).
	Una función aritmética $f$ es \strong{multiplicativa}\index{multiplicativa (función)} (resp.\ \strong{completamente multiplicativa}%
	\index{completamente multiplicativa (función)}) si es no nula y $f(ab) = f(a)f(b)$ cuando $a$ y $b$ son coprimos (resp.\ para todos $a, b$).
\end{mydef}

\begin{prop}
	Se cumplen:
	\begin{enumerate}
		\item Toda función completamente multiplicativa es multiplicativa.
		\item Las potencias $n \mapsto n^k$ para $k$ entero fijo son completamente multiplicativas.
			En particular, la identidad lo es.
		\item Si $f$ es multiplicativa, entonces $f(1) = 1$.
	\end{enumerate}
\end{prop}
\begin{proof}
	Probaremos la 3:
	Basta notar que todo $n$ es coprimo con 1, luego $f(n) = f(n)f(1)$.
	Además, como $f$ no es nula, existe un $n$ donde no lo es y se concluye que $f(1) = 1$.
\end{proof}

Una aplicación inmediata del teorema fundamental de la aritmética es el siguiente:
\begin{thm}
	Se cumplen:
	\begin{enumerate}
		\item Sea $f$ una función definida sobre las potencias de los números primos de codominio $\C$,
			entonces existe una única función $f^*$ multiplicativa que extiende a $f$.
			Y de hecho, $f^*$ está dada por:
			$$ f^*(p_1^{k_1}\cdots p_n^{k_n}) = f(p_1^{k_1}) \cdots f(p_n^{k_n}), $$
			donde los $p_i$'s son primos distintos y donde los $k_i$'s son naturales no nulos.

		\item Así mismo, si $f$ está definida sobre los números primos con codominio $\C$,
			entonces existe una única función $f^*$ completamente multiplicativa que extiende a $f$.
			Y de hecho, $f^*$ está dada por:
			$$ f^*(p_1^{k_1}\cdots p_n^{k_n}) = f(p_1)^{k_1} \cdots f(p_n)^{k_n}, $$
			donde los $p_i$'s son primos distintos y donde los $k_i$'s son naturales no nulos.
	\end{enumerate}
\end{thm}

% Otro ejemplo, menos trivial, de las funciones multiplicativas es el siguiente, que ya vimos en el libro de álgebra:
% \begin{mydef}
% 	Se define la función $\phi$ o indicatriz de Euler a
% 	$$ \phi(x) := |\{ n : n\le x \wedge (n; x) = 1 \}|, $$
% 	% \nomenclature{$\phi(x)$}{$= |\{ n : n\le x \wedge (n; x) = 1 \}|$; función $\phi$ o indicatriz de Euler}
% 	\nomenclature{$\phi(x)$}{$\displaystyle = \sum_{n\le x \atop (n; x) = 1} 1$; función $\phi$ o indicatriz de Euler}
% 	es decir, la cantidad de coprimos a $x$ menores que él (ver fig.~\ref{fig:euler-phi-function}).
% \end{mydef}
% \begin{figure}[!hbt]
% 	\centering
% 	\includegraphics[scale=1]{euler-phi-function.pdf}
% 	\caption{Función $\phi$ de Euler.}%
% 	\label{fig:euler-phi-function}
% \end{figure}

% Otra proposición que demostramos en \cite[Cor.~1.31, Prop.~1.32]{Alg}:
En \S\ref{sec:unidades_Zn} ya definimos a la función $\phi$ de Euler y vimos que es multiplicativa (teorema~\ref{thm:euler_phi_func_multiplicative}),
por lo que el teorema anterior induce lo siguiente:
\begin{prop}
	% La función $\phi$ de Euler es multiplicativa.
	Si $p$ es primo y $k \ne 0$, entonces
	$$ \phi(p^k) = p^k - p^{k-1} = p^k\left(1 - \frac{1}{p}\right), $$
	por lo que, para todo $n > 1$ se da que
	$$ \phi(n) = n\prod_{p \mid n}\left( 1 - \frac{1}{p} \right). $$
\end{prop}
Ésto permite entender mejor el gráfico que habíamos visto de la función $\phi$ y nos debería permitir reconocer las siguientes rectas
(ver fig.~\ref{fig:euler-phi-explained}).
\begin{figure}[!hbt]
	\centering
	\includegraphics[scale=1]{euler-phi-explained.pdf}
	\caption{Función $\phi$ de Euler (con rectas de apoyo).}%
	\label{fig:euler-phi-explained}
\end{figure}

\begin{prop}
	Para todo $n\in\N_{\ne 0}$ se cumple
	$$ n = \sum_{d \mid n} \phi(d). $$
\end{prop}
\begin{proof}
	Sabemos que
	$$ S := \{1, 2, \dots, n\} $$
	tiene cardinalidad $n$, y definamos
	$$ C_x := \{ m : m\le x \wedge (m; x) = 1 \}, $$
	claramente $\phi(x) := |C_x|$.
	\par
	Sea $k \in S$, veremos que existe una biyección entre $S$ y $\coprod_{d\mid n} C_d$ (la unión disjunta de los $C_d$'s).
	Para ello, sea $d := (k; n)$, entonces $(k/d; n/d) = 1$ con $n/d \mid n$, así que $k \mapsto k/d \in C_{n/d}$.
	Claramente ésta aplicación es intectiva y queda al lector ver que es suprayectiva.
\end{proof}

\begin{mydef}
	Se define la función de Möbius $\mu$ como la única función multiplicativa tal que
	$$ \mu(p^k) :=
	\begin{cases}
		-1, &k = 1 \\
		0, &k > 1
	\end{cases} $$
	\nomenclature{$\mu(n)$}{$=0$ si es divisible por $p^2$, $=-1$ si es producto de impares primos, $=1$ si es producto de pares primos. Función de Möbius}
	Nótese que si $n = p_1 \cdots p_m$ con $p_1, \dots, p_m$ primos distintos, entonces $\mu(n) = (-1)^m$, pero
	$\mu$ es nula si $n$ es divisible por el cuadrado de un primo.

	Se define la función unitaria $\varepsilon$ como
	$$ \varepsilon(n) :=
	\begin{cases}
		1, &n = 1\\
		0, &n\ne1
	\end{cases} $$
	\nomenclature{$\varepsilon(n)$}{$= 1$ si $n=1$ y $= 0$ en otro caso. Función unitaria}
\end{mydef}

\begin{prop}\label{thm:moebius_inversion}
	Para todo $n\in\N_{\ne 0}$ se cumple que
	$$ \sum_{d \mid n} \mu(d) = \varepsilon(n) $$
\end{prop}
\begin{proof}
	Lo demostraremos por inducción fuerte, donde el caso base ($n=1$) es trivial.
	\par
	Si $n = p^k$, entonces sus divisores son $\{1, p, p^2, \dots, p^k\}$ y claramente satisface el enunciado.
	\par
	Si $n = ab$ con $a = p_1^{k_1} \cdots p_m^{k_m}$ y $b = p_{m+1}^{k_{m+1}}$,
	de modo que $a,b$ son coprimos.
	Luego si $d \mid n$, entonces tenemos tres posibilidades: $d\mid a$; $d\mid b$; o $d = d_1d_2$ con $d_1\mid a$ y $d_2\mid b$, con $d_1\ne 1\ne d_2$.
	Éstas son casi disjuntas dos a dos, sin embargo si $d = 1$ cabe en las dos primeras. Así pues
	$$ \sum_{d\mid n} \mu(d) = \cancelto{0}{ \sum_{d\mid a} \mu(d) } + \cancelto{0}{ \sum_{d\mid b} \mu(b) } - \mu(1)
	+ \sum_{\substack{ 1<d_1\mid a \\ 1<d_2\mid b }} \mu(d_1)\mu(d_2). $$
	Nótese que si $d_2 = p_{m+1}^\eta$ con $\eta \ge 2$, entonces su valor, bajo $\mu$, será nulo.
	Luego podemos asumir que $d_2 = p_{m+1}$ y
	\begin{align*}
		\sum_{d\mid n} \mu(d) &= -1 + \sum_{1 < d_1\mid a} \mu(d_1)\mu(p_{m+1}) \\
		&= -1 - \left( \sum_{d_1\mid a} \mu(d_1) - \mu(1) \right) = \sum_{d_1\mid a} \mu(d_1) = 0,
	\end{align*}
	donde hemos aplicado la hipótesis de inducción al cancelar las sumatorias.
\end{proof}

He aquí una relación curiosa:
\begin{thm}
	Para todo $n\in\N_{\ne 0}$ se cumple que
	$$ \phi(n) = \sum_{d\mid n} \mu(d) \frac{n}{d}. $$
\end{thm}
\begin{proof}
	Nótese que
	$$ \phi(n) = \sum_{(k; n) = 1} 1 = \sum_{k=1}^n \varepsilon\big( (n; k) \big) $$
	y empleando la proposición anterior:
	$$ \phi(n) = \sum_{k=1}^n \sum_{d \mid (n; k)} \mu(d) = \sum_{k=1}^n \sum_{\substack{ d\mid n \\ d\mid k }} \mu(d). $$
	Analicemos bien la sumatoria.
	Se satisface que $d\mid n$ en todo caso, y si $d\mid k$, entonces $k = qd$ con $1\le k \le n$ se ha de cumplir que $1 \le q \le n/d$.
	Es decir:
	\begin{equation}
		\phi(n) = \sum_{d\mid n} \sum_{q=1}^{n/d} \mu(d) = \sum_{d\mid n} \mu(d) \sum_{q=1}^{n/d} 1 = \sum_{d\mid n} \mu(d) \frac{n}{d}. \tqedhere
	\end{equation}
\end{proof}

Ésto motiva la siguiente definición:
\begin{mydefi}
	Sean $f, g$ funciones aritméticas.
	Entonces se le dice \strong{convolución} o \strong{producto de Dirichlet}\index{convolución de Dirichlet} a la función
	\nomenclature{$(f * g)(n)$}{$= \sum_{d\mid n} f(d) g\left( \dfrac{n}{d} \right)$; convolución de Dirichlet}
	$$ (f * g)(n) := \sum_{d\mid n} f(d) g\left( \frac{n}{d} \right) = \sum_{ab = n} f(a)g(b). $$
	Se denota por $1$ (en contexto de funciones) a la constante $1(n) = 1$, que claramente es multiplicativa.
\end{mydefi}
Así pues, el teorema anterior se formula como $\phi = \mu * \Id$, y la proposición~\ref{thm:moebius_inversion} como $\mu * 1 = \varepsilon$.
Pero la gran importancia radica en lo siguiente:

\begin{thm}
	Sea $\mathcal{M}$ el conjunto de las funciones aritméticas.
	Entonces $( \mathcal{M}, * )$ es un grupo abeliano, cuyo neutro es $\varepsilon$, es decir,
	para todas las funciones multiplicativas $f, g, h$ se cumple:
	\begin{enumerate}
		\item $f*g$ es multiplicativa (clausura).
		\item $f*g = g*f$ (conmutatividad).
		\item $f*(g*h) = (f*g)*h$ (asociatividad).
		\item $f*\varepsilon = \varepsilon*f = f$ (elemento neutro).
		\item Para todo $f$ existe una función multiplicativa $g$ tal que $f*g = g*f = \varepsilon$.
			A esta función $g$, que es única, se le llama la \strong{inversa de Dirichlet}\index{inversa!de Dirichlet} de $f$.
	\end{enumerate}
\end{thm}
\begin{proof}
	\begin{enumerate}
		\item Definamos $h := f*g$.
			Claramente
			$$ h(1) = (f*g)(1) = f(1)g(1) = 1. $$
			Y si $(n; m) = 1$, entonces
			$$ h(nm) = \sum_{d \mid nm} f(d) g\left( \frac{nm}{d} \right). $$
			Como $d \mid nm$, podemos elegir que $d = ab$ con $a\mid n$ y $b\mid m$, de modo que
			\begin{align*}
				h(nm) &= \sum_{a\mid n} \sum_{b\mid m} f(ab) g\left( \frac{n}{a} \frac{m}{b} \right)
				= \sum_{a\mid n} \sum_{b\mid m} f(a)f(b) g\left( \frac{n}{a} \right) g\left( \frac{m}{b} \right), \\
				&= \sum_{a\mid n} f(a)g\left( \frac{n}{a} \right) \sum_{b\mid m} f(b)g\left( \frac{m}{b} \right), \\
				&= h(n)h(m).
			\end{align*}

		\item Para ésto basta notar que $d \mid n$ syss $n/d \mid n$, y que $d \mapsto n/d$ es una biyección entre los divisores de $n$.
		\item Para todo $n \in \N_{\ne 0}$ sea
			\begin{align*}
				\big( f*(g*h) \big)(n) &= \sum_{ab = n} f(a)(g*h)(b) = \sum_{ab = n} f(a) \sum_{cd = b} g(c)h(d) \\
				&= \sum_{acd = n} f(a)g(c)h(d) = \sum_{md = n} \left( \sum_{uv = m}f(u)g(v) \right) h(d) \\
				&= \sum_{md = n} (f*g)(m)h(d) = \big( (f*g)*h \big)(n).
			\end{align*}

		\item Para todo $n \in \N_{\ne 0}$ se cumple
			$$ \sum_{ab = n} f(a)\varepsilon(b) = f(n) + \sum_{\substack{ ab=n \\ n\ne 1 }} f(a)\cancelto{0}{ \varepsilon(b) } = f(n). $$

		\item Vamos a usar el teorema fundamental de la aritmética para construir $g$ a partir de sus valores en $p^n$.
			Primero $g(p^0) = g(1) := 1$.
			Y por recursión, sobre un mismo $p$ primo, se define:
			$$ g(p^n) := -\sum_{k=1}^n f(p^k)g(p^{n-k}). $$
			De éste modo se satisface que
			$$ (f*g)(p^n) = \sum_{ab = p^n}f(a)g(b) = \sum_{k=1}^n f(p^k)g(p^{n-k}) + \cancelto{1}{f(1)}g(p^n) = 0. $$
			Es decir, $f*g$ es una función multiplicativa (por clausura) que toma 1 en el 1 y 0 en toda otra potencia de un primo.
			Pero como toda función multiplicativa viene únicamente determinada por sus valores en las potencias de los primos
			y $\varepsilon$ también cumple lo mismo, entonces se concluye que $f*g = \varepsilon$. \qedhere
	\end{enumerate}
\end{proof}

% En teoría de grupos se demuestra que en un grupo el inverso es único, lo que nos permite definir lo siguiente:
% \begin{mydef}
% 	Dada una función multiplicativa $f$, se dice que $g$ es su \strong{inversa de Dirichlet}\index{inversa!de Dirichlet}, denotada $g = f^{-1}$,
% 	si $f*g = \varepsilon$.
% \end{mydef}

\begin{cor}[fórmula de inversión de Möbius]\index{formula@fórmula!de inversión de Möbius}
	Dadas $f, g$ multiplicativas.
	Entonces $f = g * 1$ syss $g = f * \mu$.
\end{cor}
\begin{proof}
	Ésto es una reformulación de la proposición~\ref{thm:moebius_inversion}, puesto que $\mu * 1 = \varepsilon$.
\end{proof}

\begin{thm}
	Sea $f$ una función multiplicativa.
	Entonces $f$ es completamente multiplicativa syss su inversa de Dirichlet
	$$ f^{-1}(n) = \mu(n)f(n). $$
\end{thm}
\begin{proof}
	$\implies$.
	Sea $g(n) := \mu(n)f(n)$, entonces
	$$ (g*f)(n) = \sum_{d \mid n} g(d)f\left( \frac{n}{d} \right) = \sum_{d \mid n} \mu(d)f(d)f\left( \frac{n}{d} \right) = f(n)\sum_{d\mid n}\mu(d)
	= f(n)\varepsilon(n) $$
	que es igual a $\varepsilon$, puesto que si $n = 1$ entonces da $f(1) = 1$ y de lo contrario da 0.

	$\impliedby$.
	Basta probar que $f(p^n) = f(p)^n$, lo que probaremos por inducción sobre $n\ge 1$:
	El caso base es claro y si se cumple para $n$, entonces
	\begin{align*}
		(f^{-1}*f)(p^{n+1}) = 0 &= \sum_{k=0}^{n+1} f(p^k)\mu(p^k)f(p^{n+1-k}) \\
		&= f(1)\mu(1)f(p^{n+1}) + f(p)\mu(p)f(p^n) + \cancelto{0}{ \sum_{k=2}^{n+1} f(p^k)\mu(p^k)f(p^{n+1-k}) } \\
		&= f(p^{n+1}) - f(p)f(p)^n,
	\end{align*}
	de lo que se deduce el caso inductivo (los términos cancelados lo están por tener un factor de $\mu(p^{2+j})$).
\end{proof}

La fórmula de Möbius puede generalizarse involucrando el siguiente concepto:
\begin{mydef}
	Sea $F\colon (0, \infty) \to \C$ con $F|_{(0, 1)} = 0$, y sea $\alpha(n)$ una función aritmética.
	Entonces se denota
	$$ (\alpha \oast F)(x) := \sum_{n\le x} \alpha(n)F\left( \frac{x}{n} \right), $$
	donde $(\alpha \mathop{*'} F) \colon (0, \infty) \to \C$.
\end{mydef}
Nótese que si $m$ es natural no nulo, entonces $(\alpha \mathop{*'} F)(m) = (\alpha * F)(m)$.

\begin{lem}
	Sean $\alpha$, $\beta$ funciones aritméticas y $F\colon (0, \infty) \to \C$ con $F(x) = 0$ para $0 < x < 1$.
	Entonces
	$$ \alpha \mathop{*'} (\beta \mathop{*'} F) = (\alpha * \beta) \mathop{*'} F. $$
\end{lem}
\begin{proof}
	Basta seguir el siguiente razonamiento:
	\begin{align*}
		\alpha \mathop{*'} (\beta \mathop{*'} F)(x) &= \sum_{n\le x} \alpha(n) (\beta \mathop{*'} F)\left( \frac{x}{n} \right) \\
		&= \sum_{n\le x} \alpha(n) \sum_{m \le x/n}\beta(m) F\left( \frac{x}{mn} \right) \\
		&= \sum_{mn \le x}\sum_{n\le x} \alpha(n) \beta(m) F\left( \frac{x}{mn} \right) \\
		&= \sum_{r \le x} (\alpha*\beta)(r) F\left( \frac{x}{r} \right) = \big( (\alpha * \beta) \mathop{*'} F \big)(x) \tqedhere
	\end{align*}
\end{proof}

\begin{thm}[fórmula de inversión generalizada de Möbius]
	Sea $\alpha$ una función multiplicativa con inversa de Dirichlet $\alpha^{-1}$, luego
	$$ G(x) = (\alpha \mathop{*'} F)(x) \iff F(x) = (\alpha^{-1} \mathop{*'} G)(x). $$
	El ejemplo común es $\alpha = 1$ y $\alpha^{-1} = \mu$.
\end{thm}

Finalmente veamos el siguiente resultado que nos será útil más adelante:
\begin{thm}\label{thm:arithmetic_sum_formula}
	Si $f$ es aritmética, entonces
	$$ \sum_{n\le x} (f*1)(n) = \sum_{d\le x} f(d) \sfloor{ \frac{x}{d} }. $$
\end{thm}

\subsection{Primos de Mersenne y números perfectos}%
\label{sec:perfect_numbers}
Definamos la función
\nomenclature{$\sigma_s(n)$}{$= \sum_{d \mid n} d^s$}
$$ \sigma_s(n) := \sum_{d \mid n} d^s, $$
la cual es la convolución de $1$ con $n \mapsto n^s$, las cuales son multiplicativas, de modo que $\sigma_s$ también lo es.
En particular denotamos $\tau(n) := \sigma_0(s)$, la función que cuenta la cantidad de divisores de $n$,
y $\sigma(n) := \sigma_1(n)$, la función suma de los divisores de $n$ (incluyendo a $n$ mismo).
\nomenclature{$\tau(n)$}{$= \sum_{d\mid n} 1$, la cantidad de divisores de $n$}
\nomenclature{$\sigma(n)$}{$= \sum_{d\mid n} d$, la suma de divisores de $n$}
Como ambas funciones son multiplicativas, es fácil calcular sus valores, puesto que en potencias de primos valen:
$$ \tau(p^r) = r + 1, \qquad \sigma(p^r) = 1 + p + \cdots + p^r = \frac{p^{r+1} - 1}{p - 1}. $$

\begin{mydef}
	Un número $n$ se dice \strong{perfecto}\index{numero@número!perfecto} si es la suma de sus divisores impropios.
\end{mydef}
Por ejemplo, el 6 es perfecto pues $6 = 1 + 2 + 3$.
En lenguaje de $\sigma$: $n$ es perfecto syss $\sigma(n) = 2n$.

\begin{thm}[Euclides-Euler]
	Un número par es perfecto syss es de la forma $2^{n-1}(2^n - 1)$ donde $2^n - 1$ es un número primo.
\end{thm}
\begin{proof}
	$\impliedby$. Ésto es un mero calculo empleando que
	$$ 1 + 2 + \cdots + 2^n = 2^{n+1} - 1, $$
	luego si $M_n = 2^n - 1$ es un primo de Mersenne, entonces
	$$ \sigma\big( 2^{n-1}(2^n - 1) \big) = \sigma(2^{n-1}) \sigma(M_n) = (2^n - 1) 2^n = 2\big( 2^{n-1}(2^n - 1) \big). $$
	$\implies$. Sea $2^n m$ con $m$ impar un número par perfecto, luego
	$$ 2^{n+1} m = \sigma(2^n m) = (2^{n+1} - 1) \sigma(m). $$
	Como $2^{n+1} - 1$ es impar, se concluye que $\sigma(m) = 2^{n+1} a$ para cierto $a$ impar y luego, despejando se tiene que $m = (2^{n+1} - 1)a$
	y luego $\sigma(m) = 2^{n+1}a = m + a$.
	Si $a > 1$, entonces $a, m$ son divisores distintos de $m$ y $\sigma(m) \ge 1 + a + m$ lo cual es absurdo.
\end{proof}

Para el caso impar no se sabe mucho, pero se tiene lo siguiente:
\begin{thm}[Euler-De Souza]
	Si $n$ es un número perfecto impar, entonces $n = p m^2$ donde $p \equiv 1 \pmod{4}$ y $p \nmid m$.
	En consecuente, $n \equiv 1 \pmod 4$.
\end{thm}
Euler demostró que $n = p^r m^2$ con $p \equiv r \equiv 1 \pmod{4}$, mientras que \citeauthor{desouza2018odd}~\cite{desouza2018odd}
probó que necesariamente $r = 1$ y seguimos su demostración.
% \addtocategory{article}{desouza2018odd}
\begin{proof}
	Escribamos $n = p_1^{\alpha_1} \cdots p_m^{\alpha_m}$ donde cada $p_i$ es primo e impar.
	Como $n$ es impar, entonces $n \equiv \pm 1 \pmod{4}$ y, por tanto, $\sigma(n) = 2n \equiv 2 \pod 4$ y
	$$ 2n = \sigma(n) = \sigma(p_1^{\alpha_1}) \cdot \sigma(p_2^{\alpha_2}) \cdots \sigma(p_m^{\alpha_m}). $$
	Así pues, se tiene que $\sigma(p_i^{\alpha_i}) \equiv 2 \pod{4}$ para exactamente un índice $i$.

	Como los primos $p_i$'s son impares, tenemos dos casos para $p := p_i$ y $\alpha := \alpha_i$:
	\begin{enumerate}[(a)]
		\item \underline{$p \equiv -1 \pmod{4}$:}
			Entonces
			\begin{align*}
				\sigma(p^\alpha) &= 1 + p + p^2 + \cdots + p^{\alpha} \equiv 1 + (-1) + (-1)^2 + \cdots + (-1)^{\alpha} \\
						 &\equiv \begin{cases}
							 0 \pod{4}, & 2 \nmid \alpha \\
							 1 \pod{4}, & 2 \mid \alpha
						 \end{cases}
			\end{align*}
			Así pues, se sigue que $\alpha_i$ es par.

		\item \underline{$p \equiv 1 \pmod{4}$:}
			Entonces
			\[
				\sigma(p^\alpha) &= 1 + p + p^2 + \cdots + p^\alpha \equiv 1 + 1 + 1^2 + \cdots + 1^\alpha = \alpha + 1 \pod{4}.
			\]
			Como $\sigma(p_i^{\alpha_i}) \equiv 2 \pod{4}$ para algún $i$, entonces $\alpha_i \equiv 1 \pod{4}$.
	\end{enumerate}
	Así pues, ya vimos que todos los primos $\equiv 3 \pod{4}$ aparecen al cuadrado.
	Para los primos $\equiv 1 \pod{4}$, se cumple que alguno aparece con potencia $\alpha_i \equiv 1 \pod{4}$ para
	que $\sigma(p_i^{\alpha_i}) \equiv 2 \pod{4}$; mientras que para el resto, es necesario que $\sigma(p_i^{\alpha_i}) \equiv \pm 1 \pod{4}$,
	de modo que $\alpha_i \equiv 0$ o $ \alpha_i \equiv 2 \pod{4}$ y, en ambos casos, aparecen al cuadrado.

	Ahora, sea $n = p^r m^2$ con $p \equiv r \equiv 1 \pod{4}$ y $p \nmid m$.
	Defínase $a := \sigma(m^2)$ y $b := 2m^2$, tenemos que
	$$ a(1 + p + \cdots + p^r) = (1 + p + \cdots + p^r) \sigma(m^2) = \sigma(n) = 2n = 2p^r m^2 = b p^r, $$
	así que $p$ es raíz del polinomio
	\[
		f(x) := \left(1 - \frac{b}{a}\right) x^r + x^{r-1} + \cdots + 1 \in \Q[x],
	\]
	y, como $m^2$ no es perfecto (¿por qué?), tenemos que $b/a \ne 1$.
	Nótese que $\sigma(m^2) < 2m^2$ puesto que, de lo contrario, los coeficientes de $f(x)$ son positivos y $p$ no puede ser raíz.
	Así que, como $m^2 < \sigma(m^2) < 2m^2$, tenemos que $1 < b/a < 2$.
	Definiendo $d := (a; b)$, $a_1 := a/d$ y $b_1 := b/d$, vemos que una raíz entera $s$ de $f(x)$ tiene que satisfacer que $a_1 \mid s$ y,
	como $b_1/a_1$ no puede ser entero, entonces $a_1 > 1$ y necesariamente $a_1 = p$.
	Finalmente si, por contradicción, $r$ fuese mayor que $1$, tendríamos que
	$$ f(a_1) = (a_1 - b_1)a_1^{r-1} + a_1^{r-1} + \cdots + a_1 + 1 = 0, $$
	y, por tanto, $a_1 \mid 1$ lo cual es absurdo.
\end{proof}
% \begin{cor}
% 	No hay números perfectos que sean cubos perfectos.
% \end{cor}

Si bien el caso de los números perfectos pares es fácilmente soluble, el caso de los impares es otra historia.
\begin{con}
	No existen números perfectos impares.
\end{con}
Se ha comprobado que no existen impares perfectos menores que $\le 10^{1500}$ (cfr. \citeauthor{ochem2012perfect}~\cite{ochem2012perfect}).
\addtocategory{historical}{ochem2012perfect}

% \begin{thm}[Makowski]
% 	El único número par perfecto que es suma de dos cubos es 28.
% \end{thm}
% \begin{proof}
% 	Escribamos
% 	$$ n = a^3 + b^3 = (a + b)(a^2 - ab + b^2) $$
% 	para $a > 0$ y $b \ge 0$.
% 	Es claro que $a, b$ tienen que compartir paridad.
% 	Sea $\delta$ el discriminante de $x^2 - (a+1)x + a^2 - a$
% \end{proof}

% Antes de seguir con la pregunta de los números perfectos, vamos a enfocarnos en la siguiente pregunta, ¿cuándo es $2^n - 1$ primo?
% La primera acotación es:
Volviendo al caso par, ¿cuando es $2^n - 1$ primo? Nótese que:
\begin{prop}
	Sean $a > 0$ y $n\ge 2$.
	Si $a^n - 1$ es primo, entonces $a = 2$ y $n$ es primo.
\end{prop}
\begin{proof}
	En primer lugar, podemos factorizar
	$$ a^k - 1 = (a - 1)(a^{k-1} + a^{k-2} + \cdots + a + 1), $$
	donde
	$$ a^{k-1} + a^{k-2} + \cdots + a + 1 \ge a + 1 > a - 1 > 0. $$
	De modo que $a - 1 = 1$ y $a = 2$.

	Ahora procedemos por contrarrecíproca.
	Si $n$ no es primo, tenemos que $n = qd$ con $q, d > 1$.
	Luego $2^d \equiv 1 \pmod{2^d - 1}$ y
	\[
		2^n - 1 = (2^d)^q - 1 \equiv 1^q - 1 = 0 \pmod{2^d - 1},
	\]
	de modo que $2^d - 1 \mid 2^n - 1$, por lo que $2^n - 1$ no es primo.
\end{proof}

\begin{prob}
	Un número perfecto par $n$ tiene por últimos dígitos 6 u 8;
	equivalentemente, $n \equiv 6 \pmod{10}$ o $n \equiv 8 \pod{10}$.
\end{prob}
\begin{sol}
	Por el teorema de Euclides-Euler tenemos que $n = 2^{p-1}(2^p - 1)$ y, por el lema anterior, $p$ es primo.
	Luego tenemos dos casos:
	\begin{enumerate}[(a)]
		\item \underline{$p \equiv 1 \pmod{4}$:}
			Entonces $p = 4m+1$ y
			$$ n = 2^{4m}(2^{4m+1} - 1) = 2^{8m+1} - 2^{4m} = 2\cdot 16^{2m} - 16^m. $$
			Y como $16 \equiv 6 \pod{10}$ y $6^2 = 36 \equiv 6 \pod{10}$, entonces $16^m \equiv 6 \pod{10}$ para todo $m > 1$,
			así que $n \equiv 2\cdot 6 - 6 = 6 \pod{10}$.

		\item \underline{$p \equiv 3 \pmod{4}$:}
			Entonces $p = 4m + 3$ y
			\begin{align}
				n &= 2^{4m+2}(2^{4m+3} - 1) = 2^{8m + 5} - 2^{4m + 2} = 2\cdot 16^{2m + 1} - 4 \cdot 16^m \notag \\
				  &\equiv 2 \cdot 6 - 4\cdot 6 = -12 \equiv 8 \pmod{10}. \tqedhere
			\end{align}
	\end{enumerate}
\end{sol}

\begin{mydef}
	Los \strong{números de Mersenne} son aquellos de la forma $M_p := 2^p - 1$ con $p$ primo.
	Los primos que son números de Mersenne se dicen \strong{primos de Mersenne}\index{primos!de Mersenne}.
\end{mydef}
\begin{con}
	Existen infinitos primos de Mersenne, o equivalentemente, existen infinitos números pares perfectos.
\end{con}

\begin{prop}
	Si $p$ es un número primo impar, entonces todo divisor primo de $M_p$ es de la forma $2pk + 1$.
\end{prop}
\begin{proof}
	Basta notar que $q \mid M_p$ syss $2^p \equiv 1 \pmod q$, luego $d := \order_q(2) \mid p$ y $d \mid q - 1$,
	por lo que $d = p$ y $p \mid q - 1$, con lo que $q = pr + 1$.
	Finalmente, nótese que $M_p$ es impar, luego sus divisores son impares por lo que $q = 2pk + 1$.
\end{proof}

\begin{mydef}
	Se dice que un número $p$ es un \strong{primo de Sophie Germain}\index{primo!de Sophie Germain} si $p$ y $2p + 1$ son primos.
\end{mydef}

% \begin{thm}
% 	Si $q := 2n + 1$ es primo, se cumplen
% 	\begin{enumerate}[(a)]
% 		\item Si $q \equiv \pm 1 \pmod{8}$, entonces $q \mid M_n$.
% 		\item Si $q \equiv \pm 3 \pmod{8}$, entonces $q \mid M_n + 2$.
% 	\end{enumerate}
% \end{thm}

\begin{thm}
	Si $p \equiv 3 \pmod 4$ es un primo de Sophie Germain, entonces $2p + 1 \mid M_p$.
	En consecuencia, $M_p$ no es primo, exeptuando por $p = 3$ en cuyo caso $2\cdot 3 + 1 = 7 = M_3$.
\end{thm}
\begin{proof}
	Definamos $q := 2p + 1$ el cual es primo.
	Por el pequeño teorema de Fermat vemos que $2^{2p} \equiv 1 \pmod q$, por lo que $2^p \equiv \pm 1 \pmod q$.
	Si $2^p \equiv -1$, entonces se tendría que
	$$ -2 \equiv 2^{p+1} = \left( 2^{\frac{p+1}{2}} \right)^2 \pmod q, $$
	pero nótese que si $p = 4k - 1$, entonces $q = 2p + 1 = 8k + 1$, por lo que $(-2/q) = -1$ lo cual es absurdo.
\end{proof}

\begin{prop}
	Si $p, q$ son primos impares tales que $q \mid 2^p - 1$, entonces $q \equiv \pm 1 \pmod 8$.
\end{prop}
\begin{proof}
	Por hipótesis tenemos que $2^p \equiv 1 \pmod q$, y luego, por el pequeño teorema de Fermat, se cumple que $\order(2) = p \mid q - 1$.
	Como $p$ y $q$ son impares, entonces $p \mid \frac{q - 1}{2}$ por lo que
	$$ 2^{ \frac{q-1}{2} } \equiv 1 \pmod q, $$
	pero por el criterio de Euler se cumple que $(2/q) = 1$, luego $q \equiv \pm 1 \pmod 8$.
\end{proof}

\subsection{El producto de Euler y la infinitud de primos}
Terminamos ésta sección con una aplicación del producto de Cauchy a las funciones multiplicativas:

\begin{thm}[producto de Euler]
	Sea $f$ una función multiplicativa, entonces si alguna de las siguientes se cumplen:
	\begin{enumerate}[a)]
		\item $\sum_{n=1}^\infty |f(n)|$ converge.
		\item $\prod_p (1 + |f(p)| + |f(p^2)| + \cdots)$ converge.
	\end{enumerate}
	Entonces ambas se cumplen y de hecho
	$$ \sum_{n=1}^\infty f(n) = \prod_p (1 + f(p) + f(p^2) + \cdots) $$
\end{thm}
\begin{proof}
	Supongamos que se cumple la condición a), entonces definamos $L := \sum_{n=1}^\infty |f(n)|$.
	También definamos
	$$ P(x) := \prod_{p\le x} (1 + f(p) + f(p^2) + \cdots) $$
	que está bien definido pues es un producto finito de una serie absolutamente convergente.
	Más aún, nótese que en el caso de $P(3)$, por el producto de Cauchy de series absolutamente convergentes, se da que
	$$ P(3) = f(1) + ( f(2) + f(3) ) + ( f(2^2) + f(2\cdot 3) + f(3^2) ) + \cdots $$
	donde los paréntesis rodean los términos formales de la serie producto.
	Ésto se generaliza a notar que
	$$ P(x) = \sum_{\forall p\le x \; p\mid n} f(n). $$
	Por convergencia absoluta se concluye que
	$$ S := \sum_{n=1}^\infty f(n) $$
	existe y luego
	$$ S - P(x) = \sum_{\exists p>x : p\mid n} f(n), $$
	en particular,
	$$ |S - P(x)| \le \sum_{n > x} |f(n)| \le L $$
	pero por convergencia de $L$ se cumple que el término del medio converge a 0 cuando $x \to \infty$,
	que es justamente lo que se quería probar.

	Supongamos ahora que se cumple la condición b), entonces sea
	$$ L := \prod_p ( 1 + |f(p)| + |f(p^2)| + \cdots ) $$
	luego, sea
	$$ L(x) := \prod_{p\le x} ( 1 + |f(p)| + |f(p^2)| + \cdots ) $$
	que claramente satisface $\lim_n L(n) = L$.
	Además, nótese que por las observaciones anteriores
	$$ L(x) = \sum_{\forall p\le x \; p\mid n} |f(n)| \ge \sum_{n\le x} |f(x)| =: S(x) $$
	es decir, $S(x) \le L(x) \le L$, de modo que $S(x)$ está acotada superiormente y es una suma de términos positivos, así que converge
	absultamente y nos remitimos a la situación a).
\end{proof}

\begin{cor}
	Si $f$ es completamente multiplicativa, entonces dadas las condiciones del teorema anterior se cumple que
	$$ \sum_{n=1}^\infty f(n) = \prod_p \frac{1}{1 - f(p)}. $$
\end{cor}
\begin{proof}
	Basta notar que
	$$ 1 + f(p) + f(p^2) + \cdots = 1 + f(p) + f(p)^2 + \cdots = \frac{1}{1 - f(p)} $$
	por ser serie geométrica.
\end{proof}

\begin{cor}
	Para todo $s > 1$ se satisface
	$$ \zeta(s) = \prod_p \frac{1}{1 - p^{-s}}. $$
\end{cor}

Éstos resultados por sí solos ya son bastante fuertes, pero Euler ofrece dos aplicaciones curiosas.
La primera, otra demostración de la infinitud de números primos:
\begin{Proof}{{Euler \ref{thm:infinitude_primes}}}
	% \todo{Arreglar campo opcional que señale que demuestra infinitud de primos.}
	Considerese la función multiplicativa $f(n) := 1/n$, por los corolarios anteriores se tiene que
	$$ \prod_p \frac{1}{1 - p^{-1}} = \sum_{n=1}^\infty \frac{1}{n} = \infty $$
	luego no pueden haber finitos primos pues claramente el término de la izquierda convergería.
\end{Proof}

La segunda que es una afirmación más fuerte que la anterior:
\begin{thm}
	La serie $\sum_p \frac{1}{p}$ diverge.
\end{thm}
\begin{proof}
	Supongamos, por el contrario, que converge a un valor $C$.
	Entonces, siguiendo el argumento anterior nótese que
	$$ \prod_{p\le x} \frac{1}{1 - p^{-1}} = \prod_{p\le x} \left( 1 + \frac{1}{p - 1} \right) = \prod_{p\le x} \left( 1 + \frac{2}{p} \right). $$
	Luego, recordemos que $e^x = 1 + x + x^2/2! + x^3/3! + \cdots$, es decir,
	$$ \prod_{p\le x} \left( 1 + \frac{2}{p} \right) \le \prod_{p\le x} e^{2/p} = \exp\left( \sum_{p\le x} \frac{2}{p} \right) \le \exp(2C). $$
	Pero entonces se daría que
	$$ \prod_{p\le x} \frac{1}{1 - p^{-1}} = \sum_{n \le x} \frac{1}{n} \le \exp(2C), $$
	de modo que $\sum_{n=1}^\infty \frac{1}{n}$ convergería, lo que es absurdo.
\end{proof}

\section{Promedios de funciones aritméticas}
\begin{mydef}
	Se denota la \strong{parte fraccionaria}\index{parte!fraccionaria} de un real $x$ como:
	$$ \{x\} := x - \sfloor{x}. $$
	En general se subentenderá que la notación representa ésta función si está contenida en una fórmula.
\end{mydef}

Ofreceremos dos clases de demostraciones a los dos siguientes resultados:
la primera prueba es elemental y solo manipula sumatorias, mientras que la segunda emplea integrales de Riemann-Stieltjes.
Un enfoque conceptual se encuentra en \S X.1 de \citeauthor{lang:analysis}~\cite[278\psqq]{lang:analysis}.
\begin{thm}[fórmula de Abel]\index{formula@fórmula!de Abel}
	Sean $a(n)$ una función aritmética con
	$$ A(x) := \sum_{n \le x} a(n) $$
	y sea $f \in C^1\big( [y, x] \big)$. Entonces
	$$ \sum_{y < n\le x} a(n)f(n) = A(x)f(x) - A(y)f(y) - \int_x^y A(t)f'(t) \, \ud t. $$
\end{thm}
\begin{proof}
	Sean $m := \sfloor{y}$, y $k := \sfloor{x}$.
	Nótese que si $n$ es tal que $y \le n-1 \le n \le x$, entonces
	\begin{align*}
		\int_{n-1}^{n} A(t)f'(t) \, \ud t &= A(n-1)\big( f(n) - f(n-1) \big) \\
		&= \big( A(n)f(n) - A(n-1)f(n-1) \big) - a(n)f(n).
	\end{align*}
	Luego
	$$ \int_{m+1}^{k} A(t)f'(t) \, \ud t = A(k)f(k) - A(m+1)f(m+1) - \sum_{y < n\le x} a(n)f(n). $$
	Finalmente basta refinar que
	$$ \int_{k}^{x} A(t)f'(t) \, \ud t = A(k)\big( f(x) - f(k) \big) $$
	y que
	\begin{equation}
		\int_{y}^{m} A(t)f'(t) \, \ud t = A(m-1)\big( f(m) - f(y) \big) = A(y)\big( f(m) - f(y) \big). \tqedhere
	\end{equation}
\end{proof}
\begin{proof}
	Nótese que $A(t)$, en un intervalo acotado, es de variación acotada, por lo que la expresión deseada es
	$$ \sum_{y < n \le x} a(n)f(n) = \int_{y}^{x} f(t) \, \ud A(t), $$
	lo que, por integración por partes (cfr.\ \cite[282]{lang:analysis}, Prop.~X.1.4) es igual a
	\[
		\int_{y}^{x} f(t) \, \ud A(t) + \int_{y}^{x} A(t) \, \ud f(t) = \Big[ f(t)A(t) \Big]_y^x = A(x)f(x) - A(y)f(y),
	\]
	donde $\int_{y}^{x} A(t) \, \ud f(t) = \int_{x}^{y} A(t)f'(t) \, \ud t$ por cambio de variables.
\end{proof}

\begin{thmi}[Fórmula de Euler]\index{formula@fórmula!de Euler}
	Sean $y < x$ reales, $f \in C^1\big( [y, x] \big)$ y.
	Entonces
	$$ \sum_{y < n\le x} f(n) =  -\{x\}f(x) + \{y\}f(y) + \int_y^x f(t)\, \ud t + \int_y^x \{t\}f'(t)\, \ud t. $$
\end{thmi}
\begin{proof}
	Basta aplicar el teorema anterior con $a(n) = 1$ y notar que $A(x) = \sfloor{x}$.
	Así se obtiene
	\begin{align*}
		\sum_{y < n\le x} f(n) &= \sfloor{x}f(x) - \sfloor{y}f(y) - \int_y^x \sfloor{t}f'(t)\, \ud t \\
		&= xf(x) - \{x\}f(x) + \{y\}f(y) - yf(y) - \int_y^x (t - \{t\})f'(t)\, \ud t \\
		&= - \{x\}f(x) + \{y\}f(y) + \int_y^x \{t\}f'(t)\, \ud t + {\color{nicered} xf(x) - yf(y) - \int_y^x tf'(t)\, \ud t } \\
		&= - \{x\}f(x) + \{y\}f(y) + \int_y^x \{t\}f'(t)\, \ud t + {\color{nicered} \int_x^y f(t) \, \ud t },
	\end{align*}
	donde el último paso fue integración por partes.
\end{proof}
\begin{proof}
	Integramos
	\[
		\sum_{y < n\le x} f(n) = \int_{y}^{x} f(t) \, \ud\lfloor t \rfloor = \int_{y}^{x} f(t) \, \ud t - \int_{y}^{x} f(t) \, \ud\{ t \},
	\]
	donde aplicamos integración por partes al sustraendo.
\end{proof}

La fórmula anterior se generaliza en la siguiente:
\begin{thm}[fórmula de Euler-Maclaurin]
	Sean $x, y \in \Z$ enteros, $r \ge 0$ entero y sea $f \in C^{r+1}([x, y])$ diferenciable. Entonces
	\begin{multline*}
		\sum_{x < n \le y} f(n) = \int_{x}^{y} f(t) \, \ud t + \sum_{j=0}^{r} \frac{(-1)^{j+1} B_{j+1}}{(j+1)!} \cdot (f^{(j)}(y) - f^{(j)}(x)) \\
		{} + \frac{(-1)^r}{(r + 1)!} \int_{x}^{y} b_{r+1}\{ t \} f^{(r+1)}(t) \, \ud t,
	\end{multline*}
	donde $b_j(t) \in \Q[t]$ denota el $j$-ésimo polinomio de Bernoulli y $B_j = b_j(0)$ el $j$-ésimo número de Bernoulli.
\end{thm}
\begin{proof}
	Basta aplicar suma por partes a
	\begin{align*}
		\int_{a}^{b} f(t) \, \ud b_1\{ t \} &= B_1 \cdot (f(b) - f(a)) - \int_{a}^{b} b_1\{ t \}f'(t) \, \ud t, \\
		                                    &= B_1 \cdot (f(b) - f(a)) - \frac{1}{2!}\int_{a}^{b} f'(t) \, \ud b_2^\prime\{ t \},
	\end{align*}
	recordando que $b_1\{ t \} = \{ t \} - 1/2$ y que $B_1 = -1/2$.
	Luego, se prosigue inductivamente.
\end{proof}

% \begin{mydefi}[Notación <<O mayúscula>> o de Bachmann-Landau]
\begin{mydefi}[Notación de Bachmann-Landau]
	% Sea $X$ un conjunto, y sean $f \colon X \to \C$ y $g \colon X \to (0, \infty)$
	Se denota $f(x) = O(g(x))$ ó $f \ll g$ (<<notación de Vinogradov>>) si $|f|$ está eventualmente acotada por $g$;
	vale decir, si $\lambda := \limsup_{x\to\infty} |f(x)| / g(x) < \infty$.
	Ocasionalmente, a $\lambda$ se le conoce como la \strong{constante implícita}\index{constante implícita}.
	Se denota $f(x) = o(g(x))$ si $\limsup_{x\to\infty} |f(x)| / g(x) = 0$.
	\nomenclature{$f(x) = O(g(x))$}{Si $\limsup_{x\to\infty} |f(x)| / g(x) < \infty$}
	\nomenclature{$f(x) = o(g(x))$}{Si $\limsup_{x\to\infty} |f(x)| / g(x) = 0$}
	% De éste modo $O(1)$ representa cualquier función acotada y $o(1)$ representa cualquier función que converja a 0 cuando $x \to \infty$.

	Si $f(x) \ge 0$, se denota\footnotemark{} $f(x) \asymp g(x)$ cuando $f \ll g$ y $g \ll f$.
	\nomenclature{$f(x) \asymp g(x)$}{${}\iff f(x) \ll g(x), g(x) \ll f(x)$.}
	Se denota $f(x) \sim g(x)$ cuando $\lim_{x\to\infty} f(x) / g(x) = 1$,
	\nomenclature{$f(x) \sim g(x)$}{${}\iff\displaystyle\lim_{x\to\infty} \frac{f(x)}{g(x)} = 1$}
	en cuyo caso se dice que $f$ y $g$ son \strong{asintóticamente equivalentes}.
% 	Éstos símbolos fueron introducidos por Paul Bachmann y Edmund Landau junto a muchos otros símbolos más.
\end{mydefi}
\footnotetext{Algunos libros suelen emplear $f \mathop{\gg\ll} g$.}
\begin{obs}
	Una función $f$ está acotada syss $f(x) = O(1)$.
	Se cumple que \smash{$\displaystyle\lim_{x\to\infty} f(x) = 0$} syss $f(x) = o(1)$.
\end{obs}
\begin{ex}
	Se cumple que $\{ x \} \ll 1$, por lo que, $\lfloor x \rfloor = x + O(1)$.
\end{ex}
% Ésta notación nos permitirá precisar varios teoremas; por ejemplo, sabemos que hay infinitos primos así que anotar $\pi(n) \to \infty$ sería
% redundante, pero $\pi(n) \sim \frac{n}{\log n}$ es mucho más preciso, pese a que sabemos que la expresión de la derecha diverge.
% Usualmente ésta es una notación propia de la programación y elaboración de algoritmos, pero aquí también tiene un gran uso.

Para acostumbrarnos un poco a la notación de Bachmann-Landau, unas propiedades elementales:
\begin{prop}
	Sean $f, g, h$ funciones%
	\footnote{No especificamos dominio pues dependiendo del contexto se puede hablar de $\N$, $\Z$, $\Q$ o $\R$.} 
	de codominio $\R$. Entonces:
	\begin{enumerate}
		\item Para todo $\lambda\in\R$ se cumple que $f = O(g)$ implica $\lambda f = O(g)$.
		\item $O(f)\cdot O(g) = O(fg)$.
		\item $O(f) + O(g) = O\big( |f| + |g| \big)$.
		\item Si $f = O(g)$, entonces $O(f) + O(g) = O(g)$.
		\item Si $f = O(g)$ y $g = O(h)$, entonces $f = O(h)$.
		\item Si $f\sim g$ y $g\sim h$, entonces $f\sim h$.
	\end{enumerate}
\end{prop}

\begin{mydef}
	Se define la constante de Euler-Mascheroni
	\todoref{Justificar convergencia con referencia al libro de \textit{Análisis}.}
	\nomenclature{$\gamma$}{$= \lim_n \left( 1 + \frac{1}{2} + \cdots + \frac{1}{n} - \log n \right)$; constante de Euler-Mascheroni}
	$$ \gamma := \lim_n \left( 1 + \frac{1}{2} + \frac{1}{3} + \cdots + \frac{1}{n} - \log n \right) = 0.5772156649... $$
\end{mydef}

Éstas herramientas nos permiten refinar varias series divergentes:
\begin{thm}\label{thm:big-oh-calculus}
	Si $x\ge 1$, entonces:
	\begin{enumerate}
		\item $\displaystyle \sum_{n\le x} \frac{1}{n} = \log x + \gamma + O\left( \frac{1}{x} \right)$.
		\item Si $s > 0$ y $s\ne 1$, entonces
			\begin{gather*}
				\sum_{n\le x} \frac{1}{n^s} = \frac{x^{1-s}}{1-s} + C(s) + O(x^{-s}) \\
				C(s) =
				\begin{cases}
					\zeta(s), & s > 1 \\
					\displaystyle \lim_{x\to\infty} \left( \sum_{n\le x} \frac{1}{n^s} - \frac{x^{1-s}}{1-s} \right), & s < 1 \\
				\end{cases}
			\end{gather*}
		\item Si $s > 1$, entonces $\displaystyle \sum_{n\le x} \frac{1}{n^s} = O(x^{1-s})$.
		\item Si $\alpha \ge 0$, entonces $\displaystyle \sum_{n\le x} n^\alpha = \frac{x^{\alpha + 1}}{\alpha + 1} + O(x^\alpha)$.
		\item $\displaystyle \sum_{n\le x} \log n = x\log x - x + O(\log x) $.
	\end{enumerate}
\end{thm}

\begin{proof}
	\begin{enumerate}
		\item Empleamos $f(n) = 1/n$ en la fórmula de Euler:
			\begin{align*}
				\sum_{n \le x} \frac{1}{n} &= -\frac{\{x\}}{x} + 1 + \int_1^x \frac{\ud t}{t} - \int_{1}^{x} \frac{\{t\}}{t^2} \, \ud t \\
				&= O\left( \frac{1}{x} \right) + 1 + \log x - \int_{1}^{x} \frac{\{t\}}{t^2} \, \ud t \\
				&= \log x + 1 - \int_{1}^{\infty} \frac{ \{t\} }{t^2} \, \ud t + \int_{x}^{\infty} \frac{ \{t\} }{t^2} \, \ud t
				+ O\left( \frac{1}{x} \right)
			\end{align*}
			Nótese que
			$$ 0 \le \int_{x}^{\infty} \frac{ \{t\} }{t^2} \, \ud t \le \int_{x}^{\infty} \frac{1}{t^2} \, \ud t = \frac{1}{x}. $$
			Reemplazando en la fórmula se tiene que
			$$ \sum_{n\le x} \frac{1}{n} = \log x + 1 - \int_{1}^{\infty} \frac{ \{t\} }{t^2} \, \ud t + O\left( \frac{1}{x} \right) $$
			así pues reordenando los términos y considerando el límite cuando $x \to \infty$ se obtiene que
			$$ 1 - \int_{1}^{\infty} \frac{ \{t\} }{t^2} \, \ud t = \lim_{x\to\infty} \left( \sum_{n\le x} \frac{1}{n} - \log x \right) = \gamma. $$

		\item Proseguimos de manera análoga al inciso anterior:
			\begin{align*}
				\sum_{n\le x} \frac{1}{n^s} &= -\frac{\{x\}}{x^s} + 1 + \int_{1}^{x} \frac{\ud t}{t^s}
				- s\int_{1}^{x} \frac{\{t\}}{t^{s-1}} \, \ud t \\
				&= O(x^{-s}) + 1 + \frac{x^{1-s}}{1-s} - \frac{1}{1-s} - s \int_{1}^{x} \frac{\{t\}}{t^{s+1}} \, \ud t.
			\end{align*}
			Luego se cumple que
			$$ 0 \le \int_{x}^{\infty} \frac{\{t\}}{t^{s+1}} \, \ud t \le \int_{x}^{\infty} \frac{1}{t^{s+1}} \, \ud t \le \frac{s}{x^s}. $$
			Igualando las constantes se tiene que
			$$ C(s) := 1 - \frac{1}{1-s} - s\int_{1}^{\infty} \frac{\{t\}}{t^{s+1}} \, \ud t
			= \lim_{x\to\infty} \left( \sum_{n\le x} \frac{1}{n^s} - \frac{x^{1-s}}{1-s} \right). $$
			Si $s > 1$, entonces $x^{1 - s} \to 0$ y la serie converge a $\zeta(s)$ por definición de tal.
			% El resto de incisos quedan al lector.

		\item[5.] Por la suma de Euler:
			\begin{align*}
				\sum_{n\le x} \log n &= \int_{1}^{x} \log t \, \ud t + \int_{1}^{x} \frac{\{t\}}{t} \, \ud t - \{x\}\log x \\
				&= x\log x - x + \theta \log x - \{x\}\log x = x\log x - x + O(\log x).
			\end{align*}
			Aquí el $\theta \in [0, 1]$ viene inducido del hecho de que
			\begin{equation}
				0 \le \int_{1}^{x} \frac{\{t\}}{t} \, \ud t \le \int_{1}^{x} \frac{1}{t} \, \ud t = \log x.
				\tqedhere
			\end{equation}
	\end{enumerate}
\end{proof}

% \begin{mydef}
% 	Se define la función aritmética $d(n)$ como la cantidad de divisores positivos de $n$, es decir
% 	$$ d(n) := \sum_{d\mid n} 1. $$
% 	Por la ecuación anterior es claro que $d := 1*1$, y que por lo tanto $d$ es multiplicativa.
% 	Luego basta notar que $d(p^n) = n+1$ para todo $n\in\N$ para obtener una fórmula generalizada para $d(m)$.
% \end{mydef}

\begin{lem}
	Sea $f$ una función multiplicativa.
	Si $\lim_{p^m} f(p^m) = 0$, donde el subíndice recorre todas las potencias de todos los primos, entonces $\lim_n f(n) = 0$.
\end{lem}
\begin{proof}
	Como $f(p^m) \to 0$, entonces existe $A$ tal que $|f(p^m)| \le A$ para toda potencia de primos $p^m$.
	También existe $B \in \N$ tal que si $p^m \ge B$, entonces $|f(p^m)| \le 1$.

	Para todo $\epsilon > 0$ existe $C \in \N$ tal que si $p^m \ge C$, entonces
	$$ |f(p^m)| \le \frac{\epsilon}{A^B}. $$
	Sea $n \ge 1$ con factorización prima $n = p_1^{e_1} \cdots p_r^{e_r}$.
	Si cada $p_i^{e_i} < C$, entonces $n < C^C$ (pues como los $p_i$'s son crecientes, necesariamente $r < C$);
	por lo tanto, si $n \ge C^C$ entonces existe $p_i^{e_i} \mid n$ con $p_i^{e_i} \ge C$.
	Luego
	\begin{align}
		f(n) &=   \prod_{p_i^{e_i} < B} |f(p_i^{e_i})| \cdot \prod_{B \le p_i^{e_i} < C} |f(p_i^{e_i})|
		     \cdot \prod_{C \le p_i^{e_i}} |f(p_i^{e_i})| \notag \\
		     &\le \prod_{p_i^{e_i} < B} A              \cdot \prod_{B \le p_i^{e_i} < C} 1
		     \cdot \prod_{C \le p_i^{e_i}} \frac{\epsilon}{A^B} \notag \\
		     &<    A^B \cdot 1 \cdot \frac{\epsilon}{A^B} = \epsilon. \tqedhere
	\end{align}
\end{proof}

\warn
El enunciado es bien específico, nótese que si cambiamos la hipótesis a que $\lim_r f(p^r) = 0$ para todo primo $p$,
la conclusión es falsa.
En efecto, la función multiplicativa dada por $f(p^r) := p^{2 - r}$ satisface que $\lim_r f(p^r) = 0$, pero $\lim_n f(n) \ne 0$
pues la sucesión $\{ f(n) \}_{n = 1}^\infty$ posee la subsucesión $f(p) = p$ que es divergente.
Otro ejemplo es la función de Möbius.

\begin{prop}
	Se cumplen:
	\begin{enumerate}
		\item Para todo $\epsilon > 0$ se tiene que $\tau(n) = o(n^\epsilon)$.
		% \item $\phi(n) = O(n)$.
		\item $\phi(n) \ll n$.
		\item Para todo $\epsilon > 0$ se tiene que $n^{1 - \epsilon} = o(\phi(n))$.
		% \item $n = O(\sigma(n))$.
		\item $n \ll \sigma(n)$.
		\item Para todo $\epsilon > 0$ se tiene que $\sigma(n) = O(n^{1 + \epsilon})$.
	\end{enumerate}
\end{prop}
\begin{proof}
	\begin{enumerate}
		\item Definamos $f(n) := \tau(n)/n^\epsilon$, la cual es una función multiplicativa.
			Por el lema anterior, basta probar que
			\[
				\lim_{p^m} f(p^m) = \lim_{p^m} \frac{\tau(p^m)}{p^{\epsilon m}} = \lim_{p^m} \frac{m+1}{p^{\epsilon m}} = 0
			\]
			lo que se sigue de que
			\[
				\frac{m+1}{p^{\epsilon m}} \le \frac{2m}{p^{\epsilon m}} = \frac{2 \log(p^m)}{p^{\epsilon m} \log p}
				\le \frac{2}{\log 2} \cdot \frac{\log(p^m)}{p^{\epsilon m}}.
			\]
		\item Basta notar que trivialmente $\varphi(n) / n \le 1$ y que
			$$ \frac{\phi(p^m)}{p^m} = 1 - \frac{1}{p} \to 1, $$
			de modo que $\limsup_n \phi(n)/n = 1$.
		\item Definamos $f(n) := n^{1 - \epsilon}/\phi(n)$, la cual es una función multiplicativa.
			Basta notar que
			$$ f(p^m) = \frac{p^{m(1 - \epsilon)}}{p^m(1 - 1/p)} = \frac{p^{-m \epsilon}}{1 - 1/p} \le 2p^{-m \epsilon} $$
			la cual converge a 0, por lo que, aplicando el lema anterior vemos que $\lim_n n^{1 - \epsilon}/\phi(n) = 0$.
		\item Como $n < 1 + n \le \sigma(n)$ y $\sigma(p) = p + 1$ se tiene que
			$$ \limsup_n \frac{n}{\sigma(n)} = 1. $$
		\item Definamos $f(n) := \sigma(n) / n^{1 + \epsilon}$, la cual es una función multiplicativa.
			Nótese que
			\[
				f(p^m) = \frac{p^{m+1} - 1}{p - 1} \cdot \frac{1}{p^{m(1 + \epsilon)}} = \frac{1}{p^{m\epsilon}} \cdot \frac{1 - 1/p^{m+1}}{1 - p}
			\]
			el cual converge a 0 para todo $\epsilon > 0$. \qedhere
	\end{enumerate}
\end{proof}

\begin{prop}
	Para $s > 1$ se tiene la siguiente identidad:
	$$ \sum_{n=1}^{\infty} \frac{\mu(n)}{n^s} = \frac{1}{\zeta(s)}. $$
\end{prop}
\begin{proof}
	Basta notar que
	\begin{align*}
		\left( \sum_{k=1}^{\infty} \frac{1}{k^s} \right)\left( \sum_{n=1}^{\infty} \frac{\mu(n)}{n^s} \right)
		&= \sum_{k\ge 1} \sum_{n\ge 1} \frac{\mu(n)}{(kn)^s} = \sum_{n=1}^{\infty} \sum_{d\mid n} \frac{\mu(d)}{n^s} \\
		&= \sum_{n=1}^{\infty} \frac{1}{n^s}\left( \sum_{d\mid n} \mu(d) \right) = 1.
	\end{align*}
	Donde aquí empleamos el que la $s$-serie $\sum_{k=1}^{\infty} 1/k^s$ converge absolutamente, así como la serie de Möbius
	(puesto que $|\mu(n)| \le 1$ aplicando comparación de series).
\end{proof}

\begin{lem}[método de la hipérbola de Dirichlet]\index{metodo@método!de la hipérbola de Dirichlet}
	Sean $f, g$ funciones aritméticas, entonces sean $F, G$ sus funciones sumatorias, es decir
	$$ F(x) := \sum_{n\le x} f(n), \qquad G(x) := \sum_{n\le x} g(n). $$
	Entonces, para todo $1\le y\le x$ se cumple que
	$$ \sum_{n\le x}(f*g)(n) = \sum_{n\le y} F\left( \frac{x}{n} \right) g(n) + \sum_{m \le x/y} f(m) G\left( \frac{x}{m} \right)
	- F\left( \frac{x}{y} \right) G(y). $$
\end{lem}
\begin{proof}
	Basta ver que
	\begin{align}
		\sum_{n\le x}(f*g)(n) &= \sum_{md \le x} f(m)g(d) \notag \\
		&= \sum_{\substack{ md\le x \\ d\le y }} f(m)g(d) + \sum_{\substack{ md\le x \\ d>y }} f(m)g(d) \notag \\
		&= \sum_{d\le y} g(d) \sum_{m \le x/d} f(m) + \sum_{ m \le x/y } f(m) \sum_{y < d \le x/m} g(d) \notag \\
		&= \sum_{d\le y} g(d) F\left( \frac{x}{d} \right) + \sum_{ m \le x/y } f(m)\left( G\left( \frac{x}{m}\right) - G(y) \right). \tqedhere
	\end{align}
\end{proof}

\begin{mydef}
	Sea $r(n) := |\{ (a, b) \in \Z^2 : a^2 + b^2 = n \}$, la función que cuenta la cantidad de formas de representar a un natural $n$ como suma de cuadrados.
\end{mydef}
% Nótese que como los pares $(a, b)$ se pueden intercambiar por $(\pm a, \pm b)$ y $(\pm b, \pm a)$,
% entonces genéricamente $r(n)$ suele ser múltiplo de 8.
% Geométricamente $r(n)$ puede verse como los puntos del reticulado $\Z^2$ en la esfera de radio $\sqrt{n}$ en $\R^2$.

\nocite{hlawka:number}
\begin{thm}
	Se cumplen:
	\begin{enumerate}
		\item $\displaystyle \frac{1}{x} \sum_{n\le x} \tau(n) = \log x + (2\gamma - 1) + O\left( \frac{1}{\sqrt{x}} \right).$
			% En consecuencia $\displaystyle 1/x \sum_{n\le x} \tau(n) \sim \log x$.
		\item (Problema del círculo de Gauss)
			$\displaystyle \frac{1}{x} \sum_{n\le x} r(n) = \pi + O\left( \frac{1}{\sqrt{x}} \right).$
		\item $\displaystyle \frac{1}{x} \sum_{n\le x} \phi(n) = \frac{3}{\pi^2} x + O(\log x). $
	\end{enumerate}
\end{thm}
\begin{proof}
	\begin{enumerate}
		\item Nótese que la función contadora de divisores $\tau(n)$ se puede escribir como la convolución $\tau = 1*1$.
			Basta emplear el resultado anterior con $f = g = 1$, y con $y = \sqrt{x}$, en cuyo caso
			nótese que $F(x) = G(x) = \sfloor{x}$ y que:
			\begin{align*}
				\sum_{n\le x} \tau(n) &= \sum_{n\le\sqrt{x}} \sfloor{ \frac{x}{n} } + \sum_{m \le \sqrt{x}} \sfloor{ \frac{x}{m} }
				- \sfloor{ \sqrt{x} }^2 \\
						      &= 2x\sum_{n \le \sqrt{x}} \frac{1}{n} - x + O(\sqrt{x}) \\
						      &= 2x( \log\sqrt{x} + \gamma + O( x^{-1/2} ) ) - x
						      + O(\sqrt{x}).
			\end{align*}
			que reordenando términos concluye el enunciado.
		\item Nótese que $r(n)$ puede verse como la cantidad de puntos del reticulado $\Z^2$ en la esfera $S(\sqrt{n}) := \{ (x, y) \in \R^2 :
			\|(x, y)\| = \sqrt{n} \}$.
			Así que, para un entero $N$ la cantidad total de puntos del reticulado $\Z^2$ en la bola cerrada $\overline{B}_N(0)$ es
			$$ R(N) := 1 + \sum_{n \le N} r(n). $$
			Podemos asociar a cada punto de dicho reticulado, un cuadrado de $1\times 1$ (ver fig.~\ref{fig:num/gauss_circle})
			y así notar que $R(N)$ está incluido en la bola cerrada de radio $\sqrt{N} + \sqrt{2}$ (donde el $+ \sqrt{2}$ viene de que los cuadrados
			que se <<salen>> tienen una diagonal de $\sqrt{2}$) y a su vez incluyen la bola cerrada de radio $\sqrt{N} - \sqrt{2}$.
			\begin{figure}[!hbtp]
				\centering
				\includegraphics{num/gauss_circle.pdf}
				\caption{}%
				\label{fig:num/gauss_circle}
			\end{figure}

			Así vemos que
			$$ \pi\left(\sqrt{N} - \sqrt{2}\right)^2 \le 1 + \sum_{n\le N} r(n) \le \pi\left(\sqrt{N} + \sqrt{2}\right)^2, $$
			de modo que $R(N) \le \pi N + 2\pi\sqrt{2N} + 2\pi$ y $R(N) \ge \pi N - 2 \pi \sqrt{2N} + 2\pi$.

		\item Ya sabemos que $\phi(n) = \sum_{d\mid n} \mu(d) n/d$, por lo que
			\begin{align*}
				\frac{1}{x} \sum_{n\le x} \phi(n) &= \frac{1}{x} \sum_{n\le x} \sum_{ab=n} a \mu(b) = \frac{1}{x} \sum_{a} \sum_{b \le x/a} a \mu(b) \\
						      &= \frac{1}{x} \sum_{b \le x} \mu(b) \cdot \sum_{a \le x/b} a
						      = \frac{1}{x} \sum_{b\le x} \mu(b) \frac{\lfloor x/b \rfloor (\lfloor x/b \rfloor + 1)}{2} \\
						      &= \frac{1}{2x} \sum_{n\le x} \mu(n) \left( \frac xn - \left\{ \frac xn \right\} \right)
						      \left( \frac xn - \left\{ \frac xn \right\} + 1 \right) \\
						      &= \frac{x}{2} \sum_{n \le x} \frac{\mu(n)}{n^2}
						      + \frac{1}{2} \sum_{n\le x} \frac{\mu(n)(1 - 2\{ x/n \})}{n} \\
						      &\qquad {} + \frac{1}{2x} \sum_{n\le x} \mu(n) \left\{ \frac xn \right\}
						      \left(1 - \left\{ \frac xn \right\}\right)
			\end{align*}
			Por las proposiciones anteriores tenemos que
			\begin{gather*}
				\frac{1}{2} \left| \sum_{n\le x} \frac{\mu(n)(1 - 2\{ x/n \})}{n} \right| \le \frac{1}{2} \sum_{n\le x} \frac{1}{n} = O(\log x), \\
				\frac{1}{2x} \sum_{n\le x} \mu(n) \left\{ \frac xn \right\}\left(1 - \left\{ \frac xn \right\}\right) = O(1).
			\end{gather*}
			Y finalmente, acotamos el término restante:
			\begin{equation}
				\sum_{n \le x} \frac{\mu(n)}{n^2} = \sum_{n=1}^{\infty} \frac{\mu(n)}{n^2}
				- \sum_{n=\lfloor x \rfloor +1}^{\infty} \frac{\mu(n)}{n^2} \notag
								  = \frac{1}{\zeta(2)} + O\left( \frac{1}{x} \right).
								  \tqedhere
			\end{equation}
	\end{enumerate}
\end{proof}

\begin{prop}
	\smash{$ \displaystyle \frac{1}{x} \sum_{n\le x} \sigma(n) = \frac{\pi^2}{12}x + O(\log x). $}
\end{prop}
\begin{proof}
	Escribamos
	\begin{align*}
		\sum_{n \le x} \sigma(n) &= \sum_{md \le x} d
		= \frac{1}{2} \sum_{m \le x} \lfloor \frac{x}{m} \rfloor \left( \lfloor \frac{x}{m} \rfloor + 1 \right) \\
					 &= \frac{1}{2}\sum_{m \le x} \frac{x^2}{m^2} + O\left( x \sum_{m \le x} \frac{1}{m} \right),
	\end{align*}
	nótese que $x^2 \sum_{m \le x} \frac{1}{m^2} = x^2\zeta(2) + O(x)$ y que
	\begin{equation}
		x \sum_{m \le x} \frac{1}{m} = x\log x + \gamma x + O(1).
		\tqedhere
	\end{equation}
\end{proof}

Más adelante veremos que podemos agudizar la cota del problema del círculo de Gauss.
Se conjetura lo siguiente:
\begin{con}
	Se creen:
	\begin{enumerate}
		\item Es cierto que $\displaystyle \sum_{n\le x} r(n) = \pi x + O(x^{1/4 + \epsilon})$.
		\item Es falso que $\displaystyle \sum_{n\le x} r(n) = \pi x + O(x^{1/4 - \epsilon})$.
	\end{enumerate}
\end{con}

\section{Distribución de números primos}
El objetivo principal es llegar a probar uno de los resultados más importantes de la teoría analítica de números,
primero veamos las funciones involucradas:
\begin{mydefi}
	Se definen las siguientes funciones:
	\begin{align*}
		\pi(x) &:= |\{ p\in\N : p\text{ primo} \wedge p\le x \}| = \sum_{p \le x} 1, \\
		\Li(x) &:= \int_2^x \frac{\ud t}{\log t},
	\end{align*}
	a $\pi$ se le dice \strong{función contadora de primos} y a $\Li$ se le conoce como \strong{función logaritmo integral}.
	\nomenclature{$\pi(x)$}{Función de conteo de primos, $= \sum_{p \le x} 1$}%
	\nomenclature{$\Li(x)$}{Función logaritmo integral, $= \int_2^x \frac{\ud t}{\log t}$}%
\end{mydefi}

Así la observación empírica a la que llego Gauss fue a ver que éstas dos funciones <<se parecían mucho cuando $x$ era muy grande>> (ver fig.~\ref{fig:PNT}),
o más rigurosamente que daba la impresión de que eran asintoticamente equivalentes.
A éste resultado se le conoce como el \textit{teorema de números primos} (abrev., TNP) y uno de los objetivos fundamentales de éste texto será probarlo.
% Más aún, $\Li(x) \sim \frac{x}{\log x}$ que es usualmente como se expresa el teorema.
\begin{figure}[!hbt]
	\centering
	\includegraphics[scale=1]{PNT.pdf}
	\caption{Teorema de los números primos.}%
	\label{fig:PNT}
\end{figure}

En primer lugar veamos como deducir un par de cotas iniciales para $\pi(x)$.
La primera se sigue inmediata de la demostración de Euclides:
\begin{prop}
	Para $x \ge 2$ se cumple que $\pi(x) \ge \log\log x$.
\end{prop}
\begin{proof}
	Sea $p_n \colon \N \to \N$ la enumeración creciente de los números primos, de modo que $p_0 = 2$, $p_1 = 3$, etc.
	Veamos que $p_n \le 2^{2^n}$ por inducción sobre $n$:
	El caso base $p_0 = 2 \le 2^{2^0}$, y la demostración de Euclides induce que
	$$ p_{n+1} \le p_0\cdots p_n + 1 \le 2^{2^0}\cdots 2^{2^n} + 1 = 2^{2^{n+1} - 1} + 1, $$
	y, considerando $x = 2^{2^{n+1}}$, vemos que $\frac{x}{2} + 1 \le x$ syss $2 \le x$.
	Ésto nos permite concluir que $\pi(2^{2^n}) \ge n+1$.
	Así pues, si $x \ge 2$ podemos encontrar un único $n$ tal que $2^{2^n} \le x < 2^{2^{n+1}}$, o lo que es equivalente,
	$n \le \log_2\log_2 x < n+1$ y luego
	$$ \pi(x) \ge \pi(2^{2^n}) \ge n+1 > \log_2\log_2 x; $$
	finalmente, como $0 < \log 2 < 1$ vemos que $\log_2 x = (\log x)/(\log 2) > \log x$ de lo que se concluye la demostración.
\end{proof}

Curiosamente, podemos sacar una mucho mejor cota con incluso menos trabajo. Vamos a requerir la siguiente definición:
\begin{mydef}
	Un entero $n$ se dice \strong{libre de cuadrados}\index{libre de cuadrados (número)} si no existe otro $m > 1$ tal que $m^2 \mid n$.
\end{mydef}
El 1 es el único número que es un cuadrado y está libre de cuadrados.

Por el teorema fundamental de la aritmética, notamos que un natural libre de cuadrados es un producto de primos distintos (con exponente 1),
de hecho, la totalidad de números libres de cuadrados está en biyección con los subconjuntos finitos de números primos (donde al vacío $\emptyset$
se le asocia el 1), de modo que bastaría probar que hay infinitos números libres de cuadrados para probar la infinitud de los primos.
Con ésto otorgamos dos demostraciones y cotas (en simultáneo):
\begin{Proof}{Perott \ref{thm:infinitude_primes}}
	Considere el conjunto $S_N := \{1, 2, \dots, N\}$ que posee exactamente $N$ elementos.
	Nótese que podemos acotar la cantidad de números libres de cuadrados $A(N)$ en $S_N$, mediante la siguiente criba:
	eliminamos a los múltiplos de $2^2$, a los de $3^2$, a los de $4^2$ y así:
	$$ \sum_{k=2}^{\infty} \lfloor N/k^2 \rfloor \le N \sum_{k=2}^{\infty} \frac{1}{k^2} = N( \zeta(2) - 1 ),  $$
	de modo que:
	$$ A(N) \ge N - N(\zeta(2) - 1) = N(2 - \zeta(2)). $$
	Ahora, nótese que $\zeta(2) < 2$ puesto que
	$$ \zeta(2) = 1 + \sum_{k=2}^{\infty} \frac{1}{k^2} < 1 + \sum_{k=2}^{\infty} \int_{k-1}^{k} \frac{1}{t^2} \, \ud t
	= 1 + \int_{1}^{\infty} \frac{1}{t^2} \, \ud t = 2. $$
	Luego tenemos que $A(N)/N \to 2 - \zeta(2)$ el cual es $>0$, de modo que $A(N)$ es infinito.
\end{Proof}

Y de la misma demostración se sigue que:
\begin{cor}
	Se satisface que $\pi(N) \ge \log N / \log 2 + O(1)$.
\end{cor}
\begin{proof}
	Basta notar que $A(N) \le 2^{\pi(N)}$ pues, como dijimos, los enteros libres de cuadrados están en biyección con los subconjuntos finitos de los primos.
\end{proof}

Otra demostración, y más sencilla es la siguiente:
\begin{Proof}{{Erd\H os \ref{thm:infinitude_primes}}}
	Considere el conjunto $S_N := \{1, 2, \dots, N\}$ que posee exactamente $N$ elementos.
	Nótese que hay a lo más $2^{\pi(N)}$ números libres de cuadrados en $S_N$ y que hay a lo más $\sqrt{N}$ cuadrados en $S_N$.
	Finalmente, todo elemento puede expresarse como un número cuadrado por un número libre de cuadrados, de modo que
	$$ N = S_N \le \sqrt{N} \cdot 2^{\pi(N)} \implies \pi(N) \ge \log_2(\sqrt{N}) = c\cdot\log N, $$
	donde $c = \frac{1}{2}\log(2)^{-1} \approx 0.72$.
	Como $\log N \to \infty$ cuando $N \to \infty$, entonces se concluye que $\pi(N) \to \infty$.
\end{Proof}

\begin{mydef}
	Dada una función aritmética $f$, se define la \strong{función sumatoria de $F$}\index{función!sumatoria} como
	$$ F(n) := \sum_{n\le x} f(n). $$
	Y vamos a definir varias funciones aritméticas:
	\begin{align*}
		\Lambda(x) &:=
		\begin{cases}
			\log p &x = p^k, k\ge 1 \\
			0 &\text{en otro caso}
		\end{cases} \\
		\vartheta(x) &:= \sum_{p \le x} \log p, \\
		\psi(x) &:= \sum_{p^k \le x} \log p = \sum_{n \le x} \Lambda(n), \\
		T(x) &:= \sum_{n \le x} \log n.
	\end{align*}
	A $\Lambda$ se le conoce como la \strong{función de von Mangoldt},
	\nomenclature{$\Lambda(x)$}{Función de von Mangoldt, $\Lambda(p^k) = \log p$ si $k\ge 1$ y $\Lambda(x) = 0$ en otro caso}%
	\nomenclature{$\vartheta(x)$}{Primera función de Chebyshev, $= \sum_{p \le x} \log p$}%
	\nomenclature{$\psi(x)$}{Segunda función de Chebyshev, función suma de $\Lambda$; $= \sum_{n \le x} \Lambda(n)$}%
	a $\vartheta$ y $\psi$ se le conocen como \strong{primera} y \strong{segunda función de Chebyshev} resp.,
	y $T$ es simplemente la función sumatoria de logaritmo.
	Nótese que $\psi$ es la función sumatoria de $\Lambda$, sin embargo, $\Lambda$ no es una función multiplicativa.
\end{mydef}
Uno de los descubrimientos vitales de la teoría analítica de números es que los primos se <<llevan bien con la función $\log$>>,
lo cual tiene cierto sentido si pensamos que están definidos a partir de una propiedad del producto.

En general veremos que resolver el TNP de forma elemental es difícil, por lo que debemos cambiar ligeramente el problema,
y el siguiente teorema caracteriza la necesidad de las funciones de Chebyshev:
\begin{thm}
	Son equivalentes:
	\begin{enumerate}
		\item $ \pi(x) \sim \Li(x) $.
		\item $ \displaystyle\pi(x) \sim \frac{x}{\log x} $.
		\item $ \vartheta(x) \sim x $.
		\item $ \psi(x) \sim x $.
	\end{enumerate}
\end{thm}
\begin{proof}
	Ver que $1 \iff 2$ se reduce a notar que $\Li(x) \sim \frac{x}{\log x}$ por regla de L'Hôpital.

	Ahora definamos la función aritmética:
	$$ a(n) :=
	\begin{cases}
		1, & n\text{ primo} \\
		0, & n\text{ no primo}
	\end{cases} $$
	Nótese que $\sum_{n\le x} a(n) = \pi(x)$.
	Por fórmula de Abel se tiene que
	$$ \vartheta(x) = \sum_{n\le x} a(n)\log n = \pi(x)\log x - \int_{1}^{x} \frac{\pi(t)}{t} \, \ud t, $$
	y viceversa, si $b(n) := a(n)\log n$, nótese que $\sum_{n\le x} b(n) = \vartheta(x)$, por lo que
	\begin{equation}\label{eq:pi_abel_int}
		\pi(x) = \sum_{n\le x} b(n) \frac{1}{\log n} = \frac{\vartheta(x)}{\log x} - \int_{1}^{x} \frac{\vartheta(t)}{t \log^2 t} \, \ud t.
	\end{equation}
	Luego para probar que $2 \implies 3$ basta demostrar que
	$$ \lim_{x\to\infty} \frac{1}{x} \int_{1}^{x} \frac{\pi(t)}{t} \, \ud t = 0, $$
	para ello, nótese que 2 se traduce en que $\pi(t)/t = O(1/\log t)$ cuando $t \ge 2$, luego
	$$ \frac{1}{x} \int_{2}^{x} O\left( \frac{1}{\log t} \right) \, \ud t = O\left( \frac{1}{x} \int_{2}^{x} \frac{\ud t}{\log t} \right) $$
	Y nótese que
	$$ \int_{2}^{x} \frac{\ud t}{\log t} \le \int_{2}^{\sqrt{x}} \frac{\ud t}{\log t} + \int_{\sqrt{x}}^{x} \frac{\ud t}{\log t}
	\le \frac{\sqrt{x} - 2}{\log 2} + \frac{x - \sqrt{x}}{\log( \sqrt{x} )}. $$
	Luego al multiplicar por $1/x$ se comprueba que
	$$ \lim_{x\to\infty} \frac{1}{x} \int_{2}^{x} \frac{\ud t}{\log t} = 0. $$

	Para probar que $3 \implies 2$, notamos que mirando la fórmula \eqref{eq:pi_abel_int} y multiplicando por $\frac{\log x}{x}$ se nota
	que basta demostrar que
	$$ \lim_{x\to\infty} \frac{\log x}{x} \int_{1}^{x} \frac{\vartheta(t)}{t \log^2 t} \, \ud t. $$
	Igual que antes, por 3, sea $\vartheta(t) = O(t)$, luego
	$$ \frac{\log x}{x} \int_{1}^{x} \frac{O(t)}{t \log^2 t} \, \ud t = O\left( \frac{\log x}{x} \int_{1}^{x} \frac{1}{\log^2 t} \, \ud t \right). $$
	Y nótese que
	$$ \int_{2}^{x} \frac{\ud t}{\log^2 t} \le \int_{2}^{\sqrt{x}} \frac{\ud t}{\log^2 t} + \int_{\sqrt{x}}^{x} \frac{\ud t}{\log^2 t}
	\le \frac{\sqrt{x} - 2}{\log^2 2} + \frac{x - \sqrt{x}}{\log^2( \sqrt{x} )}. $$
	Luego se concluye análogamente.

	Veamos que $3 \iff 4$.
	Así pues, comencemos por notar que
	\begin{align}
		\psi(x) &= \sum_{p^k \le x} \log p = \sum_{k\le x}\sum_{p \le x^{1/k}} \log p = \sum_{k\le x} \vartheta(x^{1/k}) \notag \\
		&= \vartheta(x) + \vartheta(x^{1/2}) + \vartheta(x^{1/3}) + \cdots \label{eq:psi_theta_formula}
	\end{align}
	nótese que la suma es finita, dado que $\vartheta(x^{1/m}) = 0$ si $x^{1/m} < 2$ para $m > \log_2 x$.
	% \todo{Completar demostración.}
	Luego
	$$ 0 \le \psi(x) - \vartheta(x) = \sum_{m \le \log_2 x} \vartheta(x^{1/m}). $$
	Además, notemos que
	$$ \vartheta(x) \le \sum_{n \le x} \log n \le x\log x. $$
	Por lo que
	\begin{align*}
		0 \le \psi(x) - \vartheta(x) &\le \sum_{m \le \log_2 x} x^{1/m} \log(x^{1/m}) \le \log_2 x x^{1/2} \log(x^{1/2}) \\
		&= \frac{\log x}{\log 2} \frac{\sqrt{x}}{2} \log x = \frac{\sqrt{x} (\log x)^2}{2\log 2}.
	\end{align*}
	Finalmente, dividiendo por $x$ se puede comprobar que el límite converge a 0.
	Es decir, hemos probado que $ \psi(x)/x \sim \vartheta(x)/x $.
\end{proof}

Uno de los incisos del teorema~\ref{thm:big-oh-calculus} se se traduce en que
\begin{equation}\label{eq:log_sum_approx}
	T(x) = x\log x - x + O(\log x).
\end{equation}

\begin{thm}[Shapiro]
	Sea $\{a(n)\}_{n=1}^\infty$ una sucesión de reales positivos tales que
	$$ \sum_{n \le x} a(n) \sfloor{\frac{x}{n}} = x\log x + O(x). $$
	Entonces:
	\begin{enumerate}
		\item Para $x\ge 1$ se cumple que
			$$ \sum_{n \le x} \frac{a(n)}{n} = \log x + O(1). $$
		\item Existen $0 < c_1 < c_2$ tales que para $x$ suficientemente grande
			$$ c_1x \le \sum_{n \le x} a(n) \le c_2x. $$
	\end{enumerate}
\end{thm}
\begin{proof}
	Definamos
	$$ S_1(x) := \sum_{n \le x} a(n), \quad S_2(x) := \sum_{n\le x} a(n)\sfloor{\frac{x}{n}}. $$
	% \underline{Probemos que $S_1 - S_2(x/2) \le S_2(x) - 2S_2(x/2)$:}
	Primero notemos que
	\begin{align*}
		S_2(x) - 2S_2\left( \frac{x}{2} \right) &\ge \sum_{n\le x} a(n)\sfloor{\frac{x}{n}} - 2\sum_{n\le x/2} a(n)\sfloor{\frac{x}{2n}} \\
		&\ge \sum_{n\le x} \underbrace{ \left( \sfloor{\frac{x}{n}} - 2\sfloor{\frac{x}{2n}} \right) }_{\ge 0} a(n)
		+ \sum_{x/2 < n \le x} a(n)\sfloor{\frac{x}{n}} \\
		&\ge \sum_{x/2 < n \le x} a(n)\sfloor{\frac{x}{n}} \ge \sum_{x/2 < n \le x} a(n) = S_1(x) - S_1(x/2).
	\end{align*}
	Luego, por construcción se tiene que
	$$ S_2(x) - 2S_2\left( \frac{x}{2} \right) = x\log x + O(x) - 2\left( \frac{x}{2}\log\left( \frac{x}{2} \right) + O(x) \right) = O(x). $$
	En consecuencia, se tiene que
	$$ S_1(x) - S_1\left( \frac{x}{2} \right) \le Kx $$
	luego, sustituyendo $x$ por $x/2^j$ se obtiene que
	$$ S_1\left( \frac{x}{2^j} \right) - S_1\left( \frac{x}{2^{j+1}} \right) \le Kx \frac{1}{2^j} $$
	y por ende
	$$ S_1(x) \le Kx\left( 1 + \frac{1}{2} + \frac{1}{4} + \cdots \right) = 2Kx $$
	por lo que $c_2 := 2K$ aplica.

	Para el inciso 1., basta notar que $\sfloor{x/n} = x/n + O(1)$ para notar que
	\begin{align*}
		S_2(x) &= \sum_{n\le x} a(n) \left( \frac{x}{n} + O(1) \right) = x\sum_{n\le x} \frac{a(n)}{n} + O\left( \sum_{n\le x} a(n) \right) \\
		&= x\sum_{n\le x} \frac{a(n)}{n} + O(x)
	\end{align*}
	donde la última igualdad sale de la conclusión anterior.
	Luego
	$$ \sum_{n\le x} \frac{a(n)}{n} = \frac{1}{x}( S_2(x) + O(x) ) = \log x + O(1). $$

	Para completar la última desigualdad definamos
	$$ A(x) := \sum_{n\le x} \frac{a(n)}{n} = \log x + R(x) $$
	donde $R(x)$ es el término error y sabemos que $|R(x)| \le M$ para algún $M > 0$, por ser de orden $O(1)$.

	Para todo $x\ge 1$ y todo $\alpha$ tal que $\alpha x\ge 1$ se satisface que
	\begin{align*}
		A(x) - A(\alpha x) &= \log x + R(x) - \big( \log(\alpha x) + R(\alpha x) \big) = -\log\alpha + R(x) - R(\alpha x) \\
		&\ge -\log\alpha - |R(x)| - |R(\alpha x)| \ge -\log\alpha - 2M.
	\end{align*}
	Luego definamos $\alpha$ tal que $-\log\alpha - 2M = 1$, es decir, $\alpha = \exp(-1 - 2M)$ y por tanto $\alpha \in (0, 1)$.
	Es decir $A(x) - A(\alpha x) \ge 1$ para todo $x\ge 1/\alpha$.
	Pero
	$$ A(x) - A(\alpha x) = \sum_{\alpha x < n\le x} \frac{a(n)}{n} \le \frac{1}{\alpha x}\sum_{n\le x} a(n) = \frac{S_1(x)}{\alpha x}. $$
	Así pues
	$$ S_1(x) \ge \alpha x $$
	para $x\ge 1/\alpha$, que completa el inciso 2.
\end{proof}

\begin{cor}
	Se cumplen las siguientes:
	\begin{enumerate}
		\item $ (\Lambda * 1)(n) = \log n $.
		\item $\displaystyle \sum_{n\le x} \Lambda(n)\sfloor{ \frac{x}{n} } = \log x + O(1) $.
		\item $ c_1x \le \psi(x) \le c_2 x $.
			En consecuente, $\psi(x)/x = O(1)$.
	\end{enumerate}
\end{cor}
\begin{proof}
	La primera es trivial y la tercera se demuestra de la segunda, que es la que probaremos:
	Nótese que
	\begin{align}
		\sum_{n\le x} \Lambda(n)\sfloor{ \frac{x}{n} } &= \sum_{d\le x} \Lambda(d) \sum_{m \le x/d} 1 = \sum_{d\le x} (\Lambda * 1)(d) \notag \\
		&= \sum_{d\le x} \log d = T(x) \label{eq:lambda_sum_T} \\
		&= x\log x - x + O(\log x) = x\log x + O(x). \tqedhere
	\end{align}
\end{proof}

\begin{thmi}[Primer teorema de Mertens]\index{teorema!de Mertens!I}
	$$ \sum_{p \le x} \frac{\log p}{p} = \log x + O(1). $$
\end{thmi}
\begin{proof}
	Primero nótese que
	% thm:arithmetic_sum_formula
	\begin{align}
		\sum_{n \le x} \frac{\Lambda(n)}{n} &= \sum_{n\le x} \Lambda(n) \frac{1}{x}\left( \sfloor{ \frac{x}{n} } + O(1) \right) \notag \\
		&= \frac{1}{x} \sum_{n \le x} \Lambda(n)\sfloor{ \frac{x}{n} } + O\left( \frac{1}{x}\sum_{n\le x} \Lambda(n) \right) \notag \\
		&= \frac{1}{x} ( x\log x + O(1) ) + O\left( \frac{\psi(x)}{x} \right) = \log x + O(1). \label{eq:mertens_i_var}
	\end{align}
	% donde empleamos que $T(x) = \sum_{n\le x} \log n = x\log x - x + O(\log x)$.

	Ahora hay que ver que la diferencia entre ésta suma y el enunciado está acotada, para ello nótese que
	\begin{align*}
		\sum_{n\le x} \frac{\Lambda(n)}{n}          &= \sum_{p\le x} \frac{\log p}{p} + \sum_{p^{k+1} \le x} \frac{\log p}{p^{k+1}}, \\
		\intertext{y que}
		\sum_{p^{k+1} \le x} \frac{\log p}{p^{k+1}} &= \sum_{p\le\sqrt{x}} \log p \sum_{m=2}^{ \sfloor{\log x/\log p} } \frac{1}{p^m} \\
		                                            &\le \sum_{p\le\sqrt{x}} \frac{\log p}{p^2 - p} \le \sum_{p\le\sqrt{x}} \frac{\log p}{p^2} \le \sum_{n=1}^\infty \frac{\log n}{n^2},
	\end{align*}
	sin embargo, el último término es una serie convergente, así se concluye el enunciado.
\end{proof}

\begin{thmi}[Segundo teorema de Mertens]\index{teorema!de Mertens!II}
	$$ \sum_{p\le x} \frac{1}{p} = \log\log p + M + O\left( \frac{1}{\log x} \right), $$
	donde $M$ es una constante, llamada la \strong{constante de Mertens}.
\end{thmi}
\begin{figure}[!hbt]
	\centering
	\includegraphics[scale=1]{mertens.pdf}
	\caption{Segundo teorema de Mertens.}%
	\label{fig:mertens}
\end{figure}
\begin{proof}
	Definamos las siguientes:
	$$ a(n) :=
	\begin{cases}
		1, & n \text{ primo} \\
		0, & n \text{ no primo}
	\end{cases}
	\quad A(n) := \sum_{n\le x} a(n)\frac{\log n}{n} = \sum_{p\le x} \frac{\log p}{p}. $$
	Así pues, empleando $f(t) = 1/\log t$ en la fórmula de Abel nos queda que
	$$ \sum_{n\le x} \frac{1}{p} = \frac{A(x)}{\log x} + \int_{2}^{x} \frac{A(t)}{t \log^2 t} \, \ud t. $$
	Como $A(x) = \log x + R(x)$, donde $R(x) = O(1)$, por el primer teorema de Mertens, se obtiene que
	\begin{align*}
		\sum_{n\le x} \frac{1}{p} &= 1 + O\left( \frac{1}{\log x} \right) + \int_{2}^{x} \frac{\log t + R(t)}{t \log^2 t} \, \ud t \\
		&= 1 + O\left( \frac{1}{\log x} \right) + \int_{2}^{x} \frac{1}{t\log t} \, \ud t + \int_{2}^{x} \frac{R(t)}{t\log^2 t} \, \ud t.
	\end{align*}
	Sin embargo, nótese que
	$$ \frac{\ud}{\ud t}\log\log t = \frac{1}{t\log t} $$
	y de que
	$$ \int_{2}^{x} \frac{R(t)}{t\log^2 t} \, \ud t = \int_{2}^{\infty} \frac{R(t)}{t\log^2 t} \, \ud t - \int_{x}^{\infty} \frac{R(t)}{t\log^2 t} \, \ud t. $$
	El primer término es una constante, y el segundo
	$$ 0\le \int_{x}^{\infty} \frac{R(x)}{x\log^2 x} \, \ud t \le \int_{x}^{\infty} \frac{K}{x\log^2 x} \, \ud t = -\frac{K}{\log x}
	= O\left( \frac{1}{\log x} \right) $$
	Reuniendo todo se obtiene que
	\begin{equation}
		\sum_{n\le x} \frac{1}{p} = \log\log x
		+ \underbrace{ 1 -\log\log 2 + \int_{2}^{\infty} \frac{R(t)}{t\log^2 t} \, \ud t }_{M} + O\left( \frac{1}{\log x} \right). \tqedhere
	\end{equation}
\end{proof}

\addtocounter{thmi}{1}
\begin{slem}
	Para $x \ge 2$ se cumple que
	$$ \sum_{p\le x} \frac{1}{p} = \sum_{p\le x} \log\left( \frac{1}{1 - 1/p} \right) - c_0 + \frac{\theta}{2(x - 1)}, $$
	donde
	\begin{equation}
		C = \sum_{p} \left( \log\left( \frac{1}{1 - 1/p} \right) - \frac{1}{p} \right)
		\label{eqn:mertens_2.5}
	\end{equation}
	y $\theta = \theta(x) \in (0, 1)$.
\end{slem}
\begin{proof}
	Por definición de $c_0$ vemos que
	\begin{align*}
		0 < \theta(x) &= 2(x-1) \sum_{p > x} \left( \log\left( \frac{1}{1 - 1/p} \right) - \frac{1}{p} \right) \\
		              &= 2(x-1) \sum_{p > x} \sum_{r=2}^{\infty} \frac{1}{r p^r} < \sum_{p > x} \frac{2(x-1)}{2p(p-1)} \\
			      &< \sum_{n>x} \frac{(x-1)}{n(n-1)} = \frac{x - 1}{\lfloor x \rfloor} \le 1.
			      \tqedhere
	\end{align*}
\end{proof}
% \begin{slem}
% 	Para $x \ge 2$ se cumple que
% 	\[
% 		\prod_{p\le x} \left( 1 - \frac{1}{p} \right) = \frac{\exp(-M-C)}{\log x}\left( 1 + O\left( \frac{1}{\log x} \right) \right),
% 	\]
% 	donde $C$ y $M$ son las constantes dadas por los teoremas anteriores.
% \end{slem}
% \begin{proof}
% 	Basta igualar el segundo teorema de Mertens con el lema anterior y despejamos la expresión,
% 	aplicando logaritmos a ambos lados.
% \end{proof}
\addtocounter{thmi}{-1}

\begin{thmi}[Tercer teorema de Mertens]\index{teorema!de Mertens!III}
	Para $x \ge 2$ se cumple que
	\[
		\prod_{p\le x} \left( 1 - \frac{1}{p} \right) = \frac{e^{-\gamma}}{\log x}\left( 1 + O\left( \frac{1}{\log x} \right) \right),
	\]
	donde $\gamma$ es la constante de Euler-Mascheroni.
\end{thmi}
\begin{proof}
	% Nótese que, para $s > 1$ real se cumple que
	% \[
	% 	\sum_{n \le x} \frac{1}{n^{1+s}} \le \prod_{p\le x} \left( 1 - \frac{1}{p^{1+s}} \right)^{-1} \le \zeta(1 + s),
	% \]
	% esto es puesto que el producto sobre todos los primos converge a $\zeta(1+s)$, pero los productos parciales son decrecientes.
	Considere la función
	\[
		f(s) := \log\zeta(1 + s) - \sum_{p} \frac{1}{p^{1+s}} = \sum_{p} \left( \log\left( \frac{1}{1 - 1/p^{1+s}} \right) - \frac{1}{p^{1+s}} \right) ),
	\]
	la cual converge uniformemente para $s \ge 0$ ya que está acotado por
	\[
		\sum_{n=2}^{\infty} \sum_{j=2}^{\infty} \frac{1}{n^j} = \sum_{n=2}^{\infty} \frac{1}{n(n-1)} \le \zeta(2).
	\]
	Así que $f$ es continua (por la derecha al menos) en $s = 0$ y $\lim_{s \to 0^+} f(s) = f(0) = C$ dado en \eqref{eqn:mertens_2.5}.
	Ahora bien, para $s > 1$ real, se cumple que
	\[
		\zeta(1 + s) = \frac{1}{s} + O(1),
	\]
	ya que se sigue de la comparación con la integral, así que
	\begin{align*}
		\log\zeta(1 + s) &= \log\left( \frac{1}{s} + O(1) \right) = \log(1/s) + O(s) = \log\left( \frac{1}{1 - e^{-s}} \right) + O(s) \\
				 &= \sum_{n=1}^{\infty} \frac{e^{-sn}}{n} + O(s) = \int_{0}^{\infty} e^{-st} \, \ud H(t) + O(s),
	\end{align*}
	donde $H(t) := \sum_{1\le n\le t} \frac{1}{n}$.
	Así, por integración por partes, tenemos que
	\[
		\log\zeta(1+s) = s \int_{1}^{\infty} e^{-st}H(t) \, \ud t + O(s).
	\]
	Sea $P(t) := \sum_{p\le t} 1/p$, entonces
	\[
		\sum_{p} \frac{1}{p^{1+s}} = \int_{1}^{\infty} \frac{1}{t^s} \, \ud P(t) = s \int_{1}^{\infty} \frac{P(t)}{t^{1+s}} \, \ud t
		= s \int_{0}^{\infty} e^{-st} P(e^t) \, \ud t.
	\]
	Agrupando los términos, obtenemos que
	\[
		f(s) = s \int_{0}^{\infty} e^{-st}\big( H(t) - P(e^t) \big) \, \ud t + O(s).
	\]
	Finalmente, ya hemos calculado que
	\[
		H(t) = \log t + \gamma + O(1/t), \qquad P(t) = \log t + M + O(1/t).
	\]
	Por lo que, para $0 < s \le 1/2$ vemos que
	\begin{align*}
		f(s) &= s \int_{0}^{\infty} \left( \gamma - M + O\left( \frac{1}{t+1} \right) \right)e^{-st} \, \ud t + O(s) \\
		     &= \gamma - M + O\left( s + s \int_{0}^{\infty} \frac{e^{-st}}{t + 1} \, \ud t \right) = \gamma - M + O(s\log(1/s)),
	\end{align*}
	así que, con $s \to 0$, concluimos que $C = \gamma - M$.

	Para concluir, tomamos logaritmos a ambos lados y concluimos por el lema anterior y el segundo teorema de Mertens.
\end{proof}

% De momento no podemos probar el TNP, pero las funciones de Chebyshev nos otorgarán ciertas equivalencias más sencillas.
% Primero probaremos un resultado conocido como el postulado de Bertrand, para ello véase primero:

\newpage
\subsection{El postulado de Bertrand}
\begin{thmi}[Postulado de Bertrand]\index{postulado!de Bertrand}
	Para todo $n \ge 2$ existe un primo $p$ tal que $n < p \le 2n$.
\end{thmi}
\begin{proof}
	Comenzamos con la siguiente fórmula:
	\begin{align*}
		\sum_{k\ge 1} \psi\left( \frac{x}{k} \right) &= \sum_{k\ge 1} \sum_{n \le x/k} \Lambda(n) = \sum_{\substack{ kn\le x \\ k\ge 1 }} \Lambda(n) \\
		&= \sum_{n\le x} \sum_{k\le x/n} \Lambda(n) = \sum_{n\le x} \Lambda(n) \sfloor{ \frac{x}{n} } = T(x)
	\end{align*}
	donde la última igualdad corresponde a la ecuación~\eqref{eq:lambda_sum_T}.
	De modo que
	$$ T(x) = \psi(x) + \psi\left( \frac{x}{2} \right) + \psi\left( \frac{x}{3} \right) + \cdots $$
	Luego se obtiene que
	$$ T(x) - 2T\left( \frac{x}{2} \right) = \psi(x) - \psi\left( \frac{x}{2} \right) + \psi\left( \frac{x}{3} \right) - \cdots $$
	Ahora, la fórmula \eqref{eq:log_sum_approx} nos permite notar que
	\begin{align*}
		T(x) - 2T(x/2) &= x\log x - x - 2\left( \frac{x}{2}\log \frac{x}{2} - \frac{x}{2} \right) + O(\log x) \\
		&= x\log 2 - \frac{x}{2} + O(\log x)
	\end{align*}
	Y como $\psi$ es monótona se tiene que
	$$ \psi(x) - \psi\left( \frac{x}{2} \right) \le T(x) - 2T\left( \frac{x}{2} \right) \le \psi(x) $$
	En definitiva
	$$ \psi(x) - \psi(x/2) \le x\log 2 - \frac{x}{2} + O(\log x) \le \psi(x). $$
	Por otro lado:
	\begin{align*}
		\psi(x) &= \big( \psi(x) - \psi(x/2) \big) + \big( \psi(x/2) - \psi(x/4) \big) + \cdots \\
		&\le \left( 1 + \frac{1}{2} + \frac{1}{4} + \cdots \right) \big( x\log 2 - x/2 + O(\log x) \big) \\
		&= x\log 4 - x + O(\log x).
	\end{align*}
	Luego como $\psi(x/2) \ge x\log(2^{1/2}) - x/4 + O(\log x)$, entonces
	$$ \psi(x) - \psi(x/2) \ge \frac{3x}{2}\big( \log(2) - 1 \big) + O(\log x) $$
\end{proof}
% Ésta subsección sigue la demostración original de Ramanujan.

Ahora incluimos la demostración de Ramanujan mediante la modificación de Erd\H os.
\begin{thm}\label{thm:prime_prod_ineq}
	Para todo $n\ge 2$ se cumple que
	$$ \prod_{p\le n} p \le 4^n \iff \vartheta(x) \le 2x\log 2. $$
\end{thm}
\begin{proof}
	Ésto se realiza por inducción:
	Primero, podemos notar que la relación se cumple trivialmente para $n=2$, así que podremos asumir $n$ más grande.

	Si $n$ es par, entonces no es primo, luego
	$$ \prod_{p\le n} p = \prod_{p\le n-1} p \le 4^{n-1} \le 4^n. $$
	Si $n = 2m+1$, entonces, considere primero al coeficiente binomial:
	$$ \binom{2m+1}{m} = \frac{(2m+1)!}{m!(m+1)!} = \binom{2m+1}{m+1} $$
	y nótese que todos los primos $>m+1$ no están en el denominador, de modo que
	$$ \prod_{m+1 < p \le n} p \mid \binom{2m+1}{m} = \frac{1}{2}\left( \binom{2m+1}{m} + \binom{2m+1}{m+1} \right) \le \frac{1}{2}(1 + 1)^{2m+1} = 4^m. $$
	Luego, por hipótesis inductiva se concluye que
	\begin{equation}
		\prod_{p \le n} p = \left( \prod_{p \le m+1} p \right) \cdot \left( \prod_{m+1<p\le n}p \right) \le 4^{m+1} \cdot 4^m = 4^n. \tqedhere
	\end{equation}
\end{proof}

\begin{lem}
	Para todo $n \ge 1$ tenemos que
	$$ \binom{n}{\lfloor n/2 \rfloor} \ge \frac{2^n}{n}, \qquad \binom{2n}{n} \ge \frac{4^n}{2n}. $$
\end{lem}
\begin{proof}
	En efecto, basta notar que, por el binomio de Newton
	$$ 2^n = (1 + 1)^n = \sum_{j=0}^{n} \binom{n}{j}, $$
	luego los coeficientes binomiales suman $2^n$ y agrupando $\binom{n}{0} + \binom{n}{n}, \binom{n}{1}, \dots, \binom{n}{n-1}$
	tenemos $n$ sumandos.
	Finalmente, basta notar que $\binom{2n}{\lfloor n/2 \rfloor}$ es el máximo valor en la lista (esto se puede comprobar analizando el triángulo
	de Pascal-Tartaglia).
\end{proof}

\begin{Proof}{Erd\H os-Ramanujan}
	En primer lugar, nótese que el coeficiente binomial $\binom{2n}{n} = \frac{(2n)!}{n! n!}$ incluye al primo $p$ exactamente
	$$ \sum_{k \ge 1} \left( \left\lfloor \frac{2n}{p^k} \right\rfloor - 2 \left\lfloor \frac{n}{p^k} \right\rfloor \right). $$
	Nótese que cada sumando es $\le 1$ pues es entero y
	$$ \left( \left\lfloor \frac{2n}{p^k} \right\rfloor - 2 \left\lfloor \frac{n}{p^k} \right\rfloor \right)
	< \frac{2n}{p^k} - 2\left( \frac{n}{p^k} - 1 \right) = 2. $$
	Además, cada sumando es cero cuando $p^k > 2n$, de modo que tenemos
	\begin{equation}
		\sum_{k \ge 1} \left( \left\lfloor \frac{2n}{p^k} \right\rfloor - 2 \left\lfloor \frac{n}{p^k} \right\rfloor \right) \le \max\{ r : p^r \le 2n \},
		\label{eq:bertrand_max}
	\end{equation}
	en consecuencia, los primos $p > \sqrt{2n}$ aparecen a lo más una vez.

	Además, para todo primo $p$ entre $\frac{2}{3}n < p \le n$ tenemos que $p \nmid \binom{2n}{n}$:
	en efecto, como $3p > 2n$ entonces los múltiplos de $p$ menores que $2n$ son $p$ y $2p$, ambos de los cuales aparecen en el denominador de $\binom{2n}{n}$.
	Empleando toda esta información podemos sacar la siguiente cota:
	$$ \frac{4^n}{2n} \le \binom{2n}{n} \le \prod_{p \le \sqrt{2n}} 2n \cdot \prod_{\sqrt{2n} < p \le \frac{2}{3}n} p \cdot \prod_{n < p \le 2n} p. $$
	El primer producto no tiene más de $\sqrt{2n}$ términos y, denotando $P(n) := \sum_{n < p \le 2n} 1$, tenemos:
	$$ \frac{4^n}{2n} < (2n)^{\sqrt{2n}} \cdot 4^{2/3 n} \cdot (2n)^{P(n)} \iff 4^{n/3} < (2n)^{\sqrt{2n} + 1 + P(n)}. $$
	Aplicando $\log_2$ a ambos lados tenemos:
	$$ P(n) > \frac{2n}{3\log_2(2n)} - (\sqrt{2n} + 1). $$
	Basta ver que la función de la derecha es positiva para un $n$ suficientemente grande.
	Así
	\begin{align*}
		                      \frac{2n}{3\log_2(2n)} &> \sqrt{2n} + 1 \\
		\iff (\sqrt{2n} - 1)(\sqrt{2n} + 1) = 2n - 1 &> 3\log_2(2n)(\sqrt{2n} + 1) - 1,
	\end{align*}
	por lo que basta probar que $3\log_2(2n) < \sqrt{2n} - 1$.
	Para $n = 2^9$ tenemos $3\log_2(2n) = 30 < 31 = \sqrt{2n} - 1$ y la función $\sqrt{2n} - 1 - 3\log_2(2n)$ es creciente
	(en un intervalo apropiado) pues su derivada es
	$$ ( \sqrt{x} - 1 - 3\log_2(x) )' = \frac{1}{2\sqrt{x}} - \frac{3}{x\log 2}, $$
	y $6\sqrt{x} < x\log 2$ para $x > \left( 6/\log 2 \right)^2 \approx 75$.

	Así pues, vemos que el postulado de Bertrand vale para $n \ge 2^9 = 512$ y para $n < 512$ también por la siguiente lista de primos:
	\begin{equation}
		3, \; 5, \; 7, \; 13, \; 23, \; 43, \; 83, \; 163, \; 317, \; 631.
		\tqedhere
	\end{equation}
\end{Proof}

Un problema abierto, con el mismo espíritu del postulado de Bertrand pero severamente más estricto es el siguiente:
\begin{con}
	Para todo $n \ge 1$ existe un primo entre $n^2$ y $(n + 1)^2$.
\end{con}
El lector podrá notar que a medida que $n$ es grande, el postulado de Bertrand se puede traducir en que hay un primo a menos de $n$ unidades;
y la conjetura anterior a que hay uno a menos de $2\sqrt{n} + 1$ unidades.
En teoría de números hay matemáticos trabajando en un problema relativamente opuesto: ¿qué tan frecuentemente se dan primos a distancias cortas?
En su forma más extrema está la siguiente conjetura:
\begin{con}[de los primos gemelos]
	Hay infinitos pares de primos de la forma $\{ p, p+2 \}$.
\end{con}

% \begin{lem}
% 	Se cumplen las siguientes:
% 	\begin{align}
% 		\psi(x) &= \vartheta(x) + \vartheta(x^{1/2}) + \vartheta(x^{1/3}) + \cdots \\
% 		T(x) &= \psi(x) + \psi\left( \frac{x}{2} \right) + \psi\left( \frac{x}{3} \right) + \cdots \\
% 		\psi(x) - 2\psi(x^{1/2}) &= \vartheta(x) - \vartheta(x^{1/2}) + \vartheta(x^{1/3}) - \cdots \\
% 		T(x) - 2T\left( \frac{x}{2} \right) &= \psi(x) - \psi\left( \frac{x}{2} \right) + \psi\left( \frac{x}{3} \right) - \cdots \\
% 		\psi(x) - 2\psi(x^{1/2}) &\le \vartheta(x) \le \psi(x) \\
% 		\psi(x) - \psi\left( \frac{x}{2} \right) &\le T(x) - 2T\left( \frac{x}{2} \right)
% 		\le \psi(x) - \psi\left( \frac{x}{2} \right) + \psi\left( \frac{x}{3} \right)
% 	\end{align}
% \end{lem}
% \begin{proof}
% 	En todo momento alentamos al lector a intentar hacer las deducciones.
% 	Así pues, nótese que
% 	\begin{align*}
% 		\psi(x) &= \sum_{p^k \le x} \log p = \sum_{k\le x}\sum_{p \le x^{1/k}} \log p = \sum_{k\le x} \vartheta(x^{1/k}) \\
% 		&= \vartheta(x) + \vartheta(x^{1/2}) + \vartheta(x^{1/3}) + \cdots
% 	\end{align*}

% 	Para la segunda se tiene que
% 	\begin{align*}
% 		\sum_{k\ge 1} \psi\left( \frac{x}{k} \right) &= \sum_{k\ge 1} \sum_{n \le x/k} \Lambda(n) = \sum_{\substack{ kn\le x \\ k\ge 1 }} \Lambda(n) \\
% 		&= \sum_{n\le x} \sum_{k\le x/n} \Lambda(n) = \sum_{n\le x} \Lambda(n) \sfloor{ \frac{x}{n} } = T(x)
% 	\end{align*}
% 	donde la última igualdad corresponde a la ecuación~\eqref{eq:lambda_sum_T}.

% 	La tercera y cuarta son mera manipulación algebraica de las dos primeras.

% 	Por la tercera fórmula se tiene que
% 	$$ \psi(x) - 2\psi(x^{1/2}) = \vartheta(x) {\color{nicered} {}- \vartheta(x^{1/2}) + \vartheta(x^{1/3}) - \cdots } \le \vartheta(x) $$
% 	donde el término rojo es negativo y donde claramente $\vartheta(x) \le \psi(x)$ por definición.

% 	La sexta fórmula es parecida a la anterior.
% \end{proof}


% \begin{thmi}[Postulado de Bertrand]\index{postulado!de Bertrand}
% 	Para todo número natural $n\ge 2$ se cumple que existe un número primo $p$ tal que $n < p\le 2n$. 
% \end{thmi}
% \begin{proof}
% 	El enunciado fue inicialmente una observación de Bertrand que se cumple para $n$ pequeño, pero para ser más concisos
% 	la siguiente lista llena casos pequeños:
% 	$$ 3, \; 5, \; 7, \; 13, \; 23, \; 43, \; 83, \; 163, \; 317, \; 631. $$
% 	Así que podemos asumir que $n \ge 631$.
% 	Nótese que probar el enunciado es equivalente a probar que $\vartheta(x) - \vartheta(x/2)$ es positiva.
% 	\todo{Completar demostración.}
% 	% Nótese que $n!$ admitirá una descomposición como producto de primos con potencias, ¿así que cuál es la potencia de $p$ en $n!$?
% 	% Primero, claramente todos los múltiplos (y hay $\sfloor{n/p}$) de $p$ menores que $n$ agregan 1, y los múltiplos de $p^2$ agregan 1 más, así que
% 	% $$ \log(n!) = T(n) = \sum_{p<n} \left( \sum_{k\ge 1} \sfloor{ \frac{n}{p^k} } \right) \cdot \log p $$
% \end{proof}

Finalmente incluimos el siguiente resultado vinculado al postulado de Bertrand.
Si pensamos los números en base 2, entonces el postulado de Bertrand se traduce en que existe un primo entre $(d_md_{m-1}\dots d_1)_2$
y $(d_m\dots d_1 0)_2$; más en particular, si elegimos $d_m = 1$ y $d_j = 0$ para $j \ne m$, esto corresponde a encontrar un primo con $m$ dígitos en base 2.
Así pues, llegamos al siguiente problema:
\begin{thm}
	Sea $k$ un cuerpo finito.
	Para todo natural $n \ge 1$ existe un polinomio $f(x) \in k[x]$ irreducible (equivalentemente primo) de grado $n$.
\end{thm}
\begin{proof}
	Sea $q := |k|$ la cardinalidad del cuerpo.
	Para todo $n$, denotamos por $F_n(x)$ al producto de todos los polinomios mónicos de grado $n$ en $k[x]$.
	Nótese que hay $q^n$ polinomios mónicos de grado $n$, de modo que $\deg(F_n) = nq^n$.
	Si $g(x)$ es mónico de grado $n-1$, entonces $(x - a)g(x)$ con $a \in k$ corresponden a $q$ polinomios de grado $n$,
	por lo que, $F_{n-1}(x)^q \mid F_n(x)$ y así sea $Q_n(x) := F_n(x) / F_{n-1}(x)^q$.

	Sea $p(x)$ un polinomio mónico irreducible de grado $d$, y considere el máximo $r$ tal que $p^r \mid F_n$,
	lo que equivale al número de polinomios mónicos de grado $n$ divisibles por $p$,
	más el número de polinomios mónicos de grado $n$ divisibles por $p^2$, más los divisibles por $p^3$ y así.
	La cantidad de polinomios mónicos de grado $n$ divisbles por $p^e$ es $q^{n - de}$ si $n - de \ge 0$ y 0 de lo contrario;
	en forma cerrada queda $\lfloor q^{n - de} \rfloor$.
	En resumen:
	$$ r = \sum_{e=1}^{\infty} \lfloor q^{n - de} \rfloor. $$
	Aplicando ésta fórmula, podemos ver que la potencia de $p(x)$ que divide a $Q_n(x)$ es exactamente:
	$$ \sum_{e=1}^{\infty} (\lfloor q^{n - de} \rfloor - q \lfloor q^{n - 1 - de} \rfloor). $$
	Si $de \le n - 1$ o si $de \ge n + 1$, entonces $\lfloor q^{n - de} \rfloor - q \lfloor q^{n - 1 - de} \rfloor = 0$.
	Si $de = n$, o equivalentemente si $d \mid n$, entonces $\lfloor q^{n - de} \rfloor - q \lfloor q^{n - 1 - de} \rfloor = 1$;
	por lo tanto
	$$ \frac{F_n(x)}{F_{n-1}(x)^q} = Q_n(x) = \prod_{\deg P \mid n} P, $$
	donde el producto recorre polinomios mónicos irreducibles.

	Ahora bien, si $n = 1$, entonces claramente $x$ es un ejemplo de un polinomio irreducible de grado $n$.
	Si $n \ge 2$, entonces $\deg(Q_n) = nq^n - q((n-1)q^{n-1}) = q^n$.
	Sea $\pi(n, k)$ la cantidad de polinomios irreducibles mónicos de grado $n$, entonces lo anterior nos da que
	\begin{equation}
		\sum_{d\mid n} d\pi(d, k) = q^n.
		\label{eqn:gauss_relation_primes_fin_fld}
	\end{equation}
	Si $\pi(n, k) = 0$, entonces podemos hacer la suma sobre el resto de divisores de $n$ los cuales son $\le \lfloor n/2 \rfloor$:
	\begin{align*}
		q^n &\le \sum_{1 \le q \le \lfloor n/2 \rfloor} d\pi(d, k) \le \lfloor n/2 \rfloor \sum_{1 \le q \le \lfloor n/2 \rfloor} q^d
		= \lfloor n/2 \rfloor \cdot \frac{q^{\lfloor n/2 \rfloor+1} - q}{q - 1} \\
		    &< \lfloor n/2 \rfloor \cdot \frac{q}{q - 1} q^{\lfloor n/2 \rfloor} \le 2 \lfloor n/2 \rfloor q^{\lfloor n/2 \rfloor},
	\end{align*}
	pero esto es imposible puesto que $q^n \ge q^{\lfloor n/2 \rfloor}2^{\lfloor n/2 \rfloor} \ge 2 \lfloor n/2 \rfloor q^{\lfloor n/2 \rfloor}$
	para todo $\lfloor n/2 \rfloor \ge 1$ (vale decir, todo $n \ge 2$).
\end{proof}

Esta última demostración es original de \citeauthor{soundararajan2020bertrands}~\cite{soundararajan2020bertrands}.
La identidad \eqref{eqn:gauss_relation_primes_fin_fld} en la demostración se conoce como \strong{relación de Gauss}.
% \addtocategory{article}{soundararajan2020bertrands}

% \subsection{El teorema de Sylvester-Schur}
% En la demostración del postulado de Bertrand hubo un determinado análisis de divisibilidad sobre coeficientes binomiales.
% Un estudio más profundo lleva a consecunecias mucho más agudas, como veremos en ésta subsección.
% Seguimos una demostración de \citeauthor{erdos34sylvester}~\cite{erdos34sylvester}.
% \addtocategory{article}{erdos34sylvester}

% \begin{lem}
% 	Si $p^\alpha \mid \binom{n}{k}$, donde $p$ es primo, entonces $p^\alpha \le n$.
% \end{lem}
% \begin{prop}
% 	Son equivalentes:
% 	\begin{enumerate}
% 		\item Para todo $k \ge 1$ existe un entero $n_1(k)$ tal que

% 	\end{enumerate}
% 	Para $n \ge k \ge 1$ enteros, se tiene
% 	$$ \binom{n + k}{k} > (n + k)^{\pi(k)} $$
% \end{prop}

\subsection{El teorema de Sylvester-Schur}
Con un poco más de trabajo y estudio de los coeficientes binomiales, uno puede mejorar el postulado de Bertrand.
\begin{lem}
	Supongamos que, para $n \ge j \ge 1$, tenemos
	\begin{equation}
		\binom{n + j}{j} > (n + j)^{\pi(j)}.
		\label{eqn:binomial_growth}
	\end{equation}
	Entonces alguno de los enteros $n + 1, n + 2, \dots, n + j$ es divisible por un primo $p > j$.
	Además, si se satisface \eqref{eqn:binomial_growth} para $n = n_1(j)$, entonces se satisface para todo $n \ge n_1(j)$.
\end{lem}
\begin{proof}
	Si, por contradicción, todos los factores primos de $n + 1, \dots, n + j$ son $\le j$, entonces todos los factores primos de $\binom{n + j}{j}$
	son $\le j$ y, por \eqref{eq:bertrand_max}, vemos que si $p^r \mid \binom{n + j}{j}$ y $p^{r+1} \nmid \binom{n + j}{j}$, entonces
	$$ \binom{n + j}{j} \le \prod_{p \le j} (n + j) = (n + j)^{\pi(j)}, $$
	lo que contradice el enunciado.

	Por inducción, podemos probar que
	$$ \forall x\ge j\ge 1 \qquad \left( 1 + \frac{1}{x + j} \right)^j \le 1 + \frac{j}{x + j}. $$
	Así, probaremos que \eqref{eqn:binomial_growth} para $n \ge n_1(j)$ por inducción:
	\begin{align*}
		\binom{n + 1 + j}{j} &= \left( 1 + \frac{j}{n + 1} \right) \binom{n + j}{j} > \left( 1 + \frac{1}{n + j} \right)^j (n + j)^{\pi(j)} \\
				     &> (n + 1 + j)^{\pi(j)},
	\end{align*}
	que es lo que se quería probar.
\end{proof}

\begin{thm}[Sylvester-Schur]\index{teorema!de Sylvester-Schur}
	Dados enteros $n \ge j \ge 1$.
	Al menos uno de los enteros $n + 1, n + 2, \dots, n + j$ es divisible por un primo $p > j$.
\end{thm}
Nótese que con $n = j$ se obtiene el postulado de Bertrand usual.
\begin{proof}
	Por el lema anterior, basta demostrar que $n_1(j) = j$ en \eqref{eqn:binomial_growth}, lo cual es cierto para $202 \le j \le 1500$.
	También es fácil comprobar que $j \le n_1(j) \le j + 17$ para $j \le 201$; por lo que comprobamos mediante un ordenador que el enunciado
	es cierto con $j \le 201$ y $j \le n \le j + 17$.

	Para $j > 1500$ supongamos que \eqref{eqn:binomial_growth} no se cumple para $n_1(j) = j$.
	Ahora, es fácil comprobar que $\pi(j) < j/3$ por inducción desde un $j$ suficientemente grande y $\frac{n + j - i}{j - i} > \frac{n + j}{j}$
	para $0 \le i < j$, de modo que $\binom{n+j}{j} \ge \left( \frac{n+j}{j} \right)^j$, con lo que concluimos que
	$$ \left( \frac{n + j}{j} \right)^j \le \binom{n + j}{j} \le (n + j)^{\pi(j)} < (n + j)^{j/3}, $$
	cancelando el exponente $j$ y reordenando obtenemos que
	$$ n + j \le j^{3/2} \iff n \le j^{3/2} - j. $$
	Ahora bien, si $p > \sqrt{n + j}$ y $p^r \mid \binom{n + j}{j}$ (de modo que, $p^r \le n + j$), entonces $r \in \{ 0, 1 \}$.
	Así que
	$$ \binom{n + j}{j} \le \prod_{p \le \sqrt{n+j}} (n + j) \prod_{p \le j} p \le j^{ \frac{1}{3}j^{3/4} } 4^{j-1}, $$
	donde empleamos que $\pi( \sqrt{n+j} ) \le \frac{1}{3} (n+j)^{1/2} \le \frac{1}{3} j^{3/4}$ y el teorema~\ref{thm:prime_prod_ineq}.

	Por ello, de darse $n_1(j) \ge 3j$, entonces
	\todo{Justificar \citeauthor{granville:masterclass}~\cite[Ex.~4.14.1]{granville:masterclass}.}
	$$ \frac{(4^4/3^3)^j}{ej} \le \binom{4j}{j} \le \binom{n + j}{j} \le j^{ \frac{1}{3}j^{3/4} } 4^{j-1}, $$
	lo cual es falso para todo $j \ge 1$.
	Por ello, $n_1(j) + j \le 4j$.
	Si $n_1(j) + j > \frac{5}{2}j$, entonces
	$$ \frac{(5^5/3^3 2^2)^{j/2}}{ej} \le \binom{5j/2}{j} \le \binom{n+j}{j} \le (4j)^{j^{1/2}} 4^{j-1}, $$
	lo cual es absurdo para $j \ge 780$ (convierta lado derecho menos izquierdo en una función, estudie derivadas y haga el cálculo).

	
\end{proof}

\subsection{Cotas de Chebyshev}
Comenzaremos por definir recursivamente $d_1 = 1$ y
$$ d_{n+1} := \mcm(d_n, n+1). $$
\begin{thm}[Nair]
	Para $n \ge 7$ se satisface que $d_n \ge 2^n$.
\end{thm}
\begin{proof}
	Defínase la siguiente integral
	$$ I(m, n) := \int_{0}^{1} x^{m-1}(1 - x)^{n-m} \, \ud x, \qquad 1 \le m \le n. $$
	Por el teorema del binomio, podemos notar que la integral indefinida determina un polinomio con coeficientes racionales
	que son efectivamente computables, de modo que $I(m, n) \in \Q$ y su denominador divide a $d_n$.
	Más precisamente,
	$$ I(m, n) = \sum_{j=0}^{m-n} (-1)^j \binom{n-m}{j} \frac{1}{m + j} \in \frac{1}{d_j}\Z $$
	Por otro lado, el teorema del binomio también muestra que
	$$ \sum_{m=1}^{n} \binom{n-1}{m-1} y^{m-1} I(m, n) = \int_{0}^{1} (1 - x + xy)^{n-1} \, \ud x = \frac{1}{n} \sum_{m=1}^{n} y^{m-1}, $$
	como ambos extremos son polinomios en $y$ sus coeficientes coinciden e
	$$ I(m, n) = \frac{1}{n \binom{n-1}{m-1}} = \frac{1}{m \binom{n}{m}}. $$
	En consecuente, $m \binom{n}{m} \mid d_n$ y, en particular, con $m = n/2$ obtenemos que
	$$ n \binom{2n}{n} \mid d_{2n} \mid d_{2n+1}, \qquad (n+1) \binom{2n+1}{n} = (n+1) \binom{2n}{n} \mid d_{2n+1}. $$
	Como $n$ y $(2n+1)$ son coprimos, entonces
	$$ n(2n+1) \binom{2n}{n} \mid d_{2n+1}. $$
	Finalmente, $\binom{2n}{n}$ es el mayor coeficiente binomial en la expansión de $(1 + 1)^{2n} = 4^n$, por lo que $d_{2n+1} \ge n 4^n$
	(para $n \ge 1$).
	Se sigue que
	$$ d_{2n+1} \ge 2\cdot 4^n = 2^{2n+1}, \qquad d_{2n+2} \ge d_{2n+1} \ge 4^{n+1}, $$
	para $n \ge 4$.
	El resto de casos pueden evaluarse a mano:
	\begin{equation*}
		2^8 = 256 < 420 = d_7 = d_8, \qquad 2^{10} = 1\,024 < 1\,260 = d_9 = d_{10}.
		% 2^{13} = 8\,192 < 13\,860 = d_{11} = d_{12}, \qquad 2^{16} =  < d_{13}.
		\tqedhere
	\end{equation*}
\end{proof}

\begin{thm}[cotas de Chebyshev, 1852]
	Para todo $n \ge 4$ se cumple que
	$$ \log 2 \le \frac{\pi(n)}{n/\log n} \le \log 4 + \frac{8\log\log n}{\log n} \le 4.33. $$
	En particular, $\pi(x) \asymp x/\log x$.
\end{thm}
\begin{proof}
	La cota inferior sale del hecho de que
	$$ 2^n \le d_n = \prod_{p^r \le n} p^r \le \prod_{p \le n} n = n^{\pi(n)}, $$
	y, equivalentemente, $n\log 2 \le \pi(n)\log n$.
	% Nótese que, por \eqref{eq:bertrand_max}, tenemos:
	% $$ n^{\pi(2n) - \pi(n)} < \prod_{n < p \le 2n} \le \binom{2n}{n} \le \prod_{p^r \le 2n < p^{r+1}} p^r \le (2n)^{\pi(2n)} $$

	Es inmediato que, para todo $1 < t \le n$
	$$ t^{\pi(n) - \pi(t)} \le \prod_{t < p \le n} p \le 4^n, $$
	luego aplicando logaritmos obtenemos que
	$$ \pi(n) \le \frac{n\log 4}{\log t} + \pi(t) \le \frac{n\log 4}{\log t} + t, $$
	luego la cota superior del teorema se sigue de poner $t = n/(\log n)^2$.
	La última desigualdad se logra calculando que la función $\log\log x/\log x$ alcanza su máximo global en $x = e^e$
	y luego computando $\log 4 + 8/e \approx 4.32932989$.
\end{proof}
Hay cotas, relativamente más elementales, que se siguen principalmente del postulado de Bertrand.
Para ello, véase \S 5.7 de \citeauthor{hua:number}~\cite[79-85]{hua:number}.

% \begin{mydef}
% 	Sean
% 	$$ H(x) := \sum_{1\le n\le x} \frac{1}{n}, $$
% 	los \strong{números harmónicos}\index{numero@número!harmónico}.
% \end{mydef}
% Ya sabemos que $H(x) = \log x + \gamma + O(1/x)$ y, por tanto, $H(x) \sim \log x$.

% \begin{lem}
% 	Para todo entero $m > 0$ se tiene que $\frac{m}{2} \le H(2^m) \le m$.
% \end{lem}
% \begin{hint}
% 	Esto es parte de la demostración usual de que la serie harmónica diverge (cfr. \cite{Top}, ej.~1.54).
% 	\addtocategory{other}{Top}
% \end{hint}

% \begin{lem}
% 	Para todo entero $m > 0$ se tiene que $\pi(2^{m+1}) \le 2^m$.
% \end{lem}
% \begin{proof}
% 	Para $x \ge 3$ entonces $\pi(x) - 1 \le \frac{x - 2}{2}$ porque en el conjunto $\{ 3, 4, \dots, \lfloor x \rfloor \}$
% 	la mitad de números son pares.
% 	Para el resto funciona por inspección.
% \end{proof}

% \begin{thm}
% 	$\displaystyle \frac{1}{8} \le \frac{\pi(x)}{x/H(x)} \le 6. $
% \end{thm}
% \begin{proof}
% 	Nos ponemos en contexto de la demostración de Erd\H os-Ramanujan del postulado de Bertrand.

% 	Nótese que, por \eqref{eq:bertrand_max}, tenemos:
% 	$$ n^{\pi(2n) - \pi(n)} < \prod_{n < p \le 2n} \le \binom{2n}{n} \le \prod_{p^r \le 2n < p^{r+1}} p^r \le (2n)^{\pi(2n)} $$
% 	Además, podemos sacar la siguiente cota:
% 	\begin{align*}
% 		\binom{2n}{n} &= \frac{2n(2n - 1)(2n - 2)\cdots (n + 1)}{n(n - 1)(n - 2)\cdots 1} \\
% 			      &= 2 \cdot \left( 2 + \frac{1}{n - 1} \right)\left( 2 + \frac{2}{n - 2} \right) \cdots \left( 2 + \frac{n - 1}{1} \right) \ge 2^n.
% 	\end{align*}
% 	Así, empleando que $\binom{2n}{n} \le 4^n$ obtenemos que
% 	$$ n^{\pi(2n) - \pi(n)} < 2^{2n}, \qquad 2^n \le (2n)^{\pi(2n)}, \qquad n \ge 1. $$
% 	Reemplazando $n = 2^j$ obtenemos que
% 	\begin{equation}
% 		2^{j( \pi(2^{j+1}) - \pi(2^j) )} < 2^{2^{j+1}}, \qquad 2^{2^j} \le 2^{(j+1)\pi(2^{j+1})},
% 		\label{eq:mid_chebyshev_est}
% 	\end{equation}
% 	o equivalentemente
% 	$$ j( \pi(2^{j+1}) - \pi(2^j) ) < 2^{j+1}, \qquad 2^j \le (j+1)\pi(2^{j+1}), $$
% 	sumando $\pi(2^{j+1})$ a ambos lados y aplicando el lema anterior tenemos que
% 	$$ (j + 1)\pi(2^{j+1}) - j\pi(2^j) < 2^{j+1} + \pi(2^{j+1}) \le 3 \cdot 2^j. $$
% 	Realizando la suma telescópica con $j = 0$ hasta $j$:
% 	$$ (j + 1)\pi(2^{j+1}) < 3(2^0 + 2^1 + \cdots + 2^j) < 3\cdot 2^{j+1}. $$
% 	Juntando esto con \eqref{eq:mid_chebyshev_est}, entonces tenemos
% 	$$ \frac{1}{2} \frac{2^{j+1}}{j + 1} \pi(2^{j+1}) < 3\cdot \frac{2^{j+1}}{j + 1}. $$
% 	Sea $x \ge 2$ y elijamos $j$ tal que $2^{j+1} \le x < 2^{j+2}$, finalmente
% 	$$ \pi(x) \le \pi(2^{j+2}) < 3\cdot \frac{2^{j+2}}{j + 2} \le 6 \cdot \frac{2^{j+1}}{H(2^{j+2})} \le 6 \cdot \frac{x}{H(x)} $$
% 	y por otro lado
% 	\begin{equation}
% 		\pi(x) \ge \pi(2^{j+1}) \ge \frac{1}{2} \frac{2^{j+1}}{j + 1} \ge \frac{1}{8} \frac{2^{j + 2}}{\frac{1}{2}(j + 2)} \ge \frac{1}{8} \frac{x}{H(x)}.
% 		\tqedhere
% 	\end{equation}
% \end{proof}

% \begin{thm}
% 	$\displaystyle \frac{1}{8} \le \frac{\pi(x)}{x/\log x} \le 6, \qquad n \ge 2.$
% \end{thm}
% \begin{proof}
% 	Basta aproximar
% 	$$ \log\left(\frac{n}{2}\right) = \int_{2}^{n} \frac{1}{t} \, \ud t
% 	\le \frac{1}{2} + \frac{1}{3} + \cdots + \frac{1}{n} = H(n) - 1 < \int_{1}^{n} \frac{1}{t} \, \ud t = \log t, $$
% 	y notar que para $n \ge 4$ se tiene $\frac{1}{2}\log n \le \log(n/2)$.
% \end{proof}

\section{El teorema de Dirichlet}\label{sec:dirichlets_thm}
El teorema de Dirichlet demuestra una versión mucho más fuerte de la infinitud de primos, formalmente
demuestra que si $(m; k) = 1$ [son coprimos], entonces hay infinitos primos de la forma $mn + k$.
\begin{mydef}
	Los elementos formales de $\Z/n\Z$, es decir, las clases de equivalencia $[a]_n := \{b : n\mid b-a\}$,
	se le llaman \strong{clases de residuos módulo $n$}.
	\par
	Las clases cuyo resto es coprimo, son las clases de $U_n := (\Z/n\Z)^\times$.
\end{mydef}
Los naturales se reparten entre las clases de residuos módulo $n$, sin embargo, la clase $[a]_n$ cuando $a, n$ no son coprimos,
sólo puede tener como máximo un número primo: en efecto, al no ser coprimos sea $m := (a; n)$, luego $m \mid nk + a$.
Aún así, puede darse que $a = p$ primo y $p \mid n\cdot 0 + p = p$, sin embargo, si $k > 1$, entonces $m$ es un divisor propio y el término no es primo.
El teorema de Dirichlet pues se enunciaría así: una clase de residuos módulo $n$ que no esté en $U_n$ tiene a lo más un primo y las de $U_n$ tienen infinitos.

Un ejemplo para visualizar ésto reside en el siguiente gráfico de puntos de la forma $n \mapsto (n\cos n, n\sin n)$ (ver fig.~\ref{fig:prime-spirals}),
donde las clases de equivalencia módulo 6 están marcadas. En rojo son las clases de resto no coprimo, en azul los que sí. Los puntos en azúl son números primos.%
\footnote{La idea fue original de un video expositorio de 3Blue1Brown: \url{https://www.youtube.com/watch?v=EK32jo7i5LQ}.} 
\begin{figure}[!bht]
	\centering
	\includegraphics[scale=1]{prime-spirals.pdf}
	\caption{}%
	\label{fig:prime-spirals}
\end{figure}

Antes de empezar con la demostración, cabe observar un hecho notable:
\begin{thm}
	Las expresiones siguientes son equivalentes:
	\begin{enumerate}
		\item \textbf{Teorema de Dirichlet:} Para todo $a, n$ coprimos, existen infinitos números primos $p$ tales que $p \equiv a \mod n$.
		\item Para todo $a, n$ coprimos, existe \textit{al menos un} número primo $p$ tal que $p \equiv a \mod n$.
	\end{enumerate}
\end{thm}
Ojo, el uso de cuantificadores puede ser engañoso: el teorema no dice que saber que existe $p \equiv 4 \pmod 5$ (en este caso $p = 19$) implica la infinitud
de los primos en $[4]_5$; sino que hay que saberlo para todo par $a, n$ coprimos.
\begin{proof}
	Es claro que $1 \implies 2$, veamos el recíproco:
	Sean $a, m$ coprimos, de modo que $m > 1$ y podemos exigir que $0 < a \le m$.
	Por hipótesis se tiene que existe un primo $p$ tal que $p \equiv a \pmod m$.
	Luego, sean $p_1, \dots, p_r$ primos tales que $p \equiv a \pmod m$ y elijamos un natural $k$ de modo que $m^k > p_1\cdots p_r$.
	Nótese que
	$$ 0 < a < a + m^k \le m + m^k = m(1 + m^{k-1}) \le m^{k+1}, $$
	donde empleamos que $1 + x \le m x$ syss $ \frac{1}{m-1} < 1 \le x $.
	Es fácil notar que $(a + m^k; m^{k+1}) = 1$, de modo que existe un primo $p$ que satisface $p \equiv a + m^k \pmod{m^{k+1}}$.
	Finalmente, $p \ge a + m^k > p_j$ para todo $j$, de modo que es un nuevo primo en nuestra lista.
\end{proof}

Primero vamos a tener que definir las siguientes funciones:
\begin{mydefi}\index{carácter!de Dirichlet}
	Dado un natural $m > 1$, se dice que una función $\chi\colon \Z \to \C$ es un \strong{carácter de Dirichlet módulo $m$} si no es idénticamente nula y
	satisface lo siguiente:
	\begin{enumerate}[i)]
		\item Si $a\equiv b\pmod m$, entonces $\chi(a) = \chi(b)$.
		\item Si $a, m$ no son coprimos, entonces $\chi(a) = 0$.
		\item $\chi(ab) = \chi(a)\chi(b)$.
	\end{enumerate}
\end{mydefi}
Traduciendo las propiedades: la primera dice que un carácter módulo $m$ está bien definido sobre $\Z_m$,
la segunda que se anula fuera de $U_m$ y la tercera es que las funciones son completamente multiplicativas y, por ende, un morfismo de grupos (multiplicativo).

Un ejemplo trivial es el siguiente:
\begin{mydef}
	Se le llama carácter \strong{principal} de Dirichlet módulo $m$ a aquella tal que $\chi(c) = 1$ si $c \in U_m$ y $\chi(c) = 0$ si $c \notin U_m$.
\end{mydef}

\begin{prop}
	Sea $\chi\colon \Z/m\Z \to \C$ un carácter de Dirichlet:
	\begin{enumerate}
		\item $\chi$ está completamente determinado por su valor en $U_m$.
		\item $\chi(1) = 1$.
		\item $\chi|_{U_m}\colon U_m \to \C_{\ne 0}$ es un morfismo (multiplicativo) de grupos.
			En consecuente, $\chi(c) \ne 0$ para todo $c\in U_m$ y $\chi(c^k) = \chi(c)^k$ para todo $k\in\Z$.
		\item Para todo $c\in U_m$ se cumple que $\chi(c)$ es una $\phi(m)$-ésima raíz de la unidad.
		\item Los caracteres de Dirichlet módulo $m$ forman un grupo con el producto puntual.
	\end{enumerate}
\end{prop}
\begin{proof}
	La primera se deduce de la propiedad \textsc{(ii)}. La segunda del hecho de que $\chi$ es completamente multiplicativa.
	La tercera sale del hecho de que $\chi(c)\chi(c^{-1}) = \chi(1) = 1$, luego $\chi(c)$ es no nula y por ser completamente multiplicativa
	se deduce que es morfismo de grupos.

	La cuarta es consecuencia del teorema de Euler-Fermat
	% \cite[Teo.~1.28]{Alg}
	% \todoref{Insertar cita a \textit{Álgebra}.}
	puesto que $c^{\phi(m)} \equiv 1$, luego $\chi(c^{\phi(m)}) = \chi(c)^{\phi(m)} = 1$.

	Para la quinta basta comprobar que se cumplen los axiomas de grupo.
	La asociatividad y conmutatividad son heredados del producto complejo.
	Sea $\chi_0$ el carácter principal, es claro que éste juega el rol del neutro.
	Sólo basta notar que si $\chi\colon U_m \to \C$ es un carácter de Dirichlet,
	entonces $\chi'$ dado por $\chi'(a) = \chi(a)^{-1}$ (puesto que el dominio es $U_m$) es también un carácter de Dirichlet y es de hecho su inversa.
\end{proof}
La quinta propiedad es ya por sí sola curiosa, pero se puede refinar bastante:

\begin{thm}
	% Para todo $m > 1$ hay exactamente $\phi(m)$ caracteres de Dichlet módulo $m$.
	Para todo $m > 1$ se cumple que el grupo de los caracteres de Dirichlet módulo $m$ es isomorfo a $U_m$.
	En consecuencia hay exactamente $\phi(m)$ caracteres de Dirichlet módulo $m$.
\end{thm}
\begin{proof}
	La demostración será por casos:
	\begin{enumerate}[a)]
		\item \underline{$m = p^n$:}
			Por el teorema~\ref{thm:primitive_root_mod_pn} se cumple que $U_m$ está generado por un solo elemento, digamos $a$. 
			Sea $\zeta$ la $\phi(m)$-ésima raíz de la unidad, es decir, tal que $\order\zeta = \phi(m)$ en el grupo $\C^\times$.
			Y sea $\chi$ la función determinada por $\chi(a^j) = \zeta^j$.
			Luego la aplicación $a^j \mapsto \chi^j$ es de hecho un isomorfismo de grupos.
			% entonces como $\chi(a^j) = \chi(a)^j$, la función está completamente determinada por el valor en $a$.
			% Y hay exactamente $\phi(m)$ raíces $\phi(m)$-ésimas de la unidad.

		\item \underline{En otro caso:}
			Entonces $m = p_n^{\alpha_1} \cdots p_n^{\alpha_n}$ y basta aplicar el teorema chino del resto. \qedhere
	\end{enumerate}
\end{proof}

\begin{thm}\label{thm:chi_sums}
	Sea $m\ge 2$. Entonces:
	\begin{enumerate}
		\item Para todo carácter $\chi$ de Dirichlet módulo $m$ se cumple:
			$$ \sum_{k=1}^m \chi(k) =
			\begin{cases}
				\phi(m), & \text{$\chi$ es principal} \\
				0,       & \text{$\chi$ no es principal}
			\end{cases} $$

		\item Si $k$ es un entero fijo, entonces:
			$$ \sum_{\chi} \chi(k) =
			\begin{cases}
				\phi(m), & k \equiv 1\pmod k \\
				0,       & k\not\equiv 1\pmod k
			\end{cases} $$
			donde $\chi$ recorre todos los caracteres de Dirichlet módulo $m$.
	\end{enumerate}
\end{thm}
\begin{proof}
	\begin{enumerate}
		\item Claramente si $\chi$ es principal se cumple la relación, puesto que $\chi(c) = 1$ para $c \in U_m$ y hay $\phi(m)$ elementos en $U_m$.

			Si $\chi$ no es principal, entonces elijamos $\bar k_0$ tal que $\chi(\bar k_0) \ne 1$. Como $U_m$ es un grupo, se cumple que
			$x \mapsto \bar k_0 \cdot x$ es una permutación, luego
			$$ \sum_{k=1}^m \chi(k) = \sum_{k=1}^m \chi(k_0k) = \chi(k_0)\sum_{k=1}^m \chi(k). $$
			Por lo que
			$$ (1 - \chi(k_0)) \sum_{k=1}^m \chi(k) = 0 $$
			donde el primer término es no nulo.

		\item Es trivial si $k \equiv 1 \pmod m$ o si $(k; m) \ne 1$.
			Para el caso restante sean $\chi_1, \dots, \chi_n$ todos los caracteres de Dirichlet, entonces veamos que
			si $\psi(k)$ es el valor de algún $\chi_i(k) \ne 1$, entonces $\psi(k)\chi_1(k), \dots, \psi(k)\chi_n(k)$
			es una mera permutación de los valores originales.
			Ésto se cumple dado que $\chi \mapsto \chi\cdot\chi_j$ es una permutación en el grupo de caracteres de Dirichlet módulo $m$.
			Luego
			$$ \sum_\chi \chi(k) = \sum_\chi \psi(k)\chi(k) = \psi(k)\sum_\chi \chi(k) $$
			y procediendo igual que antes se concluye el teorema. \qedhere
	\end{enumerate}
\end{proof}

\begin{thm}\label{thm:dirichlet_char_sum}
	Sea $C \in U_m$, entonces para todo $k\in\Z$ se cumple que
	$$ \frac{1}{\phi(m)}\sum_\chi \overline{\chi}(C) \chi(k) =
	\begin{cases}
		1, &k\in C \\
		0, &k\notin C
	\end{cases} $$
\end{thm}
\begin{proof}
	Nótese que $\overline\chi(C) = \chi(C)^{-1}$ y luego basta aplicar el teorema anterior, ya que si $k \in C$,
	entonces $C^{-1}\overline k = \overline 1$ y la suma vale $\phi(m)$; y en otro caso vale 0.
\end{proof}

Estamos ya a punto de ver el teorema de Dirichlet, pero aún son necesarias algunas herramientas adicionales.
Para ello, primero veamos el siguiente teorema:

\begin{thm}
	Sea $f\colon (0, \infty) \to \R$ una función decreciente tal que $\lim_{x\to\infty} f(x) = 0$, y sea $\chi$ un carácter de Dirichlet módulo $m>1$
	no principal. Entonces, la serie $\sum_{n=1}^\infty \chi(n)f(n)$ converge y
	$$ \sum_{n \le x} \chi(n)f(n) = \sum_{n=1}^\infty \chi(n)f(n) + O(f(x)). $$
\end{thm}
\begin{proof}
	Sea $A(x) := \sum_{n\le x} \chi(n)$.
	Nótese que $\chi(n)$ es $m$-periódica y que por el teorema~\ref{thm:chi_sums} se cumple que $A(m) = 0$.
	Luego $|A(x)| \le K$ para algún $K$.

	Por fórmula de suma por partes se tiene que
	\begin{align*}
		\sum_{x < n \le y} \chi(n)f(n) &= \sum_{n = \sfloor{x}+1}^{ \sfloor{y} } \chi(n)f(n) \\
		&= A(\sfloor{y})f(\sfloor{y}) - A(\sfloor{x})f(\sfloor{x}) + \sum_{n = \sfloor{x}+1}^{\sfloor{y}-1} A(n) \big( f(n) - f(n+1) \big).
	\end{align*}
	Luego nótese que
	\begin{align*}
		\left| \sum_{n = \sfloor{x}+1}^{\sfloor{y}-1} A(n) \big( f(n) - f(n+1) \big) \right|
		&\le K \sum_{n = \sfloor{x}+1}^{\sfloor{y}-1} \big( f(n) - f(n+1) \big) \\
		&= K \big( f(\sfloor{x}+1) - f(\sfloor{y}-1) \big) \le Kf(\sfloor{x} + 1).
	\end{align*}
	De modo que
	$$ \sum_{x < n \le y} \chi(n)f(n) \le Kf(\sfloor{x} + 1) \le Kf(x) = O(f(x)) $$
	por lo que, la sumatoria converge y
	\begin{equation}
		\sum_{n\le x} \chi(n)f(n) = \sum_{n=1}^\infty \chi(n)f(n) - \sum_{n>x} \chi(n)f(n) = \sum_{n=1}^\infty \chi(n)f(n) + O(f(x)) \tqedhere
	\end{equation}
\end{proof}

El teorema anterior nos permite definir las dos siguientes funciones para un carácter de Dirichlet $\chi$ no principal:
$$ L_0(\chi) := \sum_{n=1}^\infty \frac{\chi(n)}{n}, \quad L_1(\chi) := \sum_{n=1}^\infty \frac{\chi(n)\log n}{n}. $$
% Y nos faltan los siguientes dos resultados sobre éstas funciones:

% \begin{thm}
% 	Para todo carácter $\chi$ de Dirichlet módulo $m$ no principal se cumple:
% 	\begin{enumerate}
% 		\item $\displaystyle L_0(\chi) \sum_{n\le x} \frac{\chi(n)\mu(n)}{n} = O(1)$.
% 		\item Si $L_0(\chi) = 0$, entonces
% 			$$ L_1(\chi) \sum_{n\le x} \frac{\chi(n)\mu(n)}{n} = -\log x + O(1). $$
% 	\end{enumerate}
% \end{thm}
% \begin{proof}
% 	\begin{enumerate}
% 		\item Primero aplicamos la fórmula generalizada de inversión de Möbius, donde la función multiplicativa es $\alpha(n) := \chi(n)/n$
% 			y la otra es $F(x) := 1$, con lo que
% 			$$ G(x) = \sum_{n\le x} \alpha(n) F\left( \frac{x}{n} \right) = \sum_{n\le x} \frac{\chi(n)}{n} = L_0(\chi) + R(x) $$
% 			donde $|R(x)| \le K/x$ por el teorema anterior.
% 			Conversamente
% 			\begin{align*}
% 				F(x) = 1 &= \sum_{n\le x} \frac{\chi(n)\mu(n)}{n}\big( L_0(\chi) + R(x/n) \big) \\
% 				&= L_0(\chi)\sum_{n\le x} \frac{\chi(n)\mu(n)}{n} + \sum_{n\le x} \frac{\chi(n)\mu(n)}{n}R(x/n).
% 			\end{align*}
% 			Nótese que
% 			$$ \left| \sum_{n\le x} \frac{\chi(n)\mu(n)}{n}R(x/n) \right| \le \sum_{n\le x} \frac{1}{n} \frac{K}{x/n} \le K. $$
% 			Lo que demuestra el enunciado.

% 		\item Volvemos a aplicar la fórmula generalizada de inversión de Möbius con $\alpha(n) := \chi(n)/n$ y $F(x) := \log x$:
% 			\begin{align*}
% 				G(x) &= \sum_{n\le x} \frac{\chi(n)}{n} \log\left( \frac{x}{n} \right)
% 				= \log x\sum_{n\le x} \frac{\chi(n)}{n} - \sum_{n\le x} \frac{\chi(n)\log(n)}{n} \\
% 				&= L_0(\chi)\log x + R(x) - L_1(\chi).
% 			\end{align*}
% 			donde $|R(x)| \le \frac{K\log x}{x}$.
% 			Recordemos que por hipótesis $L_0(\chi) = 0$.
% 			Y por inversión nos queda que
% 			\begin{align*}
% 				F(x) = \log x &= \sum_{n\le x} \frac{\chi(n)\mu(n)}{n} \big( -L_1(\chi) + R(x/n) \big) \\
% 				&= -L_1(\chi) \sum_{n\le x} \frac{\chi(n)\mu(n)}{n} + \sum_{n\le x} \frac{\chi(n)\mu(n)}{n} R(x/n).
% 			\end{align*}
% 			Y nótese que
% 			\begin{align*}
% 				\left| \sum_{n\le x} \frac{\chi(n)\mu(n)}{n} R(x/n) \right| &\le \sum_{n\le x} \frac{1}{n} \frac{K\log(x/n)}{x/n} \\
% 				&= \frac{K}{x} \sum_{n\le x} ( \log x - \log n ) \le \frac{K}{x}( x\log x - T(x) ) \\
% 				&= \frac{K}{x}( x - O(\log x) ) = K + O\left( \frac{\log x}{x} \right) = O(1).
% 			\end{align*}
% 			donde empleamos la identidad \eqref{eq:log_sum_approx}. \qedhere
% 	\end{enumerate}
% \end{proof}

Ahora estamos listos para ver el teorema, el cuál saldrá como una consecuencia de una versión de los teoremas de Mertens para progresiones aritméticas.
\begin{thm}
	Para todo $m > 1$ y todo $C \in U_m$ se cumple que
	$$ \sum_{\substack{ p\le x \\ p\in C }} \frac{\log p}{p} = \frac{1}{\phi(m)}\log x + O(1). $$
\end{thm}
\begin{proof}
	Primero nótese que, al igual que con el primer teorema de Mertens, primero introducimos la función de von Mangoldt:
	$$ \sum_{\substack{n\le x \\ n\in C}} \frac{\Lambda(n)}{n} = \sum_{\substack{p\le x \\ p\in C}} \frac{\log p}{p}
	+ \sum_{\substack{p^{k+1}\le x \\ p^{k+1}\in C}} \frac{\log p}{p^{k+1}} $$
	y se nota que
	$$ 0 \le \sum_{\substack{p^{k+1}\le x \\ p^{k+1}\in C}} \frac{\log p}{p^{k+1}} \le \sum_{n=1}^\infty \frac{\log n}{n^2} $$
	donde el último término converge, por lo que basta probar el enunciado para $\Lambda(n)/n$.

	Nótese que por el teorema~\ref{thm:dirichlet_char_sum} se tiene que
	\begin{align*}
		\sum_{\substack{n\le x \\ n\in C}}\frac{\Lambda(n)}{n} &= \frac{1}{\phi(m)}\sum_{n\le x} \frac{\Lambda(n)}{n}\sum_\chi \overline{\chi}(C)\chi(n) \\
		&= \frac{1}{\phi(m)}\sum_\chi \overline{\chi}(C) \sum_{n\le x} \frac{\chi(n)\Lambda(n)}{n}
	\end{align*}
	Separemos la última suma entre el carácter principal y los no principales y notemos que si $\chi$ es no principal entonces
	$$ \left| \sum_{n\le x} \frac{\chi(n)\Lambda(n)}{n} \right| \le \left| \sum_{n=1}^\infty \frac{\chi(n)\log n}{n} \right| = |L_1(\chi)|. $$
	Luego el problema se reduce a estudiar el caso de $\chi_0$:
	$$ \sum_{n\le x} \frac{\chi_0(n)\Lambda(n)}{n} = \sum_{\substack{n\le x \\ (n; m)=1}} \frac{\Lambda(n)}{n}
	= \sum_{n\le x} \frac{\Lambda(n)}{n} - \sum_{\substack{p^k\le x \\ p\mid m}} \frac{\log p}{p^k}. $$
	Pero
	$$ \sum_{\substack{p^k\le x \\ p\mid m}} \frac{\log p}{p^k} \le \sum_{p \mid m}\sum_{k=1}^\infty \frac{\log p}{p^k}
	= \sum_{p\mid m} \frac{\log p}{p(p - 1)}, $$
	que por ser suma finita de términos es también finita y acotada.
	Finalmente basta recordar la variación del primer teorema de Mertens \eqref{eq:mertens_i_var}:
	\begin{equation}
		\sum_{n\le x} \frac{\Lambda(n)}{n} = \log x + O(1). \tqedhere
	\end{equation}
\end{proof}

\begin{thmi}[Teorema de Dirichlet]\index{teorema!de Dirichlet}
	Para todo $m > 1$ y todo $C \in U_m$, existen infinitos primos en $C$.
\end{thmi}

Y el complementario:
\begin{thm}
	Para todo $m > 1$ y todo $C \in U_m$ se cumple que
	$$ \sum_{\substack{p\le x \\ p\in C}} \frac{1}{p} = \frac{1}{\phi(m)} \log\log x + M_C + O\left( \frac{1}{\log x} \right), $$
	donde $M_C$ es una constante que depende de $C$.
\end{thm}
\begin{proof}
	Definamos
	$$ a(n) :=
	\begin{cases}
		1, &n \text{ primo en $C$} \\
		0, &\text{otro caso}
	\end{cases} \quad
	A(x) := \sum_{n\le x} a(n)\frac{\log n}{n} = \frac{1}{\phi(m)}\log x + R(x) $$
	donde $|R(x)| \le K$.
	Con $f(t) := 1/\log t$ por fórmula de Abel se cumple que
	$$ \sum_{\substack{p\le x \\ p\in C}} \frac{1}{p} = \frac{A(x)}{\log x} + \int_{2}^{x} \frac{A(t)}{t\log^2 t} \, \ud t $$
	Luego basta seguir al pie de la letra la demostración del segundo teorema de Mertens para llegar a la conclusión del enunciado.
\end{proof}

% \begin{thm}
% 	Para todo $x\ge 1$, se cumple que
% 	$$ \sum_{n\le x} \phi(n) = x^2 \frac{3}{\pi^2} + O(x^{3/2}). $$
% \end{thm}
% \begin{proof}
% 	Como $\phi = \mu * \Id$
% \end{proof}

\printbibliography[segment=\therefsegment, check=onlynew, notcategory=historical, notcategory=other]
\bibbycategory[segment=\therefsegment, check=onlynew]

\end{document}
